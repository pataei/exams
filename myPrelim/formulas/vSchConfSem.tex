\begin{figure}

\begin{comment}
\textbf{Relational model generic objects:}
\begin{syntax}
%OLD
D\in \mathbf{Dom} &&& \textit{Domain}\\
A\in \mathbf{Att} &&& \textit{Attribute Name}\\
R\in \mathbf{R} &&& \textit{Relation Name}\\
t \in \mathbf{T} &&& \textit{Tuple}
\end{syntax}

\medskip
\textbf{Relational model definition:}
\begin{syntax}
l\in \mathbf{L} &=& \vn{A} &\textit{Attribute set}\\
s \in \mathbb{S} &=& R(A_1, \ldots , A_n ) & \textit{Relation specification}\\
S \in \mathcal{\mathbf{S}} &\Coloneqq& {\vn{s}} & \textit{Schema}\\
T \in \mathbf{T} &\Coloneqq& \{\llangle t(1), \ldots, t(k)\rrangle \myOR \\
&&t(i) \in D_i,
1 \leq i \leq k ,\\
&&k = \mathit{arity}(R) \} 
%v_1^1\in D_1, \ldots, v_n^1\in D_n\rrangle, \ldots, \llangle v_1^m\in D_1, \ldots, v_n^m\in D_n\rrangle|\\
%&& \hspace{0.5cm} m = \textit{number of } R_I\textit{'s tuples}\} 
&\textit{Relation Instance (Table)}\\
%I \in \mathbf{Inst} &\Coloneqq& R_{1_I}, \cdots, R_{n_I} & \textit{Database Instance}
\end{syntax}


\medskip
\textbf{Variational relational algebra objects:}
\begin{syntax}
\synDef \dimMeta \ffSet &&&\textit{Presence condition}\\
\synDef \vAtt \vAttSet &&&\textit{Variational attribute}\\
\synDef \vAttList \vAttSet &\eqq& \vAtt, \vAttList \myOR \empAtt &\textit{Variational attribute list}\\
\synDef \vRelSch \vRelSet &\eqq& \vRelDef &\textit{Variational relation schema}\\
 \vRel &&&\textit{Variational relation}\\
\synDef \vSch \vSchSet &\eqq& \vSchDef &\textit{Variational schema}\\
 &&&\textit{Variational database instance}
\end{syntax}
\end{comment}

%%%%%%%%%%%%%%%%%%%%%%%%%%%%%%%%%%%%%%%%%%%%%%%%%%%
%\medskip
\textbf{Variational Attribute Set Configuration:}
\begin{flalign*}
& \olSem [] . : \vAttSet \to \confSet \to \pAttSet&
\end{flalign*}
%
\begin{flalign*}
& \olSem {\optAtt} &\spcEq&
    \begin{cases}
        \pAtt, \If \fSem {\dimMeta \wedge \getPC{\getRel \vAtt}} = \t\\
        \empAtt, \Otherwise
     \end{cases}\\
& \olSem {\optAtt, \vAttList} &\spcEq& \olSem {\optAtt}, \olSem \vAttList&\\
& \olSem \empAtt &\spcEq& \empAtt &
\end{flalign*}

%
\medskip
\textbf{V-Relation Configuration:}
\begin{flalign*}%\raggedleft
&\orSem [] . : \vRelSchSet \to \confSet \to \pRelSchSet&
\end{flalign*}
%
\begin{flalign*}
&\orSem \vRelDef = 
	\begin{cases}
		\pRel \paran {\olSem {\vAttList}}, \If \fSem \dimMeta = \t\\
		\empRel, \Otherwise
	\end{cases}&
\end{flalign*}

%
\medskip
\textbf{V-Schema Configuration:}
\begin{flalign*}%\raggedleft
&\osSem [] . : \vSchSet \to \confSet \to \pSchSet&
\end{flalign*}
%
\begin{flalign*}
&\osSem {\annot [\fModel] {\setDef {\vRelDefNum 1, \ldots, \vRelDefNum n}}}\\
&\hspace{0.3cm}= \begin{cases}
%		\setDef {\orSem {\vRelDefNumF 1}, \ldots, \orSem {\vRelDefNumF n}},
                 \setDef {\orSem {\vRel_1\left( \vAttList_1 \right)^{\dimMeta_1 \wedge \fModel} }, \ldots, 
                 \orSem {\vRel_n\left( \vAttList_n \right)^{\dimMeta_n \wedge \fModel} }},		
        \text{ if } \fSem \fModel = \t\\
        \setDef \ ,\qquad\qquad\qquad\qquad\qquad\quad \text{ otherwise}
	\end{cases}&\\
\end{flalign*}

\caption{
Configuration of variational set of attributes, v-relations, and v-schema.
\empAtt\ denotes both an empty variational set of attributes and a non-existence 
relation.
%\getRel \vAtt\ returns the relation that attribute \vAtt\ belongs to and \getPC \elem\ 
%returns the presence condition of element \elem. 
Note that the feature model and 
relation presence condition are passed all the way to attributes due to the 
hierarchal structure of presence conditions within a v-schema.}
\label{fig:vsch-conf-sem}
\end{figure} 
\begin{figure}
%\textbf{Configuration selection semantics of \vqsTxt:}
\begin{alignat*}{1}
\eeSem [] . &: \qSet \to \confSet \to \pQSet\\
%
\eeSem \vRel &= \orSem \vRel = \pRel\\
\eeSem {\vSel \vQ}  &= \vSel [\ecSem \vCond] {\eeSem \vQ}\\
%
\eeSem {\vPrj [\vAttList] \vQ} &= \vPrj [\olSem \vAttList] {\eeSem \vQ}\\
%
\eeSem {{\vQ_1} \times {\vQ_2}} &= \eeSem {\vQ_1} \times \eeSem {\vQ_2}\\
%
%\eeSem {{\vQ_1} \Join_\vCond {\vQ_2}} &= \eeSem {\vQ_1} \Join_{\ecSem \vCond} \eeSem {\vQ_2}\\
%
\eeSem {\chc {\vQ_1, \vQ_2}} &= 
	\begin{cases}
		\eeSem {\vQ_1}, \text{ if } \fSem \dimMeta = \t\\
		\eeSem {\vQ_2}, \text{ otherwise}
	\end{cases}\\
%
\eeSem {{\vQ_1} \circ {\vQ_2}} &= \eeSem {\vQ_1} \circ \eeSem {\vQ_2}\\
%
\eeSem {\empRel} &= \underline {\empRel}
\end{alignat*}
\caption{Configuration of VRA which assumes that the given v-query
is well-typed. 
%\orSem ., \ecSem ., and \olSem . are
%configuration of v-relation, v-condition, and variational attribute
%set, respectively, defined in \figref{vdb-conf}, 
%\figref{vcond-conf-sem}, \figref{vdb-conf}.
Note that we have extended RA with an empty relation $\underline {\empRel}$.}
\label{fig:v-alg-conf-sem}
\end{figure}

\begin{figure}

\textbf{V-queries typing rules:}

  \begin{mathpar}
  \small
  
%  \inferrule[\judge]
 % 	{\env{\vQ}{\vType}}
 %   {}
    \inferrule[\relationE]
  	{\vRel (\vType)^{\VVal \dimMeta} \in \vSch \\
	\neg \sat{\vctx \wedge \neg \VVal \dimMeta} }
     {\envWithSchema{\envInContext [\vctx ] {\vType}}}

%implicitly-typed lang:
%    \inferrule[\relationE]
%  	{\vRel (\vType)^{\VVal \dimMeta} \in \vSch \\
%	\sat{\vctx \wedge \VVal \dimMeta} }
%     {\envWithSchema{\envInContext [\vctx \wedge \VVal \dimMeta] {\vType}}}
  
  \inferrule[\prjE]
  	{\envPrime \\
    	\subsumeExpl {\annot \vType}  {\annot [\VVal \vctx] {\VVal \vType}}}
    {\env{\vPrj[\vType] \vQ} {\envInContext [\vctx] \vType}}

%implicitly-typed lang:
%  \inferrule[\prjE]
%  	{\envPrime \\
%    	\subsume {\annot \vType}  {\annot [\VVal \vctx] {\VVal \vType}}}
%    {\env{\vPrj[\vType] \vQ} {\envInContext [\VVal \vctx] {\left(\annot {\vType} \cap {\VVal \vType} \right)}}}


  \inferrule[\selE]
  	{\env \vQ {\envInContext [\VVal \vctx] \vType} \\
    	\envCondAnnot \vCond}
    {\env{\vSel \vQ}{\envInContext [\VVal \vctx] \vType}}
    
  \inferrule[\choiceE]
  	{\envOne[\vctx \wedge \VVal \dimMeta] \\
    	\envTwo[\vctx \wedge \neg \VVal \dimMeta]}
    {\env{\chc[\VVal \dimMeta]{\vQ_1, \vQ_2}}{
     \envInContext [\vctx_1 \vee \vctx_2] {\left({\envInContext [\vctx_1] \vType_1} \cup
    							{\envInContext [\vctx_2] \vType_2}\right)}}}
    
  \inferrule[\productE]
  	{\envOne \\
    	\envTwo\\
	\annot [\vctx_1] \vType_1 \cap \annot [\vctx_2] \vType_2 = \{\}}
    {\env{\vQ_1 \times \vQ_2}{\envInContext [\vctx_1 \wedge \vctx_2] {\left(\vType_1 \cup \vType_2\right)}}}


  \inferrule[\setopE]
  	{\envOne \\
    	\envTwo \\
	\envEval {\annot [\vctx_1] \vType_1} {\annot [\vctx_2] \vType_2}}
%        \envEval{\envInContext{\vType_1}} \vType \\
%        \envEval{\envInContext{\vType_2}} \vType}
    {\env{\vQ_1 \circ \vQ_2} {\envInContext [\vctx_1] \vType_1} }

%  \inferrule[\diffE]
%  	{\envOne \\
%    	\envTwo \\
%        \envEval{\envInContext{\vType_1}} \vType \\
%        \envEval{\envInContext{\vType_2}} \vType}
%    {\env{\vQ_1 \setminus \vQ_2} \vType}
  \end{mathpar}
  
\medskip
\textbf{V-condition typing rules (b: boolean tag, \pAtt: plain attribute, k: constant value):}
%(b: boolean tag, A: plain attribute, k: constant value)}
  \begin{mathpar}
  \small    

  \inferrule[\conjC]
  	{\envCond \vCond_1\\
    	\envCond \vCond_2}
    {\envCond{\vCond_1 \wedge \vCond_2}}
    
  \inferrule[\disjC]
  	{\envCond \vCond_1\\
    	\envCond \vCond_2}
    {\envCond{\vCond_1 \vee \vCond_2}}
    


  \inferrule[\choiceC]
%  	{\defType{\relInContext{\vContext''}}\in \vSch \\
    	{\envCond[\vctx \wedge \VVal \dimMeta, \vType]{\vCond_1} \\
        \envCond[\vctx \wedge \neg \VVal \dimMeta, \vType]{\vCond_2}}
    {\envCond{\chc[\VVal \dimMeta]{\vCond_1, \vCond_2}}}
    

  \inferrule[\notC]
  	{\envCond \vCond}
    {\envCond \neg \vCond}
        
%  \inferrule[]
%  	{\envCond[\vContext \wedge \dimMeta]{\vCond_1} \\
%    	\envCond[\vContext \wedge \neg\dimMeta]{\vCond_2}}
%    {\envCond{\chc{\vCond_1, \vCond_2}}}
    

    
  \inferrule[\attValC]
  	{
	%\defType{\relInContext{\vContext'}}\in \vSch \\
    	\optAtt [\VVal \dimMeta] \in \vType \\
	\taut{{\VVal \dimMeta} \imply \vctx} \\
        \cte \in \dom \vAtt}
    {\envCond{\op \pAtt \cte}}
    
  \inferrule[\boolC]
  	{}
    {\envCond \bTag}
    

    
  \inferrule[\attAttC]
  	{
	%\defType{\relInContext{\vContext'}}\in \vSch \\
    	\optAtt [\dimMeta_1] [\vAtt_1]\in \vType \\
         {\optAtt [\dimMeta_2] [\vAtt_2]} \in \vType \\
         \taut{\dimMeta_1 \imply \vctx} \\
         \taut{\dimMeta_2 \imply \vctx} \\
        \type[\vAtt_1] = \type[ \vAtt_2]}
    {\envCond{\op{\pAtt_1}{\pAtt_2}}}
    
  \end{mathpar}

%\caption{V-condition typing relation. A v-condition \vCond\ is well-typed if 
%it is valid in the variational context \vctx\ and type environment \vType, i.e., 
%\envCond \vCond. Note that the type rules for v-conditions return a boolean, if
%the v-condition is type-correct the rules return \t, otherwise they return \f.}
\caption{Explicitly-annotated VRA and v-condition typing relation. The explicitly-annotated
query is generated by pushing the schema to the implicitly-annotated query. Hence,
all annotations are explicit and included in the query, i.e., they are at least as specific as
corresponding presence conditions encoded in the v-schema. We use 
\ensuremath{\annot \vRel} as syntactic sugar for \chc {\vRel, \empRel}.
}
\label{fig:vq-stat-sem}
\end{figure}


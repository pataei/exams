\subsubsection{Property Checking over a VDB}
\label{sec:vdb-props}


Since
a \emph{single} database can supply data for \emph{many} different
database variants \emph{at the same time}
encoding variability explicitly in a database  allows the developers to 
check for different properties over all database variants
For example, to ensure that variability associated with each variant is 
valid we provide a set of validity checks.
% constructed VDB is valid we provide a set of validity checks.
These checks ensure that the presence conditions both at the schema level and
data level are consistent and satisfiable, that is, they are present in at
least one database variant. In the following, the function $\sat\dimMeta$
denotes a satisfiability check that returns \t\ if the feature expression
\dimMeta\ is satisfiable and \f\ otherwise.


At the schema level we check the following properties:
%
\begin{enumerate}
%
\item That there is at least one valid configuration of the feature model $m$:
%
$\sat\fModel$
%
\item That every relation $r$ is present in at least one configuration of the
variational schema:
%
$\forall\vRel\in\vSch, \sat{\fModel\wedge\getPC\vRel}$
%
\item That every attribute $a$ in every relation $r$ is present in at least one
configuration of the variational schema:
%
$\forall\vAtt\in\vRel, \forall\vRel\in\vSch,
\sat{\fModel\wedge\getPC\vRel\wedge\getPC\vAtt}$
%
\item That if $\vSch_\config$ denotes the expected plain relational schema for
configuration $c$ of the variational schema \vSch, then configuring the
variational schema with that configuration, written $\sem[\config]{\vSch}$,
actually yields that variant:
%
$\forall\config\in\confSet, \sem[\config]{\vSch} = \vSch_\config$
%
\end{enumerate}


\noindent
%
At the data level we check the following properties:
%
\begin{enumerate}
%
\item That every tuple $u$ in relation $r$ is present in at least one variant:
%
$\forall\vTuple\in\vRel, \forall\vRel\in\vSch,
\sat{\fModel\wedge\getPC\vRel\wedge\getPC\vTuple}$ 
%
\item That for every tuple $u$ in relation $r$, if an attribute $a$ in $r$ is
not present in any variants of the tuple, then the value of that attribute in
the tuple, written $\mathit{value}_\vTuple(\vAtt)$, should be NULL:
$\forall\vTuple\in\vRel, \forall\vAtt\in\vRel, \forall\vRel\in\vSch,
\neg\sat{\fModel\wedge\getPC\vRel\wedge\getPC\vAtt\wedge\getPC\vTuple}
\Rightarrow \mathit{value}_\vTuple(\vAtt) = \text{NULL}$
%
\end{enumerate}


\noindent
%
We implemented these checks in our VDBMS tool and verified that both case
studies described in \appref{db} satisfy all of them. 
%
Depending on the context of the VDB, 
more specialized properties can be checked too. For example, if temporal 
variability in a database is accumulated over variants, i.e., the old data
is also included in a more recent variant in addition to the newly added data,
it is desirable
to ensure that older variants are subsets of newer variants. 
%\TODO{I don't
%understand the thing we're assuming here.} 
Assume that configurations \ensuremath{\config_1, \config_2, \cdots}
represents time-orderly configurations, we formulated
and checked this for the employee use case:
\ensuremath{
\forall \config_i, \config_j \in \confSet, i \le j, \sem[\config_i] {\vDB} \subseteq \sem[\config_j]{\vDB}
}, 
where \ensuremath{\sem[\config]{\vDB}} denotes configuring a variational database instance
\vDB\ for configuration \config. 
%\parisa{note to myself, impl todo: actually check this for employee db when you got the time!}

%v-table checks:
%- \ensuremath{\forall tuple \in relation \in schema : sat (fm \wedge pc_relation \wedge pc_tuple)}\\
%- \ensuremath{\forall attribute \in relation \in schema, \forall val : if unsat (fm \wedge pc_relation \wedge pc_attribute \wedge pc_tuple)} then value must be null\\


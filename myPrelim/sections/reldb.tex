\subsection{Relational Databases and Relational Algebra}
\label{sec:rel-db}

%\point{database schema.}
A relational database \pDB\ stores information in a structured manner by forcing
data to conform to a \emph{schema} \pSch\ that is a finite set 
$\setDef {{\pRelSch}_1, \ldots, {\pRelSch}_n}$ of \emph{relation schemas}.
A relation schema is defined as
$\pRelSch = \vRel \paran {\vi \pAtt k}$ where each $\pAtt_i$ is an
\emph{attribute} contained in a relation named \vRel. 
$\getRel {\pAtt}$ returns the relation that contains the attribute.
\type [\pAtt]\ returns the \emph{type} of values associated with attribute \pAtt.

%\point{database content.}
The content of database \pDB\ is stored in the form of \emph{tuples}. A tuple \pTuple\
is a mapping between a list of relation schema attributes and their values,
%\footnote{\TODO{\*we can probably remove this if run out of space! \*Note that we only have
%pure relational values, thus, values do not have variational counterparts. Hence,
%we do not use our convention of underlying them to show they are variational.}}, 
i.e.,
$\pTuple = \paran {\vi {\underline v} k}$ for the relation schema \vRel \paran {\vi \pAtt k}.
Hence a \emph{relation content}, \pRelCont, is a set of tuples \setDef {\vi \pTuple m}.
$\getAtt {\underline v}$ returns the attribute the value corresponds to.
A \emph{table} \pTab\ is a pair of relation content and relation schema.
A \emph{database instance}, \pInst, of the database \pDB\ with the
schema \pSch, is a set of tables $\setDef {\pTab_1, \ldots, \pTab_n}$.
%relation contents 
%$\setDef  {{\pRelCont}_1, \ldots, {\pRelCont}_n}$ corresponding
%to a set of relation schemas $\setDef {{\pRelSch}_1, \ldots, {\pRelSch}_n}$ 
%defined in \pSch. 
For brevity, when it is clear from the context we refer to a database instance
by \emph{database}.


\begin{figure}

\begin{syntax}

% feature expressions
%\synDef{\dimMeta}{\ffSet}
%  &\eqq& \multicolumn{2}{l}{%
%         \t \myOR \f \myOR \fName \myOR \neg\fName
%         \myOR \dimMeta\wedge\dimMeta \myOR \dimMeta\vee\dimMeta}
%\\[1.5ex]

% relation conditions
\synDef{\pCond}{\pCondSet}
  &\eqq& \multicolumn{2}{l}{%
         \t \myOR \f \myOR \att\bullet\cte \myOR \att\bullet\att
         \myOR \neg\pCond \myOR \pCond\vee\pCond} \\
%     &|& \multicolumn{2}{l}{\vCond\wedge\vCond \myOR \chc{\vCond,\vCond}}
\\[1.5ex]

% variational relational algebra
\synDef{\pQ}{\pQSet}
  &\eqq& \pRel                 & \textit{Relation reference} \\
     &|& \pRen[\pRel]{\pQ}     & \textit{Renaming} \\
     &|& \pPrj[\pAttList]{\pQ} & \textit{Projection} \\
     &|& \pSel\pQ              & \textit{Selection} \\
     &|& \pQ \Join_{\pCond} \pQ  & \textit{Join} \\
%     &|& \chc{\vQ,\vQ}         & \textit{Choice} \\
%     &|& \empRel               & \textit{Empty relation} \\
%    &|& \vQ \times \vQ        & \textit{Cartesian Product} \\
%    &|& \vQ \circ \vQ         & \textit{Set operation} \\
\end{syntax}

\caption{Syntax of  relational algebra, where $\bullet$ ranges over
comparison operators ($<, \leq, =, \neq, >, \geq$), \cte\ over constant values,
\att\ over attribute names, and \pAttList\ over lists of attributes.
The syntactic category
% \dimMeta\ represents feature expressions, 
 \pCond\
is relational conditions, and \pQ\ is  relational algebra terms.
}
%\vspace{-20pt}
\label{fig:v-alg-def}
\end{figure}
%\vspace{-20pt}


\figref{rel-alg} defines the syntax of 
relational algebra which allows users to query a relational database~\cite{AliceBook}.
%
The first five constructs are adapted from relational algebra:
%
A query may simply \emph{reference} a relation \pRel\ in the schema.
\emph{Renaming} allows giving a name to an intermediate query to be referenced
 later. Note that \pRel\ is an overloaded symbol that indicates both a relation
 and a relation name. 
%
A \emph{projection} enables selecting a subset of attributes from the results
of a subquery, for example, \vPrj[\pAtt_1]{\pRel} would return only attribute $\pAtt_1$
from $\pRel$.
%; we extend the standard project operator to work with annotated lists
%of attributes, for example, \vPrj[a_1,a_2^e]{r} would include $a_1$ for all
%configurations and also $a_2$ for configurations where $e$ is true.
%
A \emph{selection} enables filtering the tuples returned by a subquery based on a
given condition \pCond, for example, \vSel[\pAtt_1 > 3]{\pRel} would return all tuples
from $\pRel$ where the value for $\pAtt_1$ is greater than 3.
%; these conditions may be
%variational to enable returning different tuples for different configurations
%of the VDB.
%
A \emph{Cartesian products} simply cross products every tuple from its
left subquery with every tuple from its right subquery. 
%
The \emph{join} operation joins two subqueries based on a condition and
omitting its condition implies it is a natural join (i.e., join on the
shared attribute of the two subqueries).
For example, $\pRel_1 \bowtie_{\pAtt_1 = \pAtt_2} \pRel_2$ joins tuples from $\pRel_1$ 
and $\pRel_2$ where the attribute $\pAtt_1$ from relation $\pRel_1$ is equal to
attribute $\pAtt_2$ from relation $\pRel_2$. However, if we have $\pRel_1(\pAtt_1, \pAtt_3)$
and $\pRel_2 (\pAtt_1, \pAtt_2)$ then
$\pRel_1 \bowtie \pRel_2$ joins tuples from $\pRel_1$ and $\pRel_2$ where
attribute $\pAtt_1$ has the same value in $\pRel_1$ and $\pRel_2$. 
%
Also, note that join is simply a syntactic sugar for selection of cross product,
that is $\pQ_1 \bowtie_{\pCond} \pQ_2 = \vSel [\pCond] {(\pQ_1 \times \pQ_2)}$.
%
The set operations, union and intersection, require two subqueries to have the same set of attributes
and simply apply the operation, either union or intersection, to the tuples returned by
the subqueries.
For example, if we have $\pRel_1(\pAtt_1, \pAtt_2)$ with 
tuples $\{(1,2)(3,4)\}$
and $\pRel_2(\pAtt_1, \pAtt_2)$ with tuples $\{(1,2),(5,6) \}$
then $\pRel_1 \cup \pRel_2 $ returns the tuples $\{(1,2), (3,4), (5,6)\}$.

%, except
%that again we allow conditions to be variational.
%
%A \emph{choice} encodes a variation point between two subquery alternatives based on a
%given feature expression, e.g., \chc[f_1\wedge f_2]{\vQ_l,\vQ_r} yields
%the results of $\vQ_l$ alternative for configurations where $f_1$ and $f_2$ are enabled,
%and in other configurations yields the results of $\vQ_r$ alternative. Note that the
%conditions $\vCond$ used by selections and joins also contain choices, and
%these behave similarly.
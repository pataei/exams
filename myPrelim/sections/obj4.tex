\subsection{Mechanize proofs of properties of the language and the system}
\label{sec:ro4}

Having established that our framework  effectively and explicitly encode variation 
within the database and its query language, we need to ensure that it satisfies
the properties we desire. Objective 4 aims to define such properties and
prove that they hold for our framework. 
%Additionally, it aims to prove that
%VDBMS implements the semantics of VRA correctly. 
\tabref{ro4} presents
individual research questions we need to answer for this objective. 



\begin{table}[H]
\caption{Objective 4 research questions.}
\label{tab:ro4}
\centering
\begin{tabularx}{\textwidth}{X}
\toprule
 \textbf{Objective 4: Mechanize proofs of properties of the language and the system}
\tabularnewline
\midrule
RQ4.1: What are the desired properties for a VDB and do they hold for our VDB representation? (\vamos)
\tabularnewline[0.2cm]
RQ4.2: What are the desired properties for VRA
and do they hold? (In progress)
\tabularnewline[0.2cm]
RQ4.3: Is the implementation of VDBMS consistent with the semantics of VRA? (Not started)
\tabularnewline
\bottomrule
\end{tabularx}
\end{table}

\begin{comment}
\end{comment}

For RQ4.1, we identified properties that must hold for a variational database 
to ensure the variation has been encoded correctly in both the schema and content.
\secref{vdb-props} provides these properties and discusses how a database
administrator can write their own properties over a variational database and
ensure that they hold.


\begin{comment}
* type system
* explicit annotation
* var-pres
* eq rules
\end{comment}

For RQ4.2, we pay attention to the variational nature of VRA.
%
Consequently, the expressiveness of v-queries may cause them to be 
more complicated than relational queries.
%, discussed in \secref{type-sys}. 
Hence, we introduce a 
\emph{type system} for VRA that statically checks if a 
v-query conforms to the underlying v-schema and encoded variability within the VDB.
VRA's type system is formally defined in \secref{type-sys}.
%

Additionally, 
similar to other applications of variational research~\cite{CEW16ecoop,CEW14toplas},
the type system must preserve the variation encoded in a v-query.
We call this property \emph{variation preserving} with respect to the
variational schema and formally define it in \secref{var-pres}.
%
In order for the variation-preserving property to hold we need to define 
a function that decouples a query from the underlying variational schema.
We achieve this by \emph{explicitly annotating} a query with the variational
schema. This function is formally defined in \secref{constrain}.
We are currently proving this property mechanically using the Coq
proof assistant.
%

%Moreover, since variation points in a v-query can move throughout 
%the query it is important to
We provide a set of rules that minimize
the variation in a v-query. We provide some of these rules in \secref{var-min}.
We plan to mechanically prove that these equivalence rules
are correct~\footnote{The mechanical proofs are being conducted with the help of a colleague}. 
%
We also plan to define the formal semantics of VRA to prove
that the variation-preserving property also holds at the semantic 
level to ensure that queries behave as expected.




For RQ4.3, we plan to use the the formal semantics of VRA 
%to ensure that queries behave in the expected manner.
to
prove that the implementation of VRA in VDBMS aligns with
the semantics of it. 

\secref{ro5} discusses a stretch goal that we would like to pursue if time permits.

\subsubsection{Property Checking over a VDB}
\label{sec:vdb-props}


Since
a \emph{single} database can supply data for \emph{many} different
database variants \emph{at the same time}
encoding variability explicitly in a database  allows the developers to 
check for different properties over all database variants
For example, to ensure that variability associated with each variant is 
valid we provide a set of validity checks.
% constructed VDB is valid we provide a set of validity checks.
These checks ensure that the presence conditions both at the schema level and
data level are consistent and satisfiable, that is, they are present in at
least one database variant. In the following, the function $\sat\dimMeta$
denotes a satisfiability check that returns \t\ if the feature expression
\dimMeta\ is satisfiable and \f\ otherwise.


At the schema level we check the following properties:
%
\begin{enumerate}
%
\item That there is at least one valid configuration of the feature model $m$:\\
%
$\sat\fModel$
%
\item That every relation $r$ is present in at least one configuration of the
variational schema:\\
%
$\forall\vRel\in\vSch, \sat{\fModel\wedge\getPC\vRel}$
%
\item That every attribute $a$ in every relation $r$ is present in at least one
configuration of the variational schema:\\
%
$\forall\vAtt\in\vRel, \forall\vRel\in\vSch,
\sat{\fModel\wedge\getPC\vRel\wedge\getPC\vAtt}$
%
\item That if $\vSch_\config$ denotes the expected plain relational schema for
configuration $c$ of the variational schema \vSch, then configuring the
variational schema with that configuration, written $\sem[\config]{\vSch}$,
actually yields that variant:\\
%
$\forall\config\in\confSet, \sem[\config]{\vSch} = \vSch_\config$
%
\end{enumerate}


\noindent
%
At the data level we check the following properties:
%
\begin{enumerate}
%
\item That every tuple $u$ in relation $r$ is present in at least one variant:\\
%
$\forall\vTuple\in\vRel, \forall\vRel\in\vSch,
\sat{\fModel\wedge\getPC\vRel\wedge\getPC\vTuple}$ 
%
\item That for every tuple $u$ in relation $r$, if an attribute $a$ in $r$ is
not present in any variants of the tuple, then the value of that attribute in
the tuple, written $\mathit{value}_\vTuple(\vAtt)$, should be NULL:\\
$\forall\vTuple\in\vRel, \forall\vAtt\in\vRel, \forall\vRel\in\vSch,
\neg\sat{\fModel\wedge\getPC\vRel\wedge\getPC\vAtt\wedge\getPC\vTuple}
\Rightarrow \mathit{value}_\vTuple(\vAtt) = \NULL$
%
\end{enumerate}


\noindent
%
We implemented these checks in our VDBMS tool and verified that both case
studies described in \appref{db} satisfy all of them. 
%
Depending on the context of the VDB, 
more specialized properties can be checked too. For example, if temporal 
variability in a database is accumulated over variants, i.e., the old data
is also included in a more recent variant in addition to the newly added data,
it is desirable
to ensure that older variants are subsets of newer variants. 
%\TODO{I don't
%understand the thing we're assuming here.} 
Assume that configurations \ensuremath{\config_1, \config_2, \cdots}
represents time-orderly configurations, we formulated
and checked this for the employee use case:\\
\ensuremath{
\forall \config_i, \config_j \in \confSet, i \le j, \sem[\config_i] {\vDB} \subseteq \sem[\config_j]{\vDB}
}, 
where \ensuremath{\sem[\config]{\vDB}} denotes configuring a variational database instance
\vDB\ for configuration \config. 
%\parisa{note to myself, impl todo: actually check this for employee db when you got the time!}

%v-table checks:
%- \ensuremath{\forall tuple \in relation \in schema : sat (fm \wedge pc_relation \wedge pc_tuple)}\\
%- \ensuremath{\forall attribute \in relation \in schema, \forall val : if unsat (fm \wedge pc_relation \wedge pc_attribute \wedge pc_tuple)} then value must be null\\


\subsubsection{Well-Typed (Valid) Query}
\label{sec:type-sys}

%\input{formulas/vRelAlgTypingRulesExplicitAnnotation}

%\point{aspects we need to type check v-queries.}
To prevent running v-queries that have errors
we implement a \emph{static type system} for VRA. The 
type system ensures queries are \emph{well-typed}, i.e., they comply
with the underlying v-schema, both w.r.t. the traditional structure of 
the database and the variability encoded in the database. 
%
Assume we have the VDB given in \exref{conf-vq} with the only 
relation
\ensuremath{
\vRel \left( \optAtt [\fOne] [\vAtt_1], \vAtt_2, \vAtt_3 \right)^{\fOne \vee \fTwo}
}. 
Attribute \ensuremath{\vAtt_4} cannot be projected from \vRel\ because
it is not present in \vRel, thus,
the query \ensuremath{\vPrj [\vAtt_4] \vRel} is invalid.
Similarly, the query 
\ensuremath{\vPrj [{\optAtt [\neg \fOne] [\vAtt_1]}] \vRel
} has an error because \ensuremath{\vAtt_1} is not present in 
\vRel\ for 
\ensuremath{
\forall \config \in \confSet. \fSem {\neg \fOne} = \t}, but
these are the only configurations where the query desires to project attribute
\ensuremath{\vAtt_1} from \vRel.
%is invalid because the variation encoded in the query is violating the 
%corresponding variation in v-schema, i.e., the feature expression
%\ensuremath{\neg \fOne \wedge \getPC{\vAtt_1}} is not satisfiable.
%For example, while projecting an annotated attribute \optAtt\ from a 
%v-relation \vRel\ not only the attribute must belong to the v-relation, i.e., 
%$\vRel \annot [\dimMeta_\vRel] {\paran {\optAtt [\dimMeta_1], \vAttList}}$, but 
%the feature expression $\dimMeta \wedge \dimMeta_\vRel \wedge \dimMeta_1$ 
%must also be satisfiable, i.e., 
%%a database variant in the intersection of variability
%%encoded in the v-schema and the query exists s.t. attribute \vAtt\ is present in 
%%relation \vRel.
%the attribute \vAtt\ must be present in the relation \vRel\
%under the condition imposed by the query and the v-schema. 
%
%
%A v-query that is not well-typed is \emph{ill-typed}.

\begin{figure}
%\begin{minipage}[t]{0.5\textwidth}
\textbf{V-queries typing rules:}

  \begin{mathpar}
  \small
  
  \inferrule[\empRelE]
  {}
  {\env {\empRel} {\annot [\f] {\setDef \ }}}
%  \inferrule[\judge]
 % 	{\env{\vQ}{\vType}}
 %   {}
%
% explicitly-typed vra:
%    \inferrule[\relationE]
%  	{\vRel (\vType)^{\VVal \dimMeta} \in \vSch \\
%	\neg \sat{\vctx \wedge \neg \VVal \dimMeta} }
%     {\envWithSchema{\envInContext [\vctx ] {\vType}}}

%implicitly-typed lang:
    \inferrule[\relationE]
  	{ \vRel (\vType)^{\VVal \dimMeta} \in \vSch \\
	\sat {\vctx \wedge \getPCfrom \vRelSch \vSch}}
%	\sat{\vctx \wedge \VVal \dimMeta} }
     {\envWithSchema{\envInContext [\vctx \wedge \VVal \dimMeta] {\vType}}}

% explicitly-typed vra:  
%  \inferrule[\prjE]
%  	{\envPrime \\
%    	\subsume {\annot \vType}  {\annot [\VVal \vctx] {\VVal \vType}}}
%    {\env{\vPrj[\vType] \vQ} {\envInContext [\vctx] \vType}}

%implicitly-typed lang:
  \inferrule[\prjE]
  	{\envPrime \\
	|\pushIn {\annot \vType}| = | \vType | \\
%	\pushIn {\annot \vType} \neq \setDef \ \\
%	\pushIn {\annot [\VVal \dimMeta] {\VVal \vType}} \neq \setDef \ \\
    	\subsume { \vType}  {\pushIn {\annot [\VVal \vctx] {\VVal \vType}}}}
    {\env{\vPrj[\vType] \vQ} {\envInContext [\VVal \vctx] {\left(\vType \cap {\VVal \vType} \right)}}}
%    the older version 8/8/20:
%    	\subsume {\pushIn {\annot \vType}}  {\pushIn {\annot [\VVal \vctx] {\VVal \vType}}}}
%    {\env{\vPrj[\vType] \vQ} {\envInContext [\VVal \vctx] {\left(\pushIn{\annot {\vType}} \cap {\VVal \vType} \right)}}}


  \inferrule[\selE]
  	{\env \vQ {\envInContext [\VVal \vctx] \vType} \\
    	\envCondAnnot \vCond}
    {\env{\vSel \vQ}{\envInContext [\VVal \vctx] \vType}}
    
  \inferrule[\choiceE]
  	{\envOne[\vctx \wedge \VVal \dimMeta] \\
    	\envTwo[\vctx \wedge \neg \VVal \dimMeta]}
    {\env{\chc[\VVal \dimMeta]{\vQ_1, \vQ_2}}{
     \envInContext [(\vctx_1 \wedge \VVal \dimMeta) \vee (\vctx_2 \wedge \neg \VVal \dimMeta)] 
     {\left({\pushIn {\envInContext [\vctx_1] \vType_1}} \cup
    							{\pushIn {\envInContext [\vctx_2] \vType_2}}\right)}}}
% older version 8/8/20:
%    {\env{\chc[\VVal \dimMeta]{\vQ_1, \vQ_2}}{
%     \envInContext [(\vctx_1 \wedge \VVal \dimMeta) \vee (\vctx_2 \wedge \neg \VVal \dimMeta)] 
%     {\left({\pushIn {\envInContext [\vctx_1 \wedge \VVal \dimMeta] \vType_1}} \cup
%    							{\pushIn {\envInContext [\vctx_2 \wedge \neg \VVal \dimMeta] \vType_2}}\right)}}}
    
  \inferrule[\productE]
  	{\envOne \\
    	\envTwo\\
	\pushIn {\annot [\vctx_1] \vType_1} \cap \pushIn {\annot [\vctx_2] \vType_2} = \{\}}
    {\env{\vQ_1 \times \vQ_2}{\envInContext [\vctx_1 \wedge \vctx_2] 
      {\left(\pushIn {\annot [\vctx_1] \vType_1} \cup \pushIn {\annot [\vctx_2] \vType_2} \right)}}}


  \inferrule[\setopE]
  	{\envOne \\
    	\envTwo \\
	\envEval {\pushIn {\annot [\vctx_1] \vType_1}} {\pushIn {\annot [\vctx_2] \vType_2}}}
%        \envEval{\envInContext{\vType_1}} \vType \\
%        \envEval{\envInContext{\vType_2}} \vType}
    {\env{\vQ_1 \circ \vQ_2} {\envInContext [\vctx_1] \vType_1} }

%  \inferrule[\diffE]
%  	{\envOne \\
%    	\envTwo \\
%        \envEval{\envInContext{\vType_1}} \vType \\
%        \envEval{\envInContext{\vType_2}} \vType}
%    {\env{\vQ_1 \setminus \vQ_2} \vType}
  \end{mathpar}
  
\medskip
\textbf{V-condition typing rules:}
% (b: boolean tag, \pAtt: plain attribute, k: constant value):}
%(b: boolean tag, A: plain attribute, k: constant value)}
  \begin{mathpar}
  \small    

  \inferrule[\conjC]
  	{\envCond \vCond_1\\
    	\envCond \vCond_2}
    {\envCond{\vCond_1 \wedge \vCond_2}}
    
  \inferrule[\disjC]
  	{\envCond \vCond_1\\
    	\envCond \vCond_2}
    {\envCond{\vCond_1 \vee \vCond_2}}
    


  \inferrule[\choiceC]
%  	{\defType{\relInContext{\vContext''}}\in \vSch \\
    	{\envCond[\vctx \wedge \VVal \dimMeta, \vType]{\vCond_1} \\
        \envCond[\vctx \wedge \neg \VVal \dimMeta, \vType]{\vCond_2}}
    {\envCond{\chc[\VVal \dimMeta]{\vCond_1, \vCond_2}}}
    

  \inferrule[\notC]
  	{\envCond \vCond}
    {\envCond \neg \vCond}
        
%  \inferrule[]
%  	{\envCond[\vContext \wedge \dimMeta]{\vCond_1} \\
%    	\envCond[\vContext \wedge \neg\dimMeta]{\vCond_2}}
%    {\envCond{\chc{\vCond_1, \vCond_2}}}
    

    
  \inferrule[\attValC]
  	{
	%\defType{\relInContext{\vContext'}}\in \vSch \\
    	\optAtt [\VVal \dimMeta] \in \vType \\
%	\taut{{\VVal \dimMeta} \imply \vctx} \\
        \sat {\VVal \dimMeta \wedge \vctx}}
%        \\
%        \cte \in \dom \vAtt}
    {\envCond{\op \pAtt \cte}}
    
  \inferrule[\boolC]
  	{}
    {\envCond \bTag}
    

    
  \inferrule[\attAttC]
  	{
	%\defType{\relInContext{\vContext'}}\in \vSch \\
    	\optAtt [\dimMeta_1] [\vAtt_1]\in \vType \\
         {\optAtt [\dimMeta_2] [\vAtt_2]} \in \vType \\
%         \taut{\dimMeta_1 \imply \vctx} \\
%         \taut{\dimMeta_2 \imply \vctx} \\
        \sat { \dimMeta_1 \wedge \dimMeta_2 \wedge \vctx}}
%        \\
%        \type[\vAtt_1] = \type[ \vAtt_2]}
    {\envCond{\op{\pAtt_1}{\pAtt_2}}}
    
  \end{mathpar}

%\caption{V-condition typing relation. A v-condition \vCond\ is well-typed if 
%it is valid in the variational context \vctx\ and type environment \vType, i.e., 
%\envCond \vCond. Note that the type rules for v-conditions return a boolean, if
%the v-condition is type-correct the rules return \t, otherwise they return \f.}
\caption[\TODO{shortcaption}]{VRA and v-condition typing relation. 
The rules assume that the underlying VDB is well-formed. 
Remember that our theory assumes all attributes have the same type
and all constants belong to attributes' domain. 
%The typing rule of a join query is the combination
%of rules \selE\ and \productE.
}
\label{fig:vq-stat-sem}
%\end{minipage}
\end{figure}


%\input{formulas/vRelAlgTypingRulesExplicitAnnotation}

\figref{vq-stat-sem} defines VRA's \emph {typing relation}
as a set of inference rules assigning \emph{types} to queries. 
%
The type of a query is a v-relation schema \ensuremath{\mathit{result}\annot {(\vAttList)}},
however, for brevity and since the relation name is the same for all 
queries we consider the type of a query an annotated v-set of attributes
where attributes are projected by the query from the VDB
and their presence conditions determine their valid variants.
The presence condition of attributes in the type of a query may differ 
from their presence conditions in v-schema due to variation constraints
imposed by the query.
For example, continuing with relation
\ensuremath{
\vRel \left( \optAtt [\fOne] [\vAtt_1], \vAtt_2, \vAtt_3 \right)^{\fOne \vee \fTwo}
},
the query \ensuremath{\vPrj [{\optAtt [\fOne] [\vAtt_2]}] \vRel} has 
the type \ensuremath{\{\annot [\fOne] {\vAtt_2} \}^{\fOne \vee \fTwo}} while 
according to \vRel's schema 
\ensuremath{\getPC {\vAtt_2} = \fOne \vee \fTwo}, i.e., the presence
condition of attribute \ensuremath{{\vAtt_2}} changes through the query.
The presence condition of the entire set determines the condition under
which the entire table (i.e., attributes and tuples) are valid. 
Note that it is essential to consider the type of a query an \emph{annotated}
v-set to account for the presence condition of the entire table.
%similar to how we encode v-relation schemas. 
%If we
%consider the type of a query a variational attribute set we lose information (i.e.,
%the condition under which tuples are valid).
%The final variation context, after running a v-query, is the
%presence condition of the returned v-table. That is why we 
%consider the type environment as a variational set of attributes instead of 
%a relation schema. 

%\begin{example}
%\label{eg:vq-affect-vctx}
%Assume we have the VDB defined in \exref{vsch}. 
%Consider the query $\pi_{\name} \empbio$. The returned 
%table has the type: $(\optAtt [\vFour] [\name])^\fModel$. 
%However, if we change the query to: 
%$\pi_{\optAtt [\edu] [\name]} \empbio$, the returned table has the
%type: $(\optAtt [\vFour \wedge \edu] [\name])^\fModel$. 
%\end{example}

%
%\eric{Eric, feel free to summarize rule explanations. I basically
%explained them how I'd read them.}
VRA's typing relation, as defined in \figref{vq-stat-sem}, 
has the judgement form \env \vQ {\envInContext [\VVal \vctx] \vType}. 
This states
that in \emph \vctxTxt\ \vctx\ within v-schema \vSch, 
v-query \vQ\ has type \envInContext [\VVal \vctx] \vType. 
If a query does not have a type, it is \emph{ill-typed}.
\emph{Variation context} is a feature expression that the type system 
keeps and refines to keep track of variation encoded by a query.
%To capture the variation encoded in a query,
%we keep and refine a feature expression, called a \emph{variation context}.
The variation context is initiated by the feature model. For brevity,
we use the judgment form \envWithoutVctx \vQ {\envInContext [\VVal \vctx] \vType}\
for  \env [\getPC \vSch] \vQ {\envInContext [\VVal \vctx] \vType}
i.e., the variation context is initialized. Note that attributes with an
unsatisfiable presence condition are not present in any 
database variant, i.e., they are not present for any configuration.
Thus, the existence of such attribute in a type does not change
the type semantically, based on the defined equivalence rule for 
v-sets, given in \defref{vset-eq}. Hence, we do not filter out such attributes
explicitly in \figref{vq-stat-sem}, however, for simplicity, 
the implemented type system
drops the attributes with an unsatisfiable presence condition.
 

%
The rule \relationE\ states that, in \vctxTxt\ \vctx\ with
underlying \vschTxt\ \vSch, assuming that
1) \vSch\ contains
the relation \vRel\ with presence condition $\VVal \dimMeta$
and v-set of attributes \vType\ 
and
2) there exists a valid variant in the intersection of variation context \vctx\
and \vRel's presence condition \ensuremath{\VVal \dimMeta}, i.e., 
\ensuremath {\sat {\vctx \wedge \VVal \dimMeta}},
%\vctxTxt\ \vctx\ of query \vRel\
%is more specific than the \presCondTxt\ of the relation $\VVal \dimMeta$, denoted by
%$\taut {\vctx \to \VVal \dimMeta}$, 
then query \vRel\ has type \vType\ annotated with \ensuremath { \vctx \wedge \VVal \dimMeta}.
%denoted by \envInContext \vType, formally defined in \defref{vCtxtAppliedType}.

% 
The rule \prjE\ states that, in \vctxTxt\ \vctx\ within v-schema \vSch, assuming 
that the subquery \vQ\ has type $\envInContext [\VVal \vctx] {\VVal \vType}$,
v-query $\pi_\vType \vQ$
has type \ensuremath {\envInContext [\VVal \vctx] {\left( \vType \cap \VVal \vType\right)}}, 
if  all attributes in \vType\ are present in \vctx\
%\pushIn {\annot \vType} is not an empty v-set
and
 \ensuremath {\pushIn {\envInContext [\VVal \vctx] {\VVal \vType}}} subsumes \vType.
%the variational 
%attribute set \vType\ constrained (annotated) with variational context \vctx, i.e., \annot \vType.
The subsumption, defined in \defref{vset-subsumption},
ensures that the subquery \vQ\ does not have an empty type
and it includes all attributes in 
the projected attribute set and attributes' presence conditions do not 
contradict each other. Returning the intersection of types, defined in 
\defref{vset-intersect}, filters both 
attributes and their presence conditions.
\exref{type} illustrates generating the type of a query step by step.


\begin{example}
\label{eg:type}
We illustrate how a query enforces variation encoded within it to the result.
We do this by illustrating how the type system generates two different 
types for queries
\ensuremath{\vQ_1} and \ensuremath{\vQ_2}
given in \exref{vq-specific}.
%\footnote{Derivation trees of these examples can be fine here! 
%%\TODO{derive the trees using the new rules.}
%\TODO{Eric, do we need them? If yes, where should we put them?}}. 
For brevity, we simplify feature expressions when possible.
%
For \ensuremath{\vQ_1}, it applies the \prjE\ rule under
the variation context initiated to 
\ensuremath{\fModel_2 = \vThree \oplus \vFour \oplus \vFive}
and schema \ensuremath{\vSch_2}.
It now has to apply the
\relationE\ rule to the subquery \empbio\ under the same variation context and schema,
resulting in the type
\ensuremath{
\vAttList_\empbio =  \{\empno, \sex, \birthdate,}
\ensuremath{ 
\optAtt [\vFour] [\name], \optAtt [\vFive] [\fname], \optAtt [\vFive] [\lname]\}^{\fModel_2}}.
Now that it has the type of the subquery \empbio\ 
it verifies that the projected attribute v-set
% annotated with the variation context, i.e.,
\ensuremath{
\vAttList_{\mathit{prj}} =
 \{\optAtt [\vFour \vee \vFive] [\empno],
\name,}
\ensuremath{ \fname, \lname\}^{\fModel_2}},
%^{\vThree \oplus \vFour \oplus \vFive}},
is subsumed by \ensuremath{\vAttList_\empbio}. 
Thus, it generates the type of query \ensuremath{\vQ_1} by
intersecting \ensuremath{\vAttList_{\mathit{prj}}} and \ensuremath{\vAttList_\empbio}
annotated with \ensuremath{\vAttList_\empbio}'s presence condition resulting in the type
\ensuremath{
\vAttList_{\vQ_1} = 
\{\optAtt [\vFour \vee \vFive] [\empno],
\optAtt [\vFour] [\name], }
\ensuremath{
\optAtt [\vFive] [\fname], \optAtt [\vFive] [\lname]\}^{\fModel_2}}.
%
This type demonstrates the structure of the result of query \ensuremath{\vQ_1}.
%
As for \ensuremath{\vQ_2}, the type system applies the \choiceE\ rule
under the variation 
context initiated to \ensuremath{\fModel_2} and schema \ensuremath{\vSch_2}.
It then applies the \prjE\ and \empRelE\ rules to the left and right
alternatives of the choice, respectively, which generates the types
\ensuremath{
\vType_\mathit{left} = \annot [\fModel_2 \wedge (\vFour \vee \vFive)] {(\empno, \annot [\vFour] \name,
 \annot [\vFive] \fname,\annot [\vFive] \lname)}}
and \ensuremath{\vType_\mathit{right} = \annot [\f] {\setDef \ }}, respectively.
Finally, it generates the type of \ensuremath{\vQ_2} by 
annotating the union of \ensuremath{\vType_\mathit{left}} and \ensuremath{\vType_\mathit{right}}
with \ensuremath{\fModel_2 \wedge (\vFour \vee \vFive)}, resulting in the 
final type of \\
\ensuremath{\vType_{\vQ_2} = 
\annot [\fModel_2 \wedge (\vFour \vee \vFive)] {(\empno, \annot [\vFour] \name,
 \annot [\vFive] \fname,\annot [\vFive] \lname)}}.
 Note that \ensuremath{\vType_{\vQ_2}}'s presence condition 
 explicitly accounts for only two variants
 while \ensuremath{\vType_1} does not do so even though \ensuremath{\vQ_1}
 does not return any tuple that belong to variant \ensuremath{\setDef \vThree} because
 of its attributes presence condition. 
\end{example}


%
The rule \selE\ states that, in \vctxTxt\ \vctx\ within v-schema \vSch, assuming 
that the subquery \vQ\ has type {\envInContext [\VVal \vctx] \vType}, 
the v-query $\sigma_{\vCond} \vQ$
has type {\envInContext [\VVal \vctx] \vType},
if the \vCondTxt\ \vCond\ is well-formed w.r.t.
 \vctxTxt\ \vctx\ and \tenvTxt\ {\envInContext [\VVal \vctx] \vType}, 
denoted by v-condition's typing relation 
\envCondAnnot \vCond.
Note that in variational condition typing rules, 
the presence condition of the query type is pushed in.
% applied to the 
%variational attribute set, thus, they have the form \envCond \vCond\ 
%instead of \envCondAnnot \vCond. 
The rules state that attributes used in a
\vCondTxt\ must be valid in \vType\ and 
attribute's \presCondTxt\ \ensuremath {\VVal \dimMeta} 
in type \vType\ must exists within \vctxTxt\ \vctx,
denoted by \ensuremath{\sat {\VVal \dimMeta \wedge \vctx}}.
%be more specific than the \vctxTxt\ \vctx,
%denoted by \ensuremath{\taut {\VVal \dimMeta \to \vctx}}, 
%since \vType\ is the exact type and specification of the subquery within
%a selection query which is at least as specific as the \vctxTxt\ under which
%the selection query is written. 
%the \fexpTxt\ attached to an attribute in a 
%\vCondTxt\ must be more specific than its \presCondTxt\ in type \vType. 
They also
check the constraints of traditional relational databases, such as the type of two 
compared attributes must be the same.

%
The rule \choiceE\ states that, in \vctxTxt\ \vctx\ within v-schema \vSch, the type of 
a choice of two subqueries is the \emph{union of types}, defined in 
\defref{vset-union}, of its subqueries annotated with the disjunction of their presence
conditions conjuncted with the corresponding condition of the choice's dimension.
%annotated 
%with their corresponding \vctxTxt s, which depends on the feature expression of the choice
%query $\VVal \dimMeta$. 
A choice query is well-typed iff both of 
its subqueries $\vQ_1$ and $\vQ_2$ are well-typed.
%Note that we do not simplify 
%\ensuremath{
%\annot [\vctx_1 \vee \vctx_2] {\left(\envInContext [\vctx_1] \vType_1 \cup \envInContext [\vctx_2] \vType_2\right)}
%}
%to 
%\ensuremath{
%\envInContext [\vctx_1] \vType_1 \cup \envInContext [\vctx_2] \vType_2
%}
%because we need to know the presence condition of the entire type, i.e., \ensuremath{\vctx_1 \vee \vctx_2},
%to know the condition under which tuples are valid\footnote{The simplification holds because
%\ensuremath{
%\annot [\vctx_1 \vee \vctx_2] {\left(\envInContext [\vctx_1] \vType_1 \cup \envInContext [\vctx_2] \vType_2\right)}
%\equiv
%\envInContext [\vctx_1\wedge (\vctx_1 \vee \vctx_2)] \vType_1 \cup \envInContext [\vctx_2\wedge (\vctx_1 \vee \vctx_2)] \vType_2
%\equiv
%\envInContext [\vctx_1] \vType_1 \cup \envInContext [\vctx_2] \vType_2
%}
%}.
Note that \choiceE\ is the only rule that refines the variation context. 


% 
The rule \productE\ states that the type of a product query in \vctxTxt\
\vctx\ is the union of the type of its subqueries annotated with the 
conjunction of their presence conditions, assuming that 
they are disjoint. 
%Note that 
%\ensuremath{
%\annot [\vctx_1 \wedge \vctx_2] {\left(\envInContext [\vctx_1] \vType_1 \cup \envInContext [\vctx_2] \vType_2\right)}
%\equiv 
%\envInContext [\vctx_1 \wedge \vctx_2] \vType_1 \cup \envInContext [\vctx_1 \wedge \vctx_2] \vType_2
%\equiv
%\annot [\vctx_1 \wedge \vctx_2] {\left(\vType_1 \cup \vType_2\right)}
%}.

% 
The rule \setopE\ denotes the typing rule for set operation queries such as 
union and difference. It states that, if the subqueries $\vQ_1$ and $\vQ_2$
have types $\envInContext [\vctx_1] \vType_1$ and 
$\envInContext [\vctx_2] \vType_2$, respectively, in \vctxTxt\ \vctx,
then the v-query of their set operation has type $\envInContext [\vctx_1] \vType_1$, iff 
$\pushIn {\envInContext [\vctx_1] \vType_1}$ and $\pushIn {\envInContext [\vctx_2] \vType_2}$ are \emph{equivalent}.
%their annotated type with
%\vctx\ are \emph{equivalent} to \vType. 
The \emph{type equivalence} is v-set equivalence, defined in \defref{vset-eq},
for v-sets of attributes.


%9-19-18 notes:
%* type soundness theorem: 
%    * ideally we want to prove this, but at least we should formulate it
%    * the way you do this usually, is that you have a semantics and you want to show 
%        * progress: if your program is well-typed your either done evaluating it or you can keep evaluate it -> in our case is kind of given, because we don't have a turing complete language!!
%        * preservation: when we evaluate sth it doesn't change its type! -> what we want to show is the relation that semantics gives us back fits the schema that the type system gives us. so there is consistency between type system and semantics. 
%    * so in the context of variational queries what that would mean is that when we evaluate a query, the relation that we get back actually has the type that we said it has via the type system
%    * PROBLEM: our semantics is via SQL so proving this will be kind of hairy but we should at least write this down and try to convince ourselves that it's correct, and we'll figure out what we can say in the context of the paper.
%    * [TODO] writing test cases that it?s correct!! in terms of mini database at two levels, haskell and database. the test cases will be queries against this mini database with lots of variability in it. you can write a quick check property. so you basically are testing the commuting diagram, so you either:
%        1. configure the query first and then give its type
%        2. give its variational type first and then configure it
%        3. and you should get the same thing from both
 


%The purpose of establishing \emph{type safety} is to ensure that the
%static and dynamic semantics are consistent with each other. 
%@Eric do we need to say anything about why we don't take the standard 
%approach?
%\NEED{Since 
%we define \vqTxt\ dynamic semantics in terms of relational algebra,
%i.e. we translate a \vqTxt\ into a set of relational algebra expression 
%and then combine the result of them into a \vrelTxt, we do not take 
%the standard approach of defining and proving \emph{progress} and 
%\emph{preservation} properties.
%}

%We 
%follow the approach developed by \TODO{fill in later!!}, which distinguishes
%two type safety properties, preservation and progress. The preservation
%theorem establishes that \vqsTxt\ preserve type assignments, i.e. that the
%type of a \vqTxt\ accurately predicates the type of the result of evaluating 
%that \vqTxt. \TODO{state the context and type you start out with}


%pierce def:
%progress: a well-typed term is not stuck (either it is a value or it can take a step 
%according to the evaluation rules)
%\begin{theorem}
%\label{thm:progress}
%Suppose \vQ\ is a closed, well-typed \vqTxt\ (that is, \env {\vQ} {\vType} for some 
%\vType). Then either \vQ\ is a value (a \vrelTxt ) or else there is some 
%\TODO{in order to define this we need to define:\\
%small step of relational algebra\\
%translation rules to rel alg\\
%combining the set of queries resulted from translation to output a var table\\
%canonical forms\\
%
%\end{theorem}


%pierce def:
%preservation: if a well-typed term takes a step of evaluation, then the resulting 
%term is also well typed
%\begin{theorem}
%\label{thm:type-pres}
%
%\end{theorem}

%\TODO{double check the following ph and write the properties based on it:}
%Conceptually, a variational query describes a query that can
%be executed over any database instance consistent with the 
%variational schema. The property that must hold between a 
%variational query \vQ\ and a variational schema \vSch
%is that for every plain query $\pQ_\config$ obtained from \vQ 
%by configuring with a function $\config: \fSet \to \bSet$, $\pQ_\config$ 
%is consistent with the corresponding plain schema $\pSch_\config$ 
%obtained by $\osSem  \vSch$ with the same function \config. That 
%is, every variant query matches the corresponding variant schema.
%%\REMEMBER{In \secref{prop-q-lang}, we prove that our query language,
%%variational relational algebra, can encode this idea and also
%%recover any of the conceptually potential results for any instance
%%of the variational database.}

\section{Explicitly Annotating Queries}
\label{sec:constrain}

%\point{type system allows the ql to be flexible and usable.}
%The type system is designed s.t. it relieves the user from necessarily incorporating
%the v-schema variability into their queries as long as the variational queries variability
%does not violate the v-schema, 
Variational queries do not need to repeat information that can be inferred from the variational schema
or the type of a query.
%
For example, the query \ensuremath{\vQ_1} shown in \exref{vq-specific} 
does not contradict the schema and
thus is type correct. However,
 it does not include the presence conditions of attributes and the relation encoded in
the schema while \ensuremath{\vQ_6} repeats this information:\\
%
\centerline{
\ensuremath{
\vQ_6 =
\pi_{\optAtt [(\vFour \vee \vFive) \wedge \neg \vThree] [\empno], \optAtt [\neg \vThree \wedge \vFour \wedge \neg \vFive] [\name], \optAtt [\neg \vThree \wedge \neg \vFour \wedge \vFive] [\fname], \optAtt [\neg \vThree \wedge \neg \vFour \wedge \vFive] [\lname]  } \left(\chc [\dimMeta_2] {\empbio, \empRel} \right)}}.%
%
\footnote{
The query $\vQ_6$ is the simplified version of
\[\constrain [\vSch_2] {\vQ_1} = 
\pi_{\optAtt [(\vFour \vee \vFive) \wedge \neg \vThree] [\empno], \optAtt [\neg \vThree \wedge \vFour \wedge \neg \vFive] [\name], \optAtt [\neg \vThree \wedge \neg \vFour \wedge \vFive] [\fname], \optAtt [\neg \vThree \wedge \neg \vFour \wedge \vFive] [\lname]  } \left(\constrain [\vSch_2] {\empbio}\right)\]
where $\constrain [\vSch_2] {\empbio} = \chc [\dimMeta_2] {\pi_{\empno, \annot [\vFour] \name, \annot [\vFive] \fname, \annot [\vFive] \lname} (\empbio),\empRel} $.
}

%\pi_{\optAtt [(\vFour \vee \vFive) \wedge \fModel_2] [\empno], \optAtt [\vFour \wedge \fModel_2] [\name], \optAtt [\vFive \wedge \fModel_2] [\fname], \optAtt [\vFive \wedge \fModel_2] [\lname]  } \empbio}}.
%

%\NOTE{
%This is the unsimplified version:
%\begin{align*}
%\VVal {\vQ_5} &= 
%\pi_{\optAtt [\vFour \vee \vFive] [\empno], \optAtt [\vFour] [\name], \optAtt [\vFive] [\fname], \optAtt [\vFive] [\lname]  } \\
%&(\chc [ \fModel_2 ] {\pi_{\empno, \sex, \birthdate, \optAtt [\vFour ] [\name], \optAtt [\vFive] [\fname], \optAtt [\vFive] [\lname]} \empbio, \empRel  })
%\end{align*}
%}
Similarly, the projection in the query 
\ensuremath{\vQ_7 = \pi_{\name, \fname} (\mathit{subq}_7)}
where 
\ensuremath{
\mathit{subq}_7 = \chc [ \vFour] {\pi_\name (\vQ_6), \pi_\fname (\vQ_6)}
}
is written over 
\ensuremath{\vSch_2} and it 
%\centerline{
%\ensuremath{
%\vQ_6 =
%\pi_{\name, \fname} \mathit{subq}_6
%} 
%}}
does not repeat the presence conditions of attributes from its \ensuremath{\mathit{subq}_7}'s type.
The query
%\centerline{
\ensuremath{
\vQ_8 =
\pi_{\optAtt [\vFour ] [\name],\optAtt [\neg \vFour] [\fname]} (\mathit{subq}_7)
%\chc [ \vFour] {\pi_\name \vQ_5, \pi_\fname \vQ_5}
}
%}
makes the annotations of projected attributes \emph{explicit} with respect to both 
the variational schema \ensuremath{\vSch_2} and its subquery's type.
%\TODO {give an example, schema: R(A,B), query: $\pi_{A,B} (F<\pi_A R, \pi_B R>)$
%becomes $\pi_{A^F, B^{\neg F}} ...$}
%The variation encoded in variational queries can
%be more restrictive or more loose than v-schema variation without violating them.
Although relieving the user from explicitly repeating variation makes VRA easier to use, 
queries still have to state variation explicitly to avoid losing information when 
decoupled from the schema.
%We do this by defining a function, 
%\ensuremath {\constrain \vQ}, with type \ensuremath{ \qSet \to \vSchSet \to \qSet
%},
%that \emph{explicitly annotates a query \vQ\ given the underlying schema \vSch}.
We do this by defining the function 
\ensuremath {\constrain \vQ : \qSet \totype \vSchSet \totype \qSet
},
that \emph{explicitly annotates a query \vQ\ with the  schema \vSch}.
%Note that \ensuremath {\constrain \vQ} needs to take the underlying schema as
%an input since it is using the type system (which relies on the schema) as a helper function.
The explicitly annotating query function, 
formally defined in \figref{constrain}, 
conjoins attributes and relations
presence conditions with the corresponding annotations in the query 
and wraps subqueries in a choice when needed. 
Note that, $\vQ_8$ and $\vQ_6$ are the result of $\constrain [\vSch_2] {\vQ_7}$
and $\constrain [\vSch_2] {\vQ_1}$, respectively, after simplification~\footnote{More specifically,
they are simpilified using rules defined in \figref{var-min}}.
%Queries $\vQ_7$ and $\vQ_5$ are examples of applying the 
%explicitly annotation function to queries $\vQ_6$ and $\vQ_1$, respectively,
%after simplifying them.
%\exref{constrain} illustrates how the constrain function transforms queries
%and allows users to be more flexible with their queries. 

\section{Explicitly Annotating Queries}
\label{sec:constrain}

%\point{type system allows the ql to be flexible and usable.}
%The type system is designed s.t. it relieves the user from necessarily incorporating
%the v-schema variability into their queries as long as the variational queries variability
%does not violate the v-schema, 
Variational queries do not need to repeat information that can be inferred from the variational schema
or the type of a query.
%
For example, the query \ensuremath{\vQ_1} shown in \exref{vq-specific} 
does not contradict the schema and
thus is type correct. However,
 it does not include the presence conditions of attributes and the relation encoded in
the schema while \ensuremath{\vQ_6} repeats this information:\\
%
\centerline{
\ensuremath{
\vQ_6 =
\pi_{\optAtt [(\vFour \vee \vFive) \wedge \neg \vThree] [\empno], \optAtt [\neg \vThree \wedge \vFour \wedge \neg \vFive] [\name], \optAtt [\neg \vThree \wedge \neg \vFour \wedge \vFive] [\fname], \optAtt [\neg \vThree \wedge \neg \vFour \wedge \vFive] [\lname]  } \left(\chc [\dimMeta_2] {\empbio, \empRel} \right)}}.%
%
\footnote{
The query $\vQ_6$ is the simplified version of
\[\constrain [\vSch_2] {\vQ_1} = 
\pi_{\optAtt [(\vFour \vee \vFive) \wedge \neg \vThree] [\empno], \optAtt [\neg \vThree \wedge \vFour \wedge \neg \vFive] [\name], \optAtt [\neg \vThree \wedge \neg \vFour \wedge \vFive] [\fname], \optAtt [\neg \vThree \wedge \neg \vFour \wedge \vFive] [\lname]  } \left(\constrain [\vSch_2] {\empbio}\right)\]
where $\constrain [\vSch_2] {\empbio} = \chc [\dimMeta_2] {\pi_{\empno, \annot [\vFour] \name, \annot [\vFive] \fname, \annot [\vFive] \lname} (\empbio),\empRel} $.
}

%\pi_{\optAtt [(\vFour \vee \vFive) \wedge \fModel_2] [\empno], \optAtt [\vFour \wedge \fModel_2] [\name], \optAtt [\vFive \wedge \fModel_2] [\fname], \optAtt [\vFive \wedge \fModel_2] [\lname]  } \empbio}}.
%

%\NOTE{
%This is the unsimplified version:
%\begin{align*}
%\VVal {\vQ_5} &= 
%\pi_{\optAtt [\vFour \vee \vFive] [\empno], \optAtt [\vFour] [\name], \optAtt [\vFive] [\fname], \optAtt [\vFive] [\lname]  } \\
%&(\chc [ \fModel_2 ] {\pi_{\empno, \sex, \birthdate, \optAtt [\vFour ] [\name], \optAtt [\vFive] [\fname], \optAtt [\vFive] [\lname]} \empbio, \empRel  })
%\end{align*}
%}
Similarly, the projection in the query 
\ensuremath{\vQ_7 = \pi_{\name, \fname} (\mathit{subq}_7)}
where 
\ensuremath{
\mathit{subq}_7 = \chc [ \vFour] {\pi_\name (\vQ_6), \pi_\fname (\vQ_6)}
}
is written over 
\ensuremath{\vSch_2} and it 
%\centerline{
%\ensuremath{
%\vQ_6 =
%\pi_{\name, \fname} \mathit{subq}_6
%} 
%}}
does not repeat the presence conditions of attributes from its \ensuremath{\mathit{subq}_7}'s type.
The query
%\centerline{
\ensuremath{
\vQ_8 =
\pi_{\optAtt [\vFour ] [\name],\optAtt [\neg \vFour] [\fname]} (\mathit{subq}_7)
%\chc [ \vFour] {\pi_\name \vQ_5, \pi_\fname \vQ_5}
}
%}
makes the annotations of projected attributes \emph{explicit} with respect to both 
the variational schema \ensuremath{\vSch_2} and its subquery's type.
%\TODO {give an example, schema: R(A,B), query: $\pi_{A,B} (F<\pi_A R, \pi_B R>)$
%becomes $\pi_{A^F, B^{\neg F}} ...$}
%The variation encoded in variational queries can
%be more restrictive or more loose than v-schema variation without violating them.
Although relieving the user from explicitly repeating variation makes VRA easier to use, 
queries still have to state variation explicitly to avoid losing information when 
decoupled from the schema.
%We do this by defining a function, 
%\ensuremath {\constrain \vQ}, with type \ensuremath{ \qSet \to \vSchSet \to \qSet
%},
%that \emph{explicitly annotates a query \vQ\ given the underlying schema \vSch}.
We do this by defining the function 
\ensuremath {\constrain \vQ : \qSet \totype \vSchSet \totype \qSet
},
that \emph{explicitly annotates a query \vQ\ with the  schema \vSch}.
%Note that \ensuremath {\constrain \vQ} needs to take the underlying schema as
%an input since it is using the type system (which relies on the schema) as a helper function.
The explicitly annotating query function, 
formally defined in \figref{constrain}, 
conjoins attributes and relations
presence conditions with the corresponding annotations in the query 
and wraps subqueries in a choice when needed. 
Note that, $\vQ_8$ and $\vQ_6$ are the result of $\constrain [\vSch_2] {\vQ_7}$
and $\constrain [\vSch_2] {\vQ_1}$, respectively, after simplification~\footnote{More specifically,
they are simpilified using rules defined in \figref{var-min}}.
%Queries $\vQ_7$ and $\vQ_5$ are examples of applying the 
%explicitly annotation function to queries $\vQ_6$ and $\vQ_1$, respectively,
%after simplifying them.
%\exref{constrain} illustrates how the constrain function transforms queries
%and allows users to be more flexible with their queries. 

\section{Explicitly Annotating Queries}
\label{sec:constrain}

%\point{type system allows the ql to be flexible and usable.}
%The type system is designed s.t. it relieves the user from necessarily incorporating
%the v-schema variability into their queries as long as the variational queries variability
%does not violate the v-schema, 
Variational queries do not need to repeat information that can be inferred from the variational schema
or the type of a query.
%
For example, the query \ensuremath{\vQ_1} shown in \exref{vq-specific} 
does not contradict the schema and
thus is type correct. However,
 it does not include the presence conditions of attributes and the relation encoded in
the schema while \ensuremath{\vQ_6} repeats this information:\\
%
\centerline{
\ensuremath{
\vQ_6 =
\pi_{\optAtt [(\vFour \vee \vFive) \wedge \neg \vThree] [\empno], \optAtt [\neg \vThree \wedge \vFour \wedge \neg \vFive] [\name], \optAtt [\neg \vThree \wedge \neg \vFour \wedge \vFive] [\fname], \optAtt [\neg \vThree \wedge \neg \vFour \wedge \vFive] [\lname]  } \left(\chc [\dimMeta_2] {\empbio, \empRel} \right)}}.%
%
\footnote{
The query $\vQ_6$ is the simplified version of
\[\constrain [\vSch_2] {\vQ_1} = 
\pi_{\optAtt [(\vFour \vee \vFive) \wedge \neg \vThree] [\empno], \optAtt [\neg \vThree \wedge \vFour \wedge \neg \vFive] [\name], \optAtt [\neg \vThree \wedge \neg \vFour \wedge \vFive] [\fname], \optAtt [\neg \vThree \wedge \neg \vFour \wedge \vFive] [\lname]  } \left(\constrain [\vSch_2] {\empbio}\right)\]
where $\constrain [\vSch_2] {\empbio} = \chc [\dimMeta_2] {\pi_{\empno, \annot [\vFour] \name, \annot [\vFive] \fname, \annot [\vFive] \lname} (\empbio),\empRel} $.
}

%\pi_{\optAtt [(\vFour \vee \vFive) \wedge \fModel_2] [\empno], \optAtt [\vFour \wedge \fModel_2] [\name], \optAtt [\vFive \wedge \fModel_2] [\fname], \optAtt [\vFive \wedge \fModel_2] [\lname]  } \empbio}}.
%

%\NOTE{
%This is the unsimplified version:
%\begin{align*}
%\VVal {\vQ_5} &= 
%\pi_{\optAtt [\vFour \vee \vFive] [\empno], \optAtt [\vFour] [\name], \optAtt [\vFive] [\fname], \optAtt [\vFive] [\lname]  } \\
%&(\chc [ \fModel_2 ] {\pi_{\empno, \sex, \birthdate, \optAtt [\vFour ] [\name], \optAtt [\vFive] [\fname], \optAtt [\vFive] [\lname]} \empbio, \empRel  })
%\end{align*}
%}
Similarly, the projection in the query 
\ensuremath{\vQ_7 = \pi_{\name, \fname} (\mathit{subq}_7)}
where 
\ensuremath{
\mathit{subq}_7 = \chc [ \vFour] {\pi_\name (\vQ_6), \pi_\fname (\vQ_6)}
}
is written over 
\ensuremath{\vSch_2} and it 
%\centerline{
%\ensuremath{
%\vQ_6 =
%\pi_{\name, \fname} \mathit{subq}_6
%} 
%}}
does not repeat the presence conditions of attributes from its \ensuremath{\mathit{subq}_7}'s type.
The query
%\centerline{
\ensuremath{
\vQ_8 =
\pi_{\optAtt [\vFour ] [\name],\optAtt [\neg \vFour] [\fname]} (\mathit{subq}_7)
%\chc [ \vFour] {\pi_\name \vQ_5, \pi_\fname \vQ_5}
}
%}
makes the annotations of projected attributes \emph{explicit} with respect to both 
the variational schema \ensuremath{\vSch_2} and its subquery's type.
%\TODO {give an example, schema: R(A,B), query: $\pi_{A,B} (F<\pi_A R, \pi_B R>)$
%becomes $\pi_{A^F, B^{\neg F}} ...$}
%The variation encoded in variational queries can
%be more restrictive or more loose than v-schema variation without violating them.
Although relieving the user from explicitly repeating variation makes VRA easier to use, 
queries still have to state variation explicitly to avoid losing information when 
decoupled from the schema.
%We do this by defining a function, 
%\ensuremath {\constrain \vQ}, with type \ensuremath{ \qSet \to \vSchSet \to \qSet
%},
%that \emph{explicitly annotates a query \vQ\ given the underlying schema \vSch}.
We do this by defining the function 
\ensuremath {\constrain \vQ : \qSet \totype \vSchSet \totype \qSet
},
that \emph{explicitly annotates a query \vQ\ with the  schema \vSch}.
%Note that \ensuremath {\constrain \vQ} needs to take the underlying schema as
%an input since it is using the type system (which relies on the schema) as a helper function.
The explicitly annotating query function, 
formally defined in \figref{constrain}, 
conjoins attributes and relations
presence conditions with the corresponding annotations in the query 
and wraps subqueries in a choice when needed. 
Note that, $\vQ_8$ and $\vQ_6$ are the result of $\constrain [\vSch_2] {\vQ_7}$
and $\constrain [\vSch_2] {\vQ_1}$, respectively, after simplification~\footnote{More specifically,
they are simpilified using rules defined in \figref{var-min}}.
%Queries $\vQ_7$ and $\vQ_5$ are examples of applying the 
%explicitly annotation function to queries $\vQ_6$ and $\vQ_1$, respectively,
%after simplifying them.
%\exref{constrain} illustrates how the constrain function transforms queries
%and allows users to be more flexible with their queries. 

\input{formulas/constrainVQbySch}

\begin{theorem}
\label{thm:expl-same-type}
If the query \vQ\ has the type \annot \vType\ then its explicitly annotated counterpart has an equivalent type, that is: \\
%
\centerline{
\ensuremath{%\raggedleft
\envWithoutVctx {\vQ} {\annot \vType} \Rightarrow \envWithoutVctx {\constrain \vQ} {\annot [\VVal \dimMeta] {\VVal \vType}} \textit{ and } \annot \vType \equiv \annot [\VVal \dimMeta] {\VVal \vType}
}}
%
\end{theorem}

\begin{proof}
By structural induction. We encoded and proved this theorem in the Coq proof assistant~\cite{Khan21}.
\end{proof}

This theorem shows that the type system applies the schema to the type of a query although it does not apply it to the query. 
The \emph{type equivalence} is variational set equivalence, defined 
in \figref{vset}, for normalized variational sets of attributes.
%\footnote{We proved this theorem in the Coq proof assistant. The encoding of the theorem and the proof can be found in second author's MS thesis~\cite{FaribaThesis}.}.

%We encode and prove \thmref{expl-same-type} in the Coq proof assistant~\cite{Khan21}.
We illustrate the application of \thmref{expl-same-type} to queries
\ensuremath{\vQ_1} and \ensuremath{\vQ_6}.
%
\exref{type} explained how \ensuremath{\vQ_1}'s type is generated step-by-step.
The variation context and underlying schema are
the same and the subquery \empbio\ has the same type. 
The projected attribute set annotated with the variation context is:\\
\ensuremath{
\vAttList_2 =  \{\annot [(\vFour \vee \vFive) \wedge \neg \vThree] \empno, 
%\ensuremath{ 
\optAtt [\neg \vThree \wedge \vFour \wedge \neg \vFive] [\name], \optAtt [\neg \vThree \wedge \neg \vFour \wedge \vFive] [\fname], \optAtt [\neg \vThree \wedge \neg \vFour \wedge \vFive] [\lname]\}^{\dimMeta_2}}, which is clearly subsumed by \ensuremath{\vAttList_\empbio}, thus, 
%the type of \empbio, \vAttList, and
its intersection with \ensuremath{\vAttList_\empbio} annotated
with the presence condition of \ensuremath{\vAttList_\empbio} is itself,
hence, \ensuremath{\vAttList_{\vQ_1} \equiv \vAttList_{\vQ_6}}.
%which makes it obvious that \ensuremath{\vAttList_{\vQ_1} \equiv \vAttList_{\vQ_6}}.
%\end{example}


%
Explicitly annotating variational queries not only relieves the user from repeating the
database's variation in their queries but it is also necessary for the functions that 
take a query without taking the schema, such as the query configuration function 
which is explained in \secref{vraconf}.
This is contra to other functions that have to take both the query and the 
schema, such as the type system. 
We explain this in more details in \secref{vraconf}.
%\exref{exp-annot-nec} illustrates why a query passed to the configuration function 
%must be explicitly annotated.
%
%\begin{example}
%\label{eg:exp-annot-nec}
%Consider the variational query $\vQ_5= \vPrj [{\vAtt_1, \optAtt [\fOne \wedge \fTwo] [\vAtt_2], \optAtt [\fTwo] [\vAtt_3]}] (\vRel)$ given in
%\exref{conf-vq}. This query is not explicitly annotated since attribute $\vAtt_1$ does not
%carry its variational encoding from the database, that is, it does not have the presence
%condition $\A$. Explicitly annotating this query gives us query $\VVal {\vQ_5} =  \vPrj [{\optAtt [\A][\vAtt_1], \optAtt [\fOne \wedge \fTwo] [\vAtt_2], \optAtt [\fTwo] [\vAtt_3]}] (\vRel)$.
%Configuring $\VVal {\vQ_5}$ results in the same query as configuring $\vQ_5$ except for 
%configuration \setDef {\ }, that is, $\eeSem [\setDef \ ] {\VVal {\vQ_5}} = \pi_{\setDef {\ }} \pRel = \empRel$. The reason why $\eeSem [\setDef \ ] {\vQ_5} $ is incorrect is that $\vQ_5$ is missing
%the variation attached to attribute $\vAtt_1$ and the configuration function does not consider
%the schema of a database while configuring variational queries written over that database. 
%\end{example}


\begin{theorem}
\label{thm:expl-same-type}
If the query \vQ\ has the type \annot \vType\ then its explicitly annotated counterpart has an equivalent type, that is: \\
%
\centerline{
\ensuremath{%\raggedleft
\envWithoutVctx {\vQ} {\annot \vType} \Rightarrow \envWithoutVctx {\constrain \vQ} {\annot [\VVal \dimMeta] {\VVal \vType}} \textit{ and } \annot \vType \equiv \annot [\VVal \dimMeta] {\VVal \vType}
}}
%
\end{theorem}

\begin{proof}
By structural induction. We encoded and proved this theorem in the Coq proof assistant~\cite{Khan21}.
\end{proof}

This theorem shows that the type system applies the schema to the type of a query although it does not apply it to the query. 
The \emph{type equivalence} is variational set equivalence, defined 
in \figref{vset}, for normalized variational sets of attributes.
%\footnote{We proved this theorem in the Coq proof assistant. The encoding of the theorem and the proof can be found in second author's MS thesis~\cite{FaribaThesis}.}.

%We encode and prove \thmref{expl-same-type} in the Coq proof assistant~\cite{Khan21}.
We illustrate the application of \thmref{expl-same-type} to queries
\ensuremath{\vQ_1} and \ensuremath{\vQ_6}.
%
\exref{type} explained how \ensuremath{\vQ_1}'s type is generated step-by-step.
The variation context and underlying schema are
the same and the subquery \empbio\ has the same type. 
The projected attribute set annotated with the variation context is:\\
\ensuremath{
\vAttList_2 =  \{\annot [(\vFour \vee \vFive) \wedge \neg \vThree] \empno, 
%\ensuremath{ 
\optAtt [\neg \vThree \wedge \vFour \wedge \neg \vFive] [\name], \optAtt [\neg \vThree \wedge \neg \vFour \wedge \vFive] [\fname], \optAtt [\neg \vThree \wedge \neg \vFour \wedge \vFive] [\lname]\}^{\dimMeta_2}}, which is clearly subsumed by \ensuremath{\vAttList_\empbio}, thus, 
%the type of \empbio, \vAttList, and
its intersection with \ensuremath{\vAttList_\empbio} annotated
with the presence condition of \ensuremath{\vAttList_\empbio} is itself,
hence, \ensuremath{\vAttList_{\vQ_1} \equiv \vAttList_{\vQ_6}}.
%which makes it obvious that \ensuremath{\vAttList_{\vQ_1} \equiv \vAttList_{\vQ_6}}.
%\end{example}


%
Explicitly annotating variational queries not only relieves the user from repeating the
database's variation in their queries but it is also necessary for the functions that 
take a query without taking the schema, such as the query configuration function 
which is explained in \secref{vraconf}.
This is contra to other functions that have to take both the query and the 
schema, such as the type system. 
We explain this in more details in \secref{vraconf}.
%\exref{exp-annot-nec} illustrates why a query passed to the configuration function 
%must be explicitly annotated.
%
%\begin{example}
%\label{eg:exp-annot-nec}
%Consider the variational query $\vQ_5= \vPrj [{\vAtt_1, \optAtt [\fOne \wedge \fTwo] [\vAtt_2], \optAtt [\fTwo] [\vAtt_3]}] (\vRel)$ given in
%\exref{conf-vq}. This query is not explicitly annotated since attribute $\vAtt_1$ does not
%carry its variational encoding from the database, that is, it does not have the presence
%condition $\A$. Explicitly annotating this query gives us query $\VVal {\vQ_5} =  \vPrj [{\optAtt [\A][\vAtt_1], \optAtt [\fOne \wedge \fTwo] [\vAtt_2], \optAtt [\fTwo] [\vAtt_3]}] (\vRel)$.
%Configuring $\VVal {\vQ_5}$ results in the same query as configuring $\vQ_5$ except for 
%configuration \setDef {\ }, that is, $\eeSem [\setDef \ ] {\VVal {\vQ_5}} = \pi_{\setDef {\ }} \pRel = \empRel$. The reason why $\eeSem [\setDef \ ] {\vQ_5} $ is incorrect is that $\vQ_5$ is missing
%the variation attached to attribute $\vAtt_1$ and the configuration function does not consider
%the schema of a database while configuring variational queries written over that database. 
%\end{example}


\begin{theorem}
\label{thm:expl-same-type}
If the query \vQ\ has the type \annot \vType\ then its explicitly annotated counterpart has an equivalent type, that is: \\
%
\centerline{
\ensuremath{%\raggedleft
\envWithoutVctx {\vQ} {\annot \vType} \Rightarrow \envWithoutVctx {\constrain \vQ} {\annot [\VVal \dimMeta] {\VVal \vType}} \textit{ and } \annot \vType \equiv \annot [\VVal \dimMeta] {\VVal \vType}
}}
%
\end{theorem}

\begin{proof}
By structural induction. We encoded and proved this theorem in the Coq proof assistant~\cite{Khan21}.
\end{proof}

This theorem shows that the type system applies the schema to the type of a query although it does not apply it to the query. 
The \emph{type equivalence} is variational set equivalence, defined 
in \figref{vset}, for normalized variational sets of attributes.
%\footnote{We proved this theorem in the Coq proof assistant. The encoding of the theorem and the proof can be found in second author's MS thesis~\cite{FaribaThesis}.}.

%We encode and prove \thmref{expl-same-type} in the Coq proof assistant~\cite{Khan21}.
We illustrate the application of \thmref{expl-same-type} to queries
\ensuremath{\vQ_1} and \ensuremath{\vQ_6}.
%
\exref{type} explained how \ensuremath{\vQ_1}'s type is generated step-by-step.
The variation context and underlying schema are
the same and the subquery \empbio\ has the same type. 
The projected attribute set annotated with the variation context is:\\
\ensuremath{
\vAttList_2 =  \{\annot [(\vFour \vee \vFive) \wedge \neg \vThree] \empno, 
%\ensuremath{ 
\optAtt [\neg \vThree \wedge \vFour \wedge \neg \vFive] [\name], \optAtt [\neg \vThree \wedge \neg \vFour \wedge \vFive] [\fname], \optAtt [\neg \vThree \wedge \neg \vFour \wedge \vFive] [\lname]\}^{\dimMeta_2}}, which is clearly subsumed by \ensuremath{\vAttList_\empbio}, thus, 
%the type of \empbio, \vAttList, and
its intersection with \ensuremath{\vAttList_\empbio} annotated
with the presence condition of \ensuremath{\vAttList_\empbio} is itself,
hence, \ensuremath{\vAttList_{\vQ_1} \equiv \vAttList_{\vQ_6}}.
%which makes it obvious that \ensuremath{\vAttList_{\vQ_1} \equiv \vAttList_{\vQ_6}}.
%\end{example}


%
Explicitly annotating variational queries not only relieves the user from repeating the
database's variation in their queries but it is also necessary for the functions that 
take a query without taking the schema, such as the query configuration function 
which is explained in \secref{vraconf}.
This is contra to other functions that have to take both the query and the 
schema, such as the type system. 
We explain this in more details in \secref{vraconf}.
%\exref{exp-annot-nec} illustrates why a query passed to the configuration function 
%must be explicitly annotated.
%
%\begin{example}
%\label{eg:exp-annot-nec}
%Consider the variational query $\vQ_5= \vPrj [{\vAtt_1, \optAtt [\fOne \wedge \fTwo] [\vAtt_2], \optAtt [\fTwo] [\vAtt_3]}] (\vRel)$ given in
%\exref{conf-vq}. This query is not explicitly annotated since attribute $\vAtt_1$ does not
%carry its variational encoding from the database, that is, it does not have the presence
%condition $\A$. Explicitly annotating this query gives us query $\VVal {\vQ_5} =  \vPrj [{\optAtt [\A][\vAtt_1], \optAtt [\fOne \wedge \fTwo] [\vAtt_2], \optAtt [\fTwo] [\vAtt_3]}] (\vRel)$.
%Configuring $\VVal {\vQ_5}$ results in the same query as configuring $\vQ_5$ except for 
%configuration \setDef {\ }, that is, $\eeSem [\setDef \ ] {\VVal {\vQ_5}} = \pi_{\setDef {\ }} \pRel = \empRel$. The reason why $\eeSem [\setDef \ ] {\vQ_5} $ is incorrect is that $\vQ_5$ is missing
%the variation attached to attribute $\vAtt_1$ and the configuration function does not consider
%the schema of a database while configuring variational queries written over that database. 
%\end{example}

\subsection{Type Safety}
\label{sec:var-pres}

%\arashComment{ The property of variation preservation is too short and it is not clear why it is important. I also did not find the proofs in Appendix D.}
%\resp{I established why such a property is important in the variational context where we are putting all variants together throughout the paper. here we just define the property and say that it holds at the type level. We also decided to just introduced this as a property for VLDB submission. We'll have the proofs and other properties in another paper.}
%\responded
To show that VRA is type safe we benefit from RA's type safety~\cite{RAtypeSys}
by defining the \emph{variation-preserving} property for VRA which connects VRA to RA.
%Variation-preserving property of VRA's type system and RA's type safety~\cite{RAtypeSys} 
%implies that VRA's type system is also type safe.
%Similar to other applications of variational research~\cite{CEW16ecoop,CEW14toplas},
%the type system must preserve
% the variation encoded in a variational query.
%
The 
\emph{variation-preserving property with respect to variational schema} states that
if a query \vQ\ has type \vType\ then 
configuring the type of a valid explicitly annotated query
is the same as the type of its configured
corresponding query. 
%
\thmref{var-pres} proves this property.

%, 
%i.e., no matter which path the constrained query takes in the diagram it will results
%to the same set of attributes.
%
% the code that produces the diagram
%\hspace{-2cm}
\begin{wrapfigure}{r}{0.24\textwidth}
\begin{center}
\begin{tikzcd}[column sep=2.3em]
  \constrain \vQ   \rar{\mathit{type}}  \dar[swap,dashed]{\eeSem . }
& {\annot \vType}  \dar[dashed]{\olSem . } \\
  \pQ \rar{\underline{type}}
& \pAttList
\end{tikzcd}
\end{center}
\end{wrapfigure}
%
\thmref{var-pres} is visualized in the diagram below, where 
the vertical arrows indicate corresponding configure functions,
\ensuremath{\mathit{type}} indicates VRA's type system, 
that is, \ensuremath{\mathit{type}(\vQ, \vSch) = \annot \vAttList} is 
\ensuremath{\envWithoutVctx \vQ {\envInContext [ \vctx] \vType}},
% of variational query \vQ\
%generated by VRA's type system and 
and
\ensuremath{\underline{\mathit{type}} (\pQ, \underline {\vSch})} indicates RA's type system
for the relational query \pQ\ over the relational database schema $\underline {\vSch}$,
that is, \ensuremath{\pEnv {\pQ} {\pAttList}}.
%Note that for simplicity, w
We assume that corresponding variation schema and schema is
passed to type systems.
% of relational query \pQ.
Simply put, 
the relational type of the configured variational query \vQ\ with configuration \config, 
that is, \ensuremath{\underline {\mathit{type}} (\eeSem {\vQ}, \osSem \vSch)},
must be the same as the configured variational type 
of the variational query \vQ\ with configuration \config, 
that is, \ensuremath{\olSem {\mathit{type} (\vQ, \vSch)}}.
\emph{Clearly the diagram commutes}: taking either path of 1) configuring \constrain \vQ\ first and 
then obtaining the relational type of it or 
2) obtaining the variational type of \constrain \vQ\ first and then configuring it results
in the same set of attributes. 
The variation-preserving property enforces the maintenance of variants that a tuple
belongs to through running a query at the schema level.%
%, partially satisfying second part of 
%\nTwo~
\footnote{
We define this property as a test at the semantics level and show that
%We have not proved this property at the semantics level, however, 
all our experimental
queries passed it.}.
%the test for variation-preserving property at the semantics level.}.
%configuring a variational query \vQ\ for configuration \config\ first and then 
%if we configure variational query \vQ\ for a given configuration \config\ its type (a set of attributes)
%must be the same as if we generate the variational attribute set for
%\vQ\ by VRA's type system and then configure it with \config,
%
%\appref{type-sys-prop-proof} sketches the proof of 
%VRA's type system being variation-preserving.
\exref{var-pres} illustrates why the query must be constrained by the variation schema
in the variation-preserving diagram.

\begin{theorem}
\label{thm:var-pres}
For all configurations \config, if a query \vQ\ has type \annot \vType\ 
then its configured query \ensuremath{\eeSem {\constrain \vQ}}
has type \ensuremath{\olSem {\annot \vType}}, i.e., \\
\centerline{
\ensuremath{
\forall \config \in \confSet. \envWithoutVctx { \vQ} {\annot \vType} \Rightarrow 
\pEnv [\osSem {\vSch}] {\eeSem {\constrain \vQ}} {\olSem {\annot \vType}}
}}.
\end{theorem}

\begin{proof}
By structural induction. We proved this theorem in the Coq proof assistant~\cite{Khan21}.
\end{proof}


\thmref{var-pres} implies that for all valid configurations of a VDB, any variational
query is correlated to a relational query and since RA is type safe its queries are
type safe. Thus, variational queries are type safe. 

\begin{example}
\label{eg:var-pres}
Consider the variational query 
\ensuremath{\vQ_5 = \vPrj [{\vAtt_1, \optAtt [\fOne \wedge \fTwo] [\vAtt_2], \optAtt [\fTwo] [\vAtt_3]}] \vRel} 
given in \exref{conf-vq}. It is well-typed
and  it has the type
\ensuremath{\vAttList =
\setDef {\optAtt [\fOne] [\vAtt_1], 
\optAtt [\fOne \wedge \fTwo] [\vAtt_2], 
\optAtt [\fTwo] [\vAtt_3]}
}.
Configuring \vAttList\ for the variant that both \fOne\ and \fTwo\ are disabled
results is an empty attribute set. However, the type of its configured query
for this variant, i.e., \ensuremath{\eeSem [\setDef \ ] {\vQ_5} =  \pi_{\pAtt_1} \pRel}, is the 
attribute set \ensuremath{\setDef {\pAtt_1}}. This violates the
variation-preserving property. A similar problem happens for the variant of
\setDef {\fTwo}, i.e., \ensuremath{
\underline{\mathit{type}} \left( \eeSem [\setDef \fTwo] {\vQ_5} \right) = 
\underline{\mathit{type}} \left( \pi_{\pAtt_1, \pAtt_3} \pRel \right) = 
\setDef{\pAtt_1, \pAtt_3} \not = \setDef{\pAtt_3} 
= \olSem [\setDef \fTwo] {\vAttList}
= \olSem [\setDef \fTwo] {\mathit{type} \left( \vQ_5 \right)}
}. However, the variation-preserving property holds for the 
constrained query by variation schema, i.e., 
\ensuremath{
\constrain [\vSch_3] {\vQ_5} = 
\vPrj [{\optAtt [\fOne] [\vAtt_1], \optAtt [\fOne \wedge \fTwo] [\vAtt_2], \optAtt [\fTwo] [\vAtt_3]}] \vRel
}.
Thus, the input query to the configuration function \eeSem . \emph{must} be explicitly
annotated by the underlying variation schema for the configured query to match the underlying 
configured schema.
%We can restrict VRA's type system to enforce users to incorporate the
%variation schema into their queries, e.g., \ensuremath{\vQ_5} becomes
%\ensuremath{\VVal \vQ_5 = 
%\vPrj [{\optAtt [\fOne] [\vAtt_1], 
%\optAtt [\fOne \wedge \fTwo] [\vAtt_2], 
%\optAtt [\fTwo] [\vAtt_3]}] \vRel
%}. However, one of the purposes of our type system is to relieve the users 
%from having to encode the VDB's variability into their queries.
%% this burdens the user to know the exact variation encoded in
%%the database in addition to the original variation they want to encode in their query.
%To avoid this violation without requiring users to repeat VDB's variability in their queries,
%after type checking a query we push the variation schema onto the variational query,
%e.g., doing so for \ensuremath{\vQ_5} results in \ensuremath{\VVal \vQ_5}.
\end{example}


\subsubsection{Variation Minimization}
\label{sec:var-min}

\TODO {add more rules}
%
%\TODO{Eric, I kept this here and I just point out this property of VRA in 
%\secref{vrel-alg} in a note box (could you please review that too?). 
%How do you feel about moving this subsection to appendix?}
VRA is flexible since an information need can be represented via multiple
v-queries as demonstrated in \exref{vq-specific} and \exref{vq-same-intent-mult-vars}.
It allows users to incorporate their personal taste and task requirements
into v-queries they write by 
having different levels of variation. For example, consider the explicitly annotated query
\ensuremath{\vQ_5} 
in \secref{constrain}:\\
\ensuremath {
\vQ_5 =
\pi_{\optAtt [\vFour \vee \vFive] [\empno], \optAtt [\vFour] [\name], \optAtt [\vFive] [\fname], \optAtt [\vFive] [\lname]  } \chc [\fModel_2] {\empbio, \empRel}}
%\vQ_5 =  \pi_{\optAtt [\vFour \vee \vFive] [\empno], \optAtt [\vFour] [\name], \optAtt [\vFive] [\fname], \optAtt [\vFive] [\lname]  } \empbio}.
%from \exref{vq-specific}. 
To be explicit about the exact query that will be run for 
each variant 
%and knowing that 
%\ensuremath{
%\getPC \empbio = \vThree \vee \vFour \vee \vFive
%},
the user can \emph{lift up} the variation and rewrite the query as\\
\ensuremath{
\small
\VVal \vQ_5 = \chc [\vFour] {\pi_{\empno, \name} \empbio, 
\chc [\vFive] {\pi_{\empno, \fname, \lname} \empbio, \emp}} 
}.
While \ensuremath{\vQ_5} contains less redundancy \ensuremath{\VVal \vQ_5}
is more comprehensible. 
Thus, \emph{supporting multiple levels of variation 
creates a tension between reducing redundancy and maintaining comprehensibility.}

We define \emph{variation minimization} rules, \figref{var-min}.
% and include 
%interesting ones in \secref{var-min}.
Pushing in variation into a query, i.e., applying rules left-to-right, 
reduces redundancy
% and improves performance
while lifting them up, i.e., applying rules right-to-left, 
makes a query more understandable. 
When applied left-to-right, the rules are terminating since the scope of variation 
%always gets smaller.
monotonically decreases in size.


\begin{figure}
\textbf{Choice Distributive Rules:}
\begin{alignat*}{1}
\small
%-- f<? l? q?, ? l? q?> ? ? (f<l?, l?>) f<q?, q?>
%\inferrule
%{}
%\chc {\pi_{\vAttList_1} \vQ_1, \pi_{\vAttList_2} \vQ_2 } 
%&\equiv
%\pi_{\chc {\vAttList_1, \vAttList_2}} \chc {\vQ_1, \vQ_2}\\
%-- f<? l? q?, ? l? q?> ? ? ((l??), (l? \^�f )) f<q?, q?>
%\inferrule
%{}
\chc {\pi_{\vAttList_1} \vQ_1, \pi_{\vAttList_2} \vQ_2}
&\equiv
\pi_{\annot \vAttList_1, \annot [\neg \dimMeta] \vAttList_2} \chc {\vQ_1, \vQ_2}\\
%-- f<? c? q?, ? c? q?> ? ? f<c?, c?> f<q?, q?>
%\inferrule
%{}
\chc {\sigma_{\vCond_1} \vQ_1, \sigma_{\vCond_2} \vQ_2} 
&\equiv
\sigma_{\chc {\vCond_1, \vCond_2}} \chc {\vQ_1, \vQ_2}\\
%-- f<q? � q?, q? � q?> ? f<q?, q?> � f<q?, q?>
%\inferrule
%{}
\chc {\vQ_1 \times \vQ_2, \vQ_3 \times \vQ_4}
&\equiv
\chc {\vQ_1, \vQ_3} \times \chc {\vQ_2, \vQ_4}\\
%-- f<q? ?\_c? q?, q? ?\_c? q?> ? f<q?, q?> ?\_(f<c?, c?>) f<q?, q?>
%\inferrule
%{}
\chc {\vQ_1 \Join_{\vCond_1} \vQ_2, \vQ_3 \Join_{\vCond_2} \vQ_4}
&\equiv
\chc {\vQ_1, \vQ_3} \Join_{\chc {\vCond_1, \vCond_2}} \chc {\vQ_2, \vQ_4}\\
%-- f<q? ? q?, q? ? q?> ? f<q?, q?> ? f<q?, q?>
%\inferrule
%{}
\chc {\vQ_1 \circ \vQ_2, \vQ_3 \circ \vQ_4}
&\equiv
\chc {\vQ_1, \vQ_3} \circ \chc {\vQ_2, \vQ_4}
%-- f<q? ? q?, q? ? q?> ? f<q?, q?> ? f<q?, q?>
%\inferrule
%{}
%{-}
\end{alignat*}

\medskip
\textbf{CC and RA Optimization Rules Combined:}
\begin{alignat*}{1}
\small
%-- f<? (c? ? c?) q?, ? (c? ? c?) q?> ? ? (c? ? f<c?, c?>) f<q?, q?>
%\inferrule
%{}
\chc {\sigma_{\vCond_1 \wedge \vCond_2} \vQ_1, \sigma_{\vCond_1 \wedge \vCond_3} \vQ_2}
&\equiv
\sigma_{\vCond_1 \wedge \chc {\vCond_2, \vCond_3}} \chc {\vQ_1, \vQ_2}\\
%-- ? c? (f<? c? q?, ? c? q?>) ? ? (c? ? f<c?, c?>) f<q?, q?>
%\inferrule
%{}
\sigma_{\vCond_1} \chc {\sigma_{\vCond_2} \vQ_1, \sigma_{\vCond_3} \vQ_2}
&\equiv
\sigma_{\vCond_1 \wedge \chc {\vCond_2, \vCond_3}} \chc {\vQ_1, \vQ_2}\\
%-- f<q? ?\_(c? ? c?) q?, q? ?\_(c? ? c?) q?> ? ? (f<c?, c?>) (f<q?, q?> ?\_c? f<q?, q?>)
%\inferrule
%{}
\chc {\vQ_1 \Join_{\vCond_1 \wedge \vCond_2} \vQ_2, \vQ_3 \Join_{\vCond_1 \wedge \vCond_3} \vQ_4}
&\equiv
\sigma_{\chc {\vCond_2, \vCond_3}} \left( \chc {\vQ_1, \vQ_3} \Join_{\vCond_1} \chc {\vQ_2, \vQ_4} \right)
\end{alignat*}

\caption{Some of variation minimization rules.}
\label{fig:var-min}
\end{figure}



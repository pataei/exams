\subsection{Identify the kinds of variation existing in relational databases in 
different application domains}
\label{sec:ro1}

To encode variation explicitly in databases we investigate different kinds of 
variation that appears in databases in various application domains. 
Objective 1 aims to represent an encoding for  variation that is 
generic enough that can encode different kinds of variation and is
not bind to a specific instance of variation or application domain. 
%
\tabref{ro1} presents individual research questions we need to answer for
this objective. 

\begin{table}
\caption{Objective 1 research questions.}
\label{tab:ro1}
\centering
\begin{tabularx}{\textwidth}{X}
\toprule
 \textbf{Objective 1: Identify the kinds of variation existing in relational databases in 
different application domains}\tabularnewline
\midrule
RQ1.1: What are the application domains that variation appears in a database? 
What are the dimensions of variation in a database? (\poly, \vamos)
\tabularnewline[0.2cm]
RQ1.2: How can all identified variation instances be represented  in a generic encoding without
regards for the application domain? (\dbpl)
\tabularnewline[0.2cm]
RQ1.3: Can we encode identified instances of variation using our encoding? 
What are the steps one need to take to encode an instance of variation in our
encoding? (\vamos)
\tabularnewline
\bottomrule
\end{tabularx}
\end{table}

\begin{comment}
* motivating example
*dimensions of variation

* variation appears when ...
* schema evolution
* repeated pattern in database versioning, data integration
* this also comes up in software development
* and software also evolves in time
* two dimension that can interact with each other: time and space
\end{comment}

\TODO{for all these research questions here explain what you do intuitively
and then refer to the formal definition and examples in subsection where they 
can skip ahead if they want to.}

For RQ1.1 \TODO{complete with how variation arises in databases}
we explored different application domains where databases change,
yet, there is a need to keep and access the older versions (variants) of the 
database for business reasons. 

\begin{comment}
* introduce a feature space
* propositional formulas of features
\end{comment}

For RQ1.2 we introduce a \emph{feature space} that captures dimensions of 
variation that may appear in a variational database scenario.
We then introduce \emph{feature expressions} as propositional formulas of
features. \secref{encode-var} explains this encoding in details. 

\begin{comment}
* show case the applicability and feasibility of encoding instances of variation using feature expression
\end{comment}

For RQ1.3 ...

\input{sections/encodeVar}


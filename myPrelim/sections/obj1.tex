\subsection{Identify the dimension of variation space and design an encoding to represent parts of a variation space}
\label{sec:ro1}

After studying
different application domains where multiple database instances 
exists simultaneously or over time and they conceptually represent the same or similar
data, but, differ slightly in their schema and/or content for business or experimental reasons,
we needed to understand the variation space that appears in databases.
%Some of these application domains in \secref{intro}. 
Objective 1 aims to identify the dimensions of variation space in databases
and design a generic encoding to represent parts of a variation space independent 
from the application domain.

%To encode variation explicitly in databases we investigate different kinds of 
%variation that appear in databases in various application domains. 
%Objective 1 aims to represent an encoding for  variation that is 
%generic enough that can encode different kinds of variation and is
%not bind to a specific instance of variation or application domain. 
%
\tabref{ro1} presents individual research questions we need to answer for
this objective. 

\begin{table}[H]
\caption{Objective 1 research questions.}
\label{tab:ro1}
\centering
\begin{tabularx}{\textwidth}{X}
\toprule
 \textbf{Objective 1: Identify the dimension of variation space and design an encoding to represent parts of a variation space}\tabularnewline
\midrule
RQ1.1: What are the dimensions of variation space appearing in databases? (\poly, \vamos)
\tabularnewline[0.2cm]
RQ1.2: How can parts of the variation space be described and encoded? (\dbpl)
\tabularnewline[0.2cm]
RQ1.3: Can the encoding chosen in RQ1.2 represent realistic variation spaces in databases? (\vamos)
\tabularnewline
\bottomrule
\end{tabularx}
\end{table}


For RQ1.1 
we investigated the application domains and realized that variation
in databases is similar to variation in software systems with regards
to the dimensions of variation.
%
Software systems can vary in two dimensions: ``space'' and ``time''.
Variation over space refers to the simultaneous development and maintenance of
related systems with different feature sets, which is the focus of work on
SPLs.
%
Variation over time refers to the incremental evolution of a system as it is
developed and maintained, and is the focus of revision control systems and
configuration management (CM)~\cite{Dart91}.
%
Although these two aspects of variation have traditionally been studied and
addressed separately, recent work has sought to unify the treatment of them
%both
%kinds of variation 
to support new kinds
of analyses that consider both dimensions of variation at once
in order to ease maintenance, support the reuse of
techniques developed in their respective communities, and to support new kinds
of analyses that consider both dimensions of variation at once~\cite{Thu19vv}.
%
%The case studies we present illustrate the impact of both dimensions of
%variation on relational databases, and demonstrate how our conception of
%variational databases is generic with respect to the underlying dimension(s) of
%variation.



Similarly, variation in databases exists in time and space. 
Variation in time  appears when a database evolves 
over time while variation in space  appears due
to different information requirements by software or different
sources of information. 
%
However, these variation dimensions in databases have not
been studied. Instead, as we mentioned in \secref{intro},
instances of them have been studied and addressed separately. 
%
For example, schema evolution is an instance of variation in
time while variation appearing in database of the software 
when developing software systems using a SPL is variation in 
space.
%
However, software evolution is unavoidable, so is its artifacts evolution, 
including databases~\cite{dbSPLevolve}.
This is where two instances of managing 
variation in databases (schema evolution and database-backed software 
developed by SPL) interact.
% and variation occurs both in time and space. 
While there are solutions to schema evolution
they cannot adapt to a new situation because they only provide a solution to
variation of databases in time and cannot encode the interaction of database variation in
time and space.
We motivate this case through an example in \secref{mot} and we use this
example throughout this proposal to elaborate more on the introduced 
concepts. 

\begin{comment}
* introduce a feature space
* propositional formulas of features
\end{comment}

For RQ1.2 we take advantage of SPL literature and use a \emph{feature space} 
that captures aspects of 
variation that may appear in a variational database scenario. A feature is
basically some characteristic that is important to the database administrator
and developers while interacting with the database.
We then construct propositional formulas of features, called \emph{feature expressions},
to indicate a specific partition of the variation spaces.
\secref{encode-var} provides the formal definition and examples of
features and feature expressions.


For RQ1.3 we provide two case studies by taking 
two real-world scenarios where variation appears
in databases and encode their variation space using feature expressions.
%
In our first case study, we focus on variation in ``space''.
It shows the use of feature expression to encode the variation
needed for a database-backed SPL. We consider an email
SPL which has been used in several previous SPL research projects (e.g.\
\cite{Apel13:SSP,AlHaj19}).
Since the SPL has a set of features we use the same feature space 
to account for variation in the database of the software.
%It develops a variational
%schema that captures the information needs of a SPL based on Hall's
%decomposition of an email system into its component features~\cite{Hall05}. The
%email SPL has been used in several previous SPL research projects (e.g.\
%\cite{Apel13:SSP,AlHaj19}). The variational email database is populated using
%the Enron email dataset, adapted to fit our variational schema~\cite{Shetty04}.
%
%Our first case study demonstrates the use of a VDB to encode the variational
%information needs of a database-backed SPL.
%
Our case study is formed by systematically combining two pre-existing works:
%
\begin{enumerate}
%
\item 
We use Hall's decomposition of an email system into its component
features~\cite{Hall05} as high-level specification of a SPL.
%
\item 
We use the Enron email
dataset\footnote{\url{http://www.ahschulz.de/enron-email-data/}} as 
%a source of
a realistic email database.
\end{enumerate}


In our second case study, we focus on variation that occurs in ``time''.
It demonstrates the use of a feature expressions to encode the variation created by
 an employee database evolution
scenario. We systematically adapt an existing database evolution scenario from
\citet{prima08Moon} into a VDB and populate it by a dataset that is widely used
in databases research.\footnote{\url{https://github.com/datacharmer/test_db}}
%
For this case, the variation space must include features that indicate
each version of the database, thus,
we assign a feature to each version of the schema.
For example, for the original schema we have the feature \vOne\ and
as the schema is evolving and each time we have a new schema we 
assign it a new feature. That is, for the second schema we have the 
feature \vTwo\ and so on. 
%
Objective 3 (\secref{ro3}) demonstrates how these features are used later on.

\secref{ro2} discusses the objective 2 and its research questions.

\subsubsection{Motivating Example}
\label{sec:mot}



\begin{table}[t]
\caption{Employee schema evolution of a database for a SPL.
%%(evolution features \vOne -- \vFive\ for employee part of the schema (the left schema column), evolution features \tOne -- \tFive\ for education part
%%of the schema (the right schema column), and \edu\ feature representing the education feature of SPL). 
A feature (a boolean variable) represents 
inclusion/exclusion of tables/attributes.  
%The dash-underlined attributes in the basic schema 
%have \emph{variational} present in their relation, i.e.,
%they only exist in their relation when \edu\ = \t.
%%in left schema column only exist when the \edu\ feature of SPL is enabled. 
%%This table represents 1056 possible schema variants:
%%\ensuremath{2^5} schema variants when \edu\ is disabled 
%%(by enabling any combination of \vOne--\vFive) 
%%in addition to \ensuremath{2^5 \times 2^5} when \edu\ is enabled.
}
\label{tab:mot}
\begin{center}
%\small
\footnotesize
%\scriptsize
\begin{tabular} {| c | l || l | c |}
\hline
\textbf{\tiny Temporal} & \multicolumn{2}{ c |}{\textbf{Schemas of Databases for SPL Software Variants}} & \textbf{\tiny Temporal}\\
\cline{2-3}
\textbf{\tiny Features} & \multicolumn{1}{ c ||} {\basic} & \multicolumn{1}{ c |} {\educational} & \textbf{\tiny Features}\\
\hline
%\cline{2-3}
%\cline{2-3}
%\hhline{-==}
\multirow{3}{*}{\vOne} &  \engemp\ (\empno, \name, \hiredate, \titleatt, \deptname) & 
\course\ (\cname, \tno) & \multirow{3}{*}{\tOne}\\
& \othemp\ (\empno, \name, \hiredate, \titleatt, \deptname)  & \student\ (\sno, \cname) &\\
& \job\ (\titleatt, \salary) &  &\\
\hline
\multirow{2}{*}{\vTwo} & \empacct\ (\empno, \name, \hiredate, \titleatt, \deptname) & \course\ (\cno, \cname, \tno) & \multirow{2}{*}{\tTwo}\\
%\cdashline{2-3}
& \job\ (\titleatt, \salary) & \student\ (\sno, \cno) & \\
\hline
\multirow{4}{*}{\vThree} & \empacct\ (\empno, \name, \hiredate, \titleatt, \deptno) & \course\ (\cno, \cname) & \multirow{4}{*}{\tThree}\\
& \job\ (\titleatt, \salary) & \teach\ (\tno, \cno) &\\
& \dept\ (\deptname, \deptno, \managerno) & \student\ (\sno, \cno, \grade) &\\
& \empbio\ (\empno, \sex, \birthdate) & &\\
\hline
\multirow{4}{*}{\vFour} & \empacct\ (\empno, \hiredate, \titleatt, \deptno, \dashuline{\isstudent}, \dashuline{\isteacher}) & \ecourse\ (\cno, \cname) & \multirow{4}{*}{\tFour}\\
& \job\ (\titleatt, \salary) & \course\ (\cno, \cname, \timeatt, \class) & \\
& \dept\ (\deptname, \deptno, \managerno) & \teach\ (\tno, \cno) & \\
& \empbio\ (\empno, \sex, \birthdate, \name) & \student\ (\sno, \cno, \grade) & \\
\hline
\multirow{4}{*}{\vFive} & \empacct\ (\empno, \hiredate, \titleatt, \deptno,  \dashuline{\isstudent}, \dashuline{\isteacher}, \salary) & \ecourse\ (\cno, \cname, \deptno) & \multirow{4}{*}{\tFive}\\
& \dept\ (\deptname, \deptno, \managerno,  \dashuline{\studentnum}, \dashuline{\teachernum}) & \course\ (\cno, \cname, \timeatt, \class, \deptno) & \\
& \empbio\ (\empno, \sex, \birthdate, \fname, \lname) & \teach\ (\tno, \cno) & \\
&& \take\ (\sno, \cno, \grade) & \\
\hline
\end{tabular}
\end{center}
\end{table}

%\arashComment{Section 2 has a useful example but it is unnecessarily too complicated as combines SPL and schema evolution right off the bat.
%1) If you want to use both, you may start with one and then enrich the example with the other. 2) It may be useful to combine and summarize Sections 1 and 2 and use the example to ground the abstract description in Section 1. 3) You might have to make the example shorter to page limits for conferences.} 
%\resp{I numbered your concerns and responded to them one by one: 1) I introduced schema evolution first and then its happening in spl. 2) I still prefer separating motivating ex and introduction to avoid complicating the intro. I want the intro to be to the point and brief, explaining that variational databases have multiple application (instances), they have appeared in literature except that researchers have failed to recognize the commonality in all those instance which is the fact that they are all variation appearing in different contexts and applications. The purpose of motivating example is to draw the reader to continue reading the paper that is more structured. It respects the reader time (if they're interested they'll continue reading into motivating example) and is more organized. 3) I'll come back to this in the final version. However, I think it's important to put the time in and use the space to establish the functionality of VDB.}
%\responded

%\point{Overall scenario.}
In this section, we motivate the interaction of two kinds of variation in databases:
database-backed software produced by SPL and schema evolution.
Consider a SPL that generates management software for companies. 
The SPL has an optional feature: \edu, 
indicating whether a company
provides educational means such as courses for its 
employees\footnote{In practice, such a SPL would have two features: \base\ and \edu,
where \base\ is an arbitrary feature, i.e., for all variants it must be enabled, and it 
indicates software variants that provide just basic functionalities. For simplicity
and without loss of generality, we drop this feature.}. 
%NOTE: YOU CAN OMIT THE FOOTNOTE IF YOU'RE RUNNING OUT OF SPACE!!!!!!!
Software variants that disable \edu\ (\edu\ = \f) only provide basic 
functionalities while ones that enable \edu\ provide educational functionalities
as well as basic ones. Thus, this SPL yields two types of products:
\basic\ and \educational.

%\point{Database of this SPL.}
Software produced by this SPL needs a database to store information
about employees, but SPL features impacts the database:
while \basic\ variants do not need to store any education-related records 
\educational\ variants do. In practice, SPL developers use 
only one database for both variant categories~\cite{vdbSpl18ATW}, 
which can easily be separated
by using the \edu\ feature.
We visualize this idea in \tabref{mot} with two
schema types: \basic\ and \educational.
Additionally, we introduce \emph{temporal features} to tag schemas when they change.
A \basic\ schema is associated with a temporal feature of \vOne\ -- \vFive\ while 
an \educational\ schema is associated with a temporal feature of \tOne\ -- \tFive.
We have two sets of temporal features because when \edu\ is enabled
any \educational\ and \basic\ schemas can be grouped to form a complete
schema.
For example, a valid software variant 
can have the \basic\ schema associated with \vThree\ and
the \educational\ schema associated with \tFour. 
%%\point{schemas could evolve independent from each other.}
%The \educational\ schema evolves independently from the 
%\basic\ schema. For example, due to new requirements SPL DBAs need to 
%add online courses to the schema. Schema variant associated with
%\tFour\ demonstrate this change. That is why we need to have
%two sets of temporal features.
%For example, a valid software variant 
%can have the \basic\ schema associated with \vThree\ in addition
%to the \educational\ schema associated with \tFour. 
%
%the \educational\ category 
%includes tables from the \basic\ schema in addition to ones from the \educational\ schema 
%while the \basic\
%category only consists of its own tables). Separating these database
%variants using SPL features results in much cleaner databases as
%opposed to databases created for SPL in practice.

%\point{Component evolution.}
Now, consider the following scenario:
in the initial design of the \basic\ database, SPL database administrators (DBAs)
settle on three tables \engemp, \othemp, and \job; shown in \tabref{mot} and associated
with temporal feature \vOne. 
After some time, they decide
to refactor the schema to remove redundant tables,
thus, they combine the two
relations \engemp\ and \othemp\ into one, \empacct; associated with temporal feature \vTwo. 
Since some of clients' 
software relies on a previous design the two schemas have to coexist in parallel.
%However, to capture what schema a software variant is using we introduce
%temporal features, e.g., \vOne\ and \vTwo, and tag corresponding schemas 
%with these features. 
Therefore, the existence (presence) of \engemp\ and \othemp\
relations is \emph{variational}, i.e., they only exist in the \basic\
schema when \vOne\ = \t.
This scenario describes \emph{component evolution}:
database evolution in SPL resulted from developers
update, refactor, improve, and components~\cite{dbSPLevolve}.
% is called 
%\emph{component evolution}~\cite{dbSPLevolve}.

%\point{Product evolution.}
Now, consider the case where a client that previously requested a \basic\ variant of the
management software has recently added courses to educate its
employees in subjects they need. Hence, the SPL needs to enable
the \edu\ feature for this client, forcing the adjustment of the schema to \educational. 
This case describes \emph{product evolution}:
database evolution in SPL resulted
from clients adding/removing features/components~\cite{dbSPLevolve}. 
%is called 
%\emph{product evolution}~\cite{dbSPLevolve}. 

%\point{\edu\ affects basic too.}
The two \basic\ and \educational\ schemas are not independent of each other:
%enabling the \edu\ feature not only adds the relations in the \educational\
%schema but it also affects the \basic\ schema. 
consider the \basic\ schema 
variant for temporal feature \vFour. Attributes \isstudent\ and \isteacher\ only exists
in the \empacct\ relation when \edu\ = \t, represented by \dashuline{dash-underlining} them,
otherwise the \empacct\
relation has only four attributes: \empno, \hiredate, \titleatt, and \deptno.
Hence, the presence of attributes \isstudent\ and \isteacher\ in \empacct\ relation is
\emph{variational}, i.e., they only exist in \empacct\ relation
when \edu\ = \t.
%similar to the presence of \engemp\ which
%is variational depending on whether \vOne is enabled or not. 

%%\point{schemas could evolve independent from each other.}
%The \educational\ schema evolves independently from the 
%\basic\ schema. For example, due to new requirements SPL DBAs need to 
%add online courses to the schema. Schema variant associated with
%\tFour\ demonstrate this change. That is why we need to have
%two sets of temporal features.
%For example, a valid software variant 
%can have the \basic\ schema associated with \vThree\ in addition
%to the \educational\ schema associated with \tFour. 

Our motivating example demonstrates how variation in time and space
arises and how they interact, an unavoidable consequence of modern software
development.
To be concrete throughout the proposal we itemize
the needs of users working with a database with variation
through the needs of SPL developers and DBAs:
% for our motivating example.
%These needs cover all general needs when variation appears in databases
%because our motivating example considers variation in both possible 
%dimensions; time and space:
%
\begin{enumerate}
%[wide, labelwidth=!, labelindent=0pt, topsep=1pt]
%[leftmargin=*]
%\itemsep0em
\item [\textbf{(\nZero)}]
SPL developers need to access all database variants while
writing code to be able to extract information for all software variants
they are developing. 
%
Hence,
\emph{users need to have access to all database variants at a given time}.
%
%\secref{vtab} explains how a VDB achieves this.
%%\tabref{mot-vsch} shows how v-schema achieves
%\exref{vsch} shows how v-schema achieves
%this at schema level for a part of our motivating example 
%and \secref{db} discusses two databases
%that achieve this at both content and schema level.
%
\item [\textbf{(\nOne)}]
Depending on what component they are working on,
they need to be able to query all or some of the variants.
%
Hence, 
\emph{users need to query multiple database variants simultaneously and selectively}.
%
%Our query language achieves this by introducing variation into queries, \secref{vq}.
%\exref{vq-specific} and \exref{vq-same-intent-mult-vars} illustrate this 
%for our motivating example.
%
\item [\textbf{(\nTwo)}]
Given that users have access to all database variants and that they
can query all variants simultaneously, for test purposes, 
they desire to know which variant a tuple belongs to.
%
Hence,
\emph{the framework needs to 
keep track of which variants a piece of data belongs to and ensuring that 
it is maintained throughout a query}.
%
%Storing variants that a tuple belongs to in a VDB achieves the first part, \secref{vdb}, and
%VRA's type system ensures the second part, \secref{type-sys}.
%\exref{var-pres} illustrates this for a given query.
%
\item [\textbf{(\nThree)}]
SPL developers need to deploy the database and its queries
to generate a specific software product for a client based on their
requested features. 
%
Hence,
\emph{users need to deploy one variant of the database and its associated queries}.
%
%We define \emph{configure} function for a VDB and its elements, \figref{vdb-conf}, 
%in addition to queries, \figref{v-alg-conf-sem}, that achieves this. 
%\exref{conf-vq} illustrate deploying an example query.
%
\end{enumerate}


%%\point{Why the needs are required and where we address them.}
%%\wrrite{add needs here!}
%We describe the needs of users working with a database with variation
%through the needs of SPL developers and DBAs.
%%
%SPL developers need to have access to all database variants while
%writing code to be able to extract information for all software variants
%they are developing (\nZero). \tabref{mot-vsch} shows how our framework achieves
%this for schema level and \secref{gen-vdb} shows two databases
%that achieve this at both content and schema level.
%%
%Additionally, depending on what component they are working on,
%they need to be able to query all or some of the variants (\nOne).
%We show how our framework achieves this for our motivating example
%in \exref{vq-specific} and \exref{vq-same-intent-mult-vars}.
%%
%Given that they have access to all database variants and that they
%can query all variants simultaneously, for test purposes, 
%they desire to know which variant a tuple belongs to (\nTwo).
%\exref{var-pres} illustrates this for a given query.
%%
%Finally, SPL developers need to deploy the database and its queries
%to generate a specific software product for a client based on their
%requested features (\nThree). \exref{conf-vq} illustrate deploying an example query.

%\point{Current solutions shortage.}
Current solutions to database variation not only cannot adapt to a new
kind of variation but also cannot satisfy all these needs.
Current SPLs generate and use messy 
databases by employing a universal schema.
However, this burdens DBAs heavily because they need to write specific
queries for each variant in order to avoid getting messy data, thus,
current SPLs cannot satisfy \nOne\ and \nTwo.
%
As stated in \secref{intro}, schema evolution systems only consider evolution
in time and do not provide any means for \nOne-\nThree.
%
Also, the current solution to the problem of schema evolution within a SPL is addressed by
designing
a new domain-specific language so that SPL developers 
can write scripts of the schema changes~\cite{dbSPLevolve},
which still requires a great effort by DBAs and SPL developers
and it does not address \nZero-\nTwo\ sufficiently.
%This amount of work grows exponentially 
%as the number of potential variants grow, which in our example depends on the temporal
%changes and SPL features. As a reminder, a SPL usually has hundreds of 
%features~\cite{cppSpl}. As the SPL and its database evolve,
%manually managing the variants becomes impossible. 


%We categorize these needs and explain them throughout the paper:
%\begin{enumerate}[leftmargin=*]
%\item [\textbf{N0}]Have access to all database variants at a given time
%\item [\textbf{N1}]Query multiple database variants simultaneously and selectively
%\item [\textbf{N2}]Keep track of which variants a piece of data belongs to and ensuring that 
%         it is maintained throughout a query
%\item [\textbf{N3}]Deploy one variant of the database and its associated queries.
%\end{enumerate}


\begin{comment}
writing workshop Oct. 7:
motivating example
-early in the paper
-somehow convincing
-realistic
-small
-show off solution. show how you solve it later.

\end{comment}

\subsubsection{Encoding Variability}
\label{sec:encode-var}


%\point{using a feature set to represent variability within a context.}
To account for variability in a database we need a way to 
encode it.
%
%The first challenge of incorporating variability into a database
%is to represent variability. 
%
To encode variability we first organize the configuration space into
a set of features, denoted by \fSet.
%
%we require a \emph{set of features}, denoted by \fSet, 
%appropriate for the context that the database is used for.
%
For example, in the context of schema evolution, features can be generated from version 
numbers (e.g. features \vOne\ to \vFive\ and \tOne\ to \tFive\ in the 
motivating example, \tabref{mot}); for SPLs, 
the features can be adopted from the SPL feature set (e.g. the \edu\ feature in
our motivating example, \tabref{mot}); and 
for data integration, the features can be representatives of resources.  
For simplicity, the set of features is assumed to be closed and features are
assumed to be boolean variables, however, it is easy to extend them
to multi-valued variables that have finite set of values.
% and without loss of generality, 
%features are assumed to be boolean variables, although, it is easy to extend them
%to multi-valued variables. 
A feature \ensuremath{\fName \in \fSet} can be enabled (i.e., \fName = \t) or disabled (\fName = \f).
%\point{configuration.}
%Assuming that all the features by default are set to \prog{false},
%enabling some of them specifies a variant. 

The features in \fSet\ are used to indicate which parts 
of a variational entity within the database are different 
among different variants. Enabling or disabling each of the 
features in \fSet\ produces a particular \emph{variant} 
of the entity in which all variation has been removed. 
%Enabling or disabling all features of \fSet\ specifies a non-variational \emph{variant}
%that can potentially be generated by \emph{configur}ing its variational counterpart 
%with the variant's \emph{configuration}.
%Hence, to specify a variant
%we define a function, called 
A \emph{configuration} is a \emph{total} function
that maps every feature in the feature set to a boolean value.
%By definition, a configuration is a \emph{total} function,
%i.e., it is defined for \emph{all} features in the feature set. 
For brevity, we represent a configuration  by the set of enabled features.
%which represents a variant. 
%\ensure{make sure referring to a variant can be done by the set of enabled features. HERE!}
For example, in our motivating scenario, the configuration \ensuremath{
\setDef {\vTwo,\tThree,\edu}
}
represents a database variant where only features \vTwo, \tThree, and \edu\ are enabled.
This database variant contains relation schemas of the
employee and education sub-schemas associated with \vTwo\ and
\tThree\ in \tabref{mot}, respectively.
For brevity, we refer to a variant with configuration \config\ as variant \config.
For example, variant \setDef {\vTwo,\tThree,\edu} refers to the variant
with configuration \setDef {\vTwo,\tThree,\edu}.
%and education sub-schema associated with \tThree\ in \tabref{mot}.

%\point{represent variability in db by prop formula of features, called feature expression.}
%Having defined a set of features, we need to incorporate them into the database.
%To encode features in the database, we construct propositional formulas of features
%such that the formula evaluates to \t\ for a set of configurations. 
When describing variation points in the database, we need to 
refer to subsets of the configuration space. We achieve this by 
constructing propositional formulas of features.
Thus, 
such a propositional formula defines a condition that holds for 
a subset of configurations and their corresponding variants. 
%
%describing the condition where one or more variants are present,
%i.e., assigning features to their values defined in variant's configuration and 
%evaluating the propositional formula results in \prog{true}.
%
For example,
the propositional formula $\neg \edu$ represents all variants of our
motivating example that do not 
have the education part of the schema, i.e., variant schemas of the 
left schema column. 

We call a propositional formula of features a \emph{feature expression} and define
it formally in \figref{fexp-def}. 
%\figref{fexp-def} defines the syntax of feature expressions.
The evaluation function of feature expressions 
$\fSem \dimMeta : \ffSet \to \confSet \to \bSet$ evaluates the 
feature expression \dimMeta\ w.r.t. the configuration \config.
%and defined in
%our technical report~\cite{vldbArXiv}, 
%and defined in \appref{fexp},
%evaluates feature expression \dimMeta\ under configuration \config,
%also called \emph{configuration of feature expression \dimMeta\ under \config}. 
For example, $\fSem [\{\A\}] {\A \vee \B} = \t$, however,
%which states that the feature expression
%$\A \vee \B$ evaluates to \t\ w.r.t. configuration that only enables $\A$, however,
$\fSem [\{\}] {\A \vee \B} = \f$, where the empty set indicates neither \A\ nor \B\
are enabled.
%which states that the same feature expression evaluates
%to \f\ when neither $\A$ nor $\B$ are enabled.
Additionally, we define the binary \emph{equivalence ($\equiv$)} relation and
the unary \emph{satisfiable (sat)} and \emph{unsatisfiable (unsat)}
relations over feature expressions in \figref{fexp-def}.
%
% some of the functions needed over feature
%expressions:
%1) \emph{equivalence} of two feature expressions, 
%\ensuremath{\dimMeta_1 \equiv \dimMeta_2}
%and
%2) \emph{satisfiability} of a feature expression, 
%\ensuremath{\sat \dimMeta}.
%However, we define the evaluation of feature expressions and functions over them
%in \appref{fexp}.
%We define two functions over feature expressions, as shown in \figref{fexp-def}:
%1) \emph{satisfiability}: feature expression \dimMeta\ is \emph{satisfiable} if there 
%exists configuration \config\ s.t. \fSem \dimMeta = \t\
%and 2) \emph{tautology}: feature expression \dimMeta\ is a \emph{tautology} if  
%for all valid configurations we have: \fSem \dimMeta = \t.

%\point{annotating elements of database with feature expressions.}
To incorporate feature expressions into the database,
we \emph{annotate/tag} database elements (including attributes, relations, and tuples) 
with feature expressions. An \emph{annotated element} \elem\ with feature expression \dimMeta\
is denoted by \annot \elem. 
%The feature expression \dimMeta\ represents
%the set of configurations where their variants contain element \elem\ because
%
The feature expression attached to an element is called a \emph{presence condition}
since it determines the condition (set of configurations) under which the element is present.
\getPC \elem\ returns the presence condition of the element
\elem.
For example, the
annotated number $2^{\A \vee \B}$ is present in variants with a configuration
that enables either $\A$ or $\B$ or both
but it does not exist in variants that disable both $\A$ and $\B$.
Here, $\getPC {2} = \A \vee \B$.

%\point{relationship between features is captured by a propositional formula, called feature model.} 
No matter the context, features often have a relationship with each other that
constrains configurations. For example, only one of the temporal features of $\vOne - \vFive$
can be \t\ for a given variant.
This relationship can be captured by a feature expression, called a \emph{feature model} and
denoted by \fModel,
which restricts the set of \emph{valid configurations}:
if configuration \config\ violates the relationship then 
%evaluating the feature model \fModel\
%under this configuration evaluates to \f: 
\fSem \fModel = \f.
For example, the restriction that at a given time only one of temporal features $\vOne - \vFive$
can be enabled is represented by:
%the feature expression:
\ensuremath{
\vOne \oplus \vTwo \oplus \vThree \oplus \vFour \oplus \vFive
},
where $\A \oplus \B \oplus \ldots \oplus \fName_n$ is syntactic sugar for $(\A \wedge \neg \B \wedge \ldots \wedge \neg \fName_n) \vee (\neg \A \wedge \B \wedge \ldots \wedge \neg \fName_n) \vee (\neg \A \wedge \neg \B \wedge \ldots \wedge \fName_n)$
, i.e., features are mutually exclusive.
%multiple features cannot be enabled simultaneously.
%$\left(\vOne \wedge \neg \left(\vTwo \vee \hdots \vee \vFive \right) \right)
%\vee \left(\vTwo \wedge \neg \left(\vOne \vee \vThree \vee \vFour \vee \vFive \right) \right) 
%\vee \hdots 
%\vee \left(\vFive \wedge \neg \left(\vOne \vee \hdots \vee \vFour \right) \right)$.
%Note that this is not the feature model for the entire motivating example.



\begin{figure}
%\textbf{Feature expression generic object:}
%\begin{syntax}
%\synDef \fName \fSet &\textit{Feature Name}
%%c \in \mathbf{C} &\textit{Configuration}
%\end{syntax}
%
%\medskip
\textbf{Feature expression syntax:}
\begin{syntax}
\synDef \fName \fSet &&&\textit{Feature Name}\\
\synDef \bTag \bSet &\eqq& \t \myOR \f & \textit{Boolean Value}\\
\synDef \dimMeta \ffSet &\eqq& \bTag \myOR \fName \myOR \neg \dimMeta \myOR \dimMeta \wedge \dimMeta \myOR \dimMeta \vee \dimMeta & \textit{Feature Expression}\\
\synDef \config \confSet &:& \fSet \totype \bSet &\textit{Configuration}
\end{syntax}

\medskip
\textbf{Evaluation of feature expressions:}
\begin{alignat*}{1}
\fSem [] . &: \ffSet \totype \confSet \totype \bSet\\
\fSem \bTag &= \bTag\\
\fSem \fName &= \config \ \fName\\
\fSem {\neg \fName} &= \neg \fSem \fName\\
\fSem {\annd \dimMeta} &= \fSem {\dimMeta_1} \wedge \fSem {\dimMeta_2}\\
\fSem {\orr \dimMeta} &= \fSem {\dimMeta_1} \vee \fSem {\dimMeta_2}\\
\end{alignat*}

\medskip
\textbf{Relations over feature expressions:}
\begin{alignat*}{1}
\dimMeta_1 \equiv \dimMeta_2 &\textit{ iff \ } \forall \config \in \confSet. \fSem {\dimMeta_1} = \fSem {\dimMeta_2}\\
\sat {\dimMeta} &\textit{ iff \ } \exists \config \in \confSet. \fSem {\dimMeta} = \t\\
\unsat {\dimMeta} &\textit{ iff \ } \forall \config \in \confSet. \fSem {\dimMeta} = \f\\
\oneof {\A, \B, \ldots, \fName_n}
&= (\A \wedge \neg \B \wedge \ldots \wedge \neg \fName_n)
\vee (\neg \A \wedge \B \wedge \ldots \wedge \neg \fName_n)\\
&\qquad\vee (\neg \A \wedge \neg \B \wedge \ldots \wedge \fName_n)
%\dimMeta_1 \oplus \dimMeta_2 = (\dimMeta_1 \wedge \neg \dimMeta_2) \vee (\dimMeta_2 \wedge \neg \dimMeta_1)
\end{alignat*}

%\medskip
%\textbf{Syntactic sugar for mutually exclusive features:}
%\begin{alignat*}{1}
%\A \oplus \B \oplus \ldots \oplus \fName_n
%= (\A \wedge \neg \B \wedge \ldots \wedge \neg \fName_n)\\
%\vee (\neg \A \wedge \B \wedge \ldots \wedge \neg \fName_n)\\
%\vee (\neg \A \wedge \neg \B \wedge \ldots \wedge \fName_n)
%\end{alignat*}


\begin{comment}
\medskip
\textbf{Configuration constraint:}
\begin{equation*}
\forall \overrightarrow{f_i} \in c :
%( \neg o_1 \wedge o_2 \wedge o_3 \wedge \ldots \wedge o_{|f_i|})
\bigvee_{1\leq j \leq |f_i|}(\neg o_j \wedge \bigwedge_{\substack{k\not = j\\1 \leq k\leq |f_i|}} o_k)
= \prog{true}
\end{equation*}
\end{comment}

\caption[Feature expression syntax and evaluation]{Feature expression syntax, relations, and evaluation.
% and functions over feature expressions. 
%We informally define \emph{exclusive or} \ensuremath{\oplus} of \ensuremath{n} features to be \t\ 
%only when one feature is \t.
}
\label{fig:fexp-def}
\end{figure}






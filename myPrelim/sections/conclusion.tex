\section{Conclusion}
\label{sec:con}

Informed by different instances of variation appearing in databases, we hypothesize that considering 
variation as an orthogonal concern to database provides benefits researchers, database administrators, and developers.
To investigate this hypothesis, we plan the following:

\begin{itemize}
\item Activity 1: We have studied various instances of variation in databases and based on them
we have provided a framework that considers variation as an orthogonal concern of databases~\cite{ATW17dbpl,ATW18poly}.
\item Activity 2: We have used the framework to represent real-world instances of variation in database,
showing the applicability of our frames~\cite{ALW21vamos}.
\item Activity 3: We are implementing and refactoring VDBMS, a database management system for our
framework.
\item Activity 4: We will mechanically prove the properties of our framework.
\end{itemize}

\noindent
The result of these activities will provide the following contributions:

\begin{itemize}
\item The first work to establish theoretical framework for variational databases where the database and the query language both 
allow explicit encoding of variation. (Activity 1 and 4)
\item The first database management system that allows developers and database administrators to interact with variational databases. (Activity 1, 2, and 3)
\item Real-world case studies of variation in databases that can be used in future research on variational data. (Activity 2)
\end{itemize}

The database community has researched instances of variation in databases extensively without 
acknowledging that they are instances of the same problem, variation in a database. On the other
hand, the SPL community has realized the need for encoding variation at the data model, however,
they do not go beyond the data model. Our work on RQ1.1 and RQ1.3~\cite{ALW21vamos} has
shown that different instances of variation in databases could interact with each other, thus, 
having a generic encoding of variation in databases is beneficial. 
\subsubsection{Encoding Variability}
\label{sec:encode-var}

\TODO{fix examples to ones that you have introduced here}

%\point{using a feature set to represent variability within a context.}
To account for variability in a database we need a way to 
encode it.
%
%The first challenge of incorporating variability into a database
%is to represent variability. 
%
To encode variability we first organize the configuration space into
a set of features, denoted by \fSet.
%
%we require a \emph{set of features}, denoted by \fSet, 
%appropriate for the context that the database is used for.
%
For example, in the context of schema evolution, features can be generated from version 
numbers (e.g. features \vOne\ to \vFive\ and \tOne\ to \tFive\ in the 
motivating example, \tabref{mot}); for SPLs, 
the features can be adopted from the SPL feature set (e.g. the \edu\ feature in
our motivating example, \tabref{mot}); and 
for data integration, the features can be representatives of resources.  
For simplicity, the set of features is assumed to be closed and features are
assumed to be boolean variables, however, it is easy to extend them
to multi-valued variables that have finite set of values.
% and without loss of generality, 
%features are assumed to be boolean variables, although, it is easy to extend them
%to multi-valued variables. 
A feature \ensuremath{\fName \in \fSet} can be enabled (i.e., \fName = \t) or disabled (\fName = \f).
%\point{configuration.}
%Assuming that all the features by default are set to \prog{false},
%enabling some of them specifies a variant. 

The features in \fSet\ are used to indicate which parts 
of a variational entity within the database are different 
among different variants. Enabling or disabling each of the 
features in \fSet\ produces a particular \emph{variant} 
of the entity in which all variation has been removed. 
%Enabling or disabling all features of \fSet\ specifies a non-variational \emph{variant}
%that can potentially be generated by \emph{configur}ing its variational counterpart 
%with the variant's \emph{configuration}.
%Hence, to specify a variant
%we define a function, called 
A \emph{configuration} is a \emph{total} function
that maps every feature in the feature set to a boolean value.
%By definition, a configuration is a \emph{total} function,
%i.e., it is defined for \emph{all} features in the feature set. 
For brevity, we represent a configuration  by the set of enabled features.
%which represents a variant. 
%\ensure{make sure referring to a variant can be done by the set of enabled features. HERE!}
For example, in our motivating scenario, the configuration \ensuremath{
\setDef {\vTwo,\tThree,\edu}
}
represents a database variant where only features \vTwo, \tThree, and \edu\ are enabled.
This database variant contains relation schemas of the
employee and education sub-schemas associated with \vTwo\ and
\tThree\ in \tabref{mot}, respectively.
For brevity, we refer to a variant with configuration \config\ as variant \config.
For example, variant \setDef {\vTwo,\tThree,\edu} refers to the variant
with configuration \setDef {\vTwo,\tThree,\edu}.
%and education sub-schema associated with \tThree\ in \tabref{mot}.

%\point{represent variability in db by prop formula of features, called feature expression.}
%Having defined a set of features, we need to incorporate them into the database.
%To encode features in the database, we construct propositional formulas of features
%such that the formula evaluates to \t\ for a set of configurations. 
When describing variation points in the database, we need to 
refer to subsets of the configuration space. We achieve this by 
constructing propositional formulas of features.
Thus, 
such a propositional formula defines a condition that holds for 
a subset of configurations and their corresponding variants. 
%
%describing the condition where one or more variants are present,
%i.e., assigning features to their values defined in variant's configuration and 
%evaluating the propositional formula results in \prog{true}.
%
For example,
the propositional formula $\neg \edu$ represents all variants of our
motivating example that do not 
have the education part of the schema, i.e., variant schemas of the 
left schema column. 

We call a propositional formula of features a \emph{feature expression} and define
it formally in \figref{fexp-def}. 
%\figref{fexp-def} defines the syntax of feature expressions.
The evaluation function of feature expressions 
$\fSem \dimMeta : \ffSet \to \confSet \to \bSet$ evaluates the 
feature expression \dimMeta\ w.r.t. the configuration \config.
%and defined in
%our technical report~\cite{vldbArXiv}, 
%and defined in \appref{fexp},
%evaluates feature expression \dimMeta\ under configuration \config,
%also called \emph{configuration of feature expression \dimMeta\ under \config}. 
For example, $\fSem [\{\A\}] {\A \vee \B} = \t$, however,
%which states that the feature expression
%$\A \vee \B$ evaluates to \t\ w.r.t. configuration that only enables $\A$, however,
$\fSem [\{\}] {\A \vee \B} = \f$, where the empty set indicates neither \A\ nor \B\
are enabled.
%which states that the same feature expression evaluates
%to \f\ when neither $\A$ nor $\B$ are enabled.
Additionally, we define the binary \emph{equivalence ($\equiv$)} relation and
the unary \emph{satisfiable (sat)} and \emph{unsatisfiable (unsat)}
relations over feature expressions in \figref{fexp-def}.
%
% some of the functions needed over feature
%expressions:
%1) \emph{equivalence} of two feature expressions, 
%\ensuremath{\dimMeta_1 \equiv \dimMeta_2}
%and
%2) \emph{satisfiability} of a feature expression, 
%\ensuremath{\sat \dimMeta}.
%However, we define the evaluation of feature expressions and functions over them
%in \appref{fexp}.
%We define two functions over feature expressions, as shown in \figref{fexp-def}:
%1) \emph{satisfiability}: feature expression \dimMeta\ is \emph{satisfiable} if there 
%exists configuration \config\ s.t. \fSem \dimMeta = \t\
%and 2) \emph{tautology}: feature expression \dimMeta\ is a \emph{tautology} if  
%for all valid configurations we have: \fSem \dimMeta = \t.

%\point{annotating elements of database with feature expressions.}
To incorporate feature expressions into the database,
we \emph{annotate/tag} database elements (including attributes, relations, and tuples) 
with feature expressions. An \emph{annotated element} \elem\ with feature expression \dimMeta\
is denoted by \annot \elem. 
%The feature expression \dimMeta\ represents
%the set of configurations where their variants contain element \elem\ because
%
The feature expression attached to an element is called a \emph{presence condition}
since it determines the condition (set of configurations) under which the element is present.
\getPC \elem\ returns the presence condition of the element
\elem.
For example, the
annotated number $2^{\A \vee \B}$ is present in variants with a configuration
that enables either $\A$ or $\B$ or both
but it does not exist in variants that disable both $\A$ and $\B$.
Here, $\getPC {2} = \A \vee \B$.

%\point{relationship between features is captured by a propositional formula, called feature model.} 
No matter the context, features often have a relationship with each other that
constrains configurations. For example, only one of the temporal features of $\vOne - \vFive$
can be \t\ for a given variant.
This relationship can be captured by a feature expression, called a \emph{feature model} and
denoted by \fModel,
which restricts the set of \emph{valid configurations}:
if configuration \config\ violates the relationship then 
%evaluating the feature model \fModel\
%under this configuration evaluates to \f: 
\fSem \fModel = \f.
For example, the restriction that at a given time only one of temporal features $\vOne - \vFive$
can be enabled is represented by:
%the feature expression:
\ensuremath{
\vOne \oplus \vTwo \oplus \vThree \oplus \vFour \oplus \vFive
},
where $\A \oplus \B \oplus \ldots \oplus \fName_n$ is syntactic sugar for $(\A \wedge \neg \B \wedge \ldots \wedge \neg \fName_n) \vee (\neg \A \wedge \B \wedge \ldots \wedge \neg \fName_n) \vee (\neg \A \wedge \neg \B \wedge \ldots \wedge \fName_n)$
, i.e., features are mutually exclusive.
%multiple features cannot be enabled simultaneously.
%$\left(\vOne \wedge \neg \left(\vTwo \vee \hdots \vee \vFive \right) \right)
%\vee \left(\vTwo \wedge \neg \left(\vOne \vee \vThree \vee \vFour \vee \vFive \right) \right) 
%\vee \hdots 
%\vee \left(\vFive \wedge \neg \left(\vOne \vee \hdots \vee \vFour \right) \right)$.
%Note that this is not the feature model for the entire motivating example.



\begin{figure}
%\textbf{Feature expression generic object:}
%\begin{syntax}
%\synDef \fName \fSet &\textit{Feature Name}
%%c \in \mathbf{C} &\textit{Configuration}
%\end{syntax}
%
%\medskip
\textbf{Feature expression syntax:}
\begin{syntax}
\synDef \fName \fSet &&&\textit{Feature Name}\\
\synDef \bTag \bSet &\eqq& \t \myOR \f & \textit{Boolean Value}\\
\synDef \dimMeta \ffSet &\eqq& \bTag \myOR \fName \myOR \neg \dimMeta \myOR \dimMeta \wedge \dimMeta \myOR \dimMeta \vee \dimMeta & \textit{Feature Expression}\\
\synDef \config \confSet &:& \fSet \totype \bSet &\textit{Configuration}
\end{syntax}

\medskip
\textbf{Evaluation of feature expressions:}
\begin{alignat*}{1}
\fSem [] . &: \ffSet \totype \confSet \totype \bSet\\
\fSem \bTag &= \bTag\\
\fSem \fName &= \config \ \fName\\
\fSem {\neg \fName} &= \neg \fSem \fName\\
\fSem {\annd \dimMeta} &= \fSem {\dimMeta_1} \wedge \fSem {\dimMeta_2}\\
\fSem {\orr \dimMeta} &= \fSem {\dimMeta_1} \vee \fSem {\dimMeta_2}\\
\end{alignat*}

\medskip
\textbf{Relations over feature expressions:}
\begin{alignat*}{1}
\dimMeta_1 \equiv \dimMeta_2 &\textit{ iff \ } \forall \config \in \confSet. \fSem {\dimMeta_1} = \fSem {\dimMeta_2}\\
\sat {\dimMeta} &\textit{ iff \ } \exists \config \in \confSet. \fSem {\dimMeta} = \t\\
\unsat {\dimMeta} &\textit{ iff \ } \forall \config \in \confSet. \fSem {\dimMeta} = \f\\
\oneof {\A, \B, \ldots, \fName_n}
&= (\A \wedge \neg \B \wedge \ldots \wedge \neg \fName_n)
\vee (\neg \A \wedge \B \wedge \ldots \wedge \neg \fName_n)\\
&\qquad\vee (\neg \A \wedge \neg \B \wedge \ldots \wedge \fName_n)
%\dimMeta_1 \oplus \dimMeta_2 = (\dimMeta_1 \wedge \neg \dimMeta_2) \vee (\dimMeta_2 \wedge \neg \dimMeta_1)
\end{alignat*}

%\medskip
%\textbf{Syntactic sugar for mutually exclusive features:}
%\begin{alignat*}{1}
%\A \oplus \B \oplus \ldots \oplus \fName_n
%= (\A \wedge \neg \B \wedge \ldots \wedge \neg \fName_n)\\
%\vee (\neg \A \wedge \B \wedge \ldots \wedge \neg \fName_n)\\
%\vee (\neg \A \wedge \neg \B \wedge \ldots \wedge \fName_n)
%\end{alignat*}


\begin{comment}
\medskip
\textbf{Configuration constraint:}
\begin{equation*}
\forall \overrightarrow{f_i} \in c :
%( \neg o_1 \wedge o_2 \wedge o_3 \wedge \ldots \wedge o_{|f_i|})
\bigvee_{1\leq j \leq |f_i|}(\neg o_j \wedge \bigwedge_{\substack{k\not = j\\1 \leq k\leq |f_i|}} o_k)
= \prog{true}
\end{equation*}
\end{comment}

\caption[Feature expression syntax and evaluation]{Feature expression syntax, relations, and evaluation.
% and functions over feature expressions. 
%We informally define \emph{exclusive or} \ensuremath{\oplus} of \ensuremath{n} features to be \t\ 
%only when one feature is \t.
}
\label{fig:fexp-def}
\end{figure}




\subsubsection{Variation Minimization}
\label{sec:var-min}

%
%\TODO{Eric, I kept this here and I just point out this property of VRA in 
%\secref{vrel-alg} in a note box (could you please review that too?). 
%How do you feel about moving this subsection to appendix?}
VRA is flexible since an information need can be represented via multiple
v-queries as demonstrated in \exref{vq-specific} and \exref{vq-same-intent-mult-vars}.
It allows users to incorporate their personal taste and task requirements
into v-queries they write by 
having different levels of variation. For example, consider the explicitly annotated query
\ensuremath{\vQ_5} 
in \secref{constrain}:\\
\ensuremath {
\vQ_5 =
\pi_{\optAtt [\vFour \vee \vFive] [\empno], \optAtt [\vFour] [\name], \optAtt [\vFive] [\fname], \optAtt [\vFive] [\lname]  } \chc [\fModel_2] {\empbio, \empRel}}
%\vQ_5 =  \pi_{\optAtt [\vFour \vee \vFive] [\empno], \optAtt [\vFour] [\name], \optAtt [\vFive] [\fname], \optAtt [\vFive] [\lname]  } \empbio}.
%from \exref{vq-specific}. 
To be explicit about the exact query that will be run for 
each variant 
%and knowing that 
%\ensuremath{
%\getPC \empbio = \vThree \vee \vFour \vee \vFive
%},
the user can \emph{lift up} the variation and rewrite the query as\\
\ensuremath{
\small
\VVal \vQ_5 = \chc [\vFour] {\pi_{\empno, \name} \empbio, 
\chc [\vFive] {\pi_{\empno, \fname, \lname} \empbio, \emp}} 
}.
While \ensuremath{\vQ_5} contains less redundancy \ensuremath{\VVal \vQ_5}
is more comprehensible. 
Thus, \emph{supporting multiple levels of variation 
creates a tension between reducing redundancy and maintaining comprehensibility.}

We define \emph{variation minimization} rules, \figref{var-min}.
% and include 
%interesting ones in \secref{var-min}.
Pushing in variation into a query, i.e., applying rules left-to-right, 
reduces redundancy
% and improves performance
while lifting them up, i.e., applying rules right-to-left, 
makes a query more understandable. 
When applied left-to-right, the rules are terminating since the scope of variation 
%always gets smaller.
monotonically decreases in size.


\begin{figure}
\textbf{Choice Distributive Rules:}
\begin{alignat*}{1}
\small
%-- f<? l? q?, ? l? q?> ? ? (f<l?, l?>) f<q?, q?>
%\inferrule
%{}
%\chc {\pi_{\vAttList_1} \vQ_1, \pi_{\vAttList_2} \vQ_2 } 
%&\equiv
%\pi_{\chc {\vAttList_1, \vAttList_2}} \chc {\vQ_1, \vQ_2}\\
%-- f<? l? q?, ? l? q?> ? ? ((l??), (l? \^�f )) f<q?, q?>
%\inferrule
%{}
\chc {\pi_{\vAttList_1} \vQ_1, \pi_{\vAttList_2} \vQ_2}
&\equiv
\pi_{\annot \vAttList_1, \annot [\neg \dimMeta] \vAttList_2} \chc {\vQ_1, \vQ_2}\\
%-- f<? c? q?, ? c? q?> ? ? f<c?, c?> f<q?, q?>
%\inferrule
%{}
\chc {\sigma_{\vCond_1} \vQ_1, \sigma_{\vCond_2} \vQ_2} 
&\equiv
\sigma_{\chc {\vCond_1, \vCond_2}} \chc {\vQ_1, \vQ_2}\\
%-- f<q? � q?, q? � q?> ? f<q?, q?> � f<q?, q?>
%\inferrule
%{}
\chc {\vQ_1 \times \vQ_2, \vQ_3 \times \vQ_4}
&\equiv
\chc {\vQ_1, \vQ_3} \times \chc {\vQ_2, \vQ_4}\\
%-- f<q? ?\_c? q?, q? ?\_c? q?> ? f<q?, q?> ?\_(f<c?, c?>) f<q?, q?>
%\inferrule
%{}
\chc {\vQ_1 \Join_{\vCond_1} \vQ_2, \vQ_3 \Join_{\vCond_2} \vQ_4}
&\equiv
\chc {\vQ_1, \vQ_3} \Join_{\chc {\vCond_1, \vCond_2}} \chc {\vQ_2, \vQ_4}\\
%-- f<q? ? q?, q? ? q?> ? f<q?, q?> ? f<q?, q?>
%\inferrule
%{}
\chc {\vQ_1 \circ \vQ_2, \vQ_3 \circ \vQ_4}
&\equiv
\chc {\vQ_1, \vQ_3} \circ \chc {\vQ_2, \vQ_4}
%-- f<q? ? q?, q? ? q?> ? f<q?, q?> ? f<q?, q?>
%\inferrule
%{}
%{-}
\end{alignat*}

\medskip
\textbf{CC and RA Optimization Rules Combined:}
\begin{alignat*}{1}
\small
%-- f<? (c? ? c?) q?, ? (c? ? c?) q?> ? ? (c? ? f<c?, c?>) f<q?, q?>
%\inferrule
%{}
\chc {\sigma_{\vCond_1 \wedge \vCond_2} \vQ_1, \sigma_{\vCond_1 \wedge \vCond_3} \vQ_2}
&\equiv
\sigma_{\vCond_1 \wedge \chc {\vCond_2, \vCond_3}} \chc {\vQ_1, \vQ_2}\\
%-- ? c? (f<? c? q?, ? c? q?>) ? ? (c? ? f<c?, c?>) f<q?, q?>
%\inferrule
%{}
\sigma_{\vCond_1} \chc {\sigma_{\vCond_2} \vQ_1, \sigma_{\vCond_3} \vQ_2}
&\equiv
\sigma_{\vCond_1 \wedge \chc {\vCond_2, \vCond_3}} \chc {\vQ_1, \vQ_2}\\
%-- f<q? ?\_(c? ? c?) q?, q? ?\_(c? ? c?) q?> ? ? (f<c?, c?>) (f<q?, q?> ?\_c? f<q?, q?>)
%\inferrule
%{}
\chc {\vQ_1 \Join_{\vCond_1 \wedge \vCond_2} \vQ_2, \vQ_3 \Join_{\vCond_1 \wedge \vCond_3} \vQ_4}
&\equiv
\sigma_{\chc {\vCond_2, \vCond_3}} \left( \chc {\vQ_1, \vQ_3} \Join_{\vCond_1} \chc {\vQ_2, \vQ_4} \right)
\end{alignat*}

\caption{Some of variation minimization rules.}
\label{fig:var-min}
\end{figure}

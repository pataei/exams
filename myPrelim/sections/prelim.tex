\section{Preliminaries}
\label{sec:prelim}

In this section, we introduce concepts and notations that we use
throughout the paper. \tabref{notations} 
provides a short overview 
and is meant as an aid to find definitions
faster.
% (\S\ indicates the section(s) containing the definition).
Throughout the proposal, we discuss relational concepts and their
variational counterparts. For clarity, when we need to emphasize 
an entity is not variational we underline it, e.g., \pElem\ is a 
non-variational entity while \elem\ is its variational counterpart,
if it exists.

\begin{table}
\caption{Introduced notations and terminologies with their corresponding section(s).}
\label{tab:notations}
\begin{center}
\small
\begin{tabular}{ |l c r| } 
 \hline
Name & Notation & Section \\
\hline
Feature & \fName & \multirow{7}{*} {\secref{encode-var}} \\
Feature expression & \dimMeta & \\
Annotated element \elem\ by \dimMeta & \annot \elem & \\
Configuration & \config & \\
%True feature set of \config & \fct & \\
Evaluation of \dimMeta\ under \config & \fSem \dimMeta & \\
Presence condition of entity \elem & \getPC \elem & \\
\hline
Optional attribute & \vAtt & \multirow{4}{*}{\secref{vsch}}\\
Variational attribute set & \vAttList & \\
Variational relation schema & \vRelSch & \\
Variational schema & \vSch & \\
\hline
Variational tuple & \tuple & \multirow{3}{*}{\secref{vtab}} \\
Variational relation content & \vRelCont & \\
Variational table & \vTab\ = (\vRelSch, \vRelCont) & \\
\hline
Choice & $\chc {x, y}$ & \multirow{3}{*}{\secref{vrel-alg}}\\
Variational condition & \vCond & \\
Variational query & \vQ & \\
\hline
%Configure query \vQ\ with configuration \config & \eeSem \vQ & \multirow{2}{*}{\secref{vra-sem}}\\
%Group query \vQ & \qGroup \vQ &\\
%\hline
\end{tabular}
\end{center}
\end{table}

\subsection{Relational Databases and Relational Algebra}
\label{sec:rel-db}

%\point{database schema.}
A relational database \pDB\ stores information in a structured manner by forcing
data to conform to a \emph{schema} \pSch\ that is a finite set 
$\setDef {{\pRelSch}_1, \ldots, {\pRelSch}_n}$ of \emph{relation schemas}.
A relation schema is defined as
$\pRelSch = \vRel \paran {\vi \pAtt k}$ where each $\pAtt_i$ is an
\emph{attribute} contained in a relation named \vRel. 
$\getRel {\pAtt}$ returns the relation that contains the attribute.
\type [\pAtt]\ returns the \emph{type} of values associated with attribute \pAtt.

%\point{database content.}
The content of database \pDB\ is stored in the form of \emph{tuples}. A tuple \pTuple\
is a mapping between a list of relation schema attributes and their values,
%\footnote{\TODO{\*we can probably remove this if run out of space! \*Note that we only have
%pure relational values, thus, values do not have variational counterparts. Hence,
%we do not use our convention of underlying them to show they are variational.}}, 
i.e.,
$\pTuple = \paran {\vi {\underline v} k}$ for the relation schema \vRel \paran {\vi \pAtt k}.
Hence a \emph{relation content}, \pRelCont, is a set of tuples \setDef {\vi \pTuple m}.
$\getAtt {\underline v}$ returns the attribute the value corresponds to.
A \emph{table} \pTab\ is a pair of relation content and relation schema.
A \emph{database instance}, \pInst, of the database \pDB\ with the
schema \pSch, is a set of tables $\setDef {\pTab_1, \ldots, \pTab_n}$.
%relation contents 
%$\setDef  {{\pRelCont}_1, \ldots, {\pRelCont}_n}$ corresponding
%to a set of relation schemas $\setDef {{\pRelSch}_1, \ldots, {\pRelSch}_n}$ 
%defined in \pSch. 
For brevity, when it is clear from the context we refer to a database instance
by \emph{database}.


\begin{figure}

\begin{syntax}

% feature expressions
%\synDef{\dimMeta}{\ffSet}
%  &\eqq& \multicolumn{2}{l}{%
%         \t \myOR \f \myOR \fName \myOR \neg\fName
%         \myOR \dimMeta\wedge\dimMeta \myOR \dimMeta\vee\dimMeta}
%\\[1.5ex]

% relation conditions
\synDef{\pCond}{\pCondSet}
  &\eqq& \multicolumn{2}{l}{%
         \t \myOR \f \myOR \att\bullet\cte \myOR \att\bullet\att
         \myOR \neg\pCond \myOR \pCond\vee\pCond} \\
%     &|& \multicolumn{2}{l}{\vCond\wedge\vCond \myOR \chc{\vCond,\vCond}}
\\[1.5ex]

% variational relational algebra
\synDef{\pQ}{\pQSet}
  &\eqq& \pRel                 & \textit{Relation reference} \\
     &|& \pRen[\pRel]{\pQ}     & \textit{Renaming} \\
     &|& \pPrj[\pAttList]{\pQ} & \textit{Projection} \\
     &|& \pSel\pQ              & \textit{Selection} \\
     &|& \pQ \Join_{\pCond} \pQ  & \textit{Join} \\
%     &|& \chc{\vQ,\vQ}         & \textit{Choice} \\
%     &|& \empRel               & \textit{Empty relation} \\
%    &|& \vQ \times \vQ        & \textit{Cartesian Product} \\
%    &|& \vQ \circ \vQ         & \textit{Set operation} \\
\end{syntax}

\caption{Syntax of  relational algebra, where $\bullet$ ranges over
comparison operators ($<, \leq, =, \neq, >, \geq$), \cte\ over constant values,
\att\ over attribute names, and \pAttList\ over lists of attributes.
The syntactic category
% \dimMeta\ represents feature expressions, 
 \pCond\
is relational conditions, and \pQ\ is  relational algebra terms.
}
%\vspace{-20pt}
\label{fig:v-alg-def}
\end{figure}
%\vspace{-20pt}



Relational algebra allows users to query a relational database~\cite{AliceBook}.
%
The first five constructs are adapted from relational algebra:
%
A query may simply \emph{reference} a relation \pRel\ in the schema.
\emph{Renaming} allows giving a name to an intermediate query to be referenced
 later. Remember that \pRel\ is an overloaded symbol that indicates both a relation
 and a relation name. 
%
A \emph{projection} enables selecting a subset of attributes from the results
of a subquery, for example, \vPrj[\pAtt_1]{\pRel} would return only attribute $\pAtt_1$
from $\pRel$.
%; we extend the standard project operator to work with annotated lists
%of attributes, for example, \vPrj[a_1,a_2^e]{r} would include $a_1$ for all
%configurations and also $a_2$ for configurations where $e$ is true.
%
A \emph{selection} enables filtering the tuples returned by a subquery based on a
given condition \pCond, for example, \vSel[\pAtt_1 > 3]{\pRel} would return all tuples
from $\pRel$ where the value for $\pAtt_1$ is greater than 3.
%; these conditions may be
%variational to enable returning different tuples for different configurations
%of the VDB.
%
The \emph{join} operation joins two subqueries based on a condition and
omitting its condition implies it is a natural join (i.e., join on the
shared attribute of the two subqueries).
For example, $\pRel_1 \bowtie_{\pAtt_1 = \pAtt_2} \pRel_2$ joins tuples from $\pRel_1$ 
and $\pRel_2$ where the attribute $\pAtt_1$ from relation $\pRel_1$ is equal to
attribute $\pAtt_2$ from relation $\pRel_2$. However, if we have $\pRel_1(\pAtt_1, \pAtt_3)$
and $\pRel_2 (\pAtt_1, \pAtt_2)$ then
$\pRel_1 \bowtie \pRel_2$ joins tuples from $\pRel_1$ and $\pRel_2$ where
attribute $\pAtt_1$ has the same value in $\pRel_1$ and $\pRel_2$. 
\TODO{add cross product, add union intersection}

%, except
%that again we allow conditions to be variational.
%
%A \emph{choice} encodes a variation point between two subquery alternatives based on a
%given feature expression, e.g., \chc[f_1\wedge f_2]{\vQ_l,\vQ_r} yields
%the results of $\vQ_l$ alternative for configurations where $f_1$ and $f_2$ are enabled,
%and in other configurations yields the results of $\vQ_r$ alternative. Note that the
%conditions $\vCond$ used by selections and joins also contain choices, and
%these behave similarly.
%\subsubsection{Encoding Variability}
\label{sec:encode-var}


%\point{using a feature set to represent variability within a context.}
To account for variability in a database we need a way to 
encode it.
%
%The first challenge of incorporating variability into a database
%is to represent variability. 
%
To encode variability we first organize the configuration space into
a set of features, denoted by \fSet.
%
%we require a \emph{set of features}, denoted by \fSet, 
%appropriate for the context that the database is used for.
%
For example, in the context of schema evolution, features can be generated from version 
numbers (e.g. features \vOne\ to \vFive\ and \tOne\ to \tFive\ in the 
motivating example, \tabref{mot}); for SPLs, 
the features can be adopted from the SPL feature set (e.g. the \edu\ feature in
our motivating example, \tabref{mot}); and 
for data integration, the features can be representatives of resources.  
For simplicity, the set of features is assumed to be closed and features are
assumed to be boolean variables, however, it is easy to extend them
to multi-valued variables that have finite set of values.
% and without loss of generality, 
%features are assumed to be boolean variables, although, it is easy to extend them
%to multi-valued variables. 
A feature \ensuremath{\fName \in \fSet} can be enabled (i.e., \fName = \t) or disabled (\fName = \f).
%\point{configuration.}
%Assuming that all the features by default are set to \prog{false},
%enabling some of them specifies a variant. 

The features in \fSet\ are used to indicate which parts 
of a variational entity within the database are different 
among different variants. Enabling or disabling each of the 
features in \fSet\ produces a particular \emph{variant} 
of the entity in which all variation has been removed. 
%Enabling or disabling all features of \fSet\ specifies a non-variational \emph{variant}
%that can potentially be generated by \emph{configur}ing its variational counterpart 
%with the variant's \emph{configuration}.
%Hence, to specify a variant
%we define a function, called 
A \emph{configuration} is a \emph{total} function
that maps every feature in the feature set to a boolean value.
%By definition, a configuration is a \emph{total} function,
%i.e., it is defined for \emph{all} features in the feature set. 
For brevity, we represent a configuration  by the set of enabled features.
%which represents a variant. 
%\ensure{make sure referring to a variant can be done by the set of enabled features. HERE!}
For example, in our motivating scenario, the configuration \ensuremath{
\setDef {\vTwo,\tThree,\edu}
}
represents a database variant where only features \vTwo, \tThree, and \edu\ are enabled.
This database variant contains relation schemas of the
employee and education sub-schemas associated with \vTwo\ and
\tThree\ in \tabref{mot}, respectively.
For brevity, we refer to a variant with configuration \config\ as variant \config.
For example, variant \setDef {\vTwo,\tThree,\edu} refers to the variant
with configuration \setDef {\vTwo,\tThree,\edu}.
%and education sub-schema associated with \tThree\ in \tabref{mot}.

%\point{represent variability in db by prop formula of features, called feature expression.}
%Having defined a set of features, we need to incorporate them into the database.
%To encode features in the database, we construct propositional formulas of features
%such that the formula evaluates to \t\ for a set of configurations. 
When describing variation points in the database, we need to 
refer to subsets of the configuration space. We achieve this by 
constructing propositional formulas of features.
Thus, 
such a propositional formula defines a condition that holds for 
a subset of configurations and their corresponding variants. 
%
%describing the condition where one or more variants are present,
%i.e., assigning features to their values defined in variant's configuration and 
%evaluating the propositional formula results in \prog{true}.
%
For example,
the propositional formula $\neg \edu$ represents all variants of our
motivating example that do not 
have the education part of the schema, i.e., variant schemas of the 
left schema column. 

We call a propositional formula of features a \emph{feature expression} and define
it formally in \figref{fexp-def}. 
%\figref{fexp-def} defines the syntax of feature expressions.
The evaluation function of feature expressions 
$\fSem \dimMeta : \ffSet \to \confSet \to \bSet$ evaluates the 
feature expression \dimMeta\ w.r.t. the configuration \config.
%and defined in
%our technical report~\cite{vldbArXiv}, 
%and defined in \appref{fexp},
%evaluates feature expression \dimMeta\ under configuration \config,
%also called \emph{configuration of feature expression \dimMeta\ under \config}. 
For example, $\fSem [\{\A\}] {\A \vee \B} = \t$, however,
%which states that the feature expression
%$\A \vee \B$ evaluates to \t\ w.r.t. configuration that only enables $\A$, however,
$\fSem [\{\}] {\A \vee \B} = \f$, where the empty set indicates neither \A\ nor \B\
are enabled.
%which states that the same feature expression evaluates
%to \f\ when neither $\A$ nor $\B$ are enabled.
Additionally, we define the binary \emph{equivalence ($\equiv$)} relation and
the unary \emph{satisfiable (sat)} and \emph{unsatisfiable (unsat)}
relations over feature expressions in \figref{fexp-def}.
%
% some of the functions needed over feature
%expressions:
%1) \emph{equivalence} of two feature expressions, 
%\ensuremath{\dimMeta_1 \equiv \dimMeta_2}
%and
%2) \emph{satisfiability} of a feature expression, 
%\ensuremath{\sat \dimMeta}.
%However, we define the evaluation of feature expressions and functions over them
%in \appref{fexp}.
%We define two functions over feature expressions, as shown in \figref{fexp-def}:
%1) \emph{satisfiability}: feature expression \dimMeta\ is \emph{satisfiable} if there 
%exists configuration \config\ s.t. \fSem \dimMeta = \t\
%and 2) \emph{tautology}: feature expression \dimMeta\ is a \emph{tautology} if  
%for all valid configurations we have: \fSem \dimMeta = \t.

%\point{annotating elements of database with feature expressions.}
To incorporate feature expressions into the database,
we \emph{annotate/tag} database elements (including attributes, relations, and tuples) 
with feature expressions. An \emph{annotated element} \elem\ with feature expression \dimMeta\
is denoted by \annot \elem. 
%The feature expression \dimMeta\ represents
%the set of configurations where their variants contain element \elem\ because
%
The feature expression attached to an element is called a \emph{presence condition}
since it determines the condition (set of configurations) under which the element is present.
\getPC \elem\ returns the presence condition of the element
\elem.
For example, the
annotated number $2^{\A \vee \B}$ is present in variants with a configuration
that enables either $\A$ or $\B$ or both
but it does not exist in variants that disable both $\A$ and $\B$.
Here, $\getPC {2} = \A \vee \B$.

%\point{relationship between features is captured by a propositional formula, called feature model.} 
No matter the context, features often have a relationship with each other that
constrains configurations. For example, only one of the temporal features of $\vOne - \vFive$
can be \t\ for a given variant.
This relationship can be captured by a feature expression, called a \emph{feature model} and
denoted by \fModel,
which restricts the set of \emph{valid configurations}:
if configuration \config\ violates the relationship then 
%evaluating the feature model \fModel\
%under this configuration evaluates to \f: 
\fSem \fModel = \f.
For example, the restriction that at a given time only one of temporal features $\vOne - \vFive$
can be enabled is represented by:
%the feature expression:
\ensuremath{
\vOne \oplus \vTwo \oplus \vThree \oplus \vFour \oplus \vFive
},
where $\A \oplus \B \oplus \ldots \oplus \fName_n$ is syntactic sugar for $(\A \wedge \neg \B \wedge \ldots \wedge \neg \fName_n) \vee (\neg \A \wedge \B \wedge \ldots \wedge \neg \fName_n) \vee (\neg \A \wedge \neg \B \wedge \ldots \wedge \fName_n)$
, i.e., features are mutually exclusive.
%multiple features cannot be enabled simultaneously.
%$\left(\vOne \wedge \neg \left(\vTwo \vee \hdots \vee \vFive \right) \right)
%\vee \left(\vTwo \wedge \neg \left(\vOne \vee \vThree \vee \vFour \vee \vFive \right) \right) 
%\vee \hdots 
%\vee \left(\vFive \wedge \neg \left(\vOne \vee \hdots \vee \vFour \right) \right)$.
%Note that this is not the feature model for the entire motivating example.



\begin{figure}
%\textbf{Feature expression generic object:}
%\begin{syntax}
%\synDef \fName \fSet &\textit{Feature Name}
%%c \in \mathbf{C} &\textit{Configuration}
%\end{syntax}
%
%\medskip
\textbf{Feature expression syntax:}
\begin{syntax}
\synDef \fName \fSet &&&\textit{Feature Name}\\
\synDef \bTag \bSet &\eqq& \t \myOR \f & \textit{Boolean Value}\\
\synDef \dimMeta \ffSet &\eqq& \bTag \myOR \fName \myOR \neg \dimMeta \myOR \dimMeta \wedge \dimMeta \myOR \dimMeta \vee \dimMeta & \textit{Feature Expression}\\
\synDef \config \confSet &:& \fSet \totype \bSet &\textit{Configuration}
\end{syntax}

\medskip
\textbf{Evaluation of feature expressions:}
\begin{alignat*}{1}
\fSem [] . &: \ffSet \totype \confSet \totype \bSet\\
\fSem \bTag &= \bTag\\
\fSem \fName &= \config \ \fName\\
\fSem {\neg \fName} &= \neg \fSem \fName\\
\fSem {\annd \dimMeta} &= \fSem {\dimMeta_1} \wedge \fSem {\dimMeta_2}\\
\fSem {\orr \dimMeta} &= \fSem {\dimMeta_1} \vee \fSem {\dimMeta_2}\\
\end{alignat*}

\medskip
\textbf{Relations over feature expressions:}
\begin{alignat*}{1}
\dimMeta_1 \equiv \dimMeta_2 &\textit{ iff \ } \forall \config \in \confSet. \fSem {\dimMeta_1} = \fSem {\dimMeta_2}\\
\sat {\dimMeta} &\textit{ iff \ } \exists \config \in \confSet. \fSem {\dimMeta} = \t\\
\unsat {\dimMeta} &\textit{ iff \ } \forall \config \in \confSet. \fSem {\dimMeta} = \f\\
\oneof {\A, \B, \ldots, \fName_n}
&= (\A \wedge \neg \B \wedge \ldots \wedge \neg \fName_n)
\vee (\neg \A \wedge \B \wedge \ldots \wedge \neg \fName_n)\\
&\qquad\vee (\neg \A \wedge \neg \B \wedge \ldots \wedge \fName_n)
%\dimMeta_1 \oplus \dimMeta_2 = (\dimMeta_1 \wedge \neg \dimMeta_2) \vee (\dimMeta_2 \wedge \neg \dimMeta_1)
\end{alignat*}

%\medskip
%\textbf{Syntactic sugar for mutually exclusive features:}
%\begin{alignat*}{1}
%\A \oplus \B \oplus \ldots \oplus \fName_n
%= (\A \wedge \neg \B \wedge \ldots \wedge \neg \fName_n)\\
%\vee (\neg \A \wedge \B \wedge \ldots \wedge \neg \fName_n)\\
%\vee (\neg \A \wedge \neg \B \wedge \ldots \wedge \fName_n)
%\end{alignat*}


\begin{comment}
\medskip
\textbf{Configuration constraint:}
\begin{equation*}
\forall \overrightarrow{f_i} \in c :
%( \neg o_1 \wedge o_2 \wedge o_3 \wedge \ldots \wedge o_{|f_i|})
\bigvee_{1\leq j \leq |f_i|}(\neg o_j \wedge \bigwedge_{\substack{k\not = j\\1 \leq k\leq |f_i|}} o_k)
= \prog{true}
\end{equation*}
\end{comment}

\caption[Feature expression syntax and evaluation]{Feature expression syntax, relations, and evaluation.
% and functions over feature expressions. 
%We informally define \emph{exclusive or} \ensuremath{\oplus} of \ensuremath{n} features to be \t\ 
%only when one feature is \t.
}
\label{fig:fexp-def}
\end{figure}




%\section{Annotations and Variational Sets}
\label{sec:vset}


%\point{annotating elements of database with feature expressions.}
We now introduce the first approach used to incorporate variation into a database.
To incorporate feature expressions into the database,
we \emph{annotate} database elements (including attributes, relations, and tuples) 
with feature expressions. An \emph{annotated element} \elem\ with feature expression \dimMeta\
is denoted by \annot \elem, 
that is, if \elem\ has type \typevar\ (i.e., $\elem \in \typevar$)
then $\annot \elem$ has the corresponding variational type 
$\vartype \typevar$ (i.e., $\annot \elem \in \vartype \typevar$).
%The feature expression \dimMeta\ represents
%the set of configurations where their variants contain element \elem\ because
%
The feature expression attached to an element is called its \emph{presence
condition} since it determines the condition (set of configurations) under
which the element is present in the database. 
This is done by the \emph{configuration} function $\xeSem [] . : \elemSet \totype \confSet \totype \maybe {\pelemSet}$ defined in \figref{vset}.
For example, assuming
$\features=\set{\A,\B}$, the annotated number $\annot [\A \vee \B] 2$ is present
in variants \setDef{\A} (i.e., $\xeSem [\setDef{\A}] {\annot [\A \vee \B] 2}$ = 2), 
\setDef{\B} (i.e., $\xeSem [\setDef{\A}] {\annot [\A \vee \B] 2}$ = 2), 
and \setDef{\A,\B} (i.e.,$\xeSem [\setDef{\A, \B}] {\annot [\A \vee \B] 2} = 2$) 
but not in variant
\setDef{} (i.e., $\xeSem [\setDef { }] {\annot [\A \vee \B] 2} = \bot$). 
%
The operation $\getPC{\annot{\elem}}=e$ returns the presence condition of an
annotated element.
% with a configuration
%that enables either $\A$ or $\B$ or both
%variants that disable both $\A$ and $\B$.
% Here, $\getPC {\annot [\A \vee \B] 2} = \A \vee \B$.

\section{Annotations and Variational Sets}
\label{sec:vset}


%\point{annotating elements of database with feature expressions.}
We now introduce the first approach used to incorporate variation into a database.
To incorporate feature expressions into the database,
we \emph{annotate} database elements (including attributes, relations, and tuples) 
with feature expressions. An \emph{annotated element} \elem\ with feature expression \dimMeta\
is denoted by \annot \elem, 
that is, if \elem\ has type \typevar\ (i.e., $\elem \in \typevar$)
then $\annot \elem$ has the corresponding variational type 
$\vartype \typevar$ (i.e., $\annot \elem \in \vartype \typevar$).
%The feature expression \dimMeta\ represents
%the set of configurations where their variants contain element \elem\ because
%
The feature expression attached to an element is called its \emph{presence
condition} since it determines the condition (set of configurations) under
which the element is present in the database. 
This is done by the \emph{configuration} function $\xeSem [] . : \elemSet \totype \confSet \totype \maybe {\pelemSet}$ defined in \figref{vset}.
For example, assuming
$\features=\set{\A,\B}$, the annotated number $\annot [\A \vee \B] 2$ is present
in variants \setDef{\A} (i.e., $\xeSem [\setDef{\A}] {\annot [\A \vee \B] 2}$ = 2), 
\setDef{\B} (i.e., $\xeSem [\setDef{\A}] {\annot [\A \vee \B] 2}$ = 2), 
and \setDef{\A,\B} (i.e.,$\xeSem [\setDef{\A, \B}] {\annot [\A \vee \B] 2} = 2$) 
but not in variant
\setDef{} (i.e., $\xeSem [\setDef { }] {\annot [\A \vee \B] 2} = \bot$). 
%
The operation $\getPC{\annot{\elem}}=e$ returns the presence condition of an
annotated element.
% with a configuration
%that enables either $\A$ or $\B$ or both
%variants that disable both $\A$ and $\B$.
% Here, $\getPC {\annot [\A \vee \B] 2} = \A \vee \B$.

\section{Annotations and Variational Sets}
\label{sec:vset}


%\point{annotating elements of database with feature expressions.}
We now introduce the first approach used to incorporate variation into a database.
To incorporate feature expressions into the database,
we \emph{annotate} database elements (including attributes, relations, and tuples) 
with feature expressions. An \emph{annotated element} \elem\ with feature expression \dimMeta\
is denoted by \annot \elem, 
that is, if \elem\ has type \typevar\ (i.e., $\elem \in \typevar$)
then $\annot \elem$ has the corresponding variational type 
$\vartype \typevar$ (i.e., $\annot \elem \in \vartype \typevar$).
%The feature expression \dimMeta\ represents
%the set of configurations where their variants contain element \elem\ because
%
The feature expression attached to an element is called its \emph{presence
condition} since it determines the condition (set of configurations) under
which the element is present in the database. 
This is done by the \emph{configuration} function $\xeSem [] . : \elemSet \totype \confSet \totype \maybe {\pelemSet}$ defined in \figref{vset}.
For example, assuming
$\features=\set{\A,\B}$, the annotated number $\annot [\A \vee \B] 2$ is present
in variants \setDef{\A} (i.e., $\xeSem [\setDef{\A}] {\annot [\A \vee \B] 2}$ = 2), 
\setDef{\B} (i.e., $\xeSem [\setDef{\A}] {\annot [\A \vee \B] 2}$ = 2), 
and \setDef{\A,\B} (i.e.,$\xeSem [\setDef{\A, \B}] {\annot [\A \vee \B] 2} = 2$) 
but not in variant
\setDef{} (i.e., $\xeSem [\setDef { }] {\annot [\A \vee \B] 2} = \bot$). 
%
The operation $\getPC{\annot{\elem}}=e$ returns the presence condition of an
annotated element.
% with a configuration
%that enables either $\A$ or $\B$ or both
%variants that disable both $\A$ and $\B$.
% Here, $\getPC {\annot [\A \vee \B] 2} = \A \vee \B$.

\input{formulas/vset}

%\point{vset.}
A \emph{variational set} (\emph{v-set}) $\vset = \setDef {\annot [\dimMeta_1] {\elem_1},\ldots, \annot [\dimMeta_n] {\elem_n}}$ 
is a set of annotated elements, 
that is,
$\vset \in \vsetSet$~\cite{EWC13fosd,Walk14onward,ATW17dbpl}.
We typically omit the presence condition \t\ in a variational set,
e.g., $\annot [\t] 4 = 4$.
% where the presence condition of elements is satisfiable~\cite{EWC13fosd,Walk14onward,vdb17ATW}. 
%
Conceptually, a variational set represents many different plain sets simultaneously.
These plain sets can be generated by \emph{configuring} a variational set with a configuration.
This is done by the \emph{variational set configuration} function
\ensuremath{\osetSem \vset: \vsetSet \totype \confSet \totype \psetSet}, defined in \figref{vset}.
The configuration function evaluates the presence condition $\dimMeta_i$ of each 
element $\elem_i$ of the variational set with the configuration \config. 
If the evaluation results in \t\ it includes $\elem_i$ in the plain set and otherwise it
does not. \exref{vset-conf} illustrates the configuration of a variational set for all
possible configurations. 
\structure{it'd be nice to have the entire ex in the same page.}

\begin{example}
\label{eg:vset-conf}
Assume we have the feature space $\features = \setDef {\A, \B}$ 
and the variational set $\vset_1 = \setDef {\annot [\A] 2, \annot [\B] 3, 4}$.
$\vset_1$ represents four plain sets:
\begin{alignat*}{1}
\osetSem {\vset_1} &=
\begin{cases}
  \setDef{2,3,4}, & \config = \setDef{\A,\B}\\
  \setDef{2,4}, & \config = \setDef{\A}\\
  \setDef{3,4}, & \config = \setDef{\B}\\
  \setDef{4}, & \config = \setDef { }
\end{cases}
\end{alignat*}
This states that, for example, configuring $\vset_1$ for the variant that enables 
bot \A\ and \B\ (that is, \ensuremath{\A = \t, \B = \t}) results in the plain set
\ensuremath{ \osetSem [\setDef {\A, \B}] {\vset_1} = \setDef {2,3,4} }.
\end{example}

%
%\noindent
Following database notational conventions
we drop the brackets of a variational set when used in database
schema definitions and queries.

%\point{annotated vset.}
A variational set itself can also be annotated with a feature expression.
%
%An \emph{annotated variational set} 
$\annot \vset = \setDef {\annot [\dimMeta_1] {\elem_1},\ldots,\annot [\dimMeta_n] {\elem_n}}^\dimMeta$ is an
\emph{annotated variational set}, 
that is, $\annot \vset \in \annotvsetSet$.
% that it is annotated itself by a \emph{feature expression} \dimMeta.
%We denote an annotated variational set of elements $\elem \in \mathbf{\elemSet}$ with
%\annot \elemSet.
Annotating a variational set with the feature expression \dimMeta\ means that all
elements in the variational set are only present when \dimMeta\ evaluates to \t.
The \emph{normalization} operation $\pushIn {\annot \vset}$ applies this
restriction by pushing it into the presence conditions of the individual
elements:
\ensuremath{
\pushIn {\annot \vset}
= 
\setDef{\annot [\dimMeta_i \wedge \dimMeta] {\elem_i} \myOR 
\annot [\dimMeta_i] \elem_i \in \annot \vset, \sat {\dimMeta_i \wedge \dimMeta}
}}.
%\eric{added that both v-set and annot v-set are of the same type.}
%Thus, we consider both variational sets and annotated variational sets to 
%belong to the set of variational set \vsetSet, that is, we consider them to have the same type. 
Note that both the normalization operation and variational set configuration
are overloaded, that is, they are defined for both variational sets and 
annotated variational sets. 
Also, note that the \emph{normalization} operation also removes elements
with unsatisfiable presence conditions and may also be applied
to an unannotated variational set \vset\ since $\annot[\t]{\vset} = \vset$.
%\ensuremath{
%\vset = \setDef {\annot [\dimMeta_1] \elem_1, \ldots, \annot [\dimMeta_n] \elem_n}}, 
%which is equivalent to the annotated v-set \annot [\t] \vset. Thus,
%\ensuremath{
%\pushIn \vset = \setDef {
%\annot [\dimMeta_i] \elem_i \myOR \annot [\dimMeta_i] \elem_i, \sat {\dimMeta_i}
%}
%}.}
%This restriction
%can be captured by the property:
%$\setDef {\annot [\dimMeta_1] {\elem_1} ,\ldots, \annot [\dimMeta_n] {\elem_n}}^\dimMeta
%\equiv 
%\setDef {\annot [\dimMeta_1 \wedge \dimMeta] {\elem_1},\ldots, \annot [\dimMeta_n \wedge \dimMeta] {\elem_n}}
%$.
%
For example, the annotated variational set
$\vset_1 = \{\annot [\A] 2, \annot [\neg \B] 3, 4, \annot [\C] 5\}^{\A \wedge \B}$
indicates that all the elements of the set can only exist
when both $\A$ and $\B$ are enabled. Thus, normalizing the variational set $\vset_1$
%the set's feature expression
results in
$\{\annot [\A \wedge \B] 2,\annot [\A \wedge \B] 4,\annot [\A \wedge \B \wedge \C] 5\}$. The element $3$ is dropped 
%from the set 
since 
\ensuremath{\neg \sat {\getPCfrom 3 {\vset_1} }},
where
\ensuremath{
{\getPCfrom 3 {\vset_1} } = \neg \B \wedge (\A \wedge \B)}.
%its presence condition is unsatisfiable, i.e., $\neg \sat {\neg \fName_2 \wedge (\fName_1 \wedge \fName_2)}$.
%%
Note that we use the function \getPCfrom \elem {\annot \vset} to 
return the presence condition of a unique variational element within a bigger
variational structure. 
Note that,
without loss of generality, we assume that elements in a variational set
are unique since we can simply disjoin the presence conditions of a repeated 
element, that is, 
\ensuremath{\setDef {\annot [\dimMeta] \elem, \annot [\dimMeta] \elem, \annot [\dimMeta_1] \elem_1, \ldots, \annot [\dimMeta_n] \elem_n} = \setDef {\annot [\dimMeta \vee \VVal \dimMeta] \elem, \annot [\dimMeta_1] \elem_1, \ldots, \annot [\dimMeta_n] \elem_n}}.
% by just referring to the element itself without its
%annotation, i.e., \elem.

In \figref{vset}, we also define several operations, such as union and
intersection, over variational sets; these operations are used in \secref{type-sys}. The
semantics of a variational set operation is equivalent to applying the corresponding
plain set operation to every corresponding variant of the argument variational sets. For
example, the union of two variational sets $\vset_1\cup\vset_2$ should produce a new
variational set $\vset_3$ such that
%
$\forall c\in\confSet.\;
\osetSem{\vset_3} = \osetSem{\vset_1}\,\underline{\cup}\,\osetSem{\vset_2}$,
where $\underline{\cup}$ is the plain set union operation.
%
 This property must hold for all operations over variational sets, that is, for all possible operations, \vsetOp, defined on variational sets the property 
 \ensuremath{
 \Pone: 
 \forall \config \in \confSet. \osetSem {\pushIn {\vset_1} \vsetOp \pushIn {\vset_2}} 
 = \osetSem {\vset_1} \psetOp \osetSem {\vset_2}
 } must hold, where \psetOp\ is the counterpart operation on plain sets.%
\footnote{This property is proved for the operations we define over variational sets in Coq proof assistant~\cite{Khan21}.}



%\point{vset.}
A \emph{variational set} (\emph{v-set}) $\vset = \setDef {\annot [\dimMeta_1] {\elem_1},\ldots, \annot [\dimMeta_n] {\elem_n}}$ 
is a set of annotated elements, 
that is,
$\vset \in \vsetSet$~\cite{EWC13fosd,Walk14onward,ATW17dbpl}.
We typically omit the presence condition \t\ in a variational set,
e.g., $\annot [\t] 4 = 4$.
% where the presence condition of elements is satisfiable~\cite{EWC13fosd,Walk14onward,vdb17ATW}. 
%
Conceptually, a variational set represents many different plain sets simultaneously.
These plain sets can be generated by \emph{configuring} a variational set with a configuration.
This is done by the \emph{variational set configuration} function
\ensuremath{\osetSem \vset: \vsetSet \totype \confSet \totype \psetSet}, defined in \figref{vset}.
The configuration function evaluates the presence condition $\dimMeta_i$ of each 
element $\elem_i$ of the variational set with the configuration \config. 
If the evaluation results in \t\ it includes $\elem_i$ in the plain set and otherwise it
does not. \exref{vset-conf} illustrates the configuration of a variational set for all
possible configurations. 
\structure{it'd be nice to have the entire ex in the same page.}

\begin{example}
\label{eg:vset-conf}
Assume we have the feature space $\features = \setDef {\A, \B}$ 
and the variational set $\vset_1 = \setDef {\annot [\A] 2, \annot [\B] 3, 4}$.
$\vset_1$ represents four plain sets:
\begin{alignat*}{1}
\osetSem {\vset_1} &=
\begin{cases}
  \setDef{2,3,4}, & \config = \setDef{\A,\B}\\
  \setDef{2,4}, & \config = \setDef{\A}\\
  \setDef{3,4}, & \config = \setDef{\B}\\
  \setDef{4}, & \config = \setDef { }
\end{cases}
\end{alignat*}
This states that, for example, configuring $\vset_1$ for the variant that enables 
bot \A\ and \B\ (that is, \ensuremath{\A = \t, \B = \t}) results in the plain set
\ensuremath{ \osetSem [\setDef {\A, \B}] {\vset_1} = \setDef {2,3,4} }.
\end{example}

%
%\noindent
Following database notational conventions
we drop the brackets of a variational set when used in database
schema definitions and queries.

%\point{annotated vset.}
A variational set itself can also be annotated with a feature expression.
%
%An \emph{annotated variational set} 
$\annot \vset = \setDef {\annot [\dimMeta_1] {\elem_1},\ldots,\annot [\dimMeta_n] {\elem_n}}^\dimMeta$ is an
\emph{annotated variational set}, 
that is, $\annot \vset \in \annotvsetSet$.
% that it is annotated itself by a \emph{feature expression} \dimMeta.
%We denote an annotated variational set of elements $\elem \in \mathbf{\elemSet}$ with
%\annot \elemSet.
Annotating a variational set with the feature expression \dimMeta\ means that all
elements in the variational set are only present when \dimMeta\ evaluates to \t.
The \emph{normalization} operation $\pushIn {\annot \vset}$ applies this
restriction by pushing it into the presence conditions of the individual
elements:
\ensuremath{
\pushIn {\annot \vset}
= 
\setDef{\annot [\dimMeta_i \wedge \dimMeta] {\elem_i} \myOR 
\annot [\dimMeta_i] \elem_i \in \annot \vset, \sat {\dimMeta_i \wedge \dimMeta}
}}.
%\eric{added that both v-set and annot v-set are of the same type.}
%Thus, we consider both variational sets and annotated variational sets to 
%belong to the set of variational set \vsetSet, that is, we consider them to have the same type. 
Note that both the normalization operation and variational set configuration
are overloaded, that is, they are defined for both variational sets and 
annotated variational sets. 
Also, note that the \emph{normalization} operation also removes elements
with unsatisfiable presence conditions and may also be applied
to an unannotated variational set \vset\ since $\annot[\t]{\vset} = \vset$.
%\ensuremath{
%\vset = \setDef {\annot [\dimMeta_1] \elem_1, \ldots, \annot [\dimMeta_n] \elem_n}}, 
%which is equivalent to the annotated v-set \annot [\t] \vset. Thus,
%\ensuremath{
%\pushIn \vset = \setDef {
%\annot [\dimMeta_i] \elem_i \myOR \annot [\dimMeta_i] \elem_i, \sat {\dimMeta_i}
%}
%}.}
%This restriction
%can be captured by the property:
%$\setDef {\annot [\dimMeta_1] {\elem_1} ,\ldots, \annot [\dimMeta_n] {\elem_n}}^\dimMeta
%\equiv 
%\setDef {\annot [\dimMeta_1 \wedge \dimMeta] {\elem_1},\ldots, \annot [\dimMeta_n \wedge \dimMeta] {\elem_n}}
%$.
%
For example, the annotated variational set
$\vset_1 = \{\annot [\A] 2, \annot [\neg \B] 3, 4, \annot [\C] 5\}^{\A \wedge \B}$
indicates that all the elements of the set can only exist
when both $\A$ and $\B$ are enabled. Thus, normalizing the variational set $\vset_1$
%the set's feature expression
results in
$\{\annot [\A \wedge \B] 2,\annot [\A \wedge \B] 4,\annot [\A \wedge \B \wedge \C] 5\}$. The element $3$ is dropped 
%from the set 
since 
\ensuremath{\neg \sat {\getPCfrom 3 {\vset_1} }},
where
\ensuremath{
{\getPCfrom 3 {\vset_1} } = \neg \B \wedge (\A \wedge \B)}.
%its presence condition is unsatisfiable, i.e., $\neg \sat {\neg \fName_2 \wedge (\fName_1 \wedge \fName_2)}$.
%%
Note that we use the function \getPCfrom \elem {\annot \vset} to 
return the presence condition of a unique variational element within a bigger
variational structure. 
Note that,
without loss of generality, we assume that elements in a variational set
are unique since we can simply disjoin the presence conditions of a repeated 
element, that is, 
\ensuremath{\setDef {\annot [\dimMeta] \elem, \annot [\dimMeta] \elem, \annot [\dimMeta_1] \elem_1, \ldots, \annot [\dimMeta_n] \elem_n} = \setDef {\annot [\dimMeta \vee \VVal \dimMeta] \elem, \annot [\dimMeta_1] \elem_1, \ldots, \annot [\dimMeta_n] \elem_n}}.
% by just referring to the element itself without its
%annotation, i.e., \elem.

In \figref{vset}, we also define several operations, such as union and
intersection, over variational sets; these operations are used in \secref{type-sys}. The
semantics of a variational set operation is equivalent to applying the corresponding
plain set operation to every corresponding variant of the argument variational sets. For
example, the union of two variational sets $\vset_1\cup\vset_2$ should produce a new
variational set $\vset_3$ such that
%
$\forall c\in\confSet.\;
\osetSem{\vset_3} = \osetSem{\vset_1}\,\underline{\cup}\,\osetSem{\vset_2}$,
where $\underline{\cup}$ is the plain set union operation.
%
 This property must hold for all operations over variational sets, that is, for all possible operations, \vsetOp, defined on variational sets the property 
 \ensuremath{
 \Pone: 
 \forall \config \in \confSet. \osetSem {\pushIn {\vset_1} \vsetOp \pushIn {\vset_2}} 
 = \osetSem {\vset_1} \psetOp \osetSem {\vset_2}
 } must hold, where \psetOp\ is the counterpart operation on plain sets.%
\footnote{This property is proved for the operations we define over variational sets in Coq proof assistant~\cite{Khan21}.}



%\point{vset.}
A \emph{variational set} (\emph{v-set}) $\vset = \setDef {\annot [\dimMeta_1] {\elem_1},\ldots, \annot [\dimMeta_n] {\elem_n}}$ 
is a set of annotated elements, 
that is,
$\vset \in \vsetSet$~\cite{EWC13fosd,Walk14onward,ATW17dbpl}.
We typically omit the presence condition \t\ in a variational set,
e.g., $\annot [\t] 4 = 4$.
% where the presence condition of elements is satisfiable~\cite{EWC13fosd,Walk14onward,vdb17ATW}. 
%
Conceptually, a variational set represents many different plain sets simultaneously.
These plain sets can be generated by \emph{configuring} a variational set with a configuration.
This is done by the \emph{variational set configuration} function
\ensuremath{\osetSem \vset: \vsetSet \totype \confSet \totype \psetSet}, defined in \figref{vset}.
The configuration function evaluates the presence condition $\dimMeta_i$ of each 
element $\elem_i$ of the variational set with the configuration \config. 
If the evaluation results in \t\ it includes $\elem_i$ in the plain set and otherwise it
does not. \exref{vset-conf} illustrates the configuration of a variational set for all
possible configurations. 
\structure{it'd be nice to have the entire ex in the same page.}

\begin{example}
\label{eg:vset-conf}
Assume we have the feature space $\features = \setDef {\A, \B}$ 
and the variational set $\vset_1 = \setDef {\annot [\A] 2, \annot [\B] 3, 4}$.
$\vset_1$ represents four plain sets:
\begin{alignat*}{1}
\osetSem {\vset_1} &=
\begin{cases}
  \setDef{2,3,4}, & \config = \setDef{\A,\B}\\
  \setDef{2,4}, & \config = \setDef{\A}\\
  \setDef{3,4}, & \config = \setDef{\B}\\
  \setDef{4}, & \config = \setDef { }
\end{cases}
\end{alignat*}
This states that, for example, configuring $\vset_1$ for the variant that enables 
bot \A\ and \B\ (that is, \ensuremath{\A = \t, \B = \t}) results in the plain set
\ensuremath{ \osetSem [\setDef {\A, \B}] {\vset_1} = \setDef {2,3,4} }.
\end{example}

%
%\noindent
Following database notational conventions
we drop the brackets of a variational set when used in database
schema definitions and queries.

%\point{annotated vset.}
A variational set itself can also be annotated with a feature expression.
%
%An \emph{annotated variational set} 
$\annot \vset = \setDef {\annot [\dimMeta_1] {\elem_1},\ldots,\annot [\dimMeta_n] {\elem_n}}^\dimMeta$ is an
\emph{annotated variational set}, 
that is, $\annot \vset \in \annotvsetSet$.
% that it is annotated itself by a \emph{feature expression} \dimMeta.
%We denote an annotated variational set of elements $\elem \in \mathbf{\elemSet}$ with
%\annot \elemSet.
Annotating a variational set with the feature expression \dimMeta\ means that all
elements in the variational set are only present when \dimMeta\ evaluates to \t.
The \emph{normalization} operation $\pushIn {\annot \vset}$ applies this
restriction by pushing it into the presence conditions of the individual
elements:
\ensuremath{
\pushIn {\annot \vset}
= 
\setDef{\annot [\dimMeta_i \wedge \dimMeta] {\elem_i} \myOR 
\annot [\dimMeta_i] \elem_i \in \annot \vset, \sat {\dimMeta_i \wedge \dimMeta}
}}.
%\eric{added that both v-set and annot v-set are of the same type.}
%Thus, we consider both variational sets and annotated variational sets to 
%belong to the set of variational set \vsetSet, that is, we consider them to have the same type. 
Note that both the normalization operation and variational set configuration
are overloaded, that is, they are defined for both variational sets and 
annotated variational sets. 
Also, note that the \emph{normalization} operation also removes elements
with unsatisfiable presence conditions and may also be applied
to an unannotated variational set \vset\ since $\annot[\t]{\vset} = \vset$.
%\ensuremath{
%\vset = \setDef {\annot [\dimMeta_1] \elem_1, \ldots, \annot [\dimMeta_n] \elem_n}}, 
%which is equivalent to the annotated v-set \annot [\t] \vset. Thus,
%\ensuremath{
%\pushIn \vset = \setDef {
%\annot [\dimMeta_i] \elem_i \myOR \annot [\dimMeta_i] \elem_i, \sat {\dimMeta_i}
%}
%}.}
%This restriction
%can be captured by the property:
%$\setDef {\annot [\dimMeta_1] {\elem_1} ,\ldots, \annot [\dimMeta_n] {\elem_n}}^\dimMeta
%\equiv 
%\setDef {\annot [\dimMeta_1 \wedge \dimMeta] {\elem_1},\ldots, \annot [\dimMeta_n \wedge \dimMeta] {\elem_n}}
%$.
%
For example, the annotated variational set
$\vset_1 = \{\annot [\A] 2, \annot [\neg \B] 3, 4, \annot [\C] 5\}^{\A \wedge \B}$
indicates that all the elements of the set can only exist
when both $\A$ and $\B$ are enabled. Thus, normalizing the variational set $\vset_1$
%the set's feature expression
results in
$\{\annot [\A \wedge \B] 2,\annot [\A \wedge \B] 4,\annot [\A \wedge \B \wedge \C] 5\}$. The element $3$ is dropped 
%from the set 
since 
\ensuremath{\neg \sat {\getPCfrom 3 {\vset_1} }},
where
\ensuremath{
{\getPCfrom 3 {\vset_1} } = \neg \B \wedge (\A \wedge \B)}.
%its presence condition is unsatisfiable, i.e., $\neg \sat {\neg \fName_2 \wedge (\fName_1 \wedge \fName_2)}$.
%%
Note that we use the function \getPCfrom \elem {\annot \vset} to 
return the presence condition of a unique variational element within a bigger
variational structure. 
Note that,
without loss of generality, we assume that elements in a variational set
are unique since we can simply disjoin the presence conditions of a repeated 
element, that is, 
\ensuremath{\setDef {\annot [\dimMeta] \elem, \annot [\dimMeta] \elem, \annot [\dimMeta_1] \elem_1, \ldots, \annot [\dimMeta_n] \elem_n} = \setDef {\annot [\dimMeta \vee \VVal \dimMeta] \elem, \annot [\dimMeta_1] \elem_1, \ldots, \annot [\dimMeta_n] \elem_n}}.
% by just referring to the element itself without its
%annotation, i.e., \elem.

In \figref{vset}, we also define several operations, such as union and
intersection, over variational sets; these operations are used in \secref{type-sys}. The
semantics of a variational set operation is equivalent to applying the corresponding
plain set operation to every corresponding variant of the argument variational sets. For
example, the union of two variational sets $\vset_1\cup\vset_2$ should produce a new
variational set $\vset_3$ such that
%
$\forall c\in\confSet.\;
\osetSem{\vset_3} = \osetSem{\vset_1}\,\underline{\cup}\,\osetSem{\vset_2}$,
where $\underline{\cup}$ is the plain set union operation.
%
 This property must hold for all operations over variational sets, that is, for all possible operations, \vsetOp, defined on variational sets the property 
 \ensuremath{
 \Pone: 
 \forall \config \in \confSet. \osetSem {\pushIn {\vset_1} \vsetOp \pushIn {\vset_2}} 
 = \osetSem {\vset_1} \psetOp \osetSem {\vset_2}
 } must hold, where \psetOp\ is the counterpart operation on plain sets.%
\footnote{This property is proved for the operations we define over variational sets in Coq proof assistant~\cite{Khan21}.}


\section{The Formula Choice Calculus}
\label{sec:fcc}


%To account for variation, VRA combines relational algebra (RA) with 
%\emph{choices}~\cite{EW11tosem,HW16fosd,Walk13thesis}.
%%\point{choice.}
%A choice $\chc{\elem_1,\elem_2}$ consists of a feature expression \dimMeta, called
%the \emph{dimension} of the choice, and 
%two \emph{alternatives} $\elem_1$ and $\elem_2$. For a given configuration \config, 
%the choice $\chc{\elem_1, \elem_2}$ can be replaced by $\elem_1$ if \dimMeta\
%evaluates to \t\ under configuration \config, (i.e., \fSem{\dimMeta}),
%or $\elem_2$ otherwise. 

\eric{please read the entire section. thx!}
The second approach we use to incorporate variation into queries is
the formula choice calculus~\cite{HW16fosd} which is an extension of 
the choice calculus~\cite{Walk13thesis,EW11tosem}. 
%
The choice calculus~\cite{Walk13thesis,EW11tosem} is a metalanguage for
describing variation in programs and its elements such as data 
structures~\cite{Walk14onward,EWC13fosd}.
In the choice calculus, variation is represented in-place as
choices between alternative subexpressions. For example, 
the variational expression 
$\mathit{expr} = \chc [\A] {1,2} + \chc [\B] {3,4} + \chc [\A] {5,6}$
 contains three choices.
Each choice has an associated \emph{dimension}, which is used to
synchronize the choice with other choices in different parts
of the expression. For example, expression $\mathit{expr}$ contains
two dimensions, $\A$ and $\B$, and the two choices in dimension
$\A$ are synchronized. Therefore, the variational expression
$\mathit{expr}$ represents four different plain expressions, depending
on whether the left or right alternatives are selected from each
dimension. Assuming that dimensions may be set to boolean values
where \t\ indicates the left alternative and \f\ indicates the
right alternative, we have: 
%(1) $1+3+5$, when $A$ and $B$ are \t,
%(2) $1+4+5$, when $A$ is \t\ and $B$ is \f,
%(3) $2+3+6$, when $A$ is \f\ and $B$ is \t,
%and (4) $2+4+6$, when $A$ and $B$ are \f.
\begin{alignat*}{1}
\chc [\A] {1,2} + \chc [\B] {3,4} + \chc [\A] {5,6} &=
\begin{cases}
  1+3+5,& \A =\t, \B = \t\\
  1+4+5,& \A =\t, \B = \f\\
  2+3+6,& \A =\f, \B = \t\\
  2+4+6,& \A =\f, \B = \f
\end{cases}
\end{alignat*}
%
\noindent
The formula
choice calculus extends the choice calculus 
by allowing dimensions to be propositional formulas~\cite{HW16fosd}. For example,
the variational expression $\VVal {\mathit{expr}} = \chc [\A \vee \B] {1,2} + \chc [\B] {3,4} + \chc [\A] {5,6}$ represents
four plain expressions: 
%(1) $1$, when $\A \vee \B$ evaluates to \t\
%and (2) $2$, when $\A \vee \B$ evaluates to \f. More explicitly, we have:
\begin{alignat*}{1}
\chc [\A \vee \B] {1,2} + \chc [\B] {3,4} + \chc [\A] {5,6}&=
\begin{cases}
  1+3+5,& \A =\t, \B = \t\\
  1+4+5,& \A =\t, \B = \f\\
  1+3+6,& \A =\f, \B = \t\\
  2+4+6,& \A =\f, \B = \f
\end{cases}
\end{alignat*}



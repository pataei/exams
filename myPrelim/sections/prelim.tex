\section{Preliminaries}
\label{sec:prelim}

$1^{\vOne}$
\\

$\chc [\vOne \wedge \vTwo] {1,2} + \chc [\vOne] {3,6} 
= \left\{ \begin{array}{rl}
1+3, & \vOne = \t, \vTwo = \t\\
2+3, & \vOne = \t, \vTwo = \f\\
2+6, & \vOne = \f, \vTwo = \t\\
2+6, & \vOne = \f, \vTwo = \f
\end{array} \right.
$
\\

$\{1^{\vOne}, 2, 3^{\vOne \vee \vTwo} \}
=\left\{ \begin{array} {ll}
\{1,2,3\}, & \vOne = \t, \vTwo = \t\\
\{1,2,3\}, & \vOne = \t, \vTwo = \f\\
\{2,3\}, & \vOne = \f, \vTwo = \t\\
\{2\}, & \vOne = \f, \vTwo = \f
\end{array} \right.
$
\\

$\{1^{\vOne}, 2, 3^{\vOne \vee \vTwo} \}^{\vTwo}$
\\

$\pushIn {\{1^{\vOne}, 2, 3^{\vOne \vee \vTwo} \}^{\vTwo}} = 
\{1^{\vOne \wedge \vTwo}, 2^{\vTwo}, 3^{\vTwo} \}$
\\

In this section, we introduce concepts and notations that we use
throughout the proposal. \tabref{notations} 
provides a short overview 
and is meant as an aid to find definitions
faster.
% (\S\ indicates the section(s) containing the definition).
Throughout the proposal, we discuss relational concepts and their
variational counterparts. For clarity, when we need to emphasize 
an entity is not variational we underline it, e.g., \pElem\ is a 
non-variational entity while \elem\ is its variational counterpart,
if it exists.

\begin{table}[H]
\caption{Introduced notations and terminologies with their corresponding section(s).}
\label{tab:notations}
\begin{center}
\small
\begin{tabular}{ |l c r| } 
 \hline
Name & Notation & Section \\
\hline
Feature & \fName & \multirow{7}{*} {\secref{encode-var}} \\
Feature expression & \dimMeta & \\
Annotated element \elem\ by \dimMeta & \annot \elem & \\
Configuration & \config & \\
%True feature set of \config & \fct & \\
Evaluation of \dimMeta\ under \config & \fSem \dimMeta & \\
Presence condition of entity \elem & \getPC \elem & \\
\hline
Optional attribute & \vAtt & \multirow{4}{*}{\secref{vsch}}\\
Variational attribute set & \vAttList & \\
Variational relation schema & \vRelSch & \\
Variational schema & \vSch & \\
\hline
Variational tuple & \tuple & \multirow{3}{*}{\secref{vtab}} \\
Variational relation content & \vRelCont & \\
Variational table & \vTab\ = (\vRelSch, \vRelCont) & \\
\hline
Choice & $\chc {x, y}$ & \multirow{3}{*}{\secref{vrel-alg}}\\
Variational condition & \vCond & \\
Variational query & \vQ & \\
\hline
%Configure query \vQ\ with configuration \config & \eeSem \vQ & \multirow{2}{*}{\secref{vra-sem}}\\
%Group query \vQ & \qGroup \vQ &\\
%\hline
\end{tabular}
\end{center}
\end{table}

\subsection{Relational Databases and Relational Algebra}
\label{sec:rel-db}

%\point{database schema.}
A relational database \pDB\ stores information in a structured manner by forcing
data to conform to a \emph{schema} \pSch\ that is a finite set 
$\setDef {{\pRelSch}_1, \ldots, {\pRelSch}_n}$ of \emph{relation schemas}.
A relation schema is defined as
$\pRelSch = \vRel \paran {\vi \pAtt k}$ where each $\pAtt_i$ is an
\emph{attribute} contained in a relation named \vRel. 
$\getRel {\pAtt}$ returns the relation that contains the attribute.
\type [\pAtt]\ returns the \emph{type} of values associated with attribute \pAtt.

%\point{database content.}
The content of database \pDB\ is stored in the form of \emph{tuples}. A tuple \pTuple\
is a mapping between a list of relation schema attributes and their values,
%\footnote{\TODO{\*we can probably remove this if run out of space! \*Note that we only have
%pure relational values, thus, values do not have variational counterparts. Hence,
%we do not use our convention of underlying them to show they are variational.}}, 
i.e.,
$\pTuple = \paran {\vi {\underline v} k}$ for the relation schema \vRel \paran {\vi \pAtt k}.
Hence a \emph{relation content}, \pRelCont, is a set of tuples \setDef {\vi \pTuple m}.
$\getAtt {\underline v}$ returns the attribute the value corresponds to.
A \emph{table} \pTab\ is a pair of relation content and relation schema.
A \emph{database instance}, \pInst, of the database \pDB\ with the
schema \pSch, is a set of tables $\setDef {\pTab_1, \ldots, \pTab_n}$.
%relation contents 
%$\setDef  {{\pRelCont}_1, \ldots, {\pRelCont}_n}$ corresponding
%to a set of relation schemas $\setDef {{\pRelSch}_1, \ldots, {\pRelSch}_n}$ 
%defined in \pSch. 
For brevity, when it is clear from the context we refer to a database instance
by \emph{database}.


\begin{figure}

\begin{syntax}

% feature expressions
%\synDef{\dimMeta}{\ffSet}
%  &\eqq& \multicolumn{2}{l}{%
%         \t \myOR \f \myOR \fName \myOR \neg\fName
%         \myOR \dimMeta\wedge\dimMeta \myOR \dimMeta\vee\dimMeta}
%\\[1.5ex]

% relation conditions
\synDef{\pCond}{\pCondSet}
  &\eqq& \multicolumn{2}{l}{%
         \t \myOR \f \myOR \att\bullet\cte \myOR \att\bullet\att
         \myOR \neg\pCond \myOR \pCond\vee\pCond} \\
%     &|& \multicolumn{2}{l}{\vCond\wedge\vCond \myOR \chc{\vCond,\vCond}}
\\[1.5ex]

% variational relational algebra
\synDef{\pQ}{\pQSet}
  &\eqq& \pRel                 & \textit{Relation reference} \\
     &|& \pRen[\pRel]{\pQ}     & \textit{Renaming} \\
     &|& \pPrj[\pAttList]{\pQ} & \textit{Projection} \\
     &|& \pSel\pQ              & \textit{Selection} \\
     &|& \pQ \times \pQ & \textit{Cartesian product}\\
     &|& \pQ \Join_{\pCond} \pQ  & \textit{Join} \\
     &|& \pQ \circ \pQ & \textit{Set operation}\\
%     &|& \chc{\vQ,\vQ}         & \textit{Choice} \\
%     &|& \empRel               & \textit{Empty relation} \\
%    &|& \vQ \times \vQ        & \textit{Cartesian Product} \\
%    &|& \vQ \circ \vQ         & \textit{Set operation} \\
\end{syntax}

\caption{Syntax of  relational algebra, where $\bullet$ ranges over
comparison operators ($<, \leq, =, \neq, >, \geq$), $\circ$ over 
set operations ($\cap, \cup$), \cte\ over constant values,
\att\ over attribute names, and \pAttList\ over lists of attributes.
The syntactic category
% \dimMeta\ represents feature expressions, 
 \pCond\
is relational conditions, and \pQ\ is  relational algebra terms.
}
%\vspace{-20pt}
\label{fig:rel-alg}
\end{figure}
%\vspace{-20pt}



Relational algebra allows users to query a relational database~\cite{AliceBook}.
%
The first five constructs are adapted from relational algebra:
%
A query may simply \emph{reference} a relation \pRel\ in the schema.
\emph{Renaming} allows giving a name to an intermediate query to be referenced
 later. Remember that \pRel\ is an overloaded symbol that indicates both a relation
 and a relation name. 
%
A \emph{projection} enables selecting a subset of attributes from the results
of a subquery, for example, \vPrj[\pAtt_1]{\pRel} would return only attribute $\pAtt_1$
from $\pRel$.
%; we extend the standard project operator to work with annotated lists
%of attributes, for example, \vPrj[a_1,a_2^e]{r} would include $a_1$ for all
%configurations and also $a_2$ for configurations where $e$ is true.
%
A \emph{selection} enables filtering the tuples returned by a subquery based on a
given condition \pCond, for example, \vSel[\pAtt_1 > 3]{\pRel} would return all tuples
from $\pRel$ where the value for $\pAtt_1$ is greater than 3.
%; these conditions may be
%variational to enable returning different tuples for different configurations
%of the VDB.
%
The \emph{join} operation joins two subqueries based on a condition and
omitting its condition implies it is a natural join (i.e., join on the
shared attribute of the two subqueries).
For example, $\pRel_1 \bowtie_{\pAtt_1 = \pAtt_2} \pRel_2$ joins tuples from $\pRel_1$ 
and $\pRel_2$ where the attribute $\pAtt_1$ from relation $\pRel_1$ is equal to
attribute $\pAtt_2$ from relation $\pRel_2$. However, if we have $\pRel_1(\pAtt_1, \pAtt_3)$
and $\pRel_2 (\pAtt_1, \pAtt_2)$ then
$\pRel_1 \bowtie \pRel_2$ joins tuples from $\pRel_1$ and $\pRel_2$ where
attribute $\pAtt_1$ has the same value in $\pRel_1$ and $\pRel_2$. 
\TODO{add cross product, add union intersection}

%, except
%that again we allow conditions to be variational.
%
%A \emph{choice} encodes a variation point between two subquery alternatives based on a
%given feature expression, e.g., \chc[f_1\wedge f_2]{\vQ_l,\vQ_r} yields
%the results of $\vQ_l$ alternative for configurations where $f_1$ and $f_2$ are enabled,
%and in other configurations yields the results of $\vQ_r$ alternative. Note that the
%conditions $\vCond$ used by selections and joins also contain choices, and
%these behave similarly.
%\input{sections/encodeVar}
\section{The Formula Choice Calculus}
\label{sec:fcc}


%To account for variation, VRA combines relational algebra (RA) with 
%\emph{choices}~\cite{EW11tosem,HW16fosd,Walk13thesis}.
%%\point{choice.}
%A choice $\chc{\elem_1,\elem_2}$ consists of a feature expression \dimMeta, called
%the \emph{dimension} of the choice, and 
%two \emph{alternatives} $\elem_1$ and $\elem_2$. For a given configuration \config, 
%the choice $\chc{\elem_1, \elem_2}$ can be replaced by $\elem_1$ if \dimMeta\
%evaluates to \t\ under configuration \config, (i.e., \fSem{\dimMeta}),
%or $\elem_2$ otherwise. 

The second approach we use to incorporate variation into queries is
the formula choice calculus~\cite{HW16fosd} which is an extension of 
the choice calculus~\cite{Walk13thesis,EW11tosem}. 
%
The choice calculus is a metalanguage for
describing variation in programs and its elements such as data 
structures~\cite{Walk14onward,EWC13fosd}.
In the choice calculus, variation is represented in-place as
choices between alternative subexpressions. For example, 
the variational expression 
$\mathit{expr} = \chc [\A] {1,2} + \chc [\B] {3,4} + \chc [\A] {5,6}$
 contains three choices.
Each choice has an associated \emph{dimension}, which is a boolean
variable equivalent to a feature and is used to
synchronize the choice with other choices in different parts
of the expression. For example, expression $\mathit{expr}$ contains
two dimensions, $\A$ and $\B$, and the two choices in dimension
$\A$ are synchronized. Therefore, the variational expression
$\mathit{expr}$ represents four different plain expressions, depending
on whether the left or right alternatives are selected from each
dimension. Assuming that dimensions may be set to boolean values
where \t\ indicates the left alternative and \f\ indicates the
right alternative, we have: 
%(1) $1+3+5$, when $A$ and $B$ are \t,
%(2) $1+4+5$, when $A$ is \t\ and $B$ is \f,
%(3) $2+3+6$, when $A$ is \f\ and $B$ is \t,
%and (4) $2+4+6$, when $A$ and $B$ are \f.
\begin{alignat*}{1}
\chc [\A] {1,2} + \chc [\B] {3,4} + \chc [\A] {5,6} &=
\begin{cases}
  1+3+5,& \A =\t, \B = \t\\
  1+4+5,& \A =\t, \B = \f\\
  2+3+6,& \A =\f, \B = \t\\
  2+4+6,& \A =\f, \B = \f
\end{cases}
\end{alignat*}
%
\noindent
The formula
choice calculus extends the choice calculus 
by allowing dimensions to be propositional formulas~\cite{HW16fosd}. For example,
the variational expression $\VVal {\mathit{expr}} = \chc [\A \vee \B] {1,2} + \chc [\B] {3,4} + \chc [\A] {5,6}$ represents
four plain expressions: 
%(1) $1$, when $\A \vee \B$ evaluates to \t\
%and (2) $2$, when $\A \vee \B$ evaluates to \f. More explicitly, we have:
\begin{alignat*}{1}
\chc [\A \vee \B] {1,2} + \chc [\B] {3,4} + \chc [\A] {5,6}&=
\begin{cases}
  1+3+5,& \A =\t, \B = \t\\
  1+4+5,& \A =\t, \B = \f\\
  1+3+6,& \A =\f, \B = \t\\
  2+4+6,& \A =\f, \B = \f
\end{cases}
\end{alignat*}



%\subsection{Variational Set}
\label{sec:vlist-vset}

%\point{vset.}
A \emph{variational set (v-set)} $\vset = \setDef {\annot [\dimMeta_1] {\elem_1},\ldots, \annot [\dimMeta_n] {\elem_n}}$ 
is a set of annotated elements~\cite{EWC13fosd,Walk14onward,ATW17dbpl}.
% where the presence condition of elements is satisfiable~\cite{EWC13fosd,Walk14onward,vdb17ATW}. 
%
Conceptually, a \emph{variational set} represents many different plain sets
that can be generated by enabling or disabling features
and including only the elements whose feature expressions evaluate to \t.
We typically omit the presence condition \prog{true} in a variational set,
e.g., the v-set 
$\setDef {\annot [\A] 2, \annot [\B] 3, 4}$
represents four plain sets under different configurations. These plain
sets can be generated by \emph{configuring} the variational set with a
given configuration: 
\setDef {2,3,4}, when $\A$ and $\B$
are enabled; \setDef {2,4}, when $\A$ is enabled but $\B$ is disabled;
\setDef {3,4}, when $\B$ is enabled but $\A$ is disabled;
and \setDef {4}, when both $\A$ and $\B$ are disabled.
%
%We indicate variational sets of elements $\elem \in \mathbf{\elemSet}$ with \elemSet.
%A variational set is conceptually a function from a configuration of its
%features to the corresponding plain set. 
%We typically omit the feature
%expression \prog{true} in a variational set, for example, in the
%variational set $\{5,6^{f_1}\}$, the feature expression for the value $5$ is
%implicitly \prog{true}, and so the element is included in both variants:
%$\{5,6\}$ when feature $f_1$ is enabled and $\{5\}$ when feature $f_1$ is
%disabled.
Note that elements with presence condition \prog{false} can be omitted
from the v-set, e.g., the v-set \ensuremath{\setDef {\annot [\f] {1}}} is 
equivalent to an empty v-set.
For simplicity and to comply with database notational conventions
we drop the brackets of a variational set when used in database
schema definitions and queries.
%for defining 
%variational relation schemas and the variational attribute set to be projected in a query.

%\point{annotated vset.}
A variational set itself can also be annotated with a feature expression.
%
%An \emph{annotated variational set} 
$\annot \vset = \setDef {\annot [\dimMeta_1] {\elem_1},\ldots,\annot [\dimMeta_n] {\elem_n}}^\dimMeta$ is an
\emph{annotated v-set}.
% that it is annotated itself by a \emph{feature expression} \dimMeta.
%We denote an annotated variational set of elements $\elem \in \mathbf{\elemSet}$ with
%\annot \elemSet.
Annotating a v-set with the feature expression \dimMeta\ 
restricts the condition under which its elements are present, i.e., it forces
elements' presence conditions to be more specific. This restriction 
can be applied to all elements of the set by \emph{pushing} in the
feature expression \dimMeta, done by the operation
%\NOTE{
\ensuremath{
\pushIn {\setDef {\annot [\dimMeta_1] {\elem_1},\ldots,\annot [\dimMeta_n] {\elem_n}}^\dimMeta}
= 
%\annot {\setDef{\annot [\dimMeta_i] \elem_i \myOR \sat {\dimMeta_i \wedge \dimMeta}, 1 \leq i \leq n}}}.}
\setDef {\annot [\dimMeta_1 \wedge \dimMeta] {\elem_1},\ldots, \annot [\dimMeta_n \wedge \dimMeta] {\elem_n}}
}.
%This restriction
%can be captured by the property:
%$\setDef {\annot [\dimMeta_1] {\elem_1} ,\ldots, \annot [\dimMeta_n] {\elem_n}}^\dimMeta
%\equiv 
%\setDef {\annot [\dimMeta_1 \wedge \dimMeta] {\elem_1},\ldots, \annot [\dimMeta_n \wedge \dimMeta] {\elem_n}}
%$.
%
For example, the annotated v-set
$\{\annot [\A] 2, \annot [\neg \B] 3, 4, \annot [\C] 5\}^{\A \wedge \B}$
indicates that all the elements of the set can only exist
when both $\A$ and $\B$ are enabled. Thus, pushing in the set's feature expression
results in
$\{\annot [\A \wedge \B] 2,\annot [\A \wedge \B] 4,\annot [\A \wedge \B \wedge \C] 5\}$. The element $3$ is dropped 
%from the set 
since 
\ensuremath{\neg \sat {\neg \B \wedge (\A \wedge \B)}},
where
\ensuremath{
\getPC 3 = \neg \B \wedge (\A \wedge \B)}.
%its presence condition is unsatisfiable, i.e., $\neg \sat {\neg \fName_2 \wedge (\fName_1 \wedge \fName_2)}$.
%%

We provide some operations over v-sets. Intuitively, these operations should 
behave such that configuring the result of applying a variational set operation
should be equivalent to applying the plain set operation on the configured 
input v-sets. 
 
%These operations are vastly used
%in \secref{type-sys}.

%
\begin{definition}[V-set union]
\label{def:vset-union}
The \emph {union} of two v-sets is the union of their elements with the disjunction of 
presence conditions if an element exists in both v-sets:
\ensuremath{
\vset_1 \cup \vset_2 = \setDef {\annot [\dimMeta_1] \elem \myOR \annot [\dimMeta_1] \elem \in \vset_1, \annot [\dimMeta_2] \elem \not \in \vset_2}
\cup \setDef {\annot [\dimMeta_2] \elem \myOR \annot [\dimMeta_2] \elem \in \vset_2, \annot [\dimMeta_1] \elem  \not \in \vset_1}
\cup \setDef {\annot [\dimMeta_1 \vee \dimMeta_2] \elem \myOR 
\annot [\dimMeta_1] \elem \in \vset_1, \annot [\dimMeta_2] \elem \in \vset_2}
}.
For example, \\
\ensuremath{
\setDef {2,\annot [\dimMeta_1] 3, \annot [\dimMeta_1] 4} \cup \setDef {\annot [\dimMeta_2] 3, \annot [\neg \dimMeta_1] 4} = \setDef {2, \annot [\dimMeta_1 \vee \dimMeta_2] 3, 4}
}.
\end{definition}

% 
% is needed for the implicitly-type lang:
\begin{definition}[V-set intersection]
\label{def:vset-intersect}
The \emph{intersection} of two v-sets is a v-set of their shared elements
annotated with the conjunction of their presence conditions, i.e., 
\ensuremath{
\vset_1 \cap \vset_2 = \setDef {
\annot [\dimMeta_1 \wedge \dimMeta_2 ]\elem \myOR
\annot [\dimMeta_1] \elem \in \vset_1, \annot [\dimMeta_2] \elem \in \vset_2,
\sat {\dimMeta_1 \wedge \dimMeta_2}
}
}.
For example, \ensuremath{
\setDef {2, \annot [\A] 3, \annot [\neg \B] 4} \cap
\pushIn {\annot [\B] {\setDef{2,3,4,5}}} =
\setDef{\annot [\B] 2, \annot [\A \wedge \B] 3}
}.
\end{definition}

\begin{definition} [V-set cross product]
\label{def:vset-cross}
The \emph{cross product} of two v-sets is a pair of every two elements of 
them annotated with the conjunction of their presence conditions.
\ensuremath{
\vset_1 \times \vset_2 = \setDef{
\annot [\dimMeta_1 \wedge \dimMeta_2] {(\elem_1, \elem_2)} \myOR
\annot [\dimMeta_1] \elem_1 \in \vset_1, \annot [\dimMeta_2] \elem_2 \in \vset_2
%\vset_1 \cap \vset_2 = \setDef \
}
}
%
\end{definition}

\begin{definition} [V-set equivalence]
\label{def:vset-eq}
Two v-sets are \emph{equivalent}, denoted by
\ensuremath{\vset_1 \equiv \vset_2}, iff
\ensuremath{
\forall \annot  \elem \in (\vset_1 \cup \vset_2). 
\annot [\dimMeta_1] \elem \in \vset_1, \annot [\dimMeta_2] \elem \in \vset_2, 
\dimMeta_1 \equiv \dimMeta_2},
i.e., they both cover the same set of elements and the presence conditions
of elements from the two v-sets are equivalent.
\end{definition}

%
\begin{definition} [V-set subsumption]
\label{def:vset-subsumption}
The v-set \ensuremath{\vset_1} \emph {subsumes} the v-set
\ensuremath{\vset_2}, $\subsume {\vset_2} {\vset_1}$, iff
\ensuremath{ \forall \annot [\dimMeta_2] \elem \in \vset_2.
\annot [\dimMeta_1] \elem \in \vset_1, 
%\neg \sat {\dimMeta_2 \wedge \neg \dimMeta_1}
\sat {\dimMeta_2 \wedge  \dimMeta_1}
},
i.e., all elements in $\vset_2$ also exist in $\vset_1$ 
s.t. the element is valid in a shared configuration between the v-sets.
For example, 
\ensuremath{
 \subsume {\pushIn {\annot [\A] {\setDef {2,3}}}} {\setDef {2, \annot [\A \vee \B] 3, 4}}},
however, 
\ensuremath{
 \nsubsume {\pushIn {\annot [\A] {\setDef {2,3}}}} {\setDef {2, \annot [\neg \A \wedge \B] 3}}}
and
\ensuremath{
\nsubsume {\setDef {\annot [\A] 2,\annot [\A] 3, 4}} {\setDef {2, \annot [\A \wedge \B] 3}}}.
\end{definition}

%\begin{definition} [V-set explicit subsumption]
%\dropit{drop this for vldb submission. remember you need it for popl.}
%\label{def:vset-strict-subsumption}
%The v-set \ensuremath{\vset_1} \emph {explicitly subsumes} the v-set
%\ensuremath{\vset_2}, $\subsumeExpl {\vset_2} {\vset_1}$, iff
%\ensuremath{ \forall \annot [\dimMeta_2] \elem \in \vset_2.
%\annot [\dimMeta_1] \elem \in \vset_1, 
%\neg \sat {\dimMeta_2 \wedge \neg \dimMeta_1}
%},
%i.e., all elements in $\vset_2$ also exist in $\vset_1$ 
%s.t. its presence condition in \ensuremath{\vset_2} is more specific than 
%its presence condition in \ensuremath{\vset_1}, captured by 
%\ensuremath{\neg \sat {\dimMeta_2 \wedge \neg \dimMeta_1}}
%which could also be defined as 
%\ensuremath{
%\nexists \config \in \confSet . \fSem {\dimMeta_1} = \t , \fSem {\dimMeta_2} = \f.
%%i.e. in set theory:
%% \dimMeta_2 \subset \dimMeta_1
%%\dimMeta_2 - \dimMeta_1 = \emptyset
%%i.e.
%%\dimMeta_2 \cap \bar{\dimMeta_1} = \emptyset 
%}
%\end{definition}





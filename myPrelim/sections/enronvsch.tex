\subsubsection{Generating V-Schema of the Email SPL VDB}
\label{sec:enron-vsch}

\begin{table}
\caption{Original Enron email dataset schema.}
\vspace{-8pt}
\label{tab:enron}
%\begin{center}
\small
\begin{tabular} {|l|}
%\hline
%\textbf{Enron Schema} \\
\hline 
\employees(\eid, \fname, \lname, \emailid, $\mathit{email2}$, \\
\hspace{40pt} $\mathit{email3}$, $\mathit{email4}$, \folder, \status) \\
\messages(\midatt, \sender, \dateatt, \messageid, \subject, \body, \folder)  \\ 
\recipientinfo(\rid, \midatt, \rtype, \rvalue)  \\
\referenceinfo(\rid, \midatt, \reference)  \\
\hline
\end{tabular}
%\end{center}
\vspace{-13pt}
\end{table}


To produce a v-schema for the email VDB, we start from plain schema
of the Enron email dataset shown in \tabref{enron}, then systematically adjust
its schema to align with the information needs of the email SPL described by
\citet{Hall05}.
%
The \employees\ table contains information about the employees of the company
including the employee identification number (\eid), their first name and last
name (\fname\ and \lname), their primary email address (\emailid), alternative
email addresses (e.g.\ $\mathit{email2}$), a path to the folder that contains
their data (\folder), and their last status in the company (\status).
%
The \messages\ table contains information about the email messages
 including
the message ID (\midatt), the sender of the message (\sender), the date
(\dateatt), the internal message ID (\messageid), the subject and body of the
message (\subject\ and \body), and the exact folder of the email (\folder).
% 
The \recipientinfo\ table contains information about the recipient of a message
including the recipient ID (\rid), the message ID (\midatt), the type of the
message (\rtype), and the email address of the recipient (\rvalue).
%
The \referenceinfo\ table contains messages that have been referenced in other
email messages,
for example, in a forwarded message; it contains a
reference-info ID (\rid), the message ID (\midatt), and the entire message
(\reference). 
This table simply backs up the emails.


\begin{table*}
\caption{V-schema of the email VDB with feature model
\ensuremath{\fModel_\enron}. 
Presence conditions are colored blue for clarity.
}
\vspace{-8pt}
\label{tab:enron-vsch}
% \begin{center}
\small
\begin{tabular} {|l|}
%\hline
%\textbf{Variational Schema for Email SPL} \\
\hline 
 % \annot [\vFour] \name
$\employees(\eid, \fname, \lname, \emailid, \folder, \status,$ 
%\\ \hspace{40pt} 
$\annot[\fsignature]{\verificationkey},
  \annot[\fencryption]{\publickey})$ \hspace{20pt}
 $\recipientinfo(\rid, \midatt, \rtype, \rvalue)$ 
  \\ %[1.1ex]
$\messages(\midatt, \sender, \dateatt, \messageid, \subject, \body, \folder,$ 
%\\ \hspace{30pt} 
$\issystemnotification,
  \annot[\fencryption]{\isencrypted}, 
  \annot[\fautoresponder]{\isautoresponse},\annot[\fsignature]{\issigned} $ 
  \\ \hspace{30pt} 
  $,
  \annot[\fforwardmsg]{\isforwardmsg})$  \hspace{40pt}
%   \\ %[1.1ex]
%$\recipientinfo(\rid, \midatt, \rtype, \rvalue)$   
${\forwardmsg(\eid, \forwardaddr)}^{\fforwardmsg} $\hspace{40pt}
${\mailhost(\eid, \username, \mailhost)}^{\fmailhost}$ 
\\
%\hspace{20pt}
% \\ %[1.1ex]
%\referenceinfo(\rid, \midatt, \reference)  \\
  ${\filtermsg(\eid, \suffix)}^{\ffiltermsg} $ \hspace{3pt}
%\\ %[1.1ex]
%  \hspace{47pt}
%\\ %[1.1ex]
${\remailmsg(\eid, \pseudonym)}^{\fremailmsg}$    \hspace{3pt}
%\hspace{20pt}
%\\ %[1.1ex]
${\automsg(\eid, \subject, \body)}^{\fautoresponder} $  \hspace{3pt}
%\\ %[1.1ex]
${\alias(\eid, \emailAtt, \nickname)}^{\faddressbook}$\\
%\\ %[1.1ex]
%${\filtermsg(\eid, \suffix)}^{\ffiltermsg} $\\ %[1.1ex]
\hline
\end{tabular}
\vspace{-13pt}
% \end{center}
\end{table*}


From this starting point, we introduce new attributes and relations that are
needed to implement the features in the email SPL. We attach presence
conditions to new attributes and relations corresponding to the features
they are needed to support, which ensure they will \emph{not} be present
in configurations that do not include the relevant features.
%
The resulting v-schema is given in \tabref{enron-vsch}.


%As an example of this process, 
For example, consider the \signature\ feature. In the
software, implementing this feature requires new operations for signing an
email before sending it out and for verifying the signature of a received
email. These new operations suggest new information needs: we need a way to
indicate that a message has been signed, and we need access to each user's
public key to verify those signatures (private keys used to sign a message
would not be stored in the database). These  needs are reflected in
the v-schema by the new attributes \verificationkey\ and \issigned,
added to the relations \employees\ and \messages, respectively. The new
attributes are annotated by the \signature\ presence condition, indicating that
they correspond to the \signature\ feature and are unused in configurations
that exclude this feature.
%
Additionally, several features require adding entirely new relations, e.g.,
%
%For example, 
when the \forwardmsg\ feature is enabled, the system must keep
track of which users have forwarding enabled and the address to forward the messages to.
%should be forwarded to. 
This need is reflected by the new
\forwardmsg\ relation, which is correspondingly annotated by the \forwardmsg\
presence condition.


A main focus of Hall's decomposition~\cite{Hall05} is on the many feature
interactions.
% in the email SPL. 
Several of the features may interact in
undesirable ways if special precautions are not taken. For example, any
combination of the \forwardmsg, \remailmsg, and \autoresponder\ features can
trigger an infinite messaging loop if users configure the features in the wrong
way; preventing this creates an information need to identify auto-generated
emails, which is realized in the variational schema by attributes like
\isforwardmsg\ and \isautoresponse.
%
% occurs between the \signature\ and \fremailmsg\ features: the \fremailmsg\
% feature enables anonymously sending messages by replacing the sender with a
% pseudonym, but this prevents the recipient from being able to verify a signed
% email.
%
% occurs between the \signature\ and \forwardmsg\ features: if Sarah signs a
% message and sends it to Ina, and Ina forwards the message to Philippe, then
% the signature verification operation may incorrectly interpret Ina as the
% sender rather than Sarah and fail to verify the message.

%p: moved the following paragraph to discussion.
%For each feature, we (1) enumerated the operations that must be supported both
%to implement the feature itself and to resolve undesirable feature
%interactions, (2) identified the information needs to implement these
%operations, and (3) extended the variational schema to satisfy these
%information needs.
%
% The changes made to accommodate the features \addressbook, \encryption,
% \autoresponder, \remailmessage, \filtermessages, and \mailhost\ are similar. 
%

% are provided in the
%second author's MS project report~\citet{Li19}.
%
For brevity, we omit some attributes and relations from the original schema
that are  irrelevant to the email SPL as described by Hall, such as the
\referenceinfo\ relation and alternative email addresses. 
%
%p: will ref qiaoran's work somewhere else, if space permits.
%\citet{Li19} provides
%a complete description of adaptations done in the process in her MS project report.
% \t presence condition can be omitted.


%We distribute the variational schema for the email VDB in two formats.
%%
%First, we provide the schema in the encoding used by our prototype VDBMS tool.%
%\footnote{\href{https://github.com/lambda-land/VDBMS/blob/master/usecases/space-emailSPL/schema/EnronSchema.hs}{usecases/space-emailSPL/schema}}
%
%Second, 
In addition to providing the schema in the encoding used by our 
prototype VDBMS tool, we also provide a direct encoding in SQL
%The SQL encoding 
which generates the
%\eric{Eric, I don't like universal here! it may give the wrong idea!
%maybe just say v-schema?}
universal schema for the VDB.
% in either MySQL or Postgres.%
%\footnote{\href{https://github.com/lambda-land/VDBMS/tree/master/usecases/space-emailSPL/database/create}{usecases/space-emailSPL/database/create}}
%
Variation is encoded as an additional relation of the form \vdbpc\ that
captures all of the relevant presence conditions: that of the 
v-schema itself (i.e.\ the feature model), and those of each relation and
attribute.%
%\footnote{\href{https://github.com/lambda-land/VDBMS/tree/master/usecases/space-emailSPL/database/withSchema}{usecases/space-emailSPL/database/withSchema}}
%
The $\mathit{element\_id}$ of the feature model is
$\mathit{variational\_schema}$; the $\mathit{element\_id}$ of a relation \vRel\
is its name \vRel, and of attribute \vAtt\ in relation \vRel\ is $\vRel.\vAtt$.
%
The plain SQL encoding of the v-schema supports the use
of the case studies for research on the effective management of variation in
databases independent of VDBMS.

%feature
%model of the variational schema, a relation \tablespc\ that stores the presence
%condition of each relation, and for each table $R\left(a_1, \cdots, a_n\right)$
%in the variational schema there is a table $t\left(\mathit{attribute\_name},
%\mathit{pres\_cond}\right)$ in the database that stores its attributes'
%presence conditions.

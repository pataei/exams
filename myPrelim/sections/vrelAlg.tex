\subsubsection{Variational Relational Algebra}
\label{sec:vrel-alg}

\TODO{adjust for proposal}

%\point{vra = cc + ra}
%Considering the variational nature of a VDB, to satisfy a user's information 
%need when extracting information, 
%we need a query language that not only considers the structure of 
%relational databases (such as SQL and relational algebra (RA)) but also 
%accounts for the variation encoded in the VDB. We achieve this by:
%1) picking relational algebra as our main query language and
%2) using \emph{choices}~\cite{Walk13thesis, EW11tosem} 
%and presence conditions to account for variation. 

To account for variation, VRA combines relational algebra (RA) with 
\emph{choices}~\cite{Walk13thesis, EW11tosem}.
%\point{choice.}
A choice $\chc{x,y}$ consists of a feature expression \dimMeta\ and 
two alternatives $x$ and $y$. For a given configuration \config, 
the choice $\chc{x, y}$ can be replaced by $x$ if \dimMeta\
evaluates to \t\ under configuration \config, (i.e., \fSem{\dimMeta}),
or $y$ otherwise. 
In essence, choices allow a v-queries
to encode variation in a structured and systematic manner. 

\begin{figure}
\begin{syntax}

\multicolumn{4}{l}{\textbf{Operators:}} \\[1ex]
\bullet
  &\eqq& \multicolumn{2}{l}{< \myOR \leq \myOR = \myOR \neq \myOR > \myOR \geq} \\
\circ
  &\eqq& \cup \myOR \cap \\[2ex]

\multicolumn{4}{l}{\textbf{Variational conditions:}} \\[1ex]
\vCond\in\vCondSet
  &\eqq&  \multicolumn{2}{l}{
          \bTag
   \myOR  \pAtt \bullet \cte
   \myOR  \pAtt \bullet \pAtt
   \myOR  \neg \vCond
   \myOR  \vCond \vee \vCond} \\
  &\myOR& \multicolumn{2}{l}{
          \vCond \wedge \vCond
   \myOR \chc{\vCond,\vCond}} \\[2ex]

\multicolumn{4}{l}{\textbf{Variational queries:}} \\[1ex]
\vQ\in\qSet
  &\eqq&  \vRel     & \textit{Relation}\\
  &\myOR& \vSel \vQ & \textit{Selection}\\
  &\myOR& \vPrj[\vAttList]{\vQ} & \textit{Projection}\\
  &\myOR& \chc{\vQ,\vQ} & \textit{Choice}\\
% &\myOR& \vQ \Join_\vCond \vQ & \textit{Variational Join}\\
  &\myOR& \vQ \times \vQ & \textit{Cartesian Product}\\
  &\myOR& \vQ \circ \vQ  & \textit{Set Operation}\\
% &\myOR& \vQ \backslash \vQ &\textit{Variational Set Difference}\\
  &\myOR& \empRel & \textit{Empty Relation}

\end{syntax}

\caption[Syntax of variational relational algebra]{Syntax of variational relational algebra.}
%\TODO{remember that
%you removed join (also removed it from query config def and constrain query 
%by schema). if you want use it just say it's a syntactic sugar.}
%$<, \leq, =, \neq, >, \geq$.
%$\circ$ denotes set operators: union and difference.}
\label{fig:v-alg-def}
\end{figure}



%\point{explain notation and VRA operations.}
The syntax of VRA is given in \figref{v-alg-def}.
%
In VRA, the selection operation is similar to standard RA selection except
that the condition parameter is \emph{variational} meaning that it may contain
choices.
For example, the query 
\ensuremath{\sigma_{\chc {\vAtt_1=\vAtt_2,\vAtt_1=\vAtt_3}} \vRel}
selects a v-tuple \vTuple\ if it satisfies
the condition \ensuremath{\vAtt_1 = \vAtt_2} 
and  \ensuremath{\sat {\dimMeta \wedge \getPC \vTuple}}
or
if \ensuremath{\vAtt_1 = \vAtt_3} 
and \ensuremath{\sat {\neg \dimMeta \wedge \getPC \vTuple }}.
%
The projection operation is parameterized by a v-set of attributes. For
example,
the query $\pi_{\vAtt_1, \optAtt [\dimMeta] [\vAtt_2]} \vRel$
projects $\vAtt_1$ from relation \vRel\ unconditionally, and $\vAtt_2$ 
when \sat{\dimMeta}.
%
The choice operation enables combining two v-queries to be used in different
variants based on a given feature expression. In practice,
it is often useful to return information in some variants and nothing at all in
others. We introduce an explicit \emph{empty} query \empRel\ to facilitate
this. The empty query is used, for example, in 
\ensuremath{\vQ_2} in \exref{vq-specific}. 
%The set operations between queries are v-set operations defined in \secref{vset}.
The rest of VRA's operations are similar to RA, where all set operations
(union, intersection, and cross product) are changed to the corresponding
variational set operations defined in \secref{vlist-vset}.
%
In examples, we also use a join operation with a variational condition,
$\vQ_1\bowtie_\vCond\vQ_2$, which is syntactic sugar for
$\sigma_\vCond(\vQ_1\times\vQ_2)$.


Our implementation of VRA also provides mechanisms for renaming queries and
qualifying attributes with relation/subquery names. These features are needed
to support self joins and projecting attributes with the same name in different
relations. However, for simplicity, we omit these features from the formal
definition in this paper.


%A query can simply 
%refer to a relation, filter tuples based on a variational condition 
%(which is a relational condition with choices of two conditions), and
%project a variational list of attributes. Besides production of two queries and
%set operations, VRA allows for a choice of two v-queries. This demands an
%\emph{empty} query since an alternative of a choice can very well inquire 
%no information at all. 
%For example, the query $\chc {\vQ_1, \}$
%

%\subsubsection{Running a V-Query on a VDB Results in a V-Table}
%\label{sec:run-vq-get-vtab}
%A v-query systematically represents a set of relational query variants associated to their
%corresponding database variants. Hence, intuitively the user expects to 
%get such variation in their result as well. 

The result of a v-query is a v-table with the relation name $\mathit{result}$.
%
For example, assume that v-tuples $\annot[\fOne]{(1,2)}$ and $\annot[\neg
f_3]{(3,4)}$ belong to a v-relation $\vRel(\vAtt_1,\vAtt_2)$, which is the only
relation in a VDB with the trivial feature model \t.
%
The query $\chc[f_3]{\pi_{\optAtt[f_2][\vAtt_1]}\vRel,\empRel}$ returns a
v-table with relation schema $\annot[f_3]{\mathit{result}(\annot[f_2]{a_1})}$,
which indicates that the result is only non-empty when $f_3$ is true and that the
result includes attribute $a_1$ when $f_2$ is true. \secref{type-sys} defines a
type system that yields the relation schema for any well-formed query.
%
The content of the result relation is a single v-tuple $\annot[f_1]{(1)}$. The
tuple $\annot[\neg f_3]{(3)}$ is not included since the projection occurs in
the context of the choice in $f_3$, which is incompatible with the presence
condition of the tuple (i.e., $\unsat{f_3 \wedge\neg f_3}$). This illustrates
how choices can effectively filter the tuples in a VDB based on their presence
conditions.

% Although there is no need to update the presence condition of the returned
% tuples, yet choices can filter the returned v-tuples.

% Note that here the value \ensuremath{1}
% of attribute \ensuremath{\vAtt_1} is present in VDB variants where 
% \ensuremath{\sat {\A \wedge \B \wedge \C}} although the presence 
% condition of the returned v-tuple does not have to state this condition
% since 
%
% overall presence 
%condition and the presence conditions of attributes and tuples are
%restricted by the variation enforced by the query.
%
%Note that the presence condition of tuples, attributes, and the return relation
%is restricted by the variation enforced by the query. 
% correct this so that you don't conjunct the pc of relation and clarify that it's relations's pc and not the attributes. although the conjunction should be satisfiable.}
%




%%Hence, VRA is more expressive than RA 
%%because it can encode variational queries.
%The variational nature allows users to write interesting queries in many ways:
%1) to express their variational information need or to filter returned tuples
%they can use annotations or 
%choices, \exref{vq-specific},
%2) to express the same intent over several database variants they can 
%use choices in queries or conditions, \exref{vq-same-intent-mult-vars},
%and 
%3) they can also use choices to express different intents over database variants.
%\TODO{Eric, should we drop the last since it creates messy results and isn't really useful?}.
%%The expressiveness of VRA satisfies \textbf{N1}, this is illustrated in 
%%\exref{vq-specific} and \exref{vq-same-intent-mult-vars}.
%%Interestingly, VRA's expressiveness enables users to express 
%%their information need more specifically by stating the exact condition
%%under which an information need is inquired. \exref{vq-specific} illustrates this.
%%It also allows users to express the same intent over several database 
%%variants

% \NOTE{
% To express the variational information need or to filter returned tuples
% users can use annotations or choices. \exref{vq-specific} illustrates this.
% }

The following example illustrates, in the context of our running example, how
a v-query can be used to express variational information needs.

\begin{example}
\label{eg:vq-specific}
%VRA's expressiveness consequently facilitates expressing exactly the condition
%under which an information need is inquired. 
Assume a VDB with
features \vThree, \vFour, and \vFive, and
the corresponding \empbio\ schema variants in \tabref{mot}. 
The v-schema for this VDB is:
%
\begin{align*}
\vSch_2 &=
\{\empbio (\empno, \sex, \birthdate,
\optAtt [\vFour] [\name], \optAtt [\vFive] [\fname],\\
& \hspace{16pt} \optAtt [\vFive] [\lname] )\}^{\fModel_2}\\
& \hspace{4pt} \textit{where } \fModel_2 = {\vThree \oplus \vFour \oplus \vFive}.
%\left(\vThree \wedge \neg \vFour \wedge \neg \vFive\right)
%  \vee \left(\vFour \wedge \neg \vThree \wedge \neg \vFive\right) 
%   \vee \left(\vFive \wedge \neg \vThree \wedge \neg \vFour\right)}.
\end{align*}
%
The user wants the employee ID numbers (\empno) and names for variants 
\vFour\ and \vFive.
The user needs to project the \name\ attribute 
for variant \vFour, the \fname\ and \lname\ attributes for variant \vFive,
and \empno\ attribute for both variants.
This can be expressed with the following v-query.
\[
\vQ_1 = \pi_{\optAtt [\vFour \vee \vFive] [\empno], \name, \fname, \lname} \empbio
\]
\end{example}

Note that the user does not need to repeat the variability information encoded
in the v-schema in their query, that is, they do not need to annotate \name,
\fname, and \lname\ with \vFour, \vFive, and \vFive, respectively. We discuss
this in more detail in \secref{type-sys} and \secref{constrain}. $\vQ_1$
queries all three variants simultaneously although the returned results are
only associated with variants \vFour\ and \vFive\ due to the annotation of the
attribute \empno\ in the query and the presence conditions of the rest of the
projected attributes in the schema.
%
Yet, selecting only two out of the three variants can be written more
explicitly in a query by using a choice:
$\vQ_2=\chc[\vFour\vee\vFive]{\pi_{\empno,\name,\fname,\lname}\empbio,\empRel}$. 
%
Note that queries $\vQ_1$ and $\vQ_2$ return the same set of v-tuples since
neither returns tuples associated with variant \vThree, but their returned
v-tables have different presence conditions, as will be discussed later in
\exref{type}. 
%

\NOTE{
VRA has semantic-preserving equivalence rules that allow users 
to incorporate their taste and preference of where and how
 they want to encode variation in their queries. These rules
 factor out the commonality of subqueries and generate queries
 with less variation. \secref{var-min} expands on these rules.
}

The next example illustrates how a v-query can be used to express the same
intent over several database variants using choices and conditions. Expressing
the same intent over several instances by a single query relieves the DBA from
maintaining separate queries for different versions or configurations of the
schema.

\begin{example}
\label{eg:vq-same-intent-mult-vars}
Assume a VDB with features \vOne--\vFive\ and the corresponding \basic\ schema
variants in \tabref{mot}. The user wants to get all employee names across all
variants, which they can express by the following v-query.
%
\begin{align*}
\vQ_3 &= 
  \vOne\chcL
    \pi_{\name}\engemp \cup \pi_{\name}\othemp, \\
 & \hspace{18pt}
    (\vTwo\vee\vThree)\chcL
      \pi_{\name}\empacct, \\
 & \hspace{24pt}
      \chc[(\vFour\vee\vFive)]{\pi_{\name,\fname,\lname}\empbio, \emp}\chcR\chcR
\end{align*}
%
Since the v-schema enforces that exactly one of \vOne--\vFive\ be enabled, we
can simplify the query by omitting the final choice.
%
\begin{align*}
\vQ_4 &= 
  \vOne\chcL
    \pi_{\name}\engemp \cup \pi_{\name}\othemp, \\
 & \hspace{18pt}
    \chc[(\vTwo\vee\vThree)]{
      \pi_{\name}\empacct,
      \pi_{\name,\fname,\lname}\empbio}
\end{align*}
%
\end{example}

In principle, v-queries can also express arbitrarily different intents over
different database variants. However, we expect that v-queries are best used to
capture single (or at least related) intents that vary in their realization
since this is easier to understand and increases the potential for sharing in
both the representation and execution of a v-query.


% \subsubsection{VRA Semantics}
% \label{sec:vra-sem}

\begin{figure}
%\textbf{Configuration selection semantics of \vqsTxt:}
\begin{alignat*}{1}
\eeSem [] . &: \qSet \to \confSet \to \pQSet\\
%
\eeSem \vRel &= \orSem \vRel = \pRel\\
\eeSem {\vSel \vQ}  &= \vSel [\ecSem \vCond] {\eeSem \vQ}\\
%
\eeSem {\vPrj [\vAttList] \vQ} &= \vPrj [\olSem \vAttList] {\eeSem \vQ}\\
%
\eeSem {{\vQ_1} \times {\vQ_2}} &= \eeSem {\vQ_1} \times \eeSem {\vQ_2}\\
%
%\eeSem {{\vQ_1} \Join_\vCond {\vQ_2}} &= \eeSem {\vQ_1} \Join_{\ecSem \vCond} \eeSem {\vQ_2}\\
%
\eeSem {\chc {\vQ_1, \vQ_2}} &= 
	\begin{cases}
		\eeSem {\vQ_1}, \text{ if } \fSem \dimMeta = \t\\
		\eeSem {\vQ_2}, \text{ otherwise}
	\end{cases}\\
%
\eeSem {{\vQ_1} \circ {\vQ_2}} &= \eeSem {\vQ_1} \circ \eeSem {\vQ_2}\\
%
\eeSem {\empRel} &= \underline {\empRel}
\end{alignat*}
\caption{Configuration of VRA which assumes that the given v-query
is well-typed. 
%\orSem ., \ecSem ., and \olSem . are
%configuration of v-relation, v-condition, and variational attribute
%set, respectively, defined in \figref{vdb-conf}, 
%\figref{vcond-conf-sem}, \figref{vdb-conf}.
Note that we have extended RA with an empty relation $\underline {\empRel}$.}
\label{fig:v-alg-conf-sem}
\end{figure}


The semantics of VRA can be understood as a combination of the
\emph{configuration semantics} of VRA, defined in \figref{v-alg-conf-sem}, the
configuration semantics of VDBs, defined in \figref{vdb-conf}, and the
semantics of plain RA.
%
%\TODO{Make the following a more precise description of how these three
%semantics work together, i.e.\ for every valid configuration of the feature
%model, we can configure the v-query and VDB in the same way to yield a plain RA
%query that is then executed over the corresponding plain RDB.}
%
Thus, the v-query
semantics is the set of semantics of its configured relational queries over
their corresponding configured relational database variant for every valid
configuration of the feature model of the VDB.
%
The configuration function maps a v-query under
a given configuration
to a pure relational query, defined in \figref{v-alg-conf-sem}.
%Configuring a v-query
%for all valid configurations, accessible from VDB's feature model,
%provides the complete meaning of a v-query in terms of RA semantics.
%
Users can deploy queries for a specific variant by configuring 
them,
%The configuration of a query allows users to deploy queries for a
%specific variant when they desire, 
satisfying query part of \nThree\ need stated in \secref{mot}.

%To define VRA semantics we map 
%a v-query to a pure relational query to re-use RA's semantics.
%However, to avoid losing the variation encoded 
%in the v-query, 
%we need to determine the variant under which such a
%mapping is valid. Thus, we introduce the semantic functions that 
%relate a v-query to a relational query.

%
%\textbf{Configuring a v-query:} 
%It maps a v-query under a 
%given configuration to a relational query, denoted by \eeSem . 
%and defined in \figref{v-alg-conf-sem}. Configuring a v-query
%for all valid configurations, accessible from VDB's feature model,
%provides the complete meaning of a v-query in terms of RA semantics.
%Users can deploy queries for a specific variant by configuring 
%them,
%%The configuration of a query allows users to deploy queries for a
%%specific variant when they desire, 
%satisfying query part of \nThree.

\begin{example}
\label{eg:conf-vq}
Assume the underlying VDB has the v-schema
% \t\ feature model and the v-relation
\ensuremath{
\vSch_3 = \{ \vRel \left( \optAtt [\fOne] [\vAtt_1], \vAtt_2, \vAtt_3 \right)^{\fOne \vee \fTwo}
\}} 
and only two features \fOne\ and \fTwo.
The v-query 
\ensuremath{
\vQ_5 = \vPrj [{\vAtt_1, \optAtt [\fOne \wedge \fTwo] [\vAtt_2], \optAtt [\fTwo] [\vAtt_3]}] \vRel
}
is configured to the following relational queries:
\ensuremath{\eeSem [\setDef \fOne] {\vQ_5} = \eeSem [\setDef \ ] {\vQ_5} = \pi_{\pAtt_1} \pRel},
\ensuremath{\eeSem [\setDef \fTwo] {\vQ_5} =
 \pi_{\pAtt_1, \pAtt_3} \pRel},
\ensuremath{\eeSem [\setDef {\fOne, \fTwo}] {\vQ_5} = \pi_{\pAtt_1, \pAtt_2, \pAtt_3} \pRel}.
\end{example}




%\textbf{Expressiveness of VRA:} \point{VRA is more expressive than RA since it
%can express all queries in RA and a couple of them all together in one query.}

VRA enables querying multiple database variants encoded as a singled VDB
simultaneously and selectively, satisfying the query need state in \secref{mot}
(\textbf{N1}).
%
More precisely, VRA is \emph{maximally expressive} in the sense that it can
express any set of plain RA queries over any subset of relational database
variants encoded as a VDB. This claim is captured by the following theorem.

\begin{theorem}
\label{thm:max-expr}
Given a set of plain RA queries $\pQ_1,\ldots,\pQ_n$ where each query $\pQ_i$
is to be executed over a disjoint subset $\vdbInst_i$ of variants of the VDB
instance \vdbInst, there exists a v-query $q$ such that
$\forall c.\; \odbSem[c]{\vdbInst} = \vdbInst_i \implies \eeSem[c]{q} = \pQ_i$.
\end{theorem}

\begin{proof}
By construction. Let $f_i$ be the feature expression that uniquely
characterizes the variants in each $\vdbInst_i$.
Then $q =$ \\
\(
\chc[(f_1\wedge\neg f_2\wedge\ldots\wedge\neg f_n)]{\pQ_1,
  \chc[(f_2\wedge\ldots\wedge\neg f_n)]{\pQ_2,\ldots
    \chc[f_n]{\pQ_n,\emp}\ldots}}.
\)\\
\end{proof}

\noindent
%
The above construction relies on the fact that every RA query is a valid VRA
(sub)query in which every presence condition is true.
%
Of course, in most realistic scenarios, we expect that v-queries can be encoded
more efficiently by sharing commonalities and embedding relevant choices and
presence conditions within the v-query.


%\textbf{Grouping a v-query:} 
%maps a v-query to a set of
%relational queries annotated with feature expressions, denoted by \qGroup .
%and defined in \figref{vq-group}. The presence condition of relational queries 
%indicate the group of configurations where the mapping holds. In essence, 
%grouping of v-query \vQ\ groups together all configurations with the same relational
%query produced from configuring \vQ. 
%Hence, the generated set
%%\dropit{could drop this if it's confusing!}
%of relational queries from grouping a v-query contains distinct (unique) queries.
%For example, consider the query \ensuremath {\vQ_5} in \exref{conf-vq}.
%Grouping \ensuremath{\vQ_5} results in the set:
%\ensuremath{
%\setDef{
%\left( \pi_{\pAtt_1, \pAtt_2, \pAtt_3} \pRel \right)^{\fOne \wedge \fTwo},
%\left(\pi_{\pAtt_1, \pAtt_3} \pRel \right)^{\neg \fOne \wedge \fTwo},
%\left(  \pi_{\pAtt_1} \pRel \right)^{( \fOne \wedge \neg \fTwo) \vee (\neg \fOne \wedge \neg \fTwo)}
%}
%}.
%
%
%
%\begin{figure}
%\textbf{Configuration selection semantics of \vqsTxt:}
\begin{alignat*}{1}
\qGroup . &: \qSet \totype \settype {\bm{(} \vartype \pQSet \bm{)}}\\
%
\qGroup \vRel &= \setDef {\annot [\t] \pRel}\\
\qGroup {\vSel \vQ}  &=  
\setDef {\annot [\dimMeta \wedge \dimMeta_\vCond] {\left(\sigma_{\pCond} \pQ\right)} \myOR
\annot \pQ \in \qGroup \vQ, \annot [\dimMeta_\vCond] \pCond \in \cGroup}
\\
%
\qGroup {\vPrj [\vAttList] \vQ} &= 
\setDef {\annot [\dimMeta \wedge \dimMeta_\vAttList] {\left(\pi_{\pAttList} \pQ \right)} \myOR
\annot \pQ \in \qGroup \vQ, \annot [\dimMeta_\vAttList] \pAttList \in \aGroup}
\\
%
\qGroup {{\vQ_1} \times {\vQ_2}} &= 
\setDef {\annot [\dimMeta_1 \wedge \dimMeta_2] {\left(\pQ_1 \times \pQ_2\right)} \myOR
\annot [\dimMeta_1] \pQ_1 \in \qGroup {\vQ_1}, \annot [\dimMeta_2] \pQ_2 \in \qGroup {\vQ_2} }
\\
%
\qGroup {{\vQ_1} \Join_\vCond {\vQ_2}} &= 
\setDef {\annot [\dimMeta_1 \wedge \dimMeta_2 \wedge \dimMeta_\vCond] {\left(\pQ_1 \Join_{\pCond} \pQ_2 \right)} \myOR 
\annot [\dimMeta_1] \pQ_1 \in \qGroup {\vQ_1}, \annot [\dimMeta_2] \pQ_2 \in \qGroup {\vQ_2}
%& \hspace{104pt}
,\annot [\dimMeta_\vCond] \pCond \in \cGroup  }
\\
%
\qGroup {\chc {\vQ_1, \vQ_2}} &= 
\setDef {\annot [\dimMeta \wedge \dimMeta_1] \pQ_1 \myOR  \annot [\dimMeta_1] \pQ_1 \in \qGroup {\vQ_1} }
\cup 
\setDef {\annot [\neg \dimMeta \wedge \dimMeta_2] \pQ_2 \myOR  \annot [\dimMeta_2] \pQ_2 \in \qGroup {\vQ_2}}  \\
%
\qGroup {{\vQ_1} \circ {\vQ_2}} &= 
\setDef {\annot [\dimMeta_1 \wedge \dimMeta_2] {\left(\pQ_1 \circ \pQ_2\right)} \myOR
\annot [\dimMeta_1] \pQ_1 \in \qGroup {\vQ_1}, \annot [\dimMeta_2] \pQ_2 \in \qGroup {\vQ_2} }\\
%
\qGroup {\empRel} &= \annot [\t] { \empRel}
\end{alignat*}
\caption[Unique configuration of variational queries]{Unique configuration of variational queries. 
The unique configuration function assumes that the input is well-typed.
}
\label{fig:vq-group}
\end{figure}




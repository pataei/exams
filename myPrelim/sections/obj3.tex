\subsection{Demonstrate how the proposed system manages
variation in databases in different application domains and how effective and efficient it is}
\label{sec:ro3}


Having a variational database framework, we need to examine how effectively
it represents instances of variation in databases in different application domains.
Objective 3 focuses on this goal and \tabref{ro3} represents individual research 
question we need to answer for this objective.

\begin{table}[H]
\caption{Objective 3 research questions.}
\label{tab:ro3}
\centering
\begin{tabularx}{\textwidth}{X}
\toprule
 \textbf{Objective 3: Demonstrate how the proposed system can be used to manage
variation in databases in different application domains}
\tabularnewline
\midrule
RQ3.1: Can real-world instances of variation in databases be encoded as a VDB
and can their variational information need be queried by VRA?
What are the steps to generate a VDB from a scenario of variation in a database? (\vamos)
\tabularnewline[0.2cm]
RQ3.2: What are the benefits and drawbacks of representing variation generically
in databases as opposed to having scenario-tailored approaches? What are 
possible improvements in the future?(\vamos)
\tabularnewline[0.2cm]
RQ3.3: How effective and efficient is VDBMS? (\vldb)
\tabularnewline
\bottomrule
\end{tabularx}
\end{table}

For RQ3.1 we develop the two case studies discussed in \secref{ro1} as
variational databases with a set of variational queries that capture
a variety of their variational information needs. \secref{db} provides
the systematic approach we took to generate these databases and
\secref{q} provides their set of queries. 

\secref{db} and \secref{q} illustrate steps of explicitly encoding
variation in the database and queries. The first step 
is recognizing features involved in the variational scenario. 
A feature is a trait that is essential for convey some information
need. 
%
Thus, for each feature, one should (1) enumerate the operations that must be supported both
to implement the feature itself and to resolve undesirable feature
conflicts, (2) identified the information needs to implement these
operations, and (3) map features to database elements accordingly.
%extended the variational schema to satisfy these
%information needs.
%
This process outlines generating the v-schema for a variational
scenario as show cased for our case studies in 
\secref{emp-vsch} and \secref{enron-vsch}.
%
Note that we generated our VDBs from scratch, that is,
we did not have any database variants and their corresponding
configuration. It is easy to generate a VDB from a set of 
database variants with their corresponding configuration.

%\TODO{rewrite the discussion}
Generating the case studies provides us with an insight for RQ3.2.
%
%They helped us understand the flow of
%data and the query through VDBMS better and allowed us to refactor
%our data types to achieve a more useable and understandable encoding. 
%%
%We are using the introduced case studies to debug and evaluate the
%performance of different approaches implemented in VDBMS 
%for running a variational query on a VDB. Profiling the result
%of said evaluation will help us recognize where we can optimize
%our system. 
%
%Additionally, 
We expected our language to be more useable, readable, and
easier to understand. However, we realize that there is a buy-in cost
for programmers to pay when they first learn the language. Furthermore,
the variation in queries may get complicated, yet the programmer can 
still express their information need the way they are most comfortable
since variability flows in different parts of a query as shown in \secref{emp-qs}, 
that is, the overall query
can be wrapped in a choice, attributes can be projected variationally and
conditions can be encoded variationally. 
Still, we believe that the expressiveness of our
language allows for type checking queries (with variation encoded in them)
even though it adds more complexity 
to the language.

%\textbf{How can other researchers benefit from the introduced case studies?}
%
%
The case studies provide an accessible example of how one can encode different 
kinds of variability in a database using a VDB. 
The capability to encode variation directly in databases allows one to have a 
starting point for attacking a new kind of variability appearing in databases,
for example, one could have used the employee case study to attack database 
evolution without developing an entirely new system. 
%For example, 
We suspect that VDBs can be used to also model database 
versioning~\cite{datasetVersioning,dbVersioning} and
data provenance~\cite{bt07sigmod}. Such applications can be pursued in future.
%
%Additionally, the distribution of our databases and queries in a generic
%manner allows others to parse them easily to fit their research.
 %
Furthermore, our encoding can be used to emulate the encoding of 
 variation within other sorts of data such as spreadsheets.
 
 
%\textbf{What are the lessons learned while generating these case studies?}
%do they illustrate common challenges/problems when handling variational data?
%
%Beyond the use of these studies for development of our system,
%%The case studies 
%they illustrate that the most critical step of
%generating a VDB is recognizing the features and configurations
%in order to correctly encode variability of each variant.
%
Encoding variation explicitly in both the database and the query
provides developers with tracing variation and connecting
the variation in data to variation in software development.
For example,
the ability to check properties over data as mentioned in
\secref{vdb-props} is really useful for debugging the
human mistakes. Similarly, the type system ensures that 
queries do not contain undesired behavior. 
%
Thus, we believe that explicitly
encoding variability in the data and the queries gives developers the 
power of testing properties that otherwise would not be possible. 
This is arguably the strongest advantage of VDBs. Recently
lots of attempts have been done
on statically analysis, theorem proving, model checking, and program verification
of variable programs and 
variable structure~\cite{brkts20vamos, bks11fvoos, ldl07jss, tmbhvs14}, especially 
for safety-critical systems,
%\eric{Eric, feel free to add other related work to support this argument.}
 and since databases are part of 
software system the ability to analyze, check, and verify 
properties over variational databases provides programmers
with a seamless framework where they can verify software
and its artifacts.
%


%\textbf{What are the future research directions?}
%The case studies can be used to develop further research directions
%to make programming with variational databases easier and safer.
%%
%%The first direct future work following these case studies is generating
%%a VDB automatically from a set of databases with their corresponding
%%configuration. 
%%\parisa{remove this if we have a semi-auto gen vdb approach.}
%%
%%considering the new complexity and expressiveness added 
%%to the language (explicit encoding of variability), 
%For example, one can extend the query language with holes and even suggestions for 
%the holes would make programmers job easier in both writing queries
%and debugging them.
%%
%It will be very interesting to see the application of VDBs and 
%variational queries in software testing.

For RQ3.3 we are currently evaluating VDBMS with our case studies and we plan
to compare our different implementation approaches in terms of performance.
 
\secref{ro4} discusses the objective 4 and its research questions.
 
%\section{Variation in Space: Email SPL Use Case}
\label{sec:enron-vdb}


In our first case study, we focus on variation that occurs in ``space'', that
is, where multiple software variants are developed and maintained in parallel.
In software, variation in space corresponds to a SPL, where many distinct
variants (products) can be produced from a single shared code base by enabling
or disabling features. A variety of representations and tools have been
developed for indicating which code belongs to which feature(s) and supporting
the process of configuring a SPL to obtain a particular variant.


% E: This is repetitive with the intro, but I think it's OK to remind them here
% of what the current bad solutions are. Plus, we have space to fill. :-)

Naturally, different variants of a SPL have different information needs. For
example, an optional feature in the SPL may require a corresponding attribute
or relation in the database that is not needed by the other features in the
SPL.
%
Currently, there is no good solution to managing the varying information needs
of different variants at the level of the database.
%
One possible solution is to manually maintain a separate database schema for
each variant of the SPL. This works for some SPLs where the number of products
is relatively small and the developer has control over the configuration
process. However, it does not scale to open-source SPLs or other scenarios
where the number of products is large and/or configuration is out of the
developer's hands.
%
Another possible solution is to use and maintain a single universal schema that
includes all of the relations and attributes used by any feature in the SPL. In
this solution, every product will use the same database schema regardless of
the features that are enabled. This solves the problem of scaling to large
numbers of products but is dangerous because it means that potentially several
attributes and relations will be unused in any given product. Unused attributes
will typically be populated by \nul\ values, which are a well-known source of
errors in relational databases~\cite{AliceBook}.


VDBs solve the problem by allowing the structure of a relational database to
vary in a synchronous way with the SPL. Attributes and relations may be
annotated by presence conditions to indicate in which feature(s) those
attributes and relations are needed.
%
An implementation of the VDB model might use a universal schema under-the-hood
to realize VDBs on top of a standard relational database management system
(indeed, this is exactly how our prototype VDBMS implementation works), but by
capturing the variation in the schema explicitly, we can validate (potentially
variational) queries against the relevant variants of the variational schema to
statically ensure that no \nul\ values will be referenced.

%Our first use case focuses on variation in space.
The email SPL use case shows the use of VDB to encode the variational
information needs of a database-backed SPL. We consider an email
SPL that has been used in several previous SPL research projects (e.g.\
\cite{apel2013strategies,AlHaj19}).
It develops a variational
schema that captures the information needs of a SPL based on Hall's
decomposition of an email system into its component features~\cite{Hall05}. The
email SPL has been used in several previous SPL research projects (e.g.\
\cite{Apel13:SSP,AlHaj19}). The variational email database is populated using
the Enron email dataset, adapted to fit our variational schema~\cite{Shetty04}.
%
%Our first case study demonstrates the use of a VDB to encode the variational
%information needs of a database-backed SPL.
%
Our use case is formed by systematically combining two pre-existing works:
%
\begin{enumerate}
%
\item 
 We use Hall's decomposition of an email system into its component
features~\cite{Hall05} as high-level specification of a SPL.
%
\item 
 We use the Enron email
dataset\footnote{\url{http://www.ahschulz.de/enron-email-data/}} as 
a source of
a realistic email database.
%
\end{enumerate}
%
In combining these works, we show how variation in space in an email SPL
requires corresponding variation in a supporting database, how we can link the
variation in the software to variation in the database, and how all of these
variants can be encoded in a single VDB.


\subsection{Variation Scenario: An Email SPL}
\label{sec:enron-scenario}


The email SPL consists of the following features from \citet{Hall05}: 
%
\begin{itemize}
%[leftmargin=*]
%\itemsep0em
%
\item 
\addressbook, users can maintain lists of known email addresses with
corresponding aliases, which may be used in place of recipient addresses;
%
\item 
\signature, messages may be digitally signed and verified using
cryptographic keys;
%
\item 
\encryption, messages may be encrypted before sending and decrypted upon
receipt using cryptographic keys;
%
\item 
\autoresponder, users can enable automatically generated email responses
to incoming messages;
%
\item 
\forwardmessages, users can forward all incoming messages automatically to another
address;
%
\item 
\remailmessage, users may send messages anonymously;
%
\item 
\filtermessages, incoming messages can be filtered according to a
provided white list of known sender address suffixes; and
%
\item 
\mailhost, a list of known users is maintained and known users may
retrieve messages on demand while messages sent to unknown users are rejected.
%
\end{itemize}

%\noindent
%
Note that 
Hall's decomposition separates \signature\ and \encryption\ into two
features each (corresponding to signing and verifying, encrypting and
decrypting). Since these pairs of features must always be enabled together
 and
they are so closely conceptually related, 
we 
reduce them to one feature each for simplicity.


The listed features are used in presence conditions within the
variational schema for the email VDB, linking the software variation to
variation in the database.
%
In the email SPL, each feature is optional and independent, resulting in the
  simple feature model $\fModel_\enron = \t$,
% , which is equivalent to \t, 
given as a feature expression.
%%\eric{should we just consider this feature model as true? that is how 
%%we're implementing it.}
%%
%\begin{align*}
%\fModel_\enron
%  &= \t \vee \addressbook \vee \signature \vee \encryption \\
%  &\quad \vee \autoresponder \vee \forwardmessages \\
%  &\quad \vee \remailmessage \vee \filtermessages \vee \mailhost
%\end{align*}
%
The feature model $\fModel_\enron$ is used as the root presence condition of
the variational schema for the email VDB, implicitly applying it to all
relations, attributes, and tuples in the database.

%For the rest of this section, we study a SPL that 
%generates email systems with different features and demonstrate
%how variation appears in its database in addition to how all variants
%of a database can be encoded in a single variational database. We adapt the
%email system explained in~\cite{} by considering the following features:
%...
%
%We consider the simple feature model of:
%...
%The feature model is part of the variational schema too and is applied
%to all its tables and attributes. 

\subsubsection{Generating V-Schema of the Email SPL VDB}
\label{sec:enron-vsch}

\begin{table}
\caption{Original Enron email dataset schema.}
\vspace{-8pt}
\label{tab:enron}
%\begin{center}
\small
\begin{tabular} {|l|}
%\hline
%\textbf{Enron Schema} \\
\hline 
\employees(\eid, \fname, \lname, \emailid, $\mathit{email2}$, \\
\hspace{40pt} $\mathit{email3}$, $\mathit{email4}$, \folder, \status) \\
\messages(\midatt, \sender, \dateatt, \messageid, \subject, \body, \folder)  \\ 
\recipientinfo(\rid, \midatt, \rtype, \rvalue)  \\
\referenceinfo(\rid, \midatt, \reference)  \\
\hline
\end{tabular}
%\end{center}
\vspace{-13pt}
\end{table}


To produce a v-schema for the email VDB, we start from plain schema
of the Enron email dataset shown in \tabref{enron}, then systematically adjust
its schema to align with the information needs of the email SPL described by
\citet{Hall05}.
%
The \employees\ table contains information about the employees of the company
including the employee identification number (\eid), their first name and last
name (\fname\ and \lname), their primary email address (\emailid), alternative
email addresses (e.g.\ $\mathit{email2}$), a path to the folder that contains
their data (\folder), and their last status in the company (\status).
%
The \messages\ table contains information about the email messages
 including
the message ID (\midatt), the sender of the message (\sender), the date
(\dateatt), the internal message ID (\messageid), the subject and body of the
message (\subject\ and \body), and the exact folder of the email (\folder).
% 
The \recipientinfo\ table contains information about the recipient of a message
including the recipient ID (\rid), the message ID (\midatt), the type of the
message (\rtype), and the email address of the recipient (\rvalue).
%
The \referenceinfo\ table contains messages that have been referenced in other
email messages,
for example, in a forwarded message; it contains a
reference-info ID (\rid), the message ID (\midatt), and the entire message
(\reference). 
This table simply backs up the emails.


\begin{table*}
\caption{V-schema of the email VDB with feature model
\ensuremath{\fModel_\enron}. 
Presence conditions are colored blue for clarity.
}
\vspace{-8pt}
\label{tab:enron-vsch}
% \begin{center}
\small
\begin{tabular} {|l|}
%\hline
%\textbf{Variational Schema for Email SPL} \\
\hline 
 % \annot [\vFour] \name
$\employees(\eid, \fname, \lname, \emailid, \folder, \status,$ 
%\\ \hspace{40pt} 
$\annot[\fsignature]{\verificationkey},
  \annot[\fencryption]{\publickey})$ \hspace{20pt}
 $\recipientinfo(\rid, \midatt, \rtype, \rvalue)$ 
  \\ %[1.1ex]
$\messages(\midatt, \sender, \dateatt, \messageid, \subject, \body, \folder,$ 
%\\ \hspace{30pt} 
$\issystemnotification,
  \annot[\fencryption]{\isencrypted}, 
  \annot[\fautoresponder]{\isautoresponse},\annot[\fsignature]{\issigned} $ 
  \\ \hspace{30pt} 
  $,
  \annot[\fforwardmsg]{\isforwardmsg})$  \hspace{40pt}
%   \\ %[1.1ex]
%$\recipientinfo(\rid, \midatt, \rtype, \rvalue)$   
${\forwardmsg(\eid, \forwardaddr)}^{\fforwardmsg} $\hspace{40pt}
${\mailhost(\eid, \username, \mailhost)}^{\fmailhost}$ 
\\
%\hspace{20pt}
% \\ %[1.1ex]
%\referenceinfo(\rid, \midatt, \reference)  \\
  ${\filtermsg(\eid, \suffix)}^{\ffiltermsg} $ \hspace{3pt}
%\\ %[1.1ex]
%  \hspace{47pt}
%\\ %[1.1ex]
${\remailmsg(\eid, \pseudonym)}^{\fremailmsg}$    \hspace{3pt}
%\hspace{20pt}
%\\ %[1.1ex]
${\automsg(\eid, \subject, \body)}^{\fautoresponder} $  \hspace{3pt}
%\\ %[1.1ex]
${\alias(\eid, \emailAtt, \nickname)}^{\faddressbook}$\\
%\\ %[1.1ex]
%${\filtermsg(\eid, \suffix)}^{\ffiltermsg} $\\ %[1.1ex]
\hline
\end{tabular}
\vspace{-13pt}
% \end{center}
\end{table*}


From this starting point, we introduce new attributes and relations that are
needed to implement the features in the email SPL. We attach presence
conditions to new attributes and relations corresponding to the features
they are needed to support, which ensure they will \emph{not} be present
in configurations that do not include the relevant features.
%
The resulting v-schema is given in \tabref{enron-vsch}.


%As an example of this process, 
For example, consider the \signature\ feature. In the
software, implementing this feature requires new operations for signing an
email before sending it out and for verifying the signature of a received
email. These new operations suggest new information needs: we need a way to
indicate that a message has been signed, and we need access to each user's
public key to verify those signatures (private keys used to sign a message
would not be stored in the database). These  needs are reflected in
the v-schema by the new attributes \verificationkey\ and \issigned,
added to the relations \employees\ and \messages, respectively. The new
attributes are annotated by the \signature\ presence condition, indicating that
they correspond to the \signature\ feature and are unused in configurations
that exclude this feature.
%
Additionally, several features require adding entirely new relations, e.g.,
%
%For example, 
when the \forwardmsg\ feature is enabled, the system must keep
track of which users have forwarding enabled and the address to forward the messages to.
%should be forwarded to. 
This need is reflected by the new
\forwardmsg\ relation, which is correspondingly annotated by the \forwardmsg\
presence condition.


A main focus of Hall's decomposition~\cite{Hall05} is on the many feature
interactions.
% in the email SPL. 
Several of the features may interact in
undesirable ways if special precautions are not taken. For example, any
combination of the \forwardmsg, \remailmsg, and \autoresponder\ features can
trigger an infinite messaging loop if users configure the features in the wrong
way; preventing this creates an information need to identify auto-generated
emails, which is realized in the variational schema by attributes like
\isforwardmsg\ and \isautoresponse.
%
% occurs between the \signature\ and \fremailmsg\ features: the \fremailmsg\
% feature enables anonymously sending messages by replacing the sender with a
% pseudonym, but this prevents the recipient from being able to verify a signed
% email.
%
% occurs between the \signature\ and \forwardmsg\ features: if Sarah signs a
% message and sends it to Ina, and Ina forwards the message to Philippe, then
% the signature verification operation may incorrectly interpret Ina as the
% sender rather than Sarah and fail to verify the message.

%p: moved the following paragraph to discussion.
%For each feature, we (1) enumerated the operations that must be supported both
%to implement the feature itself and to resolve undesirable feature
%interactions, (2) identified the information needs to implement these
%operations, and (3) extended the variational schema to satisfy these
%information needs.
%
% The changes made to accommodate the features \addressbook, \encryption,
% \autoresponder, \remailmessage, \filtermessages, and \mailhost\ are similar. 
%

% are provided in the
%second author's MS project report~\citet{Li19}.
%
For brevity, we omit some attributes and relations from the original schema
that are  irrelevant to the email SPL as described by Hall, such as the
\referenceinfo\ relation and alternative email addresses. 
%
%p: will ref qiaoran's work somewhere else, if space permits.
%\citet{Li19} provides
%a complete description of adaptations done in the process in her MS project report.
% \t presence condition can be omitted.


%We distribute the variational schema for the email VDB in two formats.
%%
%First, we provide the schema in the encoding used by our prototype VDBMS tool.%
%\footnote{\href{https://github.com/lambda-land/VDBMS/blob/master/usecases/space-emailSPL/schema/EnronSchema.hs}{usecases/space-emailSPL/schema}}
%
%Second, 
In addition to providing the schema in the encoding used by our 
prototype VDBMS tool, we also provide a direct encoding in SQL
%The SQL encoding 
which generates the
%\eric{Eric, I don't like universal here! it may give the wrong idea!
%maybe just say v-schema?}
universal schema for the VDB.
% in either MySQL or Postgres.%
%\footnote{\href{https://github.com/lambda-land/VDBMS/tree/master/usecases/space-emailSPL/database/create}{usecases/space-emailSPL/database/create}}
%
Variation is encoded as an additional relation of the form \vdbpc\ that
captures all of the relevant presence conditions: that of the 
v-schema itself (i.e.\ the feature model), and those of each relation and
attribute.%
%\footnote{\href{https://github.com/lambda-land/VDBMS/tree/master/usecases/space-emailSPL/database/withSchema}{usecases/space-emailSPL/database/withSchema}}
%
The $\mathit{element\_id}$ of the feature model is
$\mathit{variational\_schema}$; the $\mathit{element\_id}$ of a relation \vRel\
is its name \vRel, and of attribute \vAtt\ in relation \vRel\ is $\vRel.\vAtt$.
%
The plain SQL encoding of the v-schema supports the use
of the case studies for research on the effective management of variation in
databases independent of VDBMS.

%feature
%model of the variational schema, a relation \tablespc\ that stores the presence
%condition of each relation, and for each table $R\left(a_1, \cdots, a_n\right)$
%in the variational schema there is a table $t\left(\mathit{attribute\_name},
%\mathit{pres\_cond}\right)$ in the database that stores its attributes'
%presence conditions.

\subsection{Populating the Email SPL VDB}
\label{sec:enron-pop}

The final step to create the email VDB is to populate the database with data
from the Enron email dataset, adapted to fit our variational schema~\cite{Shetty04}.
%
For evaluation purposes, we want the data from the dataset to be distributed
across multiple variants of the VDB. To simulate this, we identified five
plausible configurations of the email SPL, which we divide the data among. The
five configurations of the email SPL 
 we considered are:
%
\begin{itemize}
%
\item 
\emph{basic email}, which includes only basic email functionality
and does not include any of the optional features
 from the SPL.
%
\item 
\emph{enhanced email}, which extends \emph{basic email} by
enabling two of the most commonly used email features, \forwardmessages\ and
\filtermessages.
%
\item 
\emph{privacy-focused email}, which extends \emph{basic email}
with features that focus on privacy, specifically, the
\signature, \encryption, and \remailmessage\ features.
%
\item 
\emph{business email}, which extends \emph{basic email} with
features tailored to an environment where most emails are expected to be among
users within the same business network, specifically, \addressbook,
\signature, \encryption, \autoresponder, and \mailhost.
%
\item 
\emph{premium email}, in which all of the optional features
in the SPL are enabled.
%
\end{itemize}
%
For all variants, any features that are not enabled are disabled. 


The original Enron dataset has 150 employees with 252,759 email messages. 
%
We load this data into the \employees\ and \messages\ tables defined in
\secref{enron-vsch}, initializing all attributes that are not present in the
original dataset to \nul.


For the \employees\ table, we construct five views corresponding to the five
variants of the email system described above. We allocate 30 employees to each
view based on their employee ID, that is, the first 30 employees sorted by
employee ID are associated with the basic email variant, the next 30 with the
enhanced email variant, and so on. The presence condition for each tuple is set
to the conjunction of features enabled in that view.
%
We then modify each of the views of the \employees\ table by adding randomly
generated values for attributes associated with the enabled features; 
e.g., in the view for the privacy-focused variant, we populate the
\verificationkey\ and \publickey\ attributes.
%
Any attribute that is not present in the given tuple due to a conflicting
presence condition will remain \nul. For example, both the \verificationkey\
and \publickey\ attributes remain \nul\ for employees in the enhanced variant
view since the presence condition does not include the corresponding features.


For the \messages\ table, we again create five views corresponding to each of
the variants. Each tuple is added to the view of the variant that contains the
message's sender, which updates the tuple's presence condition accordingly.
%
The \messages\ table also contains several additional attributes corresponding
to optional features, which we populate in a systematic way.
%
We set \issigned\ to \t\ if the message sender has the \signature\ feature
enabled, and we set \isencrypted\ to \t\ if \emph{both} the message sender
and recipient have \encryption\ enabled.
%
We populate the \isforwardmsg, \isautoresponse, and \issystemnotification\
attributes by doing a lightweight analysis of message subjects to determine
whether the email is any of these special kinds of messages; for example, if
the subject begins with ``FWD'', we set the \isforwardmsg\ attribute to \t.
%
If a forward or auto-reply message was sent by a user that does not have the
corresponding feature enabled, we filter it out of the dataset. After
filtering, the \messages\ relation contains 99,727 messages.
%
For each forward or auto-reply message, we also add a tuple with the relevant
information to the new \forwardmsg\ and \automsg\ tables.
%
For employees belonging to database variants that enable \remailmessage,
\autoresponder, \addressbook, or \mailhost\ we randomly generate tuples in the
tables that are specific to each of these features.
%
Finally, the \recipientinfo\ relation is imported directly from the dataset. We
set each tuple's presence condition to a conjunction of the presence conditions
of the sender and recipient.
% \eric{actually I conjuncted them, which now I'm not so sure about it!}


%We provide more detailed instructions for systematically constructing the email
%VDB in both MySQL and PostgreSQL in a wiki page associated with the
%repository.%
%\footnote{\url{https://github.com/lambda-land/VDBMS/wiki/Enron-Email-Database-UseCase-Doc\#steps-take-to-build-vdb-for-enron-case-study}}
%%
We provide SQL scripts to automate the creation of views for each
variant%
%
%\footnote{\href{https://github.com/lambda-land/VDBMS/blob/master/usecases/space-emailSPL/database/build/step1_build_email_variants.sql}{usecases/space-emailSPL/database/build/step1\_build\_email\_variants.sql}}
%%
and to automate the population of these views with tuples from the original
dataset,%
%%
%\footnote{\href{https://github.com/lambda-land/VDBMS/blob/master/usecases/space-emailSPL/database/build/step2_build_email_vdb.sql}{usecases/space-emailSPL/database/build/step2\_build\_email\_vdb.sql}} 
%%
which also sets each tuple's presence condition.
%
%
The resulting database is distributed in two forms, one with the embedded
variational schema as described in \secref{enron-vsch},%
%
%\footnote{\href{https://github.com/lambda-land/VDBMS/tree/master/usecases/space-emailSPL/database/withSchema}{usecases/space-emailSPL/database/withSchema}}
%
and one without the embedded schema%
%\footnote{\href{https://github.com/lambda-land/VDBMS/tree/master/usecases/space-emailSPL/database/withoutSchema}{usecases/space-emailSPL/database/withoutSchema}}
%
for use with our VDBMS tool in which the variational schema is provided
separately.%
\footnote{Both the scripts and different encodings of the email SPL VDB are
available at: \url{https://zenodo.org/record/4321921}.} 
%
We have tested the email SPL VDB for the properties described in \secref{vdbfprop} 
and all of them hold.

%\point{VDBs are a good fit to encode variation in a database of a 
%database-backed SPL.}
%%variation over space
%
%
%\point{To showcase how a VDB captures different needs arising in SPLs 
%w.r.t. variation in their databases we generate a VDB for an email SPL.}\\
%- where data is coming from\\
%- where spl is coming from
%
%\point{We adopt the features and feature model from the SPL and 
%consider five database variants.}\\
%- describe 5 variants and their config.
%
%\point{Features are incorporated into both the schema and data.}\\
%- feature model\\
%- schema\\
%- data\\


%\subsection{Variation in Time: Employee Case Study}
\label{sec:emp-vdb}

%\TODO{from old intro}
%The second case study demonstrates the changing information needs of a system
%that varies over time by developing a variational schema corresponding to an
%employee-system evolution scenario described in \citet{prima08Moon}. The
%variational employee database is populated by adapting a large, fabricated
%employee dataset\footnote{\url{https://github.com/datacharmer/test_db}} that
%has been widely used in the databases community.


%In our second case study, we focus on variation that occurs in ``time'', that
%is, where the software variants are produced sequentially by incrementally
%extending and modifying the previous variant in order to accommodate new
%features or changing business requirements. Although new variants conceptually
%replace older variants, in practice, older variants must often be maintained in
%parallel; external dependencies, requirements, and other issues may prevent
%clients from updating to the latest version.
%%
%Variation in software over time directly affects the databases such software
%depends on~\cite{dbDecay16Stonebraker}, and dealing with such changes is a
%well-studied problem in the database community known as \emph{database
%evolution}~\cite{schVersioningSurvey95Roddick}.
%
%
%Although research on database evolution has produced a variety of solutions for
%managing database variation over time, these solutions do not treat variation
%as an orthogonal property and so cannot also accommodate variation in space.
%The goal of our work on variational databases is not to directly compete with
%database evolution solutions for time-only variation scenarios, but rather to
%present a more general model of database variation that can accommodate
%variation in both time and space, and that integrates with related software via
%feature annotations.

In our second case study, we focus on variation that occurs in ``time''.
It demonstrates the use of a VDB to encode an employee database evolution
scenario. We systematically adapt an existing database evolution scenario from
\citet{prima08Moon} into a VDB and populate it by a dataset that is widely used
in databases research.\footnote{\url{https://github.com/datacharmer/test_db}}


\subsubsection{Variation Scenario: An Evolving Employee Database}
\label{sec:emp-scenario}

\begin{table}
\caption[Evolution of an employee database schema]{Evolution of an employee database schema from \citet{prima08Moon}.
}
\label{tab:emp-sch}
\begin{center}
\small
\begin{tabular} {!{\color{black}\vrule}l !{\color{black}\vrule} l !{\color{black}\vrule}}
\arrayrulecolor{black}
\hline
\textbf{Version} & \textbf{Schema}\\
% & \multirow{1}{0.3cm}{\textbf{\sI}}&  \\
\hline 
\hline 
% \dashuline{\isstudent}
\multirow{3}{0.3cm}{\vOne} &  \engemp\ (\empno, \name, \hiredate, \titleatt, \deptname) \\
%& &  \multirow{3}{2cm}{hiredate $<$ 1988-01-01}\\
& \othemp\ (\empno, \name, \hiredate, \titleatt, \deptname) \\
%& \sOne &\\
& \job\ (\titleatt, \salary)\\
% & &\\
\hline
\multirow{2}{0.3cm}{\vTwo} & \empacct\ (\empno, \name, \hiredate, \titleatt, \deptname) \\
%& \multirow{2}{0.5cm}{\sTwo}&  \multirow{2}{2cm}{hiredate $<$ 1991-01-01}\\
& \job\ (\titleatt, \salary) \\
%&  &\\
\hline
\multirow{3}{0.3cm}{\vThree} & \empacct\ (\empno, \name, \hiredate, \titleatt, \deptno) \\
%&\multirow{3}{0.5cm}{\sThree}&  \multirow{3}{2cm}{hiredate $<$ 1994-01-01}\\
& \job\ (\titleatt, \salary)\\
% &  &\\
& \dept\ (\deptname, \deptno, \managerno) \\
%& &\\
\hline
\multirow{4}{0.3cm}{\vFour} & \empacct\ (\empno, \hiredate, \titleatt, \deptno) \\
%& \multirow{4}{0.5cm}{\sFour} &  \multirow{3}{2cm}{hiredate $<$ 1997-01-01}\\
& \job\ (\titleatt, \salary) \\
%& & \\
& \dept\ (\deptname, \deptno, \managerno)\\
%& & \\
& \empbio\ (\empno, \sex, \birthdate, \name)\\
%  & & \\
\hline
\multirow{3}{0.3cm}{\vFive} & \empacct\ (\empno, \hiredate, \titleatt, \deptno, \salary) \\
%& \multirow{3}{0.5cm}{\sFive} &  \multirow{3}{2.3cm}{hiredate $<$ 2000-01-28}\\
& \dept\ (\deptname, \deptno, \managerno) \\
%& & \\
& \empbio\ (\empno, \sex, \birthdate, \fname, \lname)\\
% & & \\
\hline
\end{tabular}
\end{center}
\end{table}


\citet{prima08Moon} describe an evolution scenario in which the schema of a
company's employee management system changes over time, yielding the five
versions of the schema shown in \tabref{emp-sch}.
%
In \vOne, employees are split into two separate relations for
engineer and non-engineer personnel.
%
In \vTwo, these two tables are merged into one relation, \empacct.
%
In \vThree, departments are factored out of the \empacct\ relation and
into a new \dept\ relation to reduce redundancy in the database.
%
In \vFour, the company decides to start collecting more personal
information about their employees and stores all personal information in the
new relation \empbio.
%
Finally, in \vFive, the company decides to decouple salaries from
job titles and instead base salaries on individual employee's qualifications
and performance; this leads to dropping the \job\ relation and adding a new
\salary\ attribute to the \empacct\ relation. This version also separates the
\name\ attribute in \empbio\ into \fname\ and \lname\ attributes.


We associate a feature with each version of the schema, named 
$\vOne\ldots\vFive$.
%
These features are mutually exclusive since only one version of the
schema is valid at a time. This yields the  feature model:
%$\fModel_\employee$:
%
% Also, note that the feature model represent a restriction on the entire
% database.
%
\begin{align*}
\fModel_\employee
  &=   \left(\vOne\wedge\neg\vTwo\wedge\neg\vThree\wedge\neg\vFour\wedge\neg\vFive\right)\\
  &\hspace{-15pt} \vee\left(\neg\vOne\wedge\vTwo\wedge\neg\vThree\wedge\neg\vFour\wedge\neg\vFive\right)
%  \\
%  &
  \vee\left(\neg\vOne\wedge\neg\vTwo\wedge\vThree\wedge\neg\vFour\wedge\neg\vFive\right)\\
   &\hspace{-15pt} \vee\left(\neg\vOne\wedge\neg\vTwo\wedge\neg\vThree\wedge\vFour\wedge\neg\vFive\right)
%  \\
%  &
  \vee\left(\neg\vOne\wedge\neg\vTwo\wedge\neg\vThree\wedge\neg\vFour\wedge\vFive\right)
\end{align*}

%As a reminder, based on the hierarchy of presence conditions, 
%the feature model $\fModel_\employee$ is used as the root presence condition of
%the variational schema for the employee VDB, implicitly applying it to all
%relations, attributes, and tuples in the database.


\subsubsection{Generating V-Schema of the Employee VDB}
\label{sec:emp-vsch}

\begin{table}
\caption{Employee v-schema with feature model.}
% \ensuremath{\fModel_\employee}.}
\vspace{-8pt}
\label{tab:emp-vsch}
\begin{center}
\small
\begin{tabular} {|l|l|}
\hline
%\textbf{Variational Schema for Employee Evolution} \\
%\hline 
 % \annot [\vFour] \name
\rule{0pt}{3ex}%
$\engemp(\empno, \name, \hiredate,\titleatt,\deptname)^{\textcolor{blue}
\vOne}$ \\[1.1ex]
$\othemp(\empno, \name, \hiredate, \titleatt, \deptname)^{\textcolor{blue}
\vOne}$ \\[1.1ex]
$\empacct(\empno, \annot [\textcolor{blue}{\vTwo \vee \vThree}] \name,
\hiredate, \titleatt,$ \\
$\qquad\annot[\textcolor{blue} \vTwo]\deptname, \annot
[\textcolor{blue} {\vThree \vee \vFour \vee \vFive}]\deptno,\annot
[\textcolor{blue} \vFive]\salary)^{\textcolor{blue}{\vTwo \vee \vThree \vee
\vFour \vee \vFive}}$ \\[1.1ex]
$\job(\titleatt, \salary)^{\textcolor{blue}{\vTwo \vee \vThree \vee \vFour}}$
\\[1.1ex]
$\dept(\deptname, \deptno, \managerno)^{\textcolor{blue}{\vThree \vee \vFour
\vee \vFive}}$ \\[1.1ex]
$\empbio(\empno, \sex, \birthdate, \annot [\textcolor{blue} \vFour] \name,
\annot[\textcolor{blue}\vFive]{\fname}, \annot[\textcolor{blue}
\vFive]{\lname})^{\textcolor{blue} {\vFour \vee \vFive}}$ \\
\hline
\end{tabular}
\vspace{-12pt}
\end{center}
\end{table}


The v-schema for this scenario is given in \tabref{emp-vsch}. It
encodes all five of the schema versions in \tabref{emp-sch} and was
systematically generated by the following process. First, generate a universal
schema from all of the plain schema versions; the universal schema contains
every relation and attribute appearing in any of the five versions. Then,
annotate the attributes and relations in the universal schema according to the
versions they are present in.
%
For example, the \empacct\ relation is present in versions \vTwo--\vFive, so it
will be annotated by the feature expression
$\vTwo\vee\vThree\vee\vFour\vee\vFive$, while the \salary\ attribute within the
\empacct\ relation is present only in version \vFive, so it will be annotated
by simply \vFive.
%
% The overall variational schema will be annotated by the feature model
% $\fModel_\employee$, described in \secref{emp-scenario}.
%
Since the presence conditions of attributes are implicitly conjuncted with the
presence condition of their relation,
% that contains them, 
 we can avoid redundant
annotations when an attribute is present in all instances of its parent
relation. For example, the \empbio\ relation is present in $\vFour\vee\vFive$,
and the \birthdate\ attribute is present in the same versions, so we do not
need to redundantly annotate it.
% \birthdate.


%Similar to the email SPL VDB, we distribute the variational schema for the
%employee VDB in two formats:
%%
%First, we provide the schema in the encoding used by our prototype VDBMS tool.%
%\footnote{\href{https://github.com/lambda-land/VDBMS/blob/master/usecases/time-employee/schema/EmployeeSchema.hs}{usecases/time-employee/schema/EmployeeSchema.hs}}
%%
%Second, we provide a direct encoding in SQL that generates the universal schema
%for the VDB in either MySQL or Postgres.%
%\footnote{\href{https://github.com/lambda-land/VDBMS/tree/master/usecases/time-employee/database/create}{usecases/time-employee/database/create}}
%%
%The variability of the schema is embedded within the employee VDB%
%\footnote{\href{https://github.com/lambda-land/VDBMS/tree/master/usecases/time-employee/database/withSchema}{usecases/time-employee/database/withSchema}}
%%
%using the same encoding as described at the end of \secref{enron-vsch}.


\subsubsection{Populating the Employee VDB}
\label{sec:emp-pop}

Finally, we populate the employee VDB using data from the widely used employee
database linked to in this subsection's lede.
%
This database contains information for $240,124$ employees. To simulate the
evolution of the database over time, we divide the employees into five roughly
equal groups based on their hire date within the company. 
%For example, the
%first group consists of employees hired before $1988-01-01$, while the second
%group contains employees hired from $1988-01-01$ to $1991-01-01$.
%
Each group is assumed to have been hired during the lifetime of a particular
version of the database, and is therefore added to that version of the database
and \emph{also} to all subsequent versions of the database. This simulates the
fact that as a database evolves, older records are typically forward propagated
to the new schema~\cite{schVersioningSurvey95Roddick}. Thus, \vFive\ contains
the records for all $240,124$ employees, while older versions will contain
progressively fewer records.
%
The final employee VDB has $954,762$ employee due to this forward propagation,
despite having the same number of employees as the original database.


The schema of the employee database used to populate the employee VDB
 is different from all versions of the
v-schema, yet it includes all required information. Thus,
%To populate
%the VDB, 
we manually mapped data from the original schema onto each version of
the v-schema.


%We provide SQL scripts of required queries to automatically 
%generate the employee VDB.
%\footnote{\href{https://github.com/lambda-land/VDBMS/blob/master/usecases/time-employee/database/build}{usecases/time-employee/database/build}}
%%We provide SQL scripts to automate the separation of each group of employees
%%into views according to their hire date%
%%\footnote{\href{https://github.com/lambda-land/VDBMS/blob/master/usecases/time-employee/database/build/step1_chop_employees.sql}{usecases/time-employee/database/build/step1\_chop\_employees.sql}}
%%%
%%and populating those views from data in the employee database.%
%%\footnote{\href{https://github.com/lambda-land/VDBMS/blob/master/usecases/time-employee/database/build/step2_build_vdb.sql}{usecases/time-employee/database/build/step2\_build\_vdb.sql}}
%%
%%As for any VDB, if an attribute is not present in any of the variants covered
%%by a tuple's presence condition, that attribute will be set to NULL in the
%%tuple. We do this even though the relevant information may be contained in the
%%original employee database to ensure that we have a consistent VDB. For
%%example, while inserting tuples into the \vFour\ view of the \empbio\ table, we
%%always insert NULL values attributes \fname\ and \lname.
%%
%We also provide the final employee VDB in four flavors: both with and without the
%embedded schema, and in both cases, encoded in MySQL and PostgreSQL format.%
%\footnote{\href{https://github.com/lambda-land/VDBMS/tree/master/usecases/time-employee/database}{usecases/time-employee/database}}


%\point{VDBs are a good fit to capture schema evolution of a database.}
%%, i.e., its variation over time.}
%
%\point{To showcase how a VDB captures variation over time we encode
%the schema evolution of an employee database as a VDB.}\\
%- data comes from blah. schema evolution comes from blah.\\ 
%- data stat. schema evolution. \\
%
%\point{We consider each version of the schema evolution as a database 
%variant and assign a feature to them.}
%
%\point{We incorporate the features into both the schema and data.}\\
%- how we incorporate features into schema. \\
%- how we incorporate features into tables.
%\rewrite{rewrite tables to your liking! taken from @Q}

%%\vspace{-7.5pt}
\subsection{Email Query Set}
\label{sec:enron-qs}

%\point{We write queries based on SPL developers needs when different SPL features
%interact with each other~\cite{blah}.}\\
%- give an example.

To produce a set of queries for the email SPL use case, we collected all of
the information needs that we could identify in the description of the email
SPL by \citet{Hall05}.. In order to make the information needs more concrete, we
viewed the requirements of the email SPL mostly through the lens of
constructing an email header.
%
An email header includes all of the relevant information needed to send an
email and is used by email systems and clients to ensure that an email is sent to
the right place and interpreted correctly. 
More specifically, the email header
includes the sender and receiver of the email, whether an email is signed and
the location of a signature verification key, whether an email is encrypted and
the location of the corresponding public key, the subject and body of the
email, the mail host it belongs to, whether the email should be filtered,
% the verification status of the email and who has verified the email, 
and so on.
%
Although there is obviously other infrastructure involved, the fundamental
information needs of an email system can be understood by considering how to
construct email headers
 that ensures the email would get where it needs to go
 and be interpreted correctly on the other end.


Hall's decomposition focuses on enumerating the features of the email SPL and
enumerating the potential interactions of those features.
%
We deduce the information need for each feature by asking: ``what information
is needed to modify the email header in a way that incorporates the new
functionality?''. We deduce the information need for each interaction by
asking: ``what information is needed to modify the email header in a way that
avoids the undesirable feature interaction?''.
%
We can then translate these information needs into queries on the underlying
variational database.


In total, we provide $27$ queries for the email SPL.
%, encoded in both VRA
%format usable by our VDBMS tool and as \cpp{ifdef} annotated SQL, as described
%above.%
%
%\footnote{\href{https://github.com/lambda-land/VDBMS/tree/master/usecases/space-emailSPL/queries}{usecases/space-emailSPL/queries}}
%
This consists of $1$ query for constructing the basic email header, $8$ queries
for realizing the information needs corresponding to each feature, and $18$
queries for realizing the information needs to correctly handle the feature
interactions described by Hall.


We start by presenting the query to assemble the basic email header, \Qbasic.
This corresponds to the information need of a system with no features enabled.
We use $X$ to stand for the specific message ID (\midatt) of the email whose
header we want to construct.
%
%\vspace{-2pt}
\begin{align*}
\Qbasic &= 
%\pi_{\sender,\rvalue,\subject,\body} 
%((\selectRel\midCond\messages)\\
%&\hspace{20pt} \bowtie_{\messages.\midatt = \recipientinfo.\midatt} \recipientinfo)
\projectRel{\sender,\rvalue,\subject,\body} \mrtable \\
\mrtable &\leftarrow (\selectRel\midCond\messages) \bowtie \recipientinfo
%\mrtable &\leftarrow (\selectRel\midCond\messages) \\
%&\qquad \bowtie_{\messages.\midatt = \recipientinfo.\midatt} \recipientinfo
\end{align*}
%\vspace{-13pt}
%
This query extracts the sender, recipient, subject, and body of the email to
populate the header. The projection is applied to an intermediate result
\mrtable\ constructed by joining the \messages\ table with the \recipientinfo\
table on recipient IDs; we reuse this intermediate result also in subsequent
queries.


Taking \Qbasic\ as our starting point, we next construct our set of $8$
\emph{single-feature queries} that capture the information needs specific to
each feature.
%
When a feature is enabled in the SPL, more information is needed to construct
the header of email $X$. For example, if the feature \filtermessages\ is
enabled, then the query \Qfilter\ extends \Qbasic\ with the \suffix\ attribute
used in filtering. This additional information allows the system to filter a
message if its address contains any of the suffixes set by the receiver.
%
%% Since this queries involve features that are optional, these
%% queries are variational and correspond to multiple different plain queries.
%% However, as described in \secref{background}...
%
%\vspace{-2pt}
\begin{align*}
\Qfilter &= 
%\pi_{\sender,\rvalue,\suffix,\subject,\body}
%((\employees \\
%&\bowtie_{\rvalue=\emailid}
% \mrtable) \bowtie\filtermsg )
%%_{\employees.\eid = \filtermsg.\eid}\filtermsg )\\
%% \bowtie_{\rvalue=\emailid}{\employees}\\
\projectRel{\sender,\rvalue,\suffix,\subject,\body} \temp \\
\temp & \leftarrow \mretable
 \bowtie \filtermsg \\
%&\qquad \bowtie_{\employees.\eid = \filtermsg.\eid}\filtermsg \\
\mretable &\leftarrow \joinRel{\mrtable}{\employees}{\rvalue=\emailid}
\end{align*}
%\vspace{-1pt}
%
%The intermediate result \mretable\ formed by joining \mrtable\ with the
%\employees\ relation will be reused in later queries.
%
We can construct a query that retrieves the required header information whether
\filtermessages\ is enabled or not by combining \Qbasic\ and \Qfilter\ in a
choice, as $\Qbf=\chc[\filtermessages]{\Qfilter,\Qbasic}$. 
%
Although we do not show the process in this paper, we can use equivalence laws
from the choice calculus~\cite{EW11tosem,HW16fosd} to factor commonalities out
of choices and reduce redundancy in queries like \Qbf.
% For clarity, we distribute the un-factored form presented here in our case
% studies.
The other single-feature queries are written similarly.
% to the one shown here.


% \centerline{$\choiceS \forwardmessages {\Qforward} {\Qbasic}$}

As another example of a single-feature query, \Qforward\ captures the
information needs for implementing the \forwardmessages\ feature. It is similar
to the previous queries except that it extracts the \forwardaddr\ from the
\automsg\ table, which is needed to construct the message header for the new
email to be forwarded when email $X$ is received by a user with a \forwardaddr\
set.
%
\begin{align*}
\Qforward &=
\projectRel{\rvalue,\forwardaddr,\subject,\body}{\temp} \\
\temp &\leftarrow \mretable \\
&\qquad \bowtie_{\employees.\eid = \forwardmsg.\eid} \automsg
\end{align*}

The other single-feature queries are similar to those shown here.


Besides single-feature queries, we also provide queries that gather information
needed to identify and address the undesirable feature interactions described
by \citet{Hall05}. Out of Hall's 27 feature interactions, we determined 16 of
them to have corresponding information needs related to the database; 2 of the
interactions require 2 separate queries to resolve. Therefore, we define and
provide 18 queries addressing all 16 of the relevant feature interactions.
%
As before, we deduced the information needs through the lens of constructing an
email header; in these cases, the header would correspond to an email produced
after successfully resolving the interaction.
%
However, some interactions can only be detected but not automatically resolved.
In these cases, we constructed a query that would retrieve the relevant
information to detect and report the issue.


One undesirable feature interaction occurs between the \signature\ and
\forwardmessages\ features: if Philippe signs a message and sends it to Sarah,
and Sarah forwards the message to an alternate address Sarah-2, then signature
verification may incorrectly interpret Sarah as the sender rather than Philippe
and fail to verify the message (Hall's interaction \#4).
%
A solution to this interaction is to embed the original sender's verification
information into the email header of the forwarded message so that it can be
used to verify the message, rather than relying solely on the message's
``from'' field.


Below, we show a variational query \Qsf\ that includes four variants
corresponding to whether \signature\ and \forwardmessages\ are enabled or not
independently. The information need for resolving the interaction is satisfied
by the first alternative of the outermost choice with condition
$\signature\wedge\forwardmessages$. The alternatives of the choices nested to
the right satisfy the information needs for when only \signature\ is enabled,
only \forwardmessages\ is enabled, or neither is enabled (\Qbasic). We don't
show the single-feature \Qsig\ query, but it is similar to other
single-feature queries shown above.
%
\begin{align*}
\Qsf &=  \signature\wedge\forwardmessages \\
&\quad\langle
  \vPrj[\rvalue,\forwardaddr,\mathit{emp1}.\issigned,\mathit{emp1}.\verificationkey]{\temp}, \\
&\quad\signature\langle\Qsig, 
\chc[\forwardmessages]{\Qforward,\Qbasic} \rangle\rangle \\
\temp &\leftarrow
  ((((\sigma_{\midCond} \messages) 
 \bowtie
%_{\messages.\midatt = \recipientinfo.\midatt}
 \recipientinfo) \\
& \bowtie_{\sender = \mathit{emp1}.\emailid}
    (\vRen[\mathit{emp1}]{\employees})) \\
& \bowtie_{\rvalue = \mathit{emp2}.\emailid}
    (\vRen[\mathit{emp2}]{\employees})) \bowtie \forwardmsg
%\temp &\leftarrow
%  ((((\sigma_{\midCond} \messages) \\
%& \bowtie_{\messages.\midatt = \recipientinfo.\midatt} \recipientinfo) \\
%& \bowtie_{\messages.\sender = \mathit{emp1}.\emailid}
%    (\vRen[\mathit{emp1}]{\employees})) \\
%& \bowtie_{\recipientinfo.\rvalue = \mathit{emp2}.\emailid}
%    (\vRen[\mathit{emp2}]{\employees}))\\
%& \bowtie_{\mathit{emp2}.\eid = \forwardmsg.\eid} \forwardmsg
\end{align*}
%
 The query $\Qsf$ also resolves another consequence of the interaction between
 these two features. This time Sam successfully verifies message $X$ and
 forwards it to Sam2 which changes the header in the system such that it states
 message $X$ has been successfully verified, thus, the message could be
 altered by hackers while it is being forwarded (Hall's interaction \#27). The
 system can use $\Qsf$ to generate the correct header in this scenario again.

%\begin{comment}
\noindent
%
Some feature interactions require more than one query to satisfy their
information need.
% due to VRA's limitation that values cannot be variational.
For example, assume both \encryption\ and \forwardmessages\ are enabled.
Philippe sends an encrypted email $X$ to Sarah; upon receiving it the message
is decrypted and forwarded it to Sarah-2 (Hall's interaction \#9). This
violates the intention of encrypting the message and the system should warn the
user.
%
Queries \Qef\ and $\Qef'$ satisfy the information need for this interaction
when a message is encrypted or unencrypted, respectively.
%
\begin{align*}
\small
\Qef &= \encryption\wedge\forwardmessages \\
  &\langle
    \projectRel{\rvalue}{(\selectRel{\midCond\wedge\isencrypted}{\messages})},
    \encryption\langle\Qencrypt, \\
  &\qquad \chc[\forwardmessages]{\Qforward,\Qbasic} \rangle \\
\OB{\Qef'} &= \encryption\wedge\forwardmessages
  \langle \temp, \encryption \langle \Qencrypt, \\
  &\qquad \chc[\forwardmessages]{\Qforward,\Qbasic} \rangle \rangle \\
% \end{align*}
% %
% \begin{align*}
\temp &\leftarrow
  \pi_{\rvalue,\forwardaddr,\subject,\body} (\sigma_{\midCond \wedge \neg\isencrypted} \\
  & (\mretable \bowtie_{\employees.\eid = \forwardmsg.\eid} \forwardmsg))
\end{align*}

\noindent
%
However, managing feature interactions is not necessarily complicated. Some
interactions simply require projecting more attributes from the corresponding
single-feature queries. For example, assume both \filtermessages\ and
\mailhost\ features are enabled. Philippe sends a message to a non-existant
user in a mailhost that he has filtered. The mailhost generates a non-delivery
notification and sends it to Philippe, but he never receives it since it is filtered out
(Hall's interaction \#26). The system can check the \issystemnotification\
attribute for the $\Qfilter$ query and decide whether to filter a message or
not. Therefore, we can resolve this interaction by extending the single-feature
query for \filtermessages\ to $\Qfilter'$.
%
\begin{align*}
\small
\OB{\Qfilter'} &= 
\projectRel {\sender, \rvalue, \suffix, \issystemnotification, \subject, \body} \temp\\
\temp &\leftarrow 
{\joinRel \mretable
\filtermsg
{\employees . \eid = \filtermsg . \eid }
}
\end{align*}
%\end{comment}
\noindent
%
Overall, for the 18 interaction queries we provide, 12 have 4 variants, 3 have
3 variants, 2 have 2 variants, and 1 has 1 variant.

%%\input{formulas/enronQs}

%\subsection{Employee Query Set}
\label{sec:emp-qs}

%\point{We adapt and adjust the queries from~\cite{prima08Moon}.}\\
%- there are two kinds of queries\\
%- give an example of each

For this use case, we have a set of existing plain queries to start from.
\cite{prima08Moon} \TODO{fix citet} provides 12 queries to evaluate the Prima schema evolution
system. We adapt these queries to fit our encoding of the employee VDB
described in \secref{emp-vdb}.
%
%We provide the queries in both the VRA format usable by VDBMS and as
%\cpp{ifdef} annotated SQL, as described in \secref{enron-qs}.%
%\footnote{\href{https://github.com/lambda-land/VDBMS/tree/master/usecases/time-employee/queries}{usecases/time-employee/queries}}
%
9 of these queries have one variant, 2 have two variants, and 1 has three
variants. 


Moon's queries are of two types: 6 retrieve data valid on a particular date
(corresponding to \vThree\ in our encoding), while 6 retrieve data valid on or
after that date (\vThree--\vFive\ in our encoding).
%
For example, one query expresses the intent ``return the salary of employee
number $10004$'' at a time corresponding to \vThree, which we encode:
%\[
\centerline{
\ensuremath{
Q_1 = \vPrj[\oatt{\salary}{\vThree}]{%
  \joinRelR{\selectRel{\empno=10004}{\empacct}}{\job}
           {\empacct.\titleatt = \job.\titleatt}}
%\]
}.}
% Note that the presence condition of the only attribute \salary\ determines the
% presence condition of the resulting table.
%
% In general and for simplicity, the shared part of presence conditions of
%  projected attributes is factored out and applied to 
%  the entire table. Assume the returned table as a result of
%  query has the schema $\left(\oatt {\vAtt_1} {\dimMeta \wedge \dimMeta_1}, 
%  \oatt {\vAtt_2} {\dimMeta \wedge \dimMeta_2} \right)$.
% The shared restriction can be factored out and applied
% to the entire table, i.e., $\left( \oatt {\vAtt_1} {\dimMeta_1},
% \oatt {\vAtt_2} {\dimMeta_2} \right)^\dimMeta$.
%
We encode the same intent, but for all times at or after \vThree\ as follows:
%
\begin{align*}
Q_2 &=  \vThree\vee\vFour\vee\vFive \langle \pi_\salary 
 (\vThree\vee\vFour\langle\\
&    \joinRelR{(\vSel[\empno=10004]{\empacct})}{\job}
             {}, 
%&\quad\qquad 
{\vSel[\empno=10004]{\empacct}}\rangle), \empRel\rangle
\end{align*}
%
There are a variety of ways we could have encoded both $Q_1$ and $Q_2$.
%
For $Q_1$ we could equivalently have embedded the projection in a choice,
$\chc[\vThree]{\vPrj[\salary]{(\ldots)},\empRel}$, however attaching the presence
condition to the only projected attribute determines the presence condition of
the resulting table and so achieves the same effect.
%
In $Q_2$ we use choices to structure the query since we have to project on a
different intermediate result for \vFive\ than for \vThree\ and \vFour.

% The feature expression $\vThree \vee \vFour \vee \vFive$
% determines the database variants to be inquired. 
% Since the schema of \empacct\ and \job\ tables are the same
% in variants \vThree\ and \vFour\ they both have the same 
% query. Note that one could move the condition 
% $\vThree \vee \vFour \vee \vFive$ to the projected attribute
% which results in $\VVal {\eq_2}$, however, this query is wrong 
% because the last alternative of the choice projects attribute \salary\
% from an empty relation which is incorrect. It is important to 
% understand that the behavior of an empty relation is exactly the
% same as its behavior in relational algebra and one
% should be careful of using it in operations such as projection,
% selection, and join.

% \begin{align*}
% \small
% \VVal {\eq_2} &= 
% \pi_{\oatt \salary {\vThree \vee \vFour \vee \vFive}}\\
% &\hspace{-15pt}
% (\vThree \vee \vFour\langle
% (\sigma_{\empno=10004} \empacct)
% %&\hspace{0pt}
% \bowtie_{\empacct . \titleatt = \job . \titleatt}
% \job\\
% &\hspace{20pt},
% \vFive\langle{\selectRel {\empno=10004} \empacct},
% \empRel
% \rangle
% \rangle)
% \end{align*}

As another example, the following query realizes the intent to ``return the
name of the manager of department d001'' during the time frame of
\vThree--\vFive:
%
%\begin{align*}
\ensuremath{ Q_3 = \vThree\vee\vFour\vee\vFive \langle
  \pi_{\name,\fname,\lname} }
%  \\
%&
\centerline{
\ensuremath{  
%\hspace{10pt}
%\qquad
(\joinRel{\chc[\vThree]{\empacct,\empbio}}{(\vSel[\deptno=\deptOne]{\dept})}
           {\empno=\managerno}),
  \empRel \rangle
  }.}
%\end{align*}
%
%\noindent
Note that even though the attributes \name, \fname, and \lname\ are not present
in all three of the variants corresponding to \vThree--\vFive, the VRA encoding
permits omitting presence conditions that can be completely determined by the
presence conditions of the corresponding relations or attributes in the
variational schema. So, $Q_3$ is equivalent to the following query in which the
presence conditions of the attributes from the variational schema are listed
explicitly in the projection:
%
%\begin{align*}
\ensuremath{
Q_3' = \vThree\vee\vFour\vee\vFive \langle
  \pi_{\oatt{\name}{\vThree\vee\vFour},\oatt{\fname}{\vFive},\oatt{\lname}{\vFive}} }
%  \\
%&
\centerline{
\ensuremath{
  (\joinRel{\chc[\vThree]{\empacct,\empbio}}{(\vSel[\deptno=\deptOne]{\dept})}
           {\empno=\managerno}),
  \empRel \rangle}.}
%\end{align*}
%
%
% \noindent
Allowing developers to encode variation in v-queries based on their
preference makes VRA more flexible and easy to use. 
Also, v-queries are statically type-checked to ensure that
the variation encoded in them does not conflict the variation encoded
in the v-schema. 

 %
% Finally, we want to briefly illustrate what queries look like in the
% \cpp{ifdef}-annotated SQL format that we distribute as a potentially more
% portable and easy-to-use format for other researchers. Below is the query $Q_3$
% in this format.
% 
% \begin{lstlisting}[language={}]
% #ifdef V3 || V4 || v5
%   #ifdef V3 || V4
% SELECT name
%   #else
% SELECT firstname, lastname
%   #endif
% FROM
%   #ifdef V3
%     empacct
%   #else
%     empbio
%   #endif
% JOIN (SELECT dept WHERE deptno="d001")
%   ON empno=manageno  
% \end{lstlisting}

% \begin{lstlisting}
% #ifdef v3
% SELECT salary
% FROM empacct JOIN job ON empacct.title=job.title
% WHERE empno=10004 
% #endif
% \end{lstlisting}

%\input{formulas/empQs}

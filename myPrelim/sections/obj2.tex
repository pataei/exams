\subsection{Design and implement a database framework
%a query language and implement a database management 
%system 
that accommodate identified variations}
\label{sec:ro2}

Having an encoding that represent variation we need to incorporate it within the 
database and the query language to allow explicit storing and manipulation of 
variation in a database. Objective 2 aims to design and implement a database framework
that considers variation as a first-class citizen.
% for a variational
%database and variational query language and implement them as 
%a variational database management system that allows users to interact with a
%variational database. 
\tabref{ro2} presents individual research questions we need
to answer for this objective. 

\begin{table}
\caption{Objective 2 research questions.}
\label{tab:ro2}
\centering
\begin{tabularx}{\textwidth}{X}
\toprule
 \textbf{Objective 2: Design and implement a database framework
%  a query language and implement a database management 
%system 
that accommodates identified variations}
\tabularnewline
\midrule
RQ2.1: How should variation in form of feature expression be incorporated in the database as a first-class citizen? (\dbpl, \poly)
\tabularnewline[0.2cm]
RQ2.2: What are appropriate query languages to interact with a database that accounts for variation explicitly? And how should variation in form of feature expression be incorporated in the query language? (\dbpl, \poly)
\tabularnewline[0.2cm]
RQ2.3: Having a theoretical database framework that accounts for variation explicitly, how 
should we implement a database management system that uses that framework? (In progress)
\tabularnewline
\bottomrule
\end{tabularx}
\end{table}


\begin{comment}
* annotations and choices
\end{comment}

For RQ2.1 we annotate elements of a database with feature expression,
as introduced in \secref{encode-var}. 
We use annotated elements both in the schema and content.
Within a schema we allow attributes and relations to exist 
conditionally based on the feature expression assigned to them (\secref{vsch}).
At the content level, we annotate each tuple with a feature expression, indicating when the tuple 
is present (\secref{vtab}). 

\TODO{add def of vdb and what it is conceptually}

\TODO{intro vset first}

\begin{comment}
The variational nature of a VDB requires a query language that
accounts for variation directly.
To express and represent variation in queries,
we incorporate choice calculus~\cite{Walk13thesis, EW11tosem}  into a 
structured query language. We formally define 
\emph{variational relational algebra (VRA)} in \secref{vrel-alg}
as our algebraic query language.
A query written in VRA is called a \emph{variational query (v-query)};
when it is clear from context we use query and v-query interchangeably. 
A v-query typically conveys the same intent over several 
relational database variants, however, a single v-query is also capable of capturing different 
intents over different database variants.
Consequently, the expressiveness of v-queries may cause them to be 
more complicated than relational queries, discussed in \secref{type-sys}. 
Hence, we introduce a 
\emph{type system} for VRA that statically checks if a 
v-query conforms to the underlying v-schema and encoded variability within the VDB.
Finally, we close out this section by providing a set of rules in \secref{var-min} 
for reducing a query's variation.

\end{comment}

For RQ2.2 ...
\TODO{ref to v-rel alg}

\begin{comment}
\end{comment}

For RQ2.3 ...
\TODO{ref impl}

\subsection{Variational Set}
\label{sec:vlist-vset}

%\point{vset.}
A \emph{variational set (v-set)} $\vset = \setDef {\annot [\dimMeta_1] {\elem_1},\ldots, \annot [\dimMeta_n] {\elem_n}}$ 
is a set of annotated elements~\cite{EWC13fosd,Walk14onward,ATW17dbpl}.
% where the presence condition of elements is satisfiable~\cite{EWC13fosd,Walk14onward,vdb17ATW}. 
%
Conceptually, a \emph{variational set} represents many different plain sets
that can be generated by enabling or disabling features
and including only the elements whose feature expressions evaluate to \t.
We typically omit the presence condition \prog{true} in a variational set,
e.g., the v-set 
$\setDef {\annot [\A] 2, \annot [\B] 3, 4}$
represents four plain sets under different configurations. These plain
sets can be generated by \emph{configuring} the variational set with a
given configuration: 
\setDef {2,3,4}, when $\A$ and $\B$
are enabled; \setDef {2,4}, when $\A$ is enabled but $\B$ is disabled;
\setDef {3,4}, when $\B$ is enabled but $\A$ is disabled;
and \setDef {4}, when both $\A$ and $\B$ are disabled.
%
%We indicate variational sets of elements $\elem \in \mathbf{\elemSet}$ with \elemSet.
%A variational set is conceptually a function from a configuration of its
%features to the corresponding plain set. 
%We typically omit the feature
%expression \prog{true} in a variational set, for example, in the
%variational set $\{5,6^{f_1}\}$, the feature expression for the value $5$ is
%implicitly \prog{true}, and so the element is included in both variants:
%$\{5,6\}$ when feature $f_1$ is enabled and $\{5\}$ when feature $f_1$ is
%disabled.
Note that elements with presence condition \prog{false} can be omitted
from the v-set, e.g., the v-set \ensuremath{\setDef {\annot [\f] {1}}} is 
equivalent to an empty v-set.
For simplicity and to comply with database notational conventions
we drop the brackets of a variational set when used in database
schema definitions and queries.
%for defining 
%variational relation schemas and the variational attribute set to be projected in a query.

%\point{annotated vset.}
A variational set itself can also be annotated with a feature expression.
%
%An \emph{annotated variational set} 
$\annot \vset = \setDef {\annot [\dimMeta_1] {\elem_1},\ldots,\annot [\dimMeta_n] {\elem_n}}^\dimMeta$ is an
\emph{annotated v-set}.
% that it is annotated itself by a \emph{feature expression} \dimMeta.
%We denote an annotated variational set of elements $\elem \in \mathbf{\elemSet}$ with
%\annot \elemSet.
Annotating a v-set with the feature expression \dimMeta\ 
restricts the condition under which its elements are present, i.e., it forces
elements' presence conditions to be more specific. This restriction 
can be applied to all elements of the set by \emph{pushing} in the
feature expression \dimMeta, done by the operation
%\NOTE{
\ensuremath{
\pushIn {\setDef {\annot [\dimMeta_1] {\elem_1},\ldots,\annot [\dimMeta_n] {\elem_n}}^\dimMeta}
= 
%\annot {\setDef{\annot [\dimMeta_i] \elem_i \myOR \sat {\dimMeta_i \wedge \dimMeta}, 1 \leq i \leq n}}}.}
\setDef {\annot [\dimMeta_1 \wedge \dimMeta] {\elem_1},\ldots, \annot [\dimMeta_n \wedge \dimMeta] {\elem_n}}
}.
%This restriction
%can be captured by the property:
%$\setDef {\annot [\dimMeta_1] {\elem_1} ,\ldots, \annot [\dimMeta_n] {\elem_n}}^\dimMeta
%\equiv 
%\setDef {\annot [\dimMeta_1 \wedge \dimMeta] {\elem_1},\ldots, \annot [\dimMeta_n \wedge \dimMeta] {\elem_n}}
%$.
%
For example, the annotated v-set
$\{\annot [\A] 2, \annot [\neg \B] 3, 4, \annot [\C] 5\}^{\A \wedge \B}$
indicates that all the elements of the set can only exist
when both $\A$ and $\B$ are enabled. Thus, pushing in the set's feature expression
results in
$\{\annot [\A \wedge \B] 2,\annot [\A \wedge \B] 4,\annot [\A \wedge \B \wedge \C] 5\}$. The element $3$ is dropped 
%from the set 
since 
\ensuremath{\neg \sat {\neg \B \wedge (\A \wedge \B)}},
where
\ensuremath{
\getPC 3 = \neg \B \wedge (\A \wedge \B)}.
%its presence condition is unsatisfiable, i.e., $\neg \sat {\neg \fName_2 \wedge (\fName_1 \wedge \fName_2)}$.
%%

We provide some operations over v-sets. Intuitively, these operations should 
behave such that configuring the result of applying a variational set operation
should be equivalent to applying the plain set operation on the configured 
input v-sets. 
 
%These operations are vastly used
%in \secref{type-sys}.

%
\begin{definition}[V-set union]
\label{def:vset-union}
The \emph {union} of two v-sets is the union of their elements with the disjunction of 
presence conditions if an element exists in both v-sets:
\ensuremath{
\vset_1 \cup \vset_2 = \setDef {\annot [\dimMeta_1] \elem \myOR \annot [\dimMeta_1] \elem \in \vset_1, \annot [\dimMeta_2] \elem \not \in \vset_2}
\cup \setDef {\annot [\dimMeta_2] \elem \myOR \annot [\dimMeta_2] \elem \in \vset_2, \annot [\dimMeta_1] \elem  \not \in \vset_1}
\cup \setDef {\annot [\dimMeta_1 \vee \dimMeta_2] \elem \myOR 
\annot [\dimMeta_1] \elem \in \vset_1, \annot [\dimMeta_2] \elem \in \vset_2}
}.
For example, \\
\ensuremath{
\setDef {2,\annot [\dimMeta_1] 3, \annot [\dimMeta_1] 4} \cup \setDef {\annot [\dimMeta_2] 3, \annot [\neg \dimMeta_1] 4} = \setDef {2, \annot [\dimMeta_1 \vee \dimMeta_2] 3, 4}
}.
\end{definition}

% 
% is needed for the implicitly-type lang:
\begin{definition}[V-set intersection]
\label{def:vset-intersect}
The \emph{intersection} of two v-sets is a v-set of their shared elements
annotated with the conjunction of their presence conditions, i.e., 
\ensuremath{
\vset_1 \cap \vset_2 = \setDef {
\annot [\dimMeta_1 \wedge \dimMeta_2 ]\elem \myOR
\annot [\dimMeta_1] \elem \in \vset_1, \annot [\dimMeta_2] \elem \in \vset_2,
\sat {\dimMeta_1 \wedge \dimMeta_2}
}
}.
For example, \ensuremath{
\setDef {2, \annot [\A] 3, \annot [\neg \B] 4} \cap
\pushIn {\annot [\B] {\setDef{2,3,4,5}}} =
\setDef{\annot [\B] 2, \annot [\A \wedge \B] 3}
}.
\end{definition}

\begin{definition} [V-set cross product]
\label{def:vset-cross}
The \emph{cross product} of two v-sets is a pair of every two elements of 
them annotated with the conjunction of their presence conditions.
\ensuremath{
\vset_1 \times \vset_2 = \setDef{
\annot [\dimMeta_1 \wedge \dimMeta_2] {(\elem_1, \elem_2)} \myOR
\annot [\dimMeta_1] \elem_1 \in \vset_1, \annot [\dimMeta_2] \elem_2 \in \vset_2
%\vset_1 \cap \vset_2 = \setDef \
}
}
%
\end{definition}

\begin{definition} [V-set equivalence]
\label{def:vset-eq}
Two v-sets are \emph{equivalent}, denoted by
\ensuremath{\vset_1 \equiv \vset_2}, iff
\ensuremath{
\forall \annot  \elem \in (\vset_1 \cup \vset_2). 
\annot [\dimMeta_1] \elem \in \vset_1, \annot [\dimMeta_2] \elem \in \vset_2, 
\dimMeta_1 \equiv \dimMeta_2},
i.e., they both cover the same set of elements and the presence conditions
of elements from the two v-sets are equivalent.
\end{definition}

%
\begin{definition} [V-set subsumption]
\label{def:vset-subsumption}
The v-set \ensuremath{\vset_1} \emph {subsumes} the v-set
\ensuremath{\vset_2}, $\subsume {\vset_2} {\vset_1}$, iff
\ensuremath{ \forall \annot [\dimMeta_2] \elem \in \vset_2.
\annot [\dimMeta_1] \elem \in \vset_1, 
%\neg \sat {\dimMeta_2 \wedge \neg \dimMeta_1}
\sat {\dimMeta_2 \wedge  \dimMeta_1}
},
i.e., all elements in $\vset_2$ also exist in $\vset_1$ 
s.t. the element is valid in a shared configuration between the v-sets.
For example, 
\ensuremath{
 \subsume {\pushIn {\annot [\A] {\setDef {2,3}}}} {\setDef {2, \annot [\A \vee \B] 3, 4}}},
however, 
\ensuremath{
 \nsubsume {\pushIn {\annot [\A] {\setDef {2,3}}}} {\setDef {2, \annot [\neg \A \wedge \B] 3}}}
and
\ensuremath{
\nsubsume {\setDef {\annot [\A] 2,\annot [\A] 3, 4}} {\setDef {2, \annot [\A \wedge \B] 3}}}.
\end{definition}

%\begin{definition} [V-set explicit subsumption]
%\dropit{drop this for vldb submission. remember you need it for popl.}
%\label{def:vset-strict-subsumption}
%The v-set \ensuremath{\vset_1} \emph {explicitly subsumes} the v-set
%\ensuremath{\vset_2}, $\subsumeExpl {\vset_2} {\vset_1}$, iff
%\ensuremath{ \forall \annot [\dimMeta_2] \elem \in \vset_2.
%\annot [\dimMeta_1] \elem \in \vset_1, 
%\neg \sat {\dimMeta_2 \wedge \neg \dimMeta_1}
%},
%i.e., all elements in $\vset_2$ also exist in $\vset_1$ 
%s.t. its presence condition in \ensuremath{\vset_2} is more specific than 
%its presence condition in \ensuremath{\vset_1}, captured by 
%\ensuremath{\neg \sat {\dimMeta_2 \wedge \neg \dimMeta_1}}
%which could also be defined as 
%\ensuremath{
%\nexists \config \in \confSet . \fSem {\dimMeta_1} = \t , \fSem {\dimMeta_2} = \f.
%%i.e. in set theory:
%% \dimMeta_2 \subset \dimMeta_1
%%\dimMeta_2 - \dimMeta_1 = \emptyset
%%i.e.
%%\dimMeta_2 \cap \bar{\dimMeta_1} = \emptyset 
%}
%\end{definition}





\begin{table}
\caption[Examples of encoding variation at the schema level]{The relational tables of \empbio\ for variants that enable one of the features \vThree, \vFour, or \vFive\ and 
the variational relation \empbio\ that encompasses
the three variants of the plain table \empbio\ without accounting for variation at the content level.
Note that data from earlier variants like \setDef \vThree\ is propagated to the later variants like \setDef \vFour\ and \setDef \vFive.}
\label{tab:empbio-tab}
\centering
\small
%\footnotesize
%\scriptsize
\begin{subtable}[t]{\textwidth}
\centering
%\footnotesize
\scriptsize
\caption{The relational table of \empbio\ for variants that only enables the feature \vThree\ out of
the features \vOne--\vFive. The relation schema is captured by the name of the relation and its attributes.}
\label{tab:empbio-v3}
\begin{tabular} {c | l l l}
%\hline
\multirow{2}{*}{\empbio} & \empno & \sex & \birthdate\\
\cline{2-4}
%\hline
 &12001 & F& 1960-11-06\\
  &12002 & M& 1961-04-15\\
   &12003 & M& 1958-07-27\\
   &\ldots & \ldots & \ldots \\
\arrayrulecolor{white}\hline
\end{tabular}
\end{subtable}

\medskip
\medskip
\medskip
\begin{subtable}[t]{\textwidth}
\centering
%\footnotesize
\scriptsize
\caption{The relational table of \empbio\ for variants that only enables the feature \vFour\ out of
the features \vOne--\vFive.}
%The relation schema is captured by the name of the relation and attributes.}
\label{tab:empbio-v4}
\begin{tabular} {c | l l l l}
\multirow{2}{*}{\empbio}  & \empno & \sex & \birthdate & \name\\
\cline{2-5}
 &12001 & F& 1960-11-06 & Ulf Hofstetter\\
  &12002 & M& 1961-04-15 &Luise McFarlan \\
   &12003 & M& 1958-07-27 & Shir DuCasse \\
 &80001 & M & 1956-09-30 & Nagui Merli \\
 & 80002 & M & 1963-04-25 & Mayuko Meszaros\\
 & 80003 & F & 1960-10-26 & Theirry Viele\\
 & \ldots & \ldots & \ldots & \ldots \\
\arrayrulecolor{white}\hline
\end{tabular}
\end{subtable}

\medskip
\medskip
\medskip
\begin{subtable}[t]{\textwidth}
\centering
%\footnotesize
\scriptsize
\caption{The relational table of \empbio\ for variants that only enables the feature \vFive\ out of
the features \vOne--\vFive.}
%The relation schema is captured by the name of the relation and attributes.}
\label{tab:empbio-v5}
\begin{tabular} {c | l l l l l}
\multirow{2}{*}{\empbio}  & \empno & \sex & \birthdate & \fname & \lname\\
\cline{2-6}
 &12001 & F& 1960-11-06 &Ulf & Hofstetter\\
  &12002 & M& 1961-04-15 &Luise & McFarlan \\
   &12003 & M& 1958-07-27 & Shir & DuCasse \\
 &80001 & M & 1956-09-30 & Nagui & Merli \\
 & 80002 & M & 1963-04-25 & Mayuko & Meszaros\\
 & 80003 & F & 1960-10-26 & Theirry & Viele\\
 & 200001 & M & 1960-01-11 & Selwyn & Koshiba \\
 & 200002 & M & 1957-09-10 & Bedrich & Markovitch\\
 & 200003 & F & 1961-02-07 & Pascal & Benzmuller \\
 & \ldots & \ldots & \ldots & \ldots & \ldots\\
\arrayrulecolor{white}\hline
\end{tabular}
\end{subtable}

\medskip
\medskip
\medskip
\begin{subtable}[t]{\textwidth}
\centering
%\footnotesize
\tiny
\caption{The variational relation of \empbio\ without accounting for variation at the content level.
The relation schema is captured by the name of the relation and attributes in addition to their presence
conditions which are colored blue. }
%This table is present under a presence condition that applies
%to the entire database $\dimMeta_{\mathit{mot}}$ which is given in \exref{vsch-mot}.}
\label{tab:empbio-vsch}
\arrayrulecolor{blue}
\begin{tabular} {c !{\color{black}\vrule} l l l l l l }
%\hline
%\hhline{-==}
\tiny {\textcolor{blue}{$\vThree \vee \vFour \vee \vFive$} }& \tiny{\textcolor{blue}{\texttt{true}}} & \tiny{\textcolor{blue}{\texttt{true}}} & \tiny{\textcolor{blue}{\texttt{true}}} & \tiny {\textcolor{blue}{$\vFour \wedge \neg \vThree \wedge \neg \vFive$}} & \tiny {\textcolor{blue}{$\vFive \wedge \neg \vThree \wedge \neg \vFour$}} & \tiny {\textcolor{blue}{$\vFive \wedge \neg \vThree \wedge \neg \vFour$}}\\
\arrayrulecolor{blue}\hdashline
\multirow{2}{*}{\empbio}  & \empno & \sex & \birthdate & \name & \fname & \lname\\
\arrayrulecolor{black}\cline{2-7}
 &12001 & F& 1960-11-06 & Ulf Hofstetter & Ulf & Hofstetter \\
  &12002 & M& 1961-04-15 & Luise McFarlan & Luise & McFarlan \\
   &12003 & M& 1958-07-27 & Shir DuCasse & Shir & DuCasse \\
 &80001 & M & 1956-09-30 & Nagui Merli & Nagui & Merli \\
 & 80002 & M & 1963-04-25 & Mayuko Meszaros & Mayuko & Meszaros\\
 & 80003 & F & 1960-10-26 & Theirry Viele & Theirry & Viele \\
 & 200001 & M & 1960-01-11 & Selwyn Koshiba & Selwyn & Koshiba \\
 & 200002 & M & 1957-09-10 & Bedrich Markovitch & Bedrich & Markovitch\\
 & 200003 & F & 1961-02-07 & Pascal Benzmuller & Pascal & Benzmuller  \\
 & \ldots & \ldots & \ldots & \ldots & \ldots & \ldots\\
\arrayrulecolor{white}\hline
%\job & \titleatt & \salary\\
%\cline{2-3}
%& Assistant Engineer & 61594\\
%& Senior Engineer & 96646\\
%& \ldots & \ldots \\
%& Staff & 77935\\
%& Technique Leader & 58345
\end{tabular}
\end{subtable}
\end{table}

%\begin{table}
%\caption[Examples of encoding variation at the schema level]{The relational tables of \empbio\ for variants that enable one of the features \vThree, \vFour, or \vFive\ and 
%the variational relation \empbio\ that encompasses
%the three variants of the plain table \empbio\ without accounting for variation at the content level.
%}
%\label{tab:empbio-sch}
%\centering
%\small
%\footnotesize
%%\scriptsize
%\begin{subtable}[t]{\textwidth}
%\centering
%\caption{The relational table of \empbio\ for variants that only enables the feature \vThree\ out of
%the features \vOne--\vFive. The relation schema is captured by the name of the relation and its attributes.}
%\label{tab:empbio-v3}
%\begin{tabular} {c | l l l}
%%\hline
%\multirow{2}{*}{\empbio} & \empno & \sex & \birthdate\\
%\cline{2-4}
%%\hline
% &12001 & F& 1960-11-06\\
%  &12002 & M& 1961-04-15\\
%   &12003 & M& 1958-07-27\\
%   &\ldots & \ldots & \ldots \\
%\arrayrulecolor{white}\hline
%\end{tabular}
%\end{subtable}
%
%\medskip
%\medskip
%\medskip
%\begin{subtable}[t]{\textwidth}
%\centering
%\footnotesize
%\caption{The relational table of \empbio\ for variants that only enables the feature \vFour\ out of
%the features \vOne--\vFive.}
%%The relation schema is captured by the name of the relation and attributes.}
%\label{tab:empbio-v4}
%\begin{tabular} {c | l l l l}
%\multirow{2}{*}{\empbio}  & \empno & \sex & \birthdate & \name\\
%\cline{2-5}
% &80001 & M & 1956-09-30 & Nagui Merli \\
% & 80002 & M & 1963-04-25 & Mayuko Meszaros\\
% & 80003 & F & 1960-10-26 & Theirry Viele\\
% & \ldots & \ldots & \ldots & \ldots \\
%\arrayrulecolor{white}\hline
%\end{tabular}
%\end{subtable}
%
%\medskip
%\medskip
%\medskip
%\begin{subtable}[t]{\textwidth}
%\centering
%\caption{The relational table of \empbio\ for variants that only enables the feature \vFive\ out of
%the features \vOne--\vFive.}
%%The relation schema is captured by the name of the relation and attributes.}
%\label{tab:empbio-v5}
%\footnotesize
%\begin{tabular} {c | l l l l l}
%\multirow{2}{*}{\empbio}  & \empno & \sex & \birthdate & \fname & \lname\\
%\cline{2-6}
% & 200000 & M & 1960-01-11 & Selwyn & Koshiba \\
% & 200001 & M & 1957-09-10 & Bedrich & Markovitch\\
% & 200002 & F & 1961-02-07 & Pascal & Benzmuller \\
% & \ldots & \ldots & \ldots & \ldots & \ldots\\
%\arrayrulecolor{white}\hline
%\end{tabular}
%\end{subtable}
%
%\medskip
%\medskip
%\medskip
%\begin{subtable}[t]{\textwidth}
%\centering
%%\footnotesize
%\scriptsize
%%\tiny
%\caption{The variational relation of \empbio\ without accounting for variation at the content level.
%The relation schema is captured by the name of the relation and attributes in addition to their presence
%conditions which are colored blue. }
%%This table is present under a presence condition that applies
%%to the entire database $\dimMeta_{\mathit{mot}}$ which is given in \exref{vsch-mot}.}
%\label{tab:empbio-vsch}
%\begin{tabular} {c | l l l l l l l}
%%\hline
%%\hhline{-==}
%\textcolor{blue}{$\vThree \vee \vFour \vee \vFive$} & \textcolor{blue}{\texttt{true}} & \textcolor{blue}{\texttt{true}} & \textcolor{blue}{\texttt{true}} & \textcolor{blue}{$\vFour \wedge \neg \vThree \wedge \neg \vFive$} & \textcolor{blue}{$\vFive \wedge \neg \vThree \wedge \neg \vFour$} & \textcolor{blue}{$\vFive \wedge \neg \vThree \wedge \neg \vFour$}\\
%\arrayrulecolor{blue}\hdashline
%\multirow{2}{*}{\empbio}  & \empno & \sex & \birthdate & \name & \fname & \lname\\
%\arrayrulecolor{black}\cline{2-7}
% &12001 & F& 1960-11-06 & & & \\
%  &12002 & M& 1961-04-15 & & & \\
%   &12003 & M& 1958-07-27 & & & \\
% &80001 & M & 1956-09-30 & Nagui Merli & & \\
% & 80002 & M & 1963-04-25 & Mayuko Meszaros & & \\
% & 80003 & F & 1960-10-26 & Theirry Viele & & \\
% & 200001 & M & 1960-01-11 & & Selwyn & Koshiba \\
% & 200002 & M & 1957-09-10 & & Bedrich & Markovitch \\
% & 200003 & F & 1961-02-07 & & Pascal & Benzmuller  \\
% & \ldots & \ldots & \ldots & \ldots & \ldots & \ldots \\
%\arrayrulecolor{white}\hline
%%\job & \titleatt & \salary\\
%%\cline{2-3}
%%& Assistant Engineer & 61594\\
%%& Senior Engineer & 96646\\
%%& \ldots & \ldots \\
%%& Staff & 77935\\
%%& Technique Leader & 58345
%\end{tabular}
%\end{subtable}
%
%\end{table}

\section{Variational Table}
\label{sec:vtab}

\TODO{vtab}

\subsection{Variational Table Configuration}
\label{sec:vtabconf}

\TODO{vtab configuration}

\input{sections/vrelAlg}
\subsubsection{VDBMS Architecture}
\label{sec:impl}

\TODO{adjust for proposal}

%-haskell
%-any dbms
%-manually generate vdb explained in expr
%-provide v-sch
%-input: vdb, vsch(==> valid configs), vq
%-arch fig
%-explain arch
\TODO{the annotation of tuples is encoded as an attribute in implementation.}
%\point{impl in haskell on top of any dbms using haskell's type class.}
%To interact with VDBs using v-queries, we implement 
%\emph{Variational Database Management System (VDBMS)}.
%VDBMS is implemented in Haskell. VDBMS sit on 
%top of any DBMS that the user desires and used to store their data 
%%\arashComment{I did not find any explanation on how v-tables are stored in an RDBMS.} 
%%\resp{it is exactly implemented as formalized in v-table section.}
%%\responded
%in form of v-tables, explained in \secref{vtab}.
%%To acquire an extensible system we implement 
%To support running VDBMS with multiple different plain relational DBMS backends,
%we provide
%a shared interface
%for connecting to and inquiring information from a DBMS and
%instantiate it for different database engines such as PostgreSQL and
%MySQL. 
%%\rewrite{any dbms that has a library in haskell that has a function
%%that returns the result to the user. eg that doesn't satisfy this is 
%%database.sqlite3. } --> The following addresses this:
%An expert can extend VDBMS to another database engine by
%writing methods for connecting to and querying from the database.

%\input{sections/implVar}
%\point{vdb and vschema and config (bottom of fig).}
\textbf{VDBMS architecture:}
\figref{arch} shows the architecture of VDBMS and its modules.
For now, we assume a VDB and its v-schema are generated by an 
expert and are stored in a DBMS, we return to generation of VDBs in 
\secref{exp-disc}. A VDB can be \emph{configured} to its pure relational 
database variants, if desired by a user, by providing the configuration
of the desired variant, \figref{vdb-conf}.
For example, a SPL developer configures a VDB to produce 
software and its database for a client.
%To configure a VDB, VDBMS requires a list of valid configurations.
%Remember that the feature model is a feature expression that 
%encodes all valid configurations. Hence, solving the feature model
%by a SAT solver results in the list of valid configurations.

%\point{flow of vq in vdbms.}
Given a VDB and its v-schema, a user inputs a v-query \vQ\ to VDBMS.
%
First, \vQ\ is checked by the \emph{type system} to determine if it is invalid, explained in 
%First, \vQ\ is type-checked by the VRA type system introduced in 
\secref{type-sys}. 
If so, the user gets errors explaining what part of the 
query violated the v-schema.
%, shown in \exref{q-violate-sch}.
%\moredet{maybe give an ex of an error user will see! ref to ex of error given
%in \secref{type-sys}}
Otherwise, 
\vQ\ is constrained by the schema,
defined in \secref{constrain},
to ensure variation-preserving property w.r.t. v-schema throughout the execution flow of v-query 
in the system and then
%
it is passed to the \emph{variation minimization} module, introduced in 
\secref{var-min}, to minimize the variation of \vQ\ and apply
relational algebra optimization rules. 
%
The optimized query is then sent to the \emph{generator} module where
SQL queries are generated from v-queries, \secref{apps} provides three
approaches for this.
\exref{q-flow} in \appref{sql-gen} demonstrates the flow of a v-query through
VDBMS.

\begin{comment}
To generate runnable queries w.r.t. the underlying DBMS,
the minimized query \ensuremath {\VVal \vQ} is passed to 
the \emph{translate to RA} module that could use either 
configuring or grouping of v-queries, explained in \secref{vra-sem},
to generate RA queries. The generated 
queries are then sent to the \emph{SQL generator} module which generates
SQL queries in various ways from the relational algebra queries, explained
in \secref{sql-gen}.
%\moredet{in app have an ex of all this happening!}
\end{comment}

%\point{vtab builder.}
Having generated SQL queries, they are now run over the underlying 
VDB (stored in a DBMS desired by the user). The result could be either 
a v-table or a list of v-tables, depending on the approach chosen in 
the translator to RA and SQL generator modules. The v-table(s) is passed
to the \emph{v-table builder}
%\dropit{could drop \secref{vtab-build} and explain it here!}
%explained in \secref{vtab-build}, 
to create one v-table that filters out 
duplicate and invalid tuples, shrinks presence conditions, and 
eventually, returns the final v-table to the user.

\begin{figure}
\includegraphics[width = \linewidth] {figs/arch7.pdf}
\caption{VDBMS architecture and execution flow of a v-query. 
The dotted double-line from v-query to pushing v-schema module
indicates the dependency of passing the v-query to this module
only if it is valid. 
The dashed gray arrows with diamond heads demonstrate
an option for the flow of input. 
%We examine taking different routes
%to evaluate a v-query, resulting in various approaches in \secref{apps}.
The blue filled arrows track the data flow, the green hollow arrows 
indicate an input to a module.}
\label{fig:arch}
\end{figure}



%\input{sections/sqlGen}


\subsection{Formula Choice Calculus}
\label{sec:fcc}

The formula
choice calculus~\cite{HW16fosd} is a formal language for representing
variation.

%%\point{vset.}
%A \emph{variational set (v-set)} $\vset = \setDef {\annot [\dimMeta_1] {\elem_1},\ldots, \annot [\dimMeta_n] {\elem_n}}$ 
%is a set of annotated elements~\cite{EWC13fosd,Walk14onward,vdb17ATW}.
%% where the presence condition of elements is satisfiable~\cite{EWC13fosd,Walk14onward,vdb17ATW}. 
%%
%Conceptually, a \emph{variational set} represents many different plain sets
%that can be generated by enabling or disabling features
%and including only the elements whose feature expressions evaluate to \t.
%We typically omit the presence condition \prog{true} in a variational set,
%e.g., the v-set 
%$\setDef {\annot [\A] 2, \annot [\B] 3, 4}$
%represents four plain sets under different configurations. These plain
%sets can be generated by \emph{configuring} the variational set with a
%given configuration: 
%\setDef {2,3,4}, when $\A$ and $\B$
%are enabled; \setDef {2,4}, when $\A$ is enabled but $\B$ is disabled;
%\setDef {3,4}, when $\B$ is enabled but $\A$ is disabled;
%and \setDef {4}, when both $\A$ and $\B$ are disabled.
%%
%%We indicate variational sets of elements $\elem \in \mathbf{\elemSet}$ with \elemSet.
%%A variational set is conceptually a function from a configuration of its
%%features to the corresponding plain set. 
%%We typically omit the feature
%%expression \prog{true} in a variational set, for example, in the
%%variational set $\{5,6^{f_1}\}$, the feature expression for the value $5$ is
%%implicitly \prog{true}, and so the element is included in both variants:
%%$\{5,6\}$ when feature $f_1$ is enabled and $\{5\}$ when feature $f_1$ is
%%disabled.
%Note that elements with presence condition \prog{false} can be omitted
%from the v-set, e.g., the v-set \ensuremath{\setDef {\annot [\f] {1}}} is 
%equivalent to an empty v-set.
%For simplicity and to comply with database notational conventions
%we drop the brackets of a variational set when used in database
%schema definitions and queries.
%%for defining 
%%variational relation schemas and the variational attribute set to be projected in a query.
%
%%\point{annotated vset.}
%A variational set itself can also be annotated with a feature expression.
%%
%%An \emph{annotated variational set} 
%$\annot \vset = \setDef {\annot [\dimMeta_1] {\elem_1},\ldots,\annot [\dimMeta_n] {\elem_n}}^\dimMeta$ is an
%\emph{annotated v-set}.
%% that it is annotated itself by a \emph{feature expression} \dimMeta.
%%We denote an annotated variational set of elements $\elem \in \mathbf{\elemSet}$ with
%%\annot \elemSet.
%Annotating a v-set with the feature expression \dimMeta\ 
%restricts the condition under which its elements are present, i.e., it forces
%elements' presence conditions to be more specific. This restriction 
%can be applied to all elements of the set by \emph{pushing} in the
%feature expression \dimMeta, done by the operation
%%\NOTE{
%\ensuremath{
%\pushIn {\setDef {\annot [\dimMeta_1] {\elem_1},\ldots,\annot [\dimMeta_n] {\elem_n}}^\dimMeta}
%= 
%%\annot {\setDef{\annot [\dimMeta_i] \elem_i \myOR \sat {\dimMeta_i \wedge \dimMeta}, 1 \leq i \leq n}}}.}
%\setDef {\annot [\dimMeta_1 \wedge \dimMeta] {\elem_1},\ldots, \annot [\dimMeta_n \wedge \dimMeta] {\elem_n}}
%}.
%%This restriction
%%can be captured by the property:
%%$\setDef {\annot [\dimMeta_1] {\elem_1} ,\ldots, \annot [\dimMeta_n] {\elem_n}}^\dimMeta
%%\equiv 
%%\setDef {\annot [\dimMeta_1 \wedge \dimMeta] {\elem_1},\ldots, \annot [\dimMeta_n \wedge \dimMeta] {\elem_n}}
%%$.
%%
%For example, the annotated v-set
%$\{\annot [\A] 2, \annot [\neg \B] 3, 4, \annot [\C] 5\}^{\A \wedge \B}$
%indicates that all the elements of the set can only exist
%when both $\A$ and $\B$ are enabled. Thus, pushing in the set's feature expression
%results in
%$\{\annot [\A \wedge \B] 2,\annot [\A \wedge \B] 4,\annot [\A \wedge \B \wedge \C] 5\}$. The element $3$ is dropped 
%%from the set 
%since 
%\ensuremath{\neg \sat {\neg \B \wedge (\A \wedge \B)}},
%where
%\ensuremath{
%\getPC 3 = \neg \B \wedge (\A \wedge \B)}.
%%its presence condition is unsatisfiable, i.e., $\neg \sat {\neg \fName_2 \wedge (\fName_1 \wedge \fName_2)}$.
%%%
%
%We provide some operations over v-sets. Intuitively, these operations should 
%behave such that configuring the result of applying a variational set operation
%should be equivalent to applying the plain set operation on the configured 
%input v-sets. 
% 
%%These operations are vastly used
%%in \secref{type-sys}.
%
%%
%\begin{definition}[V-set union]
%\label{def:vset-union}
%The \emph {union} of two v-sets is the union of their elements with the disjunction of 
%presence conditions if an element exists in both v-sets:
%\ensuremath{
%\vset_1 \cup \vset_2 = \setDef {\annot [\dimMeta_1] \elem \myOR \annot [\dimMeta_1] \elem \in \vset_1, \annot [\dimMeta_2] \elem \not \in \vset_2}
%\cup \setDef {\annot [\dimMeta_2] \elem \myOR \annot [\dimMeta_2] \elem \in \vset_2, \annot [\dimMeta_1] \elem  \not \in \vset_1}
%\cup \setDef {\annot [\dimMeta_1 \vee \dimMeta_2] \elem \myOR 
%\annot [\dimMeta_1] \elem \in \vset_1, \annot [\dimMeta_2] \elem \in \vset_2}
%}.
%For example, \\
%\ensuremath{
%\setDef {2,\annot [\dimMeta_1] 3, \annot [\dimMeta_1] 4} \cup \setDef {\annot [\dimMeta_2] 3, \annot [\neg \dimMeta_1] 4} = \setDef {2, \annot [\dimMeta_1 \vee \dimMeta_2] 3, 4}
%}.
%\end{definition}
%
%% 
%% is needed for the implicitly-type lang:
%\begin{definition}[V-set intersection]
%\label{def:vset-intersect}
%The \emph{intersection} of two v-sets is a v-set of their shared elements
%annotated with the conjunction of their presence conditions, i.e., 
%\ensuremath{
%\vset_1 \cap \vset_2 = \setDef {
%\annot [\dimMeta_1 \wedge \dimMeta_2 ]\elem \myOR
%\annot [\dimMeta_1] \elem \in \vset_1, \annot [\dimMeta_2] \elem \in \vset_2,
%\sat {\dimMeta_1 \wedge \dimMeta_2}
%}
%}.
%For example, \ensuremath{
%\setDef {2, \annot [\A] 3, \annot [\neg \B] 4} \cap
%\pushIn {\annot [\B] {\setDef{2,3,4,5}}} =
%\setDef{\annot [\B] 2, \annot [\A \wedge \B] 3}
%}.
%\end{definition}
%
%\begin{definition} [V-set cross product]
%\label{def:vset-cross}
%The \emph{cross product} of two v-sets is a pair of every two elements of 
%them annotated with the conjunction of their presence conditions.
%\ensuremath{
%\vset_1 \times \vset_2 = \setDef{
%\annot [\dimMeta_1 \wedge \dimMeta_2] {(\elem_1, \elem_2)} \myOR
%\annot [\dimMeta_1] \elem_1 \in \vset_1, \annot [\dimMeta_2] \elem_2 \in \vset_2
%%\vset_1 \cap \vset_2 = \setDef \
%}
%}
%%
%\end{definition}
%
%\begin{definition} [V-set equivalence]
%\label{def:vset-eq}
%Two v-sets are \emph{equivalent}, denoted by
%\ensuremath{\vset_1 \equiv \vset_2}, iff
%\ensuremath{
%\forall \annot  \elem \in (\vset_1 \cup \vset_2). 
%\annot [\dimMeta_1] \elem \in \vset_1, \annot [\dimMeta_2] \elem \in \vset_2, 
%\dimMeta_1 \equiv \dimMeta_2},
%i.e., they both cover the same set of elements and the presence conditions
%of elements from the two v-sets are equivalent.
%\end{definition}
%
%%
%\begin{definition} [V-set subsumption]
%\label{def:vset-subsumption}
%The v-set \ensuremath{\vset_1} \emph {subsumes} the v-set
%\ensuremath{\vset_2}, $\subsume {\vset_2} {\vset_1}$, iff
%\ensuremath{ \forall \annot [\dimMeta_2] \elem \in \vset_2.
%\annot [\dimMeta_1] \elem \in \vset_1, 
%%\neg \sat {\dimMeta_2 \wedge \neg \dimMeta_1}
%\sat {\dimMeta_2 \wedge  \dimMeta_1}
%},
%i.e., all elements in $\vset_2$ also exist in $\vset_1$ 
%s.t. the element is valid in a shared configuration between the v-sets.
%For example, 
%\ensuremath{
% \subsume {\pushIn {\annot [\A] {\setDef {2,3}}}} {\setDef {2, \annot [\A \vee \B] 3, 4}}},
%however, 
%\ensuremath{
% \nsubsume {\pushIn {\annot [\A] {\setDef {2,3}}}} {\setDef {2, \annot [\neg \A \wedge \B] 3}}}
%and
%\ensuremath{
%\nsubsume {\setDef {\annot [\A] 2,\annot [\A] 3, 4}} {\setDef {2, \annot [\A \wedge \B] 3}}}.
%\end{definition}
%
%%\begin{definition} [V-set explicit subsumption]
%%\dropit{drop this for vldb submission. remember you need it for popl.}
%%\label{def:vset-strict-subsumption}
%%The v-set \ensuremath{\vset_1} \emph {explicitly subsumes} the v-set
%%\ensuremath{\vset_2}, $\subsumeExpl {\vset_2} {\vset_1}$, iff
%%\ensuremath{ \forall \annot [\dimMeta_2] \elem \in \vset_2.
%%\annot [\dimMeta_1] \elem \in \vset_1, 
%%\neg \sat {\dimMeta_2 \wedge \neg \dimMeta_1}
%%},
%%i.e., all elements in $\vset_2$ also exist in $\vset_1$ 
%%s.t. its presence condition in \ensuremath{\vset_2} is more specific than 
%%its presence condition in \ensuremath{\vset_1}, captured by 
%%\ensuremath{\neg \sat {\dimMeta_2 \wedge \neg \dimMeta_1}}
%%which could also be defined as 
%%\ensuremath{
%%\nexists \config \in \confSet . \fSem {\dimMeta_1} = \t , \fSem {\dimMeta_2} = \f.
%%%i.e. in set theory:
%%% \dimMeta_2 \subset \dimMeta_1
%%%\dimMeta_2 - \dimMeta_1 = \emptyset
%%%i.e.
%%%\dimMeta_2 \cap \bar{\dimMeta_1} = \emptyset 
%%}
%%\end{definition}
%
%
%
%

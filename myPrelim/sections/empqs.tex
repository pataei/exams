\subsection{Employee Query Set}
\label{app:emp-qs}

%\point{We adapt and adjust the queries from~\cite{prima08Moon}.}\\
%- there are two kinds of queries\\
%- give an example of each

For this case study, we have a set of existing plain queries to start from.
\citet{prima08Moon} provides 12 queries to evaluate the Prima schema evolution
system. We adapt these queries to fit our encoding of the employee VDB
described in \secref{emp-vdb}.
%
%We provide the queries in both the VRA format usable by VDBMS and as
%\cpp{ifdef} annotated SQL, as described in \secref{enron-qs}.%
%\footnote{\href{https://github.com/lambda-land/VDBMS/tree/master/usecases/time-employee/queries}{usecases/time-employee/queries}}
%
9 of these queries have one variant, 2 have two variants, and 1 has three
variants. 


Moon's queries are of two types: 6 retrieve data valid on a particular date
(corresponding to \vThree\ in our encoding), while 6 retrieve data valid on or
after that date (\vThree--\vFive\ in our encoding).
%
For example, one query expresses the intent ``return the salary of employee
number $10004$'' at a time corresponding to \vThree, which we encode:
\[
%\centerline{
%\ensuremath{
Q_1 = \vPrj[\oatt{\salary}{\vThree}].}
% Note that the presence condition of the only attribute \salary\ determines the
% presence condition of the resulting table.
%
% In general and for simplicity, the shared part of presence conditions of
%  projected attributes is factored out and applied to 
%  the entire table. Assume the returned table as a result of
%  query has the schema $\left(\oatt {\vAtt_1} {\dimMeta \wedge \dimMeta_1}, 
%  \oatt {\vAtt_2} {\dimMeta \wedge \dimMeta_2} \right)$.
% The shared restriction can be factored out and applied
% to the entire table, i.e., $\left( \oatt {\vAtt_1} {\dimMeta_1},
% \oatt {\vAtt_2} {\dimMeta_2} \right)^\dimMeta$.
%
We encode the same intent, but for all times at or after \vThree\ as follows:
%
\begin{align*}
Q_2 &=  \vThree\vee\vFour\vee\vFive \langle \pi_\salary 
 (\vThree\vee\vFour\langle
% \\
%&
    \joinRelR{(\vSel[\empno=10004]{\empacct})}{\job}
             {}, 
%&\quad\qquad 
{\vSel[\empno=10004]{\empacct}}\rangle), \empRel\rangle
\end{align*}
%
There are a variety of ways we could have encoded both $Q_1$ and $Q_2$.
%
For $Q_1$ we could equivalently have embedded the projection in a choice,\\

$Q_1' = \chc[\vThree]{\vPrj[\salary]{  \left(\joinRelR{\selectRel{\empno=10004}{\empacct}}{\job}
           {\empacct.\titleatt = \job.\titleatt}\right)},\empRel}$
           
           , however attaching the presence
condition to the only projected attribute determines the presence condition of
the resulting table and so achieves the same effect.
%
In $Q_2$ we use choices to structure the query since we have to project on a
different intermediate result for \vFive\ than for \vThree\ and \vFour.

% The feature expression $\vThree \vee \vFour \vee \vFive$
% determines the database variants to be inquired. 
% Since the schema of \empacct\ and \job\ tables are the same
% in variants \vThree\ and \vFour\ they both have the same 
% query. Note that one could move the condition 
% $\vThree \vee \vFour \vee \vFive$ to the projected attribute
% which results in $\VVal {\eq_2}$, however, this query is wrong 
% because the last alternative of the choice projects attribute \salary\
% from an empty relation which is incorrect. It is important to 
% understand that the behavior of an empty relation is exactly the
% same as its behavior in relational algebra and one
% should be careful of using it in operations such as projection,
% selection, and join.

% \begin{align*}
% \small
% \VVal {\eq_2} &= 
% \pi_{\oatt \salary {\vThree \vee \vFour \vee \vFive}}\\
% &\hspace{-15pt}
% (\vThree \vee \vFour\langle
% (\sigma_{\empno=10004} \empacct)
% %&\hspace{0pt}
% \bowtie_{\empacct . \titleatt = \job . \titleatt}
% \job\\
% &\hspace{20pt},
% \vFive\langle{\selectRel {\empno=10004} \empacct},
% \empRel
% \rangle
% \rangle)
% \end{align*}

As another example, the following query realizes the intent to ``return the
name of the manager of department d001'' during the time frame of
\vThree--\vFive:
%
%\begin{align*}
%\ensuremath{ 
\[
Q_3 = \vThree\vee\vFour\vee\vFive \langle
  \pi_{\name,\fname,\lname} 
%  }
%  \\
%&
%\centerline{
%\ensuremath{  
%\hspace{10pt}
%\qquad
(\joinRel{\chc[\vThree]{\empacct,\empbio}}{(\vSel[\deptno=\deptOne]{\dept})}
           {\empno=\managerno}),
  \empRel \rangle
\]  
%  }.}
%\end{align*}
%
%\noindent
Note that even though the attributes \name, \fname, and \lname\ are not present
in all three of the variants corresponding to \vThree--\vFive, the VRA encoding
permits omitting presence conditions that can be completely determined by the
presence conditions of the corresponding relations or attributes in the
variational schema. So, $Q_3$ is equivalent to the following query in which the
presence conditions of the attributes from the variational schema are listed
explicitly in the projection:
%
%\begin{align*}
%\ensuremath{
\[
Q_3' = \vThree\vee\vFour\vee\vFive \langle
  \pi_{\oatt{\name}{\vThree\vee\vFour},\oatt{\fname}{\vFive},\oatt{\lname}{\vFive}} 
%  }
%  \\
%&
%\centerline{
%\ensuremath{
  (\joinRel{\chc[\vThree]{\empacct,\empbio}}{(\vSel[\deptno=\deptOne]{\dept})}
           {\empno=\managerno}),
  \empRel \rangle
  \]
%  }.}
%\end{align*}
%
%
% \noindent
Allowing developers to encode variation in v-queries based on their
preference makes VRA more flexible and easy to use. 
Also, v-queries are statically type-checked to ensure that
the variation encoded in them does not conflict the variation encoded
in the v-schema. 

\[
\pi_{\mathit{middleName}} \mathit{employee}
\]

\[
\sigma_{\mathit{middleName} \neq \NULL} \mathit{employee}
\]

\[
\empacct
\]

\[
\pi_{\empno, \name, \hiredate, \titleatt, \deptno,  \deptname, \salary} \empacct
\]


\[
\chc [m_{\employee} \wedge (\vTwo \vee \vThree \vee \vFour \vee \vFive)] {\pi_{\empno, \name, \hiredate, \titleatt, \deptno,  \deptname, \salary} \empacct, \empRel}
\]

\[
\chc [m_{\employee} \wedge (\vTwo \vee \vThree \vee \vFour \vee \vFive)] {\pi_{\empno, \optAtt [\vTwo \vee \vThree] [\name], \hiredate, \titleatt, \optAtt [\vThree \vee \vFour \vee \vFive] [\deptno], \optAtt [\vTwo] [\deptname], \optAtt [\vFive] [\salary]} \empacct, \empRel}
\]

\[
q_1 \bowtie_{a_1=a_2} q_2 \equiv \sigma_{a_1=a_2} (q_1 \times q_2)
\]\cite{Li19}

 %
% Finally, we want to briefly illustrate what queries look like in the
% \cpp{ifdef}-annotated SQL format that we distribute as a potentially more
% portable and easy-to-use format for other researchers. Below is the query $Q_3$
% in this format.
% 
% \begin{lstlisting}[language={}]
% #ifdef V3 || V4 || v5
%   #ifdef V3 || V4
% SELECT name
%   #else
% SELECT firstname, lastname
%   #endif
% FROM
%   #ifdef V3
%     empacct
%   #else
%     empbio
%   #endif
% JOIN (SELECT dept WHERE deptno="d001")
%   ON empno=manageno  
% \end{lstlisting}

% \begin{lstlisting}
% #ifdef v3
% SELECT salary
% FROM empacct JOIN job ON empacct.title=job.title
% WHERE empno=10004 
% #endif
% \end{lstlisting}

%\input{formulas/empQs}

\section{Research Goals and Methods}
\label{sec:goals}

\subsection{Identify the kinds of variation existing in relational databases in 
different application domains}
\label{sec:ro1}

To encode variation explicitly in databases we investigate different kinds of 
variation that appears in databases in various application domains. 
Objective 1 aims to represent an encoding for  variation that is 
generic enough that can encode different kinds of variation and is
not bind to a specific instance of variation or application domain. 
%
\tabref{ro1} presents individual research questions we need to answer for
this objective. 

\begin{table}
\caption{Objective 1 research questions.}
\label{tab:ro1}
\centering
\begin{tabularx}{\textwidth}{X}
\toprule
 \textbf{Objective 1: Identify the kinds of variation existing in relational databases in 
different application domains}\tabularnewline
\midrule
RQ1.1: What are the application domains that variation appears in a database? 
What are the dimensions of variation in a database? (\poly, \vamos)
\tabularnewline[0.2cm]
RQ1.2: How can all identified variation instances be represented  in a generic encoding without
regards for the application domain? (\dbpl)
\tabularnewline[0.2cm]
RQ1.3: Can we encode identified instances of variation using our encoding? 
What are the steps one need to take to encode an instance of variation in our
encoding? (\vamos)
\tabularnewline
\bottomrule
\end{tabularx}
\end{table}

\begin{comment}
* schema evolution
* repeated pattern in database versioning, data integration
* this also comes up in software development
* and software also evolves in time
* two dimension that can interact with each other: time and space
\end{comment}

For RQ1.1 we explored different application domains where databases change,
yet, there is a need to keep and access the older versions (variants) of the 
database for business reasons. 

\begin{comment}
* introduce a feature space
* propositional formulas of features
\end{comment}

For RQ1.2 ...

\begin{comment}
* show case the applicability and feasibility of encoding instances of variation using feature expression
\end{comment}

For RQ1.3 ...

\subsection{Design and implement a database framework
%a query language and implement a database management 
%system 
that accommodate identified variations}
\label{sec:ro2}

Having an encoding that represent variation we need to incorporate it within the 
database and the query language to allow explicit storing and manipulation of 
variation in a database. Objective 2 aims to design and implement a database framework
that considers variation as a first-class citizen.
% for a variational
%database and variational query language and implement them as 
%a variational database management system that allows users to interact with a
%variational database. 
\tabref{ro2} presents individual research questions we need
to answer for this objective. 

\begin{table}
\caption{Objective 2 research questions.}
\label{tab:ro2}
\centering
\begin{tabularx}{\textwidth}{X}
\toprule
 \textbf{Objective 2: Design and implement a database framework
%  a query language and implement a database management 
%system 
that accommodates identified variations}
\tabularnewline
\midrule
RQ2.1: How should variation in form of feature expression be incorporated in the database as a first-class citizen? (\dbpl, \poly)
\tabularnewline[0.2cm]
RQ2.2: What are appropriate query languages to interact with a database that accounts for variation explicitly? And how should variation in form of feature expression be incorporated in the query language? (\dbpl, \poly)
\tabularnewline[0.2cm]
RQ2.3: Having a theoretical database framework that accounts for variation explicitly, how 
should we go to implement a database management system that uses that framework? (In progress)
\tabularnewline
\bottomrule
\end{tabularx}
\end{table}


\begin{comment}
* annotations and choices
\end{comment}

For RQ2.1 ...

\begin{comment}

\end{comment}

For RQ2.2 ...

\begin{comment}
\end{comment}

For RQ2.3 ...

\subsection{Demonstrate how the proposed system can be used to manage
variation in databases in different application domains}
\label{sec:ro3}

Having a variational database framework, we need to examine how effectively
it represents instances of variation in databases in different application domains.
Objective 3 focuses on this goal and \tabref{ro3} represents individual research 
question we need to answer for this objective.

\begin{table}
\caption{Objective 3 research questions.}
\label{tab:ro3}
\centering
\begin{tabularx}{\textwidth}{X}
\toprule
 \textbf{Objective 3: Demonstrate how the proposed system can be used to manage
variation in databases in different application domains}
\tabularnewline
\midrule
RQ3.1: Can real-world instances of variation in databases be encoded as a VDB?
What are the steps to generate a VDB from a scenario of variation in a database? (\vamos)
\tabularnewline[0.2cm]
RQ3.2: What are the benefits and drawbacks of representing variation generically
in databases as opposed to having scenario-tailored approaches? (\vamos)
%\tabularnewline[0.2cm]
\tabularnewline
\bottomrule
\end{tabularx}
\end{table}

\begin{comment}
\end{comment}

For RQ3.1 ...

\begin{comment}
\end{comment}

For RQ3.2 ...

\subsection{Mechanize proofs of properties of the language and the system}
\label{sec:ro4}

Having established that our framework  effectively and explicitly encode variation 
within the database and its query language, we need to ensure that it satisfies
the properties we desire. Objective 4 aims to define such properties and
prove that they hold for our framework. 
%Additionally, it aims to prove that
%VDBMS implements the semantics of VRA correctly. 
\tabref{ro4} presents
individual research questions we need to answer for this objective. 



\begin{table}
\caption{Objective 4 research questions.}
\label{tab:ro4}
\centering
\begin{tabularx}{\textwidth}{X}
\toprule
 \textbf{Objective 4: Mechanize proofs of properties of the language and the system}
\tabularnewline
\midrule
RQ4.1: What are the desired properties for a VDB and do they hold for VDB? (\vamos)
\tabularnewline[0.2cm]
RQ4.2: What are the desired properties for VRA? 
Can they be mechanically proved? (In progress)
\tabularnewline[0.2cm]
RQ4.3: Is the implementation of VDBMS compliant to semantics of VRA? (Not started)
\tabularnewline
\bottomrule
\end{tabularx}
\end{table}

\begin{comment}
\end{comment}

For RQ4.1 ...

\begin{comment}
\end{comment}

For RQ4.2 ..

\begin{comment}
\end{comment}

For RQ4.3 ...


\subsection{Stretch goal: Generalize the encoding of variation to make the framework
customizable for different application domains}
\label{sec:ro5}

The main goal of this research is to add variation as a first-class citizen to 
databases to separate the concern of dealing with variation from the application
domain, however, as discussed in \secref{ro3} there is a trade-off between 
 expressiveness and complexity. In other words, although VDB allows 
one to encode different kinds of variation it introduces more complexity than a
specialized system that only manages a specific instance of variation
since it is not bind to the application domain. 
A possible workaround this problem is to generalize the encoding of
variation to allow developers to customize variation to their use case such 
that it straps away unneeded complexity from the query language. We will 
explore this idea once objective 4 is completed and if time permits. 

\subsection{Summary}
\label{sec:sum}

\figref{conn} summarizes the connections between the research questions and activities.
\tabref{timeline} provides the timeline for this proposal.

\begin{figure}
\label{fig:conn}
\end{figure}

\begin{table}
\label{tab:timeline}
\caption{Summary of projected and completed dates for each of the proposed research questions.}
\centering
\begin{tabular}{|c|c|c|c|}
\hline
Research Question & Target conference & Projected Date & Status\\
\hline
1.1 & \poly & N/A & Complete\\
\hline
1.2 & \dbpl & N/A & Complete\\
\hline
1.3 &  \vamos &N/A & Complete\\
\hline
2.1 & \poly, \dbpl, \vldb &Early 2021 & Complete\\
\hline
2.2 & \poly, \dbpl, \vldb &Early 2021 & Complete\\
\hline
2.3 & \vldb & Early 2021 & In progress\\
\hline
3.1 & \vamos &N/A & Complete\\
\hline
3.2 & \vamos &N/A & Complete\\
\hline
4.1 & \vamos &N/A & Complete\\
\hline
4.2 & \vldb, \toplas & Mid 2021 & In progress\\
\hline
4.3 & \toplas & Mid 2021 & Not started\\
\hline
\end{tabular}
\end{table}


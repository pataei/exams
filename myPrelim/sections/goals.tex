\section{Research Goals and Methods}
\label{sec:goals}

\subsection{Identify the kinds of variation existing in relational databases in 
different application domains}
\label{sec:ro1}

To encode variation explicitly in databases we investigate different kinds of 
variation that appears in databases in various application domains. 
Objective 1 aims to represent an encoding for  variation that is 
generic enough that can encode different kinds of variation and is
not bind to a specific instance of variation or application domain. 
%
\tabref{ro1} presents individual research questions we need to answer for
this objective. 

\begin{table}
\caption{Objective 1 research questions.}
\label{tab:ro1}
\centering
\begin{tabularx}{\textwidth}{X}
\toprule
 \textbf{Objective 1: Identify the kinds of variation existing in relational databases in 
different application domains}\tabularnewline
\midrule
RQ1.1: What are the application domains that variation appears in a database? 
What are the dimensions of variation in a database? (\poly)
\tabularnewline[0.2cm]
RQ1.2: How can all identified variation instances be represented  in a generic encoding without
regards for the application domain? (\dbpl)
\tabularnewline[0.2cm]
RQ1.3: Can we encode identified instances of variation using our encoding? 
What are the steps one need to take to encode an instance of variation in our
encoding? (\vamos)
\tabularnewline
\bottomrule
\end{tabularx}
\end{table}



\subsection{Design and implement a database framework
%a query language and implement a database management 
%system 
that accommodate identified variations}
\label{sec:ro2}

Having an encoding that represent variation we need to incorporate it within the 
database and the query language to allow explicit storing and manipulation of 
variation in a database. Objective 2 aims to design and implement a database framework
that considers variation as a first-class citizen.
% for a variational
%database and variational query language and implement them as 
%a variational database management system that allows users to interact with a
%variational database. 
\tabref{ro2} presents individual research questions we need
to answer for this objective. 

\begin{table}
\caption{Objective 2 research questions.}
\label{tab:ro2}
\centering
\begin{tabularx}{\textwidth}{X}
\toprule
 \textbf{Objective 2: Design and implement a database framework
%  a query language and implement a database management 
%system 
that accommodates identified variations}
\tabularnewline
\midrule
RQ2.1: How should variation in form of feature expression be incorporated in the database as a first-class citizen? (\dbpl, \poly, \vldb)
\tabularnewline[0.2cm]
RQ2.2: What are appropriate query languages to interact with a database that accounts for variation explicitly? And how should variation in form of feature expression be incorporated in the query language? (\dbpl, \poly, \vldb)
\tabularnewline[0.2cm]
RQ2.3: 
\tabularnewline
\bottomrule
\end{tabularx}
\end{table}


\subsection{Demonstrate how the proposed system can be used to manage
variation in databases in different application domains}
\label{sec:ro3}

Having a variational database framework, we need to examine how effectively
it represents instances of variation in databases in different application domains.
Objective 3 focuses on this goal and \tabref{ro3} represents individual research 
question we need to answer for this objective.

\begin{table}
\caption{Objective 3 research questions.}
\label{tab:ro3}
\centering
\begin{tabularx}{\textwidth}{X}
\toprule
 \textbf{Objective 3: Demonstrate how the proposed system can be used to manage
variation in databases in different application domains}
\tabularnewline
\midrule
\tabularnewline[0.2cm]
\tabularnewline[0.2cm]

\tabularnewline
\bottomrule
\end{tabularx}
\end{table}

\subsection{Mechanize proofs of properties of the language and the system}
\label{sec:ro4}

Having established that our framework  effectively and explicitly encode variation 
within the database and its query language, we need to ensure that it satisfies
the properties we desire. Objective 4 aims to define such properties and
prove that they hold for our framework. 
%Additionally, it aims to prove that
%VDBMS implements the semantics of VRA correctly. 
\tabref{ro4} presents
individual research questions we need to answer for this objective. 



\begin{table}
\caption{Objective 4 research questions.}
\label{tab:ro4}
\centering
\begin{tabularx}{\textwidth}{X}
\toprule
 \textbf{Objective 4: Mechanize proofs of properties of the language and the system}
\tabularnewline
\midrule
\tabularnewline[0.2cm]
\tabularnewline[0.2cm]

\tabularnewline
\bottomrule
\end{tabularx}
\end{table}


\subsection{Stretch goal: Generalize the encoding of variation and the design of the query 
language to cover more kinds of variation}
\label{sec:ro5}



\subsection{Summary}
\label{sec:sum}

\figref{conn} summarizes the connections between the research questions and activities.
\tabref{timeline} provides the timeline for this proposal.

\begin{figure}
\label{fig:conn}
\end{figure}

\begin{table}
\label{tab:timeline}
\end{table}


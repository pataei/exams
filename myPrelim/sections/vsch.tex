\subsubsection{Variational Schema}
\label{sec:vsch}


A variational schema captures variation in the structure of a database by indicating which attributes and relations are included or excluded in which variants.
%
%\point{using annotated elements to show variability in schema.}
%Variation can exist in the structure of data, i.e., the schema.
%As motivated in \secref{mot}, schema variations include/exclude relations/attributes. 
To achieve this we annotate attributes, relations, and the schema itself with feature expressions,
which describe the condition under which they are present.
%
A \emph{\vrelTxt\ schema (v-relation schema)}, \vRelSch, is a relation name
accompanied with an annotated variational set of attributes:
%\centerline{
$\synDef {\vRelSch} \vRelSchSet \eqq \vRelDef$.
%}
The presence condition of the v-relation schema, \dimMeta, determines the
set of all possible relation schema variants for relation \vRel.
%A \emph{variational attribute set}, \vAttList, is a variational set of attributes.
%%i.e.,
%%$\synDef \vAttList \vAttSet \eqq \optAtt, \vAttList \myOR \empAtt$,
%%where \empAtt\ denotes an empty attribute. 
%%\exref{vsch} illustrates creating
%%a v-schema.
%
A \emph{\vschTxt\ (v-schema)} is an annotated set of v-relation 
schemas:
%\centerline{ 
$\synDef \vSch \vSchSet \eqq \vSchDef$.
%} 
The presence condition of the v-schema, \fModel, determines all  
configurations for non-empty schema variants. We call such configurations
\emph{valid} configurations. The v-schema's presence condition 
is the VDB feature model since it captures the relationship
between features of the underlying application and their constraints,
%as explained in \secref{encode-var} and introduced as the VDB feature model. 
Hence, the v-schema defines all valid schema variants of a VDB. 


\begin{example}
\label{eg:vsch}
$\vSch_1$ is the v-schema of a VDB including only relations \empacct\ and \ecourse\ in the last two rows
of \tabref{mot}, where only features are \vFour, \vFive, \edu, \tFour, \tFive.
Note that attributes that exist conditionally are annotated with a feature expression
to account for such a condition, e.g., the \salary\ attribute only exists when \vFive\ = \t.
%
\begin{align*}
\vSch_1 &=
\{ \empacct ( \empno, \hiredate, \titleatt, \deptno, \annot [\vFive] \salary, 
%&\hspace{55pt}
 \annot [\edu] \isstudent,
\annot [\edu] \isteacher )^{\vFour \vee \vFive}\\
%
&\hspace{17pt} ,\ecourse ( \cno, \cname, \annot [\tFive] \deptno )^{\tFour \vee \tFive} \}^{\fModel_1}\\
\fModel_1 &= 
\paran {\neg \edu \wedge \paran {\vFour \oplus \vFive}
 }
\vee
\paran {\edu \wedge \paran {\vFour \oplus \vFive}
\wedge
\paran {\tFour \oplus \tFive}
}
\end{align*}
where \ensuremath{\fModel_1} allows only one temporal feature for each schema column 
be enabled at a given time.
\end{example}

%\begin{table*}
\caption[Variational schema of the motivating example]{Variational schema $\vSch_\mot$ 
%of the motivating example given in \secref{mot} 
with feature model $\dimMeta_\mot$.
This variational schema encompasses 30 relational schemas: five schemas when \edu\ = \f\ and 25 schemas otherwise. 
%A v-schema encoding the variation of the employee schema introduced in \tabref{mot}. The feature model \fModel\ only allows one temporal feature to be true from a set of temporal features at the time: 
%\ensuremath{
%\fModel = \edu \vee 
%\left( 
%\vOne \oplus \vTwo \oplus \vThree \oplus \vFour \oplus \vFive
%\right) 
%\vee 
%\left( 
%\tOne \oplus \tTwo \oplus \tThree \oplus \tFour \oplus \tFive
%\right)
%}. 
%%where \ensuremath{\fName_1 \oplus \fName_2 = (\fName_1 \wedge \neg \fName_2) \vee (\fName_2 \wedge \neg \fName_1)}.
%However, the feature model can be encoded differently to allow more than one temporal feature to be true at the time. Hence, 
%this is not the only v-schema capturing variation of the employee schema. 
%Additionally, the encoding can change by formulating presence conditions differently while representing the same v-schema, 
%e.g., the presence condition of the \job\ relation can be changed to \ensuremath{\neg \vFive}. 
%This v-schema encompasses 30 relational schemas: five schemas when \edu\ = \f\ and 25 schemas otherwise. 
%%Note that the feature model restricts the schema s.t. only one variant of a sub-schema can exists in the schema, e.g., both \vOne\ and \vTwo cannot be enabled at the same time.
}
\label{tab:mot-vsch}
\arrayrulecolor{black}
\begin{center}
\small
\begin{tabular} {| l |}
\hline
\ensuremath{
\engemp (\empno, \name, \hiredate,\titleatt,\deptname )^{\textcolor{blue}{\vOne}}
}\\
\ensuremath{
\othemp (\empno, \name, \hiredate,\titleatt,\deptname )^{\textcolor{blue}{\vOne}}
}\\
\ensuremath{
\empacct (\empno, \optAtt [{\textcolor{blue}{\vTwo \vee \vThree}}] [\name], \hiredate, \titleatt, \optAtt [{\textcolor{blue}{\vTwo}}] [\deptname], \optAtt [{\textcolor{blue}{\vThree \vee \vFour \vee \vFive}}] [\deptno], \optAtt [{\textcolor{blue}{\vFive}}] [\salary],} \\
\hspace{40pt} \ensuremath{\optAtt [{\textcolor{blue}{\vFour \vee \vFive}}] [\isstudent], \optAtt [{\textcolor{blue}{\vFour \vee \vFive}}] [\isteacher] )^{\textcolor{blue}{\vTwo \vee \vThree \vee \vFour \vFive}}
}\\
\ensuremath{
\job \left(\titleatt, \salary  \right)^{\textcolor{blue}{\vOne \vee \vTwo \vee \vThree \vee \vFour}}
}\\
\ensuremath{
\dept \left(\deptname, \deptno, \managerno, \optAtt [{\textcolor{blue}{\vFive}}] [\studentnum], \optAtt [{\textcolor{blue}{\vFive}}] [\teachernum] \right)^{\textcolor{blue}{\vThree \vee \vFour \vee \vFive}}
}\\
\ensuremath{
\empbio \left(\empno, \sex, \birthdate, \optAtt [{\textcolor{blue}{\vFour}}] [\name], \optAtt [{\textcolor{blue}{\vFive}}] [\fname], \optAtt [{\textcolor{blue}{\vFive}}] [\lname] \right)^{\textcolor{blue}{\vThree \vee \vFour \vee \vFive}}
}\\
%\hdashline
\ensuremath{
\course \left(\optAtt [{\textcolor{blue}{\neg \tOne}}] [\cno], \cname, \optAtt [{\textcolor{blue}{\tOne \vee \tTwo}}] [\tno], \optAtt [{\textcolor{blue}{\tFour \vee \tFive}}] [\timeatt], \optAtt [{\textcolor{blue}{\tFour \vee \tFive}}] [\class], \optAtt [{\textcolor{blue}{\tFive}}] [\deptno] \right)^{\textcolor{blue}{\edu}}
}\\
\ensuremath{
\student \left(\sno, \optAtt [{\textcolor{blue}{\tOne}}] [\cname], \optAtt [{\textcolor{blue}{\neg \tOne}}] [\cno], \optAtt [{\textcolor{blue}{\tThree \vee \tFour}}] [\grade] \right)^{\textcolor{blue}{\edu \wedge \neg \tFive}}
}\\
\ensuremath{
\teach \left(\tno, \cno \right)^{\textcolor{blue}{\edu \wedge \left(\tThree \vee \tFour \vee \tFive\right)}}
}\\
\ensuremath{
\ecourse \left(\cno, \cname, \optAtt [{\textcolor{blue}{\tFive}}] [\deptno] \right)^{\textcolor{blue}{\edu \wedge \left(\tFour \vee \tFive\right)}}
}\\
\ensuremath{
\take \left(\sno, \cno, \grade \right)^{\textcolor{blue}{\edu \wedge \tFive}}
}\\
%\multirow{3}{*}{\vOne} &  \engemp\ (\empno, \name, \hiredate,\titleatt,\deptname) & 
%\course\ (\cname, \tno) & \multirow{3}{*}{\tOne}\\
%& \othemp\ (\empno, \name, \hiredate, \title, \deptname)  & \student\ (\sno, \cname) &\\
%& \job\ (\titleatt, \salary) &  &\\
%\hline
%\multirow{2}{*}{\vTwo} & \empacct\ (\empno, \name, \hiredate, \titleatt, \deptname) & \course\ (\cno, \cname, \tno) & \multirow{2}{*}{\tTwo}\\
%%\cdashline{2-3}
%& \job\ (\titleatt, \salary) & \student\ (\sno, \cno) & \\
%\hline
%\multirow{4}{*}{\vThree} & \empacct\ (\empno, \name, \hiredate, \titleatt, \deptno) & \course\ (\cno, \cname) & \multirow{4}{*}{\tThree}\\
%& \job\ (\titleatt, \salary) & \teach\ (\tno, \cno) &\\
%& \dept\ (\deptname, \deptno, \managerno) & \student\ (\sno, \cno, \grade) &\\
%& \empbio\ (\empno, \sex, \birthdate) & &\\
%\hline
%\multirow{4}{*}{\vFour} & \empacct\ (\empno, \hiredate, \titleatt, \deptno, \dashuline{\isstudent}, \dashuline{\isteacher}) & \ecourse\ (\cno, \cname) & \multirow{4}{*}{\tFour}\\
%& \job\ (\titleatt, \salary) & \course\ (\cno, \cname, \timeatt, \class) & \\
%& \dept\ (\deptname, \deptno, \managerno) & \teach\ (\tno, \cno) & \\
%& \empbio\ (\empno, \sex, \birthdate, \name) & \student\ (\sno, \cno, \grade) & \\
%\hline
%\multirow{4}{*}{\vFive} & \empacct\ (\empno, \hiredate, \titleatt, \deptno,  \dashuline{\isstudent}, \dashuline{\isteacher}, \salary) & \ecourse\ (\cno, \cname, \deptno) & \multirow{4}{*}{\tFive}\\
%& \dept\ (\deptname, \deptno, \managerno,  \dashuline{\studentnum}, \dashuline{\teachernum}) & \course\ (\cno, \cname, \timeatt, \class, \deptno) & \\
%& \empbio\ (\empno, \sex, \birthdate, \fname, \lname) & \teach\ (\tno, \cno) & \\
%&& \take\ (\sno, \cno, \grade) & \\
\hline
\end{tabular}
\end{center}
\end{table*}



%\textbf{Hierarchal structure of feature expressions in a v-schema:}
The presence of an attribute naturally depends on the presence of its parent v-relation, which in turn depends on the feature model.
% presence of the v-schema. 
%
Thus, the presence condition of an attribute is  
the conjunction of its presence condition with its v-relation's presence condition
and the feature model.
%
Similarly,
the presence condition of a v-relation is the conjunction of its
presence condition and the feature model.
%
In other words, the annotated attribute \optAtt\ of v-relation \vRel\ with 
$\dimMeta_r = \getPC \vRel$
defined in the v-schema \vSch\ with feature model \fModel\
is valid if: $\sat {\dimMeta \wedge \dimMeta_r \wedge \fModel }$.
For example, the \isstudent\ attribute described in \exref{vsch} 
is only valid if its presence condition is
satisfiable, i.e.,
\ensuremath {\sat {\edu \wedge (\vFour \vee \vFive) \wedge \fModel }}.

%\subsubsection{Configuring a V-Schema}
%\label{sec:conf-vsch}
In essence, a v-schema is a systematic  
compact representation of all schema variants of the 
underlying application of interest.
%, e.g., SPL, that encodes 
%the variation effectively inside the database schema by means of 
%feature expressions.
%Consequently, v-schemas relieve the
%need to define an intermediate schema and state mappings 
%between it and source schemas, like the approach that 
%data integration systems employ. 
%One can still obtain the 
A specific pure relational schema for
a database variant can be obtained 
by \emph{configuring} the v-schema with that variant's configuration.
We define the configuration function for v-schemas and its elements in \figref{vdb-conf}.
For example, consider the v-schema in \exref{vsch}.
Configuring the variational attribute set of the \empacct\ v-relation for 
the variant \setDef {\vFive}, i.e., 
\ensuremath {\olSem [\setDef {\vFive}] {\empno, \hiredate, \titleatt, \deptno}},
yields the 
%relational 
attribute set of
\ensuremath {\setDef {\underline \empno,\underline \hiredate,\underline \titleatt,\underline \deptno }}.




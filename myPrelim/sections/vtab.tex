\subsubsection{Variational Table}
\label{sec:vtab}

%\arashComment{I did not find the connection between schematic and content variations. Using an example as discussed before, you may explain that annotating only attributes and tables is not enough to create a complete system to express schematic variation and the framework has to support variation in the level of values or tuples. }
%\resp{addressed in schematic variation vs content-level variation.}
%\responded
%\point{tagging tuples, creating v-table, and hierarchal structure of pcs within it.}
Variation also exists in database content. To account 
for content variability, we tag tuples with 
presence conditions, e.g., the tuple $(1,2)^{\A}$ only exists
when \A\  is enabled. 
%
Thus, a \emph{variational tuple (v-tuple)} is an annotated tuple where 
there is a mapping between the v-relation attribute list and  tuple's values:
\ensuremath{
\vTuple \in \vRelCont \eqq \annot  [\dimMeta_\vTuple] {(\vi v \numAtts)}
} where the a relation schema is 
\ensuremath{
\vRel  \annot [\dimMeta_\vRel] {(\vi \vAtt \numAtts)}
}.
%
The content of a v-relation
%  \emph{variational relation content} 
is a set of v-tuples 
$ \vRelCont \in \vRelContSet \eqq \setDef {\vi \vTuple \numTuples}$
and 
%
a  \emph{variational table (v-table)} is the pair of its relation schema and content: 
\ensuremath{
\vTab = (\vRelSch, \vRelCont)
}.
%
A \emph{variational database instance}, \ensuremath{\vdbInst_\vSch}, 
of VDB \vDB\ with v-schema \vSch, 
is a set of v-tables: 
$\synDef \vdbInst  \vdbInstSet \eqq \setDef {\vi \vTab \numRels} $.
%

Note that the value $v_i$ is present iff 
$\sat {\dimMeta_\vTuple \wedge \dimMeta_\vRel \wedge \dimMeta_\vAtt \wedge \fModel}$,
where, 
$\dimMeta_\vAtt = \getPC {\getAtt v_i}$ and
%,
%$\dimMeta_\vTuple = \getPC \vTuple$,
%\dimMeta = \getPC \vRel,
%and 
%\fModel\ is the feature model.
%
for simplicity, 
%Also, note that to avoid overcrowding the database with variation and feature 
%expressions
we only annotate tuples and not cells. 
This design decision causes some redundancy.
For example, the two tuples: 
\ensuremath{\annot [\fOne] {(1,2)}} and \ensuremath{\annot [\neg\fOne] {(1,3)}}
cannot be represented as a single tuple \ensuremath{(1, \chc [\fOne] {2,3})} 
that has variation at its cell-level.
%

This encoding of variational databases satisfies all of the needs for a variational database described in \secref{mot}.
Similar to v-schema, a user can configure a v-table or a VDB
for a specific variant,
formally defined in \figref{vdb-conf}. 
This allows users to deploy a VDB for a specific configuration (variant),
satisfying database part of \nThree\ need.
%
Additionally, 
our VDB framework puts all variants of a database into
one VDB (satisfying \textbf{N0}) 
and it keep tracks of which variant a tuple belongs to by 
annotating them with presence conditions. 
For example, consider tuples
\ensuremath{\annot [\tFive] {(38, PL, 678)}}
and 
\ensuremath{\annot [\tFour] {(23, DB, \nul)}}
that belong to the \ecourse\ table. 
The presence conditions \tFive\ and \tFour\ state that tuples belong to
variants four and five of this VDB, respectively.
Hence, this framework tracks which variants a tuple belongs to 
(first part of \nTwo).
%
Note that 
the VDB framework encodes both schematic and content-level
variation. A simpler framework could be used to encode 
only content-level variation (where tables consist of v-tuples but
they have plain relational schema), similar to frameworks used for 
database versioning and experimental databases~\cite{dbVersioning}.
However, schematic variation cannot be encoded without 
accounting for content-level variation in a framework where
variants coexist in parallel and they are all put into one database,
e.g., while 
\ensuremath{
\ecourse \left(\cno, \cname, \optAtt [\tFive] [\deptno] \right)^{\edu \wedge \left(\tFour \vee \tFive\right)}
} encodes variation at the schema-level for relation \ecourse,
dropping presence conditions of given tuples (i.e., resulting in tuples
\ensuremath{(38, PL, 678)}
and 
\ensuremath{(23, DB, \nul)})
where it is unclear which variant each tuple belongs to
and there is no way to recover such information.


\begin{figure}

\begin{comment}
\textbf{Relational model generic objects:}
\begin{syntax}
%OLD
D\in \mathbf{Dom} &&& \textit{Domain}\\
A\in \mathbf{Att} &&& \textit{Attribute Name}\\
R\in \mathbf{R} &&& \textit{Relation Name}\\
t \in \mathbf{T} &&& \textit{Tuple}
\end{syntax}

\medskip
\textbf{Relational model definition:}
\begin{syntax}
l\in \mathbf{L} &=& \vn{A} &\textit{Attribute set}\\
s \in \mathbb{S} &=& R(A_1, \ldots , A_n ) & \textit{Relation specification}\\
S \in \mathcal{\mathbf{S}} &\Coloneqq& {\vn{s}} & \textit{Schema}\\
T \in \mathbf{T} &\Coloneqq& \{\llangle t(1), \ldots, t(k)\rrangle \myOR \\
&&t(i) \in D_i,
1 \leq i \leq k ,\\
&&k = \mathit{arity}(R) \} 
%v_1^1\in D_1, \ldots, v_n^1\in D_n\rrangle, \ldots, \llangle v_1^m\in D_1, \ldots, v_n^m\in D_n\rrangle|\\
%&& \hspace{0.5cm} m = \textit{number of } R_I\textit{'s tuples}\} 
&\textit{Relation Instance (Table)}\\
%I \in \mathbf{Inst} &\Coloneqq& R_{1_I}, \cdots, R_{n_I} & \textit{Database Instance}
\end{syntax}


\medskip
\textbf{Variational relational algebra objects:}
\begin{syntax}
\synDef \dimMeta \ffSet &&&\textit{Presence condition}\\
\synDef \vAtt \vAttSet &&&\textit{Variational attribute}\\
\synDef \vAttList \vAttSet &\eqq& \vAtt, \vAttList \myOR \empAtt &\textit{Variational attribute list}\\
\synDef \vRelSch \vRelSet &\eqq& \vRelDef &\textit{Variational relation schema}\\
 \vRel &&&\textit{Variational relation}\\
\synDef \vSch \vSchSet &\eqq& \vSchDef &\textit{Variational schema}\\
 &&&\textit{Variational database instance}
\end{syntax}
\end{comment}

%%%%%%%%%%%%%%%%%%%%%%%%%%%%%%%%%%%%%%%%%%%%%%%%%%%
\textbf{Variational Condition Configuration:}
\begin{alignat*}{1}
\ecSem [] . &: \vCondSet \to \confSet \to \pCondSet\\
%
\ecSem \bTag &= \bTag \\
%
\ecSem \vAttOpCte &= 
    \vAttOpCte\\
%	\begin{cases}
%		\vAttOpCte, &\text{ if } \pAtt \in \attr [\eeSem \vRel]\\
%		\f, &\text{ otherwise}
%	\end{cases}\\
%
\ecSem \vAttOpAtt &= 
       \vAttOpAtt\\
%	\begin{cases}
%		\pAttOpAtt, &\text{ if } \pAtt_1 \in \attr [\eeSem \vRel] \&\ 
%		                                   \pAtt_2 \in \attr [\eeSem \vRel] \\
%		\f,  &\text{ otherwise}
%	\end{cases}\\
%
\ecSem {\neg \vCond} &= \neg \ecSem \vCond\\
%
\ecSem {\orr \vCond} &= \ecSem {\vCond_1} \vee \ecSem {\vCond_2}\\
%
\ecSem {\annd \vCond} &= \ecSem {\vCond_1} \wedge \ecSem {\vCond_2}\\
%
\ecSem {\chc {\vCond_1, \vCond_2}} &=
	\begin{cases}
		\ecSem {\vCond_1}, &\text{ if } \fSem \dimMeta  \\
		\ecSem {\vCond_2}, &\text{ otherwise}
	\end{cases}
\end{alignat*}

\medskip
\textbf{Variational Set of Attributes Configuration:}
\begin{flalign*}
& \olSem [] . &: &\ \vAttSet \to \confSet \to \pAttSet\\
%\end{flalign*}
%
%\begin{flalign*}
& \olSem {\{\optAtt\} \cup \vAttList} &\spcEq &\ 
    \begin{cases}
        \{\pAtt\} \cup \olSem{\vAttList},
                            & \If \fSem {\dimMeta \wedge \getPC{\getRel \vAtt} \wedge \fModel} \\
        \olSem{\vAttList} , & \Otherwise
     \end{cases} \\
% & \olSem {\{\optAtt\} \cup \vAttList} &\spcEq &\  \olSem {\{\optAtt\}} \cup \olSem {\vAttList}\\
& \olSem {\setDef{}} &\spcEq & \setDef{}
\end{flalign*}

%
\medskip
\textbf{V-Relation Schema Configuration:}
\begin{flalign*}%\raggedleft
&\orSem [] . : \vRelSchSet \to \confSet \to \pRelSchSet&\\
%\end{flalign*}
%
%\begin{flalign*}
&\orSem \vRelDef = 
	\begin{cases}
		\vRel({\olSem {\vAttList}}, &\If \fSem {\dimMeta \wedge \fModel}) \\
		\empRel, &\Otherwise
	\end{cases}&
\end{flalign*}

%
\medskip
\textbf{V-Schema Configuration:}
\begin{flalign*}%\raggedleft
&\osSem [] . : \vSchSet \to \confSet \to \pSchSet&\\
%\end{flalign*}
%
%\begin{flalign*}
&\osSem {\annot [\fModel] {\setDef {\vRelDefNum 1, \ldots, \vRelDefNum \numRels}}}\\
&\hspace{0.3cm}= \begin{cases}
%		\setDef {\orSem {\vRelDefNumF 1}, \ldots, \orSem {\vRelDefNumF n}},
                 \setDef {\orSem {\vRel_1( \vAttList_1 )^{\dimMeta_1 \wedge \fModel} }, \ldots, 
                 \orSem {\vRel_\numRels( \vAttList_\numRels)^{\dimMeta_\numRels \wedge \fModel} }},		
        & \If \fSem \fModel \\
        \setDef{}, & \text{otherwise}
	\end{cases}&
\end{flalign*}

\medskip
\textbf{V-Tuple Configuration:}
%
\begin{flalign*}%\raggedleft
&\ouSem [] . : \vRelCont \to \confSet \to \pRelCont&\\
%\end{flalign*}
%
%\begin{flalign*}%\raggedleft
&\ouSem {\annot [ \dimMeta_\tuple] {\left( {\vi v \numAtts}\right)}} = \left( \ovSem {v_1}, \hdots, \ovSem {v_\numAtts} \right) &\\
& \textit{ where } \forall 1 \leq i \leq \numAtts: \\
&\hspace{5pt} \ovSem {v_i} = 
\begin{cases}
v_i, & \If \fSem {\fModel \wedge \getPC{\getRel{\getAtt{v_i}}} \wedge \getPC {\getAtt {v_i}} \wedge \dimMeta_\tuple} \\
\varepsilon, & \Otherwise
\end{cases}
\end{flalign*}

\medskip
\textbf{V-Relation Content Configuration:}
%
\begin{flalign*}%\raggedleft
&\otSem [] . : \vRelContSet \to \confSet \to \pRelContSet&\\
%\end{flalign*}
%
%\begin{flalign*}%\raggedleft
&\otSem {\setDef {\vi \tuple \numTuples}} = \setDef {\ouSem {\tuple_1}, \hdots, \ouSem {\tuple_\numTuples}}&
\end{flalign*}

\medskip
\textbf{VDB Instance Configuration:}
%
\begin{flalign*}%\raggedleft
&\odbSem [] . : \vInstSet \to \confSet \to \pInstSet&\\
%\end{flalign*}
%
%\begin{flalign*}%\raggedleft
&\odbSem { {\setDef {\vi \vTab \numRels}}} = \setDef {(\orSem {\vRelSch_1}, \otSem {\vRelCont_1}), \hdots, (\orSem {\vRelSch_\numRels}, \otSem {\vRelCont_\numRels} )}&
\end{flalign*}

\caption{
V-cond and VDB instance configurations.
% of variational set of attributes, v-relations, and v-schema.
$\varepsilon$ denotes a non-existent relation and value.
%Note that the feature model and 
%relation presence condition are passed all the way to attributes due to the 
%hierarchal structure of presence conditions within a v-schema.
}
\label{fig:vdb-conf}
\end{figure} 


\section{Introduction}
\label{sec:intro}


\begin{comment}
simple streamline direction:
- variation in db is everywhere
- active research for some instances of it
- other instances aren't well-covered and well-supported
- the combination of instances

a general solution that 
- can model variation for well-covered variation
- others
\end{comment}

%\structure{openning}
%Variation in databases appears abundantly in different forms and contexts.

Variation in databases arises when multiple database instances 
conceptually represent the same database, but, differ
slightly either in their schema and/or content. 
%
Managing variation in databases is a perennial problem in database literature
and appears in different forms and contexts.
Specific cases of this problem has been extensively studied including 
schema evolution~\cite{SchEvolRA90McKenzie, schVersioning97Castro, 
tempSchEvol91Ariav,prima08Moon}, 
data integration~\cite{dataIntegBook}, 
and database versioning~\cite{datasetVersioning,dbVersioning},
where each instance has a context-specific solution that is tailored
to the constrained problem definition. 
%
For example, schema
evolution approaches deal with variation in a schema (i.e., the structure
of the data) over time, and database versioning and data integration systems 
manage variation in the content of the database. Yet, the database
community does not consider all of these as instances of a similar problem.
%For example, 
%approaches designed to manage schema evolution over time force
%database administrators to add timestamps to each version of the schema
%in the format that is used by the approach~\cite{} while systems designed
%to manage database versioning force database administrators to use 
%version numbers .

Consider schema evolution which is an instance of schematic variation in databases
that is well-supported~\cite{SchEvolRA90McKenzie, 
schVersioning97Castro, tempSchEvol91Ariav, tsql95Snodgrass, 
prima08Moon}.
Changes applied to the schema over time are \emph{variation} 
in the database and every time the database evolves, a new
\emph{variant} is generated.
%which needs to coexists in parallel
%with other variants.
Current solutions addressing schema
evolution rely on the temporal nature of schema evolution by using
timestamps~\cite{SchEvolRA90McKenzie, schVersioning97Castro, 
tempSchEvol91Ariav, tsql95Snodgrass} 
or keeping an external history of 
changes applied to the database~\cite{prima08Moon}. 
These approaches only consider variation in time and do not
%However, none of them 
incorporate the time-based changes into
the database directly, rather they \emph{simulate} the effect of these changes,
resulting in context-specific systems that break when used in another instance of
variation in databases. 

%Database-backed software produced by software product line (SPL) 
%is an example where variation arises in databases and is poorly 
%supported. SPL is an
%approach to developing and maintaining software-intensive systems 
%in a cost-effective, easy to maintain manner by accommodating variation
%in the software that is being reused. 
%%The products of a SPL pertain to a
%%common application domain or business goal. 
%%They also have a common
%%managed set of features that describe the specific need for a product. 
%%They share 
%%a common codebase which is used to produce a product with respect to its set of 
%%selected (enabled) features
%In SPL, a common codebase is shared and used to produce products w.r.t.
%a set of selected (enabled) features~\cite{splBook}. 
%% from the SPL feature set designed to describe
%%specific needs for products~\cite{splBook}. 
%Different products of a SPL typically have different
%sets of enabled features or are tailored to run in different environments. These
%differences impose different data requirements which creates variation in space 
%in the shared database used in the common codebase. 
%%Hence, 
%%each product has its own database variant.
%%For example, different legal
%%requirements often require tracking different data in products tailored for use
%%in different countries or regions~\cite{splBook}.
%%Different data requirements results in different desired data which 
%%creates database variants. 
%%
%%
%%The database variants corresponding to software products differ mainly 
%%in their schema, a relation/attribute can either be included or excluded for a 
%%specific software product~\cite{vdbSpl18ATW}.
%The variation is in the form of exclusion/inclusion of tables/attributes based on
%selected features for a product~\cite{vdbSpl18ATW}.
%%
%In practice, software systems produced by a SPL are accommodated with a database that
%has all attributes and tables available for all variants-- a database with universal schema~\cite{vdbSpl18ATW}. 
%Unfortunately, this approach is
%inefficient, error-prone, and filled with lots of null values since not all attributes and tables
%are valid for all variant products. A possible solution to this could be defining views on 
%the universal database per software variant and write queries for each variant against its 
%view~\cite{vdbSpl18ATW}.
%However, this is burdensome, expensive, and costly to maintain since it 
%requires developers to generate and maintain numerous view definitions
%in addition to manually generating
%and managing the mappings between views and the universal schema for each product.


Additionally, the software product line (SPL) community has realized that 
variation in software propagates to the artifacts the software relies on, 
including the database.
%
\TODO{revise def. Eric said it's awkward!}
SPL is an
approach to developing and maintaining software-intensive systems 
in a cost-effective, easy to maintain manner by accommodating variation
in the software that is being reused. 
%
In SPL, a common codebase is shared and used to produce products w.r.t.
a set of selected (enabled) features~\cite{splBook}. 
%
Different products of an SPL typically have different
sets of enabled features or are tailored to run in different environments. These
differences impose different data requirements, which in turn requires a different database 
for each product.
% creates variation in space 
%in the shared database used in the common codebase. 
%
%The SPL community has closed
%the gap of variation appearing in database schema while developing software
%and the database schema used in client's application 
Researchers have addressed this problem by
encoding variation explicitly in the data model, called a \emph{variable data model},
by associating features with elements of the schema.
% and deploying the variable
%data model for 
The specific database for a client's application can then be generated 
 by only selecting the set of elements 
associated with features that have been selected for the client's application~\cite{skrhas09DBIS, slrs12CAiSE, 
ad11varDataModel}. 
However, this approach breaks when software itself evolves over time, 
which is unavoidable~\cite{dbSPLevolve}.
%\TODO{it suffices for their exact usecase but cannot manage evolution of spl.
%Software evolution is unavoidable, so is its artifacts evolution, 
%including databases~\cite{dbSPLevolve}.
%%Variation in software development is unavoidable and 
%This is where two instances of managing 
%variation in databases (schema evolution and database-backed software 
%developed by SPL) interact.
%% and variation occurs both in time and space. 
%While there are solutions to schema evolution
%they cannot adapt to a new situation because they only provide a solution to
%variation of databases in time and cannot encode the interaction of database variation in
%time and space.
%}
%However, the direct encoding of variation has not been extended to the
%database itself by these approaches. 

To the best of our knowledge, 
there is no generic solution that manages all possible kinds of
variation in databases. Thus, we explore the idea of considering
variation as an orthogonal concern in databases. Such exploration
poses questions such as: How can variation be represented in a generic and
expressive manner? 
%How can variation be encoded explicitly in databases?
What are the benefits and drawbacks of explicitly encoding variation in 
databases? How feasible is it to encode variation 
directly in databases for realistic problems?

To answer these research questions we propose the following research goals:

\begin{itemize}
\item Objective 1: Identify the kinds of variation existing in relational databases in 
different application domains.
\item Objective 2: Design a query language and implement a database management 
system that accommodate  variations identified in Objective 1.
\item Objective 3: Demonstrate how the proposed system manages
variation in databases in different application domains and how effective and efficient it is.
\item Objective 4: Mechanize proofs of properties of the language and the system.
% maybe as a stretch goal
%\item Objective 5: Generalize the encoding of variation and the design of the query 
%language to cover more kinds of variation.
\end{itemize}

\TODO{at the end, make sure the abstract and objectives are promising the same thing}

%The proposal is structured such that 
%for 
Each objective has some research questions that need to be answered
which are followed by how we approached or will approach questions with references to
sections including details of our approach.
% The details
%How we approach a question followsThe answers or our plan to explore a question follows after posing the question,
%however, the details of our approach for a question is referenced in a separate
%section so that the reader can skip ahead if they choose not to go deep in details
%for an approach.

%----------------------------------------------------------------------------
%
%are instances of the general problem that are
%not well-studied, resulting in using manual approaches that burden experts.
%Moreover, different kinds of variation can interact which cannot be addressed
%by current approaches due to lack of a general solution to managing 
%different kinds of variation in databases.
%%In this paper, 
%To understand 
%we provide a fundamental solution to managing
%variation in databases by considering variation explicitly as 
%a \emph{first-class citizen}
%in the system, allowing for encoding different kinds of variation.
%
%
%%\structure{the following paragraphs are funnel}
%%\point{Schematic and content-level variation in DBs that conceptually 
%%represent the same data.}
%Variation in databases arises when multiple database instances 
%conceptually represent the same database, but, differ
%slightly either in their schema and/or content. 
%%These database
%%instances coexist in parallel.
%The variation in schema and/or content occurs in two dimensions:
%time and space. Variation in space refers to different variants of database that
%coexist in parallel while variation in time refers to the evolution of 
%database, similar to variation observed in software~\cite{EricSPLEvolve19}. 
%Note that variation in a database can occur due to both dimensions
%at the same time.
%
%
%%\point{Schema evolution is an instance of variation in databases
%%that is well-supported.}
%%Explains how schema evolution (which is unavoidable) 
%%is an instance of variation in databases.
%%And mentions some of the current solutions, emphasizing that
%%they cannot address other instances of the problem.
%Schema evolution is an instance of schematic variation in databases
%that is well-supported~\cite{SchEvolRA90McKenzie, 
%schVersioning97Castro, tempSchEvol91Ariav, tsql95Snodgrass, 
%prima08Moon}.
%Changes applied to the schema over time are \emph{variation} 
%in the database and every time the database evolves, a new
%\emph{variant} is generated.
%%which needs to coexists in parallel
%%with other variants.
%Current solutions addressing schema
%evolution rely on temporal nature of schema evolution by using
%timestamps~\cite{SchEvolRA90McKenzie, schVersioning97Castro, 
%tempSchEvol91Ariav, tsql95Snodgrass} 
%or keeping an external file of time-line history of 
%changes applied to the database~\cite{prima08Moon}. 
%These approaches only consider variation in time and do not
%%However, none of them 
%incorporate the time-based changes into
%the database directly, rather they \emph{simulate} the effect of these changes,
%resulting in brittle systems.
%
%
%%\point{Database-backed software produced by SPL is an instance
%%of variation in databases that is poorly managed in practice.}
%%Explains SPL briefly and how variation appears in databases used to
%%store data for software produced by SPL. As well as how poorly it is 
%%managed in practice.
%Database-backed software produced by software product line (SPL) 
%is an example where variation arises in databases and is poorly 
%supported. SPL is an
%approach to developing and maintaining software-intensive systems 
%in a cost-effective, easy to maintain manner by accommodating variation
%in the software that is being reused. 
%%The products of a SPL pertain to a
%%common application domain or business goal. 
%%They also have a common
%%managed set of features that describe the specific need for a product. 
%%They share 
%%a common codebase which is used to produce a product with respect to its set of 
%%selected (enabled) features
%In SPL, a common codebase is shared and used to produce products w.r.t.
%a set of selected (enabled) features~\cite{splBook}. 
%% from the SPL feature set designed to describe
%%specific needs for products~\cite{splBook}. 
%Different products of a SPL typically have different
%sets of enabled features or are tailored to run in different environments. These
%differences impose different data requirements which creates variation in space 
%in the shared database used in the common codebase. 
%%Hence, 
%%each product has its own database variant.
%%For example, different legal
%%requirements often require tracking different data in products tailored for use
%%in different countries or regions~\cite{splBook}.
%%Different data requirements results in different desired data which 
%%creates database variants. 
%%
%%
%%The database variants corresponding to software products differ mainly 
%%in their schema, a relation/attribute can either be included or excluded for a 
%%specific software product~\cite{vdbSpl18ATW}.
%The variation is in the form of exclusion/inclusion of tables/attributes based on
%selected features for a product~\cite{vdbSpl18ATW}.
%%
%In practice, software systems produced by a SPL are accommodated with a database that
%has all attributes and tables available for all variants-- a database with universal schema~\cite{vdbSpl18ATW}. 
%Unfortunately, this approach is
%inefficient, error-prone, and filled with lots of null values since not all attributes and tables
%are valid for all variant products. A possible solution to this could be defining views on 
%the universal database per software variant and write queries for each variant against its 
%view~\cite{vdbSpl18ATW}.
%However, this is burdensome, expensive, and costly to maintain since it 
%requires developers to generate and maintain numerous view definitions
%in addition to manually generating
%and managing the mappings between views and the universal schema for each product.
%
%
%%\point{Schema evolution meets SPL evolution and results in databases
%%used in SPL that their schemas evolve over time.}EricSPLEvolve19, splEvolveBP14
%%Mentions that SPL evolution is hot topic. Part of this evolution is database
%%evolution. This is where two instances of managing variation in databases
%%interact and even the well-supported systems for schema evolution cannot
%%address it. 
%Software evolution is unavoidable, so is its artifacts evolution, 
%including databases~\cite{dbSPLevolve}.
%%Variation in software development is unavoidable and 
%This is where two instances of managing 
%variation in databases (schema evolution and database-backed software 
%developed by SPL) interact.
%% and variation occurs both in time and space. 
%While there are solutions to schema evolution
%they cannot adapt to a new situation because they only provide a solution to
%variation of databases in time and cannot encode the interaction of database variation in
%time and space.
%%dismiss variation of databases in space, 
%%i.e., they dismiss that database variants that have been created due to variation
%%in time must coexist in parallel (a property of variation in 
%%space~\cite{EricSPLEvolve19}). 
%We motivate this case through an example in \secref{mot}.
%%We use this case as our motivating example in \secref{mot} and explain it 
%%in more details.
%
%%\structure{knowledge gap + challenge}
%%\point{Reiterate the knowledge gap. Introduce the challenge.} 
%%Reiterates the knowledge gap. Explains the challenge:
%%The challenge then becomes encoding variation in databases
%%that can model different instances and satisfy different specialists' needs 
%%at different stages (like development, deployment, information extraction, etc).
%%%Explains the challenge as incorporating variation in databases s.t. all 
%%%database variants are gathered in one place. 
%%Also, itemizes the users' 
%%needs in such an environment: 
%%1) Query some/all variants simultaneously and selectively.
%%2) Track the original variant of a piece of data and the variation applied
%%to it through a query.
%%3) Deploy the database and its queries to a variant of it.
%As we have shown, 
%variation in databases w.r.t. time and space is abundant and inexorable~\cite{dbDecay16Stonebraker};
%impacts DBAs, data scientists, and developers significantly~\cite{dbSPLevolve}; 
%and appears in different contexts. Various research has studied variation of databases
%in time and space, however, they all consider only one of these dimensions and
%are extremely tailored to a specific context. This becomes a problem when 
%the two dimensions and different contexts interact. Hence, the challenge becomes
%%offering a fundamental generic framework that 
%incorporating variation in databases s.t. it can model different instances of variation
%in different contexts and it satisfies different specialists' needs at different stages,
%e.g., development, information extraction, and deployment. 
%%
%%We categorize these needs and explain them throughout the paper:
%%\begin{enumerate}[leftmargin=*]
%%\item [\textbf{N0}]Have access to all database variants at a given time
%%\item [\textbf{N1}]Query multiple database variants simultaneously and selectively
%%\item [\textbf{N2}]Keep track of which variants a piece of data belongs to and ensuring that 
%%         it is maintained throughout a query
%%\item [\textbf{N3}]Deploy one variant of the database and its associated queries.
%%\end{enumerate}
%%\structure{proposed solution}
%%%\point{challenge: capturing/representing variability in the database + 
%%%(put-all-together origin-tracking framework) + solution (contributions).}
%%%\structure{point-last.}
%%Therefore, 
%%\emph{
%%how can we incorporate variability in databases to allow for
%%expressive queries over multiple
%%database variants selectively and simultaneously while keeping
%%track of the original variation of data and variation applied to it by
%%a query
%%%without violating data provenance
%%%and 
%%s.t. it is applicable in different contexts?}
%Our contributions in this paper address this challenge:
%\begin{itemize}[leftmargin=*]
%\itemsep0em
%%[wide, labelwidth=!, labelindent=0pt, topsep=1pt]
%%[leftmargin=*]
%\item We provide a framework to capture variation within a database using
%propositional formulas over  
%sets of features, called \emph{feature expressions}, following~\cite{vdb17ATW}.
%\item We incorporate feature expressions  into both the structure (schema) and
%content (tuples) of the database, introducing \emph{variational schemas}, \secref{vsch}, 
%and \emph{variational tables}, \secref{vtab}, and together \emph{VDBs},
%\secref{vdb}.
%%, satisfying \textbf{N0} and first part of \textbf{N2}.
%\item To express user information needs we define the 
%\emph{variational relational algebra} query language,
%%which is 
%a combination of relational algebra and 
%choice calculus~\cite{EW11tosem,Walk13thesis}, \secref{vrel-alg}.
%%, 
%%satisfying \textbf{N1}.
%Users query a VDB by a \emph{variational query}, \secref{vq}.
%\item 
%To make variational queries more useable and easier to understand, respectively,
%we define 
%a static type system, \secref{type-sys},
%and \emph{variation-minimiztion} rules, \secref{var-min}.
%%to make it easier to understand and more useable. 
%%rules to minimize 
%%variation in v-query, \secref{var-min}, to provide better
%%efficiency and usability. 
%%This completes satisfiability of \textbf{N2}.
%\item To query a 
%variational database and receive clear results
%we implement \emph{Variational Database Management System (VDBMS)}, \secref{impl}.
%%,
%%satisfying all four needs: \textbf{N0}-\textbf{N3}.
%%\textbf{N1}, \textbf{N2}, and \textbf{N3}.
%\end{itemize}
%
%%\point{evaluate VDBMS + its pay off}
%%To evaluate VDBMS, 
%VDBMS provides new functionality to databases by accounting for 
%schematic and content-level variation. We provide two use cases of how VDBMS can
%model the schema evolution and SPL instances of variation in databases
%for two real-world databases,
%%(employee database\footnote{\url{https://github.com/datacharmer/test_db}} 
%%and Enron email data corpus\footnote{\url{http://www.ahschulz.de/enron-email-data/}}),
%\secref{exp-disc}. 
%%We also adopt practical queries from~\cite{prima08Moon} and~\cite{emailSPL},
%%respectively,
%%for our use cases to assess VDBMS, \secref{exp-qs}.
%We hypothesize that it especially pays
%off when there is lots of variation in the database and the user needs to query
%many of the variants simultaneously, \secref{exp-disc}. 
%%To examine our hypothesis,  
%%we adopt two real-world databases 
%%from two different contexts (schema evolution and SPL) and
%%generate their counterpart VDBs (based on the changes applied to them).
%%, assess
%%the performance of VDBMS using a set of queries within each context, and
%%finally analyze and discuss the results, \secref{exp-disc}.
%%Our experiments show ...
%%, elucidated more in \secref{exp-disc}.
%
%
%
%%%\begin{comment}
%%%vamos intro:
%%\section{Introduction}
%%\label{sec:intro}
%%
%%%\NOTE{High-level comment: avoid abbreviations like ``w.r.t.'' (I usually edit
%%%most of these out when revising your writing), or at least use them very
%%%sparingly. ``i.e.'' and ``e.g.'' are OK in parentheticals (e.g.\ this is fine),
%%%but should be avoided in the flow of text, e.g.\ this is awkward.}
%%
%%%\NOTE{Maybe first paragraph on how variation in software leads to variation in
%%%databases and vice-versa? Then get into how this breaks down into time and
%%%space.}
%%
%%%\TODO{we show feasibility of our approach. feasibility study.}
%%%
%%%\TODO{var in softwar. var in db. var time space. db has looked ...
%%%vdb more general and complicated. feasibility study.}
%%%
%%%\TODO{bulleted contribution list}
%%%
%%%\TODO{less harsh claim instead of miss variablity. say they don't recognize these are instances of the same problem.}
%%%
%%%\TODO{instead of brittle say specific or not general. very specific to a defined problem. cite a ppr that shows schema evolution sol cannot handle all situations.}
%%%
%%%\TODO{remember to go back to unification of time and space.}
%%
%%Variation in software leaks into its development code and data structures~\cite{Walk14onward,MMWWK17vamos,alkubaish20},
%%tools used by the software~\cite{ywt20splc}, and its artifacts. Here we focus on 
%%one kind of external artifact that is ubiquitous in software to store and manipulate
%%data: \emph{relational databases}.
%%%
%%For example, different clients of a SPL require different information need
%%and consequently, different databases~\cite{skrhas09DBIS}.
%%%
%%Furthermore, software evolves over time resulting in requiring 
%%new information needs or revising previous information needs.
%%This makes database evolvement over time unavoidable~\cite{dbDecay16Stonebraker}.
%%
%%
%%
%%Software systems can vary in two dimensions: ``space'' and ``time''.
%%Variation over space refers to the simultaneous development and maintenance of
%%related systems with different feature sets, which is the focus of work on
%%SPLs.
%%%
%%Variation over time refers to the incremental evolution of a system as it is
%%developed and maintained, and is the focus of revision control systems and
%%configuration management (CM)~\cite{Dart91}.
%%%
%%Although these two aspects of variation have traditionally been studied and
%%addressed separately, recent work has sought to unify the treatment of them
%%%both
%%%kinds of variation 
%%to support new kinds
%%of analyses that consider both dimensions of variation at once~\cite{Thu19vv}.
%%%in order to ease maintenance, support the reuse of
%%%techniques developed in their respective communities, and to support new kinds
%%%of analyses that consider both dimensions of variation at once~\cite{Thu19vv}.
%%%
%%%The case studies we present illustrate the impact of both dimensions of
%%%variation on relational databases, and demonstrate how our conception of
%%%variational databases is generic with respect to the underlying dimension(s) of
%%%variation.
%%
%%%\NOTE{The phrase ``dismissed the nature of the problem'' is strong and
%%%vague---a bad combination! There are also lots of citations here. Probably need
%%%to make a handful of more specific claims about the problem and provide a small
%%%number of citations for each one. Also not clear what the italicized
%%%\emph{variation} means here.}
%%
%%
%%Similarly, variation in databases exists in time and space. 
%%Variation in time  appears when a database evolves 
%%over time while variation in space  appears due
%%to different information requirements by software or different
%%sources of information. 
%%%
%%However, these variation dimensions in databases have not
%%been studied. Instead, instances of them have been studied
%%and addressed separately. 
%%
%%%
%%For example, consider the schema evolution problem:
%%the schema of a database changes over time due to new
%%business requirements~\cite{SchEvolRA90McKenzie}. 
%%This requires applying the changes
%%caused by the evolution to the queries and data. This
%%problem
%%%to the queries written on  (i.e., an instance of database
%%%variation in time) 
%%has been studied
%%thoroughly by the database community~\cite{schVersioning97Castro,SchEvolRA90McKenzie,
%%prima08Moon}.
%%Yet, it is not recognized as variation in the database by the community:
%%%
%%changes applied to the schema over time are \emph{variation} 
%%in the database and every time the database evolves, a new
%%\emph{variant} is generated.
%%%
%%Current solutions addressing schema
%%evolution rely on temporal nature of schema evolution by using
%%timestamps~\cite{SchEvolRA90McKenzie, schVersioning97Castro, 
%%tempSchEvol91Ariav} 
%%or keeping an external file of time-line history of 
%%changes applied to the database~\cite{prima08Moon}. 
%%These approaches only consider variation in time and do not
%%%However, none of them 
%%incorporate the time-based changes into
%%the database directly, rather they \emph{simulate} the effect of these changes,
%%resulting in specialized systems that cannot be extended to other instances
%%of database variation in time.
%%%
%%This observation is not limited to schema evolution;
%%some examples of this include 
%%database versioning~\cite{datasetVersioning,dbVersioning},
%%data integration~\cite{dataIntegBook}, 
%%and data provenance~\cite{bt07sigmod}.
%%%\TODO{food of thought: temporal databases~\cite{tsql95Snodgrass}.}
%%
%%%
%%On the other hand, the SPL community recognizes and addresses the need 
%%for variable data models which models a database schema 
%%(usually as an Entity-Relation model) with
%%annotations of features from SPL to indicate their variable 
%%existence~\cite{skrhas09DBIS, slrs12CAiSE, 
%%ad11varDataModel}, but, it does not expand such need to 
%%the data or the query language. 
%%%
%%%it does not address the variation that appears in queries and data. 
%%These approaches only consider variation in space over a data model
%%and do not address the variation that appears in queries and data. 
%%Thus, developers have to write the required information need as a
%%query encoded as a string per variant. Not only this is labor-some but
%%also due to the nature of queries being encoded as strings there is no
%%static check to ensure that queries are type correct. 
%%This impacts DBAs, data scientists, and developers significantly~\cite{dbSPLevolve}.
%%
%%\begin{comment}
%%Effectively managing variation is a fundamental challenge of software
%%engineering. Research on CM and SPLs
%%have developed numerous representations and strategies for effectively managing
%%different kinds of variation in software. However, these solutions typically do
%%not extend to managing variation in the artifacts and systems that software
%%uses.
%%%An especially tricky aspect is managing corresponding variation in
%%%the external artifacts and services that a software system interacts with.
%%%
%%Here, we focus on \emph{relational databases}.
%%%one kind of external artifact that is ubiquitous in
%%%software but not well-supported by current approaches to managing software
%%%variation: \emph{relational databases}.
%%%
%%Different variants of a software system have different information 
%%needs~\cite{skrhas09DBIS}, which
%%implies a corresponding need for variation in the structure and content of the
%%relational databases that these systems use and rely on. This clearly results in 
%%having variation in the database used to develop a SPL.
%%%
%%%While research on software product lines (SPLs) \cite{fospl} 
%%%has led to a variety of representations and techniques for safely working with
%%%many variants of a software system, these solutions don't extend to relational
%%%databases.
%%
%%In a database-supported SPL, 
%%typically a number of strategies are employed to
%%accommodate the different information needs of different variants.
%%%
%%The first is that a different relational database may be \emph{specified and
%%created per-variant}, according to the information needs of each
%%variant~\cite{marco13featureAdaptSch}. 
%%This approach is labor-intensive and difficult to maintain
%%since changes need to be propagated across variants manually.
%%%
%%The second strategy is to define a single \emph{global schema that applies
%%to all variants}~\cite{batini86dbSchIntegAnalysis}. 
%%This strategy is more efficient to maintain compared to the previous approach
%%but is still hard to maintain,
%%especially in face of SPL evolution. Due to lack of separation of concerns
%%and suboptimal traceability of requirements to database elements~\cite{skrhas09DBIS}
%%it is also complex, hard to understand, and unscalable~\cite{slrs12CAiSE}. 
%%Additionally, it suffers from design limitation and 
%%error-proneness since parts of the schema will be irrelevant to each variant,
%%resulting in losing database's integrity constraints~\cite{slrs12CAiSE}.
%%%Irrelevant attributes are typically populated by NULL-values, which may later
%%%be referenced since it is impossible to check or enforce that queries in each
%%%variant use the database in a safe and consistent way.
%%%
%%The third strategy is to define a \emph{variable data model}~\cite{skrhas09DBIS, 
%%slrs12CAiSE, ad11varDataModel} which models a database schema 
%%(usually as an Entity-Relation model) with
%%annotations of features from SPL to indicate their variable existence. 
%%This approach addresses problems of the previous approach, however,
%%it does not address the variation that appears in queries and data. 
%%Thus, developers have to write the required information need as a
%%query encoded as a string per variant. Not only this is labor-some but
%%also due to the nature of queries being encoded as strings there is no
%%static check to ensure that queries are type correct. 
%%\end{comment}
%%
%%Thus, an initial need for considering variation both in time and space
%%as first-class citizen in databases arises.
%%%
%%To this end, we have developed \emph{variational databases (VDBs)}
%%and \emph{variational queries}~\cite{ATW18poly,ATW17dbpl}, which extend ideas
%%developed in the SPL community to the creation, management,
%%and querying of relational databases.
%%%
%%Conceptually, a  \emph{single} variational database represents \emph{many}
%%different plain relational
%%databases, each one corresponding to a different variant of a software system,
%%\emph{at the same time}.
%%Similarly, a variational query represents potentially many different queries,
%%each one corresponding to a variant of a variational database.
%%%
%%Together, they enable safely and efficiently working with many variants of a
%%relational database at once, and reliably integrating the variants of a
%%database with the corresponding variants of a software product line.
%%%
%%We are currently implementing these ideas in \emph{VDBMS}, 
%%a practical implementation of
%%variational databases as a lightweight wrapper on top of a traditional
%%relational database management system.
%%
%%However, the generic and expressive approach of VDB in dealing
%%with database variation creates new complexity and cost of learning
%%which raises the question: is explicitly encoding variation in databases
%%a good idea?
%%To answer this question, in this paper, we
%%\begin{itemize}
%%%
%%\item show the feasibility of VDB by systematically generating
%%two VDBs from real-world instances of variation in time and space
%%from the scratch, \secref{db}
%%%
%%\item illustrate the applicability of variational queries by encoding information
%%needs for the developed VDBs from scenarios described in the literature,
%%\secref{q}
%%%
%%\item  discuss the pros and cons of explicitly
%%encoding variation in databases and future research directions, \secref{dis}
%%\end{itemize}
%%%
%%We distribute the VDBs, SQL scripts for generating them, 
%%and queries of our case studies in a GitHub repository available \href{https://github.com/lambda-land/VDBMS/tree/master/usecases}{here} for public use. 
%%%
%%We distribute the VDBs in either MySQL or Postgres and
%%in two form, one with embedded  
%%schema described in \secref{enron-vsch}, and one without the embedded schema
%%for use with our VDBMS tool in which the schema is provided
%% separately.
%%%
%%We distribute the queries in our encodings and
%%%but also 
%%as simple \cpp{ifdef}-annotated SQL files to promote their broad reuse
%%in the design and evaluation of other systems for managing variational
%%relational data.
%%
%%\begin{comment}
%%\NOTE{I like this paragraph, but the claim at the end is strong. Are database
%%evolution systems really brittle? And if so, how? For a claim like this, we
%%probably need to be a bit more specific and definitely need a citation.}
%%
%%Although we were inspired to design  VDB because of variation 
%%of databases in SPL, we realized variation in databases appear
%%in other contexts and forms. 
%%%
%%For example, schema evolution is an instance of schematic variation in databases
%%that is well-supported~\cite{SchEvolRA90McKenzie, 
%%schVersioning97Castro, tempSchEvol91Ariav, tsql95Snodgrass, 
%%prima08Moon}.
%%Changes applied to the schema over time are \emph{variation} 
%%in the database and every time the database evolves, a new
%%\emph{variant} is generated.
%%%which needs to coexists in parallel
%%%with other variants.
%%Current solutions addressing schema
%%evolution rely on temporal nature of schema evolution by using
%%timestamps~\cite{SchEvolRA90McKenzie, schVersioning97Castro, 
%%tempSchEvol91Ariav, tsql95Snodgrass} 
%%or keeping an external file of time-line history of 
%%changes applied to the database~\cite{prima08Moon}. 
%%These approaches only consider variation in time and do not
%%%However, none of them 
%%incorporate the time-based changes into
%%the database directly, rather they \emph{simulate} the effect of these changes,
%%resulting in brittle systems.
%%
%%
%%
%%%
%%%Although we have so far considered only SPL-like variation, it's important to
%%%note that software systems can vary in both ``space'' and ``time''.
%%%%
%%%Variation over space refers to the simultaneous development and maintenance of
%%%related systems with different feature sets, which is the focus of work on
%%%SPLs.
%%%%
%%%Variation over time refers to the incremental evolution of a system as it is
%%%developed and maintained, and is the focus of revision control systems and
%%%configuration management (CM)~\cite{Dart91}.
%%%%
%%%Although these two kinds of variation have traditionally been studied and
%%%addressed separately, recent work has sought to unify the treatment of both
%%%kinds of variation in order to ease maintenance, support the reuse of
%%%techniques developed in their respective communities, and to support new kinds
%%%of analyses that consider both dimensions of variation at once~\cite{Thu19vv}.
%%%%
%%%The case studies we present illustrate the impact of both dimensions of
%%%variation on relational databases, and demonstrate how our conception of
%%%variational databases is generic with respect to the underlying dimension(s) of
%%%variation.
%%%
%%
%%\NOTE{First part of this paragraph sounds like it's from a conclusion, not an
%%introduction. We haven't shown anything yet!
%%
%%\medskip
%%Is it possible to organize the outline around a bulleted list of
%%contributions?}
%%
%%So far we have shown, 
%%variation in databases w.r.t. time and space is abundant 
%%and inexorable~\cite{dbDecay16Stonebraker};
%%impacts DBAs, data scientists, and developers significantly~\cite{dbSPLevolve}; 
%%and appears in different contexts. 
%%\end{comment}
%%
%%
%%
%%\begin{comment}
%%%
%%We also showed examples
%%of instances of variation in databases, however, the question
%%becomes how would one encode each specific instance of
%%variation in a database in our VDB framework and what 
%%are the pros and cons of doing such. 
%%%
%%We introduce VDB
%%formalism in \secref{background} and use the introduced
%%notation to describe in detail two case studies (one of variation
%%w.r.t. time and the other of variation w.r.t. space) of systematically generating
%%a VDB for an instance of variation in a database from the scratch
%%(we derive the features, schema, adapt and adjust the data,
%%and populate the VDB) in \secref{db}. 
%%%
%%We then outline the steps
%%required to take in order to generate a VDB and \TODO{introduce
%%a semi-automatic approach for generating a VDB from database 
%%variants accompanied with their associated selected features} in
%%\secref{gen-vdb}. 
%%%
%%We also provide a set of variational queries
%%that capture a wide range of information needs for the introduced
%%use cases in \secref{q}. 
%%%Additionally, 
%%%we distribute the queries not only in our encodings,
%%%but also as simple \cpp{ifdef}-annotated SQL files to promote their broad reuse
%%%in the design and evaluation of other systems for managing variational
%%%relational data.
%%%
%%We distribute the VDBs, SQL scripts for generating them, 
%%and queries of our case studies in a GitHub repository available \href{https://github.com/lambda-land/VDBMS/tree/master/usecases}{here} for public use. 
%%%
%%We distribute the VDBs in either MySQL or Postgres and
%%in two form, one with embedded variational 
%%schema described in \secref{enron-vsch}, and one without the embedded schema
%%for use with our VDBMS tool in which the variational schema is provided
%%and distributed separately.
%%%
%%We distribute the queries not only in our encodings,
%%but also as simple \cpp{ifdef}-annotated SQL files to promote their broad reuse
%%in the design and evaluation of other systems for managing variational
%%relational data.
%%%
%%Finally, we thoroughly discuss the pros and cons of explicitly
%%encoding variation in databases and future research directions
%%in \secref{dis}.
%%\end{comment}
%%
%%%Thus, a generic explicit encoding of variation in databases, both
%%%in the schema and content, seems beneficial. We introduce this
%%%encoding in \secref{background}. 
%%%
%%%Our contributions in this paper are:
%%%\begin{itemize} [leftmargin=*]
%%%\itemsep0em
%%%%
%%%\item To illustrate the applicability of VDB
%%%we explain in details how we systematically 
%%%generated two VDBs by combining existing, widely
%%%used datasets with variation scenarios described in the literature 
%%%in \secref{db}. 
%%%To illustrate the generality of our encoding the 
%%%first case study focuses on variation in space
%%%while the second one focuses on variation in time.
%%%%
%%%\item We outline required steps to generate a VDB and 
%%%\TODO{provide a semi-automatic technique for generating a VDB from a 
%%%set of database variant with their corresponding selected features} 
%%%in \secref{gen-vdb}.
%%%%
%%%\item We provide a set of variational queries
%%%that captures the information needs of different database variants
%%%for each case study
%%%to demonstrate the use of our query language in \secref{q}.
%%%%
%%%However, we distribute the queries not only in our encodings,
%%%but also as simple \cpp{ifdef}-annotated SQL files to promote their broad reuse
%%%in the design and evaluation of other systems for managing variational
%%%relational data.
%%%%
%%%\item We thoroughly discuss the pros and cons of explicitly
%%%encoding variation in databases and future research directions
%%%in \secref{dis}.
%%%\end{itemize}
%%
%%%\begin{comment}
%%%In this paper, we present two case studies that demonstrate the need for a more
%%%systematic approach to managing variation in relational databases than the
%%%current status quo. Additionally, these case studies support the evaluation of
%%%systems like VDBMS that are intended to address this need.
%%%%
%%%The case studies were created by systematically combining existing, widely used
%%%datasets with software variation scenarios described in the literature.
%%%%
%%%Each case study consists of (1) a variational database whose structure is
%%%defined by a \emph{variational schema} and whose content is given by a set of
%%%\emph{variational tables}, and (2) a set of variational queries over this
%%%database that captures the information needs of different variants of the
%%%corresponding software system.
%%
%%
%%%We give a brief overview of variational databases and queries in
%%%Section~\ref{sec:background} and use the notations and terminology developed in
%%%that section throughout the paper.
%%%
%%%However, we distribute the case studies themselves not only in our encodings,
%%%but also as simple \cpp{ifdef}-annotated SQL files to promote their broad reuse
%%%in the design and evaluation of other systems for managing variational
%%%relational data.
%%
%%
%%%The variational databases for each case study are presented in
%%%Section~\ref{sec:db}.
%%%%
%%%The first case study focuses on variation over space. It develops a variational
%%%schema that captures the information needs of a SPL based on Hall's
%%%decomposition of an email system into its component features~\cite{Hall05}. The
%%%email SPL has been used in several previous SPL research projects (e.g.\
%%%\cite{Apel13:SSP,AlHaj19}). The variational email database is populated using
%%%the Enron email dataset, adapted to fit our variational schema~\cite{Shetty04}.
%%
%%
%%%The second case study demonstrates the changing information needs of a system
%%%that varies over time by developing a variational schema corresponding to an
%%%employee-system evolution scenario described in \citet{prima08Moon}. The
%%%variational employee database is populated by adapting a large, fabricated
%%%employee dataset\footnote{\url{https://github.com/datacharmer/test_db}} that
%%%has been widely used in the databases community.
%%
%%
%%%After describing the development of the database for each case study, we give a
%%%brief statistical overview of their structure and contents, and also identify
%%%some basic validation properties that all variational databases (including the
%%%two examples we provide) should satisfy.
%%
%%
%%%The variational queries for each case study are presented in
%%%Section~\ref{sec:q}. For each case study, we systematically worked through each
%%%scenario, as described in the corresponding paper, and created a query for each
%%%information need that was described.
%%%%
%%%We describe how the queries were developed and give several examples.
%%%\end{comment}
%
%
%
%
%

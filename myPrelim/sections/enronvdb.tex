\subsection{Variation in Space: Email SPL Case Study}
\label{app:enron-vdb}


%In our first case study, we focus on variation that occurs in ``space'', that
%is, where multiple software variants are developed and maintained in parallel.
%In software, variation in space corresponds to a SPL, where many distinct
%variants (products) can be produced from a single shared code base by enabling
%or disabling features. A variety of representations and tools have been
%developed for indicating which code belongs to which feature(s) and supporting
%the process of configuring a SPL to obtain a particular variant.


% E: This is repetitive with the intro, but I think it's OK to remind them here
% of what the current bad solutions are. Plus, we have space to fill. :-)

%Naturally, different variants of a SPL have different information needs. For
%example, an optional feature in the SPL may require a corresponding attribute
%or relation in the database that is not needed by the other features in the
%SPL.
%%
%Currently, there is no good solution to managing the varying information needs
%of different variants at the level of the database.
%%
%One possible solution is to manually maintain a separate database schema for
%each variant of the SPL. This works for some SPLs where the number of products
%is relatively small and the developer has control over the configuration
%process. However, it does not scale to open-source SPLs or other scenarios
%where the number of products is large and/or configuration is out of the
%developer's hands.
%%
%Another possible solution is to use and maintain a single universal schema that
%includes all of the relations and attributes used by any feature in the SPL. In
%this solution, every product will use the same database schema regardless of
%the features that are enabled. This solves the problem of scaling to large
%numbers of products but is dangerous because it means that potentially several
%attributes and relations will be unused in any given product. Unused attributes
%will typically be populated by NULL values, which are a well-known source of
%errors in relational databases~\cite{AliceBook}.


%VDBs solve the problem by allowing the structure of a relational database to
%vary in a synchronous way with the SPL. Attributes and relations may be
%annotated by presence conditions to indicate in which feature(s) those
%attributes and relations are needed.
%%
%An implementation of the VDB model might use a universal schema under-the-hood
%to realize VDBs on top of a standard relational database management system
%(indeed, this is exactly how our prototype VDBMS implementation works), but by
%capturing the variation in the schema explicitly, we can validate (potentially
%variational) queries against the relevant variants of the variational schema to
%statically ensure that no NULL values will be referenced.

In our first case study, we focus on variation in ``space''.
It shows the use of VDB to encode the variational
information needs of a database-backed SPL. We consider an email
SPL which has been used in several previous SPL research projects (e.g.\
\cite{Apel13:SSP,AlHaj19}).
%It develops a variational
%schema that captures the information needs of a SPL based on Hall's
%decomposition of an email system into its component features~\cite{Hall05}. The
%email SPL has been used in several previous SPL research projects (e.g.\
%\cite{Apel13:SSP,AlHaj19}). The variational email database is populated using
%the Enron email dataset, adapted to fit our variational schema~\cite{Shetty04}.
%
%Our first case study demonstrates the use of a VDB to encode the variational
%information needs of a database-backed SPL.
%
Our case study is formed by systematically combining two pre-existing works:
%
\begin{enumerate}
%
\item 
(1) We use Hall's decomposition of an email system into its component
features~\cite{Hall05} as high-level specification of a SPL.
%
\item 
(2) We use the Enron email
dataset\footnote{\url{http://www.ahschulz.de/enron-email-data/}} as 
%a source of
a realistic email database.
%
\end{enumerate}
%
In combining these works, we show how variation in space in an email SPL
requires corresponding variation in a supporting database, how we can link the
variation in the software to variation in the database, and how all of these
variants can be encoded in a single VDB.


\subsubsection{Variation Scenario: An Email SPL}
\label{sec:enron-scenario}


The email SPL consists of the following features from \citet{Hall05}:
%
\begin{itemize}
%[leftmargin=*]
%\itemsep0em
%
\item 
\addressbook: Users can maintain lists of known email addresses with
corresponding aliases, which may be used in place of recipient addresses.
%
\item 
\signature: Messages may be digitally signed and verified using
cryptographic keys.
%
\item 
\encryption: Messages may be encrypted before sending and decrypted upon
receipt using cryptographic keys.
%
\item 
\autoresponder: Users can enable automatically generated email responses
to incoming messages.
%
\item 
\forwardmessages: Users can forward all incoming messages automatically to another
address.
%
\item 
\remailmessage: Users may send messages anonymously.
%
\item 
\filtermessages: Incoming messages can be filtered according to a
provided white list of known sender address suffixes.
%
\item 
\mailhost: A list of known users is maintained and known users may
retrieve messages on demand. Messages sent to unknown users are rejected.
%
\end{itemize}

%\noindent
%
Note that 
Hall's decomposition separates \signature\ and \encryption\ into two
features each (corresponding to signing and verifying, encrypting and
decrypting). Since these pairs of features must always be enabled together
 and
they are so closely conceptually related, 
we 
reduce them to one feature each for simplicity.


The listed features are used in presence conditions within the
v-schema for the email VDB, linking the software variation to
variation in the database.
%
In the email SPL, each feature is optional and independent, resulting in the
  simple feature model $\fModel_\enron = \t$.
% , which is equivalent to \t, 
%given as a feature expression.
%%\eric{should we just consider this feature model as true? that is how 
%%we're implementing it.}
%%
%\begin{align*}
%\fModel_\enron
%  &= \t \vee \addressbook \vee \signature \vee \encryption \\
%  &\quad \vee \autoresponder \vee \forwardmessages \\
%  &\quad \vee \remailmessage \vee \filtermessages \vee \mailhost
%\end{align*}
%
%The feature model $\fModel_\enron$ is used as the root presence condition of
%the variational schema for the email VDB, implicitly applying it to all
%relations, attributes, and tuples in the database.

%For the rest of this section, we study a SPL that 
%generates email systems with different features and demonstrate
%how variation appears in its database in addition to how all variants
%of a database can be encoded in a single variational database. We adapt the
%email system explained in~\cite{} by considering the following features:
%...
%
%We consider the simple feature model of:
%...
%The feature model is part of the variational schema too and is applied
%to all its tables and attributes. 

\subsubsection{Generating V-Schema of the Email SPL VDB}
\label{sec:enron-vsch}

\begin{table}
\caption{Original Enron email dataset schema.}
%\vspace{-8pt}
\label{tab:enron}
\begin{center}
\small
\begin{tabular} {|l|}
%\hline
%\textbf{Enron Schema} \\
\hline 
\employees(\eid, \fname, \lname, \emailid, $\mathit{email2}$, 
%\hspace{40pt}
 $\mathit{email3}$, $\mathit{email4}$, \folder, \status) \\
\messages(\midatt, \sender, \dateatt, \messageid, \subject, \body, \folder)  \\ 
\recipientinfo(\rid, \midatt, \rtype, \rvalue)  \\
\referenceinfo(\rid, \midatt, \reference)  \\
\hline
\end{tabular}
\end{center}
%\vspace{-13pt}
\end{table}


To produce a v-schema for the email VDB, we start from plain schema
of the Enron email dataset shown in \tabref{enron}, then systematically adjust
its schema to align with the information needs of the email SPL described by
\citet{Hall05}.
%
The \employees\ table contains information about the employees of the company
including the employee identification number (\eid), their first name and last
name (\fname\ and \lname), their primary email address (\emailid), alternative
email addresses (e.g.\ $\mathit{email2}$), a path to the folder that contains
their data (\folder), and their last status in the company (\status).
%
The \messages\ table contains information about the email messages
 including
the message ID (\midatt), the sender of the message (\sender), the date
(\dateatt), the internal message ID (\messageid), the subject and body of the
message (\subject\ and \body), and the exact folder of the email (\folder).
% 
The \recipientinfo\ table contains information about the recipient of a message
including the recipient ID (\rid), the message ID (\midatt), the type of the
message (\rtype), and the email address of the recipient (\rvalue).
%
The \referenceinfo\ table contains messages that have been referenced in other
email messages,
for example, in a forwarded message; it contains a
reference-info ID (\rid), the message ID (\midatt), and the entire message
(\reference). 
This table simply backs up the emails.


\begin{table}
\caption{V-schema of the email VDB with feature model
\ensuremath{\fModel_\enron}. 
Presence conditions are colored blue for clarity.
}
%\vspace{-8pt}
\label{tab:enron-vsch}
 \begin{center}
\small
\begin{tabular} {|l|}
%\hline
%\textbf{Variational Schema for Email SPL} \\
\hline 
 % \annot [\vFour] \name
$\employees(\eid, \fname, \lname, \emailid, \folder, \status,$ 
%\\ \hspace{40pt} 
$\annot[\fsignature]{\verificationkey},
  \annot[\fencryption]{\publickey})$\\
%   \hspace{20pt}
$\messages(\midatt, \sender, \dateatt, \messageid, \subject, \body, \folder,$ 
%\\ \hspace{30pt} 
$\issystemnotification,
  \annot[\fencryption]{\isencrypted}$\\
\hspace{37pt} $ ,\annot[\fautoresponder]{\isautoresponse},\annot[\fsignature]{\issigned} $ 
%  \\ \hspace{30pt} 
  $,
  \annot[\fforwardmsg]{\isforwardmsg})$  \\
 $\recipientinfo(\rid, \midatt, \rtype, \rvalue)$ 
  \\ %[1.1ex]
%  \hspace{40pt}
%   \\ %[1.1ex]
%$\recipientinfo(\rid, \midatt, \rtype, \rvalue)$   
${\forwardmsg(\eid, \forwardaddr)}^{\fforwardmsg} $\\
%\hspace{40pt}
${\mailhost(\eid, \username, \mailhost)}^{\fmailhost}$ 
\\
%\hspace{20pt}
% \\ %[1.1ex]
%\referenceinfo(\rid, \midatt, \reference)  \\
  ${\filtermsg(\eid, \suffix)}^{\ffiltermsg} $ \\
%  \hspace{3pt}
%\\ %[1.1ex]
%  \hspace{47pt}
%\\ %[1.1ex]
${\remailmsg(\eid, \pseudonym)}^{\fremailmsg}$  \\
%  \hspace{3pt}
%\hspace{20pt}
%\\ %[1.1ex]
${\automsg(\eid, \subject, \body)}^{\fautoresponder} $ \\
% \hspace{3pt}
%\\ %[1.1ex]
${\alias(\eid, \emailAtt, \nickname)}^{\faddressbook}$\\
%\\ %[1.1ex]
%${\filtermsg(\eid, \suffix)}^{\ffiltermsg} $\\ %[1.1ex]
\hline
\end{tabular}
%\vspace{-13pt}
 \end{center}
\end{table}


From this starting point, we introduce new attributes and relations that are
needed to implement the features in the email SPL. We attach presence
conditions to new attributes and relations corresponding to the features
they are needed to support, which ensure they will \emph{not} be present
in configurations that do not include the relevant features.
%
The resulting v-schema is given in \tabref{enron-vsch}.


%As an example of this process, 
For example, consider the \signature\ feature. In the
software, implementing this feature requires new operations for signing an
email before sending it out and for verifying the signature of a received
email. These new operations suggest new information needs: we need a way to
indicate that a message has been signed, and we need access to each user's
public key to verify those signatures (private keys used to sign a message
would not be stored in the database). These  needs are reflected in
the v-schema by the new attributes \verificationkey\ and \issigned,
added to the relations \employees\ and \messages, respectively. The new
attributes are annotated by the \signature\ presence condition, indicating that
they correspond to the \signature\ feature and are unused in configurations
that exclude this feature.
%
Additionally, several features require adding entirely new relations, e.g.,
%
%For example, 
when the \forwardmsg\ feature is enabled, the system must keep
track of which users have forwarding enabled and the address to forward the messages to.
%should be forwarded to. 
This need is reflected by the new
\forwardmsg\ relation, which is correspondingly annotated by the \forwardmsg\
presence condition.


A main focus of Hall's decomposition~\cite{Hall05} is on the many feature
interactions.
% in the email SPL. 
Several of the features may interact in
undesirable ways if special precautions are not taken. For example, any
combination of the \forwardmsg, \remailmsg, and \autoresponder\ features can
trigger an infinite messaging loop if users configure the features in the wrong
way; preventing this creates an information need to identify auto-generated
emails, which is realized in the variational schema by attributes like
\isforwardmsg\ and \isautoresponse.
%
% occurs between the \signature\ and \fremailmsg\ features: the \fremailmsg\
% feature enables anonymously sending messages by replacing the sender with a
% pseudonym, but this prevents the recipient from being able to verify a signed
% email.
%
% occurs between the \signature\ and \forwardmsg\ features: if Sarah signs a
% message and sends it to Ina, and Ina forwards the message to Philippe, then
% the signature verification operation may incorrectly interpret Ina as the
% sender rather than Sarah and fail to verify the message.

%p: moved the following paragraph to discussion.
%For each feature, we (1) enumerated the operations that must be supported both
%to implement the feature itself and to resolve undesirable feature
%interactions, (2) identified the information needs to implement these
%operations, and (3) extended the variational schema to satisfy these
%information needs.
%
% The changes made to accommodate the features \addressbook, \encryption,
% \autoresponder, \remailmessage, \filtermessages, and \mailhost\ are similar. 
%

% are provided in the
%second author's MS project report~\citet{Li19}.
%
For brevity, we omit some attributes and relations from the original schema
that are  irrelevant to the email SPL as described by Hall, such as the
\referenceinfo\ relation and alternative email addresses. 
%
%p: will ref qiaoran's work somewhere else, if space permits.
%\citet{Li19} provides
%a complete description of adaptations done in the process in her MS project report.
% \t presence condition can be omitted.


%We distribute the variational schema for the email VDB in two formats.
%%
%First, we provide the schema in the encoding used by our prototype VDBMS tool.%
%\footnote{\href{https://github.com/lambda-land/VDBMS/blob/master/usecases/space-emailSPL/schema/EnronSchema.hs}{usecases/space-emailSPL/schema}}
%
%Second, 
In addition to providing the schema in the encoding used by our 
prototype VDBMS tool, we also provide a direct encoding in SQL
%The SQL encoding 
which generates the
%\eric{Eric, I don't like universal here! it may give the wrong idea!
%maybe just say v-schema?}
universal schema for the VDB.
% in either MySQL or Postgres.%
%\footnote{\href{https://github.com/lambda-land/VDBMS/tree/master/usecases/space-emailSPL/database/create}{usecases/space-emailSPL/database/create}}
%
Variation is encoded as an additional relation of the form \vdbpc\ that
captures all of the relevant presence conditions: that of the 
v-schema itself (i.e.\ the feature model), and those of each relation and
attribute.%
%\footnote{\href{https://github.com/lambda-land/VDBMS/tree/master/usecases/space-emailSPL/database/withSchema}{usecases/space-emailSPL/database/withSchema}}
%
The $\mathit{element\_id}$ of the feature model is
$\mathit{variational\_schema}$; the $\mathit{element\_id}$ of a relation \vRel\
is its name \vRel, and of attribute \vAtt\ in relation \vRel\ is $\vRel.\vAtt$.
%
The plain SQL encoding of the v-schema supports the use
of the case studies for research on the effective management of variation in
databases independent of VDBMS.

%feature
%model of the variational schema, a relation \tablespc\ that stores the presence
%condition of each relation, and for each table $R\left(a_1, \cdots, a_n\right)$
%in the variational schema there is a table $t\left(\mathit{attribute\_name},
%\mathit{pres\_cond}\right)$ in the database that stores its attributes'
%presence conditions.

\subsection{Populating the Email SPL VDB}
\label{sec:enron-pop}

The final step to create the email VDB is to populate the database with data
from the Enron email dataset, adapted to fit our variational schema~\cite{Shetty04}.
%
For evaluation purposes, we want the data from the dataset to be distributed
across multiple variants of the VDB. To simulate this, we identified five
plausible configurations of the email SPL, which we divide the data among. The
five configurations of the email SPL 
 we considered are:
%
\begin{itemize}
%
\item 
\emph{basic email}, which includes only basic email functionality
and does not include any of the optional features
 from the SPL.
%
\item 
\emph{enhanced email}, which extends \emph{basic email} by
enabling two of the most commonly used email features, \forwardmessages\ and
\filtermessages.
%
\item 
\emph{privacy-focused email}, which extends \emph{basic email}
with features that focus on privacy, specifically, the
\signature, \encryption, and \remailmessage\ features.
%
\item 
\emph{business email}, which extends \emph{basic email} with
features tailored to an environment where most emails are expected to be among
users within the same business network, specifically, \addressbook,
\signature, \encryption, \autoresponder, and \mailhost.
%
\item 
\emph{premium email}, in which all of the optional features
in the SPL are enabled.
%
\end{itemize}
%
For all variants, any features that are not enabled are disabled. 


The original Enron dataset has 150 employees with 252,759 email messages. 
%
We load this data into the \employees\ and \messages\ tables defined in
\secref{enron-vsch}, initializing all attributes that are not present in the
original dataset to \nul.


For the \employees\ table, we construct five views corresponding to the five
variants of the email system described above. We allocate 30 employees to each
view based on their employee ID, that is, the first 30 employees sorted by
employee ID are associated with the basic email variant, the next 30 with the
enhanced email variant, and so on. The presence condition for each tuple is set
to the conjunction of features enabled in that view.
%
We then modify each of the views of the \employees\ table by adding randomly
generated values for attributes associated with the enabled features; 
e.g., in the view for the privacy-focused variant, we populate the
\verificationkey\ and \publickey\ attributes.
%
Any attribute that is not present in the given tuple due to a conflicting
presence condition will remain \nul. For example, both the \verificationkey\
and \publickey\ attributes remain \nul\ for employees in the enhanced variant
view since the presence condition does not include the corresponding features.


For the \messages\ table, we again create five views corresponding to each of
the variants. Each tuple is added to the view of the variant that contains the
message's sender, which updates the tuple's presence condition accordingly.
%
The \messages\ table also contains several additional attributes corresponding
to optional features, which we populate in a systematic way.
%
We set \issigned\ to \t\ if the message sender has the \signature\ feature
enabled, and we set \isencrypted\ to \t\ if \emph{both} the message sender
and recipient have \encryption\ enabled.
%
We populate the \isforwardmsg, \isautoresponse, and \issystemnotification\
attributes by doing a lightweight analysis of message subjects to determine
whether the email is any of these special kinds of messages; for example, if
the subject begins with ``FWD'', we set the \isforwardmsg\ attribute to \t.
%
If a forward or auto-reply message was sent by a user that does not have the
corresponding feature enabled, we filter it out of the dataset. After
filtering, the \messages\ relation contains 99,727 messages.
%
For each forward or auto-reply message, we also add a tuple with the relevant
information to the new \forwardmsg\ and \automsg\ tables.
%
For employees belonging to database variants that enable \remailmessage,
\autoresponder, \addressbook, or \mailhost\ we randomly generate tuples in the
tables that are specific to each of these features.
%
Finally, the \recipientinfo\ relation is imported directly from the dataset. We
set each tuple's presence condition to a conjunction of the presence conditions
of the sender and recipient.
% \eric{actually I conjuncted them, which now I'm not so sure about it!}


%We provide more detailed instructions for systematically constructing the email
%VDB in both MySQL and PostgreSQL in a wiki page associated with the
%repository.%
%\footnote{\url{https://github.com/lambda-land/VDBMS/wiki/Enron-Email-Database-UseCase-Doc\#steps-take-to-build-vdb-for-enron-case-study}}
%%
We provide SQL scripts to automate the creation of views for each
variant%
%
%\footnote{\href{https://github.com/lambda-land/VDBMS/blob/master/usecases/space-emailSPL/database/build/step1_build_email_variants.sql}{usecases/space-emailSPL/database/build/step1\_build\_email\_variants.sql}}
%%
and to automate the population of these views with tuples from the original
dataset,%
%%
%\footnote{\href{https://github.com/lambda-land/VDBMS/blob/master/usecases/space-emailSPL/database/build/step2_build_email_vdb.sql}{usecases/space-emailSPL/database/build/step2\_build\_email\_vdb.sql}} 
%%
which also sets each tuple's presence condition.
%
%
The resulting database is distributed in two forms, one with the embedded
variational schema which is described in \secref{enron-vsch},%
%
%\footnote{\href{https://github.com/lambda-land/VDBMS/tree/master/usecases/space-emailSPL/database/withSchema}{usecases/space-emailSPL/database/withSchema}}
%
and one without the embedded schema%
%\footnote{\href{https://github.com/lambda-land/VDBMS/tree/master/usecases/space-emailSPL/database/withoutSchema}{usecases/space-emailSPL/database/withoutSchema}}
%
for use with our VDBMS tool in which the variational schema is provided
separately.%
\footnote{Both the scripts and different encodings of the email SPL VDB are
available at: \url{https://zenodo.org/record/4321921}.} 
%
We have tested the email SPL VDB for the properties described in \secref{vdbfprop} 
and all of them hold.

%\point{VDBs are a good fit to encode variation in a database of a 
%database-backed SPL.}
%%variation over space
%
%
%\point{To showcase how a VDB captures different needs arising in SPLs 
%w.r.t. variation in their databases we generate a VDB for an email SPL.}\\
%- where data is coming from\\
%- where spl is coming from
%
%\point{We adopt the features and feature model from the SPL and 
%consider five database variants.}\\
%- describe 5 variants and their config.
%
%\point{Features are incorporated into both the schema and data.}\\
%- feature model\\
%- schema\\
%- data\\


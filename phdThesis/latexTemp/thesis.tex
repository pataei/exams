
\documentclass[onehalf,11pt]{beavtex}
\title{Theory and Implementation of a Variational Database Management System}
\author{Parisa Ataei}
\degree{Doctor of Philosophy}
\doctype{Dissertation}
\department{Electrical Engineering and Computer Science}
\depttype{School}
\depthead{Director}
\major{Computer Science}
\advisor{Eric Walkingshaw}
\submitdate{June ?, 2021}
\commencementyear{2021}
%
\abstract{
Databases used in the same context that are intended to satisfy the same information need
share commonalities while varying in some aspects as well. 
% 
For example, databases used to store information for a software system are intended to 
satisfy the client's information need and they share some parts of the schema, however, 
they still vary in some parts of the schema and most of the content.
%
That is, they are \emph{variants} of a conceptual hypothetical database that captures all 
the \emph{variation} among them.
%
This pattern appears repeatedly in databases. Instances of variation occurring in databases
are: schema evolution, database integration, database versioning, data extraction, and 
software development either using software product lines approaches or not. 
%
While there are specialized approaches to some instances of variation in databases
there are no generic solutions to manage variation in databases.

In this thesis, we answer the question: ``can we abstract out this repeating pattern of 
variation appearing in databases and bring the hypothetical database that captures all
the variation among a number of databases to life?''
%
To this end, we formalize the database that captures all variation of a number of databases,
called a \emph{variational database}, and a query language that allows interaction with
the said database, called \emph{variational relational algebra}. 
\TODO{maybe two languages, another: variational SQL, depends on my timing!}
%
We implement these concepts in \emph{Variational Database Management System (VDBMS)},
demonstrate the feasibility of our concepts by developing two real-world use cases, and 
examine the performance of VDBMS over our two use cases. 
}
%
\acknowledgements{
\TODO{Eric.
Committee.
jeff.
abu.
parents.
friends.
}
}

\usepackage{algorithm}
\usepackage{algorithmic}

\usepackage{ebproof} %for derivation trees
\usepackage{lscape} %to force a page to be in landscape mode
\usepackage{graphicx}
\usepackage{balance}  % for  \balance command ON LAST PAGE  (only there!)
\usepackage{comment}
\usepackage{accents}
\usepackage{lambda,cc}
%\usepackage{mathtools} 
\usepackage{array,multirow}
%\usepackage{hhline}
\usepackage{arydshln}
%\usepackage{color}
%\usepackage[usenames, dvipsnames]{xcolor}
\usepackage{xcolor}
%\usepackage{colortbl}
%\usepackage{booktabs}
\usepackage{mathpartir}
\usepackage{hyperref}
\usepackage[normalem]{ulem}
% commuting diagram packages
\usepackage{tikz}
\usepackage{tikz-cd}
\usetikzlibrary{decorations.pathmorphing}
\usepackage{wrapfig}
%\usepackage{blindtext}% for example text here only
\usepackage{enumitem,kantlipsum} %remove enumerate indent
\usepackage{amssymb}%for arrow labels

\usepackage{listings} % for v-sql example code
%
% references
\newcommand{\tabref}[1]{\hyperref[tab:#1]{Table~\ref*{tab:#1}}}
\newcommand{\figref}[1]{\hyperref[fig:#1]{Figure~\ref*{fig:#1}}}
\newcommand{\secref}[1]{\hyperref[sec:#1]{Section~\ref*{sec:#1}}}
\newcommand{\defref}[1]{\hyperref[def:#1]{Definition~\ref*{def:#1}}}
\newcommand{\appref}[1]{\hyperref[app:#1]{Appendix~\ref*{app:#1}}}
\newcommand{\chref}[1]{\hyperref[ch:#1]{Chapter~\ref*{ch:#1}}}
\newcommand{\thmref}[1]{\hyperref[thm:#1]{Theorem~\ref*{thm:#1}}}
\newcommand{\lemref}[1]{\hyperref[lem:#1]{Lemma~\ref*{lem:#1}}}
\newcommand{\exref}[1]{\hyperref[ex:#1]{Example~\ref*{eg:#1}}}
%
%%colors
\definecolor{deepcarminepink}{rgb}{0.94, 0.19, 0.22}%all shared
\definecolor{mediumelectricblue}{rgb}{0.01, 0.31, 0.59}%middlename
\definecolor{frenchblue}{rgb}{0.0, 0.45, 0.73}%lastname
\definecolor{green(munsell)}{rgb}{0.0, 0.66, 0.47}%iceland
\definecolor{violet(ryb)}{rgb}{0.53, 0.0, 0.69}%us,invest
\definecolor{navyblue}{rgb}{0.0, 0.0, 0.5}%v-table
\definecolor{persimmon}{rgb}{0.93, 0.35, 0.0}%iran
\definecolor{Plum}{rgb}{0.78, 0.08, 0.52}%us,bank
\definecolor{ruby}{rgb}{0.88, 0.07, 0.37}%us,bank

\definecolor{light-gray}{gray}{0.95}
\newcommand{\code}[1]{\colorbox{light-gray}{\texttt{#1}}}

%names
%approaches
\newcommand{\nbf}{NBF}
\newcommand{\ubf}{UBF}
\newcommand{\uav}{UAV}

%shorthands
\newcommand{\ChcExpTxt}{Choice expression}
\newcommand{\DimTxt}{Choice dimension}
\newcommand{\ConfigTxt}{Configuration}
\newcommand{\vdbTxt}{variational database}
\newcommand{\VdbTxt}{Variational database}
\newcommand{\vdbInstTxt}{variational database instance}
\newcommand{\VdbInstTxt}{Variational database instance}
\newcommand{\vqTxt}{variational query}
\newcommand{\VqTxt}{Variational query}
\newcommand{\vqsTxt}{variational queries}
\newcommand{\VqsTxt}{Variational queries}
\newcommand{\vschTxt}{variational schema}
\newcommand{\VschTxt}{Variational schema}
\newcommand{\tenvTxt}{type}
\newcommand{\TenvTxt}{Type}
\newcommand{\vctxTxt}{variation context}
\newcommand{\VctxTxt}{Variation context}
\newcommand{\vRelSchTxt}{variational relation schema}
\newcommand{\VrelSchTxt}{Variational relation schema}
\newcommand{\vrelTxt}{variational relation}
\newcommand{\VrelTxt}{Variational relation}
\newcommand{\presCondTxt}{presence condition}
\newcommand{\PresCondTxt}{Presence condition}
\newcommand{\fexpTxt}{feature expression}
\newcommand{\FexpTxt}{Feature expression}
\newcommand{\vCondTxt}{variational condition}
\newcommand{\VcondTxt}{Variational condition}
\newcommand{\vAttListTxt}{variational attribute list}
\newcommand{\VattListTxt}{Variational attribute list}
\newcommand{\optAttTxt}{optional attribute}
\newcommand{\OptAttTxt}{Optional attribute}
\newcommand{\VlistTxt}{Variational list}
\newcommand{\VsetTxt}{Variational set}
\newcommand{\AnnotVsetTxt}{Annotated variational set}
\newcommand{\AnnotVlistTxt}{Annotated variational list}
\newcommand{\VobjTxt}{Variational object}
\newcommand{\AttTxt}{Attribute}
\newcommand{\RelSchTxt}{Relation schema}
\newcommand{\SchTxt}{Schema}
\newcommand{\RelTxt}{Relation}
\newcommand{\DbInstTxt}{Database instance}
\newcommand{\DbTxt}{Database}
\newcommand{\CondTxt}{Condition}
\newcommand{\vrelAlgTxt}{variational relational algebra}
\newcommand{\VrelAlgTxt}{variational relational algebra}
\newcommand{\typeSys}{type system}
\newcommand{\varPres}{variation preserving}
\newcommand{\VarPres}{Variation preserving}
\newcommand{\FTxt}{Feature}
\newcommand{\QTxt}{Query}
%\newcommand{\VcondTxt}{Variational condition}
\newcommand{\fctTxt}{true feature set}
\newcommand{\osemTxt}{selection semantics /selection /semantics /configuration semantics/ configuring}

%needs names
\newcommand{\nZero}{\textbf{N0}}
\newcommand{\nOne}{\textbf{N1}}
\newcommand{\nTwo}{\textbf{N2}}
\newcommand{\nThree}{\textbf{N3}}
\newcommand{\basic}{\OB{\mathsf{basic}}}
\newcommand{\educational}{\OB{\mathsf{educational}}}

%motivating ex names
\newcommand{\employee}{\OB{\mathit{emp}}}
\newcommand{\eng}{\OB{\mathit{eng}}}
\newcommand{\edu}{\OB{\mathit{edu}}}
\newcommand{\base}{\OB{\mathit{base}}}
\newcommand{\oncamp}{\OB{\mathit{onCamp}}}
\newcommand{\online}{\OB{\mathit{onLine}}}
\newcommand{\vOne}{\OB{V_1}}
\newcommand{\vTwo}{\OB{V_2}}
\newcommand{\vThree}{\OB{V_3}}
\newcommand{\vFour}{\OB{V_4}}
\newcommand{\vFive}{\OB{V_5}}
\newcommand{\tOne}{\OB{T_1}}
\newcommand{\tTwo}{\OB{T_2}}
\newcommand{\tThree}{\OB{T_3}}
\newcommand{\tFour}{\OB{T_4}}
\newcommand{\tFive}{\OB{T_5}}
\newcommand{\fOne}{\OB{f_1}}
\newcommand{\fTwo}{\OB{f_2}}
\newcommand{\engemp}{\OB{\mathit{engineerpersonnel}}}
\newcommand{\othemp}{\OB{\mathit{otherpersonnel}}}
\newcommand{\pers}{\OB{\mathit{personnel}}}
\newcommand{\empacct}{\OB{\mathit{empacct}}}
\newcommand{\empno}{\OB{\mathit{empno}}}
\newcommand{\name}{\OB{\mathit{name}}}
\newcommand{\hiredate}{\OB{\mathit{hiredate}}}
\newcommand{\titleatt}{\OB{\mathit{title}}}
\newcommand{\deptname}{\OB{\mathit{deptname}}}
\newcommand{\salary}{\OB{\mathit{salary}}}
\newcommand{\job}{\OB{\mathit{job}}}
\newcommand{\deptno}{\OB{\mathit{deptno}}}
\newcommand{\dept}{\OB{\mathit{dept}}}
\newcommand{\managerno}{\OB{\mathit{managerno}}}
\newcommand{\empbio}{\OB{\mathit{empbio}}}
\newcommand{\sex}{\OB{\mathit{sex}}}
\newcommand{\birthdate}{\OB{\mathit{birthdate}}}
\newcommand{\fname}{\OB{\mathit{firstname}}}
\newcommand{\lname}{\OB{\mathit{lastname}}}
\newcommand{\course}{\OB{\mathit{course}}}
\newcommand{\ecourse}{\OB{\mathit{ecourse}}}
\newcommand{\student}{\OB{\mathit{student}}}
\newcommand{\teacher}{\OB{\mathit{teacher}}}
\newcommand{\cname}{\OB{\mathit{coursename}}}
\newcommand{\tno}{\OB{\mathit{teacherno}}}
\newcommand{\sno}{\OB{\mathit{studentno}}}
\newcommand{\cno}{\OB{\mathit{courseno}}}
\newcommand{\class}{\OB{\mathit{class}}}
\newcommand{\timeatt}{\OB{\mathit{time}}}
\newcommand{\teach}{\OB{\mathit{teach}}}
\newcommand{\take}{\OB{\mathit{take}}}
\newcommand{\grade}{\OB{\mathit{grade}}}
\newcommand{\isstudent}{\OB{\mathit{std}}}
\newcommand{\isteacher}{\OB{\mathit{instr}}}
\newcommand{\studentnum}{\OB{\mathit{stdnum}}}
\newcommand{\teachernum}{\OB{\mathit{instrnum}}}


% enron email system example names
\newcommand{\enron}{\OB{\mathit{en}}}
\newcommand{\eid}{\OB{\mathit{eid}}}
\newcommand{\emailid}{\OB{\mathit{email\_id}}}
\newcommand{\folder}{\OB{\mathit{folder}}}
\newcommand{\status}{\OB{\mathit{status}}}
\newcommand{\verificationkey}{\OB{\mathit{verification\_key}}}
\newcommand{\publickey}{\OB{\mathit{public\_key}}}
\newcommand{\midatt}{\OB{\mathit{mid}}}
\newcommand{\sender}{\OB{\mathit{sender}}}
\newcommand{\dateatt}{\OB{\mathit{date}}}
\newcommand{\messageid}{\OB{\mathit{message\_id}}}
\newcommand{\subject}{\OB{\mathit{subject}}}
\newcommand{\body}{\OB{\mathit{body}}}
\newcommand{\issigned}{\OB{\mathit{is\_signed}}}
\newcommand{\isencrypted}{\OB{\mathit{is\_encrypted}}}
\newcommand{\isfromremailer}{\OB{\mathit{is\_from\_remaler}}}
\newcommand{\issystemnotification}{\OB{\mathit{is\_system\_notification}}}
\newcommand{\isautoresponse}{\OB{\mathit{is\_autoresponse}}}
\newcommand{\isforwardmsg}{\OB{\mathit{is\_forward\_msg}}}
\newcommand{\rid}{\OB{\mathit{rid}}}
\newcommand{\rtype}{\OB{\mathit{rtype}}}
\newcommand{\rvalue}{\OB{\mathit{rvalue}}}
\newcommand{\rfid}{\OB{\mathit{rfid}}}
\newcommand{\reference}{\OB{\mathit{reference}}}
\newcommand{\forwardaddr}{\OB{\mathit{forwardaddr}}}
\newcommand{\pseudonym}{\OB{\mathit{pseudonym}}}
\newcommand{\suffix}{\OB{\mathit{suffix}}}
\newcommand{\username}{\OB{\mathit{username}}}
\newcommand{\mailhost}{\OB{\mathit{mailhost}}}
\newcommand{\nickname}{\OB{\mathit{nickname}}}
\newcommand{\emailAtt}{\OB{\mathit{email}}}
\newcommand{\employees}{\OB{\mathit{employeelist}}}
\newcommand{\messages}{\OB{\mathit{messages}}}
\newcommand{\recipientinfo}{\OB{\mathit{recipientinfo}}}
\newcommand{\referenceinfo}{\OB{\mathit{referenceinfo}}}
\newcommand{\automsg}{\OB{\mathit{auto\_msg}}}
\newcommand{\forwardmsg}{\OB{\mathit{forward\_msg}}}
\newcommand{\remailmsg}{\OB{\mathit{remail\_msg}}}
\newcommand{\filtermsg}{\OB{\mathit{filter\_msg}}}
\newcommand{\alias}{\OB{\mathit{alias}}}
\newcommand{\id}{\OB{\mathit{id}}}
\newcommand{\scores}{\OB{\mathit{SATscores}}}
%\newcommand{\midatt}{\OB{\mathit{mid}}}
\newcommand{\midCond}{\OB{\midatt = \xvalue}}
\newcommand{\header}{\OB{\mathit{header}}}
\newcommand{\headerAtts}{\OB{\mathit{header\_attributes}}}
\newcommand{\verifiedAt}{\OB{\mathit{verified\_at}}}
\newcommand{\signedBy}{\OB{\mathit{signed\_by}}}
\newcommand{\recievedBy}{\OB{\mathit{recieved\_by}}}
\newcommand{\verStat}{\OB{\mathit{verification\_status}}}
\newcommand{\shouldFilter}{\OB{\mathit{should\_filter}}}

\newcommand{\addressbook}{\OB{\mathit{addressbook}}}
\newcommand{\signature}{\OB{\mathit{signature}}}
\newcommand{\encryption}{\OB{\mathit{encryption}}}
\newcommand{\autoresponder}{\OB{\mathit{autoresponder}}}
\newcommand{\forwardmessages}{\OB{\mathit{forwardmessages}}}
\newcommand{\remailmessage}{\OB{\mathit{remailmessage}}}
\newcommand{\filtermessages}{\OB{\mathit{filtermessages}}}
\newcommand{\basicq}{\OB{\mathit{basic}}}

\newcommand{\addressbookf}{\OB{\mathit{addressbook}}}
\newcommand{\signaturef}{\OB{\mathit{signature}}}
\newcommand{\encryptionf}{\OB{\mathit{encryption}}}
\newcommand{\autoresponderf}{\OB{\mathit{autoresponder}}}
\newcommand{\forwardmsgf}{\OB{\mathit{forwardmessages}}}
\newcommand{\remailmsgf}{\OB{\mathit{remailmessage}}}
\newcommand{\filtermsgf}{\OB{\mathit{filtermessages}}}
\newcommand{\mailhostf}{\OB{\mathit{mailhost}}}

% some specific queries
\newcommand{\Qbasic}{\OB{\nfq_\basicq}}
\newcommand{\Qfilter}{\OB{\nfq_\mathit{filter}}}
\newcommand{\Qforward}{\OB{\nfq_\mathit{forward}}}
\newcommand{\Qencrypt}{\OB{\nfq_\mathit{encrypt}}}
\newcommand{\Qsig}{\OB{\nfq_\mathit{sig}}}
\newcommand{\Qbf}{\OB{\nfq_\mathit{bf}}}
\newcommand{\Qsf}{\OB{\nfq_\mathit{sf}}}
\newcommand{\Qef}{\OB{\nfq_\mathit{ef}}}

\newcommand{\addressbooku}{{ADDRESSBOOK}}
\newcommand{\signatureu}{SIGNATURE}
\newcommand{\encryptionu}{ENCRYPTION}
\newcommand{\autoresponderu}{AUTORESPONDER}
\newcommand{\forwardmsgu}{FORWARDMESSAGES}
\newcommand{\remailmsgu}{REMAILMESSAGE}
\newcommand{\filtermsgu}{FILTERMESSAGES}
\newcommand{\mailhostu}{MAILHOST}

\newcommand{\addressbookl}{{ADDRESSBOOK}}
\newcommand{\signaturel}{SIGNATURE}
\newcommand{\encryptionl}{ENCRYPTION}
\newcommand{\autoresponderl}{AUTORESPONDER}
\newcommand{\forwardmsgl}{FORWARDMESSAGES}
\newcommand{\remailmsgl}{REMAILMESSAGES}
\newcommand{\filtermsgl}{FILTERMESSAGES}
\newcommand{\mailhostl}{MAILHOST}

\newcommand{\faddressbook}{\textcolor{blue} {\OB{\mathit{addressbook}}}}
\newcommand{\fsignature}{\textcolor{blue} {\OB{\mathit{signature}}}}
\newcommand{\fencryption}{\textcolor{blue} {\OB{\mathit{encryption}}}}
\newcommand{\fautoresponder}{\textcolor{blue} {\OB{\mathit{autoresponder}}}}
\newcommand{\fforwardmsg}{\textcolor{blue} {\OB{\mathit{forwardmessages}}}}
\newcommand{\fremailmsg}{\textcolor{blue} {\OB{\mathit{remailmessage}}}}
\newcommand{\ffiltermsg}{\textcolor{blue} {\OB{\mathit{filtermessages}}}}
\newcommand{\fmailhost}{\textcolor{blue} {\OB{\mathit{mailhost}}}}

%\newcommand{\addressbookf}{\OB{\mathit{addressbook}}}
%\newcommand{\signaturef}{\OB{\mathit{signature}}}
%\newcommand{\encryptionf}{\OB{\mathit{encryption}}}
%\newcommand{\autoresponderf}{\OB{\mathit{autoresponder}}}
%\newcommand{\forwardmsgf}{\OB{\mathit{forwardmessages}}}
%\newcommand{\remailmsgf}{\OB{\mathit{remailmessage}}}
%\newcommand{\filtermsgf}{\OB{\mathit{filtermessages}}}
%\newcommand{\mailhostf}{\OB{\mathit{mailhost}}}

\newcommand{\vemployees}{\OB{\mathit{v\_employeelist}}}
\newcommand{\vmessages}{\OB{\mathit{v\_messages}}}
\newcommand{\vrecipientinfo}{\OB{\mathit{v\_recipientinfo}}}
\newcommand{\vreferenceinfo}{\OB{\mathit{v\_referenceinfo}}}
\newcommand{\vautomsg}{\OB{\mathit{v\_auto\_msg}}}
\newcommand{\vforwardmsg}{\OB{\mathit{v\_forward\_msg}}}
\newcommand{\vremailmsg}{\OB{\mathit{v\_remail\_msg}}}
\newcommand{\vfiltermsg}{\OB{\mathit{v\_filtermsg}}}
\newcommand{\vmailhost}{\OB{\mathit{v\_mailhost}}}
\newcommand{\valias}{\OB{\mathit{v\_alias}}}
\newcommand{\midvalue}{\OB{\mathit{X}}}

%query names
\newcommand{\xvalue}{\OB{\mathit{X}}}
\newcommand{\temp}{\OB{\mathit{temp}}}
\newcommand{\tempOne}{\OB{\mathit{temp1}}}
\newcommand{\tempTwo}{\OB{\mathit{temp2}}}
\newcommand{\eq}{\OB{\mathit{empQ}}}
\newcommand{\nfq}{\OB{\mathit{Q}}}
\newcommand{\ntemp}{\OB{\mathit{temp_\mathit{enron}}}}
\newcommand{\deptOne}{\OB{\mathit{``d001"}}}
\newcommand{\mrtable}{\OB{\mathit{mes\_rec}}}
\newcommand{\mretable}{\OB{\mathit{mes\_rec\_emp}}}


% separator
\newcommand{\rSep}{\rn{-}}

%encoding of variational schema in the database
\newcommand{\vdbpc}{\OB{\mathit{vdb\_pcs}\left(\mathit{element\_id}, \mathit{pres\_cond} \right)}}
\newcommand{\tablespc}{\OB{\mathit{tables\_pc}\left(\mathit{table\_name},\mathit{pres\_cond}\right)}}

% table values
\newcommand{\tru}{\OB{\mathit{true}}}
\newcommand{\fls}{\OB{\mathit{false}}}
\newcommand{\NULL}{\OB{\prog{NULL}}}


\newcommand*{\rE}{\rSep\rn{E}}
\newcommand*{\rRefl}{\rn{Refl}}
\newcommand*{\rReflE}{\rRefl\rE}

\newcommand*{\judge}{\rn{Judgement}}
\newcommand*{\empRelRule}{\rn{EmptyRelation}}
\newcommand*{\empRelE}{\empRelRule\rE}
\newcommand*{\relation}{\rn{Relation}}
\newcommand*{\relationE}{\relation\rE}
\newcommand*{\choice}{\rn{Choice}}
\newcommand*{\choiceE}{\choice\rE}
\newcommand*{\product}{\rn{Product}}
\newcommand*{\productE}{\product\rE}
\newcommand*{\setop}{\rn{SetOp}}
\newcommand*{\setopE}{\setop\rE}
%\newcommand*{\diff}{\rn{SetDifference}}
%\newcommand*{\diffE}{\diff\rE}
\newcommand*{\prj}{\rn{Project}}
\newcommand*{\prjE}{\prj\rE}
\newcommand*{\sel}{\rn{Select}}
\newcommand*{\selE}{\sel\rE}
%\newcommand*{\rel}{\rn{RelationRef}}
%\newcommand*{\relE}{\rel\rE}

\newcommand*{\rC}{\rSep\rn{C}}
\newcommand*{\bool}{\rn{Boolean}}
\newcommand*{\boolC}{\bool\rC}
\newcommand*{\nott}{\rn{Neg}}
\newcommand*{\notC}{\nott\rC}
\newcommand*{\conj}{\rn{Conjunction}}
\newcommand*{\conjC}{\conj\rC}
\newcommand*{\disj}{\rn{Disjunction}}
\newcommand*{\disjC}{\disj\rC}
\newcommand*{\attVal}{\rn{AttOptVal}}
\newcommand*{\attValC}{\attVal\rC}
\newcommand*{\attAtt}{\rn{AttOptAtt}}
\newcommand*{\attAttC}{\attAtt\rC}
\newcommand*{\choiceC}{\choice\rC}


%\newcommand*{\rC}{\rSep\rn{C}}
\newcommand*{\rChoice}{\rn{Choice}\rSep}
\newcommand*{\rDist}{\rSep\rn{Dist}}

\newcommand*{\rOpDistC}[1]{\rChoice\rn{#1}\rDist\rC}
\newcommand*{\rSelDistC}{\rOpDistC {Sel}}
\newcommand*{\rPrjDistC}{\rOpDistC {Prj}}
\newcommand*{\rProdDistC}{\rOpDistC {Prod}}

\newcommand*{\rOpDist}[1]{\rChoice\rn{#1}\rDist}
\newcommand*{\rSelDist}{\rOpDist {Sel}}
\newcommand*{\rPrjDist}{\rOpDist {Prj}}
\newcommand*{\rProdDist}{\rOpDist {Prod}}


%intro spl examples
%\newcommand*{\ve}{\OB{\mathit{V}}}
%\newcommand*{\x}{\OB{\mathit{X}}}
%\newcommand*{\y}{\OB{\mathit{Y}}}
%\newcommand*{\msg}{\OB{\mathit{message}}}
%\newcommand*{\sender}{\OB{\mathit{sender}}}
%\newcommand*{\mdate}{\OB{\mathit{date}}}
%\newcommand*{\sbj}{\OB{\mathit{subject}}}
%\newcommand*{\rec}{\OB{\mathit{recepientInfo}}}
%\newcommand*{\rid}{\OB{\mathit{rid}}}
%\newcommand*{\rtype}{\OB{\mathit{rtype}}}
%\newcommand*{\mbody}{\OB{\mathit{body}}}
%\newcommand*{\encrypt}{\OB{\mathit{encrypt}}}
%\newcommand*{\decrypt}{\OB{\mathit{decrypt}}}
%\newcommand*{\encryption}{\OB{\mathit{encryption}}}
%\newcommand*{\decryption}{\OB{\mathit{decryption}}}
%\newcommand*{\employee}{\OB{\mathit{employee}}}
%\newcommand*{\signaturetab}{\OB{\mathit{signature}}}
%\newcommand*{\verification}{\OB{\mathit{verification}}}
%\newcommand*{\sign}{\OB{\mathit{sign}}}
%\newcommand*{\signed}{\OB{\mathit{signed}}}
%\newcommand*{\signKey}{\OB{\mathit{signKey}}}
%\newcommand*{\verify}{\OB{\mathit{verify}}}
%\newcommand*{\isVerified}{\OB{\mathit{isVerified}}}
%\renewcommand*{\mid}{\OB{\mathit{mid}}} %%FYI: you're defining \mid somewhere else, find where!!!!!!!!!!!!!!
%\newcommand*{\isEncrypted}{\OB{\mathit{isEncrypted}}}
%\newcommand*{\encryptionKey}{\OB{\mathit{encryptionKey}}}
%\newcommand*{\ID}{\OB{\mathit{ID}}}
%\newcommand{\faN}{\OB{\mathit{fatherName}}}
%\newcommand{\gender}{\OB{\mathit{gender}}}

%features
\newcommand*{\fName}{\OB{f}}
\newcommand*{\A}{\OB{\fName_1}}
\newcommand*{\B}{\OB{\fName_2}}
\newcommand*{\C}{\OB{\fName_3}}
\newcommand*{\fSet}{\OB{\mathbf{F}}}
\renewcommand*{\dimMeta}{\OB{e}}
\newcommand*{\fModel}{\OB{m}}
\newcommand*{\ffSet}{\OB{\mathbf{E}}}
\newcommand*{\bTag}{\OB{b}}
\newcommand*{\bSet}{\mathbf{B}}
\newcommand*{\config}{\OB{c}}
\newcommand*{\confSet}{\OB{\mathbf{C}}}
\newcommand*{\fct}[1][\config]{\OB{\dimMeta_{#1}^t}}
\newcommand*{\vctx}{\OB{\dimMeta}}

% example feature, feature category and relation names
%\newcommand*{\France}{\OB{\mathit{France}}}
%\newcommand*{\Iceland}{\OB{\mathit{Iceland}}}
%\newcommand*{\US}{\OB{\mathit{US}}}
%\newcommand*{\Iran}{\OB{\mathit{Iran}}}
%\newcommand*{\countryF}{\OB{\mathit{country}}}
%
%\newcommand*{\fType}{\OB{\mathit{type}}}
%\newcommand*{\bank}{\OB{\mathit{bank}}}
%\newcommand*{\stock}{\OB{\mathit{stock}}}
%\newcommand*{\invest}{\OB{\mathit{invest}}}
%
%\newcommand*{\deposit}{\OB{\mathit{deposit}}}
%\newcommand*{\account}{\OB{\mathit{account}}}
%\newcommand*{\amount}{\OB{\mathit{amount}}}
%\newcommand*{\depositor}{\OB{\mathit{depositor}}}
%
%\newcommand*{\member}{\OB{\mathit{member}}}
%\newcommand*{\id}{\OB{\mathit{ID}}}
%\newcommand*{\fN}{\OB{\mathit{firstName}}}
%\newcommand*{\mN}{\OB{\mathit{middleName}}}
%\newcommand*{\lN}{\OB{\mathit{lastName}}}

%spj objects
\newcommand*{\pDB}{\OB{\underline{\db}}}
\newcommand*{\pQ}{\OB{\underline{\q}}}
\newcommand*{\pQSet}{\OB{\underline{\qSet}}}
\newcommand*{\pAtt}{\OB{\underline{\att}}}
\newcommand*{\pAttList}{\OB{\underline{\attList}}}
\newcommand*{\pRel}{\OB{\underline{\rel}}}
\newcommand*{\pRelSch}{\OB{\underline{\relSch}}}
\newcommand*{\pRelCont}{\OB{\underline{\relCont}}}
\newcommand*{\pRelContSet}{\OB{\underline{\relContSet}}}
\newcommand*{\pSch}{\OB{\underline{\sch}}}
\newcommand*{\pInst}{\OB{\underline{\dbInst}}}
\newcommand*{\pInstSet}{\OB{\underline{\dbInstSet}}}
\newcommand*{\pAttSet}{\OB{\underline{\attSet}}}
\newcommand*{\pRelSet}{\OB{\underline{\relSet}}}
\newcommand*{\pRelSchSet}{\OB{\underline{\relSchSet}}}
\newcommand*{\pSchSet}{\OB{\underline{\schSet}}}
\newcommand*{\pAttOpCte}{\OB{\op \pAtt \cte}}
\newcommand*{\pAttOpAtt}{\OB{\op {\pAtt_1}  {\pAtt_2}}}
\newcommand*{\pElem}{\OB{\underline \elem}}
\newcommand*{\pTuple}{\OB{\underline \tuple}}
\newcommand*{\pTab}{\OB{\underline \tab}}

%sql objects
\newcommand*{\sqlQ}{\OB{\underline{sql}}}

%symbols
\newcommand*{\db}{\OB{D}}
\newcommand*{\q}{\OB{q}}
\newcommand*{\att}{\OB{a}}
\newcommand*{\rel}{\OB{r}}
\newcommand*{\tab}{\OB{t}}
\newcommand*{\relSch}{\OB{s}}
\newcommand*{\relCont}{\OB{U}}
\newcommand*{\attList}{\OB{A}}
\newcommand*{\sch}{\OB{S}}
\newcommand*{\dbInst}{\OB{\mathcal{I}}}
\newcommand*{\tuple}{\OB{u}}
\newcommand*{\vTuple}{\tuple}
\newcommand*{\annot}[2][\dimMeta]{\OB{{#2}^{#1}}}
\newcommand*{\elem}{\OB{x}}
\newcommand*{\elemSet}{\OB{X}}
\newcommand*{\cond}{\OB{\theta}}
\newcommand*{\pset}{\OB{\underline {X}}}
\newcommand*{\numRels}{\OB{n}}
\newcommand*{\numTuples}{\OB{k}}
\newcommand*{\numAtts}{\OB{l}}
\newcommand*{\nul}{\OB{\prog{NULL}}}


%set symbols
\newcommand*{\qSet}{\OB{\mathbf{Q}}}
\newcommand*{\dbInstSet}{\OB{\mathbf{I}}}
\newcommand*{\attSet}{\OB{\mathbf{A}}}
%\newcommand*{\tabSet}{\OB{\mathbf{T}}}
\newcommand*{\schSet}{\OB{\mathbf{S}}}
\newcommand*{\relSchSet}{\OB{\mathbf{R}}}
\newcommand*{\relContSet}{\OB{\mathbf{T}}}
\newcommand*{\condSet}{\OB{\boldsymbol{\Theta}}}


% vspj objects
\newcommand*{\vDB}{\db}
\newcommand*{\vSchDef}{\OB{\setDef {{\vRelSch_1 , \ldots, \vRelSch_n}}^\fModel}}
%{\{\widetilde{s}_{R_1}^{F_1}, \ldots, \widetilde{s}_{R_n}^{F_n}\}^\dimMeta}}
\newcommand*{\vdbInst}{\OB{\dbInst}}
\newcommand*{\vdbInstSet}{\OB{\dbInstSet}}
\newcommand*{\vObj}{\OB{o}}
\newcommand*{\vObjSet}{\OB{\mathbf{O}}}
\newcommand*{\vQ}{\OB{\q}}
%\newcommand*{\vRelSch}[1][\pRel]{\OB{{\VVList{s}}_{#1}}}
\newcommand*{\vRelSch}{\OB{\relSch}}
\newcommand*{\vRelCont}{\OB{\relCont}}
\newcommand*{\vSch}{\OB{\sch}}
%\newcommand*{\vSch}{\OB{\VVSet {\mathcal{S}}}}
\newcommand*{\vset}{\OB{X}}
\newcommand*{\vTab}{\tab}
\newcommand*{\vRel}{\rel}
\newcommand*{\vAtt}{\att}

%\makeatletter
%\def\vRel{%
%   \@ifnextchar[%
%     {\vRel@i}
%     {\vRel@i[\dimMeta]}%
%}
%\def\vRel@i[#1]{%
%   \@ifnextchar[%
%     {\vRel@ii{#1}}
%     {\vRel@ii{#1}[\pRel]}%
%}
%\def\vRel@ii#1[#2]{%
%\OB{{#2}^{#1}}
%}
%\makeatother

\newcommand*{\relNum}{\OB{n}}

\makeatletter
\def\vRelConfed{%
   \@ifnextchar[%
     {\vRelConfed@i}
     {\vRelConfed@i[\config]}%
}
\def\vRelConfed@i[#1]{%
   \@ifnextchar[%
     {\vRelConfed@ii{#1}}
     {\vRelConfed@ii{#1}[\pRel]}%
}
\def\vRelConfed@ii#1[#2]{%
\OB{{#2}^{#1}}
}
\makeatother

\makeatletter
\def\optAtt{%
   \@ifnextchar[%
     {\optAtt@i}
     {\optAtt@i[\dimMeta]}%
}
\def\optAtt@i[#1]{%
   \@ifnextchar[%
     {\optAtt@ii{#1}}
     {\optAtt@ii{#1}[\vAtt]}%
}
\def\optAtt@ii#1[#2]{%
%\OB{{\widetilde{#2}}^{#1}}
\OB{\annot[#1]{#2}}
}
\makeatother

%\newcommand*{\vAttt}[2]
%\newcommand*{\attPres}{\OB{F}}
\newcommand*{\vAttList}{\OB{\attList}}
%\widetilde{l}}}
%\newcommand*{\vRelDef}{\OB{\left[\pRel^{\dimMeta}:\vAttList\right]}}

\makeatletter
\def\vRelDef{
   \@ifnextchar[
      {\vRelDef@i}
      {\vRelDef@i[\vRel]}
}
\def\vRelDef@i[#1]{
   \@ifnextchar[
      {\vRelDef@ii{#1}}
      {\vRelDef@ii{#1}[\dimMeta]}
}
\def\vRelDef@ii#1[#2]{
   \@ifnextchar[
      {\vRelDef@iii{#1}{#2}}
      {\vRelDef@iii{#1}{#2}[\vAttList]}
}
\def\vRelDef@iii#1#2[#3]{
\OB{{#1}\left({#3}\right)^{#2}}
}

\newcommand*{\vRelDefNum}[1]{\OB{\vRelDef [\vRel_{#1}] [\dimMeta_{#1}] [\vAttList_{#1}] }}
\newcommand*{\vRelDefNumF}[2]{\OB{\vRelDef [\vRel_{#1}][\vAttList_{#1}][\dimMeta \wedge \dimMeta_{#1}]}}

%\newcommand*{\vRelDef}{\OB{\left[\pRel^{\dimMeta}:\vAttList\right]}}
\newcommand*{\vRelConf}[1][\vRel]{\OB{{#1}^\config}}
\newcommand*{\vAttSet}{\OB{\attSet}}
\newcommand*{\vTabSet}{\OB{\tabSet}}
\newcommand*{\vRelSchSet}{\OB{\relSchSet}}
\newcommand*{\vRelContSet}{\OB{\relContSet}}
\newcommand*{\vSchSet}{\OB{\schSet}}
\newcommand*{\vInstSet}{\OB{\dbInstSet}}
\newcommand*{\empSet}{\OB{\varnothing}}
\newcommand*{\emp}{\OB{\varepsilon}}
\newcommand*{\empAtt}{\OB{\varepsilon}}
\newcommand*{\empRel}{\OB{\varepsilon}}
\newcommand*{\vAttOpCte}{\OB{\op \pAtt \cte}}
\newcommand*{\vAttOpAtt}{\OB{\op {\pAtt_1} {\pAtt_2}}}
\renewcommand*{\tag}[2]{\OB{{#1}^{#2}}}
\newcommand*{\pc}{\OB{pc}}
\newcommand*{\getPC}[1]{\OB{\mathit{pc}{\left({#1}\right)}}}
\newcommand*{\getRel}[1]{\OB{\mathit{rel}{\left({#1}\right)}}}
\newcommand*{\getAtt}[1]{\OB{\mathit{att}{\left({#1}\right)}}}

%conditions
\newcommand*{\pCond}{\OB{\underline{\cond}}}
\newcommand*{\cte}{\OB{k}}
\newcommand*{\pCondSet}{\OB{\underline{\condSet}}}

% variational conditions
\newcommand*{\vCond}{\OB{\cond}}
\newcommand*{\vCondSet}{\OB{\condSet}}

% vspj typing
\newcommand*{\vType}{\OB{\vAttList}}
%\newcommand*{\vContext}{\OB{\vctx}} % --> remove ti

% trans part(Qiaoran)
%\newcommand*{\vSql}{\OB{vq}}
%\newcommand*{\vSqlSet}{\OB{\mathbf{VQ}}}
%\newcommand*{\vRelList}{\OB{rl}}
%\newcommand*{\vRelListSet}{\OB{\mathbf{VrSet}}}
%\newcommand*{\oAttList}{\OB{al}}
%\newcommand*{\oAttListSet}{\OB{\mathbf{Val}}}
%\newcommand*{\vSqlselect}{\textbf{Select} (\vAttList, \vRelList, \vCond)}

% Qiaoran add for translation part
% \newcommand*{\tqS}{\OB{T_Q}}
%\newcommand*{\tqTrans}[1]{\OB{\tqS({#1})}}
%\newcommand*{\trS}{\OB{T_R}}
%\newcommand*{\trTrans}[1]{\OB{\trS({#1})}}

%variational list and set
\newcommand*{\listl}{\OB{l}}
\newcommand*{\sets}{\OB{s}}
\newcommand{\VSet}[1]{\smash{\accentset{\rightarrow}{#1}}}
\newcommand{\VVSet}[1]{\smash{\accentset{\hookrightarrow}{#1}}}
\newcommand{\VList}[1]{\smash{\accentset{\rightarrowtriangle}{#1}}}
\newcommand{\VVList}[1]{\smash{\accentset{\lhook\joinrel\rightarrowtriangle}{#1}}}
\newcommand{\VMap}[1]{\smash{\accentset{\mapsto}{#1}}}


%ops
\newcommand{\Cat}[1]{\mathit{#1}}
\newcommand{\myOR}{\hspace{1.5ex}|\hspace{1.5ex}}
\newcommand{\VVal}[1]{#1'}
\newcommand{\VVVal}[1]{#1''}

\newcommand{\vlift}[1]{\lceil #1 \rceil_S^{S'}}


% defining syntax
\newcommand{\synDef}[2]{\OB{#1 \in #2 }}
\newcommand{\eqq}{\OB{\Coloneqq}}
\renewcommand{\t}{\OB{\prog{true}}}
\newcommand{\f}{\OB{\prog{false}}}

%constraining function
\newcommand{\constrain}[2][\vSch]{\OB{\lfloor #2 \rfloor_{#1}}}

% defining semantics
\newcommand*{\fS}{\OB{\mathbb{E}}}
\newcommand*{\fSem}[2][\config]{\OB{\fS\sem[#1]{#2}}}

\newcommand*{\elemS}{\OB{\mathbb{X}}}
\newcommand*{\elemSem}[2][\config]{\OB{\elemS\sem[#1]{#2}}}

\newcommand*{\elemG}{\OB{\mathcal{G}}}
\newcommand*{\elemGroup}[1][\elem]{\OB{\elemG{\left({#1}\right)}}}

\newcommand*{\qG}{\OB{\mathcal{Q}}}
\newcommand*{\qGroup}[1]{\OB{\qG{\left({#1}\right)}}}

\newcommand*{\cG}{\OB{\mathcal{C}}}
\newcommand*{\cGroup}[1][\vCond]{\OB{\cG{\left({#1}\right)}}}

\newcommand*{\aG}{\OB{\mathcal{A}}}
\newcommand*{\aGroup}[1][\vAttList]{\OB{\aG{\left({#1}\right)}}}

\newcommand*{\oS}{\OB{\mathbb{O}}}
\newcommand*{\oSem}[2][\config]{\OB{\oS\sem[#1]{#2}}}

\newcommand*{\olS}{\OB{\mathbb{A}}}
\newcommand*{\olSem}[2][\config]{\OB{\olS\sem[#1]{#2}}}

\newcommand*{\orS}{\OB{\mathbb{R}}}
\newcommand*{\orSem}[2][\config]{\OB{\orS\sem[#1]{#2}}}

\newcommand*{\otS}{\OB{\mathbb{T}}}
\newcommand*{\otSem}[2][\config]{\OB{\otS\sem[#1]{#2}}}

\newcommand*{\ouS}{\OB{\mathbb{U}}}
\newcommand*{\ouSem}[2][\config]{\OB{\ouS\sem[#1]{#2}}}

\newcommand*{\ovS}{\OB{\mathbb{V}}}
\newcommand*{\ovSem}[2][\config]{\OB{\ovS\sem[#1]{#2}}}

\newcommand*{\osS}{\OB{\mathbb{S}}}
\newcommand*{\osSem}[2][\config]{\OB{\osS\sem[#1]{#2}}}

\newcommand*{\odbS}{\OB{\mathbb{I}}}
\newcommand*{\odbSem}[2][\config]{\OB{\odbS\sem[#1]{#2}}}

%\newcommand*{\eS}{\OB{Q}}
%\newcommand*{\eSem}[2][\config]{\OB{\eS\sem[#1]{#2}}}

\newcommand*{\ecS}{\OB{\mathbb{C}}}
\newcommand*{\ecSem}[2][\config]{\OB{\ecS\sem[#1]{#2}}}

\newcommand*{\eeS}{\OB{\mathbb{Q}}}
\newcommand*{\eeSem}[2][\config]{\OB{\eeS\sem[#1]{#2}}}

%\newcommand*{\tqS}{\OB{T_Q}}
%\newcommand*{\tqSem}[1]{\OB{\tqS\sem[]{#1}}}

% vset operations
\newcommand{\pushIn}[1]{\OB{\downarrow\!\left(#1\right)}}

% spj operations
\newcommand{\pPrj}[2][\pAtt]{\OB{\pi_{#1} #2}}
\newcommand{\pSel}[2][\pCond]{\OB{\sigma_{#1} #2}}
\newcommand{\setDef}[1]{\OB{\{#1\}}}
\newcommand{\pRen}[2][\nameVar]{\OB{\rho_{#1} #2}}

% conditions
\newcommand{\op}[2]{\OB{#1\bullet#2}}
\newcommand{\annd}[1]{\OB{{#1}_1 \wedge {#1}_2}}
\newcommand{\orr}[1]{\OB{{#1}_1 \vee {#1}_2}} 

% vspj operations
\newcommand{\vPrj}[2][\vAttList]{\OB{\pi_{#1} #2}}
\newcommand{\vSel}[2][\vCond]{\OB{\sigma_{#1} #2}}
\newcommand{\vRen}[2][\nameVar]{\OB{\rho_{#1} #2}}
\newcommand{\project}[2]{\OB{\pi_{#1} \left({#2}\right)}}
\newcommand{\projectRel}[2]{\OB{\pi_{#1} {#2}}}
\newcommand{\select}[2]{\OB{\sigma_{#1} \left({#2}\right)}}
\newcommand{\selectRel}[2]{\OB{\sigma_{#1} {#2}}}
\newcommand{\oatt}[2]{\OB{{#1}^{#2}}}
\newcommand{\join}[3]{\OB{\left({#1}\right) \bowtie_{#3} \left({#2}\right)}}
\newcommand{\joinRelL}[3]{\OB{{#1} \bowtie_{#3} \left({#2}\right)}}
\newcommand{\joinRelR}[3]{\OB{\left({#1}\right) \bowtie_{#3} {#2}}}
\newcommand{\joinRel}[3]{\OB{{#1} \bowtie_{#3} {#2}}}
\newcommand{\rename}[2]{\OB{\rho_{#1}  \left({#2}\right)}}
%\newcommand{\choiice}[3]{\OB{\chc [\left({#1}\right)] {\left({#2}\right), \left({#3}\right)}}}
\newcommand{\choiice}[3]{\OB{\chc [\left({#1}\right)] {{#2}, {#3}}}}
%\newcommand{\choiceL}[3]{\OB{\chc [\left({#1}\right)] {{#2}, \left({#3}\right)}}}
%\newcommand{\choiceR}[3]{\OB{\chc [\left({#1}\right)] {\left({#2}\right), {#3}}}}
%\newcommand{\choiceS}[3]{\OB{\chc [{#1}] {\left({#2}\right), \left({#3}\right)}}}
\newcommand{\choiceS}[3]{\OB{\chc [{#1}] {{#2}, {#3}}}}
%\newcommand{\choiceSL}[3]{\OB{\chc [{#1}] {{#2}, \left({#3}\right)}}}
%\newcommand{\choiceSR}[3]{\OB{\chc [{#1}] {\left({#2}\right), {#3}}}}
%\newcommand{\chc}[2][\dimMeta]{\OB{#1\chcL#2\chcR}}

% query type environment
%\newcommand{\env}[4][\vctx][\vSch]{\OB{#1,#2 \vdash#3:#4}}
\makeatletter
\def\env{
  \@ifnextchar[
    {\env@i}
    {\env@i[\vctx]}
}
\def\env@i[#1]{
  \@ifnextchar[
    {\env@ii{#1}}
    {\env@ii{#1}[\vSch]}
}
\def\env@ii#1[#2]#3#4{
{\OB{#1,#2 \vdash#3:#4}}
}
\makeatother

\newcommand{\envWithoutVctx}[3][\vSch]{\OB{#1 \vdash #2 : #3}}
\newcommand{\explEnv}[3][\vSch]{\OB{#1 \vDash #2 : #3}}
\newcommand{\pEnv}[3][\pSch]{\OB{#1 \vdash #2 : #3}}
\newcommand{\envWithSchema}[2][\OB{\vctx}]{\env[#1]{\vRel}{#2}}
\newcommand{\envRelInSch}[2][\vSch]{\env[#1]{#2}{\vType}}
\newcommand{\envOne}[1][\vctx]{\env[#1]{\vQ_1}{\envInContext [\vctx_1] \vType_1}}
\newcommand{\envTwo}[1][\vctx]{\env[#1]{\vQ_2}{\envInContext [\vctx_2] \vType_2}}
\newcommand{\envPrime}{\env{\vQ}{\envInContext[\VVal \vctx]{\VVal \vType}}}

% condition type environment
\newcommand{\envCond}[2][\vctx, \vType]{\OB{#1\vdash #2}}
\newcommand{\envCondAnnot}[2][\vctx, {\pushIn{\annot [\VVal \vctx] \vType}}]{\OB{#1\vdash #2}}

% type and variation context
\newcommand{\envL}{\OB{\lfloor}}
\newcommand{\envR}{\OB{\rfloor}}
\newcommand{\envInContext}[2][\vctx]{\OB{{#2}^{#1}}}
\newcommand{\envEval}[2]{#1 \equiv #2}
\newcommand{\imply}{\OB{\rightarrow}}
\newcommand{\subsume}[2]{\OB{#1 \prec #2}}
\newcommand{\subsumeExpl}[2]{\OB{#1 \preccurlyeq #2}} 
\newcommand{\nsubsume}[2]{\OB{#1 \not \prec #2}}

% object in a variational context
\newcommand{\relInContext}[2][\vRel]{\OB{#1^{#2}}}
\newcommand{\attInContext}[2][\vAtt]{\OB{#1^{#2}}}
\newcommand{\defType}[2][\vType]{\OB{#2: #1}}


% satisfiability function
\newcommand{\sat}[1]{\OB{\mathit{sat}{\left(#1\right)}}}
\newcommand{\unsat}[1]{\OB{\mathit{unsat}{\left(#1\right)}}}
\newcommand{\taut}[1]{\OB{\mathit{taut}{\left(#1\right)}}}

% dbms
\renewcommand{\dom}[2][\vdbInst]{\OB{\mathit{dom}_{#1}{\left(#2\right)}}}
\newcommand{\type}[1][\pAtt]{\OB{\mathit{type}{\left({#1}\right)}}}
\newcommand{\attr}[1][\pRel]{\OB{\mathit{attr}{\left(#1\right)}}}
\newcommand{\arity}[1][\pRel]{\OB{\mathit{arity}{\left(#1\right)}}}
\newcommand{\feat}[2][\vObj]{\OB{\mathit{presCond_{#1}}\left(#2\right)}}
\newcommand{\obj}[1]{\OB{\mathit{obj}\left(#1\right)}}

%\newcommand{\chc}[2][\dimMeta]{\OB{#1\chcL#2\chcR}}
%\newcommand{\chcPP}[3][\dimMeta]{\chc[#1]{\prog{#2},\prog{#3}}}
%\newcommand{\chcPPP}[4][\dimMeta]{\chc[#1]{\prog{#2},\prog{#3},\prog{#4}}}

% special equalities
\newcommand{\spcEq}[1][0cm]{\OB{\hspace{#1}=}}
\newcommand{\spcEquiv}[1][0cm]{\OB{\hspace{#1}\equiv}}


% operations w.r.t. c
\newcommand{\equivc}{\OB{\equiv_{\config}}}


% enumeration and indexing 
\newcommand{\vvn}[2]{\OB{{#1_1}^{#2_1}, \ldots, {#1_n}^{#2_n}}}

% formulation in general
\newcommand{\paran}[1]{\OB{\left(#1\right)}}


\begin{document}
\maketitle

\mainmatter

\section{Introduction}
\label{sec:intro}


\begin{comment}
simple streamline direction:
- variation in db is everywhere
- active research for some instances of it
- other instances aren't well-covered and well-supported
- the combination of instances

a general solution that 
- can model variation for well-covered variation
- others
\end{comment}

%\structure{openning}
%Variation in databases appears abundantly in different forms and contexts.

Variation in databases arises when multiple database instances 
conceptually represent the same database, but, differ
slightly either in their schema and/or content. 
%
Managing variation in databases is a perennial problem in database literature
and appears in different forms and contexts.
Specific cases of this problem has been extensively studied including 
schema evolution~\cite{}, 
data integration~\cite{}, 
and database versioning~\cite{},
where each instance has a context-specific solution that is hard-wired
to the constrained problem definition. 
%
While schema
evolution approaches deal with variation in the schema (i.e., the structure
of the data) database versioning and data integration systems 
manage variation in the content of the database. Yet, the database
community does not consider them as instances of the same problem.
%For example, 
%approaches designed to manage schema evolution over time force
%database administrators to add timestamps to each version of the schema
%in the format that is used by the approach~\cite{} while systems designed
%to manage database versioning force database administrators to use 
%version numbers .

Consider schema evolution which is an instance of schematic variation in databases
that is well-supported~\cite{SchEvolRA90McKenzie, 
schVersioning97Castro, tempSchEvol91Ariav, tsql95Snodgrass, 
prima08Moon}.
Changes applied to the schema over time are \emph{variation} 
in the database and every time the database evolves, a new
\emph{variant} is generated.
%which needs to coexists in parallel
%with other variants.
Current solutions addressing schema
evolution rely on temporal nature of schema evolution by using
timestamps~\cite{SchEvolRA90McKenzie, schVersioning97Castro, 
tempSchEvol91Ariav, tsql95Snodgrass} 
or keeping an external file of time-line history of 
changes applied to the database~\cite{prima08Moon}. 
These approaches only consider variation in time and do not
%However, none of them 
incorporate the time-based changes into
the database directly, rather they \emph{simulate} the effect of these changes,
resulting in brittle systems.

Database-backed software produced by software product line (SPL) 
is an example where variation arises in databases and is poorly 
supported. SPL is an
approach to developing and maintaining software-intensive systems 
in a cost-effective, easy to maintain manner by accommodating variation
in the software that is being reused. 
%The products of a SPL pertain to a
%common application domain or business goal. 
%They also have a common
%managed set of features that describe the specific need for a product. 
%They share 
%a common codebase which is used to produce a product with respect to its set of 
%selected (enabled) features
In SPL, a common codebase is shared and used to produce products w.r.t.
a set of selected (enabled) features~\cite{splBook}. 
% from the SPL feature set designed to describe
%specific needs for products~\cite{splBook}. 
Different products of a SPL typically have different
sets of enabled features or are tailored to run in different environments. These
differences impose different data requirements which creates variation in space 
in the shared database used in the common codebase. 
%Hence, 
%each product has its own database variant.
%For example, different legal
%requirements often require tracking different data in products tailored for use
%in different countries or regions~\cite{splBook}.
%Different data requirements results in different desired data which 
%creates database variants. 
%
%
%The database variants corresponding to software products differ mainly 
%in their schema, a relation/attribute can either be included or excluded for a 
%specific software product~\cite{vdbSpl18ATW}.
The variation is in the form of exclusion/inclusion of tables/attributes based on
selected features for a product~\cite{vdbSpl18ATW}.
%
In practice, software systems produced by a SPL are accommodated with a database that
has all attributes and tables available for all variants-- a database with universal schema~\cite{vdbSpl18ATW}. 
Unfortunately, this approach is
inefficient, error-prone, and filled with lots of null values since not all attributes and tables
are valid for all variant products. A possible solution to this could be defining views on 
the universal database per software variant and write queries for each variant against its 
view~\cite{vdbSpl18ATW}.
However, this is burdensome, expensive, and costly to maintain since it 
requires developers to generate and maintain numerous view definitions
in addition to manually generating
and managing the mappings between views and the universal schema for each product.


Additionally, the software product line (SPL) community has realized that 
variation in software development travels to its artifact including the database.
%
SPL is an
approach to developing and maintaining software-intensive systems 
in a cost-effective, easy to maintain manner by accommodating variation
in the software that is being reused. 
%
In SPL, a common codebase is shared and used to produce products w.r.t.
a set of selected (enabled) features~\cite{splBook}. 
%
Different client's products of a SPL typically have different
sets of enabled features or are tailored to run in different environments. These
differences impose different data requirements which creates variation in space 
in the shared database used in the common codebase. 
%
The SPL community has closed
the gap of variation appearing in database schema while developing software
and the database schema used in client's application by
encoding variation explicitly in data model, called \emph{variable data model}~\cite{}. 
However, the direct encoding of variation has not been extended to the
database itself by these approaches. 

To the best of our knowledge, 
there is no generic solution that manages all possible kinds of
variation in databases. Thus, we explore the idea of considering
variation as a \emph{first-class citizen} in databases. Such exploration
poses questions such as: How can variation be represented in a generic
expressive manner? How can variation be encoded explicitly in databases?
What are the benefits and drawbacks of explicitly encoding variation in 
databases? How applicable and feasible it is to encode variation 
directly in databases?

To answer these research questions we propose the following research goals:

\begin{itemize}
\item Objective 1: Identify the kinds of variation existing in relational databases in 
different application domains.
\item Objective 2: Design a query language and implement a database management 
system that accommodate  variations identified in objective 1.
\item Objective 3: Demonstrate how the proposed system can be used to manage
variation in databases in different application domains.
\item Objective 4: Mechanize proofs of properties of the language and the system.
% maybe as a stretch goal
%\item Objective 5: Generalize the encoding of variation and the design of the query 
%language to cover more kinds of variation.
\end{itemize}


%----------------------------------------------------------------------------

%are instances of the general problem that are
%not well-studied, resulting in using manual approaches that burden experts.
%Moreover, different kinds of variation can interact which cannot be addressed
%by current approaches due to lack of a general solution to managing 
%different kinds of variation in databases.
%%In this paper, 
%To understand 
%we provide a fundamental solution to managing
%variation in databases by considering variation explicitly as 
%a \emph{first-class citizen}
%in the system, allowing for encoding different kinds of variation.
%
%
%%\structure{the following paragraphs are funnel}
%%\point{Schematic and content-level variation in DBs that conceptually 
%%represent the same data.}
%Variation in databases arises when multiple database instances 
%conceptually represent the same database, but, differ
%slightly either in their schema and/or content. 
%%These database
%%instances coexist in parallel.
%The variation in schema and/or content occurs in two dimensions:
%time and space. Variation in space refers to different variants of database that
%coexist in parallel while variation in time refers to the evolution of 
%database, similar to variation observed in software~\cite{EricSPLEvolve19}. 
%Note that variation in a database can occur due to both dimensions
%at the same time.
%
%
%%\point{Schema evolution is an instance of variation in databases
%%that is well-supported.}
%%Explains how schema evolution (which is unavoidable) 
%%is an instance of variation in databases.
%%And mentions some of the current solutions, emphasizing that
%%they cannot address other instances of the problem.
%Schema evolution is an instance of schematic variation in databases
%that is well-supported~\cite{SchEvolRA90McKenzie, 
%schVersioning97Castro, tempSchEvol91Ariav, tsql95Snodgrass, 
%prima08Moon}.
%Changes applied to the schema over time are \emph{variation} 
%in the database and every time the database evolves, a new
%\emph{variant} is generated.
%%which needs to coexists in parallel
%%with other variants.
%Current solutions addressing schema
%evolution rely on temporal nature of schema evolution by using
%timestamps~\cite{SchEvolRA90McKenzie, schVersioning97Castro, 
%tempSchEvol91Ariav, tsql95Snodgrass} 
%or keeping an external file of time-line history of 
%changes applied to the database~\cite{prima08Moon}. 
%These approaches only consider variation in time and do not
%%However, none of them 
%incorporate the time-based changes into
%the database directly, rather they \emph{simulate} the effect of these changes,
%resulting in brittle systems.
%
%
%%\point{Database-backed software produced by SPL is an instance
%%of variation in databases that is poorly managed in practice.}
%%Explains SPL briefly and how variation appears in databases used to
%%store data for software produced by SPL. As well as how poorly it is 
%%managed in practice.
%Database-backed software produced by software product line (SPL) 
%is an example where variation arises in databases and is poorly 
%supported. SPL is an
%approach to developing and maintaining software-intensive systems 
%in a cost-effective, easy to maintain manner by accommodating variation
%in the software that is being reused. 
%%The products of a SPL pertain to a
%%common application domain or business goal. 
%%They also have a common
%%managed set of features that describe the specific need for a product. 
%%They share 
%%a common codebase which is used to produce a product with respect to its set of 
%%selected (enabled) features
%In SPL, a common codebase is shared and used to produce products w.r.t.
%a set of selected (enabled) features~\cite{splBook}. 
%% from the SPL feature set designed to describe
%%specific needs for products~\cite{splBook}. 
%Different products of a SPL typically have different
%sets of enabled features or are tailored to run in different environments. These
%differences impose different data requirements which creates variation in space 
%in the shared database used in the common codebase. 
%%Hence, 
%%each product has its own database variant.
%%For example, different legal
%%requirements often require tracking different data in products tailored for use
%%in different countries or regions~\cite{splBook}.
%%Different data requirements results in different desired data which 
%%creates database variants. 
%%
%%
%%The database variants corresponding to software products differ mainly 
%%in their schema, a relation/attribute can either be included or excluded for a 
%%specific software product~\cite{vdbSpl18ATW}.
%The variation is in the form of exclusion/inclusion of tables/attributes based on
%selected features for a product~\cite{vdbSpl18ATW}.
%%
%In practice, software systems produced by a SPL are accommodated with a database that
%has all attributes and tables available for all variants-- a database with universal schema~\cite{vdbSpl18ATW}. 
%Unfortunately, this approach is
%inefficient, error-prone, and filled with lots of null values since not all attributes and tables
%are valid for all variant products. A possible solution to this could be defining views on 
%the universal database per software variant and write queries for each variant against its 
%view~\cite{vdbSpl18ATW}.
%However, this is burdensome, expensive, and costly to maintain since it 
%requires developers to generate and maintain numerous view definitions
%in addition to manually generating
%and managing the mappings between views and the universal schema for each product.
%
%
%%\point{Schema evolution meets SPL evolution and results in databases
%%used in SPL that their schemas evolve over time.}EricSPLEvolve19, splEvolveBP14
%%Mentions that SPL evolution is hot topic. Part of this evolution is database
%%evolution. This is where two instances of managing variation in databases
%%interact and even the well-supported systems for schema evolution cannot
%%address it. 
%Software evolution is unavoidable, so is its artifacts evolution, 
%including databases~\cite{dbSPLevolve}.
%%Variation in software development is unavoidable and 
%This is where two instances of managing 
%variation in databases (schema evolution and database-backed software 
%developed by SPL) interact.
%% and variation occurs both in time and space. 
%While there are solutions to schema evolution
%they cannot adapt to a new situation because they only provide a solution to
%variation of databases in time and cannot encode the interaction of database variation in
%time and space.
%%dismiss variation of databases in space, 
%%i.e., they dismiss that database variants that have been created due to variation
%%in time must coexist in parallel (a property of variation in 
%%space~\cite{EricSPLEvolve19}). 
%We motivate this case through an example in \secref{mot}.
%%We use this case as our motivating example in \secref{mot} and explain it 
%%in more details.
%
%%\structure{knowledge gap + challenge}
%%\point{Reiterate the knowledge gap. Introduce the challenge.} 
%%Reiterates the knowledge gap. Explains the challenge:
%%The challenge then becomes encoding variation in databases
%%that can model different instances and satisfy different specialists' needs 
%%at different stages (like development, deployment, information extraction, etc).
%%%Explains the challenge as incorporating variation in databases s.t. all 
%%%database variants are gathered in one place. 
%%Also, itemizes the users' 
%%needs in such an environment: 
%%1) Query some/all variants simultaneously and selectively.
%%2) Track the original variant of a piece of data and the variation applied
%%to it through a query.
%%3) Deploy the database and its queries to a variant of it.
%As we have shown, 
%variation in databases w.r.t. time and space is abundant and inexorable~\cite{dbDecay16Stonebraker};
%impacts DBAs, data scientists, and developers significantly~\cite{dbSPLevolve}; 
%and appears in different contexts. Various research has studied variation of databases
%in time and space, however, they all consider only one of these dimensions and
%are extremely tailored to a specific context. This becomes a problem when 
%the two dimensions and different contexts interact. Hence, the challenge becomes
%%offering a fundamental generic framework that 
%incorporating variation in databases s.t. it can model different instances of variation
%in different contexts and it satisfies different specialists' needs at different stages,
%e.g., development, information extraction, and deployment. 
%%
%%We categorize these needs and explain them throughout the paper:
%%\begin{enumerate}[leftmargin=*]
%%\item [\textbf{N0}]Have access to all database variants at a given time
%%\item [\textbf{N1}]Query multiple database variants simultaneously and selectively
%%\item [\textbf{N2}]Keep track of which variants a piece of data belongs to and ensuring that 
%%         it is maintained throughout a query
%%\item [\textbf{N3}]Deploy one variant of the database and its associated queries.
%%\end{enumerate}
%%\structure{proposed solution}
%%%\point{challenge: capturing/representing variability in the database + 
%%%(put-all-together origin-tracking framework) + solution (contributions).}
%%%\structure{point-last.}
%%Therefore, 
%%\emph{
%%how can we incorporate variability in databases to allow for
%%expressive queries over multiple
%%database variants selectively and simultaneously while keeping
%%track of the original variation of data and variation applied to it by
%%a query
%%%without violating data provenance
%%%and 
%%s.t. it is applicable in different contexts?}
%Our contributions in this paper address this challenge:
%\begin{itemize}[leftmargin=*]
%\itemsep0em
%%[wide, labelwidth=!, labelindent=0pt, topsep=1pt]
%%[leftmargin=*]
%\item We provide a framework to capture variation within a database using
%propositional formulas over  
%sets of features, called \emph{feature expressions}, following~\cite{vdb17ATW}.
%\item We incorporate feature expressions  into both the structure (schema) and
%content (tuples) of the database, introducing \emph{variational schemas}, \secref{vsch}, 
%and \emph{variational tables}, \secref{vtab}, and together \emph{VDBs},
%\secref{vdb}.
%%, satisfying \textbf{N0} and first part of \textbf{N2}.
%\item To express user information needs we define the 
%\emph{variational relational algebra} query language,
%%which is 
%a combination of relational algebra and 
%choice calculus~\cite{EW11tosem,Walk13thesis}, \secref{vrel-alg}.
%%, 
%%satisfying \textbf{N1}.
%Users query a VDB by a \emph{variational query}, \secref{vq}.
%\item 
%To make variational queries more useable and easier to understand, respectively,
%we define 
%a static type system, \secref{type-sys},
%and \emph{variation-minimiztion} rules, \secref{var-min}.
%%to make it easier to understand and more useable. 
%%rules to minimize 
%%variation in v-query, \secref{var-min}, to provide better
%%efficiency and usability. 
%%This completes satisfiability of \textbf{N2}.
%\item To query a 
%variational database and receive clear results
%we implement \emph{Variational Database Management System (VDBMS)}, \secref{impl}.
%%,
%%satisfying all four needs: \textbf{N0}-\textbf{N3}.
%%\textbf{N1}, \textbf{N2}, and \textbf{N3}.
%\end{itemize}
%
%%\point{evaluate VDBMS + its pay off}
%%To evaluate VDBMS, 
%VDBMS provides new functionality to databases by accounting for 
%schematic and content-level variation. We provide two use cases of how VDBMS can
%model the schema evolution and SPL instances of variation in databases
%for two real-world databases,
%%(employee database\footnote{\url{https://github.com/datacharmer/test_db}} 
%%and Enron email data corpus\footnote{\url{http://www.ahschulz.de/enron-email-data/}}),
%\secref{exp-disc}. 
%%We also adopt practical queries from~\cite{prima08Moon} and~\cite{emailSPL},
%%respectively,
%%for our use cases to assess VDBMS, \secref{exp-qs}.
%We hypothesize that it especially pays
%off when there is lots of variation in the database and the user needs to query
%many of the variants simultaneously, \secref{exp-disc}. 
%%To examine our hypothesis,  
%%we adopt two real-world databases 
%%from two different contexts (schema evolution and SPL) and
%%generate their counterpart VDBs (based on the changes applied to them).
%%, assess
%%the performance of VDBMS using a set of queries within each context, and
%%finally analyze and discuss the results, \secref{exp-disc}.
%%Our experiments show ...
%%, elucidated more in \secref{exp-disc}.
%
%
%
%%%\begin{comment}
%%%vamos intro:
%%\section{Introduction}
%%\label{sec:intro}
%%
%%%\NOTE{High-level comment: avoid abbreviations like ``w.r.t.'' (I usually edit
%%%most of these out when revising your writing), or at least use them very
%%%sparingly. ``i.e.'' and ``e.g.'' are OK in parentheticals (e.g.\ this is fine),
%%%but should be avoided in the flow of text, e.g.\ this is awkward.}
%%
%%%\NOTE{Maybe first paragraph on how variation in software leads to variation in
%%%databases and vice-versa? Then get into how this breaks down into time and
%%%space.}
%%
%%%\TODO{we show feasibility of our approach. feasibility study.}
%%%
%%%\TODO{var in softwar. var in db. var time space. db has looked ...
%%%vdb more general and complicated. feasibility study.}
%%%
%%%\TODO{bulleted contribution list}
%%%
%%%\TODO{less harsh claim instead of miss variablity. say they don't recognize these are instances of the same problem.}
%%%
%%%\TODO{instead of brittle say specific or not general. very specific to a defined problem. cite a ppr that shows schema evolution sol cannot handle all situations.}
%%%
%%%\TODO{remember to go back to unification of time and space.}
%%
%%Variation in software leaks into its development code and data structures~\cite{Walk14onward,MMWWK17vamos,alkubaish20},
%%tools used by the software~\cite{ywt20splc}, and its artifacts. Here we focus on 
%%one kind of external artifact that is ubiquitous in software to store and manipulate
%%data: \emph{relational databases}.
%%%
%%For example, different clients of a SPL require different information need
%%and consequently, different databases~\cite{skrhas09DBIS}.
%%%
%%Furthermore, software evolves over time resulting in requiring 
%%new information needs or revising previous information needs.
%%This makes database evolvement over time unavoidable~\cite{dbDecay16Stonebraker}.
%%
%%
%%
%%Software systems can vary in two dimensions: ``space'' and ``time''.
%%Variation over space refers to the simultaneous development and maintenance of
%%related systems with different feature sets, which is the focus of work on
%%SPLs.
%%%
%%Variation over time refers to the incremental evolution of a system as it is
%%developed and maintained, and is the focus of revision control systems and
%%configuration management (CM)~\cite{Dart91}.
%%%
%%Although these two aspects of variation have traditionally been studied and
%%addressed separately, recent work has sought to unify the treatment of them
%%%both
%%%kinds of variation 
%%to support new kinds
%%of analyses that consider both dimensions of variation at once~\cite{Thu19vv}.
%%%in order to ease maintenance, support the reuse of
%%%techniques developed in their respective communities, and to support new kinds
%%%of analyses that consider both dimensions of variation at once~\cite{Thu19vv}.
%%%
%%%The case studies we present illustrate the impact of both dimensions of
%%%variation on relational databases, and demonstrate how our conception of
%%%variational databases is generic with respect to the underlying dimension(s) of
%%%variation.
%%
%%%\NOTE{The phrase ``dismissed the nature of the problem'' is strong and
%%%vague---a bad combination! There are also lots of citations here. Probably need
%%%to make a handful of more specific claims about the problem and provide a small
%%%number of citations for each one. Also not clear what the italicized
%%%\emph{variation} means here.}
%%
%%
%%Similarly, variation in databases exists in time and space. 
%%Variation in time  appears when a database evolves 
%%over time while variation in space  appears due
%%to different information requirements by software or different
%%sources of information. 
%%%
%%However, these variation dimensions in databases have not
%%been studied. Instead, instances of them have been studied
%%and addressed separately. 
%%
%%%
%%For example, consider the schema evolution problem:
%%the schema of a database changes over time due to new
%%business requirements~\cite{SchEvolRA90McKenzie}. 
%%This requires applying the changes
%%caused by the evolution to the queries and data. This
%%problem
%%%to the queries written on  (i.e., an instance of database
%%%variation in time) 
%%has been studied
%%thoroughly by the database community~\cite{schVersioning97Castro,SchEvolRA90McKenzie,
%%prima08Moon}.
%%Yet, it is not recognized as variation in the database by the community:
%%%
%%changes applied to the schema over time are \emph{variation} 
%%in the database and every time the database evolves, a new
%%\emph{variant} is generated.
%%%
%%Current solutions addressing schema
%%evolution rely on temporal nature of schema evolution by using
%%timestamps~\cite{SchEvolRA90McKenzie, schVersioning97Castro, 
%%tempSchEvol91Ariav} 
%%or keeping an external file of time-line history of 
%%changes applied to the database~\cite{prima08Moon}. 
%%These approaches only consider variation in time and do not
%%%However, none of them 
%%incorporate the time-based changes into
%%the database directly, rather they \emph{simulate} the effect of these changes,
%%resulting in specialized systems that cannot be extended to other instances
%%of database variation in time.
%%%
%%This observation is not limited to schema evolution;
%%some examples of this include 
%%database versioning~\cite{datasetVersioning,dbVersioning},
%%data integration~\cite{dataIntegBook}, 
%%and data provenance~\cite{bt07sigmod}.
%%%\TODO{food of thought: temporal databases~\cite{tsql95Snodgrass}.}
%%
%%%
%%On the other hand, the SPL community recognizes and addresses the need 
%%for variable data models which models a database schema 
%%(usually as an Entity-Relation model) with
%%annotations of features from SPL to indicate their variable 
%%existence~\cite{skrhas09DBIS, slrs12CAiSE, 
%%ad11varDataModel}, but, it does not expand such need to 
%%the data or the query language. 
%%%
%%%it does not address the variation that appears in queries and data. 
%%These approaches only consider variation in space over a data model
%%and do not address the variation that appears in queries and data. 
%%Thus, developers have to write the required information need as a
%%query encoded as a string per variant. Not only this is labor-some but
%%also due to the nature of queries being encoded as strings there is no
%%static check to ensure that queries are type correct. 
%%This impacts DBAs, data scientists, and developers significantly~\cite{dbSPLevolve}.
%%
%%\begin{comment}
%%Effectively managing variation is a fundamental challenge of software
%%engineering. Research on CM and SPLs
%%have developed numerous representations and strategies for effectively managing
%%different kinds of variation in software. However, these solutions typically do
%%not extend to managing variation in the artifacts and systems that software
%%uses.
%%%An especially tricky aspect is managing corresponding variation in
%%%the external artifacts and services that a software system interacts with.
%%%
%%Here, we focus on \emph{relational databases}.
%%%one kind of external artifact that is ubiquitous in
%%%software but not well-supported by current approaches to managing software
%%%variation: \emph{relational databases}.
%%%
%%Different variants of a software system have different information 
%%needs~\cite{skrhas09DBIS}, which
%%implies a corresponding need for variation in the structure and content of the
%%relational databases that these systems use and rely on. This clearly results in 
%%having variation in the database used to develop a SPL.
%%%
%%%While research on software product lines (SPLs) \cite{fospl} 
%%%has led to a variety of representations and techniques for safely working with
%%%many variants of a software system, these solutions don't extend to relational
%%%databases.
%%
%%In a database-supported SPL, 
%%typically a number of strategies are employed to
%%accommodate the different information needs of different variants.
%%%
%%The first is that a different relational database may be \emph{specified and
%%created per-variant}, according to the information needs of each
%%variant~\cite{marco13featureAdaptSch}. 
%%This approach is labor-intensive and difficult to maintain
%%since changes need to be propagated across variants manually.
%%%
%%The second strategy is to define a single \emph{global schema that applies
%%to all variants}~\cite{batini86dbSchIntegAnalysis}. 
%%This strategy is more efficient to maintain compared to the previous approach
%%but is still hard to maintain,
%%especially in face of SPL evolution. Due to lack of separation of concerns
%%and suboptimal traceability of requirements to database elements~\cite{skrhas09DBIS}
%%it is also complex, hard to understand, and unscalable~\cite{slrs12CAiSE}. 
%%Additionally, it suffers from design limitation and 
%%error-proneness since parts of the schema will be irrelevant to each variant,
%%resulting in losing database's integrity constraints~\cite{slrs12CAiSE}.
%%%Irrelevant attributes are typically populated by NULL-values, which may later
%%%be referenced since it is impossible to check or enforce that queries in each
%%%variant use the database in a safe and consistent way.
%%%
%%The third strategy is to define a \emph{variable data model}~\cite{skrhas09DBIS, 
%%slrs12CAiSE, ad11varDataModel} which models a database schema 
%%(usually as an Entity-Relation model) with
%%annotations of features from SPL to indicate their variable existence. 
%%This approach addresses problems of the previous approach, however,
%%it does not address the variation that appears in queries and data. 
%%Thus, developers have to write the required information need as a
%%query encoded as a string per variant. Not only this is labor-some but
%%also due to the nature of queries being encoded as strings there is no
%%static check to ensure that queries are type correct. 
%%\end{comment}
%%
%%Thus, an initial need for considering variation both in time and space
%%as first-class citizen in databases arises.
%%%
%%To this end, we have developed \emph{variational databases (VDBs)}
%%and \emph{variational queries}~\cite{ATW18poly,ATW17dbpl}, which extend ideas
%%developed in the SPL community to the creation, management,
%%and querying of relational databases.
%%%
%%Conceptually, a  \emph{single} variational database represents \emph{many}
%%different plain relational
%%databases, each one corresponding to a different variant of a software system,
%%\emph{at the same time}.
%%Similarly, a variational query represents potentially many different queries,
%%each one corresponding to a variant of a variational database.
%%%
%%Together, they enable safely and efficiently working with many variants of a
%%relational database at once, and reliably integrating the variants of a
%%database with the corresponding variants of a software product line.
%%%
%%We are currently implementing these ideas in \emph{VDBMS}, 
%%a practical implementation of
%%variational databases as a lightweight wrapper on top of a traditional
%%relational database management system.
%%
%%However, the generic and expressive approach of VDB in dealing
%%with database variation creates new complexity and cost of learning
%%which raises the question: is explicitly encoding variation in databases
%%a good idea?
%%To answer this question, in this paper, we
%%\begin{itemize}
%%%
%%\item show the feasibility of VDB by systematically generating
%%two VDBs from real-world instances of variation in time and space
%%from the scratch, \secref{db}
%%%
%%\item illustrate the applicability of variational queries by encoding information
%%needs for the developed VDBs from scenarios described in the literature,
%%\secref{q}
%%%
%%\item  discuss the pros and cons of explicitly
%%encoding variation in databases and future research directions, \secref{dis}
%%\end{itemize}
%%%
%%We distribute the VDBs, SQL scripts for generating them, 
%%and queries of our case studies in a GitHub repository available \href{https://github.com/lambda-land/VDBMS/tree/master/usecases}{here} for public use. 
%%%
%%We distribute the VDBs in either MySQL or Postgres and
%%in two form, one with embedded  
%%schema described in \secref{enron-vsch}, and one without the embedded schema
%%for use with our VDBMS tool in which the schema is provided
%% separately.
%%%
%%We distribute the queries in our encodings and
%%%but also 
%%as simple \cpp{ifdef}-annotated SQL files to promote their broad reuse
%%in the design and evaluation of other systems for managing variational
%%relational data.
%%
%%\begin{comment}
%%\NOTE{I like this paragraph, but the claim at the end is strong. Are database
%%evolution systems really brittle? And if so, how? For a claim like this, we
%%probably need to be a bit more specific and definitely need a citation.}
%%
%%Although we were inspired to design  VDB because of variation 
%%of databases in SPL, we realized variation in databases appear
%%in other contexts and forms. 
%%%
%%For example, schema evolution is an instance of schematic variation in databases
%%that is well-supported~\cite{SchEvolRA90McKenzie, 
%%schVersioning97Castro, tempSchEvol91Ariav, tsql95Snodgrass, 
%%prima08Moon}.
%%Changes applied to the schema over time are \emph{variation} 
%%in the database and every time the database evolves, a new
%%\emph{variant} is generated.
%%%which needs to coexists in parallel
%%%with other variants.
%%Current solutions addressing schema
%%evolution rely on temporal nature of schema evolution by using
%%timestamps~\cite{SchEvolRA90McKenzie, schVersioning97Castro, 
%%tempSchEvol91Ariav, tsql95Snodgrass} 
%%or keeping an external file of time-line history of 
%%changes applied to the database~\cite{prima08Moon}. 
%%These approaches only consider variation in time and do not
%%%However, none of them 
%%incorporate the time-based changes into
%%the database directly, rather they \emph{simulate} the effect of these changes,
%%resulting in brittle systems.
%%
%%
%%
%%%
%%%Although we have so far considered only SPL-like variation, it's important to
%%%note that software systems can vary in both ``space'' and ``time''.
%%%%
%%%Variation over space refers to the simultaneous development and maintenance of
%%%related systems with different feature sets, which is the focus of work on
%%%SPLs.
%%%%
%%%Variation over time refers to the incremental evolution of a system as it is
%%%developed and maintained, and is the focus of revision control systems and
%%%configuration management (CM)~\cite{Dart91}.
%%%%
%%%Although these two kinds of variation have traditionally been studied and
%%%addressed separately, recent work has sought to unify the treatment of both
%%%kinds of variation in order to ease maintenance, support the reuse of
%%%techniques developed in their respective communities, and to support new kinds
%%%of analyses that consider both dimensions of variation at once~\cite{Thu19vv}.
%%%%
%%%The case studies we present illustrate the impact of both dimensions of
%%%variation on relational databases, and demonstrate how our conception of
%%%variational databases is generic with respect to the underlying dimension(s) of
%%%variation.
%%%
%%
%%\NOTE{First part of this paragraph sounds like it's from a conclusion, not an
%%introduction. We haven't shown anything yet!
%%
%%\medskip
%%Is it possible to organize the outline around a bulleted list of
%%contributions?}
%%
%%So far we have shown, 
%%variation in databases w.r.t. time and space is abundant 
%%and inexorable~\cite{dbDecay16Stonebraker};
%%impacts DBAs, data scientists, and developers significantly~\cite{dbSPLevolve}; 
%%and appears in different contexts. 
%%\end{comment}
%%
%%
%%
%%\begin{comment}
%%%
%%We also showed examples
%%of instances of variation in databases, however, the question
%%becomes how would one encode each specific instance of
%%variation in a database in our VDB framework and what 
%%are the pros and cons of doing such. 
%%%
%%We introduce VDB
%%formalism in \secref{background} and use the introduced
%%notation to describe in detail two case studies (one of variation
%%w.r.t. time and the other of variation w.r.t. space) of systematically generating
%%a VDB for an instance of variation in a database from the scratch
%%(we derive the features, schema, adapt and adjust the data,
%%and populate the VDB) in \secref{db}. 
%%%
%%We then outline the steps
%%required to take in order to generate a VDB and \TODO{introduce
%%a semi-automatic approach for generating a VDB from database 
%%variants accompanied with their associated selected features} in
%%\secref{gen-vdb}. 
%%%
%%We also provide a set of variational queries
%%that capture a wide range of information needs for the introduced
%%use cases in \secref{q}. 
%%%Additionally, 
%%%we distribute the queries not only in our encodings,
%%%but also as simple \cpp{ifdef}-annotated SQL files to promote their broad reuse
%%%in the design and evaluation of other systems for managing variational
%%%relational data.
%%%
%%We distribute the VDBs, SQL scripts for generating them, 
%%and queries of our case studies in a GitHub repository available \href{https://github.com/lambda-land/VDBMS/tree/master/usecases}{here} for public use. 
%%%
%%We distribute the VDBs in either MySQL or Postgres and
%%in two form, one with embedded variational 
%%schema described in \secref{enron-vsch}, and one without the embedded schema
%%for use with our VDBMS tool in which the variational schema is provided
%%and distributed separately.
%%%
%%We distribute the queries not only in our encodings,
%%but also as simple \cpp{ifdef}-annotated SQL files to promote their broad reuse
%%in the design and evaluation of other systems for managing variational
%%relational data.
%%%
%%Finally, we thoroughly discuss the pros and cons of explicitly
%%encoding variation in databases and future research directions
%%in \secref{dis}.
%%\end{comment}
%%
%%%Thus, a generic explicit encoding of variation in databases, both
%%%in the schema and content, seems beneficial. We introduce this
%%%encoding in \secref{background}. 
%%%
%%%Our contributions in this paper are:
%%%\begin{itemize} [leftmargin=*]
%%%\itemsep0em
%%%%
%%%\item To illustrate the applicability of VDB
%%%we explain in details how we systematically 
%%%generated two VDBs by combining existing, widely
%%%used datasets with variation scenarios described in the literature 
%%%in \secref{db}. 
%%%To illustrate the generality of our encoding the 
%%%first case study focuses on variation in space
%%%while the second one focuses on variation in time.
%%%%
%%%\item We outline required steps to generate a VDB and 
%%%\TODO{provide a semi-automatic technique for generating a VDB from a 
%%%set of database variant with their corresponding selected features} 
%%%in \secref{gen-vdb}.
%%%%
%%%\item We provide a set of variational queries
%%%that captures the information needs of different database variants
%%%for each case study
%%%to demonstrate the use of our query language in \secref{q}.
%%%%
%%%However, we distribute the queries not only in our encodings,
%%%but also as simple \cpp{ifdef}-annotated SQL files to promote their broad reuse
%%%in the design and evaluation of other systems for managing variational
%%%relational data.
%%%%
%%%\item We thoroughly discuss the pros and cons of explicitly
%%%encoding variation in databases and future research directions
%%%in \secref{dis}.
%%%\end{itemize}
%%
%%%\begin{comment}
%%%In this paper, we present two case studies that demonstrate the need for a more
%%%systematic approach to managing variation in relational databases than the
%%%current status quo. Additionally, these case studies support the evaluation of
%%%systems like VDBMS that are intended to address this need.
%%%%
%%%The case studies were created by systematically combining existing, widely used
%%%datasets with software variation scenarios described in the literature.
%%%%
%%%Each case study consists of (1) a variational database whose structure is
%%%defined by a \emph{variational schema} and whose content is given by a set of
%%%\emph{variational tables}, and (2) a set of variational queries over this
%%%database that captures the information needs of different variants of the
%%%corresponding software system.
%%
%%
%%%We give a brief overview of variational databases and queries in
%%%Section~\ref{sec:background} and use the notations and terminology developed in
%%%that section throughout the paper.
%%%
%%%However, we distribute the case studies themselves not only in our encodings,
%%%but also as simple \cpp{ifdef}-annotated SQL files to promote their broad reuse
%%%in the design and evaluation of other systems for managing variational
%%%relational data.
%%
%%
%%%The variational databases for each case study are presented in
%%%Section~\ref{sec:db}.
%%%%
%%%The first case study focuses on variation over space. It develops a variational
%%%schema that captures the information needs of a SPL based on Hall's
%%%decomposition of an email system into its component features~\cite{Hall05}. The
%%%email SPL has been used in several previous SPL research projects (e.g.\
%%%\cite{Apel13:SSP,AlHaj19}). The variational email database is populated using
%%%the Enron email dataset, adapted to fit our variational schema~\cite{Shetty04}.
%%
%%
%%%The second case study demonstrates the changing information needs of a system
%%%that varies over time by developing a variational schema corresponding to an
%%%employee-system evolution scenario described in \citet{prima08Moon}. The
%%%variational employee database is populated by adapting a large, fabricated
%%%employee dataset\footnote{\url{https://github.com/datacharmer/test_db}} that
%%%has been widely used in the databases community.
%%
%%
%%%After describing the development of the database for each case study, we give a
%%%brief statistical overview of their structure and contents, and also identify
%%%some basic validation properties that all variational databases (including the
%%%two examples we provide) should satisfy.
%%
%%
%%%The variational queries for each case study are presented in
%%%Section~\ref{sec:q}. For each case study, we systematically worked through each
%%%scenario, as described in the corresponding paper, and created a query for each
%%%information need that was described.
%%%%
%%%We describe how the queries were developed and give several examples.
%%%\end{comment}
%
%
%
%
%

%mot
%contribution
\chapter{Background}
\label{ch:bg}


The core of this thesis is injecting a new aspect to relational databases: \emph{variation}.
Thus, the goals of this chapter are twofold: 
%
first, to introduce how variation is encoded and represented in our variational database framework;
%
second, to provide the reader with the concepts and notations
used to build up the main contributions of this thesis---mainly relational databases,
relational algebra,
%
and approaches used to add variation to them.
% of converting non-variational components into
%variational counterparts by using the introduced encoding of variation. 
%third, to familiarize the reader with
%main notations and design decision of the thesis. 

%
Throughout the thesis, we use types when defining concepts. 
A type is a set of possible values. For example, the type $\mathbf{Int}$
denotes all possible integers. In our formalization, we use the notation of $i \in \mathbf{Int}$ to
state that the variable $i$ is of type $\mathbf{Int}$. 
%
Types can be more general. Consider the type \settype \typevar\ that indicates the set of all sets
 of values of type \typevar. Here, \typevar\ is a type variable and stands for any possible type. 
Note that concrete types start with a capital letters but type variables do not.
For example, the type $\settype {\mathbf{Int}}$ is the type of
sets of integers and it has values such as $\setDef {1,3,4}$ and $\setDef {\ }$.
We also use type synonyms to make formalizations easier to understand. 
For example, instead of referring to
$\settype {\mathbf{Int}}$ we can give it a new name ($\mathbf{Ints}= \settype {\mathbf{Int}}$).
Thus,  $\mathbf{Ints}$ also indicates all possible sets of integers.
%However, for brevity, we usually use 
%bold capital letters for types, for example, instead of $\mathit{Int}$ we write $\mathcal{I}$ to denote the type $\mathit{Int}$.
%
Sometimes we must extend a type with an additional ``bottom'' element (i.e., $\bot$). To account for this
extension at the type level we subscript the type with $\bot$. For example, $\maybe {\mathbf{Int}}$
denotes the extension of  type $\mathbf{Int}$ with $\bot$.
%
%\eric{pair and function types}
Furthermore, sometimes we pair two types together. For example, a variable of type $\typepair{\mathbf{Int}} {\mathbf{String}}$ can take the value $(3,``three")$. 
%
Finally, we use function types to provide the type signature of functions. For example,
the type signature $id : \typevar \totype \typevar$ for function $id$ states that the function $id$ takes a value of
type \typevar\ and returns a value of type \typevar. 

%
Throughout the thesis, we discuss relational concepts and their
variational counterparts. 
When it is unclear from context, we use
an $\underline{underline}$ to distinguish a non-variational entity
from its variational counterpart, 
both at the value level and the type level,
%when we need to emphasize 
%an entity is not variational we underline it, 
e.g., \pElem\ is a 
non-variational entity while \elem\ is its variational counterpart,
if it exists.


%\secref{types} describes types. Types provide readers with a strong tool to understand 
%some formalizations easier and more intuitively.
%
\secref{rdb} describes the database model of relational databases and 
the specification of
the structure used to store the data~\cite{AliceBook}. 
\secref{ra} describes the relational algebra, a query language used to query relational databases~\cite{AliceBook}.
%
\secref{encode-var} defines our encoding of the variation space used in the 
variational database framework and how we describe parts of that space using propositional formulas of boolean variables~\cite{ATW18poly,ATW17dbpl}.
%
Finally, we introduce the main techniques used to incorporate variation into our variational 
database framework.
\secref{vset} introduces variation into sets, which forms the basis of the variational database
framework~\cite{EWC13fosd,Walk14onward,ATW17dbpl} 
and \secref{fcc} describes the formula choice calculus used to incorporate 
variation into relational algebra~\cite{HW16fosd}.



%\section{Types}
\label{sec:types}

\TODO{types}


\begin{table}
\caption[shortcaption]{An example of a relational database corresponding to \vTwo\ of our motivating example
given in \tabref{mot-basic}.}
\label{tab:rdb}
\centering
\small
%\footnotesize
%\scriptsize
\begin{subtable}[t]{\textwidth}
\centering
\caption{The schema of a relational database.}
\label{tab:rdb-sch}
\begin{tabular} {| l | }
\hline
\empacct\ (\empno, \name, \hiredate, \titleatt, \deptname)\\
\job\ (\titleatt, \salary)\\
\hline
\end{tabular}
\end{subtable}

\medskip
\medskip
\medskip
\begin{subtable}[t]{\textwidth}
%\begin{center}
\centering
\caption{The \empacct\ table.}
\label{tab:rdb-empacct}
\begin{tabular} {c | l l l l l}
%\hline
%\hhline{-==}
\empacct & \empno & \name & \hiredate & \titleatt & \deptname\\
\cline{2-6}
& 10001 & Georgi Facello & 1986-06-26 & Senior Engineer & Development\\
& 10002 & Bezalel Simmel & 1985-11-21 & Staff & Sales\\
& \ldots & \ldots & \ldots & \ldots & \ldots \\
& 499998 & Patricia Breugel & 1993-10-13 & Senior Staff & Finance\\
& 499999 & Sachin Tsukuda & 1997-11-30 & Engineer & Production
\end{tabular}
%\end{center}
\end{subtable}

\medskip
\medskip
\medskip
\begin{subtable}[t]{\textwidth}
\centering
\caption{The \job\ table.}
\label{tab:rdb-job}
\begin{tabular} {c | l l }
%\hline
%\hhline{-==}
\job & \titleatt & \salary\\
\cline{2-3}
& Assistant Engineer & 96646\\
& Assistant Engineer & 61594\\
& \ldots & \ldots \\
& Technique Leader & 58345\\
& Technique Leader & 86641
\end{tabular}
\end{subtable}

\end{table}


\section{The SPC Relational Algebra}
\label{sec:ra}

%\maybeAdd{add type system}
%\maybeAdd{maybe add semantics later on}
Having introduced relational databases, we now shift gears into querying
these databases, that is, extracting information from tables.
For almost all relational query languages, the result of a query 
is a table called $\mathit{result}$. 
%
We base our variational query language on the SPC relational algebra.
Three primitive operators form the SPC algebra: \emph{selection}, \emph{projection},
and \emph{cross-product} (or Cartesian product)~\cite{AliceBook}.
We introduce these operators through \exref{ra} by stating an intent and then
building up a query to extract the information required by the intent. 

\begin{example}
\label{eg:ra}
Consider the database instance given in \tabref{rdb}. We want to get a list
of employees (by their names) whose salary is more than 65000 dollars. 
As the first step, we use the selection operator to get the tuples for all jobs with salaries that 
are more than 65000 dollars.\\
%
%\begin{equation*}
\centerline{
\ensuremath{
{\pQ_1} = \sigma_{\salary \ge 65000} (\job)
%\end{equation*}
}}
%
\noindent
A sample of the results returned by the query ${\pQ_1}$ is given in \tabref{ra1}.
Next a set of 
 tuples is created by taking the cross-product of ${\pQ_1}$
and \empacct.\\
%
%\begin{equation*}
\centerline{
\ensuremath{
{\pQ_2} = {\pQ_1} \times \empacct
%\end{equation*}
}}
%
\noindent
A sample of the results returned by the query ${\pQ_2}$ is given in \tabref{ra2}.
However, looking closely at these results, there is no connection between an employee
in the \empacct\ relation and their salary in the \job\ relation. Thus, we have to perform 
another selection to connect each employee with their title. \\
%
%\begin{equation*}
\centerline{
\ensuremath{
{\pQ_3} = \sigma_{\empacct.\titleatt=\job.\titleatt} ({\pQ_2})
%\end{equation*}
}}
%
\noindent
A sample of the results returned by the query ${\pQ_3}$ is given in \tabref{ra3}.
At this point, we are only interested in two attributes, that is, \name\ and \salary.
Thus, we use projection to discard the unneeded columns.\\
%
%\begin{equation*}
\centerline{
\ensuremath{
{\pQ_4} = \pi_{\name, \salary} ({\pQ_3} )
%\end{equation*}
}}
%
A sample of the results returned by the query ${\pQ_4}$ is given in \tabref{ra4}.
\end{example}

\begin{table}[!htbp]
\caption[Results of subqueries to build up the query in \exref{ra}]{Results of each step of building the final query in \exref{ra}.}
\label{tab:ra-ex}
\centering
\small
%\footnotesize
%\scriptsize
\begin{subtable}[t]{\textwidth}
\centering
\caption{Result of the query \ensuremath{\pQ_1 = \sigma_{\salary \ge 65000} (\job)}.}
\label{tab:ra1}
\begin{tabular} {c | l l }
\multirow{2}{*}{$\mathit{result}$} & \titleatt & \salary \\
\cline{2-3}
&Senior Staff & 77935 \\
& Senior Engineer & 96646\\
& \ldots & \ldots \\
& Staff & 77935\\
& Engineer & 96646
\end{tabular}
\end{subtable}

\medskip
\medskip
\medskip
\begin{subtable}[t]{\textwidth}
%\begin{center}
\centering
\tiny
\caption{Result of the query \ensuremath{\pQ_2 = (\sigma_{\salary \ge 65000} (\job)) \times \empacct}.}
\label{tab:ra2}
\begin{tabular} {c | l l l l l l l}
%\hline
%\hhline{-==}
\multirow{2}{*}{$\mathit{result}$}  & \titleatt & \salary & \empno & \name & \hiredate & \titleatt & \deptname\\
\cline{2-8}
&Senior Engineer & 96646 & 13094 & Sanjay Servieres & 1986-01-01 & Engineer & Research \\
&Staff & 77935 & 16099 & Mohan Ferretti & 1987-09-20 & Senior Staff & Human Resources\\
&\ldots & \ldots & \ldots & \ldots & \ldots & \ldots & \ldots\\
&Engineer & 80324 & 19162 & Chinho Fadgyas & 1986-05-19 & Technique Leader & Production \\
&Senior Staff & 88070 & 22255 & Kristian Merel & 1986-09-12 & Senior Engineer & Development
\end{tabular}
%\end{center}
\end{subtable}

\medskip
\medskip
\medskip
\begin{subtable}[t]{\textwidth}
%\begin{center}
\centering
%\footnotesize
\tiny
\caption{Result of the query \ensuremath{\pQ_3 = \sigma_{\empacct.\titleatt=\job.\titleatt}\left(\left(\sigma_{\salary \ge 65000} \left(\job\right)\right) \times \empacct\right)}.}
\label{tab:ra3}
\begin{tabular} {c | l l l l l l l}
%\hline
%\hhline{-==}
\multirow{2}{*}{$\mathit{result}$}  & \titleatt & \salary & \empno & \name & \hiredate & \titleatt & \deptname\\
\cline{2-8}
&Engineer & 96646 & 13094 & Sanjay Servieres & 1986-01-01 & Engineer & Research \\
&Senior Staff & 77935 & 16099 & Mohan Ferretti & 1987-09-20 & Senior Staff & Human Resources\\
&\ldots & \ldots & \ldots & \ldots & \ldots & \ldots & \ldots\\
&Senior Engineer & 96646 & 22255 & Kristian Merel & 1986-09-12 & Senior Engineer & Development\\
&Staff & 77935 & 43670 & JoAnna Randi & 1987-10-18 & Staff & Marketing
\end{tabular}
%\end{center}
\end{subtable}

\medskip
\medskip
\medskip
\begin{subtable}[t]{\textwidth}
\centering
\caption{Result of the query \ensuremath{\pQ_4 = \pi_{\name, \salary} \left(\sigma_{\empacct.\titleatt=\job.\titleatt}\left(\left(\sigma_{\salary \ge 65000} \left(\job\right)\right) \times \empacct\right)\right)}.}
\label{tab:ra4}
\begin{tabular} {c | l l }
%\hline
%\hhline{-==}
\multirow{2}{*}{$\mathit{result}$}  &\name & \salary\\
\cline{2-3}
& Sanjay Servieres & 96646\\
&Mohan Ferretti & 77935\\
&  \ldots & \ldots \\
& Kristian Mere & 96646\\
& JoAnna Randi & 77935
\end{tabular}
\end{subtable}

\end{table}



The relational algebra that we use also includes standard set operations, a join 
operation, and an empty relation. The syntax is defined in \figref{rel-alg}.
%
The set operations, union and intersection, require two subqueries to have the same relation schema
and simply applies the corresponding operation, either union or intersection, to the sets of tuples returned by
the subqueries.
%For example, if we have $\pRel_1(\pAtt_1, \pAtt_2)$ with 
%tuples $\{(1,2)(3,4)\}$
%and $\pRel_2(\pAtt_1, \pAtt_2)$ with tuples $\{(1,2),(5,6) \}$
%then $\pRel_1 \cup \pRel_2 $ returns the tuples $\{(1,2), (3,4), (5,6)\}$.
%
The \emph{join} operation is equivalent to selection applied to a cross-product, that is,
$\pQ_1 \bowtie_{\pCond} \pQ_2 = \vSel [\pCond] {(\pQ_1 \times \pQ_2)}$.
For example, ${\pQ_3}$ in \exref{ra} can be rewritten as\\
\centerline{
\ensuremath{
\VVal {{\pQ_3}} = \left(\sigma_{\salary \ge 65000} \left(\job\right)\right) \bowtie_{\empacct.\titleatt=\job.\titleatt} \empacct
}}.
%joins two subqueries based on a condition and
\noindent
Throughout our examples, omitting the condition of join  implies it is a \emph{natural join},
that is, join on the shared attribute of the two subqueries.
For example, $\VVal {{\pQ_3}}$ can be rewritten using the natural join\\
\centerline{
\ensuremath{
\VVVal {{\pQ_3}} =  \left(\sigma_{\salary \ge 65000} \left(\job\right)\right) \bowtie \empacct
}}.

\begin{figure}

\begin{syntax}

% feature expressions
%\synDef{\dimMeta}{\ffSet}
%  &\eqq& \multicolumn{2}{l}{%
%         \t \myOR \f \myOR \fName \myOR \neg\fName
%         \myOR \dimMeta\wedge\dimMeta \myOR \dimMeta\vee\dimMeta}
%\\[1.5ex]

% relation conditions
\synDef{\pCond}{\pCondSet}
  &\eqq& \multicolumn{2}{l}{%
         \t \myOR \f \myOR \att\bullet\cte \myOR \att\bullet\att
         \myOR \neg\pCond \myOR \pCond\vee\pCond} \\
%     &|& \multicolumn{2}{l}{\vCond\wedge\vCond \myOR \chc{\vCond,\vCond}}
\\[1.5ex]

% variational relational algebra
\synDef{\pQ}{\pQSet}
  &\eqq& \pRel                 & \textit{Relation reference} \\
     &|& \pRen[\pRel]{\pQ}     & \textit{Renaming} \\
     &|& \pPrj[\pAttList]{\pQ} & \textit{Projection} \\
     &|& \pSel\pQ              & \textit{Selection} \\
     &|& \pQ \Join_{\pCond} \pQ  & \textit{Join} \\
%     &|& \chc{\vQ,\vQ}         & \textit{Choice} \\
%     &|& \empRel               & \textit{Empty relation} \\
%    &|& \vQ \times \vQ        & \textit{Cartesian Product} \\
%    &|& \vQ \circ \vQ         & \textit{Set operation} \\
\end{syntax}

\caption{Syntax of  relational algebra, where $\bullet$ ranges over
comparison operators ($<, \leq, =, \neq, >, \geq$), \cte\ over constant values,
\att\ over attribute names, and \pAttList\ over lists of attributes.
The syntactic category
% \dimMeta\ represents feature expressions, 
 \pCond\
is relational conditions, and \pQ\ is  relational algebra terms.
}
%\vspace{-20pt}
\label{fig:v-alg-def}
\end{figure}
%\vspace{-20pt}


We also extend relational algebra such that projection of an empty set of
attributes is a valid query that returns an empty set of tuples. We define the
\emph{empty} query \empPRel\ as shorthand for projecting an empty set of
attributes, that is, $\empPRel = \pi_{\{\}} \pQ$.
%
Note that we do not extend the notation of using underline for relational algebra
operators. Instead, relational algebra operators are overloaded and are used
as both plain relational and variational operators. It should be clear from
context when an operation is variational or not. 


Although we do not consider renaming of queries in the formal definition of 
relational algebra, we do support this in our implementation. Furthermore, we use it
to rename subqueries of our examples to make them easier to understand. 
For example, query $\VVVal {{\pQ_3}}$ can be written as:
\begin{align*}
\VVVal {{\pQ_3}} &= \underline{temp} \bowtie \empacct\\
\underline{temp} &\leftarrow  \sigma_{\salary \ge 65000} \left(\job\right)
\end{align*}
\noindent
Making this renaming explicit is necessary to avoid names conflicting in some cases.





\section{Variation Space in a Variational Database Framework}
\label{sec:varspace}

\TODO{have to revise}
\TODO{define oplus in fig as syntactic sugar.}

%\point{using a feature set to represent variability within a context.}
To account for variability in a database we need to 
encode it.
%
%The first challenge of incorporating variability into a database
%is to represent variability. 
%
To encode variability we introduce a \emph{feature space} \fSet\ as 
a closed set of boolean variables. 
A feature \ensuremath{\fName \in \fSet} can be enabled (i.e., \fName = \t) or disabled (\fName = \f).
Features capture the variation in a given variational scenario.
%first organize the configuration space into
%a set of features \fSet.
%, denoted by \fSet.
%
%we require a \emph{set of features}, denoted by \fSet, 
%appropriate for the context that the database is used for.
%
For example, in the context of schema evolution, features can be generated from version 
numbers (e.g. features \vOne\ to \vFive\ and \tOne\ to \tFive\ in the 
motivating example, \tabref{mot}); for SPLs, 
the features can be adopted from the SPL feature set (e.g. the \edu\ feature in
our motivating example, \tabref{mot}); and 
for data integration, the features can be representatives of resources.  
%For simplicity, the set of features is assumed to be closed and features are
%assumed to be boolean variables, however, it is easy to extend them
%to multi-valued variables that have finite set of values.
% and without loss of generality, 
%features are assumed to be boolean variables, although, it is easy to extend them
%to multi-valued variables. 
%A feature \ensuremath{\fName \in \fSet} can be enabled (i.e., \fName = \t) or disabled (\fName = \f).
%\point{configuration.}
%Assuming that all the features by default are set to \prog{false},
%enabling some of them specifies a variant. 

Features describe local points of variation in a database.
%, i.e,
%in \fSet\ 
%are used to 
%they indicate which parts of the database are present conditionally.
%of a variational entity within the database 
%are different 
%among different variants. 
Thus, enabling or disabling all
the features 
%in \fSet\ 
produces a particular database \emph{variant} where
%of the entity in which 
all variation has been removed. 
%Enabling or disabling all features of \fSet\ specifies a non-variational \emph{variant}
%that can potentially be generated by \emph{configur}ing its variational counterpart 
%with the variant's \emph{configuration}.
%Hence, to specify a variant
%we define a function, called 
A \emph{configuration} is a \emph{total} function
that maps every feature in the feature set to a boolean value.
%By definition, a configuration is a \emph{total} function,
%i.e., it is defined for \emph{all} features in the feature set. 
%For brevity, 
We represent a configuration  by the set of enabled features.
%which represents a variant. 
%\ensure{make sure referring to a variant can be done by the set of enabled features. HERE!}
For example, in our motivating scenario, the configuration \ensuremath{
\setDef {\vTwo,\tThree,\edu}
}
represents a database variant where only features \vTwo, \tThree, and \edu\ are enabled.
This database variant contains relation schemas in the yellow cells of \tabref{mot}.
%of the
%employee and education sub-schemas associated with \vTwo\ and
%\tThree\ in \tabref{mot}, respectively.
%For brevity, 
We refer to a variant with configuration \config\ as variant \config, e.g.,
%For example, 
variant \setDef {\vTwo,\tThree,\edu} refers to the variant
with configuration \setDef {\vTwo,\tThree,\edu}.
%and education sub-schema associated with \tThree\ in \tabref{mot}.

%\point{represent variability in db by prop formula of features, called feature expression.}
%Having defined a set of features, we need to incorporate them into the database.
%To encode features in the database, we construct propositional formulas of features
%such that the formula evaluates to \t\ for a set of configurations. 
When describing variation points in the database, we need to 
refer to subsets of the configuration space. We achieve this by 
constructing propositional formulas of features.
Thus, 
such a propositional formula defines a condition that holds for 
a subset of configurations and their corresponding variants. 
%
%describing the condition where one or more variants are present,
%i.e., assigning features to their values defined in variant's configuration and 
%evaluating the propositional formula results in \prog{true}.
%
For example,
the propositional formula $\neg \edu$ represents all variants of our
motivating example that do not 
have the education part of the schema, i.e., variant schemas of the 
left schema column. 

We call a propositional formula of features a \emph{feature expression} and define
it formally in \figref{fexp-def}. 
%\figref{fexp-def} defines the syntax of feature expressions.
The evaluation function of feature expressions 
$\fSem \dimMeta : \ffSet \to \confSet \to \bSet$ simply substitutes each
feature \fName\ in the expression \dimMeta\ with its boolean value 
set in the given configuration \config\ and then
simplifies the propositional formula to a boolean value.
%  evaluates the 
%feature expression \dimMeta\ w.r.t. the configuration \config.
%and defined in
%our technical report~\cite{vldbArXiv}, 
%and defined in \appref{fexp},
%evaluates feature expression \dimMeta\ under configuration \config,
%also called \emph{configuration of feature expression \dimMeta\ under \config}. 
For example, assuming that 
\ensuremath{\fSet = \setDef{\A, \B}}
 $\fSem [\{\A\}] {\A \vee \B} = \t \vee \f = \t$, however,
%which states that the feature expression
%$\A \vee \B$ evaluates to \t\ w.r.t. configuration that only enables $\A$, however,
$\fSem [\{\}] {\A \vee \B} = \f \vee \f = \f$.
%, where the empty set indicates neither \A\ nor \B\
%are enabled.
%which states that the same feature expression evaluates
%to \f\ when neither $\A$ nor $\B$ are enabled.
Additionally, we define the binary \emph{equivalence ($\equiv$)} relation and
the unary \emph{satisfiable (sat)} and \emph{unsatisfiable (unsat)}
relations over feature expressions in \figref{fexp-def}.
%
% some of the functions needed over feature
%expressions:
%1) \emph{equivalence} of two feature expressions, 
%\ensuremath{\dimMeta_1 \equiv \dimMeta_2}
%and
%2) \emph{satisfiability} of a feature expression, 
%\ensuremath{\sat \dimMeta}.
%However, we define the evaluation of feature expressions and functions over them
%in \appref{fexp}.
%We define two functions over feature expressions, as shown in \figref{fexp-def}:
%1) \emph{satisfiability}: feature expression \dimMeta\ is \emph{satisfiable} if there 
%exists configuration \config\ s.t. \fSem \dimMeta = \t\
%and 2) \emph{tautology}: feature expression \dimMeta\ is a \emph{tautology} if  
%for all valid configurations we have: \fSem \dimMeta = \t.

%\point{annotating elements of database with feature expressions.}
To incorporate feature expressions into the database,
we \emph{annotate/tag} database elements (including attributes, relations, and tuples) 
with feature expressions. An \emph{annotated element} \elem\ with feature expression \dimMeta\
is denoted by \annot \elem. 
%The feature expression \dimMeta\ represents
%the set of configurations where their variants contain element \elem\ because
%
The feature expression attached to an element is called a \emph{presence condition}
since it determines the condition (set of configurations) under which the element is present.
\getPC {\annot \elem} returns the presence condition of the variational element
\annot \elem.
For example, the
annotated number $\annot [\A \vee \B] 2$ is present in variants
\setDef \A, \setDef \B, 
\ensuremath{\setDef {\A, \B}}
% with a configuration
%that enables either $\A$ or $\B$ or both
but it does not exist in variant \setDef {}.
%variants that disable both $\A$ and $\B$.
Here, $\getPC {\annot [\A \vee \B] 2} = \A \vee \B$.

%\point{relationship between features is captured by a propositional formula, called feature model.} 
No matter the context, features often have a relationship with each other that
constrains configurations. For example, only one of the temporal features of $\vOne - \vFive$
can be \t\ for a given variant.
This relationship is captured by a feature expression, called a \emph{feature model} and
denoted by \fModel,
which restricts the set of \emph{valid configurations}:
if configuration \config\ violates the relationship then 
%evaluating the feature model \fModel\
%under this configuration evaluates to \f: 
it follows from the \fSem . definition that \fSem \fModel = \f.
For example, the restriction that at a given time only one of temporal features $\vOne - \vFive$
can be enabled is represented by:
%the feature expression:
\ensuremath{
\vOne \oplus \vTwo \oplus \vThree \oplus \vFour \oplus \vFive
},
where $\A \oplus \B \oplus \ldots \oplus \fName_n$ is syntactic sugar for $(\A \wedge \neg \B \wedge \ldots \wedge \neg \fName_n) \vee (\neg \A \wedge \B \wedge \ldots \wedge \neg \fName_n) \vee (\neg \A \wedge \neg \B \wedge \ldots \wedge \fName_n)$
, i.e., features are mutually exclusive.
%multiple features cannot be enabled simultaneously.
%$\left(\vOne \wedge \neg \left(\vTwo \vee \hdots \vee \vFive \right) \right)
%\vee \left(\vTwo \wedge \neg \left(\vOne \vee \vThree \vee \vFour \vee \vFive \right) \right) 
%\vee \hdots 
%\vee \left(\vFive \wedge \neg \left(\vOne \vee \hdots \vee \vFour \right) \right)$.
%Note that this is not the feature model for the entire motivating example.



\begin{figure}
%\textbf{Feature expression generic object:}
%\begin{syntax}
%\synDef \fName \fSet &\textit{Feature Name}
%%c \in \mathbf{C} &\textit{Configuration}
%\end{syntax}
%
%\medskip
\textbf{Feature expression syntax:}
\begin{syntax}
\synDef \fName \fSet &&&\textit{Feature Name}\\
\synDef \bTag \bSet &\eqq& \t \myOR \f & \textit{Boolean Value}\\
\synDef \dimMeta \ffSet &\eqq& \bTag \myOR \fName \myOR \neg \dimMeta \myOR \dimMeta \wedge \dimMeta \myOR \dimMeta \vee \dimMeta & \textit{Feature Expression}\\
\synDef \config \confSet &:& \fSet \totype \bSet &\textit{Configuration}
\end{syntax}

\medskip
\textbf{Evaluation of feature expressions:}
\begin{alignat*}{1}
\fSem [] . &: \ffSet \totype \confSet \totype \bSet\\
\fSem \bTag &= \bTag\\
\fSem \fName &= \config \ \fName\\
\fSem {\neg \fName} &= \neg \fSem \fName\\
\fSem {\annd \dimMeta} &= \fSem {\dimMeta_1} \wedge \fSem {\dimMeta_2}\\
\fSem {\orr \dimMeta} &= \fSem {\dimMeta_1} \vee \fSem {\dimMeta_2}\\
\end{alignat*}

\medskip
\textbf{Relations over feature expressions:}
\begin{alignat*}{1}
\dimMeta_1 \equiv \dimMeta_2 &\textit{ iff \ } \forall \config \in \confSet. \fSem {\dimMeta_1} = \fSem {\dimMeta_2}\\
\sat {\dimMeta} &\textit{ iff \ } \exists \config \in \confSet. \fSem {\dimMeta} = \t\\
\unsat {\dimMeta} &\textit{ iff \ } \forall \config \in \confSet. \fSem {\dimMeta} = \f\\
\oneof {\A, \B, \ldots, \fName_n}
&= (\A \wedge \neg \B \wedge \ldots \wedge \neg \fName_n)
\vee (\neg \A \wedge \B \wedge \ldots \wedge \neg \fName_n)\\
&\qquad\vee (\neg \A \wedge \neg \B \wedge \ldots \wedge \fName_n)
%\dimMeta_1 \oplus \dimMeta_2 = (\dimMeta_1 \wedge \neg \dimMeta_2) \vee (\dimMeta_2 \wedge \neg \dimMeta_1)
\end{alignat*}

%\medskip
%\textbf{Syntactic sugar for mutually exclusive features:}
%\begin{alignat*}{1}
%\A \oplus \B \oplus \ldots \oplus \fName_n
%= (\A \wedge \neg \B \wedge \ldots \wedge \neg \fName_n)\\
%\vee (\neg \A \wedge \B \wedge \ldots \wedge \neg \fName_n)\\
%\vee (\neg \A \wedge \neg \B \wedge \ldots \wedge \fName_n)
%\end{alignat*}


\begin{comment}
\medskip
\textbf{Configuration constraint:}
\begin{equation*}
\forall \overrightarrow{f_i} \in c :
%( \neg o_1 \wedge o_2 \wedge o_3 \wedge \ldots \wedge o_{|f_i|})
\bigvee_{1\leq j \leq |f_i|}(\neg o_j \wedge \bigwedge_{\substack{k\not = j\\1 \leq k\leq |f_i|}} o_k)
= \prog{true}
\end{equation*}
\end{comment}

\caption[Feature expression syntax and evaluation]{Feature expression syntax, relations, and evaluation.
% and functions over feature expressions. 
%We informally define \emph{exclusive or} \ensuremath{\oplus} of \ensuremath{n} features to be \t\ 
%only when one feature is \t.
}
\label{fig:fexp-def}
\end{figure}


\begin{figure}
\textbf{Evaluation of feature expressions:}
\begin{alignat*}{1}
\fSem [] . &: \ffSet \to \confSet \to \bSet\\
\fSem \bTag &= \bTag\\
\fSem \fName &= \config \ \fName\\
\fSem {\neg \fName} &= \neg \fSem \fName\\
\fSem {\annd \dimMeta} &= \fSem {\dimMeta_1} \wedge \fSem {\dimMeta_2}\\
\fSem {\orr \dimMeta} &= \fSem {\dimMeta_1} \vee \fSem {\dimMeta_2}\\
\end{alignat*}

\begin{comment}

\medskip
\textbf{Functions over feature expressions:}
\begin{alignat*}{1}
\mathit{sat}, \mathit{taut} &: \ffSet \to \bSet \\
\sat \dimMeta = \t &\textit{ iff \ } \exists \config \in \confSet: \fSem \dimMeta = \t\\
\taut \dimMeta = \t &\textit{ iff \ } \forall \config \in \confSet: \fSem \dimMeta = \t
\end{alignat*}
\end{comment}

\caption{Feature expression evaluation and functions.}
\label{fig:fexp-eval}
\end{figure}





\section{Annotations and Variational Sets}
\label{sec:vset}


%\point{annotating elements of database with feature expressions.}
We now introduce the first approach used to incorporate variation into a database.
To incorporate feature expressions into the database,
we \emph{annotate} database elements (including attributes, relations, and tuples) 
with feature expressions. An \emph{annotated element} \elem\ with feature expression \dimMeta\
is denoted by \annot \elem, 
that is, if \elem\ has type \typevar\ (i.e., $\elem \in \typevar$)
then $\annot \elem$ has the corresponding variational type 
$\vartype \typevar$ (i.e., $\annot \elem \in \vartype \typevar$).
%The feature expression \dimMeta\ represents
%the set of configurations where their variants contain element \elem\ because
%
The feature expression attached to an element is called its \emph{presence
condition} since it determines the condition (set of configurations) under
which the element is present in the database. 
This is done by the \emph{configuration} function $\xeSem [] . : \elemSet \totype \confSet \totype \maybe {\pelemSet}$ defined in \figref{vset}.
For example, assuming
$\features=\set{\A,\B}$, the annotated number $\annot [\A \vee \B] 2$ is present
in variants \setDef{\A} (i.e., $\xeSem [\setDef{\A}] {\annot [\A \vee \B] 2}$ = 2), 
\setDef{\B} (i.e., $\xeSem [\setDef{\A}] {\annot [\A \vee \B] 2}$ = 2), 
and \setDef{\A,\B} (i.e.,$\xeSem [\setDef{\A, \B}] {\annot [\A \vee \B] 2} = 2$) 
but not in variant
\setDef{} (i.e., $\xeSem [\setDef { }] {\annot [\A \vee \B] 2} = \bot$). 
%
The operation $\getPC{\annot{\elem}}=e$ returns the presence condition of an
annotated element.
% with a configuration
%that enables either $\A$ or $\B$ or both
%variants that disable both $\A$ and $\B$.
% Here, $\getPC {\annot [\A \vee \B] 2} = \A \vee \B$.

\section{Annotations and Variational Sets}
\label{sec:vset}


%\point{annotating elements of database with feature expressions.}
We now introduce the first approach used to incorporate variation into a database.
To incorporate feature expressions into the database,
we \emph{annotate} database elements (including attributes, relations, and tuples) 
with feature expressions. An \emph{annotated element} \elem\ with feature expression \dimMeta\
is denoted by \annot \elem, 
that is, if \elem\ has type \typevar\ (i.e., $\elem \in \typevar$)
then $\annot \elem$ has the corresponding variational type 
$\vartype \typevar$ (i.e., $\annot \elem \in \vartype \typevar$).
%The feature expression \dimMeta\ represents
%the set of configurations where their variants contain element \elem\ because
%
The feature expression attached to an element is called its \emph{presence
condition} since it determines the condition (set of configurations) under
which the element is present in the database. 
This is done by the \emph{configuration} function $\xeSem [] . : \elemSet \totype \confSet \totype \maybe {\pelemSet}$ defined in \figref{vset}.
For example, assuming
$\features=\set{\A,\B}$, the annotated number $\annot [\A \vee \B] 2$ is present
in variants \setDef{\A} (i.e., $\xeSem [\setDef{\A}] {\annot [\A \vee \B] 2}$ = 2), 
\setDef{\B} (i.e., $\xeSem [\setDef{\A}] {\annot [\A \vee \B] 2}$ = 2), 
and \setDef{\A,\B} (i.e.,$\xeSem [\setDef{\A, \B}] {\annot [\A \vee \B] 2} = 2$) 
but not in variant
\setDef{} (i.e., $\xeSem [\setDef { }] {\annot [\A \vee \B] 2} = \bot$). 
%
The operation $\getPC{\annot{\elem}}=e$ returns the presence condition of an
annotated element.
% with a configuration
%that enables either $\A$ or $\B$ or both
%variants that disable both $\A$ and $\B$.
% Here, $\getPC {\annot [\A \vee \B] 2} = \A \vee \B$.

\input{formulas/vset}

%\point{vset.}
A \emph{variational set} (\emph{v-set}) $\vset = \setDef {\annot [\dimMeta_1] {\elem_1},\ldots, \annot [\dimMeta_n] {\elem_n}}$ 
is a set of annotated elements, 
that is,
$\vset \in \vsetSet$~\cite{EWC13fosd,Walk14onward,ATW17dbpl}.
We typically omit the presence condition \t\ in a variational set,
e.g., $\annot [\t] 4 = 4$.
% where the presence condition of elements is satisfiable~\cite{EWC13fosd,Walk14onward,vdb17ATW}. 
%
Conceptually, a variational set represents many different plain sets simultaneously.
These plain sets can be generated by \emph{configuring} a variational set with a configuration.
This is done by the \emph{variational set configuration} function
\ensuremath{\osetSem \vset: \vsetSet \totype \confSet \totype \psetSet}, defined in \figref{vset}.
The configuration function evaluates the presence condition $\dimMeta_i$ of each 
element $\elem_i$ of the variational set with the configuration \config. 
If the evaluation results in \t\ it includes $\elem_i$ in the plain set and otherwise it
does not. \exref{vset-conf} illustrates the configuration of a variational set for all
possible configurations. 
\structure{it'd be nice to have the entire ex in the same page.}

\begin{example}
\label{eg:vset-conf}
Assume we have the feature space $\features = \setDef {\A, \B}$ 
and the variational set $\vset_1 = \setDef {\annot [\A] 2, \annot [\B] 3, 4}$.
$\vset_1$ represents four plain sets:
\begin{alignat*}{1}
\osetSem {\vset_1} &=
\begin{cases}
  \setDef{2,3,4}, & \config = \setDef{\A,\B}\\
  \setDef{2,4}, & \config = \setDef{\A}\\
  \setDef{3,4}, & \config = \setDef{\B}\\
  \setDef{4}, & \config = \setDef { }
\end{cases}
\end{alignat*}
This states that, for example, configuring $\vset_1$ for the variant that enables 
bot \A\ and \B\ (that is, \ensuremath{\A = \t, \B = \t}) results in the plain set
\ensuremath{ \osetSem [\setDef {\A, \B}] {\vset_1} = \setDef {2,3,4} }.
\end{example}

%
%\noindent
Following database notational conventions
we drop the brackets of a variational set when used in database
schema definitions and queries.

%\point{annotated vset.}
A variational set itself can also be annotated with a feature expression.
%
%An \emph{annotated variational set} 
$\annot \vset = \setDef {\annot [\dimMeta_1] {\elem_1},\ldots,\annot [\dimMeta_n] {\elem_n}}^\dimMeta$ is an
\emph{annotated variational set}, 
that is, $\annot \vset \in \annotvsetSet$.
% that it is annotated itself by a \emph{feature expression} \dimMeta.
%We denote an annotated variational set of elements $\elem \in \mathbf{\elemSet}$ with
%\annot \elemSet.
Annotating a variational set with the feature expression \dimMeta\ means that all
elements in the variational set are only present when \dimMeta\ evaluates to \t.
The \emph{normalization} operation $\pushIn {\annot \vset}$ applies this
restriction by pushing it into the presence conditions of the individual
elements:
\ensuremath{
\pushIn {\annot \vset}
= 
\setDef{\annot [\dimMeta_i \wedge \dimMeta] {\elem_i} \myOR 
\annot [\dimMeta_i] \elem_i \in \annot \vset, \sat {\dimMeta_i \wedge \dimMeta}
}}.
%\eric{added that both v-set and annot v-set are of the same type.}
%Thus, we consider both variational sets and annotated variational sets to 
%belong to the set of variational set \vsetSet, that is, we consider them to have the same type. 
Note that both the normalization operation and variational set configuration
are overloaded, that is, they are defined for both variational sets and 
annotated variational sets. 
Also, note that the \emph{normalization} operation also removes elements
with unsatisfiable presence conditions and may also be applied
to an unannotated variational set \vset\ since $\annot[\t]{\vset} = \vset$.
%\ensuremath{
%\vset = \setDef {\annot [\dimMeta_1] \elem_1, \ldots, \annot [\dimMeta_n] \elem_n}}, 
%which is equivalent to the annotated v-set \annot [\t] \vset. Thus,
%\ensuremath{
%\pushIn \vset = \setDef {
%\annot [\dimMeta_i] \elem_i \myOR \annot [\dimMeta_i] \elem_i, \sat {\dimMeta_i}
%}
%}.}
%This restriction
%can be captured by the property:
%$\setDef {\annot [\dimMeta_1] {\elem_1} ,\ldots, \annot [\dimMeta_n] {\elem_n}}^\dimMeta
%\equiv 
%\setDef {\annot [\dimMeta_1 \wedge \dimMeta] {\elem_1},\ldots, \annot [\dimMeta_n \wedge \dimMeta] {\elem_n}}
%$.
%
For example, the annotated variational set
$\vset_1 = \{\annot [\A] 2, \annot [\neg \B] 3, 4, \annot [\C] 5\}^{\A \wedge \B}$
indicates that all the elements of the set can only exist
when both $\A$ and $\B$ are enabled. Thus, normalizing the variational set $\vset_1$
%the set's feature expression
results in
$\{\annot [\A \wedge \B] 2,\annot [\A \wedge \B] 4,\annot [\A \wedge \B \wedge \C] 5\}$. The element $3$ is dropped 
%from the set 
since 
\ensuremath{\neg \sat {\getPCfrom 3 {\vset_1} }},
where
\ensuremath{
{\getPCfrom 3 {\vset_1} } = \neg \B \wedge (\A \wedge \B)}.
%its presence condition is unsatisfiable, i.e., $\neg \sat {\neg \fName_2 \wedge (\fName_1 \wedge \fName_2)}$.
%%
Note that we use the function \getPCfrom \elem {\annot \vset} to 
return the presence condition of a unique variational element within a bigger
variational structure. 
Note that,
without loss of generality, we assume that elements in a variational set
are unique since we can simply disjoin the presence conditions of a repeated 
element, that is, 
\ensuremath{\setDef {\annot [\dimMeta] \elem, \annot [\dimMeta] \elem, \annot [\dimMeta_1] \elem_1, \ldots, \annot [\dimMeta_n] \elem_n} = \setDef {\annot [\dimMeta \vee \VVal \dimMeta] \elem, \annot [\dimMeta_1] \elem_1, \ldots, \annot [\dimMeta_n] \elem_n}}.
% by just referring to the element itself without its
%annotation, i.e., \elem.

In \figref{vset}, we also define several operations, such as union and
intersection, over variational sets; these operations are used in \secref{type-sys}. The
semantics of a variational set operation is equivalent to applying the corresponding
plain set operation to every corresponding variant of the argument variational sets. For
example, the union of two variational sets $\vset_1\cup\vset_2$ should produce a new
variational set $\vset_3$ such that
%
$\forall c\in\confSet.\;
\osetSem{\vset_3} = \osetSem{\vset_1}\,\underline{\cup}\,\osetSem{\vset_2}$,
where $\underline{\cup}$ is the plain set union operation.
%
 This property must hold for all operations over variational sets, that is, for all possible operations, \vsetOp, defined on variational sets the property 
 \ensuremath{
 \Pone: 
 \forall \config \in \confSet. \osetSem {\pushIn {\vset_1} \vsetOp \pushIn {\vset_2}} 
 = \osetSem {\vset_1} \psetOp \osetSem {\vset_2}
 } must hold, where \psetOp\ is the counterpart operation on plain sets.%
\footnote{This property is proved for the operations we define over variational sets in Coq proof assistant~\cite{Khan21}.}



%\point{vset.}
A \emph{variational set} (\emph{v-set}) $\vset = \setDef {\annot [\dimMeta_1] {\elem_1},\ldots, \annot [\dimMeta_n] {\elem_n}}$ 
is a set of annotated elements, 
that is,
$\vset \in \vsetSet$~\cite{EWC13fosd,Walk14onward,ATW17dbpl}.
We typically omit the presence condition \t\ in a variational set,
e.g., $\annot [\t] 4 = 4$.
% where the presence condition of elements is satisfiable~\cite{EWC13fosd,Walk14onward,vdb17ATW}. 
%
Conceptually, a variational set represents many different plain sets simultaneously.
These plain sets can be generated by \emph{configuring} a variational set with a configuration.
This is done by the \emph{variational set configuration} function
\ensuremath{\osetSem \vset: \vsetSet \totype \confSet \totype \psetSet}, defined in \figref{vset}.
The configuration function evaluates the presence condition $\dimMeta_i$ of each 
element $\elem_i$ of the variational set with the configuration \config. 
If the evaluation results in \t\ it includes $\elem_i$ in the plain set and otherwise it
does not. \exref{vset-conf} illustrates the configuration of a variational set for all
possible configurations. 
\structure{it'd be nice to have the entire ex in the same page.}

\begin{example}
\label{eg:vset-conf}
Assume we have the feature space $\features = \setDef {\A, \B}$ 
and the variational set $\vset_1 = \setDef {\annot [\A] 2, \annot [\B] 3, 4}$.
$\vset_1$ represents four plain sets:
\begin{alignat*}{1}
\osetSem {\vset_1} &=
\begin{cases}
  \setDef{2,3,4}, & \config = \setDef{\A,\B}\\
  \setDef{2,4}, & \config = \setDef{\A}\\
  \setDef{3,4}, & \config = \setDef{\B}\\
  \setDef{4}, & \config = \setDef { }
\end{cases}
\end{alignat*}
This states that, for example, configuring $\vset_1$ for the variant that enables 
bot \A\ and \B\ (that is, \ensuremath{\A = \t, \B = \t}) results in the plain set
\ensuremath{ \osetSem [\setDef {\A, \B}] {\vset_1} = \setDef {2,3,4} }.
\end{example}

%
%\noindent
Following database notational conventions
we drop the brackets of a variational set when used in database
schema definitions and queries.

%\point{annotated vset.}
A variational set itself can also be annotated with a feature expression.
%
%An \emph{annotated variational set} 
$\annot \vset = \setDef {\annot [\dimMeta_1] {\elem_1},\ldots,\annot [\dimMeta_n] {\elem_n}}^\dimMeta$ is an
\emph{annotated variational set}, 
that is, $\annot \vset \in \annotvsetSet$.
% that it is annotated itself by a \emph{feature expression} \dimMeta.
%We denote an annotated variational set of elements $\elem \in \mathbf{\elemSet}$ with
%\annot \elemSet.
Annotating a variational set with the feature expression \dimMeta\ means that all
elements in the variational set are only present when \dimMeta\ evaluates to \t.
The \emph{normalization} operation $\pushIn {\annot \vset}$ applies this
restriction by pushing it into the presence conditions of the individual
elements:
\ensuremath{
\pushIn {\annot \vset}
= 
\setDef{\annot [\dimMeta_i \wedge \dimMeta] {\elem_i} \myOR 
\annot [\dimMeta_i] \elem_i \in \annot \vset, \sat {\dimMeta_i \wedge \dimMeta}
}}.
%\eric{added that both v-set and annot v-set are of the same type.}
%Thus, we consider both variational sets and annotated variational sets to 
%belong to the set of variational set \vsetSet, that is, we consider them to have the same type. 
Note that both the normalization operation and variational set configuration
are overloaded, that is, they are defined for both variational sets and 
annotated variational sets. 
Also, note that the \emph{normalization} operation also removes elements
with unsatisfiable presence conditions and may also be applied
to an unannotated variational set \vset\ since $\annot[\t]{\vset} = \vset$.
%\ensuremath{
%\vset = \setDef {\annot [\dimMeta_1] \elem_1, \ldots, \annot [\dimMeta_n] \elem_n}}, 
%which is equivalent to the annotated v-set \annot [\t] \vset. Thus,
%\ensuremath{
%\pushIn \vset = \setDef {
%\annot [\dimMeta_i] \elem_i \myOR \annot [\dimMeta_i] \elem_i, \sat {\dimMeta_i}
%}
%}.}
%This restriction
%can be captured by the property:
%$\setDef {\annot [\dimMeta_1] {\elem_1} ,\ldots, \annot [\dimMeta_n] {\elem_n}}^\dimMeta
%\equiv 
%\setDef {\annot [\dimMeta_1 \wedge \dimMeta] {\elem_1},\ldots, \annot [\dimMeta_n \wedge \dimMeta] {\elem_n}}
%$.
%
For example, the annotated variational set
$\vset_1 = \{\annot [\A] 2, \annot [\neg \B] 3, 4, \annot [\C] 5\}^{\A \wedge \B}$
indicates that all the elements of the set can only exist
when both $\A$ and $\B$ are enabled. Thus, normalizing the variational set $\vset_1$
%the set's feature expression
results in
$\{\annot [\A \wedge \B] 2,\annot [\A \wedge \B] 4,\annot [\A \wedge \B \wedge \C] 5\}$. The element $3$ is dropped 
%from the set 
since 
\ensuremath{\neg \sat {\getPCfrom 3 {\vset_1} }},
where
\ensuremath{
{\getPCfrom 3 {\vset_1} } = \neg \B \wedge (\A \wedge \B)}.
%its presence condition is unsatisfiable, i.e., $\neg \sat {\neg \fName_2 \wedge (\fName_1 \wedge \fName_2)}$.
%%
Note that we use the function \getPCfrom \elem {\annot \vset} to 
return the presence condition of a unique variational element within a bigger
variational structure. 
Note that,
without loss of generality, we assume that elements in a variational set
are unique since we can simply disjoin the presence conditions of a repeated 
element, that is, 
\ensuremath{\setDef {\annot [\dimMeta] \elem, \annot [\dimMeta] \elem, \annot [\dimMeta_1] \elem_1, \ldots, \annot [\dimMeta_n] \elem_n} = \setDef {\annot [\dimMeta \vee \VVal \dimMeta] \elem, \annot [\dimMeta_1] \elem_1, \ldots, \annot [\dimMeta_n] \elem_n}}.
% by just referring to the element itself without its
%annotation, i.e., \elem.

In \figref{vset}, we also define several operations, such as union and
intersection, over variational sets; these operations are used in \secref{type-sys}. The
semantics of a variational set operation is equivalent to applying the corresponding
plain set operation to every corresponding variant of the argument variational sets. For
example, the union of two variational sets $\vset_1\cup\vset_2$ should produce a new
variational set $\vset_3$ such that
%
$\forall c\in\confSet.\;
\osetSem{\vset_3} = \osetSem{\vset_1}\,\underline{\cup}\,\osetSem{\vset_2}$,
where $\underline{\cup}$ is the plain set union operation.
%
 This property must hold for all operations over variational sets, that is, for all possible operations, \vsetOp, defined on variational sets the property 
 \ensuremath{
 \Pone: 
 \forall \config \in \confSet. \osetSem {\pushIn {\vset_1} \vsetOp \pushIn {\vset_2}} 
 = \osetSem {\vset_1} \psetOp \osetSem {\vset_2}
 } must hold, where \psetOp\ is the counterpart operation on plain sets.%
\footnote{This property is proved for the operations we define over variational sets in Coq proof assistant~\cite{Khan21}.}


\section{The Formula Choice Calculus}
\label{sec:fcc}


%To account for variation, VRA combines relational algebra (RA) with 
%\emph{choices}~\cite{EW11tosem,HW16fosd,Walk13thesis}.
%%\point{choice.}
%A choice $\chc{\elem_1,\elem_2}$ consists of a feature expression \dimMeta, called
%the \emph{dimension} of the choice, and 
%two \emph{alternatives} $\elem_1$ and $\elem_2$. For a given configuration \config, 
%the choice $\chc{\elem_1, \elem_2}$ can be replaced by $\elem_1$ if \dimMeta\
%evaluates to \t\ under configuration \config, (i.e., \fSem{\dimMeta}),
%or $\elem_2$ otherwise. 

\eric{please read the entire section. thx!}
The second approach we use to incorporate variation into queries is
the formula choice calculus~\cite{HW16fosd} which is an extension of 
the choice calculus~\cite{Walk13thesis,EW11tosem}. 
%
The choice calculus~\cite{Walk13thesis,EW11tosem} is a metalanguage for
describing variation in programs and its elements such as data 
structures~\cite{Walk14onward,EWC13fosd}.
In the choice calculus, variation is represented in-place as
choices between alternative subexpressions. For example, 
the variational expression 
$\mathit{expr} = \chc [\A] {1,2} + \chc [\B] {3,4} + \chc [\A] {5,6}$
 contains three choices.
Each choice has an associated \emph{dimension}, which is used to
synchronize the choice with other choices in different parts
of the expression. For example, expression $\mathit{expr}$ contains
two dimensions, $\A$ and $\B$, and the two choices in dimension
$\A$ are synchronized. Therefore, the variational expression
$\mathit{expr}$ represents four different plain expressions, depending
on whether the left or right alternatives are selected from each
dimension. Assuming that dimensions may be set to boolean values
where \t\ indicates the left alternative and \f\ indicates the
right alternative, we have: 
%(1) $1+3+5$, when $A$ and $B$ are \t,
%(2) $1+4+5$, when $A$ is \t\ and $B$ is \f,
%(3) $2+3+6$, when $A$ is \f\ and $B$ is \t,
%and (4) $2+4+6$, when $A$ and $B$ are \f.
\begin{alignat*}{1}
\chc [\A] {1,2} + \chc [\B] {3,4} + \chc [\A] {5,6} &=
\begin{cases}
  1+3+5,& \A =\t, \B = \t\\
  1+4+5,& \A =\t, \B = \f\\
  2+3+6,& \A =\f, \B = \t\\
  2+4+6,& \A =\f, \B = \f
\end{cases}
\end{alignat*}
%
\noindent
The formula
choice calculus extends the choice calculus 
by allowing dimensions to be propositional formulas~\cite{HW16fosd}. For example,
the variational expression $\VVal {\mathit{expr}} = \chc [\A \vee \B] {1,2} + \chc [\B] {3,4} + \chc [\A] {5,6}$ represents
four plain expressions: 
%(1) $1$, when $\A \vee \B$ evaluates to \t\
%and (2) $2$, when $\A \vee \B$ evaluates to \f. More explicitly, we have:
\begin{alignat*}{1}
\chc [\A \vee \B] {1,2} + \chc [\B] {3,4} + \chc [\A] {5,6}&=
\begin{cases}
  1+3+5,& \A =\t, \B = \t\\
  1+4+5,& \A =\t, \B = \f\\
  1+3+6,& \A =\f, \B = \t\\
  2+4+6,& \A =\f, \B = \f
\end{cases}
\end{alignat*}



%rdb
%fcc
\chapter{The Variational Database Framework}
\label{ch:frame}

\TODO{needs. must have configuration. }

\section{Variational Needs in a Relational Database}
\label{sec:varneeds}

\TODO{needs and examples of them.}

\section{Variation Space in a Variational Database Framework}
\label{sec:varspace}

\TODO{have to revise}
\TODO{define oplus in fig as syntactic sugar.}

%\point{using a feature set to represent variability within a context.}
To account for variability in a database we need to 
encode it.
%
%The first challenge of incorporating variability into a database
%is to represent variability. 
%
To encode variability we introduce a \emph{feature space} \fSet\ as 
a closed set of boolean variables. 
A feature \ensuremath{\fName \in \fSet} can be enabled (i.e., \fName = \t) or disabled (\fName = \f).
Features capture the variation in a given variational scenario.
%first organize the configuration space into
%a set of features \fSet.
%, denoted by \fSet.
%
%we require a \emph{set of features}, denoted by \fSet, 
%appropriate for the context that the database is used for.
%
For example, in the context of schema evolution, features can be generated from version 
numbers (e.g. features \vOne\ to \vFive\ and \tOne\ to \tFive\ in the 
motivating example, \tabref{mot}); for SPLs, 
the features can be adopted from the SPL feature set (e.g. the \edu\ feature in
our motivating example, \tabref{mot}); and 
for data integration, the features can be representatives of resources.  
%For simplicity, the set of features is assumed to be closed and features are
%assumed to be boolean variables, however, it is easy to extend them
%to multi-valued variables that have finite set of values.
% and without loss of generality, 
%features are assumed to be boolean variables, although, it is easy to extend them
%to multi-valued variables. 
%A feature \ensuremath{\fName \in \fSet} can be enabled (i.e., \fName = \t) or disabled (\fName = \f).
%\point{configuration.}
%Assuming that all the features by default are set to \prog{false},
%enabling some of them specifies a variant. 

Features describe local points of variation in a database.
%, i.e,
%in \fSet\ 
%are used to 
%they indicate which parts of the database are present conditionally.
%of a variational entity within the database 
%are different 
%among different variants. 
Thus, enabling or disabling all
the features 
%in \fSet\ 
produces a particular database \emph{variant} where
%of the entity in which 
all variation has been removed. 
%Enabling or disabling all features of \fSet\ specifies a non-variational \emph{variant}
%that can potentially be generated by \emph{configur}ing its variational counterpart 
%with the variant's \emph{configuration}.
%Hence, to specify a variant
%we define a function, called 
A \emph{configuration} is a \emph{total} function
that maps every feature in the feature set to a boolean value.
%By definition, a configuration is a \emph{total} function,
%i.e., it is defined for \emph{all} features in the feature set. 
%For brevity, 
We represent a configuration  by the set of enabled features.
%which represents a variant. 
%\ensure{make sure referring to a variant can be done by the set of enabled features. HERE!}
For example, in our motivating scenario, the configuration \ensuremath{
\setDef {\vTwo,\tThree,\edu}
}
represents a database variant where only features \vTwo, \tThree, and \edu\ are enabled.
This database variant contains relation schemas in the yellow cells of \tabref{mot}.
%of the
%employee and education sub-schemas associated with \vTwo\ and
%\tThree\ in \tabref{mot}, respectively.
%For brevity, 
We refer to a variant with configuration \config\ as variant \config, e.g.,
%For example, 
variant \setDef {\vTwo,\tThree,\edu} refers to the variant
with configuration \setDef {\vTwo,\tThree,\edu}.
%and education sub-schema associated with \tThree\ in \tabref{mot}.

%\point{represent variability in db by prop formula of features, called feature expression.}
%Having defined a set of features, we need to incorporate them into the database.
%To encode features in the database, we construct propositional formulas of features
%such that the formula evaluates to \t\ for a set of configurations. 
When describing variation points in the database, we need to 
refer to subsets of the configuration space. We achieve this by 
constructing propositional formulas of features.
Thus, 
such a propositional formula defines a condition that holds for 
a subset of configurations and their corresponding variants. 
%
%describing the condition where one or more variants are present,
%i.e., assigning features to their values defined in variant's configuration and 
%evaluating the propositional formula results in \prog{true}.
%
For example,
the propositional formula $\neg \edu$ represents all variants of our
motivating example that do not 
have the education part of the schema, i.e., variant schemas of the 
left schema column. 

We call a propositional formula of features a \emph{feature expression} and define
it formally in \figref{fexp-def}. 
%\figref{fexp-def} defines the syntax of feature expressions.
The evaluation function of feature expressions 
$\fSem \dimMeta : \ffSet \to \confSet \to \bSet$ simply substitutes each
feature \fName\ in the expression \dimMeta\ with its boolean value 
set in the given configuration \config\ and then
simplifies the propositional formula to a boolean value.
%  evaluates the 
%feature expression \dimMeta\ w.r.t. the configuration \config.
%and defined in
%our technical report~\cite{vldbArXiv}, 
%and defined in \appref{fexp},
%evaluates feature expression \dimMeta\ under configuration \config,
%also called \emph{configuration of feature expression \dimMeta\ under \config}. 
For example, assuming that 
\ensuremath{\fSet = \setDef{\A, \B}}
 $\fSem [\{\A\}] {\A \vee \B} = \t \vee \f = \t$, however,
%which states that the feature expression
%$\A \vee \B$ evaluates to \t\ w.r.t. configuration that only enables $\A$, however,
$\fSem [\{\}] {\A \vee \B} = \f \vee \f = \f$.
%, where the empty set indicates neither \A\ nor \B\
%are enabled.
%which states that the same feature expression evaluates
%to \f\ when neither $\A$ nor $\B$ are enabled.
Additionally, we define the binary \emph{equivalence ($\equiv$)} relation and
the unary \emph{satisfiable (sat)} and \emph{unsatisfiable (unsat)}
relations over feature expressions in \figref{fexp-def}.
%
% some of the functions needed over feature
%expressions:
%1) \emph{equivalence} of two feature expressions, 
%\ensuremath{\dimMeta_1 \equiv \dimMeta_2}
%and
%2) \emph{satisfiability} of a feature expression, 
%\ensuremath{\sat \dimMeta}.
%However, we define the evaluation of feature expressions and functions over them
%in \appref{fexp}.
%We define two functions over feature expressions, as shown in \figref{fexp-def}:
%1) \emph{satisfiability}: feature expression \dimMeta\ is \emph{satisfiable} if there 
%exists configuration \config\ s.t. \fSem \dimMeta = \t\
%and 2) \emph{tautology}: feature expression \dimMeta\ is a \emph{tautology} if  
%for all valid configurations we have: \fSem \dimMeta = \t.

%\point{annotating elements of database with feature expressions.}
To incorporate feature expressions into the database,
we \emph{annotate/tag} database elements (including attributes, relations, and tuples) 
with feature expressions. An \emph{annotated element} \elem\ with feature expression \dimMeta\
is denoted by \annot \elem. 
%The feature expression \dimMeta\ represents
%the set of configurations where their variants contain element \elem\ because
%
The feature expression attached to an element is called a \emph{presence condition}
since it determines the condition (set of configurations) under which the element is present.
\getPC {\annot \elem} returns the presence condition of the variational element
\annot \elem.
For example, the
annotated number $\annot [\A \vee \B] 2$ is present in variants
\setDef \A, \setDef \B, 
\ensuremath{\setDef {\A, \B}}
% with a configuration
%that enables either $\A$ or $\B$ or both
but it does not exist in variant \setDef {}.
%variants that disable both $\A$ and $\B$.
Here, $\getPC {\annot [\A \vee \B] 2} = \A \vee \B$.

%\point{relationship between features is captured by a propositional formula, called feature model.} 
No matter the context, features often have a relationship with each other that
constrains configurations. For example, only one of the temporal features of $\vOne - \vFive$
can be \t\ for a given variant.
This relationship is captured by a feature expression, called a \emph{feature model} and
denoted by \fModel,
which restricts the set of \emph{valid configurations}:
if configuration \config\ violates the relationship then 
%evaluating the feature model \fModel\
%under this configuration evaluates to \f: 
it follows from the \fSem . definition that \fSem \fModel = \f.
For example, the restriction that at a given time only one of temporal features $\vOne - \vFive$
can be enabled is represented by:
%the feature expression:
\ensuremath{
\vOne \oplus \vTwo \oplus \vThree \oplus \vFour \oplus \vFive
},
where $\A \oplus \B \oplus \ldots \oplus \fName_n$ is syntactic sugar for $(\A \wedge \neg \B \wedge \ldots \wedge \neg \fName_n) \vee (\neg \A \wedge \B \wedge \ldots \wedge \neg \fName_n) \vee (\neg \A \wedge \neg \B \wedge \ldots \wedge \fName_n)$
, i.e., features are mutually exclusive.
%multiple features cannot be enabled simultaneously.
%$\left(\vOne \wedge \neg \left(\vTwo \vee \hdots \vee \vFive \right) \right)
%\vee \left(\vTwo \wedge \neg \left(\vOne \vee \vThree \vee \vFour \vee \vFive \right) \right) 
%\vee \hdots 
%\vee \left(\vFive \wedge \neg \left(\vOne \vee \hdots \vee \vFour \right) \right)$.
%Note that this is not the feature model for the entire motivating example.



\begin{figure}
%\textbf{Feature expression generic object:}
%\begin{syntax}
%\synDef \fName \fSet &\textit{Feature Name}
%%c \in \mathbf{C} &\textit{Configuration}
%\end{syntax}
%
%\medskip
\textbf{Feature expression syntax:}
\begin{syntax}
\synDef \fName \fSet &&&\textit{Feature Name}\\
\synDef \bTag \bSet &\eqq& \t \myOR \f & \textit{Boolean Value}\\
\synDef \dimMeta \ffSet &\eqq& \bTag \myOR \fName \myOR \neg \dimMeta \myOR \dimMeta \wedge \dimMeta \myOR \dimMeta \vee \dimMeta & \textit{Feature Expression}\\
\synDef \config \confSet &:& \fSet \totype \bSet &\textit{Configuration}
\end{syntax}

\medskip
\textbf{Evaluation of feature expressions:}
\begin{alignat*}{1}
\fSem [] . &: \ffSet \totype \confSet \totype \bSet\\
\fSem \bTag &= \bTag\\
\fSem \fName &= \config \ \fName\\
\fSem {\neg \fName} &= \neg \fSem \fName\\
\fSem {\annd \dimMeta} &= \fSem {\dimMeta_1} \wedge \fSem {\dimMeta_2}\\
\fSem {\orr \dimMeta} &= \fSem {\dimMeta_1} \vee \fSem {\dimMeta_2}\\
\end{alignat*}

\medskip
\textbf{Relations over feature expressions:}
\begin{alignat*}{1}
\dimMeta_1 \equiv \dimMeta_2 &\textit{ iff \ } \forall \config \in \confSet. \fSem {\dimMeta_1} = \fSem {\dimMeta_2}\\
\sat {\dimMeta} &\textit{ iff \ } \exists \config \in \confSet. \fSem {\dimMeta} = \t\\
\unsat {\dimMeta} &\textit{ iff \ } \forall \config \in \confSet. \fSem {\dimMeta} = \f\\
\oneof {\A, \B, \ldots, \fName_n}
&= (\A \wedge \neg \B \wedge \ldots \wedge \neg \fName_n)
\vee (\neg \A \wedge \B \wedge \ldots \wedge \neg \fName_n)\\
&\qquad\vee (\neg \A \wedge \neg \B \wedge \ldots \wedge \fName_n)
%\dimMeta_1 \oplus \dimMeta_2 = (\dimMeta_1 \wedge \neg \dimMeta_2) \vee (\dimMeta_2 \wedge \neg \dimMeta_1)
\end{alignat*}

%\medskip
%\textbf{Syntactic sugar for mutually exclusive features:}
%\begin{alignat*}{1}
%\A \oplus \B \oplus \ldots \oplus \fName_n
%= (\A \wedge \neg \B \wedge \ldots \wedge \neg \fName_n)\\
%\vee (\neg \A \wedge \B \wedge \ldots \wedge \neg \fName_n)\\
%\vee (\neg \A \wedge \neg \B \wedge \ldots \wedge \fName_n)
%\end{alignat*}


\begin{comment}
\medskip
\textbf{Configuration constraint:}
\begin{equation*}
\forall \overrightarrow{f_i} \in c :
%( \neg o_1 \wedge o_2 \wedge o_3 \wedge \ldots \wedge o_{|f_i|})
\bigvee_{1\leq j \leq |f_i|}(\neg o_j \wedge \bigwedge_{\substack{k\not = j\\1 \leq k\leq |f_i|}} o_k)
= \prog{true}
\end{equation*}
\end{comment}

\caption[Feature expression syntax and evaluation]{Feature expression syntax, relations, and evaluation.
% and functions over feature expressions. 
%We informally define \emph{exclusive or} \ensuremath{\oplus} of \ensuremath{n} features to be \t\ 
%only when one feature is \t.
}
\label{fig:fexp-def}
\end{figure}


\begin{figure}
\textbf{Evaluation of feature expressions:}
\begin{alignat*}{1}
\fSem [] . &: \ffSet \to \confSet \to \bSet\\
\fSem \bTag &= \bTag\\
\fSem \fName &= \config \ \fName\\
\fSem {\neg \fName} &= \neg \fSem \fName\\
\fSem {\annd \dimMeta} &= \fSem {\dimMeta_1} \wedge \fSem {\dimMeta_2}\\
\fSem {\orr \dimMeta} &= \fSem {\dimMeta_1} \vee \fSem {\dimMeta_2}\\
\end{alignat*}

\begin{comment}

\medskip
\textbf{Functions over feature expressions:}
\begin{alignat*}{1}
\mathit{sat}, \mathit{taut} &: \ffSet \to \bSet \\
\sat \dimMeta = \t &\textit{ iff \ } \exists \config \in \confSet: \fSem \dimMeta = \t\\
\taut \dimMeta = \t &\textit{ iff \ } \forall \config \in \confSet: \fSem \dimMeta = \t
\end{alignat*}
\end{comment}

\caption{Feature expression evaluation and functions.}
\label{fig:fexp-eval}
\end{figure}





\section{Annotations and Variational Sets}
\label{sec:vset}


%\point{annotating elements of database with feature expressions.}
We now introduce the first approach used to incorporate variation into a database.
To incorporate feature expressions into the database,
we \emph{annotate} database elements (including attributes, relations, and tuples) 
with feature expressions. An \emph{annotated element} \elem\ with feature expression \dimMeta\
is denoted by \annot \elem, 
that is, if \elem\ has type \typevar\ (i.e., $\elem \in \typevar$)
then $\annot \elem$ has the corresponding variational type 
$\vartype \typevar$ (i.e., $\annot \elem \in \vartype \typevar$).
%The feature expression \dimMeta\ represents
%the set of configurations where their variants contain element \elem\ because
%
The feature expression attached to an element is called its \emph{presence
condition} since it determines the condition (set of configurations) under
which the element is present in the database. 
This is done by the \emph{configuration} function $\xeSem [] . : \elemSet \totype \confSet \totype \maybe {\pelemSet}$ defined in \figref{vset}.
For example, assuming
$\features=\set{\A,\B}$, the annotated number $\annot [\A \vee \B] 2$ is present
in variants \setDef{\A} (i.e., $\xeSem [\setDef{\A}] {\annot [\A \vee \B] 2}$ = 2), 
\setDef{\B} (i.e., $\xeSem [\setDef{\A}] {\annot [\A \vee \B] 2}$ = 2), 
and \setDef{\A,\B} (i.e.,$\xeSem [\setDef{\A, \B}] {\annot [\A \vee \B] 2} = 2$) 
but not in variant
\setDef{} (i.e., $\xeSem [\setDef { }] {\annot [\A \vee \B] 2} = \bot$). 
%
The operation $\getPC{\annot{\elem}}=e$ returns the presence condition of an
annotated element.
% with a configuration
%that enables either $\A$ or $\B$ or both
%variants that disable both $\A$ and $\B$.
% Here, $\getPC {\annot [\A \vee \B] 2} = \A \vee \B$.

\section{Annotations and Variational Sets}
\label{sec:vset}


%\point{annotating elements of database with feature expressions.}
We now introduce the first approach used to incorporate variation into a database.
To incorporate feature expressions into the database,
we \emph{annotate} database elements (including attributes, relations, and tuples) 
with feature expressions. An \emph{annotated element} \elem\ with feature expression \dimMeta\
is denoted by \annot \elem, 
that is, if \elem\ has type \typevar\ (i.e., $\elem \in \typevar$)
then $\annot \elem$ has the corresponding variational type 
$\vartype \typevar$ (i.e., $\annot \elem \in \vartype \typevar$).
%The feature expression \dimMeta\ represents
%the set of configurations where their variants contain element \elem\ because
%
The feature expression attached to an element is called its \emph{presence
condition} since it determines the condition (set of configurations) under
which the element is present in the database. 
This is done by the \emph{configuration} function $\xeSem [] . : \elemSet \totype \confSet \totype \maybe {\pelemSet}$ defined in \figref{vset}.
For example, assuming
$\features=\set{\A,\B}$, the annotated number $\annot [\A \vee \B] 2$ is present
in variants \setDef{\A} (i.e., $\xeSem [\setDef{\A}] {\annot [\A \vee \B] 2}$ = 2), 
\setDef{\B} (i.e., $\xeSem [\setDef{\A}] {\annot [\A \vee \B] 2}$ = 2), 
and \setDef{\A,\B} (i.e.,$\xeSem [\setDef{\A, \B}] {\annot [\A \vee \B] 2} = 2$) 
but not in variant
\setDef{} (i.e., $\xeSem [\setDef { }] {\annot [\A \vee \B] 2} = \bot$). 
%
The operation $\getPC{\annot{\elem}}=e$ returns the presence condition of an
annotated element.
% with a configuration
%that enables either $\A$ or $\B$ or both
%variants that disable both $\A$ and $\B$.
% Here, $\getPC {\annot [\A \vee \B] 2} = \A \vee \B$.

\input{formulas/vset}

%\point{vset.}
A \emph{variational set} (\emph{v-set}) $\vset = \setDef {\annot [\dimMeta_1] {\elem_1},\ldots, \annot [\dimMeta_n] {\elem_n}}$ 
is a set of annotated elements, 
that is,
$\vset \in \vsetSet$~\cite{EWC13fosd,Walk14onward,ATW17dbpl}.
We typically omit the presence condition \t\ in a variational set,
e.g., $\annot [\t] 4 = 4$.
% where the presence condition of elements is satisfiable~\cite{EWC13fosd,Walk14onward,vdb17ATW}. 
%
Conceptually, a variational set represents many different plain sets simultaneously.
These plain sets can be generated by \emph{configuring} a variational set with a configuration.
This is done by the \emph{variational set configuration} function
\ensuremath{\osetSem \vset: \vsetSet \totype \confSet \totype \psetSet}, defined in \figref{vset}.
The configuration function evaluates the presence condition $\dimMeta_i$ of each 
element $\elem_i$ of the variational set with the configuration \config. 
If the evaluation results in \t\ it includes $\elem_i$ in the plain set and otherwise it
does not. \exref{vset-conf} illustrates the configuration of a variational set for all
possible configurations. 
\structure{it'd be nice to have the entire ex in the same page.}

\begin{example}
\label{eg:vset-conf}
Assume we have the feature space $\features = \setDef {\A, \B}$ 
and the variational set $\vset_1 = \setDef {\annot [\A] 2, \annot [\B] 3, 4}$.
$\vset_1$ represents four plain sets:
\begin{alignat*}{1}
\osetSem {\vset_1} &=
\begin{cases}
  \setDef{2,3,4}, & \config = \setDef{\A,\B}\\
  \setDef{2,4}, & \config = \setDef{\A}\\
  \setDef{3,4}, & \config = \setDef{\B}\\
  \setDef{4}, & \config = \setDef { }
\end{cases}
\end{alignat*}
This states that, for example, configuring $\vset_1$ for the variant that enables 
bot \A\ and \B\ (that is, \ensuremath{\A = \t, \B = \t}) results in the plain set
\ensuremath{ \osetSem [\setDef {\A, \B}] {\vset_1} = \setDef {2,3,4} }.
\end{example}

%
%\noindent
Following database notational conventions
we drop the brackets of a variational set when used in database
schema definitions and queries.

%\point{annotated vset.}
A variational set itself can also be annotated with a feature expression.
%
%An \emph{annotated variational set} 
$\annot \vset = \setDef {\annot [\dimMeta_1] {\elem_1},\ldots,\annot [\dimMeta_n] {\elem_n}}^\dimMeta$ is an
\emph{annotated variational set}, 
that is, $\annot \vset \in \annotvsetSet$.
% that it is annotated itself by a \emph{feature expression} \dimMeta.
%We denote an annotated variational set of elements $\elem \in \mathbf{\elemSet}$ with
%\annot \elemSet.
Annotating a variational set with the feature expression \dimMeta\ means that all
elements in the variational set are only present when \dimMeta\ evaluates to \t.
The \emph{normalization} operation $\pushIn {\annot \vset}$ applies this
restriction by pushing it into the presence conditions of the individual
elements:
\ensuremath{
\pushIn {\annot \vset}
= 
\setDef{\annot [\dimMeta_i \wedge \dimMeta] {\elem_i} \myOR 
\annot [\dimMeta_i] \elem_i \in \annot \vset, \sat {\dimMeta_i \wedge \dimMeta}
}}.
%\eric{added that both v-set and annot v-set are of the same type.}
%Thus, we consider both variational sets and annotated variational sets to 
%belong to the set of variational set \vsetSet, that is, we consider them to have the same type. 
Note that both the normalization operation and variational set configuration
are overloaded, that is, they are defined for both variational sets and 
annotated variational sets. 
Also, note that the \emph{normalization} operation also removes elements
with unsatisfiable presence conditions and may also be applied
to an unannotated variational set \vset\ since $\annot[\t]{\vset} = \vset$.
%\ensuremath{
%\vset = \setDef {\annot [\dimMeta_1] \elem_1, \ldots, \annot [\dimMeta_n] \elem_n}}, 
%which is equivalent to the annotated v-set \annot [\t] \vset. Thus,
%\ensuremath{
%\pushIn \vset = \setDef {
%\annot [\dimMeta_i] \elem_i \myOR \annot [\dimMeta_i] \elem_i, \sat {\dimMeta_i}
%}
%}.}
%This restriction
%can be captured by the property:
%$\setDef {\annot [\dimMeta_1] {\elem_1} ,\ldots, \annot [\dimMeta_n] {\elem_n}}^\dimMeta
%\equiv 
%\setDef {\annot [\dimMeta_1 \wedge \dimMeta] {\elem_1},\ldots, \annot [\dimMeta_n \wedge \dimMeta] {\elem_n}}
%$.
%
For example, the annotated variational set
$\vset_1 = \{\annot [\A] 2, \annot [\neg \B] 3, 4, \annot [\C] 5\}^{\A \wedge \B}$
indicates that all the elements of the set can only exist
when both $\A$ and $\B$ are enabled. Thus, normalizing the variational set $\vset_1$
%the set's feature expression
results in
$\{\annot [\A \wedge \B] 2,\annot [\A \wedge \B] 4,\annot [\A \wedge \B \wedge \C] 5\}$. The element $3$ is dropped 
%from the set 
since 
\ensuremath{\neg \sat {\getPCfrom 3 {\vset_1} }},
where
\ensuremath{
{\getPCfrom 3 {\vset_1} } = \neg \B \wedge (\A \wedge \B)}.
%its presence condition is unsatisfiable, i.e., $\neg \sat {\neg \fName_2 \wedge (\fName_1 \wedge \fName_2)}$.
%%
Note that we use the function \getPCfrom \elem {\annot \vset} to 
return the presence condition of a unique variational element within a bigger
variational structure. 
Note that,
without loss of generality, we assume that elements in a variational set
are unique since we can simply disjoin the presence conditions of a repeated 
element, that is, 
\ensuremath{\setDef {\annot [\dimMeta] \elem, \annot [\dimMeta] \elem, \annot [\dimMeta_1] \elem_1, \ldots, \annot [\dimMeta_n] \elem_n} = \setDef {\annot [\dimMeta \vee \VVal \dimMeta] \elem, \annot [\dimMeta_1] \elem_1, \ldots, \annot [\dimMeta_n] \elem_n}}.
% by just referring to the element itself without its
%annotation, i.e., \elem.

In \figref{vset}, we also define several operations, such as union and
intersection, over variational sets; these operations are used in \secref{type-sys}. The
semantics of a variational set operation is equivalent to applying the corresponding
plain set operation to every corresponding variant of the argument variational sets. For
example, the union of two variational sets $\vset_1\cup\vset_2$ should produce a new
variational set $\vset_3$ such that
%
$\forall c\in\confSet.\;
\osetSem{\vset_3} = \osetSem{\vset_1}\,\underline{\cup}\,\osetSem{\vset_2}$,
where $\underline{\cup}$ is the plain set union operation.
%
 This property must hold for all operations over variational sets, that is, for all possible operations, \vsetOp, defined on variational sets the property 
 \ensuremath{
 \Pone: 
 \forall \config \in \confSet. \osetSem {\pushIn {\vset_1} \vsetOp \pushIn {\vset_2}} 
 = \osetSem {\vset_1} \psetOp \osetSem {\vset_2}
 } must hold, where \psetOp\ is the counterpart operation on plain sets.%
\footnote{This property is proved for the operations we define over variational sets in Coq proof assistant~\cite{Khan21}.}



%\point{vset.}
A \emph{variational set} (\emph{v-set}) $\vset = \setDef {\annot [\dimMeta_1] {\elem_1},\ldots, \annot [\dimMeta_n] {\elem_n}}$ 
is a set of annotated elements, 
that is,
$\vset \in \vsetSet$~\cite{EWC13fosd,Walk14onward,ATW17dbpl}.
We typically omit the presence condition \t\ in a variational set,
e.g., $\annot [\t] 4 = 4$.
% where the presence condition of elements is satisfiable~\cite{EWC13fosd,Walk14onward,vdb17ATW}. 
%
Conceptually, a variational set represents many different plain sets simultaneously.
These plain sets can be generated by \emph{configuring} a variational set with a configuration.
This is done by the \emph{variational set configuration} function
\ensuremath{\osetSem \vset: \vsetSet \totype \confSet \totype \psetSet}, defined in \figref{vset}.
The configuration function evaluates the presence condition $\dimMeta_i$ of each 
element $\elem_i$ of the variational set with the configuration \config. 
If the evaluation results in \t\ it includes $\elem_i$ in the plain set and otherwise it
does not. \exref{vset-conf} illustrates the configuration of a variational set for all
possible configurations. 
\structure{it'd be nice to have the entire ex in the same page.}

\begin{example}
\label{eg:vset-conf}
Assume we have the feature space $\features = \setDef {\A, \B}$ 
and the variational set $\vset_1 = \setDef {\annot [\A] 2, \annot [\B] 3, 4}$.
$\vset_1$ represents four plain sets:
\begin{alignat*}{1}
\osetSem {\vset_1} &=
\begin{cases}
  \setDef{2,3,4}, & \config = \setDef{\A,\B}\\
  \setDef{2,4}, & \config = \setDef{\A}\\
  \setDef{3,4}, & \config = \setDef{\B}\\
  \setDef{4}, & \config = \setDef { }
\end{cases}
\end{alignat*}
This states that, for example, configuring $\vset_1$ for the variant that enables 
bot \A\ and \B\ (that is, \ensuremath{\A = \t, \B = \t}) results in the plain set
\ensuremath{ \osetSem [\setDef {\A, \B}] {\vset_1} = \setDef {2,3,4} }.
\end{example}

%
%\noindent
Following database notational conventions
we drop the brackets of a variational set when used in database
schema definitions and queries.

%\point{annotated vset.}
A variational set itself can also be annotated with a feature expression.
%
%An \emph{annotated variational set} 
$\annot \vset = \setDef {\annot [\dimMeta_1] {\elem_1},\ldots,\annot [\dimMeta_n] {\elem_n}}^\dimMeta$ is an
\emph{annotated variational set}, 
that is, $\annot \vset \in \annotvsetSet$.
% that it is annotated itself by a \emph{feature expression} \dimMeta.
%We denote an annotated variational set of elements $\elem \in \mathbf{\elemSet}$ with
%\annot \elemSet.
Annotating a variational set with the feature expression \dimMeta\ means that all
elements in the variational set are only present when \dimMeta\ evaluates to \t.
The \emph{normalization} operation $\pushIn {\annot \vset}$ applies this
restriction by pushing it into the presence conditions of the individual
elements:
\ensuremath{
\pushIn {\annot \vset}
= 
\setDef{\annot [\dimMeta_i \wedge \dimMeta] {\elem_i} \myOR 
\annot [\dimMeta_i] \elem_i \in \annot \vset, \sat {\dimMeta_i \wedge \dimMeta}
}}.
%\eric{added that both v-set and annot v-set are of the same type.}
%Thus, we consider both variational sets and annotated variational sets to 
%belong to the set of variational set \vsetSet, that is, we consider them to have the same type. 
Note that both the normalization operation and variational set configuration
are overloaded, that is, they are defined for both variational sets and 
annotated variational sets. 
Also, note that the \emph{normalization} operation also removes elements
with unsatisfiable presence conditions and may also be applied
to an unannotated variational set \vset\ since $\annot[\t]{\vset} = \vset$.
%\ensuremath{
%\vset = \setDef {\annot [\dimMeta_1] \elem_1, \ldots, \annot [\dimMeta_n] \elem_n}}, 
%which is equivalent to the annotated v-set \annot [\t] \vset. Thus,
%\ensuremath{
%\pushIn \vset = \setDef {
%\annot [\dimMeta_i] \elem_i \myOR \annot [\dimMeta_i] \elem_i, \sat {\dimMeta_i}
%}
%}.}
%This restriction
%can be captured by the property:
%$\setDef {\annot [\dimMeta_1] {\elem_1} ,\ldots, \annot [\dimMeta_n] {\elem_n}}^\dimMeta
%\equiv 
%\setDef {\annot [\dimMeta_1 \wedge \dimMeta] {\elem_1},\ldots, \annot [\dimMeta_n \wedge \dimMeta] {\elem_n}}
%$.
%
For example, the annotated variational set
$\vset_1 = \{\annot [\A] 2, \annot [\neg \B] 3, 4, \annot [\C] 5\}^{\A \wedge \B}$
indicates that all the elements of the set can only exist
when both $\A$ and $\B$ are enabled. Thus, normalizing the variational set $\vset_1$
%the set's feature expression
results in
$\{\annot [\A \wedge \B] 2,\annot [\A \wedge \B] 4,\annot [\A \wedge \B \wedge \C] 5\}$. The element $3$ is dropped 
%from the set 
since 
\ensuremath{\neg \sat {\getPCfrom 3 {\vset_1} }},
where
\ensuremath{
{\getPCfrom 3 {\vset_1} } = \neg \B \wedge (\A \wedge \B)}.
%its presence condition is unsatisfiable, i.e., $\neg \sat {\neg \fName_2 \wedge (\fName_1 \wedge \fName_2)}$.
%%
Note that we use the function \getPCfrom \elem {\annot \vset} to 
return the presence condition of a unique variational element within a bigger
variational structure. 
Note that,
without loss of generality, we assume that elements in a variational set
are unique since we can simply disjoin the presence conditions of a repeated 
element, that is, 
\ensuremath{\setDef {\annot [\dimMeta] \elem, \annot [\dimMeta] \elem, \annot [\dimMeta_1] \elem_1, \ldots, \annot [\dimMeta_n] \elem_n} = \setDef {\annot [\dimMeta \vee \VVal \dimMeta] \elem, \annot [\dimMeta_1] \elem_1, \ldots, \annot [\dimMeta_n] \elem_n}}.
% by just referring to the element itself without its
%annotation, i.e., \elem.

In \figref{vset}, we also define several operations, such as union and
intersection, over variational sets; these operations are used in \secref{type-sys}. The
semantics of a variational set operation is equivalent to applying the corresponding
plain set operation to every corresponding variant of the argument variational sets. For
example, the union of two variational sets $\vset_1\cup\vset_2$ should produce a new
variational set $\vset_3$ such that
%
$\forall c\in\confSet.\;
\osetSem{\vset_3} = \osetSem{\vset_1}\,\underline{\cup}\,\osetSem{\vset_2}$,
where $\underline{\cup}$ is the plain set union operation.
%
 This property must hold for all operations over variational sets, that is, for all possible operations, \vsetOp, defined on variational sets the property 
 \ensuremath{
 \Pone: 
 \forall \config \in \confSet. \osetSem {\pushIn {\vset_1} \vsetOp \pushIn {\vset_2}} 
 = \osetSem {\vset_1} \psetOp \osetSem {\vset_2}
 } must hold, where \psetOp\ is the counterpart operation on plain sets.%
\footnote{This property is proved for the operations we define over variational sets in Coq proof assistant~\cite{Khan21}.}


\begin{table}
\caption[Examples of relation schemas and a variational relation schema]{The relation schema of \empbio\ for variants that enable one of the features \vThree, \vFour, or \vFive\ and the variational relation schema of \empbio\ encompassing 
the three variants of the plain relation \empbio.}
\label{tab:empbio-sch}
\centering
\small
%\footnotesize
%\scriptsize
\begin{subtable}[t]{\textwidth}
\centering
\caption{The relation schema of \empbio\ for variants that enable the feature \vThree.}
\label{tab:empbio-v3}
\begin{tabular} {c | l l l}
%\hline
\multirow{2}{*}{\empbio} & \empno & \sex & \birthdate\\
\cline{2-4}
%\hline
 &12001 & F& 1960-11-06\\
\arrayrulecolor{white}\hline
\end{tabular}
\end{subtable}

\medskip
\medskip
\medskip
\begin{subtable}[t]{\textwidth}
\centering
\caption{The relation schema of \empbio\ for variants that enable the feature \vFour.}
\label{tab:empbio-v4}
\begin{tabular} {c | l l l l}
\multirow{2}{*}{\empbio}  & \empno & \sex & \birthdate & \name\\
\cline{2-5}
 &80001 & M & 1956-09-30 & Nagui Merli \\
\arrayrulecolor{white}\hline
\end{tabular}
\end{subtable}

\medskip
\medskip
\medskip
\begin{subtable}[t]{\textwidth}
\centering
\caption{The relation schema of \empbio\ for variants that enable the feature \vFive.}
\label{tab:empbio-v5}
\begin{tabular} {c | l l l l l}
\multirow{2}{*}{\empbio}  & \empno & \sex & \birthdate & \fname & \lname\\
\cline{2-6}
 & 200000 & M & 1960-01-11 & Selwyn & Koshiba \\
\arrayrulecolor{white}\hline
\end{tabular}
\end{subtable}

\medskip
\medskip
\medskip
\begin{subtable}[t]{\textwidth}
\centering
\footnotesize
\caption{The variational relation schema of \empbio.}
\label{tab:empbio-vsch}
\begin{tabular} {c | l l l l l l l}
%\hline
%\hhline{-==}
\textcolor{blue}{$\vThree \vee \vFour \vee \vFive$} & \textcolor{blue}{\t} & \textcolor{blue}{\t} & \textcolor{blue}{\t} & \textcolor{blue}{$\vFour \wedge \neg \vThree \wedge \neg \vFive$} & \textcolor{blue}{$\vFive \wedge \neg \vThree \wedge \neg \vFour$} & \textcolor{blue}{$\vFive \wedge \neg \vThree \wedge \neg \vFour$}\\
\arrayrulecolor{blue}\hdashline
\multirow{2}{*}{\empbio}  & \empno & \sex & \birthdate & \name & \fname & \lname\\
\arrayrulecolor{black}\cline{2-7}
 &12001 & F& 1960-11-06 & & &  \\
  &80001 & M & 1956-09-30 & Nagui Merli & & \\
   & 200000 & M & 1960-01-11 & & Selwyn & Koshiba \\
\arrayrulecolor{white}\hline
%\job & \titleatt & \salary\\
%\cline{2-3}
%& Assistant Engineer & 61594\\
%& Senior Engineer & 96646\\
%& \ldots & \ldots \\
%& Staff & 77935\\
%& Technique Leader & 58345
\end{tabular}
\end{subtable}

\end{table}

\section{Variational Table}
\label{sec:vtab}

%\TODO{change vdb conf to vtab.}
%\rewrite{read and revise}
%\fromppr{vldb}
%\TODO{remember to use definitions of config elem and vset}
%\TODO{give example of vtab for mot ex}
%%\begin{figure}
%[ht]
%
%\begin{comment}
%\textbf{Relational model generic objects:}
%\begin{syntax}
%%OLD
%D\in \mathbf{Dom} &&& \textit{Domain}\\
%A\in \mathbf{Att} &&& \textit{Attribute Name}\\
%R\in \mathbf{R} &&& \textit{Relation Name}\\
%t \in \mathbf{T} &&& \textit{Tuple}
%\end{syntax}
%
%\medskip
%\textbf{Relational model definition:}
%\begin{syntax}
%l\in \mathbf{L} &=& \vn{A} &\textit{Attribute set}\\
%s \in \mathbb{S} &=& R(A_1, \ldots , A_n ) & \textit{Relation specification}\\
%S \in \mathcal{\mathbf{S}} &\Coloneqq& {\vn{s}} & \textit{Schema}\\
%T \in \mathbf{T} &\Coloneqq& \{\llangle t(1), \ldots, t(k)\rrangle \myOR \\
%&&t(i) \in D_i,
%1 \leq i \leq k ,\\
%&&k = \mathit{arity}(R) \} 
%%v_1^1\in D_1, \ldots, v_n^1\in D_n\rrangle, \ldots, \llangle v_1^m\in D_1, \ldots, v_n^m\in D_n\rrangle|\\
%%&& \hspace{0.5cm} m = \textit{number of } R_I\textit{'s tuples}\} 
%&\textit{Relation Instance (Table)}\\
%%I \in \mathbf{Inst} &\Coloneqq& R_{1_I}, \cdots, R_{n_I} & \textit{Database Instance}
%\end{syntax}
%
%
%\medskip
%\textbf{Variational relational algebra objects:}
%\begin{syntax}
%\synDef \dimMeta \ffSet &&&\textit{Presence condition}\\
%\synDef \vAtt \vAttSet &&&\textit{Variational attribute}\\
%\synDef \vAttList \vAttSet &\eqq& \vAtt, \vAttList \myOR \empAtt &\textit{Variational attribute list}\\
%\synDef \vRelSch \vRelSet &\eqq& \vRelDef &\textit{Variational relation schema}\\
% \vRel &&&\textit{Variational relation}\\
%\synDef \vSch \vSchSet &\eqq& \vSchDef &\textit{Variational schema}\\
% &&&\textit{Variational database instance}
%\end{syntax}
%\end{comment}
%
%%%%%%%%%%%%%%%%%%%%%%%%%%%%%%%%%%%%%%%%%%%%%%%%%%%%
\textbf{Variational database objects:}
\begin{syntax}
%\synDef \vAtt \attNames &&&\textit{Attribute Name}\\
%\synDef \vRel \relNames &&& \textit{Relation Name}\\
%\synDef \vAttList {\boldmth{\mathbf{V} \mathbf{A}}} &\eqq& 
%\setDef {\annot [\dimMeta_1] \vAtt_1, \annot [\dimMeta_2] \vAtt_2, \ldots, \annot [\dimMeta_k] \vAtt_k} & \textit{Variational Set of Attributes}\\
%\synDef \vRelSch \vRelSchSet &\eqq& \vRelDef & \textit{Variational Relation Schema}\\
%\synDef \vSch \vSchSet &\eqq& \vSchDef & \textit{Variational Schema}
%
\synDef \vTuple {\vartype \tupletype} &\eqq& \annot[\dimMeta]{(\vi v \numAtts)} & \textit{Variational Tuple}\\
%\vTuple\in\vRelCont \eqq \annot[\dimMeta_\vTuple]{(\vi v \numAtts)}
\synDef \vRelCont \vRelContSet &\eqq& \setDef {\vi \vTuple \numTuples} & \textit{Variational Relation Content}\\
%\vRelCont \in \vRelContSet \eqq \setDef {\vi \vTuple \numTuples}
\synDef \vTab \tabletype &\eqq& (\vRelSch, \vRelCont) & \textit{Variational Table}\\
\synDef \vdbInst \vdbInstSet &\eqq& \annot [\dimMeta] {\setDef {\vi \vTab \numRels} } & \textit{Variational Database Instance}
%\synDef \vdbInst  \vdbInstSet \eqq \annot [\dimMeta] {\setDef {\vi \vTab \numRels} }
\end{syntax}

\medskip
\textbf{Variational database type synonyms:}
\begin{alignat*}{1}
\vRelContSet &= \settype {(\vartype \tupletype)}\\
\tabletype &= \typepair{\vRelSchSet, \vRelContSet}\\
\vdbInstSet &= \vartype {\left( \settype {\left(\left(\relschtype,\relconttype\right)\right)}\right)}
\end{alignat*}

\medskip
\textbf{Variational tuple configuration:}
%
\begin{alignat*}{1}%\raggedleft
\ouSemType [] . &: {\vartype \tupletype} \to \vRelSchSet \to \confSet \to \maybe \tupletype\\
%\end{flalign*}
%
%\begin{flalign*}%\raggedleft
\ouSem{\vRelSch} {\annot  {\left( {\vi v \numAtts}\right)}}  &\\
& \hspace{-50pt} = \begin{cases}
(v_i, \cdots, v_j), &\If \forall k. 1 \leq i \leq k \leq j \leq l, \fSem {\getPCfrom {\getAtt {k}} \vRelSch \wedge \dimMeta} = \t\\
\bot, &\Otherwise
\end{cases}
%\left( \ovSem {v_1}, \hdots, \ovSem {v_\numAtts} \right) &\\
%& \textit{ where } \forall 1 \leq i \leq \numAtts: \\
%&\hspace{5pt} \ovSem {v_i} = 
%\begin{cases}
%v_i, & \If \fSem {\fModel \wedge \getPC{\getRel{\getAtt{v_i}}} \wedge \getPC {\getAtt {v_i}} \wedge \dimMeta_\tuple} \\
%\varepsilon, & \Otherwise
%\end{cases}
\end{alignat*}

%\medskip
\textbf{Variational relation content configuration:}
%
\begin{alignat*}{1}%\raggedleft
\otSemType [] . &: \vRelContSet \to \vRelSchSet \to \confSet \to \pRelContSet\\
%\end{flalign*}
%
%\begin{flalign*}%\raggedleft
\otSem {\vRelSch} {\setDef {\vi \tuple \numTuples}} &= \setDef {\ouSem {\vRelSch}{\tuple_1}, \hdots, \ouSem{\vRelSch} {\tuple_\numTuples}}
\end{alignat*}

%\medskip
\textbf{VDB instance configuration:}
%
\begin{alignat*}{1}%\raggedleft
\odbSem [] . &: \vdbInstSet \to \confSet \to \pInstSet\\
%\end{flalign*}
%
%\begin{flalign*}%\raggedleft
\odbSem { \annot  {\setDef {\vi \vTab \numRels}}} 
&=\odbSem { \annot  { \setDef {\left( \vRelSch_1, \vRelCont_1\right), \ldots, 
\left( \vRelSch_\numRels, \vRelCont_\numRels\right)}}}\\
& = \begin{cases}
\setDef{\left( \orSem {\vRel_1 \annot [\dimMeta_1 {\wedge \dimMeta}] {\left( \vAttList_1 \right)} }, 
\otSem {\vRelSch_1} {\pushIn {\annot [\dimMeta_1 {\wedge \dimMeta}] \vRelCont_1}} \right), \ldots}, &\If \fSem \dimMeta = \t \\
\setDef {}, \Otherwise
\end{cases}
%&= \setDef {(\orSem {\vRelSch_1}, \otSem {\vRelCont_1}), \hdots, (\orSem {\vRelSch_\numRels}, \otSem {\vRelCont_\numRels} )}&
\end{alignat*}

\caption{VDB instance syntax and configuration.
%The input to all configuration functions assumes a well-formed input,
%either a v-cond (see \secref{type-sys} or a (part of a) VDB.
%\TODO{you need to define well-formedness for vdb and mention it 
%for vcond somewhere in the paper.}
%%V-cond configuration only accepts conditions that are type correct. 
%%All the configuration functions are defined over a given database
%%with v-schema \vSch. 
%A set with question mark at the end, e.g., \maybe \pRelSchSet, 
%denotes an optional type, meaning that the original set is extended
%with a non-value, \ensuremath{\bot}.
%\revised{
Note that the schema of a relation must be passed to the configuration function
for its content,
however, the variational schema does not need to be passed to configuration 
functions of smaller parts of the variational schema such as \orSem .  or \olSem .
since all needed information for configuring a part of a variational schema
is propagated. 
%Note that $\vRelContSet = \settype {(\vartype \tupletype)}$.
%}
% of variational set of attributes, v-relations, and v-schema.
%$\varepsilon$ denotes a non-existent relation and value.
%Note that the feature model and 
%relation presence condition are passed all the way to attributes due to the 
%hierarchal structure of presence conditions within a v-schema.
}
\label{fig:vdb-conf}
\end{figure} 

%
%Variation also exists in database content. To account 
%for content variability, we tag tuples with 
%presence conditions. 
%%e.g., the tuple $(1,2)^{\A}$ only exists
%%when \A\  is enabled. 
%%
%Thus, a \emph{variational tuple} (\emph{v-tuple}) is an annotated tuple,
%$\vTuple\in\vRelCont \eqq \annot[\dimMeta_\vTuple]{(\vi v \numAtts)}$. A
%variational tuple corresponds to a variational relation,
%$\vRel\annot[\dimMeta_\vRel]{(\vi \vAtt \numAtts)}$,
%where each element $v_i$ is a value corresponding to attribute $\vAtt_i$
%(recall that attributes in a variational relation are ordered).
%%
%For example, $\annot[\tFive]{(38, PL, 678)}$ is a variational tuple that belongs to the
%\ecourse\ relation from \exref{vsch} and is only present when \tFive\ is
%enabled. 
%%
%The content of a variational relation
%%  \emph{variational relation content} 
%is a set of variational tuples,
%$\vRelCont \in \vRelContSet \eqq \setDef {\vi \vTuple \numTuples}$
%and 
%%
%a \emph{variational table} (\emph{v-table}) is the pair of its relation
%schema and content, $\vTab = (\vRelSch, \vRelCont)$.
%%
%A \emph{variational database instance}
%%of VDB \vDB\ with variational schema \vSch, 
%is a set of variational tables,
%$\synDef \vdbInst  \vdbInstSet \eqq \annot [\fModel] {\setDef {\vi \vTab \numRels} }$.
%%
%A VDB instance is \emph{well-formed} if its encoded variation at
%the schema and content level are consistent and satisfiable~\cite{ALW21vamos}.
%% We define properties that must hold for a VDB to be well-formed can be 
%% found in~\cite{ALW21vamos}.
%
%
%\NOTE{The explanation in the following two paragraphs is really hard to follow.
%Some ideas for improvement: (1) State the requirement that you're talking about
%first, then explain how VDB satisfies it; currently the requirement comes at
%the end, so for 1--3 sentences the reader is wondering why you're re-hashing
%this aspect of VDB. (2) The reader has forgotten by now what the requirements
%are, so a brief (few words) description is needed for each requirement as you
%discuss it. (3) It would help to follow the same order of requirements as
%\secref{mot} as closely as possible. 
%
%\medskip
%I recommend structuring this discussion more rigidly as, ``This encoding
%satisfies \nZero, which is about foo, by doing bar. It satisfies \nOne, which
%is about blah, by doing baz.'' Obviously there's plenty of room for making it
%read more nicely than that, but the lack of structure is making it hard to
%understand as-is.}
%
%
%This encoding of variational databases satisfies the requirements for a
%variational database described in \secref{mot}. Similar to a variational schema, a user
%can configure a variational table or a VDB for a specific variant, formally defined in
%\figref{vdb-conf} in \appref{vdb-conf}. This allows users to deploy a VDB for a
%specific configuration and generate the corresponding VDB variant, satisfying
%database part of \nThree\ need.
%%
%Additionally, 
%our VDB framework puts all variants of a database into
%one VDB (satisfying \nZero) 
%and it keep tracks of which variant a tuple belongs to by 
%annotating them with presence conditions. 
%For example, consider tuples
%\ensuremath{\annot [\tFive] {(38, PL, 678)}}
%and 
%\ensuremath{\annot [\tFour] {(23, DB, \nul)}}
%that belong to the \ecourse\ table. 
%The presence conditions \tFive\ and \tFour\ state that tuples belong to temporal
%variants four and five of this VDB, respectively.
%Hence, this framework tracks which variants a tuple belongs to 
%(first part of \nTwo).
%
%
%%As shown, o
%Our VDB framework encodes variation in databases 
%at two levels: schema and content.
%% We do not extend variation to 
%%the constraint level and only focus on variation at the schema 
%%and content levels. 
%In a database that is a variational artifact as defined in \secref{req},
%while content-level variation can stand on its own, such as
%frameworks used for database versioning and 
%experimental databases~\cite{dbVersioning},
%the schema level cannot, e.g., 
%\ensuremath{
%\ecourse \left(\cno, \cname, \optAtt [\tFive] [\deptno] \right)^{\edu \wedge \left(\tFour \vee \tFive\right)}
%} encodes variation at the schema level for relation \ecourse.
%Dropping the presence conditions of tuples leads to ambiguity, i.e.,
%it is unclear which variant each of the tuples
%\ensuremath{(38, PL, 678)}
%and 
%\ensuremath{(23, DB, \nul)} belongs to. We can only guess that
%they belong to variants where \tFour\ or \tFive\ are enabled, but, 
%we do not know for sure which one. Thus, it violates the \nTwo\ 
%requirement of a variational database framework.
%%where it is unclear which variant each tuple belongs to
%%and there is no way to recover such information.
%%Note that 
%%the VDB framework encodes both schema- and content-level
%%variation. A simpler framework could be used to encode 
%%only content-level variation (where tables consist of variational tuples but
%%have plain relational schema), similar to frameworks used for 
%%database versioning and experimental databases~\cite{dbVersioning}.
%%However, schema-level variation cannot be encoded without 
%%accounting for content-level variation in a framework where
%%variants coexist in parallel and they are all put into one database,
%%e.g., while 
%%\ensuremath{
%%\ecourse \left(\cno, \cname, \optAtt [\tFive] [\deptno] \right)^{\edu \wedge \left(\tFour \vee \tFive\right)}
%%} encodes variation at the schema-level for relation \ecourse,
%%dropping presence conditions of tuples results in tuples
%%\ensuremath{(38, PL, 678)}
%%and 
%%\ensuremath{(23, DB, \nul)}
%%where it is unclear which variant each tuple belongs to
%%and there is no way to recover such information.
%
%
%Note that we limit the granularity of variation in content to tuples, that is,
%the individual values within a tuple are not variational.
%%
%%Note that the value $v_i$ is present iff 
%%$\sat {\dimMeta_\vTuple \wedge \dimMeta_\vRel \wedge \dimMeta_\vAtt \wedge \fModel}$,
%%where, 
%%$\dimMeta_\vAtt = \getPC {\getAtt i}$ and
%%%,
%%%$\dimMeta_\vTuple = \getPC \vTuple$,
%%%\dimMeta = \getPC \vRel,
%%%and 
%%%\fModel\ is the feature model.
%%%
%%for simplicity, 
%%%Also, note that to avoid overcrowding the database with variation and feature 
%%%expressions
%%we only annotate tuples and not cells. 
%This design decision causes some redundancy.
%For example, the two tuples
%\ensuremath{\annot [\tFive] {(38, PL, 678)}} and 
%\ensuremath{\annot [\neg \tFive] {(38, PL, \nul)}}
%%\ensuremath{\annot [\fOne] {(1,2)}} and \ensuremath{\annot [\neg\fOne] {(1,3)}}
%cannot be represented as a single tuple
%% \ensuremath{(1, \chc [\fOne] {2,3})} 
%with variation in the third element. However, this design decision
%does not prevent us from distinguishing between a \nul\ value
%that represents a missing value and a \nul\ value that represents
%a cell that is not present. This distinction can be made by checking
%the satisfiability of 
%the presence condition of the value $v_i$ in tuple \vTuple\ of relation \vRel\ in schema \vSch:
%If $\sat{\getPCfrom {v_i} \vTuple}$ then the \nul\ indicates a missing value
%and otherwise it indicates a non-present cell, where 
%\ensuremath{\getPCfrom {v_i} \vTuple = \dimMeta_\vTuple \wedge \getPCfrom \vRel \vSch
%\wedge \getPCfrom {\getAtt i} \vRel}.
%% \dimMeta_\vRel \wedge \dimMeta_\vAtt \wedge \fModel
%%\revised{
%%where \ensuremath{\chc [\fOne] {2,3}} is a \emph{choice} of values $2$ and $3$
%%and it states that if \A\ is enabled the cell holds the value $2$ and otherwise it 
%%holds the value $3$. }
%%
%
%%\begin{figure}

%%%%%%%%%%%%%%%%%%%%%%%%%%%%%%%%%%%%%%%%%%%%%%%%%%%
\textbf{Variational Set of Attributes Configuration:}
\begin{flalign*}
& \olSem [] . : \ \vAttSet \to \confSet \to \pAttSet&\\
%\end{flalign*}
%
%\begin{flalign*}
& \olSem {\{\optAtt\} \cup \vAttList} \spcEq  
    \begin{cases}
        \{\pAtt\} \cup \olSem{\vAttList},
                            & \If \fSem {\dimMeta \wedge \getPC{\getRel \vAtt} \wedge \fModel} \\
        \olSem{\vAttList} , & \Otherwise
     \end{cases} &\\
% & \olSem {\{\optAtt\} \cup \vAttList} &\spcEq &\  \olSem {\{\optAtt\}} \cup \olSem {\vAttList}\\
& \olSem {\setDef{}} \spcEq  \setDef{}&
\end{flalign*}

%
\medskip
\textbf{Variational Relation Schema Configuration:}
\begin{flalign*}%\raggedleft
&\orSem [] . : \vRelSchSet \to \confSet \to \pRelSchSet&\\
%\end{flalign*}
%
%\begin{flalign*}
&\orSem \vRelDef = 
	\begin{cases}
		\vRel({\olSem {\vAttList}}, &\If \fSem {\dimMeta \wedge \fModel}) \\
		\empRel, &\Otherwise
	\end{cases}&
\end{flalign*}

%
\medskip
\textbf{Variational Schema Configuration:}
\begin{flalign*}%\raggedleft
&\osSem [] . : \vSchSet \to \confSet \to \pSchSet&\\
%\end{flalign*}
%
%\begin{flalign*}
&\osSem {\annot [\fModel] {\setDef {\vRelDefNum 1, \ldots, \vRelDefNum \numRels}}}
%&\hspace{0.3cm}
= \begin{cases}
%		\setDef {\orSem {\vRelDefNumF 1}, \ldots, \orSem {\vRelDefNumF n}},
                 \setDef {\orSem {\vRel_1( \vAttList_1 )^{\dimMeta_1 \wedge \fModel} }, \ldots, 
                 \orSem {\vRel_\numRels( \vAttList_\numRels)^{\dimMeta_\numRels \wedge \fModel} }},		
        & \If \fSem \fModel \\
        \setDef{}, & \text{otherwise}
	\end{cases}&
\end{flalign*}

\medskip
\textbf{Variational Tuple Configuration:}
%
\begin{flalign*}%\raggedleft
&\ouSem [] . : \vRelCont \to \confSet \to \pRelCont&\\
%\end{flalign*}
%
%\begin{flalign*}%\raggedleft
&\ouSem {\annot [ \dimMeta_\tuple] {\left( {\vi v \numAtts}\right)}} = \left( \ovSem {v_1}, \hdots, \ovSem {v_\numAtts} \right) 
%&\\
& \textit{ where } \forall 1 \leq i \leq \numAtts: 
%&\hspace{5pt} 
\ovSem {v_i} = 
\begin{cases}
v_i, & \If \fSem {\fModel \wedge \getPC{\getRel{\getAtt{v_i}}} \wedge \getPC {\getAtt {v_i}} \wedge \dimMeta_\tuple} \\
\varepsilon, & \Otherwise
\end{cases}
\end{flalign*}

\medskip
\textbf{V-Relation Content Configuration:}
%
\begin{flalign*}%\raggedleft
&\otSem [] . : \vRelContSet \to \confSet \to \pRelContSet&\\
%\end{flalign*}
%
%\begin{flalign*}%\raggedleft
&\otSem {\setDef {\vi \tuple \numTuples}} = \setDef {\ouSem {\tuple_1}, \hdots, \ouSem {\tuple_\numTuples}}&
\end{flalign*}

\medskip
\textbf{VDB Instance Configuration:}
%
\begin{flalign*}%\raggedleft
&\odbSem [] . : \vInstSet \to \confSet \to \pInstSet&\\
%\end{flalign*}
%
%\begin{flalign*}%\raggedleft
&\odbSem { {\setDef {\vi \vTab \numRels}}} = \setDef {(\orSem {\vRelSch_1}, \otSem {\vRelCont_1}), \hdots, (\orSem {\vRelSch_\numRels}, \otSem {\vRelCont_\numRels} )}&
\end{flalign*}

\caption{
V-cond and VDB instance configurations.
% of variational set of attributes, v-relations, and v-schema.
$\varepsilon$ denotes a non-existent relation and value.
%Note that the feature model and 
%relation presence condition are passed all the way to attributes due to the 
%hierarchal structure of presence conditions within a v-schema.
}
\label{fig:vdb-conf}
\end{figure} 


\section{Variational Database}
\label{sec:vdb}

\TODO{vdb}

\section{Properties of a Variational Database Framework}
\label{sec:vdbfprop}


In this section, we describe a set of basic properties that a well-formed VDB
should satisfy.
%
These checks ensure that presence conditions are consistent and satisfiable,
which ensures that each element is present in at least one variant.
%
In the following, $\sat\dimMeta$ denotes a satisfiability check
that returns \t\ if the feature expression \dimMeta\ is satisfiable and \f\
otherwise.


A well-formed v-schema should have the following properties:
%
\begin{enumerate}
%
\item There is at least one valid configuration of the VDB feature model $\getPC \vSch$:\\
%
\centerline{
$\sat {\getPC \vSch}$}
%
\item Every relation \vRel\ is present in at least one configuration of the
variational schema \vSch:\\
%
\centerline{
$\forall\vRel\in\vSch. \sat{\getPCfrom \vRel \vSch}$}
%
\item Every attribute \vAtt\ in every relation \vRel\ is present in at least one
configuration of the variational schema \vSch:\\
%
\centerline{
$\forall\vAtt\in\vRel, \forall\vRel\in\vSch.
\sat{\getPCfrom \vRel \vSch \wedge\getPCfrom \vAtt \vRel}$}
%
\item If $\vSch_\config$ denotes the expected plain relational schema for
configuration $c$ of the variational schema \vSch, then configuring the
variational schema with that configuration, written $\sem[\config]{\vSch}$,
actually yields that variant:\\
%
\centerline{
$\forall\config\in\confSet. \osSem {\vSch} = \vSch_\config$}
%
\end{enumerate}


\noindent
%
At the data level, a well-formed VDB should have these properties:
%
\begin{enumerate}
%
\item Every tuple \vTuple\ in relation \vRel\ is present in at least one variant:\\
%
\centerline{
$\forall\vTuple\in\vRel, \forall\vRel\in\vSch.
\sat{\getPCfrom\vRel \vSch\wedge\getPCfrom \vTuple \vRel}$ }
%
\item For every tuple \vTuple\ in relation \vRel, if an attribute \vAtt\ in \vRel\ is
not present in any variants of the tuple, then the value of that attribute in
the tuple, written $\mathit{value}_\vTuple(\vAtt)$, should be NULL:\\
%\centerline{
$\forall\vTuple\in\vRel, \forall\vAtt\in\vRel,\forall \vRel\in\vSch.
\neg\sat{\getPCfrom \vRel \vSch \wedge\getPCfrom \vAtt \vRel\wedge\getPCfrom \vTuple \vRel}
\Rightarrow \mathit{value}_\vTuple(\vAtt) = \nul$
%
\end{enumerate}


\noindent
%
%
 Since a single VDB can supply data for many different database variants at the
 same time, encoding variation explicitly in a database allows the developers
 to check for different properties over all database variants.
%
Thus, depending on the context of the VDB, more specialized properties can be checked.
For example, if temporal variability in a database is accumulated over
variants (i.e.\ old data is included in more recent variants in addition to
newly added data), it is desirable to ensure that older variants are subsets of
newer variants.
%
This property should hold for our employee dataset, introduced in 
\secref{emp-vdb}. To check this, 
assume that configurations \ensuremath{\config_1, \config_2, \cdots}
represent time-ordered configurations, then check
\ensuremath{
\forall \config_i, \config_j \in \confSet, i \le j, \odbSem [\config_i] {\vdbInst} \subseteq \odbSem [\config_j]{\vdbInst}
}, 
where \ensuremath{\odbSem[\config]{\vdbInst}} denotes configuring the VDB instance
\vdbInst\ for configuration \config, defined in \figref{vdb-conf}.
%\parisa{note to myself, impl todo: actually check this for employee db when you got the time!}

%v-table checks:
%- \ensuremath{\forall tuple \in relation \in schema : sat (fm \wedge pc_relation \wedge pc_tuple)}\\
%- \ensuremath{\forall attribute \in relation \in schema, \forall val : if unsat (fm \wedge pc_relation \wedge pc_attribute \wedge pc_tuple)} then value must be null\\



%encode
%vset
%vsch
%vtab
%vdb
%vdbprop
\chapter{Variational Queries}
\label{ch:vql}

%change all $\vFour \vee \vFive$ to $\neg \vThree$ in all sections and
%update the derivation tree examples to the final type system. }

Now that we have introduced the variational database framework 
we need a query language to extract information from a VDB instance.
Our approach will build on existing relational query languages (like SQL and relational algebra)
but must also account for the new aspect of our database: variation. 


%\point{queries need to be able to express variability encoded in vdb.}
%The variational nature of a VDB requires a query language that
%accounts for variation directly.
%To express and represent variation in queries,
%we incorporate choice calculus~\cite{Walk13thesis, EW11tosem}  into a 
%structured query language. 
We formally define 
\emph{variational relational algebra} (VRA) in \secref{vrel-alg}
as our algebraic query language.
A query written in VRA is called a \emph{variational query} (\emph{v-query});
we use query and variational query interchangeably when it is clear from context. 
Unlike relational queries that convey an intent over a single database, 
a variational query typically conveys the same intent over several 
relational database variants. However, a single variational query is also capable of capturing different 
intents over different database variants.
%Consequently, the expressiveness of variational queries may cause them to be 
%more complicated than relational queries, discussed in \secref{type-sys}. 
%Hence, 



%
Due to the expressiveness of variational queries, 
we define a type system for VRA that statically checks a
variational query against the underlying variational schema in \secref{type-sys}.
%
To make variational queries more useable we relieve the user from repeating 
the variational schema's variation in their variational queries. This is achieved by 
explicitly annotating queries in \secref{constrain}.
%In \secref{constrain}, we define an operation that explicitly annotates a
%variational query with information contained in a v-schema. 
%This operation is useful to
%define the \emph{variation-preservation} property for VRA and its type system,
%which is discussed in \secref{var-pres}, and demonstrates how our framework
%satisfies the information need \nTwo.

To understand the meaning of variational queries
we define the semantics of variational queries via the
semantics of relational queries in \secref{vrasem}. We define
how to configure a variational query to a relational query
in \secref{vraconf}. Then, we use the results of multiple relational
queries to accumulate the result of the original variational query 
in \secref{accum}.
%\maybeAdd{add direct sem of VRA if time allows and equiv in vql prop.}


%
We also provide 
%we close out this section by providing 
a set of syntactic rules that are semantic-preserving 
in \secref{var-min}. These rules enable factoring and distributing
variation points within a variational query, which enables syntactic refactoring
including maximizing sharing within a variational query.
%for reducing a query's variation.
%
Finally, in \secref{vqlprop}, we present some properties of the VRA including
the expressiveness and type safety of VRA in \secref{express} and \secref{var-pres}, respectively,
in addition to the \emph{variation-preservation} property of VRA at the
semantics level in \secref{var-pres-sem}.
%also define the \emph{variation-preservation} property for VRA at
%the type level in \secref{var-pres}.


\section{Variational Relational Algebra}
\label{sec:vrel-alg}

%
%\wrrite{you want to have an example here that doesn't need to be explicitly annotated. 
%then use the same example to illustrate the semantics via ra sem.}
%\TODO{we use ... that has these differences with blah
%%
%we introduce these differences through building up a query to extract info
%required by the variational intent. }

%\point{vra = cc + ra}
%Considering the variational nature of a VDB, to satisfy a user's information 
%need when extracting information, 
%we need a query language that not only considers the structure of 
%relational databases (such as SQL and relational algebra (RA)) but also 
%accounts for the variation encoded in the VDB. We achieve this by:
%1) picking relational algebra as our main query language and
%2) using \emph{choices}~\cite{Walk13thesis, EW11tosem} 
%and presence conditions to account for variation. 

To account for variation, VRA combines relational algebra (RA) with 
\emph{choices}~\cite{EW11tosem,HW16fosd,Walk13thesis}.
%\point{choice.}
Remember that a choice $\chc{\elem_1,\elem_2}$ consists of a feature expression \dimMeta, called
the \emph{dimension} of the choice, and 
two \emph{alternatives} $\elem_1$ and $\elem_2$. For a given configuration \config, 
the choice $\chc{\elem_1, \elem_2}$ can be replaced by $\elem_1$ if \dimMeta\
evaluates to \t\ under configuration \config, (i.e., \fSem{\dimMeta}),
or $\elem_2$ otherwise. 
% Choices allow a variational queries
% to encode variation in a structured and systematic manner. 

\begin{figure}
\begin{syntax}

\multicolumn{4}{l}{\textbf{Operators:}} \\[1ex]
\bullet
  &\eqq& \multicolumn{2}{l}{< \myOR \leq \myOR = \myOR \neq \myOR > \myOR \geq} \\
\circ
  &\eqq& \cup \myOR \cap \\[2ex]

\multicolumn{4}{l}{\textbf{Variational conditions:}} \\[1ex]
\vCond\in\vCondSet
  &\eqq&  \multicolumn{2}{l}{
          \bTag
   \myOR  \pAtt \bullet \cte
   \myOR  \pAtt \bullet \pAtt
   \myOR  \neg \vCond
   \myOR  \vCond \vee \vCond} \\
  &\myOR& \multicolumn{2}{l}{
          \vCond \wedge \vCond
   \myOR \chc{\vCond,\vCond}} \\[2ex]

\multicolumn{4}{l}{\textbf{Variational queries:}} \\[1ex]
\vQ\in\qSet
  &\eqq&  \vRel     & \textit{Relation}\\
  &\myOR& \vSel \vQ & \textit{Selection}\\
  &\myOR& \vPrj[\vAttList]{\vQ} & \textit{Projection}\\
  &\myOR& \chc{\vQ,\vQ} & \textit{Choice}\\
% &\myOR& \vQ \Join_\vCond \vQ & \textit{Variational Join}\\
  &\myOR& \vQ \times \vQ & \textit{Cartesian Product}\\
  &\myOR& \vQ \circ \vQ  & \textit{Set Operation}\\
% &\myOR& \vQ \backslash \vQ &\textit{Variational Set Difference}\\
  &\myOR& \empRel & \textit{Empty Relation}

\end{syntax}

\caption[Syntax of variational relational algebra]{Syntax of variational relational algebra.}
%\TODO{remember that
%you removed join (also removed it from query config def and constrain query 
%by schema). if you want use it just say it's a syntactic sugar.}
%$<, \leq, =, \neq, >, \geq$.
%$\circ$ denotes set operators: union and difference.}
\label{fig:v-alg-def}
\end{figure}



%\point{explain notation and VRA operations.}
The syntax of VRA is given in \figref{v-alg-def}.
%
The selection operation is similar to standard RA selection except
that the condition parameter is \emph{variational} meaning that it may contain
choices.
For example, the query 
\ensuremath{\sigma_{\chc {\vAtt_1=\vAtt_2,\vAtt_1=\vAtt_3}} (\vRel)}
selects a variational tuple \vTuple\ if it satisfies
the condition \ensuremath{\vAtt_1 = \vAtt_2} 
and  \ensuremath{\sat {\dimMeta \wedge \getPC \vTuple}}
or
if \ensuremath{\vAtt_1 = \vAtt_3} 
and \ensuremath{\sat {\neg \dimMeta \wedge \getPC \vTuple }}.
%
The projection operation is parameterized by a variational set of attributes, \vAttList. For
example,
the query $\pi_{\vAtt_1, \optAtt [\dimMeta] [\vAtt_2]} (\vRel)$
projects $\vAtt_1$ from relation \vRel\ unconditionally, and $\vAtt_2$ 
when \sat{\dimMeta}.
%
The choice operation enables combining two variational queries to be used in different
variants based on the dimension. In practice,
it is often useful to return information in some variants and nothing at all in
others. We introduce an explicit \emph{empty} query \empRel\ to facilitate
this. 
Similar to our definition of the empty query for relational algebra, for VRA we
also have: $\empRel=\vPrj[\set{}]{\vQ}$.
The empty query is used, for example, in 
\ensuremath{\vQ_2} in \exref{vq-specific}. 
%The set operations between queries are v-set operations defined in \secref{vset}.
The rest of VRA's operations are similar to RA, where all set operations
(union, intersection, and product) are changed to the corresponding
variational set operations defined in \secref{vset}.
%\secref{vlist-vset}.
%
%\remember{
%In examples, we also use a join operation with a variational condition,
%$\vQ_1\bowtie_\vCond\vQ_2$, which is syntactic sugar for
%$\sigma_\vCond(\vQ_1\times\vQ_2)$.}


Our implementation of VRA also provides mechanisms for renaming queries and
qualifying attributes with relation/sub\-query names. These features are needed
to support self joins and to project attributes with the same name in different
relations. However, for simplicity, we omit these features from the formal
definition in this thesis.


%A query can simply 
%refer to a relation, filter tuples based on a variational condition 
%(which is a relational condition with choices of two conditions), and
%project a variational list of attributes. Besides production of two queries and
%set operations, VRA allows for a choice of two variational queries. This demands an
%\emph{empty} query since an alternative of a choice can very well inquire 
%no information at all. 
%For example, the query $\chc {\vQ_1, \}$
%

%\subsubsection{Running a Variational query on a VDB Results in a Variational table}
%\label{sec:run-vq-get-vtab}
%A variational query systematically represents a set of relational query variants associated to their
%corresponding database variants. Hence, intuitively the user expects to 
%get such variation in their result as well. 

The result of a variational query is a variational table with the reserved relation name $\mathit{result}$.
%
For example, assume that variational tuples $\annot[\fOne]{(1,2)}$ and $\annot[\neg
f_3]{(3,4)}$ belong to a variational relation $\vRel(\vAtt_1,\vAtt_2)$, which is the only
relation in a VDB with the trivial feature model \t.
%
The query $\chc[f_3]{\pi_{\optAtt[f_2][\vAtt_1]}(\vRel),\empRel}$ returns a
variational table with relation schema $\annot[f_3]{\mathit{result}(\annot[f_2]{a_1})}$,
which indicates that the result is only non-empty when $f_3$ is \t\ and that the
result includes attribute $a_1$ when $f_2$ is \t. 
%\secref{type-sys} defines a
%type system that yields the relation schema for any well-formed query.
%
The content of the result relation for the example query is a single variational tuple
$\annot[f_1]{(1)}$. The tuple $\annot[\neg f_3]{(3)}$ is not included since the
projection occurs in the context of a choice in $f_3$, which is incompatible
with the presence condition of the tuple, i.e., $\unsat{f_3 \wedge\neg f_3}$.
This illustrates how choices can effectively filter the tuples in a VDB based
on the dimension.
%, satisfying the second part of \nOne.
%
% Although there is no need to update the presence condition of the returned
% tuples, yet choices can filter the returned variational tuples.
%
% Note that here the value \ensuremath{1}
% of attribute \ensuremath{\vAtt_1} is present in VDB variants where 
% \ensuremath{\sat {\A \wedge \B \wedge \C}} although the presence 
% condition of the returned variational tuple does not have to state this condition
% since 
%
% overall presence 
%condition and the presence conditions of attributes and tuples are
%restricted by the variation enforced by the query.
%
%Note that the presence condition of tuples, attributes, and the return relation
%is restricted by the variation enforced by the query. 
% correct this so that you don't conjunct the pc of relation and clarify that it's relations's pc and not the attributes. although the conjunction should be satisfiable.}
%
%
%%Hence, VRA is more expressive than RA 
%%because it can encode variational queries.
%The variational nature allows users to write interesting queries in many ways:
%1) to express their variational information need or to filter returned tuples
%they can use annotations or 
%choices, \exref{vq-specific},
%2) to express the same intent over several database variants they can 
%use choices in queries or conditions, \exref{vq-same-intent-mult-vars},
%and 
%3) they can also use choices to express different intents over database variants.
%\TODO{Eric, should we drop the last since it creates messy results and isn't really useful?}.
%%The expressiveness of VRA satisfies \textbf{N1}, this is illustrated in 
%%\exref{vq-specific} and \exref{vq-same-intent-mult-vars}.
%%Interestingly, VRA's expressiveness enables users to express 
%%their information need more specifically by stating the exact condition
%%under which an information need is inquired. \exref{vq-specific} illustrates this.
%%It also allows users to express the same intent over several database 
%%variants
% \NOTE{
% To express the variational information need or to filter returned tuples
% users can use annotations or choices. \exref{vq-specific} illustrates this.
% }
%
%The following example
\exref{vq-specific} illustrates
%, in the context of our running example, 
how
a variational query can be used to express variational information needs.

\begin{example}
\label{eg:vq-specific}
%VRA's expressiveness consequently facilitates expressing exactly the condition
%under which an information need is inquired. 
%\wrrite{build up this example. and show the result tables.}
Assume a VDB with
\ensuremath{\features = \setDef {\vThree, \vFour, \vFive}}, 
and the only variational table \empbio\ shown in \tabref{empbio-vtab}.
The VDB has the feature model $\dimMeta_2 = \oneof {\vThree, \vFour, \vFive}$
which states that the three \vThree--\vFive\ are mutually exclusive. 
Note that $\dimMeta_2$ is different from the feature model 
$\dimMeta_{\mathit{mot}}$ of the \empbio\ variational table
shown in \tabref{empbio-vtab} .
%the corresponding \empbio\ schema variants in \tabref{mot}. 
The variational schema for this VDB is:\\
%
\centerline{\ensuremath{
\vSch_2 =
\{\empbio (\empno, \sex, \birthdate,
\optAtt [\vFour] [\name], \optAtt [\vFive] [\fname],
 \optAtt [\vFive] [\lname] )\}^{\dimMeta_2}
% \\
%& \hspace{-38pt} \textit{where } \dimMeta_2 = \oneof {\vThree, \vFour, \vFive}
%{\vThree \oplus \vFour \oplus \vFive}.
%\left(\vThree \wedge \neg \vFour \wedge \neg \vFive\right)
%  \vee \left(\vFour \wedge \neg \vThree \wedge \neg \vFive\right) 
%   \vee \left(\vFive \wedge \neg \vThree \wedge \neg \vFour\right)}.
%\end{align*}
}}.
%
Now, the user wants the employee ID numbers (\empno) and their names for variants 
that enable either \vFour\ or \vFive\ but not \vThree.
%\set{\vFour} and \set{\vFive}.
We show the steps to build up multiple queries that can extract this information. 
First, to extract the required attributes we write the query $\vQ_0$ to project all the needed
attributes without considering the variational aspect of projection. \\
\centerline{\ensuremath{
\vQ_0 = \pi_{\empno, \name, \fname, \lname} (\empbio)
}}
Note that the presence condition attribute (\pcatt) does not need to be projected. In fact, 
the presence condition attribute is returned for every variational query since that is the only
way to keep track of variation at the content level. 
%
\tabref{vq0-res} shows
the result of query $\vQ_0$ over the described VDB.
%
Note that the presence condition of the result is $\getPCfrom \empbio {\vSch_2} = \oneof {\vThree, \vFour, \vFive} \wedge (\vThree \vee \vFour \vee \vFive)$ which can be simplified to
$\oneof {\vThree, \vFour, \vFive}$. We discuss how the 
presence conditions of the returned result and its attributes are generated in \secref{type-sys}.
%

\begin{table}[ht!]
%\caption[Results of some variational queries]{Results of some variational queries over the VDB instance described in \exref{vq-specific}.}
%\label{tab:vq-res}
%\centering
%\begin{subtable}[t]{\textwidth}
\centering
\caption[Result of a variational query]{Result of the v-query $\vQ_0 = \pi_{\empno, \name, \fname, \lname} (\empbio)$.}
\label{tab:vq0-res}
\footnotesize
\arrayrulecolor{blue}
%!{\color{black}\vrule}
\begin{tabular} {c !{\color{black}\vrule} l l l l : l }
 {\textcolor{blue}{$\oneof {\vThree, \vFour,\vFive}$} }& {\textcolor{blue}{\texttt{true}}}&  {\textcolor{blue}{$\vFour$}} &  {\textcolor{blue}{$\vFive$}} &  {\textcolor{blue}{$\vFive $}} & {\textcolor{blue}{\texttt{true}}}\\
\arrayrulecolor{blue}\hdashline
\multirow{2}{*}{$\mathit{result}$}  & \empno & \name & \fname & \lname & \pcatt \\
\arrayrulecolor{black}\cline{2-6}
& 12001 & Ulf Hofstetter & Ulf & Hofstetter  & $\textcolor{blue}{\vThree \vee \vFour \vee \vFive}$\\
& 12002 & Luise McFarlan & Luise & McFarlan  & $\textcolor{blue}{\vThree \vee \vFour \vee \vFive}$\\
& 12003 & Shir DuCasse & Shir & DuCasse  & $\textcolor{blue}{\vThree \vee \vFour \vee \vFive}$\\
 &80001  & Nagui Merli & Nagui & Merli & $\textcolor{blue}{ \vFour \vee \vFive}$\\
 & 80002 & Mayuko Meszaros & Mayuko & Meszaros & $\textcolor{blue}{ \vFour \vee \vFive}$\\
 & 80003 & Theirry Viele & Theirry & Viele & $\textcolor{blue}{ \vFour \vee \vFive}$\\
 & 200001  & Selwyn Koshiba & Selwyn & Koshiba & \textcolor{blue}{\vFive}\\
 & 200002  & Bedrich Markovitch & Bedrich & Markovitch & \textcolor{blue}{\vFive}\\
 & 200003  & Pascal Benzmuller & Pascal & Benzmuller  & \textcolor{blue}{\vFive}\\
 & \ldots  & \ldots & \ldots & \ldots & \textcolor{blue}{\ldots} \\
\arrayrulecolor{white}\hline
\end{tabular}
%\end{subtable}
%
%\medskip
%\medskip
%\medskip
%\begin{subtable}[t]{\textwidth}
%\centering
%\caption{Result of the variational queries $\vQ_1 = \pi_{\optAtt [\vFour \vee \vFive] [\empno], \name, \fname, \lname} (\empbio)$ and 
%$\VVal {\vQ_1} = \pi_{\optAtt [(\vFour \vee \vFive) \wedge \neg \vThree] [\empno], 
%\optAtt [\vFour \wedge \neg \vThree \wedge \neg \vFive] [\name], 
%\optAtt [\vFive \wedge \neg \vThree \wedge \neg \vFour] [\fname], 
%\optAtt [\vFive \wedge \neg \vThree \wedge \neg \vFour] [\lname]} (\empbio)
%$.}
%\label{tab:vq1-res}
%\footnotesize
%\arrayrulecolor{blue}
%%!{\color{black}\vrule}
%\begin{tabular} {c !{\color{black}\vrule} l l l l : l }
% {\textcolor{blue}{$\oneof {\vThree, \vFour,\vFive}$} }& {\textcolor{blue}{$\vFour \vee \vFive$}}&  {\textcolor{blue}{$\vFour $}} &  {\textcolor{blue}{$\vFive $}} &  {\textcolor{blue}{$\vFive$}} & {\textcolor{blue}{\texttt{true}}}\\
%\arrayrulecolor{blue}\hdashline
%\multirow{2}{*}{$\mathit{result}$}  & \empno & \name & \fname & \lname & \pcatt \\
%\arrayrulecolor{black}\cline{2-6}
%%& 12001 & & & & \textcolor{blue}{\vThree}\\
%%& 12002 & & & & \textcolor{blue}{\vThree}\\
%%& 12003 & & & & \textcolor{blue}{\vThree}\\
% &80001  & Nagui Merli & & & \textcolor{blue}{\vFour}\\
% & 80002 & Mayuko Meszaros & & & \textcolor{blue}{\vFour}\\
% & 80003 & Theirry Viele & & & \textcolor{blue}{\vFour}\\
% & 200001  & & Selwyn & Koshiba & \textcolor{blue}{\vFive}\\
% & 200002  & & Bedrich & Markovitch & \textcolor{blue}{\vFive}\\
% & 200003  & & Pascal & Benzmuller  & \textcolor{blue}{\vFive}\\
% & \ldots  & \ldots & \ldots & \ldots& \textcolor{blue}{\ldots} \\
%\arrayrulecolor{white}\hline
%\end{tabular}
%\end{subtable}
%
%\medskip
%\medskip
%\medskip
%\begin{subtable}[t]{\textwidth}
%\centering
%\caption{Result of the variational query $\vQ_2 = \chc[\neg \vThree]{\pi_{\empno,\name,\fname,\lname}(\empbio),\empRel}$.}
%\label{tab:vq2-res}
%\footnotesize
%\arrayrulecolor{blue}
%%!{\color{black}\vrule}
%\begin{tabular} {c !{\color{black}\vrule} l l l l : l }
% {\textcolor{blue}{$\oneof {\vThree, \vFour,\vFive} \wedge \neg \vThree$} }& {\textcolor{blue}{\t}}&  {\textcolor{blue}{$\vFour $}} &  {\textcolor{blue}{$\vFive $}} &  {\textcolor{blue}{$\vFive$}} & {\textcolor{blue}{\texttt{true}}}\\
%\arrayrulecolor{blue}\hdashline
%\multirow{2}{*}{$\mathit{result}$}  & \empno & \name & \fname & \lname & \pcatt \\
%\arrayrulecolor{black}\cline{2-6}
%%& 12001 & & & & \textcolor{blue}{\vThree}\\
%%& 12002 & & & & \textcolor{blue}{\vThree}\\
%%& 12003 & & & & \textcolor{blue}{\vThree}\\
% &80001  & Nagui Merli & & & \textcolor{blue}{\vFour}\\
% & 80002 & Mayuko Meszaros & & & \textcolor{blue}{\vFour}\\
% & 80003 & Theirry Viele & & & \textcolor{blue}{\vFour}\\
% & 200001  & & Selwyn & Koshiba & \textcolor{blue}{\vFive}\\
% & 200002  & & Bedrich & Markovitch & \textcolor{blue}{\vFive}\\
% & 200003  & & Pascal & Benzmuller  & \textcolor{blue}{\vFive}\\
% & \ldots  & \ldots & \ldots & \ldots& \textcolor{blue}{\ldots} \\
%\arrayrulecolor{white}\hline
%\end{tabular}
%\end{subtable}
%
\end{table}


Now we pay attention to the variational aspect of the query. Knowing that the variation encoded
in the VDB can be inferred (that is, the VDB exists if and only if exactly
 one of the features \vThree--\vFive\ is enabled, the \name\ attribute only exists for variants
that enable \vFour\ and the \fname\ and \lname\ attributes only exist for variants that
enable \vFive) and since we only want the
projected attributes for variants that enable \vFour\ or \vFive\ we can write the
query $\vQ_1$.\\
%
\centerline{\ensuremath{
\vQ_1 = \pi_{\optAtt [\vFour \vee \vFive] [\empno], \name, \fname, \lname} (\empbio)
}}
%
\tabref{vq1-res} shows the result of this query over the described VDB.
Note that the first three tuples from \tabref{vq0-res} are not returned since the query
does not project the \empno\
attribute for variants that enable \vThree\ and  attributes 
\name, \fname, and \lname\ do not exist for these variants in the VDB. 
Thus, the tuple will just be empty and so is dropped. 
%The user needs to project the \name\ attribute 
%for variant \set{\vFour}, the \fname\ and \lname\ attributes for variant
%\set{\vFive}, and \empno\ attribute for both variants.
%This can be expressed with the following variational query.
%If we did not know that the database enforces the variation encoded in itself
%we had to repeat that variation. 

\begin{table}[ht!]
%\caption[Results of some variational queries]{Results of some variational queries over the VDB instance described in \exref{vq-specific}.}
%\label{tab:vq-res}
%\centering
%\begin{subtable}[t]{\textwidth}
%\centering
%\caption{Result of the variational query $\vQ_0 = \pi_{\empno, \name, \fname, \lname} (\empbio)$.}
%\label{tab:vq0-res}
%\footnotesize
%\arrayrulecolor{blue}
%%!{\color{black}\vrule}
%\begin{tabular} {c !{\color{black}\vrule} l l l l : l }
% {\textcolor{blue}{$\oneof {\vThree, \vFour,\vFive}$} }& {\textcolor{blue}{\texttt{true}}}&  {\textcolor{blue}{$\vFour$}} &  {\textcolor{blue}{$\vFive$}} &  {\textcolor{blue}{$\vFive $}} & {\textcolor{blue}{\texttt{true}}}\\
%\arrayrulecolor{blue}\hdashline
%\multirow{2}{*}{$\mathit{result}$}  & \empno & \name & \fname & \lname & \pcatt \\
%\arrayrulecolor{black}\cline{2-6}
%& 12001 & & & & \textcolor{blue}{\vThree}\\
%& 12002 & & & & \textcolor{blue}{\vThree}\\
%& 12003 & & & & \textcolor{blue}{\vThree}\\
% &80001  & Nagui Merli & & & \textcolor{blue}{\vFour}\\
% & 80002 & Mayuko Meszaros & & & \textcolor{blue}{\vFour}\\
% & 80003 & Theirry Viele & & & \textcolor{blue}{\vFour}\\
% & 200001  & & Selwyn & Koshiba & \textcolor{blue}{\vFive}\\
% & 200002  & & Bedrich & Markovitch & \textcolor{blue}{\vFive}\\
% & 200003  & & Pascal & Benzmuller  & \textcolor{blue}{\vFive}\\
% & \ldots  & \ldots & \ldots & \ldots & \textcolor{blue}{\ldots} \\
%\arrayrulecolor{white}\hline
%\end{tabular}
%\end{subtable}
%
%\medskip
%\medskip
%\medskip
%\begin{subtable}[t]{\textwidth}
\centering
\caption[Result of a variational query]{Result of the v-queries $\vQ_1 = \pi_{\optAtt [\vFour \vee \vFive] [\empno], \name, \fname, \lname} (\empbio)$ and 
$\VVal {\vQ_1} = \pi_{\optAtt [(\vFour \vee \vFive) \wedge \neg \vThree] [\empno], 
\optAtt [\vFour \wedge \neg \vThree \wedge \neg \vFive] [\name], 
\optAtt [\vFive \wedge \neg \vThree \wedge \neg \vFour] [\fname], 
\optAtt [\vFive \wedge \neg \vThree \wedge \neg \vFour] [\lname]} (\empbio)
$.}
\label{tab:vq1-res}
\footnotesize
\arrayrulecolor{blue}
%!{\color{black}\vrule}
\begin{tabular} {c !{\color{black}\vrule} l l l l : l }
 {\textcolor{blue}{$\oneof {\vThree, \vFour,\vFive}$} }& {\textcolor{blue}{$\vFour \vee \vFive$}}&  {\textcolor{blue}{$\vFour $}} &  {\textcolor{blue}{$\vFive $}} &  {\textcolor{blue}{$\vFive$}} & {\textcolor{blue}{\texttt{true}}}\\
\arrayrulecolor{blue}\hdashline
\multirow{2}{*}{$\mathit{result}$}  & \empno & \name & \fname & \lname & \pcatt \\
\arrayrulecolor{black}\cline{2-6}
%%& 12001 & & & & \textcolor{blue}{\vThree}\\
%%& 12002 & & & & \textcolor{blue}{\vThree}\\
%%& 12003 & & & & \textcolor{blue}{\vThree}\\
& 12001 & Ulf Hofstetter & Ulf & Hofstetter  & $\textcolor{blue}{\vThree \vee \vFour \vee \vFive}$\\
& 12002 & Luise McFarlan & Luise & McFarlan  & $\textcolor{blue}{\vThree \vee \vFour \vee \vFive}$\\
& 12003 & Shir DuCasse & Shir & DuCasse  & $\textcolor{blue}{\vThree \vee \vFour \vee \vFive}$\\
 &80001  & Nagui Merli & Nagui & Merli & $\textcolor{blue}{\vFour \vee \vFive}$\\
 & 80002 & Mayuko Meszaros & Mayuko & Meszaros & $\textcolor{blue}{ \vFour \vee \vFive}$\\
 & 80003 & Theirry Viele & Theirry & Viele & $\textcolor{blue}{ \vFour \vee \vFive}$\\
 & 200001  & Selwyn Koshiba & Selwyn & Koshiba & \textcolor{blue}{\vFive}\\
 & 200002  & Bedrich Markovitch & Bedrich & Markovitch & \textcolor{blue}{\vFive}\\
 & 200003  & Pascal Benzmuller & Pascal & Benzmuller  & \textcolor{blue}{\vFive}\\
 & \ldots  & \ldots & \ldots & \ldots & \textcolor{blue}{\ldots} \\
% &80001  & Nagui Merli & & & \textcolor{blue}{\vFour}\\
% & 80002 & Mayuko Meszaros & & & \textcolor{blue}{\vFour}\\
% & 80003 & Theirry Viele & & & \textcolor{blue}{\vFour}\\
% & 200001  & & Selwyn & Koshiba & \textcolor{blue}{\vFive}\\
% & 200002  & & Bedrich & Markovitch & \textcolor{blue}{\vFive}\\
% & 200003  & & Pascal & Benzmuller  & \textcolor{blue}{\vFive}\\
% & \ldots  & \ldots & \ldots & \ldots& \textcolor{blue}{\ldots} \\
\arrayrulecolor{white}\hline
\end{tabular}
%\end{subtable}
%
%\medskip
%\medskip
%\medskip
%\begin{subtable}[t]{\textwidth}
%\centering
%\caption{Result of the variational query $\vQ_2 = \chc[\neg \vThree]{\pi_{\empno,\name,\fname,\lname}(\empbio),\empRel}$.}
%\label{tab:vq2-res}
%\footnotesize
%\arrayrulecolor{blue}
%%!{\color{black}\vrule}
%\begin{tabular} {c !{\color{black}\vrule} l l l l : l }
% {\textcolor{blue}{$\oneof {\vThree, \vFour,\vFive} \wedge \neg \vThree$} }& {\textcolor{blue}{\t}}&  {\textcolor{blue}{$\vFour $}} &  {\textcolor{blue}{$\vFive $}} &  {\textcolor{blue}{$\vFive$}} & {\textcolor{blue}{\texttt{true}}}\\
%\arrayrulecolor{blue}\hdashline
%\multirow{2}{*}{$\mathit{result}$}  & \empno & \name & \fname & \lname & \pcatt \\
%\arrayrulecolor{black}\cline{2-6}
%%& 12001 & & & & \textcolor{blue}{\vThree}\\
%%& 12002 & & & & \textcolor{blue}{\vThree}\\
%%& 12003 & & & & \textcolor{blue}{\vThree}\\
% &80001  & Nagui Merli & & & \textcolor{blue}{\vFour}\\
% & 80002 & Mayuko Meszaros & & & \textcolor{blue}{\vFour}\\
% & 80003 & Theirry Viele & & & \textcolor{blue}{\vFour}\\
% & 200001  & & Selwyn & Koshiba & \textcolor{blue}{\vFive}\\
% & 200002  & & Bedrich & Markovitch & \textcolor{blue}{\vFive}\\
% & 200003  & & Pascal & Benzmuller  & \textcolor{blue}{\vFive}\\
% & \ldots  & \ldots & \ldots & \ldots& \textcolor{blue}{\ldots} \\
%\arrayrulecolor{white}\hline
%\end{tabular}
%\end{subtable}
%
\end{table}


If desired, we can also make the inferred presence conditions explicit, as 
demonstrated in the following query $\VVal {\vQ_1}$.\\
%This is expressed in $\VVal {\vQ_1}$. 
\centerline{\ensuremath{
\VVal {\vQ_1} = 
\pi_{\optAtt [(\vFour \vee \vFive) \wedge \neg \vThree] [\empno], 
\optAtt [\vFour \wedge \neg \vThree \wedge \neg \vFive] [\name], 
\optAtt [\vFive \wedge \neg \vThree \wedge \neg \vFour] [\fname], 
\optAtt [\vFive \wedge \neg \vThree \wedge \neg \vFour] [\lname]} (\empbio)
}}
%
The result of the query $\VVal {\vQ_1}$ is still \tabref{vq1-res}.
%\eric{
%%isnt this a moot point? since the feature model has already been applied, the result is unchanged by whether the feature model restricts it or not? 
%Yeah, I guess. but that was the whole point of saying that fm has been applied.}
Note that all the variation encoded in the VDB is applied to the result of
a query. Thus,
the result of a variational query stands on its own, that is,
it is not part of a bigger structure like the variational tables in a VDB.
%Note that unlike the variational table \empbio\ shown in \tabref{empbio-vtab},
%%which is restricted not only by its presence condition but also by the feature model,
%the result of a variational query is not part of a bigger variational structure (the VDB).
%Thus, it is only restricted by its presence condition and not the feature model although
%the feature model has already been applied to its presence condition.
\end{example}


In the example, note that the user does not need to repeat the variability  encoded
in the variational schema in their query, that is, they do not need to annotate \name,
\fname, and \lname\ with \vFour, \vFive, and \vFive, respectively. We discuss
this in more detail in \secref{constrain}. $\vQ_1$
queries all three variants simultaneously although the returned results are
only associated with variants \vFour\ and \vFive\ due to the annotation of the
attribute \empno\ in the query and the presence conditions of the rest of the
projected attributes in the schema.
%
Yet, the query can be further simplified with a choice. $\vQ_2$ selects only two
out of the three variants explicitly:\\
%selecting only two out of the three variants can be written more
%explicitly in a query by using a choice:
\centerline{\ensuremath{
\vQ_2=\chc[\neg \vThree]{\pi_{\empno,\name,\fname,\lname}(\empbio),\empRel}}}. 
%
\tabref{vq2-res} shows the result of this query over the VDB described in \exref{vq-specific}.
%

\begin{table}[ht!]
%\caption[Results of some variational queries]{Results of some variational queries over the VDB instance described in \exref{vq-specific}.}
%\label{tab:vq-res}
%\centering
%\begin{subtable}[t]{\textwidth}
%\centering
%\caption{Result of the variational query $\vQ_0 = \pi_{\empno, \name, \fname, \lname} (\empbio)$.}
%\label{tab:vq0-res}
%\footnotesize
%\arrayrulecolor{blue}
%%!{\color{black}\vrule}
%\begin{tabular} {c !{\color{black}\vrule} l l l l : l }
% {\textcolor{blue}{$\oneof {\vThree, \vFour,\vFive}$} }& {\textcolor{blue}{\texttt{true}}}&  {\textcolor{blue}{$\vFour$}} &  {\textcolor{blue}{$\vFive$}} &  {\textcolor{blue}{$\vFive $}} & {\textcolor{blue}{\texttt{true}}}\\
%\arrayrulecolor{blue}\hdashline
%\multirow{2}{*}{$\mathit{result}$}  & \empno & \name & \fname & \lname & \pcatt \\
%\arrayrulecolor{black}\cline{2-6}
%& 12001 & & & & \textcolor{blue}{\vThree}\\
%& 12002 & & & & \textcolor{blue}{\vThree}\\
%& 12003 & & & & \textcolor{blue}{\vThree}\\
% &80001  & Nagui Merli & & & \textcolor{blue}{\vFour}\\
% & 80002 & Mayuko Meszaros & & & \textcolor{blue}{\vFour}\\
% & 80003 & Theirry Viele & & & \textcolor{blue}{\vFour}\\
% & 200001  & & Selwyn & Koshiba & \textcolor{blue}{\vFive}\\
% & 200002  & & Bedrich & Markovitch & \textcolor{blue}{\vFive}\\
% & 200003  & & Pascal & Benzmuller  & \textcolor{blue}{\vFive}\\
% & \ldots  & \ldots & \ldots & \ldots & \textcolor{blue}{\ldots} \\
%\arrayrulecolor{white}\hline
%\end{tabular}
%\end{subtable}
%
%\medskip
%\medskip
%\medskip
%\begin{subtable}[t]{\textwidth}
%\centering
%\caption{Result of the variational queries $\vQ_1 = \pi_{\optAtt [\vFour \vee \vFive] [\empno], \name, \fname, \lname} (\empbio)$ and 
%$\VVal {\vQ_1} = \pi_{\optAtt [(\vFour \vee \vFive) \wedge \neg \vThree] [\empno], 
%\optAtt [\vFour \wedge \neg \vThree \wedge \neg \vFive] [\name], 
%\optAtt [\vFive \wedge \neg \vThree \wedge \neg \vFour] [\fname], 
%\optAtt [\vFive \wedge \neg \vThree \wedge \neg \vFour] [\lname]} (\empbio)
%$.}
%\label{tab:vq1-res}
%\footnotesize
%\arrayrulecolor{blue}
%%!{\color{black}\vrule}
%\begin{tabular} {c !{\color{black}\vrule} l l l l : l }
% {\textcolor{blue}{$\oneof {\vThree, \vFour,\vFive}$} }& {\textcolor{blue}{$\vFour \vee \vFive$}}&  {\textcolor{blue}{$\vFour $}} &  {\textcolor{blue}{$\vFive $}} &  {\textcolor{blue}{$\vFive$}} & {\textcolor{blue}{\texttt{true}}}\\
%\arrayrulecolor{blue}\hdashline
%\multirow{2}{*}{$\mathit{result}$}  & \empno & \name & \fname & \lname & \pcatt \\
%\arrayrulecolor{black}\cline{2-6}
%%& 12001 & & & & \textcolor{blue}{\vThree}\\
%%& 12002 & & & & \textcolor{blue}{\vThree}\\
%%& 12003 & & & & \textcolor{blue}{\vThree}\\
% &80001  & Nagui Merli & & & \textcolor{blue}{\vFour}\\
% & 80002 & Mayuko Meszaros & & & \textcolor{blue}{\vFour}\\
% & 80003 & Theirry Viele & & & \textcolor{blue}{\vFour}\\
% & 200001  & & Selwyn & Koshiba & \textcolor{blue}{\vFive}\\
% & 200002  & & Bedrich & Markovitch & \textcolor{blue}{\vFive}\\
% & 200003  & & Pascal & Benzmuller  & \textcolor{blue}{\vFive}\\
% & \ldots  & \ldots & \ldots & \ldots& \textcolor{blue}{\ldots} \\
%\arrayrulecolor{white}\hline
%\end{tabular}
%\end{subtable}
%
%\medskip
%\medskip
%\medskip
%\begin{subtable}[t]{\textwidth}
\centering
\caption[Result of a variational query]{Result of the v-query $\vQ_2 = \chc[\neg \vThree]{\pi_{\empno,\name,\fname,\lname}(\empbio),\empRel}$.}
\label{tab:vq2-res}
\footnotesize
\arrayrulecolor{blue}
%!{\color{black}\vrule}
\begin{tabular} {c !{\color{black}\vrule} l l l l : l }
 {\textcolor{blue}{$\oneof {\vThree, \vFour,\vFive} \wedge \neg \vThree$} }& {\textcolor{blue}{\t}}&  {\textcolor{blue}{$\vFour $}} &  {\textcolor{blue}{$\vFive $}} &  {\textcolor{blue}{$\vFive$}} & {\textcolor{blue}{\texttt{true}}}\\
\arrayrulecolor{blue}\hdashline
\multirow{2}{*}{$\mathit{result}$}  & \empno & \name & \fname & \lname & \pcatt \\
\arrayrulecolor{black}\cline{2-6}
%%& 12001 & & & & \textcolor{blue}{\vThree}\\
%%& 12002 & & & & \textcolor{blue}{\vThree}\\
%%& 12003 & & & & \textcolor{blue}{\vThree}\\
& 12001 & Ulf Hofstetter & Ulf & Hofstetter  & $\textcolor{blue}{ \vFour \vee \vFive}$\\
& 12002 & Luise McFarlan & Luise & McFarlan  & $\textcolor{blue}{ \vFour \vee \vFive}$\\
& 12003 & Shir DuCasse & Shir & DuCasse  & $\textcolor{blue}{ \vFour \vee \vFive}$\\
 &80001  & Nagui Merli & Nagui & Merli & $\textcolor{blue}{ \vFour \vee \vFive}$\\
 & 80002 & Mayuko Meszaros & Mayuko & Meszaros & $\textcolor{blue}{ \vFour \vee \vFive}$\\
 & 80003 & Theirry Viele & Theirry & Viele & $\textcolor{blue}{ \vFour \vee \vFive}$\\
 & 200001  & Selwyn Koshiba & Selwyn & Koshiba & \textcolor{blue}{\vFive}\\
 & 200002  & Bedrich Markovitch & Bedrich & Markovitch & \textcolor{blue}{\vFive}\\
 & 200003  & Pascal Benzmuller & Pascal & Benzmuller  & \textcolor{blue}{\vFive}\\
 & \ldots  & \ldots & \ldots & \ldots & \textcolor{blue}{\ldots} \\
% &80001  & Nagui Merli & & & \textcolor{blue}{\vFour}\\
% & 80002 & Mayuko Meszaros & & & \textcolor{blue}{\vFour}\\
% & 80003 & Theirry Viele & & & \textcolor{blue}{\vFour}\\
% & 200001  & & Selwyn & Koshiba & \textcolor{blue}{\vFive}\\
% & 200002  & & Bedrich & Markovitch & \textcolor{blue}{\vFive}\\
% & 200003  & & Pascal & Benzmuller  & \textcolor{blue}{\vFive}\\
% & \ldots  & \ldots & \ldots & \ldots& \textcolor{blue}{\ldots} \\
\arrayrulecolor{white}\hline
\end{tabular}
%\end{subtable}
%
\end{table}



Note that, as shown in \tabref{vq1-res} and \tabref{vq2-res}, 
queries $\vQ_1$ and $\vQ_2$ return the same set of variational tuples.
However, the first three tuples in \tabref{vq1-res} could belong to a variant that 
enables any of \vThree--\vFive\ whereas the first three tuples in \tabref{vq2-res}
could only belong to variants that either enable \vFour\ or \vFive. 
This difference is due to the difference in their tables' presence conditions, 
that is, $\vQ_2$ filters out tuples that belong to variant \vThree\ at the schema 
level while $\vQ_1$ does not. We discuss this more in \exref{type}. 
More importantly, even though the first three tuples in \tabref{vq1-res} could 
belong to a variant that enables \vThree, configuring \tabref{vq1-res}
for such a variant drops the first three tuples since all their attributes would 
be \nul. We illustrate how configuring \tabref{vq1-res} for variant \setDef \vThree\
drops the first three tuples in \exref{conf-vq}.
% since
%neither returns tuples associated with variant \vThree, but their returned
%variational tables have different presence conditions, thus, $\vQ_2$ filters out
%tuples that belong to variant \vThree\ at the schema level while $\vQ_1$ does not. We discuss this
%more in \exref{type}. 
%

%\NOTE{
%\revised{VRA has \revised{syntactic} equivalence rules, described in
%\secref{var-min}, that enable semantics-preserving transformations of queries
%similar to the transformation of $\vQ_1$ into $\vQ_2$ (and vice versa). These
%rules enable factoring commonality out of subqueries, among other
%transformations.}

%The next example 
 Expressing
the same intent over several database variants by a single query relieves the DBA from
maintaining separate queries for different variants or configurations of the
schema.
\exref{vq-same-intent-mult-vars} 
illustrates this point.
% by using choices.
%how a variational query can be used to express the same
%intent over several database variants using choices and conditions.

\begin{example}
\label{eg:vq-same-intent-mult-vars}
Assume a VDB with  \ensuremath{\features = \setDef{\vOne, \ldots, \vFive}}
and the corresponding \basic\ schema
variants in \tabref{mot}. The user wants to get all employee names across all
variants. They express this intent by the query $\vQ_3$:
%
\begin{align*}
\vQ_3 &= 
  \vOne\chcL
    (\pi_{\name}(\engemp)) \cup (\pi_{\name}(\othemp)) \\
 & \hspace{32pt},
    (\vTwo\vee\vThree)\chcL
      \pi_{\name}(\empacct) \\
 & \hspace{88pt},
      \chc[(\vFour\vee\vFive)]{\pi_{\name,\fname,\lname}\empbio, \emp}\chcR\chcR
\end{align*}
%
Since the variational schema enforces that exactly one of \vOne--\ \vFive\ be enabled, we
can simplify the query by omitting the final choice.
%
\begin{align*}
\vQ_4 &= 
  \vOne\chcL
    (\pi_{\name}(\engemp)) \cup (\pi_{\name}(\othemp)) \\
 & \hspace{32pt},
    \chc[(\vTwo\vee\vThree)]{
      \pi_{\name}(\empacct),
      \pi_{\name,\fname,\lname}(\empbio)}
\end{align*}
%
\end{example}

In principle, variational queries can also express arbitrarily different intents over
different database variants. However, we expect that variational queries are best used to
capture single (or at least related) intents that vary in their realization
since this is easier to understand and increases the potential for sharing in
both the representation and execution of a variational query.







%\subsection{VRA Type System}
\label{sec:typesys}

\TODO{type sys}


\subsection{VRA Type System}
\label{sec:typesys}

\TODO{type sys}


\section{Explicitly Annotating Queries}
\label{sec:constrain}

%\point{type system allows the ql to be flexible and usable.}
%The type system is designed s.t. it relieves the user from necessarily incorporating
%the v-schema variability into their queries as long as the variational queries variability
%does not violate the v-schema, 
Variational queries do not need to repeat information that can be inferred from the v-schema
or the type of a query.
%
For example, the query \ensuremath{\vQ_1} shown in \exref{vq-specific} 
does not contradict the schema and
thus is type correct. However,
 it does not include the presence conditions of attributes and the relation encoded in
the schema while \ensuremath{\vQ_6} repeats this information:\\
%
\centerline{
\ensuremath{
\vQ_6 =
\pi_{\optAtt [\vFour \vee \vFive] [\empno], \optAtt [\vFour] [\name], \optAtt [\vFive] [\fname], \optAtt [\vFive] [\lname]  } \left(\chc [\dimMeta_2] {\empbio, \empRel} \right)}}.

%\pi_{\optAtt [(\vFour \vee \vFive) \wedge \fModel_2] [\empno], \optAtt [\vFour \wedge \fModel_2] [\name], \optAtt [\vFive \wedge \fModel_2] [\fname], \optAtt [\vFive \wedge \fModel_2] [\lname]  } \empbio}}.
%

%\NOTE{
%This is the unsimplified version:
%\begin{align*}
%\VVal {\vQ_5} &= 
%\pi_{\optAtt [\vFour \vee \vFive] [\empno], \optAtt [\vFour] [\name], \optAtt [\vFive] [\fname], \optAtt [\vFive] [\lname]  } \\
%&(\chc [ \fModel_2 ] {\pi_{\empno, \sex, \birthdate, \optAtt [\vFour ] [\name], \optAtt [\vFive] [\fname], \optAtt [\vFive] [\lname]} \empbio, \empRel  })
%\end{align*}
%}
Similarly, the projection in the query 
\ensuremath{\vQ_7 = \pi_{\name, \fname} (\mathit{subq}_7)}
where 
\ensuremath{
\mathit{subq}_7 = \chc [ \vFour] {\pi_\name (\vQ_6), \pi_\fname (\vQ_6)}
}
is written over 
\ensuremath{\vSch_2} and it 
%\centerline{
%\ensuremath{
%\vQ_6 =
%\pi_{\name, \fname} \mathit{subq}_6
%} 
%}}
does not repeat the presence conditions of attributes from its \ensuremath{\mathit{subq}_7}'s type.
The query
%\centerline{
\ensuremath{
\vQ_8 =
\pi_{\optAtt [\vFour ] [\name],\optAtt [\neg \vFour] [\fname]} (\mathit{subq}_7)
%\chc [ \vFour] {\pi_\name \vQ_5, \pi_\fname \vQ_5}
}
%}
makes the annotations of projected attributes \emph{explicit} with respect to both 
the v-schema \ensuremath{\vSch_2} and its subquery's type.
%\TODO {give an example, schema: R(A,B), query: $\pi_{A,B} (F<\pi_A R, \pi_B R>)$
%becomes $\pi_{A^F, B^{\neg F}} ...$}
%The variation encoded in variational queries can
%be more restrictive or more loose than v-schema variation without violating them.
Although relieving the user from explicitly repeating variation makes VRA easier to use, 
queries still have to state variation explicitly to avoid losing information when 
decoupled from the schema.
%We do this by defining a function, 
%\ensuremath {\constrain \vQ}, with type \ensuremath{ \qSet \to \vSchSet \to \qSet
%},
%that \emph{explicitly annotates a query \vQ\ given the underlying schema \vSch}.
We do this by defining the function 
\ensuremath {\constrain \vQ : \qSet \totype \vSchSet \totype \qSet
},
that \emph{explicitly annotates a query \vQ\ with the  schema \vSch}.
%Note that \ensuremath {\constrain \vQ} needs to take the underlying schema as
%an input since it is using the type system (which relies on the schema) as a helper function.
The explicitly annotating query function, 
formally defined in \figref{constrain}, 
conjoins attributes and relations
presence conditions with the corresponding annotations in the query 
and wraps subqueries in a choice when needed. 
Note that, $\vQ_8$ and $\vQ_6$ are the result of $\constrain [\vSch_2] {\vQ_7}$
and $\constrain [\vSch_2] {\vQ_1}$, respectively, after simplification~\footnote{More specifically,
they are simpilified using rules defined in \figref{var-min}}.
%Queries $\vQ_7$ and $\vQ_5$ are examples of applying the 
%explicitly annotation function to queries $\vQ_6$ and $\vQ_1$, respectively,
%after simplifying them.
%\exref{constrain} illustrates how the constrain function transforms queries
%and allows users to be more flexible with their queries. 

\section{Explicitly Annotating Queries}
\label{sec:constrain}

%\point{type system allows the ql to be flexible and usable.}
%The type system is designed s.t. it relieves the user from necessarily incorporating
%the v-schema variability into their queries as long as the variational queries variability
%does not violate the v-schema, 
Variational queries do not need to repeat information that can be inferred from the v-schema
or the type of a query.
%
For example, the query \ensuremath{\vQ_1} shown in \exref{vq-specific} 
does not contradict the schema and
thus is type correct. However,
 it does not include the presence conditions of attributes and the relation encoded in
the schema while \ensuremath{\vQ_6} repeats this information:\\
%
\centerline{
\ensuremath{
\vQ_6 =
\pi_{\optAtt [\vFour \vee \vFive] [\empno], \optAtt [\vFour] [\name], \optAtt [\vFive] [\fname], \optAtt [\vFive] [\lname]  } \left(\chc [\dimMeta_2] {\empbio, \empRel} \right)}}.

%\pi_{\optAtt [(\vFour \vee \vFive) \wedge \fModel_2] [\empno], \optAtt [\vFour \wedge \fModel_2] [\name], \optAtt [\vFive \wedge \fModel_2] [\fname], \optAtt [\vFive \wedge \fModel_2] [\lname]  } \empbio}}.
%

%\NOTE{
%This is the unsimplified version:
%\begin{align*}
%\VVal {\vQ_5} &= 
%\pi_{\optAtt [\vFour \vee \vFive] [\empno], \optAtt [\vFour] [\name], \optAtt [\vFive] [\fname], \optAtt [\vFive] [\lname]  } \\
%&(\chc [ \fModel_2 ] {\pi_{\empno, \sex, \birthdate, \optAtt [\vFour ] [\name], \optAtt [\vFive] [\fname], \optAtt [\vFive] [\lname]} \empbio, \empRel  })
%\end{align*}
%}
Similarly, the projection in the query 
\ensuremath{\vQ_7 = \pi_{\name, \fname} (\mathit{subq}_7)}
where 
\ensuremath{
\mathit{subq}_7 = \chc [ \vFour] {\pi_\name (\vQ_6), \pi_\fname (\vQ_6)}
}
is written over 
\ensuremath{\vSch_2} and it 
%\centerline{
%\ensuremath{
%\vQ_6 =
%\pi_{\name, \fname} \mathit{subq}_6
%} 
%}}
does not repeat the presence conditions of attributes from its \ensuremath{\mathit{subq}_7}'s type.
The query
%\centerline{
\ensuremath{
\vQ_8 =
\pi_{\optAtt [\vFour ] [\name],\optAtt [\neg \vFour] [\fname]} (\mathit{subq}_7)
%\chc [ \vFour] {\pi_\name \vQ_5, \pi_\fname \vQ_5}
}
%}
makes the annotations of projected attributes \emph{explicit} with respect to both 
the v-schema \ensuremath{\vSch_2} and its subquery's type.
%\TODO {give an example, schema: R(A,B), query: $\pi_{A,B} (F<\pi_A R, \pi_B R>)$
%becomes $\pi_{A^F, B^{\neg F}} ...$}
%The variation encoded in variational queries can
%be more restrictive or more loose than v-schema variation without violating them.
Although relieving the user from explicitly repeating variation makes VRA easier to use, 
queries still have to state variation explicitly to avoid losing information when 
decoupled from the schema.
%We do this by defining a function, 
%\ensuremath {\constrain \vQ}, with type \ensuremath{ \qSet \to \vSchSet \to \qSet
%},
%that \emph{explicitly annotates a query \vQ\ given the underlying schema \vSch}.
We do this by defining the function 
\ensuremath {\constrain \vQ : \qSet \totype \vSchSet \totype \qSet
},
that \emph{explicitly annotates a query \vQ\ with the  schema \vSch}.
%Note that \ensuremath {\constrain \vQ} needs to take the underlying schema as
%an input since it is using the type system (which relies on the schema) as a helper function.
The explicitly annotating query function, 
formally defined in \figref{constrain}, 
conjoins attributes and relations
presence conditions with the corresponding annotations in the query 
and wraps subqueries in a choice when needed. 
Note that, $\vQ_8$ and $\vQ_6$ are the result of $\constrain [\vSch_2] {\vQ_7}$
and $\constrain [\vSch_2] {\vQ_1}$, respectively, after simplification~\footnote{More specifically,
they are simpilified using rules defined in \figref{var-min}}.
%Queries $\vQ_7$ and $\vQ_5$ are examples of applying the 
%explicitly annotation function to queries $\vQ_6$ and $\vQ_1$, respectively,
%after simplifying them.
%\exref{constrain} illustrates how the constrain function transforms queries
%and allows users to be more flexible with their queries. 

\input{formulas/constrainVQbySch}

\begin{theorem}
\label{thm:expl-same-type}
If the query \vQ\ has the type \vType\ then its explicitly annotated counterpart has the same type \vType, i.e.: \\
%
\centerline{
\ensuremath{%\raggedleft
\envWithoutVctx {\vQ} {\vType} \Rightarrow \envWithoutVctx {\constrain \vQ} {\VVal \vType} \textit{ and } \vType \equiv {\VVal \vType}
}}
%
This shows that the type system applies the schema to the type of a query although it does not apply it to the query. 
The \emph{type equivalence} is variational set equivalence, defined 
in \figref{vset}, for normalized variational sets of attributes.
%\footnote{We proved this theorem in the Coq proof assistant. The encoding of the theorem and the proof can be found in second author's MS thesis~\cite{FaribaThesis}.}.
\end{theorem}

We encode and prove \thmref{expl-same-type} in the Coq proof assistant~\cite{FaribaThesis}.
We also illustrate the application of \thmref{expl-same-type} to queries
\ensuremath{\vQ_1} and \ensuremath{\vQ_6}.
%
\exref{type} explained how \ensuremath{\vQ_1}'s type is generated step-by-step.
The variation context and underlying schema are
the same and the subquery \empbio\ has the same type. 
The projected attribute set annotated with the variation context is:
\ensuremath{
\vAttList_2 =  \{\annot [\vFour \vee \vFive] \empno, }
\ensuremath{ 
\optAtt [\vFour] [\name], \optAtt [\vFive] [\fname], \optAtt [\vFive] [\lname]\}^{\dimMeta_2}}, which is clearly subsumed by \ensuremath{\vAttList_\empbio}, thus, 
%the type of \empbio, \vAttList, and
its intersection with \ensuremath{\vAttList_\empbio} annotated
with the presence condition of \ensuremath{\vAttList_\empbio} is itself,
hence, \ensuremath{\vAttList_{\vQ_1} \equiv \vAttList_{\vQ_6}}.
%which makes it obvious that \ensuremath{\vAttList_{\vQ_1} \equiv \vAttList_{\vQ_6}}.
%\end{example}

\begin{theorem}
\label{thm:expl-same-type}
If the query \vQ\ has the type \vType\ then its explicitly annotated counterpart has the same type \vType, i.e.: \\
%
\centerline{
\ensuremath{%\raggedleft
\envWithoutVctx {\vQ} {\vType} \Rightarrow \envWithoutVctx {\constrain \vQ} {\VVal \vType} \textit{ and } \vType \equiv {\VVal \vType}
}}
%
This shows that the type system applies the schema to the type of a query although it does not apply it to the query. 
The \emph{type equivalence} is variational set equivalence, defined 
in \figref{vset}, for normalized variational sets of attributes.
%\footnote{We proved this theorem in the Coq proof assistant. The encoding of the theorem and the proof can be found in second author's MS thesis~\cite{FaribaThesis}.}.
\end{theorem}

We encode and prove \thmref{expl-same-type} in the Coq proof assistant~\cite{FaribaThesis}.
We also illustrate the application of \thmref{expl-same-type} to queries
\ensuremath{\vQ_1} and \ensuremath{\vQ_6}.
%
\exref{type} explained how \ensuremath{\vQ_1}'s type is generated step-by-step.
The variation context and underlying schema are
the same and the subquery \empbio\ has the same type. 
The projected attribute set annotated with the variation context is:
\ensuremath{
\vAttList_2 =  \{\annot [\vFour \vee \vFive] \empno, }
\ensuremath{ 
\optAtt [\vFour] [\name], \optAtt [\vFive] [\fname], \optAtt [\vFive] [\lname]\}^{\dimMeta_2}}, which is clearly subsumed by \ensuremath{\vAttList_\empbio}, thus, 
%the type of \empbio, \vAttList, and
its intersection with \ensuremath{\vAttList_\empbio} annotated
with the presence condition of \ensuremath{\vAttList_\empbio} is itself,
hence, \ensuremath{\vAttList_{\vQ_1} \equiv \vAttList_{\vQ_6}}.
%which makes it obvious that \ensuremath{\vAttList_{\vQ_1} \equiv \vAttList_{\vQ_6}}.
%\end{example}
\section{VRA Semantics }
\label{sec:vrasem}

%\TODO{vra semantics. we understand it through RA sem + accumulation}

We use the semantics of relational queries to define the semantics of 
variational queries. We first define the configuration function
for variational queries which takes a configuration and a variational query
and returns a relational query, \secref{vraconf}. We also define another version of the
variational query configuration function that generates unique relational
query variants, \secref{vraconf}. Then, we define an accumulation function that accumulates
multiple (annotated) relational tables into a variational table, \secref{accum}. Finally, we  
define the denotational semantics of VRA using the defined configuration and
accumulation functions, \secref{vradensem}.
%
%\maybeAdd{if have time add VRA sem + equiv}
%dentoational semantics of VRA
%equivalence of dent sem and config and accumulation  --> in properties section



\subsection{VRA Configuration}
\label{sec:vraconf}

\begin{figure}
%\textbf{Configuration selection semantics of \vqsTxt:}
\begin{alignat*}{1}
\eeSem [] . &: \qSet \to \confSet \to \pQSet\\
%
\eeSem \vRel &= \orSem \vRel = \pRel\\
\eeSem {\vSel \vQ}  &= \vSel [\ecSem \vCond] {\eeSem \vQ}\\
%
\eeSem {\vPrj [\vAttList] \vQ} &= \vPrj [\olSem \vAttList] {\eeSem \vQ}\\
%
\eeSem {{\vQ_1} \times {\vQ_2}} &= \eeSem {\vQ_1} \times \eeSem {\vQ_2}\\
%
%\eeSem {{\vQ_1} \Join_\vCond {\vQ_2}} &= \eeSem {\vQ_1} \Join_{\ecSem \vCond} \eeSem {\vQ_2}\\
%
\eeSem {\chc {\vQ_1, \vQ_2}} &= 
	\begin{cases}
		\eeSem {\vQ_1}, \text{ if } \fSem \dimMeta = \t\\
		\eeSem {\vQ_2}, \text{ otherwise}
	\end{cases}\\
%
\eeSem {{\vQ_1} \circ {\vQ_2}} &= \eeSem {\vQ_1} \circ \eeSem {\vQ_2}\\
%
\eeSem {\empRel} &= \underline {\empRel}
\end{alignat*}
\caption{Configuration of VRA which assumes that the given v-query
is well-typed. 
%\orSem ., \ecSem ., and \olSem . are
%configuration of v-relation, v-condition, and variational attribute
%set, respectively, defined in \figref{vdb-conf}, 
%\figref{vcond-conf-sem}, \figref{vdb-conf}.
Note that we have extended RA with an empty relation $\underline {\empRel}$.}
\label{fig:v-alg-conf-sem}
\end{figure}


%\NOTE{
%Also, the following definition of the semantics contradicts with the
%description earlier in the section about producing a \emph{result}
%relation.
%
%\medskip
%Also also, maybe we should move the discussion of the semantics before the
%examples? It's a bit surprising to come across it here.}

%The semantics of VRA can be understood as a combination of the
%\emph{configuration semantics} of VRA, defined in \figref{v-alg-conf-sem}, the
%configuration semantics of VDBs, defined in \figref{vdb-conf}, and the
%semantics of plain RA.
%%
%%\TODO{Make the following a more precise description of how these three
%%semantics work together, i.e.\ for every valid configuration of the feature
%%model, we can configure the variational query and VDB in the same way to yield a plain RA
%%query that is then executed over the corresponding plain RDB.}
%%
%Thus, the variational query
%semantics is the set of semantics of its configured relational queries over
%their corresponding configured relational database variant for every valid
%configuration of the feature model of the VDB.
%
%We now embark on the formal definition of variational queries configuration.
The \emph{configuration} function maps a variational query under
a configuration
to a relational query, defined in \figref{v-alg-conf-sem}. Thus, a variational query 
can be understood as a set of relational queries, the results of which are gathered
in a single table and tagged with the feature expression stating their variants.
%Configuring a variational query
%for all valid configurations, accessible from VDB's feature model,
%provides the complete meaning of a variational query in terms of RA semantics.
%
Users can deploy queries for a specific variant by configuring 
the variational query.
%
%The configuration of a query allows users to deploy queries for a
%specific variant when they desire, 
%satisfying query part of \nThree\ requirement. 
\exref{conf-vq} illustrates configuring a query
and \exref{vq-sem} illustrates the configuration of query $\VVal {\vQ_1}$ from \exref{vq-specific} and the corresponding relational results table.

%\begin{figure}
%\textbf{Configuration selection semantics of variational conditions:}
\begin{alignat*}{1}
\ecSem [] . &: \vCondSet \to \confSet \to \pCondSet\\
%
\ecSem \bTag &= \bTag \\
%
\ecSem \vAttOpCte &= 
    \vAttOpCte\\
%	\begin{cases}
%		\vAttOpCte, &\text{ if } \pAtt \in \attr [\eeSem \vRel]\\
%		\f, &\text{ otherwise}
%	\end{cases}\\
%
\ecSem \vAttOpAtt &= 
       \vAttOpAtt\\
%	\begin{cases}
%		\pAttOpAtt, &\text{ if } \pAtt_1 \in \attr [\eeSem \vRel] \&\ 
%		                                   \pAtt_2 \in \attr [\eeSem \vRel] \\
%		\f,  &\text{ otherwise}
%	\end{cases}\\
%
\ecSem {\neg \vCond} &= \neg \ecSem \vCond\\
%
\ecSem {\orr \vCond} &= \ecSem {\vCond_1} \vee \ecSem {\vCond_2}\\
%
\ecSem {\annd \vCond} &= \ecSem {\vCond_1} \wedge \ecSem {\vCond_2}\\
%
\ecSem {\chc {\vCond_1, \vCond_2}} &=
	\begin{cases}
		\ecSem {\vCond_1}, &\text{ if } \fSem \dimMeta = \t \\
		\ecSem {\vCond_2}, &\text{ otherwise}
	\end{cases}
\end{alignat*}
\caption{V-condition configuration.
%which assumes v-conditions
%are well-typed.
}
\label{fig:vcond-conf-sem}
\end{figure}


%To define VRA semantics we map 
%a variational query to a pure relational query to re-use RA's semantics.
%However, to avoid losing the variation encoded 
%in the variational query, 
%we need to determine the variant under which such a
%mapping is valid. Thus, we introduce the semantic functions that 
%relate a variational query to a relational query.

%
%\textbf{Configuring a variational query:} 
%It maps a variational query under a 
%given configuration to a relational query, denoted by \eeSem . 
%and defined in \figref{v-alg-conf-sem}. Configuring a variational query
%for all valid configurations, accessible from VDB's feature model,
%provides the complete meaning of a variational query in terms of RA semantics.
%Users can deploy queries for a specific variant by configuring 
%them,
%%The configuration of a query allows users to deploy queries for a
%%specific variant when they desire, 
%satisfying query part of \nThree.

\begin{example}
\label{eg:conf-vq}
Assume the underlying VDB has the variational schema
% \t\ feature model and the variational relation
\ensuremath{
\vSch_3 = \{ \vRel \left( \optAtt [\fOne] [\vAtt_1], \vAtt_2, \vAtt_3 \right)^{\fOne \vee \fTwo}
\}} 
and the feature space 
\ensuremath{
\features = \setDef{ \fOne, \fTwo}}.
For valid configurations of this VDB (that is, \setDef {\ }, \setDef \A, \setDef \B, and \setDef {\A, \B}), 
the variational query 
\ensuremath{
\vQ_5 = \vPrj [{\vAtt_1, \optAtt [\fOne \wedge \fTwo] [\vAtt_2], \optAtt [\fTwo] [\vAtt_3]}] (\vRel)
}
is configured to the following relational queries:
\begin{alignat*}{1}
\eeSem [\setDef \ ] {\vQ_5} &= \pi_{\pAtt_1} \pRel\\
\eeSem [\setDef \fOne] {\vQ_5} &=  \pi_{\pAtt_1} \pRel\\
\eeSem [\setDef \fTwo] {\vQ_5} &= \pi_{\pAtt_1, \pAtt_3} \pRel\\
\eeSem [\setDef {\fOne, \fTwo}] {\vQ_5} &= \pi_{\pAtt_1, \pAtt_2, \pAtt_3} \pRel
\end{alignat*}
\end{example}





%\textbf{Grouping a variational query:} 
%maps a variational query to a set of
%relational queries annotated with feature expressions, denoted by \qGroup .
%and defined in \figref{vq-group}. The presence condition of relational queries 
%indicate the group of configurations where the mapping holds. In essence, 
%grouping of variational query \vQ\ groups together all configurations with the same relational
%query produced from configuring \vQ. 
%Hence, the generated set
%%\dropit{could drop this if it's confusing!}
%of relational queries from grouping a variational query contains distinct (unique) queries.
%For example, consider the query \ensuremath {\vQ_5} in \exref{conf-vq}.
%Grouping \ensuremath{\vQ_5} results in the set:
%\ensuremath{
%\setDef{
%\left( \pi_{\pAtt_1, \pAtt_2, \pAtt_3} \pRel \right)^{\fOne \wedge \fTwo},
%\left(\pi_{\pAtt_1, \pAtt_3} \pRel \right)^{\neg \fOne \wedge \fTwo},
%\left(  \pi_{\pAtt_1} \pRel \right)^{( \fOne \wedge \neg \fTwo) \vee (\neg \fOne \wedge \neg \fTwo)}
%}
%}.
%
%
%

\begin{example}
\label{eg:vq-sem}
Consider the query $\VVal {\vQ_1}$ given in \exref{vq-specific}: \\
\centerline{
$\VVal {\vQ_1} = 
\pi_{\optAtt [(\vFour \vee \vFive) \wedge \neg \vThree] [\empno], 
\optAtt [\vFour \wedge \neg \vThree \wedge \neg \vFive] [\name], 
\optAtt [\vFive \wedge \neg \vThree \wedge \neg \vFour] [\fname], 
\optAtt [\vFive \wedge \neg \vThree \wedge \neg \vFour] [\lname]} (\empbio)
$.}  
Configuring $\VVal {\vQ_1}$ for all valid configurations 
(\setDef \vThree, \setDef \vFour, \setDef \vFive) of the given VDB
results in three relational queries:
%
\begin{alignat*}{1}
%
\eeSem [\setDef {\vThree}] {\VVal {\vQ_1}} &= \empRel\\
%
\eeSem [\setDef {\vFour}] {\VVal {\vQ_1}} &= \pi_{\empno, \name} (\empbio)\\
%
\eeSem [\setDef {\vFive}] {\VVal {\vQ_1}} &= \pi_{\empno, \fname, \lname} (\empbio)
%
%\eeSem [\setDef {\ }] {\VVal {\vQ_1}} &= \empRel 
\end{alignat*}
%
\noindent
\tabref{vq-conf-res} shows the result of these relational queries.
\end{example}

\begin{table}[!htbp]
\caption[Results of relational queries from configuring a variational query]{Results of relational queries from configuring the variational query $\VVal {\vQ_1}$.}
\label{tab:vq-conf-res}
\centering
\small
%\footnotesize
%\scriptsize
\begin{subtable}[t]{\textwidth}
\centering
\caption{Result of the query $\eeSem [\setDef {\vThree}] {\VVal {\vQ_1}} = \empRel$.}
\label{tab:vq-conf1}
\arrayrulecolor{black}
\begin{tabular} {c | l }
\multirow{2}{*}{$\mathit{result}$} & \textcolor{white}{blah blah}\\
\cline{2-2}
&  \\
\arrayrulecolor{white}\hline
\end{tabular}
\end{subtable}

%\medskip
%\medskip
\medskip
\begin{subtable}[t]{\textwidth}
%\begin{center}
\centering
%\tiny
\caption{Result of the query $\eeSem [\setDef {\vFour}] {\VVal {\vQ_1}} = \pi_{\empno, \name} (\empbio)$.}
\label{tab:vq-conf2}
\arrayrulecolor{black}
\begin{tabular} {c | l l }
%\hline
%\hhline{-==}
\multirow{2}{*}{$\mathit{result}$}  & \empno & \name\\
\cline{2-3}
 &80001 & Nagui Merli\\
 & 80002 & Mayuko Meszaros\\
 & 80003 & Theirry Viele\\
&\ldots & \ldots \\
\arrayrulecolor{white}\hline
\end{tabular}
%\end{center}
\end{subtable}

%\medskip
%\medskip
\medskip
\begin{subtable}[t]{\textwidth}
%\begin{center}
\centering
%\footnotesize
%\tiny
\caption{Result of the query \ensuremath{\eeSem [\setDef {\vFive}] {\VVal {\vQ_1}} = \pi_{\empno, \fname, \lname} (\empbio)}.}
\label{tab:vq-conf3}
\arrayrulecolor{black}
\begin{tabular} {c | l l l}
%\hline
%\hhline{-==}
\multirow{2}{*}{$\mathit{result}$}  & \empno &\fname &\lname\\
\cline{2-4}
 & 200001 & Selwyn & Koshiba \\
 & 200002 & Bedrich & Markovitch \\
 & 200003 & Pascal & Benzmuller  \\
 & \ldots & \ldots & \ldots \\
 \arrayrulecolor{white}\hline
\end{tabular}
%\end{center}
\end{subtable}

\end{table}




Often a variational query will yield the same plain query for multiple configurations.
For our semantics, it is useful to get the set of unique variants of a variational query.
%Unfortunately, the configuration of variational queries may result in
%duplicate relational queries. In practice, this is not very efficient, as discussed
%later in \secref{exp}. 
Thus, we define the \emph{unique variants} (unique configuration) function, whose type is given below.
\[
\qGroup[\cdot]{\cdot} : \qSet \totype \settype \fSet \totype \settype {\bm{(} \vartype \pQSet \bm{)}}
\]
This function takes a variational query and VDB's set of features
and returns a set of configured relational queries annotated with
a presence condition. The presence condition is a feature expression generated from
the set of configurations that configured the variational query into the same relational query.
To generate this presence condition from configurations we need to know the closed 
set of VDB's features.
%
This is done by the $\mathit{genFexp} (\config,\features)$ that takes a configuration and a closed set of 
features and generates the feature expression \dimMeta\ that is only satisfiable by the configuration
\config. For example, $\mathit{genFexp} (\setDef {\A},\setDef {\A, \B}) = \A \wedge \neg \B$ and
$\mathit{genFexp} (\setDef {\A, \B},\setDef {\A, \B}) = \A \wedge \B$.
%
Remember that the set of enabled features of a configuration denote the said configuration,
for example, $\setDef {\A}$ denotes the configuration in which only feature \A\ has been 
enabled.

%
In essence, the unique variants function can be defined for all data types that encode variation.
For example, the unique configuration function for 
variational queries can be defined as follows.
\begin{alignat*}{1}
\qGroup{\vQ} &=
  \{ \pQ^{e_1 \vee\ldots\vee e_n}
     \myOR \pQ^{e_1}, \ldots, \pQ^{e_n}
       \in \{ (\eeSem{\vQ})^{\mathit{genFexp}(\config,\features)}
         \myOR \config\in\confSet \} \}
% \qGroup \vQ &= \{ \annot \pQ \myOR \dimMeta = \bigvee_{\dimMeta_i \in \mathit{es}} \dimMeta_i,
% \mathit{es} = \{\dimMeta_i \myOR \forall \config \in \confSet. \eeSem \vQ = \pQ, 
% \dimMeta_i = \mathit{genFexp} (\config,\features) \}\\
% &\hspace{50pt}, \exists \config \in \confSet. \fSem \dimMeta = \t, \eeSem \vQ = \pQ
% \}
\end{alignat*}
%\centerline{
%\ensuremath{
%\qGroup \vQ = \setDef {\annot \pQ \myOR \forall \config \in \confSet: \fSem \dimMeta = \t,
%\eeSem \vQ = \pQ}
%}.}
%\]
The unique configuration for variational sets of attributes
($\aG(\cdot,\cdot)$) and variational conditions ($\cG(\cdot,\cdot)$) are
defined similarly; their types are given below.
\begin{alignat*}{1}
\aG(\cdot,\cdot) &:
  \vAttSet \totype \settype \fSet \totype {\vartype {\bm{(}\settype \attnametype \bm{)}}} \\
\cG(\cdot,\cdot) &:
  \vCondSet \totype \settype \fSet \totype \vartype \pCondSet
\end{alignat*}
%
However, the definition of $\qGroup[\cdot]{\cdot}$ is not efficient since it
still enumerates all possible 
configurations. Thus, we define the more efficient unique configuration function
for variational queries in \figref{vq-group}.
%
%\exref{group-vq} and \exref{vq-group} provide the unique configuration of the queries 
%given in \exref{conf-vq} and \exref{vq-sem}, respectively.

\begin{figure}
%\textbf{Configuration selection semantics of \vqsTxt:}
\begin{alignat*}{1}
\qGroup . &: \qSet \totype \settype {\bm{(} \vartype \pQSet \bm{)}}\\
%
\qGroup \vRel &= \setDef {\annot [\t] \pRel}\\
\qGroup {\vSel \vQ}  &=  
\setDef {\annot [\dimMeta \wedge \dimMeta_\vCond] {\left(\sigma_{\pCond} \pQ\right)} \myOR
\annot \pQ \in \qGroup \vQ, \annot [\dimMeta_\vCond] \pCond \in \cGroup}
\\
%
\qGroup {\vPrj [\vAttList] \vQ} &= 
\setDef {\annot [\dimMeta \wedge \dimMeta_\vAttList] {\left(\pi_{\pAttList} \pQ \right)} \myOR
\annot \pQ \in \qGroup \vQ, \annot [\dimMeta_\vAttList] \pAttList \in \aGroup}
\\
%
\qGroup {{\vQ_1} \times {\vQ_2}} &= 
\setDef {\annot [\dimMeta_1 \wedge \dimMeta_2] {\left(\pQ_1 \times \pQ_2\right)} \myOR
\annot [\dimMeta_1] \pQ_1 \in \qGroup {\vQ_1}, \annot [\dimMeta_2] \pQ_2 \in \qGroup {\vQ_2} }
\\
%
\qGroup {{\vQ_1} \Join_\vCond {\vQ_2}} &= 
\setDef {\annot [\dimMeta_1 \wedge \dimMeta_2 \wedge \dimMeta_\vCond] {\left(\pQ_1 \Join_{\pCond} \pQ_2 \right)} \myOR 
\annot [\dimMeta_1] \pQ_1 \in \qGroup {\vQ_1}, \annot [\dimMeta_2] \pQ_2 \in \qGroup {\vQ_2}
%& \hspace{104pt}
,\annot [\dimMeta_\vCond] \pCond \in \cGroup  }
\\
%
\qGroup {\chc {\vQ_1, \vQ_2}} &= 
\setDef {\annot [\dimMeta \wedge \dimMeta_1] \pQ_1 \myOR  \annot [\dimMeta_1] \pQ_1 \in \qGroup {\vQ_1} }
\cup 
\setDef {\annot [\neg \dimMeta \wedge \dimMeta_2] \pQ_2 \myOR  \annot [\dimMeta_2] \pQ_2 \in \qGroup {\vQ_2}}  \\
%
\qGroup {{\vQ_1} \circ {\vQ_2}} &= 
\setDef {\annot [\dimMeta_1 \wedge \dimMeta_2] {\left(\pQ_1 \circ \pQ_2\right)} \myOR
\annot [\dimMeta_1] \pQ_1 \in \qGroup {\vQ_1}, \annot [\dimMeta_2] \pQ_2 \in \qGroup {\vQ_2} }\\
%
\qGroup {\empRel} &= \annot [\t] { \empRel}
\end{alignat*}
\caption[Unique configuration of variational queries]{Unique configuration of variational queries. 
The unique configuration function assumes that the input is well-typed.
}
\label{fig:vq-group}
\end{figure}


\begin{example}
\label{eg:group-vq}
Consider the query \ensuremath{
\vQ_5 = \vPrj [{\vAtt_1, \optAtt [\fOne \wedge \fTwo] [\vAtt_2], \optAtt [\fTwo] [\vAtt_3]}] (\vRel)
}
given in \exref{conf-vq}. The unique configuration of this query results in the following set of queries:
%
\[
\qGroup [\{\A,\B\}] {\vQ_5} = \setDef {
\annot [(\A \wedge \neg \B) \vee (\neg \A \wedge \neg \B)] {(\pi_{\pAtt_1} (\pRel))},
\annot [\neg \A \wedge \B] {( \pi_{\pAtt_1, \pAtt_3} (\pRel))},
\annot [\A \wedge \B] {(\pi_{\pAtt_1, \pAtt_2, \pAtt_3} (\pRel))}
}.
\]
\end{example}

\begin{example}
\label{eg:vq-group}
Consider the query $\VVal {\vQ_1}$ configured in \exref{vq-sem}:\\
\centerline{
$\VVal {\vQ_1} = 
\pi_{\optAtt [(\vFour \vee \vFive) \wedge \neg \vThree] [\empno], 
\optAtt [\vFour \wedge \neg \vThree \wedge \neg \vFive] [\name], 
\optAtt [\vFive \wedge \neg \vThree \wedge \neg \vFour] [\fname], 
\optAtt [\vFive \wedge \neg \vThree \wedge \neg \vFour] [\lname]} (\empbio)
$.}  
The unique configuration of it results in:
\begin{alignat*}{1}
\qGroup [\{\vThree, \vFour, \vFive \}] {\VVal {\vQ_1}} &= \{
\annot [\vThree \wedge \neg \vFour \wedge \neg \vFive] {\empRel},
\annot [\neg \vThree \wedge \vFour \wedge \neg \vFive] {\left(\pi_{\empno, \name} (\empbio)\right)}\\
&\qquad ,\annot [\neg \vThree \wedge \vFour \wedge \neg \vFive] {\left(\pi_{\empno, \fname, \lname} (\empbio) \right)}\}.
\end{alignat*}
\end{example}



\subsection{Accumulation of Relational Tables to a Variational Table}
\label{sec:accum}

After connecting variational queries to relational queries, to define the 
semantics of VRA we need to connect
the results of multiple relational queries to the result of a single variational 
query. 
%
Since we have two approaches to connect a variational query to relational queries 
we define two \emph{accumulation} functions that generate a 
variational table from a set of relational tables. 

%
The first accumulation function $\mathit{accum} : \settype \fSet \totype \settype {\typepair \confSet \pTabSet} \totype \tabletype$ takes the feature space of a database and a set of relational
tables with their attached configurations and generates a variational table. \figref{accum1} 
defines this function in terms of some auxiliary functions. 
%
The $\mathit{mkTable}$ function takes a variational relation schema and a set of 
variational relation contents and generates a variational table that has the given schema
and the variational tuples in the input tables. 
%
The $\mathit{addPresCondToConfTables}$ function maps the $\mathit{addPresCondToConfContent}$
over a set of tables and their attached configuration and the  $\mathit{addPresCondToConfContent}$
function adds the \pcatt\ attribute to a relational table and its corresponding value which is 
a feature expression associated with the given configuration using the closed set of
features.
%
The $\mathit{fitConfTablesToVsch}$ maps the function $\mathit{fitTableToVsch}$ to tables of a set of 
relational tables and their attached configuration.
The $\mathit{fitTableToVsch}$ function adjusts a table, both its schema and content, 
to a variational relation schema.
%
The $\mathit{tablesToVsch}$ maps the function $\mathit{schToVsch}$ to a set of 
relational tables and their attached configuration. 
The $\mathit{schToVsch}$ generates a variational relation schema from a set of
plain relation schema and their attached configuration given the closed set of 
features of the database's feature space.%
\footnote{In the implementation, for efficiency, we pass the type of the query from VRA's type system
as the variational relation schema that is generated by the $\mathit{tablesToVsch}$ function.}
%
Note that to generate a feature expression from a configuration it is essential to
pass the closed set of features.
%
\exref{acc-table-from-conf} illustrates the behavior of these auxiliary functions and the
table accumulation function over the relational tables in \tabref{vq-conf-res}.


\begin{figure}

\textbf{Table accumulation function:}
\begin{alignat*}{1}
\mathit{accum} &: \settype \fSet \totype \settype {\typepair \confSet \pTabSet} \totype \tabletype\\
\mathit{accum} \  \mathit{fs} \ \mathit{ts} &= \mathit{mkTable} \ \mathit{vsch} \ \mathit{tables}\\
%&\hspace{60pt} (\mathit{addPresCondToConfTables} \ \mathit{fs} \\
%&\hspace{140pt} (\mathit{fitConfTablesToVsch} \ \mathit{ts} \ \mathit{vsch}))\\
&\hspace{-40pt}\textit{where }
\mathit{vsch} = \mathit{tablesToVsch} \ \mathit{fs} \ \mathit{ts}\\
&\hspace{-6pt} \mathit{tables} = \mathit{addPresCondToConfTables} \ \mathit{fs} \ \mathit {fitted}\\
&\hspace{-6pt} \mathit{fitted} = \mathit{fitConfTablesToVsch} \ \mathit{ts} \ \mathit{vsch}
\end{alignat*}


\medskip 
\textbf{Auxiliary functions for table accumulation:}
\footnotesize
\begin{alignat*}{1}
\mathit{schToVsch} &: \settype \fSet \totype \settype {\typepair \confSet \pRelSchSet} \totype \vRelSchSet\\
\mathit{tablesToVsch} &: \settype \fSet \totype \settype {\typepair \confSet \pTabSet} \totype \vRelSchSet\\
\mathit{fitTableToVsch} &: \pTabSet \totype \vRelSchSet \totype \pTabSet\\
\mathit{fitConfTablesToVsch} &: \settype {\typepair \confSet \pTabSet} \totype \vRelSchSet \totype \settype {\typepair \confSet \pTabSet}\\
\mathit{addPresCondToConfContent} &: \settype \fSet \totype \typepair \confSet \pRelContSet \totype \vRelContSet\\
\mathit{addPresCondToConfTables} &: \settype \fSet \totype \settype {\typepair \confSet \pTabSet} \totype \settype \vRelContSet\\
\mathit{mkTable} &: \vRelSchSet \totype \settype \vRelContSet \totype \tabletype
\end{alignat*}


\caption[Accumulation function of a set of relational tables with their attached configuration into a variational table]{Accumulation function of a set of relational tables with their attached configuration into a variational table and its auxiliary functions. The definition uses spaces to pass parameters. For example, $f \ x$ states that the parameter $x$ is passed to the function $x$ and $f\ x\ y$ states that
parameters $x$ and $y$ are passed to $f$ as the first and second arguments, respectively.
}
\label{fig:accum1}
\end{figure}



\begin{example}
\label{eg:acc-table-from-conf}
Consider the query $\VVal {\vQ_1}$ written over the VDB with variational schema $\vSch_2$ and 
feature space $\features = \setDef {\vThree, \vFour, \vFive}$, all given in \exref{vq-specific}. 
%
All configured relational queries of $\VVal {\vQ_1}$ for VDB's valid configurations and their
corresponding results in form of a relational table are given
in \exref{vq-sem} and \tabref{vq-conf-res}, respectively. 
%
Now we show how the relational tables of the configured queries, shown in \tabref{vq-conf-res}, are accumulated 
to the variational table, shown in \tabref{vq1-res}, as the result of the variational query $\VVal \vQ_1$ by 
using the table accumulation function $\mathit{accum}$.
%
As the first step of accumulation, we generate the variational relation schema by
applying $\mathit{tablesToVsch}$ to tables in \tabref{vq-conf-res}. 
This results in the variational relation schema $\vRelSch_{\mathit{accum}}$
%
\begin{alignat*}{1}
\vRelSch_{\mathit{accum}} &= \mathit{result} (\annot [(\neg \vThree \wedge \vFour \wedge \neg \vFive)\vee(\neg \vThree \wedge \neg \vFour \wedge \vFive)] \empno, \annot [\neg \vThree \wedge \vFour \wedge \neg \vFive] \name,\\
&\hspace{30pt}\annot [\neg \vThree \wedge \neg \vFour \wedge \vFive] \fname, \annot [\neg \vThree \wedge \neg \vFour \wedge \vFive] \lname)^{\oneof {\vThree,\vFour,\vFive}}
\end{alignat*}
%
\noindent
Note that the presence conditions are generated based on the configurations attached to
the tables. For example, the presence condition $(\neg \vThree \wedge \vFour \wedge \neg \vFive)\vee(\neg \vThree \wedge \neg \vFour \wedge \vFive)$ associated with the attribute \empno\
is the disjunction of  two feature expressions $(\neg \vThree \wedge \vFour \wedge \neg \vFive)$
and $(\neg \vThree \wedge \neg \vFour \wedge \vFive)$ where they represent the configuration
\setDef \vFour\ (associated to \tabref{vq-conf2}) and \setDef \vFive\ (associated to \tabref{vq-conf3}),
respectively. That is, the configuration \setDef \vFour\ represents the variants that only enable the
feature \vFour\ from \vThree--\vFive, thus, its corresponding feature expression is $(\neg \vThree \wedge \vFour \wedge \neg \vFive)$. That is why we need to pass the closed set of features 
to the auxiliary functions (to generate feature expression corresponding to configurations).

%
In the next step, the tables in \tabref{vq-conf-res} are adjusted so that they all match a certain
relation schema. This is achieved by the $\mathit{fitConfTablesToVsch}$ which gets all the 
tables in \tabref{vq-conf-res} with their associated configurations and the variational relation
schema generated by passing them to the $\mathit{tablesToVsch}$. This is done by
mapping the  $\mathit{fitTableToVsch}$ to all the tables in 
\tabref{vq-conf-res} with their associated configurations. This function simply adds
 attributes of the variational relation schema to the table that do not exists in the table 
 and puts \nul\ as values (indicated by the white space) in the tuples for those attributes. 
%
\tabref{fitting1}--\tabref{fitting3} illustrate the application of 
$\mathit{fitTableToVsch}$ to \tabref{vq-conf1}--\tabref{vq-conf3} and variational relation 
schema $\vRelSch_{\mathit{accum}}$.
%
\begin{table}[!htbp]
%\caption[Example of step two of table accumulation]{Step two of table accumulation applies the 
%$\mathit{fitConfTablesToVsch}$ function to all tables of \tabref{vq-conf-res} and their corresponding 
%configurations and the variational relation schema $\vRelSch_{\mathit{accum}}$.}
%\label{tab:fitting}
%\centering
%\small
%%\footnotesize
%%\scriptsize
%\begin{subtable}[t]{\textwidth}
\centering
\caption[Example of step two of table accumulation]{Result of the $\mathit{fitTableToVsch}$ applied to 
\tabref{vq-conf1} and  variational relation schema $\vRelSch_{\mathit{accum}}$.}
\label{tab:fitting1}
\arrayrulecolor{black}
\begin{tabular} {c | l l l l  }
% {\textcolor{blue}{$\oneof {\vThree, \vFour,\vFive}$} }& {\textcolor{blue}{$\vFour \vee \vFive$}}&  {\textcolor{blue}{$\vFour $}} &  {\textcolor{blue}{$\vFive $}} &  {\textcolor{blue}{$\vFive$}} & {\textcolor{blue}{\texttt{true}}}\\
%\arrayrulecolor{blue}\hdashline
\multirow{2}{*}{$\mathit{result}$}  & \empno & \name & \fname & \lname\\
\arrayrulecolor{black}\cline{2-5} 
& & & & \\
\arrayrulecolor{white}\hline
\end{tabular}
%\end{subtable}
%
%%\medskip
%%\medskip
%\medskip
%\begin{subtable}[t]{\textwidth}
%%\begin{center}
%\centering
%%\tiny
%\caption{Result of the $\mathit{fitTableToVsch}$ applied to 
%\tabref{vq-conf2} and  variational relation schema $\vRelSch_{\mathit{accum}}$.}
%\label{tab:fitting2}
%\arrayrulecolor{black}
%\begin{tabular} {c | l l l l  }
%% {\textcolor{blue}{$\oneof {\vThree, \vFour,\vFive}$} }& {\textcolor{blue}{$\vFour \vee \vFive$}}&  {\textcolor{blue}{$\vFour $}} &  {\textcolor{blue}{$\vFive $}} &  {\textcolor{blue}{$\vFive$}} & {\textcolor{blue}{\texttt{true}}}\\
%%\arrayrulecolor{blue}\hdashline
%\multirow{2}{*}{$\mathit{result}$}  & \empno & \name & \fname & \lname\\
%\arrayrulecolor{black}\cline{2-5}  
%& 12001 & Ulf Hofstetter & & \\
%& 12002 & Luise McFarlan & & \\
%& 12003 & Shir DuCasse & & \\
% &80001 & Nagui Merli & & \\
% & 80002 & Mayuko Meszaros & & \\
% & 80003 & Theirry Viele & & \\
%% &80001 & Nagui Merli & & \\
%% & 80002 & Mayuko Meszaros & & \\
%% & 80003 & Theirry Viele & & \\
%&\ldots & \ldots  & \ldots & \ldots \\
%\arrayrulecolor{white}\hline
%\end{tabular}
%%\end{center}
%\end{subtable}
%
%%\medskip
%%\medskip
%\medskip
%\begin{subtable}[t]{\textwidth}
%%\begin{center}
%\centering
%%\footnotesize
%%\tiny
%\caption{Result of the $\mathit{fitTableToVsch}$ applied to 
%\tabref{vq-conf3} and  variational relation schema $\vRelSch_{\mathit{accum}}$.}
%\label{tab:fitting3}
%\arrayrulecolor{black}
%\begin{tabular} {c | l l l l  }
%% {\textcolor{blue}{$\oneof {\vThree, \vFour,\vFive}$} }& {\textcolor{blue}{$\vFour \vee \vFive$}}&  {\textcolor{blue}{$\vFour $}} &  {\textcolor{blue}{$\vFive $}} &  {\textcolor{blue}{$\vFive$}} & {\textcolor{blue}{\texttt{true}}}\\
%%\arrayrulecolor{blue}\hdashline
%\multirow{2}{*}{$\mathit{result}$}  & \empno & \name & \fname & \lname\\
%\arrayrulecolor{black}\cline{2-5}  
% & 12001 & & Ulf & Hofstetter \\
%& 12002 & & Luise & McFarlan\\
%& 12003 & & Shir & DuCasse\\
% &80001 & & Nagui & Merli\\
% & 80002 & & Mayuko & Meszaros\\
% & 80003 & & Theirry & Viele\\
% & 200001 & & Selwyn & Koshiba \\
% & 200002 & & Bedrich & Markovitch \\
% & 200003 & & Pascal & Benzmuller  \\
%% & 200001 & & Selwyn & Koshiba \\
%% & 200002 & & Bedrich & Markovitch \\
%% & 200003 & & Pascal & Benzmuller  \\
% & \ldots & \ldots & \ldots & \ldots \\
% \arrayrulecolor{white}\hline
%\end{tabular}
%%\end{center}
%\end{subtable}
%
\end{table}


\begin{table}[!htbp]
%\caption[Example of step two of table accumulation]{Step two of table accumulation applies the 
%$\mathit{fitConfTablesToVsch}$ function to all tables of \tabref{vq-conf-res} and their corresponding 
%configurations and the variational relation schema $\vRelSch_{\mathit{accum}}$.}
%\label{tab:fitting}
%\centering
%\small
%%\footnotesize
%%\scriptsize
%\begin{subtable}[t]{\textwidth}
%\centering
%\caption{Result of the $\mathit{fitTableToVsch}$ applied to 
%\tabref{vq-conf1} and  variational relation schema $\vRelSch_{\mathit{accum}}$.}
%\label{tab:fitting1}
%\arrayrulecolor{black}
%\begin{tabular} {c | l l l l  }
%% {\textcolor{blue}{$\oneof {\vThree, \vFour,\vFive}$} }& {\textcolor{blue}{$\vFour \vee \vFive$}}&  {\textcolor{blue}{$\vFour $}} &  {\textcolor{blue}{$\vFive $}} &  {\textcolor{blue}{$\vFive$}} & {\textcolor{blue}{\texttt{true}}}\\
%%\arrayrulecolor{blue}\hdashline
%\multirow{2}{*}{$\mathit{result}$}  & \empno & \name & \fname & \lname\\
%\arrayrulecolor{black}\cline{2-5} 
%& & & & \\
%\arrayrulecolor{white}\hline
%\end{tabular}
%\end{subtable}
%
%%\medskip
%%\medskip
%\medskip
%\begin{subtable}[t]{\textwidth}
%\begin{center}
\centering
%\tiny
\caption[Example of step two of table accumulation]{Result of the $\mathit{fitTableToVsch}$ applied to 
\tabref{vq-conf2} and  variational relation schema $\vRelSch_{\mathit{accum}}$.}
\label{tab:fitting2}
\arrayrulecolor{black}
\begin{tabular} {c | l l l l  }
% {\textcolor{blue}{$\oneof {\vThree, \vFour,\vFive}$} }& {\textcolor{blue}{$\vFour \vee \vFive$}}&  {\textcolor{blue}{$\vFour $}} &  {\textcolor{blue}{$\vFive $}} &  {\textcolor{blue}{$\vFive$}} & {\textcolor{blue}{\texttt{true}}}\\
%\arrayrulecolor{blue}\hdashline
\multirow{2}{*}{$\mathit{result}$}  & \empno & \name & \fname & \lname\\
\arrayrulecolor{black}\cline{2-5}  
& 12001 & Ulf Hofstetter & & \\
& 12002 & Luise McFarlan & & \\
& 12003 & Shir DuCasse & & \\
 &80001 & Nagui Merli & & \\
 & 80002 & Mayuko Meszaros & & \\
 & 80003 & Theirry Viele & & \\
% &80001 & Nagui Merli & & \\
% & 80002 & Mayuko Meszaros & & \\
% & 80003 & Theirry Viele & & \\
&\ldots & \ldots  & \ldots & \ldots \\
\arrayrulecolor{white}\hline
\end{tabular}
%\end{center}
%\end{subtable}
%
%%\medskip
%%\medskip
%\medskip
%\begin{subtable}[t]{\textwidth}
%%\begin{center}
%\centering
%%\footnotesize
%%\tiny
%\caption{Result of the $\mathit{fitTableToVsch}$ applied to 
%\tabref{vq-conf3} and  variational relation schema $\vRelSch_{\mathit{accum}}$.}
%\label{tab:fitting3}
%\arrayrulecolor{black}
%\begin{tabular} {c | l l l l  }
%% {\textcolor{blue}{$\oneof {\vThree, \vFour,\vFive}$} }& {\textcolor{blue}{$\vFour \vee \vFive$}}&  {\textcolor{blue}{$\vFour $}} &  {\textcolor{blue}{$\vFive $}} &  {\textcolor{blue}{$\vFive$}} & {\textcolor{blue}{\texttt{true}}}\\
%%\arrayrulecolor{blue}\hdashline
%\multirow{2}{*}{$\mathit{result}$}  & \empno & \name & \fname & \lname\\
%\arrayrulecolor{black}\cline{2-5}  
% & 12001 & & Ulf & Hofstetter \\
%& 12002 & & Luise & McFarlan\\
%& 12003 & & Shir & DuCasse\\
% &80001 & & Nagui & Merli\\
% & 80002 & & Mayuko & Meszaros\\
% & 80003 & & Theirry & Viele\\
% & 200001 & & Selwyn & Koshiba \\
% & 200002 & & Bedrich & Markovitch \\
% & 200003 & & Pascal & Benzmuller  \\
%% & 200001 & & Selwyn & Koshiba \\
%% & 200002 & & Bedrich & Markovitch \\
%% & 200003 & & Pascal & Benzmuller  \\
% & \ldots & \ldots & \ldots & \ldots \\
% \arrayrulecolor{white}\hline
%\end{tabular}
%%\end{center}
%\end{subtable}
%
\end{table}


\begin{table}[!htbp]
%\caption[Example of step two of table accumulation]{Step two of table accumulation applies the 
%$\mathit{fitConfTablesToVsch}$ function to all tables of \tabref{vq-conf-res} and their corresponding 
%configurations and the variational relation schema $\vRelSch_{\mathit{accum}}$.}
%\label{tab:fitting}
%\centering
%\small
%%\footnotesize
%%\scriptsize
%\begin{subtable}[t]{\textwidth}
%\centering
%\caption{Result of the $\mathit{fitTableToVsch}$ applied to 
%\tabref{vq-conf1} and  variational relation schema $\vRelSch_{\mathit{accum}}$.}
%\label{tab:fitting1}
%\arrayrulecolor{black}
%\begin{tabular} {c | l l l l  }
%% {\textcolor{blue}{$\oneof {\vThree, \vFour,\vFive}$} }& {\textcolor{blue}{$\vFour \vee \vFive$}}&  {\textcolor{blue}{$\vFour $}} &  {\textcolor{blue}{$\vFive $}} &  {\textcolor{blue}{$\vFive$}} & {\textcolor{blue}{\texttt{true}}}\\
%%\arrayrulecolor{blue}\hdashline
%\multirow{2}{*}{$\mathit{result}$}  & \empno & \name & \fname & \lname\\
%\arrayrulecolor{black}\cline{2-5} 
%& & & & \\
%\arrayrulecolor{white}\hline
%\end{tabular}
%\end{subtable}
%
%%\medskip
%%\medskip
%\medskip
%\begin{subtable}[t]{\textwidth}
%%\begin{center}
%\centering
%%\tiny
%\caption{Result of the $\mathit{fitTableToVsch}$ applied to 
%\tabref{vq-conf2} and  variational relation schema $\vRelSch_{\mathit{accum}}$.}
%\label{tab:fitting2}
%\arrayrulecolor{black}
%\begin{tabular} {c | l l l l  }
%% {\textcolor{blue}{$\oneof {\vThree, \vFour,\vFive}$} }& {\textcolor{blue}{$\vFour \vee \vFive$}}&  {\textcolor{blue}{$\vFour $}} &  {\textcolor{blue}{$\vFive $}} &  {\textcolor{blue}{$\vFive$}} & {\textcolor{blue}{\texttt{true}}}\\
%%\arrayrulecolor{blue}\hdashline
%\multirow{2}{*}{$\mathit{result}$}  & \empno & \name & \fname & \lname\\
%\arrayrulecolor{black}\cline{2-5}  
%& 12001 & Ulf Hofstetter & & \\
%& 12002 & Luise McFarlan & & \\
%& 12003 & Shir DuCasse & & \\
% &80001 & Nagui Merli & & \\
% & 80002 & Mayuko Meszaros & & \\
% & 80003 & Theirry Viele & & \\
%% &80001 & Nagui Merli & & \\
%% & 80002 & Mayuko Meszaros & & \\
%% & 80003 & Theirry Viele & & \\
%&\ldots & \ldots  & \ldots & \ldots \\
%\arrayrulecolor{white}\hline
%\end{tabular}
%%\end{center}
%\end{subtable}
%
%%\medskip
%%\medskip
%\medskip
%\begin{subtable}[t]{\textwidth}
%\begin{center}
\centering
%\footnotesize
%\tiny
\caption[Example of step two of table accumulation]{Result of the $\mathit{fitTableToVsch}$ applied to 
\tabref{vq-conf3} and  variational relation schema $\vRelSch_{\mathit{accum}}$.}
\label{tab:fitting3}
\arrayrulecolor{black}
\begin{tabular} {c | l l l l  }
% {\textcolor{blue}{$\oneof {\vThree, \vFour,\vFive}$} }& {\textcolor{blue}{$\vFour \vee \vFive$}}&  {\textcolor{blue}{$\vFour $}} &  {\textcolor{blue}{$\vFive $}} &  {\textcolor{blue}{$\vFive$}} & {\textcolor{blue}{\texttt{true}}}\\
%\arrayrulecolor{blue}\hdashline
\multirow{2}{*}{$\mathit{result}$}  & \empno & \name & \fname & \lname\\
\arrayrulecolor{black}\cline{2-5}  
 & 12001 & & Ulf & Hofstetter \\
& 12002 & & Luise & McFarlan\\
& 12003 & & Shir & DuCasse\\
 &80001 & & Nagui & Merli\\
 & 80002 & & Mayuko & Meszaros\\
 & 80003 & & Theirry & Viele\\
 & 200001 & & Selwyn & Koshiba \\
 & 200002 & & Bedrich & Markovitch \\
 & 200003 & & Pascal & Benzmuller  \\
% & 200001 & & Selwyn & Koshiba \\
% & 200002 & & Bedrich & Markovitch \\
% & 200003 & & Pascal & Benzmuller  \\
 & \ldots & \ldots & \ldots & \ldots \\
 \arrayrulecolor{white}\hline
\end{tabular}
%\end{center}
%\end{subtable}
%
\end{table}


%
Then, the $\mathit{addPresCondToConfContent}$ 
function adds the presence condition attribute and its values 
to relation contents of \tabref{fitting1}--\tabref{fitting3}, resulting in \tabref{pcadded} which illustrates a set of 
relation contents that are separated by the red bold line. Note that since \tabref{fitting1}
does not have any tuples, \tabref{pcadded} does not have any tuples associated with
the variant \setDef \vThree.
%
\begin{table}[!htbp]
\caption[Example of step three of table accumulation]{Step three of table accumulation adds the 
presence condition values to relation contents. The table illustrates a set of relation contents that 
are separated by the red bold line between them. The tuples follow the order of attributes in the
relation schema.}
\label{tab:pcadded}
\centering
\small
%\footnotesize
%\scriptsize
\arrayrulecolor{blue}
\begin{tabular} {c !{\color{black}\vrule} l l l l : l }
%\multirow{2}{*}{$\mathit{result}$}  & \empno & \name & \fname & \lname & \pcatt \\
\multirow{2}{*}{\textcolor{white}{result}} & & & & &  \\
\arrayrulecolor{black}\cline{2-6}
& & & & &  \textcolor{blue}{$\vThree \wedge \neg \vFour \wedge \neg \vFive$}\\
\arrayrulecolor{red}\specialrule{.2em}{.1em}{.1em}
 &80001 & Nagui Merli & & & \textcolor{blue}{$\neg \vThree \wedge \vFour \wedge \neg \vFive$}\\
 & 80002 & Mayuko Meszaros & & & \textcolor{blue}{$\neg \vThree \wedge \vFour \wedge \neg \vFive$}\\
 & 80003 & Theirry Viele & & & \textcolor{blue}{$\neg \vThree \wedge \vFour \wedge \neg \vFive$}\\
&\ldots & \ldots  & \ldots & \ldots & \textcolor{blue}{\ldots}\\
\arrayrulecolor{red}\specialrule{.2em}{.1em}{.1em}
 & 200001 & & Selwyn & Koshiba & \textcolor{blue}{$\neg \vThree \wedge \neg \vFour \wedge \vFive$}\\
 & 200002 & & Bedrich & Markovitch &\textcolor{blue}{$\neg \vThree \wedge \neg \vFour \wedge \vFive$} \\
 & 200003 & & Pascal & Benzmuller &\textcolor{blue}{$\neg \vThree \wedge \neg \vFour \wedge \vFive$} \\
 & \ldots & \ldots & \ldots & \ldots & \textcolor{blue}{\ldots}\\
\arrayrulecolor{white}\hline
\end{tabular}

\end{table}


%
Finally, the $\mathit{mkTable}$ function takes the variational relation schema $\vRelSch_{\mathit{accum}}$
and \tabref{pcadded}. Note that the values in tuples of \tabref{pcadded} follow the order of the
attributes in the variational relation schema. This results in \tabref{mktab} which is equivalent to
the result of $\VVal \vQ_1$ given in \tabref{vq1-res}.
%
\begin{table}
\caption[Example of the final step of table accumulation]{Final step of table accumulation passes the
variational relation schema $\vRelSch_{\mathit{accum}}$ and relation contents in \tabref{pcadded} to the $\mathit{mkTable}$ function.}
\label{tab:mktab}
\centering
%\footnotesize
%\scriptsize
\tiny
\arrayrulecolor{blue}
%!{\color{black}\vrule}
\begin{tabular} {c !{\color{black}\vrule} l l l l : l }
& {\textcolor{blue}{$(\neg \vThree \wedge \vFour \wedge \neg \vFive)$}}&   &  &  & \\
 {\textcolor{blue}{$\oneof {\vThree, \vFour,\vFive}$} }& {\textcolor{blue}{$(\vee(\neg \vThree \wedge \neg \vFour \wedge \vFive)$}}&  {\textcolor{blue}{$\neg \vThree \wedge \vFour \wedge \neg \vFive $}} &  {\textcolor{blue}{$\neg \vThree \wedge \neg \vFour \wedge \vFive $}} &  {\textcolor{blue}{$\neg \vThree \wedge \neg \vFour \wedge \vFive$}} & {\textcolor{blue}{\texttt{true}}}\\
\arrayrulecolor{blue}\hdashline
\multirow{2}{*}{$\mathit{result}$}  & \empno & \name & \fname & \lname & \pcatt \\
\arrayrulecolor{black}\cline{2-6}
& & & & &  \textcolor{blue}{$\vThree \wedge \neg \vFour \wedge \neg \vFive$}\\
%& 12001 & & & & \textcolor{blue}{\vThree}\\
%& 12002 & & & & \textcolor{blue}{\vThree}\\
%& 12003 & & & & \textcolor{blue}{\vThree}\\
 &80001  & Nagui Merli & & & \textcolor{blue}{$\neg \vThree \wedge \vFour \wedge \neg \vFive$}\\
 & 80002 & Mayuko Meszaros & & & \textcolor{blue}{$\neg \vThree \wedge \vFour \wedge \neg \vFive$}\\
 & 80003 & Theirry Viele & & & \textcolor{blue}{$\neg \vThree \wedge \vFour \wedge \neg \vFive$}\\
 & 200001  & & Selwyn & Koshiba & \textcolor{blue}{$\neg \vThree \wedge \neg \vFour \wedge \vFive$} \\
 & 200002  & & Bedrich & Markovitch & \textcolor{blue}{$\neg \vThree \wedge \neg \vFour \wedge \vFive$} \\
 & 200003  & & Pascal & Benzmuller  & \textcolor{blue}{$\neg \vThree \wedge \neg \vFour \wedge \vFive$} \\
 & \ldots  & \ldots & \ldots & \ldots& \textcolor{blue}{\ldots} \\
\arrayrulecolor{white}\hline
\end{tabular}

\end{table}

\end{example}

The second accumulation function
 $\VVal {\mathit{accum}} :  \settype {\bm{(}\vartype \pTabSet\bm{)}} \totype \tabletype$ 
 takes a set of relational tables that are annotated with
a feature expression instead of their attached configuration. \figref{accum2} defines
this function and its auxiliary functions. The auxiliary functions are similar to the ones
defined in \figref{accum2} except that they do not need to generate a feature expression
from a configuration and a set of closed features.
%
%\exref{acc-table-from-group} illustrates the behavior of these auxiliary functions and the second accumulation
%function over the tables in \tabref{vq-conf-res}.

\begin{figure}[ht!]

\textbf{Table accumulation function:}
\begin{alignat*}{1}
\VVal {\mathit{accum}} &:  \settype {\bm{(}\vartype \pTabSet\bm{)}} \totype \tabletype\\
\VVal {\mathit{accum}} \  \mathit{fs} \ \mathit{ts} &= \mathit{mkTable} \ \mathit{vsch} \ \mathit{tables}\\
&\hspace{-40pt}\textit{where }
\mathit{vsch} = \mathit{annotTablesToVsch}  \ \mathit{ts}\\
&\hspace{-6pt} \mathit{tables} = \mathit{addPresCondToVarTables} \ \mathit {fitted}\\
&\hspace{-6pt} \mathit{fitted} = \mathit{fitVarTablesToVsch} \ \mathit{ts} \ \mathit{vsch}
\end{alignat*}


\medskip 
\textbf{Auxiliary functions for table accumulation:}
\footnotesize
\begin{alignat*}{1}
\mathit{annotSchToVsch} &:  \settype {\bm{(}\vartype \pRelSchSet\bm{)}} \totype \vRelSchSet\\
\mathit{annotTablesToVsch} &:  \settype {\bm{(}\vartype \pTabSet\bm{)}} \totype \vRelSchSet\\
%\mathit{fitTableToVsch} &: \pTabSet \totype \vRelSchSet \totype \pTabSet\\
\mathit{fitVarTablesToVsch} &: \settype {\bm{(}\vartype \pTabSet\bm{)}} \totype \vRelSchSet \totype \settype {\bm{(}\vartype \pTabSet\bm{)}}\\
\mathit{addPresCondToVarContent} &:  \vartype \pRelContSet \totype \vRelContSet\\
\mathit{addPresCondToVarTables} &:  \settype {\bm{(}\vartype \pTabSet\bm{)}} \totype \settype \vRelContSet
%\mathit{mkTable} &: \vRelSchSet \totype \settype \vRelContSet \totype \tabletype
\end{alignat*}


\caption[Accumulation function of a set of relational tables annotated with a feature expression into a variational table]{Accumulation function of a set of relational tables annotated with a feature expression into a variational table and its auxiliary functions. The definition uses spaces to pass parameters, e.g., $f \ x = f(x)$ and $f \ x \ y = f(x,y)$.
}
\label{fig:accum2}
\end{figure}



%\begin{example}
%\label{eg:acc-table-from-group}
%\wrrite{write this}
%\end{example}
\subsection{VRA Denotational Semantics }
\label{sec:vradensem}


Now that we have all required parts we define the denotational semantics of 
variational relational algebra using the denotational semantics of relational 
algebra. 
We assume the existence of the function
%The denotational semantics of relational queries 
$\mathit{rqSem} : \pQSet \totype \pInstSet \totype \pTabSet$, which given a plain query and
a plain database, returns a plain table named $\mathit{result}$
according to the standard semantics of plain relational algebra.
%We do not give a formal 
%definition of $\mathit{rqSem}$, however, examples of the semantics
%of a query are given in \tabref{vq-conf-res}.
%
We then define the VRA denotational semantics 
$\mathit{vqSem} : \qSet \totype \vdbInstSet \totype \tabletype$ as the 
accumulation of relational tables resulting from the semantics of its
configured queries over their corresponding configured databases for all 
valid configurations of a variational database. 
%
The $\mathit{mapRQSem}$ function takes a set of relational queries with their attached
configurations and a variational database instance and returns the set of query 
semantics over their configured database
with their attached configurations, that is, it maps $\mathit{rqSem}$ on the 
relational queries over their corresponding relational database.%
\footnote{In the implementation, the closed set of features and valid configurations
of a VDB are contained within, instead of extracting them from the database. However,
we keep the formalization simple and assume that they can also be retrieved from
the VDB.}
%
Finally, the $\mathit{qToConfRelQs}$ takes a well-typed explicitly annotated 
variational query and the set of valid configurations and configures the variational
query for the given configurations and returns the set of configured queries paired 
with their corresponding configuration.


\begin{figure}

\textbf{VRA denotational semantics:}
\begin{alignat*}{1}
\mathit{vqSem} &: \qSet \totype \vdbInstSet \totype \tabletype\\
\mathit{vqSem} \  \vQ \ \vdbInst &= \mathit{accum} \ \mathit{fs} \ \mathit{tabs}\\
&\hspace{-30pt}\textit{where } \mathit{fs} =\mathit{featues} \ \vdbInst\\
&\hspace{2pt} \mathit{rqs} = \mathit{qToConfRelQs} \ \vQ \ (\mathit{validConfigs} \ \vdbInst)\\
&\hspace{2pt} \mathit{tabs} = \mathit{mapRQSem} \ \mathit{rqs} \ \vdbInst
\end{alignat*}


\medskip 
\textbf{Auxiliary functions for VRA denotational semantics:}
%\footnotesize
\begin{alignat*}{1}
\mathit{rqSem} &: \pQSet \totype \pInstSet \totype \pTabSet\\
\mathit{mapRQSem} &: \settype {\typepair \confSet \pQSet} \totype \vdbInstSet \totype \settype {\typepair \confSet \pTabSet}\\
\mathit{features} &: \vdbInstSet \totype \settype \fSet\\
\mathit{validConfigs} &: \vdbInstSet \totype \settype \confSet\\
\mathit{qToConfRelQs} &: \qSet \totype \settype \confSet \totype \settype {\typepair \confSet \pQSet}\\
\end{alignat*}


\caption[VRA denotational semantics]{Denotational semantics of variational relational algebra.
}
\label{fig:densem}
\end{figure}



%\begin{example}
%\label{eg:sem}
%\wrrite{write the damn thing}
%\end{example}


\section{Variation-Minimization Rules}
\label{sec:var-min}

%
%\maybeAdd{add $\vQ_6$ is simplified of  $\VVal \vQ_6$ because of rule application blah blah.}
%\maybeAdd{add example + more rules + point out interesting ones}
%
VRA is flexible since an information need can be represented via multiple
variational queries as demonstrated in \exref{vq-specific} and \exref{vq-same-intent-mult-vars}.
It allows users to incorporate their personal taste and task requirements
into variational queries they write by 
having different levels of variation. For example, consider the explicitly annotated query
\ensuremath{\vQ_6} 
in \secref{constrain}.
%\ensuremath {
\[
\vQ_6 =
\pi_{\optAtt [\vFour \vee \vFive] [\empno], \optAtt [\vFour] [\name], \optAtt [\vFive] [\fname], \optAtt [\vFive] [\lname]  } \left( \chc [\fModel_2] {\empbio, \empRel}\right)
\]
%}.
%\vQ_5 =  \pi_{\optAtt [\vFour \vee \vFive] [\empno], \optAtt [\vFour] [\name], \optAtt [\vFive] [\fname], \optAtt [\vFive] [\lname]  } \empbio}.
%from \exref{vq-specific}. 
To be explicit about the exact query that will be run for 
each variant 
%and knowing that 
%\ensuremath{
%\getPC \empbio = \vThree \vee \vFour \vee \vFive
%},
the query $\vQ_6$'s variation can be \emph{lifted up} by using choices, resulting in the query $\VVVal \vQ_6$.
%\ensuremath{
%\small
\[
\VVVal \vQ_6 = \chc [\vFour] {\pi_{\empno, \name} \empbio, 
\chc [\vFive] {\pi_{\empno, \fname, \lname} \empbio, \emp}} 
\]
%}.
While \ensuremath{\vQ_6} contains less redundancy \ensuremath{\VVVal \vQ_6}
is more comprehensible since the variants are explicitly stated in the dimension of the choice. 
Thus, \emph{supporting multiple levels of variation 
creates a tension between reducing redundancy and maintaining comprehensibility.}

We define \emph{variation minimization} rules in \figref{var-min} that are syntactic and 
preserve the semantics.
% and include 
%interesting ones in \secref{var-min}.
Pushing in variation into a query, i.e., applying rules left-to-right, 
reduces redundancy
% and improves performance
while lifting them up, i.e., applying rules right-to-left, 
makes a query more understandable. 
When applied left-to-right, the rules are terminating since the scope of variation 
%always gets smaller.
monotonically decreases in size.
%
%\revised{
%Additionally, these rules can be used to simplify queries after
%explicitly annotating them with a schema. For example, the first rule in \figref{var-min}
%is used to simplify the query \ensuremath{\constrain [\vSch_2] {\vQ_1}}, introduced in \secref{var-pres},
% which resulted
%in \ensuremath{\vQ_6}.}


\begin{figure}
\textbf{Choice Distributive Rules:}
\begin{alignat*}{1}
\small
%-- f<? l? q?, ? l? q?> ? ? (f<l?, l?>) f<q?, q?>
%\inferrule
%{}
%\chc {\pi_{\vAttList_1} \vQ_1, \pi_{\vAttList_2} \vQ_2 } 
%&\equiv
%\pi_{\chc {\vAttList_1, \vAttList_2}} \chc {\vQ_1, \vQ_2}\\
%-- f<? l? q?, ? l? q?> ? ? ((l??), (l? \^�f )) f<q?, q?>
%\inferrule
%{}
\chc {\pi_{\vAttList_1} \vQ_1, \pi_{\vAttList_2} \vQ_2}
&\equiv
\pi_{\annot \vAttList_1, \annot [\neg \dimMeta] \vAttList_2} \chc {\vQ_1, \vQ_2}\\
%-- f<? c? q?, ? c? q?> ? ? f<c?, c?> f<q?, q?>
%\inferrule
%{}
\chc {\sigma_{\vCond_1} \vQ_1, \sigma_{\vCond_2} \vQ_2} 
&\equiv
\sigma_{\chc {\vCond_1, \vCond_2}} \chc {\vQ_1, \vQ_2}\\
%-- f<q? � q?, q? � q?> ? f<q?, q?> � f<q?, q?>
%\inferrule
%{}
\chc {\vQ_1 \times \vQ_2, \vQ_3 \times \vQ_4}
&\equiv
\chc {\vQ_1, \vQ_3} \times \chc {\vQ_2, \vQ_4}\\
%-- f<q? ?\_c? q?, q? ?\_c? q?> ? f<q?, q?> ?\_(f<c?, c?>) f<q?, q?>
%\inferrule
%{}
\chc {\vQ_1 \Join_{\vCond_1} \vQ_2, \vQ_3 \Join_{\vCond_2} \vQ_4}
&\equiv
\chc {\vQ_1, \vQ_3} \Join_{\chc {\vCond_1, \vCond_2}} \chc {\vQ_2, \vQ_4}\\
%-- f<q? ? q?, q? ? q?> ? f<q?, q?> ? f<q?, q?>
%\inferrule
%{}
\chc {\vQ_1 \circ \vQ_2, \vQ_3 \circ \vQ_4}
&\equiv
\chc {\vQ_1, \vQ_3} \circ \chc {\vQ_2, \vQ_4}
%-- f<q? ? q?, q? ? q?> ? f<q?, q?> ? f<q?, q?>
%\inferrule
%{}
%{-}
\end{alignat*}

\medskip
\textbf{CC and RA Optimization Rules Combined:}
\begin{alignat*}{1}
\small
%-- f<? (c? ? c?) q?, ? (c? ? c?) q?> ? ? (c? ? f<c?, c?>) f<q?, q?>
%\inferrule
%{}
\chc {\sigma_{\vCond_1 \wedge \vCond_2} \vQ_1, \sigma_{\vCond_1 \wedge \vCond_3} \vQ_2}
&\equiv
\sigma_{\vCond_1 \wedge \chc {\vCond_2, \vCond_3}} \chc {\vQ_1, \vQ_2}\\
%-- ? c? (f<? c? q?, ? c? q?>) ? ? (c? ? f<c?, c?>) f<q?, q?>
%\inferrule
%{}
\sigma_{\vCond_1} \chc {\sigma_{\vCond_2} \vQ_1, \sigma_{\vCond_3} \vQ_2}
&\equiv
\sigma_{\vCond_1 \wedge \chc {\vCond_2, \vCond_3}} \chc {\vQ_1, \vQ_2}\\
%-- f<q? ?\_(c? ? c?) q?, q? ?\_(c? ? c?) q?> ? ? (f<c?, c?>) (f<q?, q?> ?\_c? f<q?, q?>)
%\inferrule
%{}
\chc {\vQ_1 \Join_{\vCond_1 \wedge \vCond_2} \vQ_2, \vQ_3 \Join_{\vCond_1 \wedge \vCond_3} \vQ_4}
&\equiv
\sigma_{\chc {\vCond_2, \vCond_3}} \left( \chc {\vQ_1, \vQ_3} \Join_{\vCond_1} \chc {\vQ_2, \vQ_4} \right)
\end{alignat*}

\caption{Some of variation minimization rules.}
\label{fig:var-min}
\end{figure}

\section{Variational Query Language Properties}
\label{sec:vqlprop}

\TODO{prop. show for vra.}



%if time allows have a subsection for properties of the equivalnece of dent sem and ra + accum


%vra
%%typesys
%%vminrule
%vqlprop
\chapter{Variational Database Management System (VDBMS)}
\label{ch:vdbms}

\TODO{vdbms}


\chapter{Related Work}
\label{ch:rw}

\TODO{related work! have to work on this!}


\section{Instances of Variation in Databases}
\label{sec:vardbinstance}


Database researchers have studied several kinds of variation in
both time and space. There is a substantial body of work on \emph{schema
evolution} and \emph{database
migration}~\cite{Prism08Curino,prima08Moon,schEvolUnifyApp,schEvolIssues03Ram},
which corresponds to variation in time. Typically the goal of such work is to
safely migrate existing databases forward to new versions of the schema as it
evolves. 
%
Work on \emph{database versioning}~\cite{datasetVersioning,dbVersioning}
extends this idea to a database's content. In a versioned database, 
%schema and 
content changes can be sent between different instances of a database, similar
to a distributed revision control system.
%
All of this work is different from variational databases because it encodes a
less general notion of variation and does not support querying multiple
versions of the database at once.
%
Work on \emph{data integration} can be viewed as managing variation in
space~\cite{dataIntegBook}. In data integration, the goal is to combine data
from disparate sources and provide a unified interface for querying.
This is different from VDBs, which make differences between variants
explicit. % , which is needed to manage data variation in SPLs.

 The definition of variation is very limited in these problems. Such
 limitation allows for an efficient intelligent solutions, however, it tailors
 their solutions to a specific context and prevents one from using the same
 solution/system in a similar context when variation in time or space appears
 in a database~\cite{schVersioningSurvey95Roddick}. For example, one cannot
 use a data integration system to manage variation in a database used in
 software produced by a SPL.


\subsection{Schema Evolution}
\label{sec:sch-evo}

Current solutions addressing schema evolution rely on
temporal nature of schema evolution. They use timestamps as a 
means to keep track of historical changes either in an external document~\cite{prima08Moon}
or as versions attached to 
databases~\cite{SchEvolRA90McKenzie, schVersioning97Castro, tempSchEvol91Ariav, tsql95Snodgrass}, 
i.e., either approaches fail to incorporate
the timestamps into the database. 
Then, they take one of these approaches:
1) they require the DBA to design a unified schema, map all schema variants
to the unified one, migrate the database variants to the unified schema, and
write queries only on the unified schema~\cite{schEvolUnifyApp},
2) they require the DBA to specify the version for their query and then migrating
all database variants to the queried 
version~\cite{SchEvolRA90McKenzie, schVersioning97Castro, tempSchEvol91Ariav, tsql95Snodgrass},
or 3) they require the user to specify the timestamps for their query and
then reformulate the query for other database variants~\cite{prima08Moon}.


These approaches usually do not
grant users access to old variants of data even if they desire so and it
is messy to keep both different copies of a variant, one with the old schema
and one with the unified schema, since every data addition/update now requires to 
be applied to all copies of the database variant. A better solution
is maintaining a history of the changes applied to the database and the unified schema
as an XML document and providing a language that allows users/developers to choose
the variant they desire~\cite{prima08Moon}. Unfortunately, this is achieved by limiting
the schema evolution to temporal changes, offering a beautifully tailored approach for 
temporal changes, however, resulting in a non-extensible approach for non-temporal changes.

Temporal evolution is tracked by requiring the database to always have a time-related 
attribute in tables. Thus, queries have to specify the time frame for which they are inquiring 
information~\cite{prima08Moon}. 
Now the user can choose a wide enough time frame in their queries to access 
to their desired variant(s). Aside from the detailed mapping of time frames and variants, this
approach requires a query to have one and only one information need, no matter how many
variants it is aiming. That is, if a time frame includes assumingely two variants a user cannot 
write a query that extract two separate information needs for each of them accumulatively 
in one query. Even worse, if this query does not conform to one of the variant's schema
but it conforms to the other one, the query still fails since there is no systematic way to 
identify that the query is ill (does not conform to the schema) for one of the variants. 
These limitations and constraint are the result of ignoring that temporal changes to a 
database is a form of variability.

\subsection{Database Integration}
\label{sec:db-intg}

The need for \emph{data integration} systems was raised by the invention of the Internet and 
the World Wide Web which require quick access to lots of data stored online as well
as the ability to query all of them. These systems need to query disparate data sources
which often have different formats (e.g., some are completely structured data while others are 
either semi-structured or unstructured) and have been developed independently of each 
other~\cite{dataIntegBook}.
%Furthermore, there is no gurantee that they have access to 
%
Thus, work on \emph{data integration} can be viewed as managing database variation in
space at the content, schema, and format level.
%Work on \emph{data integration} and \emph{database versioning} can be viewed as
%managing database variation in space~\cite{dataIntegBook}. In data integration,
%the goal is typically to combine data variation from disparate sources and
%provide a unified interface for querying that data. This is different from
%VDBs, which make differences between variants explicit, which is needed to
%manage data variation in SPLs.
%
Most of these systems fall somewhere on the spectrum of warehousing and 
virtual integration. In the warehousing model, data from each source are loaded
and materialized in a physical database called a warehouse whereas
in the virtual integration model, the data remain in each data source and are 
accessed as needed at query time. The VDB framework falls in the middle of 
the spectrum since all database variants (data sources) are stored in a physical database 
without materialization. 

In database integration systems, a \emph{mediated schema} is defined for the integrated data and each 
data source has a wrapper/extractor that adjusts the schema and data of the
source to the mediated schema. This is done by using schema matching and 
mapping approaches~\cite{Rahm01Survey,Doan05, schMapBook}. 
Our variational database framework skips this
step since the variational schema contains within itself the variational encoding
of the variants which is similar to the role of  wrapper/extractor in database integration
systems. 

Finally, queries are posed in terms of relations in the mediated schema. Then, the
database integration system reformulates and optimizes the query to grab parts
of the data requested by the query from 
the corresponding data sources since all data sources do not necessarily contain 
the requested information in the query. Similarly, our SQL generators reformulate a 
variational query to extract data from corresponding variants. However, variational 
queries allow one to select the database variants (or parts of it) that they want to query
whereas queries in a database integration system do not provide such an option. 
Furthermore, unlike VDBs, database integration systems are not variation-preserving, that is, the 
result has no indication as to the belonging of part of data to a specific variant (data source).



\subsection{Temporal Databases and Database Versioning}
\label{sec:db-ver}

There is a rich body of work on temporal databases which consider both the data model and the query language~\cite{tempDataMng, tempDBSurv, tempDBbook}.
These databases manage data that are temporal in nature, that is, the state of the data at a specific
time is important, such as financial and medical data or record-keeping applications. 
%
Some of these databases use traditional relational databases and extend them to meet their 
needs~\cite{stratum,Teradata, db2}, while others adopt an in-database 
approach~\cite{KaufmannMVFKFM13sigmod}.
%
These databases contain attributes indicating the start and end time, indicating either 
the transaction time or the valid time of the data. Temporal query languages usually extend 
traditional query languages by adding timestamps and conditions over 
timestamps ~\cite{chomicki95,Jensen2009,evalTempLang}.
%
Temporal databases only have variation at their content level. Thus, their model is simpler 
than the VDB framework, however, representing their temporal representation as feature expression
will require many features. Similar to VRA, their query languages can express
the desired time frame for extracting data and additional their query languages can express
interval related queries.


Temporal databases support a linear timeline for a database, however, the rise of collaborative data 
science has led to non-linear time-based changes in a database. 
This resulted in work on database versioning that aims to support curating and analyzing data 
collaboratively~\cite{datahub15cidr}. Inspired by software version control systems such as Git, 
\citet{dbVersioning} introduce \emph{\textsc{OrpheusDB}} which stores metadata information
about the version graph such as 
the version number, its parent(s), checkout time, commit time, comments, etc. 
Using this metadata the shared parts of the database can be broken down at different scales so
that the database for each version can be recreated.
For example, each version could have a table of its own or 
tuples could have a version attribute and so on. 
\textsc{OrpheusDB} supports both git-style version control 
commands and SQL-like queries. Its query language \emph{VQL} can query the data 
as well as the data and their versions. VQL supports a subset of the query language for 
versioning and provenance proposed by~\citet{vqlAndProv}. 
\citet{datasetVersioning} studies the trade-off of storage and recreation cost for 
different compression and optimization methods used to recreate a database version.
%
Similar to temporal databases, database versioning systems also only contain variation
at the content level. However, their query languages can express git-style commands
whereas VRA cannot. 


%As mentioned in \secref{vtab}, database versioning approaches only consider
%content-level variation~\cite{dbVersioning} which is usually used for experimental and
%scientific databases.
%
%Work on \emph{database versioning}~\cite{datasetVersioning,dbVersioning}
%shifts this idea to content level. In a versioned database, 
%%schema
%%and 
%content changes can be sent between different instances of a database,
%similar to a distributed revision control system.
%%
%All of this work is different from variational databases because it typically
%does not require maintaining or querying multiple versions of the database at
%once.
%

%\subsection{Integrity of Databases}
%\label{sec:integ}
%
%In this section, we highlight the importance of 



\section{Instances of Database Variation Resulted from Software Development}
\label{sec:varsoft}

\TODO{SPL. data model. query.}


\section{Variational Research}
\label{sec:varresearch}

In this section, we discuss some of the related variational research and other applications of it. 
%
The representation of variational schemas and variational tables is based on previous
work on variational sets~\cite{EWC13fosd}, which is part of a larger effort
toward developing safe and efficient variational data
structures~\cite{Walk14onward,MMWWK17vamos}. 
%
The representation of variation in variational queries is based on 
formula choice calculus~\cite{Walk13thesis, HW16fosd}.
%
The central motivation of work on
variational data structures is that many applications can benefit from
maintaining and computing with variation at
runtime~\cite{EW11gttse,CEW16ecoop}. Implementing SPL analyses
are an example of such an application, but there are many
more~\cite{Walk14onward}.The ability to maintain and query several
variants of a database at once extends the idea of computing with variation to
relational databases.

VDBMS is not the only system that extends an existing system with variation. 
\citet{Grasley18} expands on interpreters for variational imperative
languages by providing a formal operational semantics for the variational imperative
language VIMP.
\citet{Alkubaish20} investigates the use of 
algebraic effects to resolve the conflict between variation and side effects.
\citet{young20} add variation to SAT solvers and argue that the variational SAT solver
automates the interaction with the incremental solver.



%We introduce
%variational SAT solving, which differs from incremental SAT solving by accepting all related problems as a single variational input
%and returning all results as a single variational output. Our central
%idea is to make explicit the operations of incremental SAT solving, thereby encoding differences between related SAT problems as
%local points of variation. Our approach automates the interaction
%with the incremental solver and enables methods to automatically
%optimize sharing of the input. To evaluate our methods we construct a prototype variational SAT solver and perform an empirical
%analysis on two real-world datasets that applied incremental solvers
%to software evolution scenarios. We show, assuming a variational
%input, that the prototype solver scales better for these problems
%than naive incremental solving while also removing the need to
%track individual results



\fromppr{from vamos}
%In a database-supported SPL, 
%typically a number of strategies are employed to
%accommodate the different information needs of different variants.
%%
%The first is that a different relational database may be \emph{specified and
%created per-variant}, according to the information needs of each
%variant~\cite{marco13featureAdaptSch}. 
%This approach is labor-intensive and difficult to maintain
%since changes need to be propagated across variants manually.
%%
%The second strategy is to define a single \emph{global schema that applies
%to all variants}~\cite{batini86dbSchIntegAnalysis}. 
%This strategy is more efficient to maintain compared to the previous approach
%but is still hard to maintain,
%especially in face of SPL evolution. Due to lack of separation of concerns
%and suboptimal traceability of requirements to database elements~\cite{skrhas09DBIS}
%it is also complex, hard to understand, and unscalable~\cite{slrs12CAiSE}. 
%Additionally, it suffers from design limitation and 
%error-proneness since parts of the schema will be irrelevant to each variant,
%resulting in losing database's integrity constraints~\cite{slrs12CAiSE}.
%%Irrelevant attributes are typically populated by NULL-values, which may later
%%be referenced since it is impossible to check or enforce that queries in each
%%variant use the database in a safe and consistent way.
%%
%The third strategy is to define a \emph{variable data model}~\cite{skrhas09DBIS, 
%slrs12CAiSE, ad11varDataModel} which models a database schema 
%(usually as an Entity-Relation model) with
%annotations of features from SPL to indicate their variable existence. 
%This approach addresses problems of the previous approach, however,
%it does not address the variation that appears in queries and data. 
%Thus, developers have to write the required information need as a
%query encoded as a string per variant. Not only this is labor-some but
%also due to the nature of queries being encoded as strings there is no
%static check to ensure that queries are type correct. 

\begin{comment}
\TODO{organize later. just a summary of approaches recently found.}

--------------------------------------------------------------------------------------
bridging the gap between variability in client application and 
database schema~\cite{skrhas09DBIS}:
- focuses on data model (schema) without really having a database
- shows that traditional modeling techniques aren't sufficient for expressing
a variable schema
- applies SPL methodology to schemas and allows the generation of 
tailor-made schemas
- using a global schema for all variants of the client application 
leads to a gap bw variability in client application and schema
- problems with global: 1) high variability the global schema will be large
and complex, it doesn't scale. 2) the maintainability is reduced b/c
it is difficult to understand and change a global sch that is fixed
for all variants 3) the lack of separation of concerns and thus the 
suboptimal traceability of requirements of database schema elements
complicate developement and evolution which may in turn hinder application
developers to introdcue new variants in the application 
- focuses on separation of concern on the level of conceptual modeling
so that one can trace features not only to the impelementation level
but also to database schema elements. so in the target schema for a 
application variant only the required database schema elements 
should be included
- demonstrates the problem of database schema variability
- they propose two decomposition approaches: 
1) physically: schema elements are stored in sparate files, one file
per feature. to generate a schema for a specific variant fiels corresponding
to the feature selection are composed. this is well-suited for prjs starting
from the scratch, prjs with many alternative features, and prjs where 
third-party developers are involved.
 and 2) virtually: schema elements in ER (Entity Relationship) model
  are annotated with features.
 to generate a variant for a given feature selection, the annotations
 are evaluated and elements that aren't necessary are removed.
 the resulting schema in a the deployed variant is a subset of the 
 global schema. this approach can easily be adopted if there is 
 already an existing database schema b/c it makes minimal changes to 
 the development process and can easily be adopted in an industrial
 scenario.

--------------------------------------------------------------------------------------
building information system variants with tailored database schemas 
using features~\cite{slrs12CAiSE}:
- shows the challenges of tailoring db schemas in spls. 
- provides an approach to treat the client and database part of an infor sys
in the same way using a variable db sch.
- provides a semi-automatic approach of decomposing schemas that takes 
1) a previously used global db sch and 2) client implementation that is 
separated into features.
- in spl software engineers aim at creating tailor-made software systems with
the help of reusable artifact. an spl forms a group of similar software systems
sharing a set of identical and different functions. the reuse of high-quality 
artifacts in spls reduces the effort for maintenance and further development. 
although the use of spls for producing executable program code has been 
researched quite intensively, the impacts on data managemnts and esp db 
schemas are still fragmentary. 
- simply designing a distinct sch for every potential variant manually is 
practically impossible bc the number of potential variants increases exponentially
with the number of features. 
- problems with global sche: 1) increased effort for maintenance and spl evolution
2) design limitation eg no support for alternative features and table partitioning
3) complex and thus hard to understand db sch
4) data integrity problems bc of missing integrity constraints in db sch
5) solutions that do not scale bc of large number of variants
- discusses the relationship between features and db sch elements:
A feature at DB schema level contains all schema elements (relations, 
attributes, and integrity constraints) that this feature on client side needs 
to perform the read and write operations within its specific source code 
on client side. Thus, we tailor the DB schema w.r.t. the requirements of 
the feature on client side. As a result, there is one feature model for the 
whole IS allowing us to easily generate the single variants of the IS.
Note that the relationship between a feature on client and DB schema 
level ensures completeness and minimizes the complexity of the DB
schema variant (see 3.1). Ev- ery schema element that a selected 
feature needs in a variant, is contained in the fea- ture and therefore 
added to the schema variant during composition. Furthermore, the 
schema variant contains no unused schema elements for the same 
reason. Alternatively, the feature models could contain only the additionally
required schema elements. This approach has the benefit that schema 
elements cannot be mapped to multiple features. This can lead to 
conflicting definitions of a schema element in one variant during the 
evolution of the SPL. The drawback of this alternative is, that we have to 
decide to what feature a schema element belongs, especially when they 
are only required by optional features. However, a high number of 
redundancy suggests a refactoring of the client programs source code.

--------------------------------------------------------------------------------------
towards modeling data variability in spls~\cite{ad11varDataModel}:
- intros persistency features, how they can be expressed, and how 
they relate to other features of the spl. 
- show how to derive a so-called variable data model from persistency
features. and annotations provide traceability bw variability of the 
features and the variability in data model.
-In general, two scenarios exist for defining a data model: a centralized design and a decentralized design [22]. In centralized database design, the database model is defined in one step, and as a result one global database model is defined. In decentralized database design, a data model is defined for each user view resulting in a number of data model views. In case a global data model is required, it can be derived via a view integration process where the different segments of the database design are combined to create one global model.
In either case, variability information is associated with the data model to instruct how to derive the possible different data models for the different products of the product line. The data model incorporating variability information is called the variable data model.
There are basically two options for tailoring a variable data model to the needs of each product: the materialized view and the virtual view [23]. A materialized view means that the required data model elements are actually extracted from the variable data model and are stored physically. This allows creating a tailored database for the final product composed of tables materialized by these views. While with the virtual view, a central database exists, and each individual product has its own view (or sets of views) on this common database.
-we call it a variable concept, i.e. its existence is dependent on the selection of the corresponding feature in the product line.
- steps: 
1) derive persistency features from other features in other perspective defined in feature 
assembly modeling (FAM)
2) map persistency features to data concepts
resulting in representing variability in data model


--------------------------------------------------------------------------------------
* uses views to choose the subset of global schema per variant
* view integration or view merging ~\cite{sp94tkde, parsons02jmis, bln86acmcs}:
focuses on consistency checking, covering different, incomplete, and 
overlapping aspects in local view to retrieve one valid global view or sch.
* or view tailoring~\cite{bqr07cmer} aims at generating and composing views to
tailor the data of an underlying database schema for a given context. 
* View-based approaches~\cite{bqr07cmer} generate views on top of the global schema that emulate a schema variant for the client, which may be seen as an annotative approach. Thus, the global schema is still part of every DB schema vari- ant, the approach inherits the problems of the global schema. Unfortunately, the vari- ant?s schema complexity does even increase, because the additional schema elements for the view, emulating the variant, have to be included as well. Furthermore, there is additional effort to generate views when modeling the DB schema. This approach has benefits in data integrity, because the views emulating the schema variants can contain additional integrity constraints, which cannot be included into the global schema. Thus, the expressiveness of the model is also better than in the global schema approach



\end{comment}
%ATW18poly,ATW17dbpl, vldbArXiv

We proposed encoding variation explicitly in database schemas and queries
in~\cite{ATW17dbpl} and proposed applying this idea to database-backed SPLs
in~\cite{ATW18poly}. Our previous work uses slightly different encodings; the
one presented here is the basis of our VDBMS implementation.
% and follows~\cite{vldbArXiv} closely.
This is the first work that provides use cases for VDBs.


The SPL community has a tradition of developing and distributing use cases
to support research on software variation. For example, SPL2go~\cite{SPL2go}
catalogs the source code and variability models of a large number of SPLs.
Additionally, specific projects, such as Apel
et~al.'s~\cite{apel2013strategies} work on SPL verification, often distribute
use cases along with study results.
%
However, there are no existing datasets or use cases that include
corresponding relational databases and queries, despite their ubiquity in
modern software.


Many researchers have recognized the need to manage structural variation in the
databases that SPLs rely on.
%
\citet{ad11varDataModel} argue for modeling data variability as part of a
model-oriented SPL process. Their \emph{variable data models}
% marking some features as ``persistent'' and linking those to 
link features to concepts in a data model so that specialized data models can
be generated for different products.
%
\citet{dbSchVarSPL} address data model variability in the context of
delta-oriented programming. They define delta modules that can incrementally
generate a relational database schema, and so
% in SQL's Data Definition Language. By combining these deltas in various ways
% using delta-oriented programming, they
can be used to generate different schemas for each variant of a SPL.
%
\citet{varMngDBapp} present a tool to manage variation in the schema of a
relational database used by a SPL. Their tool enables
% defining feature models and 
linking features to elements of a schema, then generating different variants of
the schema for different products.
%
% \eric{Eric, to save you time only the following three sentences have been
% added. also, it'd be great if we could limit the related work to one column.}
\citet{slrs12CAiSE} generate a variable database schema from a given global schema and
software configurations by mapping schema element to features.
%
\citet{skrhas09DBIS} emphasize the need for variable database schema in SPLs and
propose two decomposition approaches: (1) \emph{physical} where database sub-schemas
associated with a feature are stored in physical files
and (2) \emph{virtual} where a global entity-relation model of a schema is annotated
with features.
%
All of these approaches address the issue of \emph{structural} database variation
in SPLs and provide a way to derive a schema per variant, 
which is also achievable by configuring a VDB.
%
The work of \citet{varMngDBapp} is most similar to our notion of a
variational schema since it is an annotative approach~\cite{KAK08} that
directly associates schema elements with features. \citet{ad11varDataModel}
is also annotative, but operates at the higher level of a data model that may
only later be realized as a relational database. \citet{dbSchVarSPL} is a
compositional approach~\cite{KAK08} to generating database schemas.
%
None of these approaches consider \emph{content-level} variation, which is
captured by VDBs and observable in our use cases, nor do they consider how
to express queries over databases with structural variation, which is addressed
by our \emph{v-queries}.


While the previous approaches all address data variation in space,
\citet{dbSPLevolve} emphasize that as an SPL evolves over time, so does its
database. Their approach adapts work on database evolution to SPLs, enabling
the safe evolution of all deployed products.
%
% They present a toolkit to address database evolution in SPLs. Their
% approach generates a global evolution script from the local evolution scripts
% by grouping them into a single database operations and executing them
% sequentially. This requires having the old and new schema of a variant to
% generate the delta scripts, which it uses to ensure correct evolution of both
% data and schema at the deployment step. 


Database researchers have studied several kinds of variation in
both time and space. There is a substantial body of work on \emph{schema
evolution} and \emph{database
migration}~\cite{Prism08Curino,prima08Moon,schEvolUnifyApp,schEvolIssues03Ram},
which corresponds to variation in time. Typically the goal of such work is to
safely migrate existing databases forward to new versions of the schema as it
evolves. 
%
Work on \emph{database versioning}~\cite{datasetVersioning,dbVersioning}
extends this idea to a database's content. In a versioned database, 
%schema and 
content changes can be sent between different instances of a database, similar
to a distributed revision control system.
%
All of this work is different from variational databases because it encodes a
less general notion of variation and does not support querying multiple
versions of the database at once.
%
Work on \emph{data integration} can be viewed as managing variation in
space~\cite{dataIntegBook}. In data integration, the goal is to combine data
from disparate sources and provide a unified interface for querying.
This is different from VDBs, which make differences between variants
explicit. % , which is needed to manage data variation in SPLs.

The representation of v-schemas and variational tables is based on previous
work on variational sets~\cite{EWC13fosd}, which is part of a larger effort
toward developing safe and efficient variational data
structures~\cite{Walk14onward,MMWWK17vamos}. The central motivation of work on
variational data structures is that many applications can benefit from
maintaining and computing with variation at
runtime~\cite{EW11gttse,CEW16ecoop}. The ability to maintain and query several
variants of a database at once extends the idea of computing with variation to
relational databases.



% \NOTE{Snipped from intro, need to integrate: code and data
% structures~\cite{Walk14onward,MMWWK17vamos,alkubaish20}, tools used by the
% software~\cite{ywt20splc}}


%
%Although we have focused on variational databases to support SPL development,
%the broader motivation of \emph{effectively computing with variability} is at
%the heart of our work. This is why VDBs support not only structural variation
%but also content-level variation. Also, while variational queries can be
%statically configured in the same way that SPLs typically are, our prototype
%VDBMS implementation also supports directly executing variational queries on
%variational databases to yield variational results.


% The definition of variation is very limited in these problems. Such
% limitation allows for an efficient intelligent solutions, however, it tailors
% their solutions to a specific context and prevents one from using the same
% solution/system in a similar context when variation in time or space appears
% in a database~\cite{schVersioningSurvey95Roddick}. For example, one cannot
% use a data integration system to manage variation in a database used in
% software produced by a SPL.


% Variational databases, originally designed to represent the intrinsic
% variability of databases used in SPL~\cite{ATW18poly}, explicitly account for
% variation in databases by considering a set of features for a database that
% encodes its variability. Unlike other approaches taken in SPL introduced at
% the beginning of this section, it collapses all the database variants into
% one variational database while tracing the variation by annotating elements
% with feature expressions~\cite{ATW17dbpl}. The variational database
% management system allows users to query all database variants, i.e., the
% overall variational database, simultaneously and
% selectively~\cite{vldbArXiv}. It also allows users to deploy the variational
% database (similar to approaches described briefly above) and variational
% queries for a specific variant (while mentioned approaches are unable to
% achieve this).

\section{Conclusion}
\label{sec:con}

Informed by different kinds of variation appearing in databases, we hypothesize that considering 
variation as an orthogonal concern to databases provides benefits to researchers, database administrators, and developers.
To investigate this hypothesis, we planed the following:

\begin{itemize}
\item Activity 1: We have studied various kinds of variation in databases and based on them
we have provided a framework that considers variation as an orthogonal concern of databases~\cite{ATW17dbpl,ATW18poly}.
\item Activity 2: We have used the framework to represent realistic scenarios of variation in databases,
showing the applicability of our framework~\cite{ALW21vamos}.
\item Activity 3: We are implementing and refactoring VDBMS, a database management system for our
framework.
\item Activity 4: We will mechanically prove the properties of our framework.
\end{itemize}

\noindent
The result of these activities will provide the following contributions:

\begin{itemize}
\item The first work to establish a theoretical framework for variational databases that considers variation as an orthogonal concern to databases. (Activity 1 and 4)
\item The first database management system that allows developers and database administrators to interact with variational databases. (Activity 1, 2, and 3)
\item Realistic case studies of variation in databases that can be used in future research on variational data. (Activity 2)
\end{itemize}

The database community has researched different kinds of variation in databases extensively without 
acknowledging that they are instances of the same problem. On the other
hand, the SPL community has realized the need for encoding variation in the data model, however,
they do not go beyond the data model. Our work on RQ1.1 and RQ1.3~\cite{ALW21vamos} has
shown that different instances of variation in databases could interact with each other, thus, 
having a generic encoding of variation in databases is beneficial. 

%\chapter{Introduction}
%I have done some excellent research \cite{matrix}.
%\section{Introduction to the Introduction}
%\begin{figure}[!ht]
%\centering
%\fbox{\huge Box}
%\caption{Go figure.}
%\end{figure}
%
%\chapter{The Body}
%This is the meat.
%\section{Meat}
%We're born meat and we die meat. Meanwhile, we learn (see Algorithm \ref{alg:learning}).
%
%\begin{algorithm}[h]
%\caption{\textsc{Learning}}
%\label{alg:learning}
%\begin{algorithmic}[1]
%\ENSURE{Optimal policy $\mathcal{C}$}
%	\STATE $\mathcal{C} \gets 42$
%	\RETURN $\mathcal{C}$
%\end{algorithmic}
%\end{algorithm}
%
%\chapter{Conclusion}
%Wow, that really was excellent.
%\section{Fin}
%This is the end, my only friend, the end.


\bibliographystyle{plain}
\bibliography{thesis}

\appendix
\chapter{Variational Database Use Cases}
\label{ch:vdbusecase}


Thus far we introduced the variational database framework and variational
queries. However, some natural questions are: ``How feasible is the variational
database framework?'', ``How would an expert generate a VDB and write 
variational queries?'', ``Can a VDB be generated automatically? And if so,
what is required to make this process automated?''. 
%
In this chapter, we describe preliminary work aimed at answering these questions.
Thus, the goal of this chapter is twofold: 
first, to describe how generating a VDB can be made automatic;
second, to guide an expert through both generating a VDB
from a variation scenario when it cannot be done automatically and
writing variational queries that express an expert's information need over
multiple database variants in the variation scenario.
%

%First, we focus on generating a VDB automatically. 
A VDB can be generated automatically when the main variational information and
database variants are available, that is, when the closed set of features, the
feature model, and closed set of database variants are provided.
%
Unfortunately, this information and these encodings are not available 
for us to use in order to evaluate variational databases.
since existing work only focuses on studying a specific 
kind of variation in databases and does not encode variation inside the database,
instead, it addresses the problem with tools that simulate the effect of the
specific kind of variation.
% following are given:
%(1) its closed feature set, (2) its feature model, and (3) its closed set of 
%database variants. \secref{genvdb} describes in detail how a VDB can 
%be generated by providing these inputs and shows the architecture of
%our module that implements automatic generation of VDBs. 
%\point{We describe how we systematically generated two 
%variational databases from real world scenarios where variation appears
%in databases.}\\
%- explain one comes from schema evlution and another from spl.
%Then, we switch gears to cases that a VDB cannot automatically be generated.
Consequently, we describe how we systematically generated two 
variational databases from real world scenarios where variation appears
in databases. We take a scenario where variation over either time or space
exists in the database, use the schema variants to generate the variational schema, 
and attach feature expressions to tables and tuples to populate the VDB with
data for each use case. 

Additionally, variation in software affects not only databases but also how developers and
database administrators interact with databases.
%
Since different software variants have different information needs, developers
must often write and maintain different queries for different software
variants. Moreover, even if a particular information need is similar across
variants, different variants of a query may need to be created and maintained
to account for structural differences in the schema for each variant.
%
Creating and maintaining different queries for each variant is tedious and
error-prone, and potentially even intractable for large and open-ended
configuration spaces, such as most open-source projects~\cite{dbDecay16Stonebraker}.
%Thus, we illustrate how variational information needs can be captured by
%variational queries written in VRA.

%expressive query languages that account
%for variation explicitly and link variation in software and databases to
%queries by using the same feature names and configuration space.
%
Thus, for each use case we present a set of variational queries 
and we illustrate how VRA realizes the
information needs of the different variants of the database and potentially the 
corresponding software systems.
%
It achieves this level of expressiveness by accounting 
for variation explicitly and linking variation in software and databases to
queries by using the same feature names and configuration space.
%
We present only a  sample of the queries,
yet we provide the full query sets online.\footnote{Complete sets of queries in both formats are available at: \url{https://zenodo.org/record/4321921}.} 
%in the link provided in \secref{intro}.
%online and will be linked to throughout this
%section.
%
The full query sets capture all of the information needs described in the
papers that  we base our variation scenarios on. It is important to note that
this makes our query sets potentially biased toward queries containing more
variation points since the focus of the papers is on variational parts of the
system. A complete query set, capturing \emph{all}
information needs for each scenario might contain more plain queries, that is,
queries that perform the same way over all variants.
%
However, we do not believe this bias is harmful for the role the case studies
are intended to serve, namely, motivating and evaluating variational database
systems. For this role, queries that contain variation are more useful than
plain queries, and additional plain queries can likely be more easily generated
if needed.

We distribute the variational queries in two formats: (1) VRA,
% introduced in \secref{background}, 
encoded in the format used by
our VDBMS tool, and (2) plain SQL queries with embedded 
\cpp{ifdef}-annotations to capture variation points.
%
The SQL format provides queries for studying variational data 
independently of VDBMS tool and
will be more immediately useful for other researchers
studying variational data independently of our VDBMS tool, 
but we use VRA in
this thesis for its brevity
because it is much more concise.


We first focus on variation in databases over space, 
\secref{enron-vdb}. \secref{enron-scenario} describes the
variation scenario from \citet{Hall05} that is the basis of this use case including the feature set and
feature model. Then, \secref{enron-vsch} and \secref{enron-pop} describe generating the variational schema for the described
variation scenario and populating the email SPL VDB with Enron email data.
%\footnote{\url{http://www.ahschulz.de/enron-email-data/}} respectively.
Finally, \secref{enron-qs} describes how variational queries capture the information need 
adapted from feature interactions described by \citet{Hall05}.
%
We then switch focus to variation in databases over time,
in \secref{emp-vdb}. Again, \secref{emp-scenario} describes the
evolution of an employee database as the variation scenario from \citet{prima08Moon} 
that is the basis of this use case.
Then, \secref{emp-vsch} and \secref{emp-pop} describe generating the variational 
schema for the described
variation scenario and populating the employee VDB with a well-known employee data set.%
\footnote{\url{https://github.com/datacharmer/test_db}}
Finally, \secref{emp-qs} describes the adapted and adjusted queries from \citet{prima08Moon}. 
%
At the end of this chapter, \secref{usecase-disc}, we discuss the trade offs of using variational databases and
attempt to answer the question: ``Should variation be encoded explicitly in databases?''.

We distribute the VDBs, SQL scripts for generating them, and queries of our
use cases.\footnote{Available at: \url{https://zenodo.org/record/4321921}}
%in a GitHub repository.%
%\footnote{\url{https://github.com/lambda-land/VDBMS/tree/master/usecases}}
%
We distribute the VDBs in both MySQL and Postgres in two forms, one intended
for use with our VDBMS tool, and one intended for more general-purpose research
on variation in databases.
%
% with an embedded  schema representation described in \secref{enron-vsch}, and
% one without the embedded schema for use with our VDBMS tool, in which the
% variational schema is provided separately.
%
We distribute the variational queries as simple \cpp{ifdef}-annotated SQL files
to promote their broad reuse in the design and evaluation of other systems for
managing variational relational data.

%\section{Automatic Generation of VDBs}
\label{sec:genvdb}

\wrrite{implement and then write}

\subsection{Variation in Space: Email SPL Case Study}
\label{sec:enron-vdb}


%In our first case study, we focus on variation that occurs in ``space'', that
%is, where multiple software variants are developed and maintained in parallel.
%In software, variation in space corresponds to a SPL, where many distinct
%variants (products) can be produced from a single shared code base by enabling
%or disabling features. A variety of representations and tools have been
%developed for indicating which code belongs to which feature(s) and supporting
%the process of configuring a SPL to obtain a particular variant.


% E: This is repetitive with the intro, but I think it's OK to remind them here
% of what the current bad solutions are. Plus, we have space to fill. :-)

%Naturally, different variants of a SPL have different information needs. For
%example, an optional feature in the SPL may require a corresponding attribute
%or relation in the database that is not needed by the other features in the
%SPL.
%%
%Currently, there is no good solution to managing the varying information needs
%of different variants at the level of the database.
%%
%One possible solution is to manually maintain a separate database schema for
%each variant of the SPL. This works for some SPLs where the number of products
%is relatively small and the developer has control over the configuration
%process. However, it does not scale to open-source SPLs or other scenarios
%where the number of products is large and/or configuration is out of the
%developer's hands.
%%
%Another possible solution is to use and maintain a single universal schema that
%includes all of the relations and attributes used by any feature in the SPL. In
%this solution, every product will use the same database schema regardless of
%the features that are enabled. This solves the problem of scaling to large
%numbers of products but is dangerous because it means that potentially several
%attributes and relations will be unused in any given product. Unused attributes
%will typically be populated by NULL values, which are a well-known source of
%errors in relational databases~\cite{AliceBook}.


%VDBs solve the problem by allowing the structure of a relational database to
%vary in a synchronous way with the SPL. Attributes and relations may be
%annotated by presence conditions to indicate in which feature(s) those
%attributes and relations are needed.
%%
%An implementation of the VDB model might use a universal schema under-the-hood
%to realize VDBs on top of a standard relational database management system
%(indeed, this is exactly how our prototype VDBMS implementation works), but by
%capturing the variation in the schema explicitly, we can validate (potentially
%variational) queries against the relevant variants of the variational schema to
%statically ensure that no NULL values will be referenced.

In our first case study, we focus on variation in ``space''.
It shows the use of VDB to encode the variational
information needs of a database-backed SPL. We consider an email
SPL which has been used in several previous SPL research projects (e.g.\
\cite{Apel13:SSP,AlHaj19}).
%It develops a variational
%schema that captures the information needs of a SPL based on Hall's
%decomposition of an email system into its component features~\cite{Hall05}. The
%email SPL has been used in several previous SPL research projects (e.g.\
%\cite{Apel13:SSP,AlHaj19}). The variational email database is populated using
%the Enron email dataset, adapted to fit our variational schema~\cite{Shetty04}.
%
%Our first case study demonstrates the use of a VDB to encode the variational
%information needs of a database-backed SPL.
%
Our case study is formed by systematically combining two pre-existing works:
%
\begin{enumerate}
%
\item 
(1) We use Hall's decomposition of an email system into its component
features~\cite{Hall05} as high-level specification of a SPL.
%
\item 
(2) We use the Enron email
dataset\footnote{\url{http://www.ahschulz.de/enron-email-data/}} as 
%a source of
a realistic email database.
%
\end{enumerate}
%
In combining these works, we show how variation in space in an email SPL
requires corresponding variation in a supporting database, how we can link the
variation in the software to variation in the database, and how all of these
variants can be encoded in a single VDB.


\subsubsection{Variation Scenario: An Email SPL}
\label{sec:enron-scenario}


The email SPL consists of the following features from \citet{Hall05}:
%
\begin{itemize}
%[leftmargin=*]
%\itemsep0em
%
\item 
\addressbook: Users can maintain lists of known email addresses with
corresponding aliases, which may be used in place of recipient addresses.
%
\item 
\signature: Messages may be digitally signed and verified using
cryptographic keys.
%
\item 
\encryption: Messages may be encrypted before sending and decrypted upon
receipt using cryptographic keys.
%
\item 
\autoresponder: Users can enable automatically generated email responses
to incoming messages.
%
\item 
\forwardmessages: Users can forward all incoming messages automatically to another
address.
%
\item 
\remailmessage: Users may send messages anonymously.
%
\item 
\filtermessages: Incoming messages can be filtered according to a
provided white list of known sender address suffixes.
%
\item 
\mailhost: A list of known users is maintained and known users may
retrieve messages on demand. Messages sent to unknown users are rejected.
%
\end{itemize}

%\noindent
%
Note that 
Hall's decomposition separates \signature\ and \encryption\ into two
features each (corresponding to signing and verifying, encrypting and
decrypting). Since these pairs of features must always be enabled together
 and
they are so closely conceptually related, 
we 
reduce them to one feature each for simplicity.


The listed features are used in presence conditions within the
v-schema for the email VDB, linking the software variation to
variation in the database.
%
In the email SPL, each feature is optional and independent, resulting in the
  simple feature model $\fModel_\enron = \t$.
% , which is equivalent to \t, 
%given as a feature expression.
%%\eric{should we just consider this feature model as true? that is how 
%%we're implementing it.}
%%
%\begin{align*}
%\fModel_\enron
%  &= \t \vee \addressbook \vee \signature \vee \encryption \\
%  &\quad \vee \autoresponder \vee \forwardmessages \\
%  &\quad \vee \remailmessage \vee \filtermessages \vee \mailhost
%\end{align*}
%
%The feature model $\fModel_\enron$ is used as the root presence condition of
%the variational schema for the email VDB, implicitly applying it to all
%relations, attributes, and tuples in the database.

%For the rest of this section, we study a SPL that 
%generates email systems with different features and demonstrate
%how variation appears in its database in addition to how all variants
%of a database can be encoded in a single variational database. We adapt the
%email system explained in~\cite{} by considering the following features:
%...
%
%We consider the simple feature model of:
%...
%The feature model is part of the variational schema too and is applied
%to all its tables and attributes. 

\subsubsection{Generating V-Schema of the Email SPL VDB}
\label{sec:enron-vsch}

\begin{table}
\caption{Original Enron email dataset schema.}
%\vspace{-8pt}
\label{tab:enron}
\begin{center}
\small
\begin{tabular} {|l|}
%\hline
%\textbf{Enron Schema} \\
\hline 
\employees(\eid, \fname, \lname, \emailid, $\mathit{email2}$, 
%\hspace{40pt}
 $\mathit{email3}$, $\mathit{email4}$, \folder, \status) \\
\messages(\midatt, \sender, \dateatt, \messageid, \subject, \body, \folder)  \\ 
\recipientinfo(\rid, \midatt, \rtype, \rvalue)  \\
\referenceinfo(\rid, \midatt, \reference)  \\
\hline
\end{tabular}
\end{center}
%\vspace{-13pt}
\end{table}


To produce a v-schema for the email VDB, we start from plain schema
of the Enron email dataset shown in \tabref{enron}, then systematically adjust
its schema to align with the information needs of the email SPL described by
\citet{Hall05}.
%
The \employees\ table contains information about the employees of the company
including the employee identification number (\eid), their first name and last
name (\fname\ and \lname), their primary email address (\emailid), alternative
email addresses (e.g.\ $\mathit{email2}$), a path to the folder that contains
their data (\folder), and their last status in the company (\status).
%
The \messages\ table contains information about the email messages
 including
the message ID (\midatt), the sender of the message (\sender), the date
(\dateatt), the internal message ID (\messageid), the subject and body of the
message (\subject\ and \body), and the exact folder of the email (\folder).
% 
The \recipientinfo\ table contains information about the recipient of a message
including the recipient ID (\rid), the message ID (\midatt), the type of the
message (\rtype), and the email address of the recipient (\rvalue).
%
The \referenceinfo\ table contains messages that have been referenced in other
email messages,
for example, in a forwarded message; it contains a
reference-info ID (\rid), the message ID (\midatt), and the entire message
(\reference). 
This table simply backs up the emails.


\begin{table}
\caption{V-schema of the email VDB with feature model
\ensuremath{\fModel_\enron}. 
Presence conditions are colored blue for clarity.
}
%\vspace{-8pt}
\label{tab:enron-vsch}
 \begin{center}
\small
\begin{tabular} {|l|}
%\hline
%\textbf{Variational Schema for Email SPL} \\
\hline 
 % \annot [\vFour] \name
$\employees(\eid, \fname, \lname, \emailid, \folder, \status,$ 
%\\ \hspace{40pt} 
$\annot[\fsignature]{\verificationkey},
  \annot[\fencryption]{\publickey})$\\
%   \hspace{20pt}
$\messages(\midatt, \sender, \dateatt, \messageid, \subject, \body, \folder,$ 
%\\ \hspace{30pt} 
$\issystemnotification,
  \annot[\fencryption]{\isencrypted}$\\
\hspace{37pt} $ ,\annot[\fautoresponder]{\isautoresponse},\annot[\fsignature]{\issigned} $ 
%  \\ \hspace{30pt} 
  $,
  \annot[\fforwardmsg]{\isforwardmsg})$  \\
 $\recipientinfo(\rid, \midatt, \rtype, \rvalue)$ 
  \\ %[1.1ex]
%  \hspace{40pt}
%   \\ %[1.1ex]
%$\recipientinfo(\rid, \midatt, \rtype, \rvalue)$   
${\forwardmsg(\eid, \forwardaddr)}^{\fforwardmsg} $\\
%\hspace{40pt}
${\mailhost(\eid, \username, \mailhost)}^{\fmailhost}$ 
\\
%\hspace{20pt}
% \\ %[1.1ex]
%\referenceinfo(\rid, \midatt, \reference)  \\
  ${\filtermsg(\eid, \suffix)}^{\ffiltermsg} $ \\
%  \hspace{3pt}
%\\ %[1.1ex]
%  \hspace{47pt}
%\\ %[1.1ex]
${\remailmsg(\eid, \pseudonym)}^{\fremailmsg}$  \\
%  \hspace{3pt}
%\hspace{20pt}
%\\ %[1.1ex]
${\automsg(\eid, \subject, \body)}^{\fautoresponder} $ \\
% \hspace{3pt}
%\\ %[1.1ex]
${\alias(\eid, \emailAtt, \nickname)}^{\faddressbook}$\\
%\\ %[1.1ex]
%${\filtermsg(\eid, \suffix)}^{\ffiltermsg} $\\ %[1.1ex]
\hline
\end{tabular}
%\vspace{-13pt}
 \end{center}
\end{table}


From this starting point, we introduce new attributes and relations that are
needed to implement the features in the email SPL. We attach presence
conditions to new attributes and relations corresponding to the features
they are needed to support, which ensure they will \emph{not} be present
in configurations that do not include the relevant features.
%
The resulting v-schema is given in \tabref{enron-vsch}.


%As an example of this process, 
For example, consider the \signature\ feature. In the
software, implementing this feature requires new operations for signing an
email before sending it out and for verifying the signature of a received
email. These new operations suggest new information needs: we need a way to
indicate that a message has been signed, and we need access to each user's
public key to verify those signatures (private keys used to sign a message
would not be stored in the database). These  needs are reflected in
the v-schema by the new attributes \verificationkey\ and \issigned,
added to the relations \employees\ and \messages, respectively. The new
attributes are annotated by the \signature\ presence condition, indicating that
they correspond to the \signature\ feature and are unused in configurations
that exclude this feature.
%
Additionally, several features require adding entirely new relations, e.g.,
%
%For example, 
when the \forwardmsg\ feature is enabled, the system must keep
track of which users have forwarding enabled and the address to forward the messages to.
%should be forwarded to. 
This need is reflected by the new
\forwardmsg\ relation, which is correspondingly annotated by the \forwardmsg\
presence condition.


A main focus of Hall's decomposition~\cite{Hall05} is on the many feature
interactions.
% in the email SPL. 
Several of the features may interact in
undesirable ways if special precautions are not taken. For example, any
combination of the \forwardmsg, \remailmsg, and \autoresponder\ features can
trigger an infinite messaging loop if users configure the features in the wrong
way; preventing this creates an information need to identify auto-generated
emails, which is realized in the variational schema by attributes like
\isforwardmsg\ and \isautoresponse.
%
% occurs between the \signature\ and \fremailmsg\ features: the \fremailmsg\
% feature enables anonymously sending messages by replacing the sender with a
% pseudonym, but this prevents the recipient from being able to verify a signed
% email.
%
% occurs between the \signature\ and \forwardmsg\ features: if Sarah signs a
% message and sends it to Ina, and Ina forwards the message to Philippe, then
% the signature verification operation may incorrectly interpret Ina as the
% sender rather than Sarah and fail to verify the message.

%p: moved the following paragraph to discussion.
%For each feature, we (1) enumerated the operations that must be supported both
%to implement the feature itself and to resolve undesirable feature
%interactions, (2) identified the information needs to implement these
%operations, and (3) extended the variational schema to satisfy these
%information needs.
%
% The changes made to accommodate the features \addressbook, \encryption,
% \autoresponder, \remailmessage, \filtermessages, and \mailhost\ are similar. 
%

% are provided in the
%second author's MS project report~\citet{Li19}.
%
For brevity, we omit some attributes and relations from the original schema
that are  irrelevant to the email SPL as described by Hall, such as the
\referenceinfo\ relation and alternative email addresses. 
%
%p: will ref qiaoran's work somewhere else, if space permits.
%\citet{Li19} provides
%a complete description of adaptations done in the process in her MS project report.
% \t presence condition can be omitted.


%We distribute the variational schema for the email VDB in two formats.
%%
%First, we provide the schema in the encoding used by our prototype VDBMS tool.%
%\footnote{\href{https://github.com/lambda-land/VDBMS/blob/master/usecases/space-emailSPL/schema/EnronSchema.hs}{usecases/space-emailSPL/schema}}
%
%Second, 
In addition to providing the schema in the encoding used by our 
prototype VDBMS tool, we also provide a direct encoding in SQL
%The SQL encoding 
which generates the
%\eric{Eric, I don't like universal here! it may give the wrong idea!
%maybe just say v-schema?}
universal schema for the VDB.
% in either MySQL or Postgres.%
%\footnote{\href{https://github.com/lambda-land/VDBMS/tree/master/usecases/space-emailSPL/database/create}{usecases/space-emailSPL/database/create}}
%
Variation is encoded as an additional relation of the form \vdbpc\ that
captures all of the relevant presence conditions: that of the 
v-schema itself (i.e.\ the feature model), and those of each relation and
attribute.%
%\footnote{\href{https://github.com/lambda-land/VDBMS/tree/master/usecases/space-emailSPL/database/withSchema}{usecases/space-emailSPL/database/withSchema}}
%
The $\mathit{element\_id}$ of the feature model is
$\mathit{variational\_schema}$; the $\mathit{element\_id}$ of a relation \vRel\
is its name \vRel, and of attribute \vAtt\ in relation \vRel\ is $\vRel.\vAtt$.
%
The plain SQL encoding of the v-schema supports the use
of the case studies for research on the effective management of variation in
databases independent of VDBMS.

%feature
%model of the variational schema, a relation \tablespc\ that stores the presence
%condition of each relation, and for each table $R\left(a_1, \cdots, a_n\right)$
%in the variational schema there is a table $t\left(\mathit{attribute\_name},
%\mathit{pres\_cond}\right)$ in the database that stores its attributes'
%presence conditions.

\subsection{Populating the Email SPL VDB}
\label{sec:enron-pop}

The final step to create the email VDB is to populate the database with data
from the Enron email dataset, adapted to fit our variational schema~\cite{Shetty04}.
%
For evaluation purposes, we want the data from the dataset to be distributed
across multiple variants of the VDB. To simulate this, we identified five
plausible configurations of the email SPL, which we divide the data among. The
five configurations of the email SPL 
 we considered are:
%
\begin{itemize}
%
\item 
\emph{basic email}, which includes only basic email functionality
and does not include any of the optional features
 from the SPL.
%
\item 
\emph{enhanced email}, which extends \emph{basic email} by
enabling two of the most commonly used email features, \forwardmessages\ and
\filtermessages.
%
\item 
\emph{privacy-focused email}, which extends \emph{basic email}
with features that focus on privacy, specifically, the
\signature, \encryption, and \remailmessage\ features.
%
\item 
\emph{business email}, which extends \emph{basic email} with
features tailored to an environment where most emails are expected to be among
users within the same business network, specifically, \addressbook,
\signature, \encryption, \autoresponder, and \mailhost.
%
\item 
\emph{premium email}, in which all of the optional features
in the SPL are enabled.
%
\end{itemize}
%
For all variants, any features that are not enabled are disabled. 


The original Enron dataset has 150 employees with 252,759 email messages. 
%
We load this data into the \employees\ and \messages\ tables defined in
\secref{enron-vsch}, initializing all attributes that are not present in the
original dataset to \nul.


For the \employees\ table, we construct five views corresponding to the five
variants of the email system described above. We allocate 30 employees to each
view based on their employee ID, that is, the first 30 employees sorted by
employee ID are associated with the basic email variant, the next 30 with the
enhanced email variant, and so on. The presence condition for each tuple is set
to the conjunction of features enabled in that view.
%
We then modify each of the views of the \employees\ table by adding randomly
generated values for attributes associated with the enabled features; 
e.g., in the view for the privacy-focused variant, we populate the
\verificationkey\ and \publickey\ attributes.
%
Any attribute that is not present in the given tuple due to a conflicting
presence condition will remain \nul. For example, both the \verificationkey\
and \publickey\ attributes remain \nul\ for employees in the enhanced variant
view since the presence condition does not include the corresponding features.


For the \messages\ table, we again create five views corresponding to each of
the variants. Each tuple is added to the view of the variant that contains the
message's sender, which updates the tuple's presence condition accordingly.
%
The \messages\ table also contains several additional attributes corresponding
to optional features, which we populate in a systematic way.
%
We set \issigned\ to \t\ if the message sender has the \signature\ feature
enabled, and we set \isencrypted\ to \t\ if \emph{both} the message sender
and recipient have \encryption\ enabled.
%
We populate the \isforwardmsg, \isautoresponse, and \issystemnotification\
attributes by doing a lightweight analysis of message subjects to determine
whether the email is any of these special kinds of messages; for example, if
the subject begins with ``FWD'', we set the \isforwardmsg\ attribute to \t.
%
If a forward or auto-reply message was sent by a user that does not have the
corresponding feature enabled, we filter it out of the dataset. After
filtering, the \messages\ relation contains 99,727 messages.
%
For each forward or auto-reply message, we also add a tuple with the relevant
information to the new \forwardmsg\ and \automsg\ tables.
%
For employees belonging to database variants that enable \remailmessage,
\autoresponder, \addressbook, or \mailhost\ we randomly generate tuples in the
tables that are specific to each of these features.
%
Finally, the \recipientinfo\ relation is imported directly from the dataset. We
set each tuple's presence condition to a conjunction of the presence conditions
of the sender and recipient.
% \eric{actually I conjuncted them, which now I'm not so sure about it!}


%We provide more detailed instructions for systematically constructing the email
%VDB in both MySQL and PostgreSQL in a wiki page associated with the
%repository.%
%\footnote{\url{https://github.com/lambda-land/VDBMS/wiki/Enron-Email-Database-UseCase-Doc\#steps-take-to-build-vdb-for-enron-case-study}}
%%
We provide SQL scripts to automate the creation of views for each
variant%
%
%\footnote{\href{https://github.com/lambda-land/VDBMS/blob/master/usecases/space-emailSPL/database/build/step1_build_email_variants.sql}{usecases/space-emailSPL/database/build/step1\_build\_email\_variants.sql}}
%%
and to automate the population of these views with tuples from the original
dataset,%
%%
%\footnote{\href{https://github.com/lambda-land/VDBMS/blob/master/usecases/space-emailSPL/database/build/step2_build_email_vdb.sql}{usecases/space-emailSPL/database/build/step2\_build\_email\_vdb.sql}} 
%%
which also sets each tuple's presence condition.
%
%
The resulting database is distributed in two forms, one with the embedded
variational schema which is described in \secref{enron-vsch},%
%
%\footnote{\href{https://github.com/lambda-land/VDBMS/tree/master/usecases/space-emailSPL/database/withSchema}{usecases/space-emailSPL/database/withSchema}}
%
and one without the embedded schema%
%\footnote{\href{https://github.com/lambda-land/VDBMS/tree/master/usecases/space-emailSPL/database/withoutSchema}{usecases/space-emailSPL/database/withoutSchema}}
%
for use with our VDBMS tool in which the variational schema is provided
separately.%
\footnote{Both the scripts and different encodings of the email SPL VDB are
available at: \url{https://zenodo.org/record/4321921}.} 
%
We have tested the email SPL VDB for the properties described in \secref{vdbfprop} 
and all of them hold.

%\point{VDBs are a good fit to encode variation in a database of a 
%database-backed SPL.}
%%variation over space
%
%
%\point{To showcase how a VDB captures different needs arising in SPLs 
%w.r.t. variation in their databases we generate a VDB for an email SPL.}\\
%- where data is coming from\\
%- where spl is coming from
%
%\point{We adopt the features and feature model from the SPL and 
%consider five database variants.}\\
%- describe 5 variants and their config.
%
%\point{Features are incorporated into both the schema and data.}\\
%- feature model\\
%- schema\\
%- data\\


\subsection{Email Query Set}
\label{sec:enron-qs}

%\point{We write queries based on SPL developers needs when different SPL features
%interact with each other~\cite{blah}.}\\
%- give an example.

To produce a set of queries for the email SPL case study, we collected all of
the information needs that we could identify in the description of the email
SPL by \citet{Hall05}. In order to make the information needs more concrete, we
viewed the requirements of the email SPL mostly through the lens of
constructing an email header.
%
An email header includes all of the relevant information needed to send an
email and is used by email systems and clients to ensure that an email is sent to
the right place and interpreted correctly. 
More specifically, the email header
includes the sender and receiver of the email, whether an email is signed and
the location of a signature verification key, whether an email is encrypted and
the location of the corresponding public key, the subject and body of the
email, the mail host it belongs to, whether the email should be filtered,
% the verification status of the email and who has verified the email, 
and so on.
%
Although there is obviously other infrastructure involved, the fundamental
information needs of an email system can be understood by considering how to
construct email headers.
% that ensures the email would get where it needs to go
% and be interpreted correctly on the other end.


Hall's decomposition focuses on enumerating the features of the email SPL and
enumerating the potential interactions of those features.
%
We deduce the information need for each feature by asking: ``what information
is needed to modify the email header in a way that incorporates the new
functionality?''. We deduce the information need for each interaction by
asking: ``what information is needed to modify the email header in a way that
avoids the undesirable feature interaction?''.
%
We can then translate these information needs into queries on the underlying
variational database.


In total, we provide $27$ queries for the email SPL.
%, encoded in both VRA
%format usable by our VDBMS tool and as \cpp{ifdef} annotated SQL, as described
%above.%
%
%\footnote{\href{https://github.com/lambda-land/VDBMS/tree/master/usecases/space-emailSPL/queries}{usecases/space-emailSPL/queries}}
%
This consists of $1$ query for constructing the basic email header, $8$ queries
for realizing the information needs corresponding to each feature, and $18$
queries for realizing the information needs to correctly handle the feature
interactions described by Hall.


We start by presenting the query to assemble the basic email header, \Qbasic.
This corresponds to the information need of a system with no features enabled.
We use $X$ to stand for the specific message ID (\midatt) of the email whose
header we want to construct.
%
%\vspace{-2pt}
\begin{align*}
\Qbasic &= 
%\pi_{\sender,\rvalue,\subject,\body} 
%((\selectRel\midCond\messages)\\
%&\hspace{20pt} \bowtie_{\messages.\midatt = \recipientinfo.\midatt} \recipientinfo)
\projectRel{\sender,\rvalue,\subject,\body} \mrtable \\
\mrtable &\leftarrow (\selectRel\midCond\messages) \bowtie \recipientinfo
%\mrtable &\leftarrow (\selectRel\midCond\messages) \\
%&\qquad \bowtie_{\messages.\midatt = \recipientinfo.\midatt} \recipientinfo
\end{align*}
%\vspace{-13pt}
%
This query extracts the sender, recipient, subject, and body of the email to
populate the header. The projection is applied to an intermediate result
\mrtable\ constructed by joining the \messages\ table with the \recipientinfo\
table on recipient IDs; we reuse this intermediate result also in subsequent
queries.


Taking \Qbasic\ as our starting point, we next construct our set of $8$
\emph{single-feature queries} that capture the information needs specific to
each feature.
%
When a feature is enabled in the SPL, more information is needed to construct
the header of email $X$. For example, if the feature \filtermessages\ is
enabled, then the query \Qfilter\ extends \Qbasic\ with the \suffix\ attribute
used in filtering. This additional information allows the system to filter a
message if its address contains any of the suffixes set by the receiver.
%
%% Since this queries involve features that are optional, these
%% queries are variational and correspond to multiple different plain queries.
%% However, as described in \secref{background}...
%
%\vspace{-2pt}
\begin{align*}
\Qfilter &= 
\pi_{\sender,\rvalue,\suffix,\subject,\body} 
(\mretable \bowtie\filtermsg )\\
%((\employees \\
%&\bowtie_{\rvalue=\emailid}
% \mrtable) \bowtie\filtermsg )
%_{\employees.\eid = \filtermsg.\eid}\filtermsg )\\
% \bowtie_{\rvalue=\emailid}{\employees}\\
%&\projectRel{\sender,\rvalue,\suffix,\subject,\body} \temp \\
%\temp & \leftarrow \mretable\\
%&\qquad \bowtie_{\employees.\eid = \filtermsg.\eid}\filtermsg \\
\mretable &\leftarrow \joinRel{\mrtable}{\employees}{\rvalue=\emailid}
\end{align*}
%\vspace{-1pt}
%
The intermediate result \mretable\ formed by joining \mrtable\ with the
\employees\ relation will be reused in later queries.
%
We can construct a query that retrieves the required header information whether
\filtermessages\ is enabled or not by combining \Qbasic\ and \Qfilter\ in a
choice, as $\Qbf=\chc[\filtermessages]{\Qfilter,\Qbasic}$. 
%
Although we do not show the process in this paper, we can use equivalence laws
from the choice calculus~\cite{EW11tosem,HW16fosd} to factor commonalities out
of choices and reduce redundancy in queries like \Qbf.
% For clarity, we distribute the un-factored form presented here in our case
% studies.
%The other single-feature queries are written similarly.
% to the one shown here.


% \centerline{$\choiceS \forwardmessages {\Qforward} {\Qbasic}$}

As another example of a single-feature query, \Qforward\ captures the
information needs for implementing the \forwardmessages\ feature. It is similar
to the previous queries except that it extracts the \forwardaddr\ from the
\automsg\ table, which is needed to construct the message header for the new
email to be forwarded when email $X$ is received by a user with a \forwardaddr\
set.
%
\begin{align*}
\Qforward &=
\projectRel{\rvalue,\forwardaddr,\subject,\body}{\temp} \\
\temp &\leftarrow \mretable 
%&\qquad 
\bowtie_{\employees.\eid = \forwardmsg.\eid} \automsg
\end{align*}

The other single-feature queries are similar to those shown here.


Besides single-feature queries, we also provide queries that gather information
needed to identify and address the undesirable feature interactions described
by \citet{Hall05}. Out of Hall's 27 feature interactions, we determined 16 of
them to have corresponding information needs related to the database; 2 of the
interactions require 2 separate queries to resolve. Therefore, we define and
provide 18 queries addressing all 16 of the relevant feature interactions.
%
As before, we deduced the information needs through the lens of constructing an
email header; in these cases, the header would correspond to an email produced
after successfully resolving the interaction.
%
However, some interactions can only be detected but not automatically resolved.
In these cases, we constructed a query that would retrieve the relevant
information to detect and report the issue.


One undesirable feature interaction occurs between the \signature\ and
\forwardmessages\ features: if Philippe signs a message and sends it to Sarah,
and Sarah forwards the message to an alternate address Sarah-2, then signature
verification may incorrectly interpret Sarah as the sender rather than Philippe
and fail to verify the message (Hall's interaction \#4).
%
A solution to this interaction is to embed the original sender's verification
information into the email header of the forwarded message so that it can be
used to verify the message, rather than relying solely on the message's
``from'' field.


Below, we show a variational query \Qsf\ that includes four variants
corresponding to whether \signature\ and \forwardmessages\ are enabled or not
independently. The information need for resolving the interaction is satisfied
by the first alternative of the outermost choice with condition
$\signature\wedge\forwardmessages$. The alternatives of the choices nested to
the right satisfy the information needs for when only \signature\ is enabled,
only \forwardmessages\ is enabled, or neither is enabled (\Qbasic). We don't
show the single-feature \Qsig\ query, but it is similar to other
single-feature queries shown above.
%
\begin{align*}
\Qsf &=  \signature\wedge\forwardmessages 
%\\
%&\quad
\langle
  \vPrj[\rvalue,\forwardaddr,\mathit{emp1}.\issigned,\mathit{emp1}.\verificationkey]{\temp}, \\
&\qquad\qquad\qquad\qquad\qquad\qquad\quad\signature\langle\Qsig, 
\chc[\forwardmessages]{\Qforward,\Qbasic} \rangle\rangle \\
\temp &\leftarrow
  ((((\sigma_{\midCond} \messages) 
 \bowtie
%_{\messages.\midatt = \recipientinfo.\midatt}
 \recipientinfo) 
% \\
%& 
\bowtie_{\sender = \mathit{emp1}.\emailid}
    (\vRen[\mathit{emp1}]{\employees})) \\
& \bowtie_{\rvalue = \mathit{emp2}.\emailid}
    (\vRen[\mathit{emp2}]{\employees})) \bowtie \forwardmsg
%\temp &\leftarrow
%  ((((\sigma_{\midCond} \messages) \\
%& \bowtie_{\messages.\midatt = \recipientinfo.\midatt} \recipientinfo) \\
%& \bowtie_{\messages.\sender = \mathit{emp1}.\emailid}
%    (\vRen[\mathit{emp1}]{\employees})) \\
%& \bowtie_{\recipientinfo.\rvalue = \mathit{emp2}.\emailid}
%    (\vRen[\mathit{emp2}]{\employees}))\\
%& \bowtie_{\mathit{emp2}.\eid = \forwardmsg.\eid} \forwardmsg
\end{align*}
%
 The query $\Qsf$ also resolves another consequence of the interaction between
 these two features. This time Sam successfully verifies message $X$ and
 forwards it to Sam2 which changes the header in the system s.t. it states
 message $X$ has been successfully verified, thus, the message could be
 altered by hackers while it is being forwarded (Hall's interaction 27). The
 system can use $\Qsf$ to generate the correct header in this scenario again.

%\begin{comment}
%\noindent
%
Some feature interactions require more than one query to satisfy their
information need
 due to VRA's limitation that values cannot be variational.
For example, assume both \encryption\ and \forwardmessages\ are enabled.
Philippe sends an encrypted email $X$ to Sarah; upon receiving it the message
is decrypted and forwarded it to Sarah-2 (Hall's interaction \#9). This
violates the intention of encrypting the message and the system should warn the
user.
%
Queries \Qef\ and $\Qef'$ satisfy the information need for this interaction
when a message is encrypted or unencrypted, respectively.
%
\begin{align*}
\small
\Qef &= \encryption\wedge\forwardmessages 
%\\
%  &
  \langle
    \projectRel{\rvalue}{(\selectRel{\midCond\wedge\isencrypted}{\messages})},\\
&\qquad\qquad\qquad\qquad\qquad\qquad\qquad    \encryption\langle\Qencrypt, 
 \chc[\forwardmessages]{\Qforward,\Qbasic} \rangle \\
\OB{\Qef'} &= \encryption\wedge\forwardmessages
  \langle \temp, \encryption \langle \Qencrypt, 
%  \\
%  &
%  \qquad 
  \chc[\forwardmessages]{\Qforward,\Qbasic} \rangle \rangle \\
% \end{align*}
% %
% \begin{align*}
\temp &\leftarrow
  \pi_{\rvalue,\forwardaddr,\subject,\body} (\sigma_{\midCond \wedge \neg\isencrypted} \\
  &\qquad\qquad\qquad\qquad\qquad\qquad\qquad  (\mretable \bowtie_{\employees.\eid = \forwardmsg.\eid} \forwardmsg))
\end{align*}

\noindent
%
However, managing feature interactions is not necessarily complicated. Some
interactions simply require projecting more attributes from the corresponding
single-feature queries. For example, assume both \filtermessages\ and
\mailhost\ features are enabled. Philippe sends a message to a non-existant
user in a mailhost that he has filtered. The mailhost generates a non-delivery
notification and sends it to Philippe, but he never receives it since it is filtered out
(Hall's interaction \#26). The system can check the \issystemnotification\
attribute for the $\Qfilter$ query and decide whether to filter a message or
not. Therefore, we can resolve this interaction by extending the single-feature
query for \filtermessages\ to $\Qfilter'$.
%
\begin{align*}
\small
\OB{\Qfilter'} &= 
\projectRel {\sender, \rvalue, \suffix, \issystemnotification, \subject, \body} \temp\\
\temp &\leftarrow 
{\joinRel \mretable
\filtermsg
{\employees . \eid = \filtermsg . \eid }
}
\end{align*}
%\end{comment}
\noindent
%
Overall, for the 18 interaction queries we provide, 12 have 4 variants, 3 have
3 variants, 2 have 2 variants, and 1 has 1 variant.

%%\input{formulas/enronQs}


\section{Variation in Time: Employee Use Case}
\label{sec:emp-vdb}

%\TODO{from old intro}
%The second case study demonstrates the changing information needs of a system
%that varies over time by developing a variational schema corresponding to an
%employee-system evolution scenario described in \citet{prima08Moon}. The
%variational employee database is populated by adapting a large, fabricated
%employee dataset\footnote{\url{https://github.com/datacharmer/test_db}} that
%has been widely used in the databases community.


In our second case study, we focus on variation that occurs in ``time'', that
is, where the software variants are produced sequentially by incrementally
extending and modifying the previous variant in order to accommodate new
features or changing business requirements. Although new variants conceptually
replace older variants, in practice, older variants must often be maintained in
parallel; external dependencies, requirements, and other issues may prevent
clients from updating to the latest version.
%
Variation in software over time directly affects the databases such software
depends on~\cite{dbDecay16Stonebraker}, and dealing with such changes is a
well-studied problem in the database community known as \emph{database
evolution}~\cite{schVersioningSurvey95Roddick}.


Although research on database evolution has produced a variety of solutions for
managing database variation over time, these solutions do not treat variation
as an orthogonal property and so cannot also accommodate variation in space.
The goal of our work on variational databases is not to directly compete with
database evolution solutions for time-only variation scenarios, but rather to
present a more general model of database variation that can accommodate
variation in both time and space, and that integrates with related software via
feature annotations.

We demonstrate variation in time by 
using a VDB to encode an employee database evolution scenario
systematically adapted from
\citet{prima08Moon} and populated by a dataset that is widely used
in databases research.\footnote{\url{https://github.com/datacharmer/test_db}}


\subsection{Variation Scenario: An Evolving Employee Database}
\label{sec:emp-scenario}

\begin{table}
\caption{Evolution of an employee database schema~\cite{prima08Moon}.
%%from \citet{prima08Moon}.
}
\vspace{-8pt}
\label{tab:emp-sch}
\begin{center}
\small
\begin{tabular} {|l|l|}
\hline
\textbf{Version} & \textbf{Schema}\\
% & \multirow{1}{0.3cm}{\textbf{\sI}}&  \\
\hline 
\hline 
% \dashuline{\isstudent}
\multirow{3}{0.3cm}{\vOne} &  \engemp\ (\empno, \name, \hiredate, \titleatt, \deptname) \\
%& &  \multirow{3}{2cm}{hiredate $<$ 1988-01-01}\\
& \othemp\ (\empno, \name, \hiredate, \titleatt, \deptname) \\
%& \sOne &\\
& \job\ (\titleatt, \salary)\\
% & &\\
\hline
\multirow{2}{0.3cm}{\vTwo} & \empacct\ (\empno, \name, \hiredate, \titleatt, \deptname) \\
%& \multirow{2}{0.5cm}{\sTwo}&  \multirow{2}{2cm}{hiredate $<$ 1991-01-01}\\
& \job\ (\titleatt, \salary) \\
%&  &\\
\hline
\multirow{3}{0.3cm}{\vThree} & \empacct\ (\empno, \name, \hiredate, \titleatt, \deptno) \\
%&\multirow{3}{0.5cm}{\sThree}&  \multirow{3}{2cm}{hiredate $<$ 1994-01-01}\\
& \job\ (\titleatt, \salary)\\
% &  &\\
& \dept\ (\deptname, \deptno, \managerno) \\
%& &\\
\hline
\multirow{4}{0.3cm}{\vFour} & \empacct\ (\empno, \hiredate, \titleatt, \deptno) \\
%& \multirow{4}{0.5cm}{\sFour} &  \multirow{3}{2cm}{hiredate $<$ 1997-01-01}\\
& \job\ (\titleatt, \salary) \\
%& & \\
& \dept\ (\deptname, \deptno, \managerno)\\
%& & \\
& \empbio\ (\empno, \sex, \birthdate, \name)\\
%  & & \\
\hline
\multirow{3}{0.3cm}{\vFive} & \empacct\ (\empno, \hiredate, \titleatt, \deptno, \salary) \\
%& \multirow{3}{0.5cm}{\sFive} &  \multirow{3}{2.3cm}{hiredate $<$ 2000-01-28}\\
& \dept\ (\deptname, \deptno, \managerno) \\
%& & \\
& \empbio\ (\empno, \sex, \birthdate, \fname, \lname)\\
% & & \\
\hline
\end{tabular}
\vspace{-5pt}
\end{center}
\end{table}

\TODO{fix caption}

\citet{prima08Moon} describe an evolution scenario in which the schema of a
company's employee management system changes over time, yielding the five
versions of the schema shown in \tabref{emp-sch}.
%
In \vOne, employees are split into two separate relations for
engineer and non-engineer personnel.
%
In \vTwo, these two tables are merged into one relation, \empacct.
%
In \vThree, departments are factored out of the \empacct\ relation and
into a new \dept\ relation to reduce redundancy in the database.
%
In \vFour, the company decides to start collecting more personal
information about their employees and stores all personal information in the
new relation \empbio.
%
Finally, in \vFive, the company decides to decouple salaries from
job titles and instead base salaries on individual employee's qualifications
and performance; this leads to dropping the \job\ relation and adding a new
\salary\ attribute to the \empacct\ relation. This version also separates the
\name\ attribute in \empbio\ into \fname\ and \lname\ attributes.


We associate a feature with each version of the schema, named 
$\vOne\ldots\vFive$.
%
These features are mutually exclusive since only one version of the
schema is valid at a time. This yields the  feature model
$\fModel_\employee$.
%
 Also, note that the feature model represent a restriction on the entire
 database.
%
\begin{align*}
\fModel_\employee
  &=   \left(\vOne\wedge\neg\vTwo\wedge\neg\vThree\wedge\neg\vFour\wedge\neg\vFive\right)\\
  &\quad
  \vee\left(\neg\vOne\wedge\vTwo\wedge\neg\vThree\wedge\neg\vFour\wedge\neg\vFive\right)
%  \\
%  &
  \vee\left(\neg\vOne\wedge\neg\vTwo\wedge\vThree\wedge\neg\vFour\wedge\neg\vFive\right)\\
   &\quad
   \vee\left(\neg\vOne\wedge\neg\vTwo\wedge\neg\vThree\wedge\vFour\wedge\neg\vFive\right)
%  \\
%  &
  \vee\left(\neg\vOne\wedge\neg\vTwo\wedge\neg\vThree\wedge\neg\vFour\wedge\vFive\right)
\end{align*}

%As a reminder, based on the hierarchy of presence conditions, 
%the feature model $\fModel_\employee$ is used as the root presence condition of
%the variational schema for the employee VDB, implicitly applying it to all
%relations, attributes, and tuples in the database.


\subsection{Generating Variational Schema of the Employee VDB}
\label{sec:emp-vsch}

\begin{table}
\caption[short caption]{Employee v-schema with feature model.
 \ensuremath{\dimMeta_\employee}.}
\vspace{-8pt}
\label{tab:emp-vsch}
\begin{center}
\small
\begin{tabular} {|l|l|}
\hline
%\textbf{Variational Schema for Employee Evolution} \\
%\hline 
 % \annot [\vFour] \name
\rule{0pt}{3ex}%
$\engemp(\empno, \name, \hiredate,\titleatt,\deptname)^{\textcolor{blue}
\vOne}$ \\[1.1ex]
$\othemp(\empno, \name, \hiredate, \titleatt, \deptname)^{\textcolor{blue}
\vOne}$ \\[1.1ex]
$\empacct(\empno, \annot [\textcolor{blue}{\vTwo \vee \vThree}] \name,
\hiredate, \titleatt,$ \\
$\qquad\annot[\textcolor{blue} \vTwo]\deptname, \annot
[\textcolor{blue} {\vThree \vee \vFour \vee \vFive}]\deptno,\annot
[\textcolor{blue} \vFive]\salary)^{\textcolor{blue}{\vTwo \vee \vThree \vee
\vFour \vee \vFive}}$ \\[1.1ex]
$\job(\titleatt, \salary)^{\textcolor{blue}{\vTwo \vee \vThree \vee \vFour}}$
\\[1.1ex]
$\dept(\deptname, \deptno, \managerno)^{\textcolor{blue}{\vThree \vee \vFour
\vee \vFive}}$ \\[1.1ex]
$\empbio(\empno, \sex, \birthdate, \annot [\textcolor{blue} \vFour] \name,
\annot[\textcolor{blue}\vFive]{\fname}, \annot[\textcolor{blue}
\vFive]{\lname})^{\textcolor{blue} {\vFour \vee \vFive}}$ \\
\hline
\end{tabular}
\vspace{-12pt}
\end{center}
\end{table}

\TODO{fix caption}

The variational schema for this scenario is given in \tabref{emp-vsch}. It
encodes all five of the schema versions in \tabref{emp-sch} and was
systematically generated by the following process. First, generate a universal
schema from all of the plain schema versions; the universal schema contains
every relation and attribute appearing in any of the five versions. Then,
annotate the attributes and relations in the universal schema according to the
versions they are present in.
%
For example, the \empacct\ relation is present in versions \vTwo--\vFive, so it
will be annotated by the feature expression
$\vTwo\vee\vThree\vee\vFour\vee\vFive$, while the \salary\ attribute within the
\empacct\ relation is present only in version \vFive, so it will be annotated
by simply \vFive.
%
 The overall variational schema will be annotated by the feature model
 $\fModel_\employee$, described in \secref{emp-scenario}.
%
Since the presence conditions of attributes are implicitly conjuncted with the
presence condition of their relation
 that contains them, 
 we can avoid redundant
annotations when an attribute is present in all instances of its parent
relation. For example, the \empbio\ relation is present in $\vFour\vee\vFive$,
and the \birthdate\ attribute is present in the same versions, so we do not
need to redundantly annotate 
 \birthdate.

Similar to the email SPL VDB, we distribute the variational schema for the
employee VDB in two formats:
%
First, we provide the schema in the encoding used by our prototype VDBMS tool.%
%\footnote{\href{https://github.com/lambda-land/VDBMS/blob/master/usecases/time-employee/schema/EmployeeSchema.hs}{usecases/time-employee/schema/EmployeeSchema.hs}}
%
Second, we provide a direct encoding in SQL that generates the universal schema
for the VDB in either MySQL or Postgres.%
%\footnote{\href{https://github.com/lambda-land/VDBMS/tree/master/usecases/time-employee/database/create}{usecases/time-employee/database/create}}
%
The variability of the schema is embedded within the employee VDB%
%\footnote{\href{https://github.com/lambda-land/VDBMS/tree/master/usecases/time-employee/database/withSchema}{usecases/time-employee/database/withSchema}}
%
using the same encoding as described at the end of \secref{enron-vsch}.%
\footnote{All encodings of the employee variational schema are available at: \url{https://zenodo.org/record/4321921}.} 

%\begin{comment}
\subsection{Populating the Employee VDB}
\label{sec:emp-pop}

Finally, we populate the employee VDB using data from the widely used employee
database linked to in this subsection's lede.
%
This database contains information for $240,124$ employees. To simulate the
evolution of the database over time, we divide the employees into five roughly
equal groups based on their hire date within the company. 
For example, the
first group consists of employees hired before $1988-01-01$, while the second
group contains employees hired from $1988-01-01$ to $1991-01-01$.
%
Each group is assumed to have been hired during the lifetime of a particular
version of the database, and is therefore added to that version of the database
and \emph{also} to all subsequent versions of the database. This simulates the
fact that as a database evolves, older records are typically forward propagated
to the new schema~\cite{schVersioningSurvey95Roddick}. Thus, \vFive\ contains
the records for all $240,124$ employees, while older versions will contain
progressively fewer records.
%
The final employee VDB has $954,762$ employee due to this forward propagation,
despite having the same number of employees as the original database.


The schema of the employee database used to populate the employee VDB
 is different from all versions of the
variational schema, yet it includes all required information. Thus,
%To populate
%the VDB, 
we manually mapped data from the original schema onto each version of
the variational schema.
%\end{comment}


We provide SQL scripts of required queries to automatically 
generate the employee VDB.
%\footnote{\href{https://github.com/lambda-land/VDBMS/blob/master/usecases/time-employee/database/build}{usecases/time-employee/database/build}}
We also provide SQL scripts to automate the separation of each group of employees
into views according to their hire date%
%\footnote{\href{https://github.com/lambda-land/VDBMS/blob/master/usecases/time-employee/database/build/step1_chop_employees.sql}{usecases/time-employee/database/build/step1\_chop\_employees.sql}}
%
and populating those views from data in the employee database.%
\footnote{All the scripts are available at: \url{https://zenodo.org/record/4321921}.} 
%\footnote{\href{https://github.com/lambda-land/VDBMS/blob/master/usecases/time-employee/database/build/step2_build_vdb.sql}{usecases/time-employee/database/build/step2\_build\_vdb.sql}}

As for any VDB, if an attribute is not present in any of the variants covered
by a tuple's presence condition, that attribute will be set to NULL in the
tuple. We do this even though the relevant information may be contained in the
original employee database to ensure that we have a consistent VDB. For
example, while inserting tuples into the \vFour\ view of the \empbio\ table, we
always insert NULL values attributes \fname\ and \lname.
%
We also provide the final employee VDB in four flavors: both with and without the
embedded schema, and in both cases, encoded in MySQL and PostgreSQL format.%
\footnote{Both formats are availabe at: \url{https://zenodo.org/record/4321921}.} 
%\footnote{\href{https://github.com/lambda-land/VDBMS/tree/master/usecases/time-employee/database}{usecases/time-employee/database}}
%
We have tested the employee  VDB for the properties described in \secref{vdbfprop} 
and all of them hold.


%\point{VDBs are a good fit to capture schema evolution of a database.}
%%, i.e., its variation over time.}
%
%\point{To showcase how a VDB captures variation over time we encode
%the schema evolution of an employee database as a VDB.}\\
%- data comes from blah. schema evolution comes from blah.\\ 
%- data stat. schema evolution. \\
%
%\point{We consider each version of the schema evolution as a database 
%variant and assign a feature to them.}
%
%\point{We incorporate the features into both the schema and data.}\\
%- how we incorporate features into schema. \\
%- how we incorporate features into tables.
%\rewrite{rewrite tables to your liking! taken from @Q}

\subsection{Employee Query Set}
\label{app:emp-qs}

%\point{We adapt and adjust the queries from~\cite{prima08Moon}.}\\
%- there are two kinds of queries\\
%- give an example of each

For this case study, we have a set of existing plain queries to start from.
\citet{prima08Moon} provides 12 queries to evaluate the Prima schema evolution
system. We adapt these queries to fit our encoding of the employee VDB
described in \secref{emp-vdb}.
%
%We provide the queries in both the VRA format usable by VDBMS and as
%\cpp{ifdef} annotated SQL, as described in \secref{enron-qs}.%
%\footnote{\href{https://github.com/lambda-land/VDBMS/tree/master/usecases/time-employee/queries}{usecases/time-employee/queries}}
%
9 of these queries have one variant, 2 have two variants, and 1 has three
variants. 


Moon's queries are of two types: 6 retrieve data valid on a particular date
(corresponding to \vThree\ in our encoding), while 6 retrieve data valid on or
after that date (\vThree--\vFive\ in our encoding).
%
For example, one query expresses the intent ``return the salary of employee
number $10004$'' at a time corresponding to \vThree, which we encode:
\[
%\centerline{
%\ensuremath{
Q_1 = \vPrj[\oatt{\salary}{\vThree}].}
% Note that the presence condition of the only attribute \salary\ determines the
% presence condition of the resulting table.
%
% In general and for simplicity, the shared part of presence conditions of
%  projected attributes is factored out and applied to 
%  the entire table. Assume the returned table as a result of
%  query has the schema $\left(\oatt {\vAtt_1} {\dimMeta \wedge \dimMeta_1}, 
%  \oatt {\vAtt_2} {\dimMeta \wedge \dimMeta_2} \right)$.
% The shared restriction can be factored out and applied
% to the entire table, i.e., $\left( \oatt {\vAtt_1} {\dimMeta_1},
% \oatt {\vAtt_2} {\dimMeta_2} \right)^\dimMeta$.
%
We encode the same intent, but for all times at or after \vThree\ as follows:
%
\begin{align*}
Q_2 &=  \vThree\vee\vFour\vee\vFive \langle \pi_\salary 
 (\vThree\vee\vFour\langle
% \\
%&
    \joinRelR{(\vSel[\empno=10004]{\empacct})}{\job}
             {}, 
%&\quad\qquad 
{\vSel[\empno=10004]{\empacct}}\rangle), \empRel\rangle
\end{align*}
%
There are a variety of ways we could have encoded both $Q_1$ and $Q_2$.
%
For $Q_1$ we could equivalently have embedded the projection in a choice,
$\chc[\vThree]{\vPrj[\salary]{(\ldots)},\empRel}$, however attaching the presence
condition to the only projected attribute determines the presence condition of
the resulting table and so achieves the same effect.
%
In $Q_2$ we use choices to structure the query since we have to project on a
different intermediate result for \vFive\ than for \vThree\ and \vFour.

% The feature expression $\vThree \vee \vFour \vee \vFive$
% determines the database variants to be inquired. 
% Since the schema of \empacct\ and \job\ tables are the same
% in variants \vThree\ and \vFour\ they both have the same 
% query. Note that one could move the condition 
% $\vThree \vee \vFour \vee \vFive$ to the projected attribute
% which results in $\VVal {\eq_2}$, however, this query is wrong 
% because the last alternative of the choice projects attribute \salary\
% from an empty relation which is incorrect. It is important to 
% understand that the behavior of an empty relation is exactly the
% same as its behavior in relational algebra and one
% should be careful of using it in operations such as projection,
% selection, and join.

% \begin{align*}
% \small
% \VVal {\eq_2} &= 
% \pi_{\oatt \salary {\vThree \vee \vFour \vee \vFive}}\\
% &\hspace{-15pt}
% (\vThree \vee \vFour\langle
% (\sigma_{\empno=10004} \empacct)
% %&\hspace{0pt}
% \bowtie_{\empacct . \titleatt = \job . \titleatt}
% \job\\
% &\hspace{20pt},
% \vFive\langle{\selectRel {\empno=10004} \empacct},
% \empRel
% \rangle
% \rangle)
% \end{align*}

As another example, the following query realizes the intent to ``return the
name of the manager of department d001'' during the time frame of
\vThree--\vFive:
%
%\begin{align*}
%\ensuremath{ 
\[
Q_3 = \vThree\vee\vFour\vee\vFive \langle
  \pi_{\name,\fname,\lname} 
%  }
%  \\
%&
%\centerline{
%\ensuremath{  
%\hspace{10pt}
%\qquad
(\joinRel{\chc[\vThree]{\empacct,\empbio}}{(\vSel[\deptno=\deptOne]{\dept})}
           {\empno=\managerno}),
  \empRel \rangle
\]  
%  }.}
%\end{align*}
%
%\noindent
Note that even though the attributes \name, \fname, and \lname\ are not present
in all three of the variants corresponding to \vThree--\vFive, the VRA encoding
permits omitting presence conditions that can be completely determined by the
presence conditions of the corresponding relations or attributes in the
variational schema. So, $Q_3$ is equivalent to the following query in which the
presence conditions of the attributes from the variational schema are listed
explicitly in the projection:
%
%\begin{align*}
%\ensuremath{
\[
Q_3' = \vThree\vee\vFour\vee\vFive \langle
  \pi_{\oatt{\name}{\vThree\vee\vFour},\oatt{\fname}{\vFive},\oatt{\lname}{\vFive}} 
%  }
%  \\
%&
%\centerline{
%\ensuremath{
  (\joinRel{\chc[\vThree]{\empacct,\empbio}}{(\vSel[\deptno=\deptOne]{\dept})}
           {\empno=\managerno}),
  \empRel \rangle
  \]
%  }.}
%\end{align*}
%
%
% \noindent
Allowing developers to encode variation in v-queries based on their
preference makes VRA more flexible and easy to use. 
Also, v-queries are statically type-checked to ensure that
the variation encoded in them does not conflict the variation encoded
in the v-schema. 

 %
% Finally, we want to briefly illustrate what queries look like in the
% \cpp{ifdef}-annotated SQL format that we distribute as a potentially more
% portable and easy-to-use format for other researchers. Below is the query $Q_3$
% in this format.
% 
% \begin{lstlisting}[language={}]
% #ifdef V3 || V4 || v5
%   #ifdef V3 || V4
% SELECT name
%   #else
% SELECT firstname, lastname
%   #endif
% FROM
%   #ifdef V3
%     empacct
%   #else
%     empbio
%   #endif
% JOIN (SELECT dept WHERE deptno="d001")
%   ON empno=manageno  
% \end{lstlisting}

% \begin{lstlisting}
% #ifdef v3
% SELECT salary
% FROM empacct JOIN job ON empacct.title=job.title
% WHERE empno=10004 
% #endif
% \end{lstlisting}

%\input{formulas/empQs}


\section{Discussion: Should Variation Be Encoded Explicitly in Databases?}
\label{sec:usecase-disc}
In this section we discuss the use cases and our encodings of VDB and variational queries
in the context of the question posed in the title of this paper: \emph{Should
variation be encoded explicitly in databases?}

% \subsection{Expressiveness of Explicit Variation}
% \label{sec:dis:good}

\paragraph{Expressiveness of explicit variation.}
%
The use cases in \chref{vdbusecase} show that by treating
variation as an orthogonal concern and embedding it directly in databases and
queries (via presence conditions and choices), one can encode data variation
scenarios in both time and space.
%
In fact, VDBs and variational queries are \emph{maximally expressive} in the sense that
any set of plain relational databases can be encoded as a single VDB and any
set of plain queries over the variants of a VDB can be encoded as a variational query.%
%
\footnote{The expressiveness of VDBs and variational queries can be proved by
construction. For VDBs, one can simply take the union of all relations,
attributes, and tuples across all variants, then attach presence conditions
corresponding to which variants each is present in. For variational queries, all variants
can be organized under a tree of choices that similarly organizes the variants
in the appropriate way.}


The expressiveness of our approach is its main advantage over other ways to
manage database variation. When working with a form of variation that already
has its own specialized solution (e.g.\ schema evolution, data integration),
the expressiveness of explicit variation is probably not worth the additional
complexity.
%
The expressiveness of explicit variation is most useful when working with a
form of variation that is not well supported (e.g.\ query-level variation in
SPLs), or when combining multiple forms of variation in one database (e.g.\
during SPL evolution).


We expect that ill-supported forms of variation are common in industry and
justify the expressiveness of explicit variation. For example, the following is
a scenario we recently discussed with an industry contact:
%
A software company develops software for different networking companies and
analyzes data from its clients to advise them accordingly. 
%
The company records information from each of its clients' networks in databases
customized to the particular hardware, operating systems, etc.\ that each
client uses.
%
The company analysts need to query information from all clients who agreed to
share their information, but the same information need will be represented
differently for each client.
%
This problem is essentially a combination of the SPL variation problem (the
company develops and maintains many databases that vary in structure and
content) and the data integration problem (querying over many databases that
vary in structure and content). However, neither the existing solutions from
the SPL community nor database integration address both sides of the problem.
%
Currently the company manually maintains variant schemas and queries, but this
does not take advantage of sharing and is a major maintenance challenge.
% , for the reasons SPL researchers are familiar with.
%
With a database encoding that supports explicit variation in schemas, content,
and queries, the company could maintain a single variational database that can
be configured for each client, import shared data into a VDB, and write
variational queries over the VDB to analyze the data, significantly reducing redundancy
across clients.
%
% While this scenario uses all of the features of VDBMS, we expect one can
% still benefit by using only a subset of its features.


% \subsection{Complexity of Explicit Variation}
% \label{sec:dis:bad}

\paragraph{Complexity of explicit variation.}
%
The generality of explicit variation comes at the cost of increased complexity.
The complexity introduced by presence conditions and choices is similar to the
complexity introduced by variation annotations in annotative approaches to SPL
implementation~\cite{KAK08}. There is widespread acknowledgment that
unrestricted use of variation annotations, such as the C Preprocessor's \cpp{ifdef}-notation~\cite{cpp}, makes software difficult to understand~\cite{LWE11vl} and
is error prone~\cite{FMKPA:SPLC16}.
%
However, so-called \emph{disciplined} use of variation annotations, where
annotations are used in a way that is consistent with the object language
syntax of variants, may suffer less from such issues~\cite{LKA:AOSD11}. In
VDBs, and in the VRA notation for variational queries, annotations are disciplined since
presence conditions and choices are integrated into the existing syntax of
relational database schemas and relational algebra.
%
Note that annotation discipline is not enforced in the \cpp{ifdef}-annotated SQL
notation that we use to distribute the variational queries associated with our use cases.


Subjectively, the development of our use cases suggests that the impact
of variation annotations on understandability is moderate for variational schemas and
VDBs, and significant for variational queries written in VRA, despite the fact that such
annotations are disciplined.
%
% Maybe cut this chunk:
%
 That is, we believe that presence conditions make clear the structural and
 content variation in our example VDBs without significantly impacting the
 understandability of the overall structure and content of the variant
 databases. However, the understandability of variational queries do seem to be
 significantly impacted by the use of presence conditions and choices, despite
 the fact that their use is disciplined in the VRA notation.


It is possible that a more restrictive and/or coarse-grained form of variation
in variational queries would make them easier to understand at the cost of increased
redundancy and (potentially) reduced expressiveness.
%
This tradeoff is one we already made when considering how to encode variation
in the \emph{content} of a VDB. Specifically, we do not support cell-level
variation in a VDB (e.g.\ choices within individual cells). This does not
reduce the expressiveness of content variation in VDBs since cell-level
variation can be simulated by row variation, but it does increase redundancy
since all non-varied cells in the row must be duplicated.
%
Similarly, variation in queries could be restricted to expression-level
choices, with no choices or annotations in conditions or attribute lists. This
would likely make understanding individual query variants easier at the cost of
increasing redundancy among the alternatives of each choice.


Alternatively, the understandability of variational queries could be improved through
tooling, for example, using background colors~\cite{feigenspan2013}, virtual
separation of concerns~\cite{KA09}, or view-based
editing~\cite{WO14gpce,SBWW16icsme}.
%
Future work should validate our subjective assessment of the understandability
VDBs and variational queries, and explore techniques for improving this concern.


% The additional expressiveness of a database and query language allowing users
% to express the variation appearing in their specific use case is essential to
% databases and users information needs. 
% %
% If you are dealing with a kind of variation that has its own specialized
% solution you probably prefer to use that, e.g., you have a database that
% evolves over time. However, if you are dealing with a kind of variation or
% combination of different kinds of variations in your database that does not
% have a specialized solution or system to it you want to be able to configure a
% generic database to your variational need instead of developing a database
% system for your specific use from scratch. That is basically what VDB does: it
% provides users with a generic framework that allows them to configure it to
% their need and benefit from its capabilities that matches their need. 

% \edit{
% The following scenario resulted from our conversation with industry.
% Consider a software company that develops software for 
% different networking companies and analyzes data from its clients to 
% advise them accordingly. 
% %
% This company uses a database to store information that it extracts from 
% each of its clients network, however, since different clients use different 
% hardware and operating systems it needs to store different information
% for each client. 
% %
% At the same time, the company analyzers need to query information
% from all clients who agreed to sharing their information. This results
% in a huge problem since each client has its own database and schema,
% thus, analyzers cannot put all clients databases in a single database. 
% %
% Currently and in practice, developers fix a schema and force all their data
% into that schema. This approach is messy and inefficient. It also creates
% lots of manual burden for developers and DBAs. 
% %
% However, by using VDB the developers can avoid lots of unnecessary work since
% }
% 
% \begin{itemize}
% \item \edit{A VDB can be configured for this scenario by creating the feature space and the v-schema.}
% \item \edit{Individual databases can be deployed for each client.}
% \item \edit{The VDB that contains information from all clients can be queried using VRA.
% These queries allow analyzers to compare different networks (variants) in addition to allowing
% developers to write v-queries in their programs instead of manually rewriting queries for each client
% such that it matches the correct schema. }
% \item \edit{Clients can still query their individual database using SQL. So they do not need to learn a new language.}
% \item \edit{Developers can express properties of each (or some) variant(s).}
% \end{itemize}
% 
% \edit{
% Note that this scenario uses all that VDB has to offer, however, one can also benefit
% from some of VDB's features and not all of them. 
% }


\paragraph{Analyzability of explicit variation.}
%
The relationship of our work to alternative approaches can be viewed through
the lens of annotative vs.\ compositional variation, familiar to the SPL
community~\cite{KAK08}.
%
VDBs and variational queries rely on generic annotations embedded directly in schemas and
queries, respectively, while approaches from the databases community often
express variation through separate artifacts, such as views~\cite{bancilhon81}.
%
Annotative vs.\ compositional representations often exhibit the same tradeoff
between expressiveness and complexity described above: annotative variation
tends to be general and expressive, while compositional variation tends to be
more restrictive but support modular reasoning~\cite{KAK08}.
%
Traditionally, another advantage of compositional approaches is that they are
more analyzable thanks to the ability to analyze components separately (i.e.\
\emph{feature-based} analysis~\citep{Thuem14}), a benefit shared by database
views.
%
However, in the last decade there has been a significant amount of work in the
SPL community to improve the analyzability of annotative variation by analyzing
whole variational artifacts directly (i.e.\ \emph{family-based}
analysis~\cite{Thuem14}). 
%
Although not presented here, we build directly on this body of work, especially
work on variational typing~\cite{CEW12icfp,CEW14toplas}, to enable efficiently
checking variational queries against all variants of a VDB, among other properties.
%
Thus, the increased complexity of explicit variation annotations does not
prevent us from verifying its correctness.

% Historically, a disadvantage of annotative variation is that it was impossible
% to ensure safety properties for all variants of a variational program since the
% number of variants grows exponentially and such properties could only be
% checked variant-by-variant (i.e.\ via product-based analysis~\cite{Thuem14}).
%
% This was in contrast to some compositional approaches to
% variation~\cite{KAK08}, which supported separately analyzing components and
% their compositions (i.e.\ feature-based analysis~\cite{Thuem14}). However,
% compositional approaches typically have the drawback of being less general than
% annotative ones, supporting only certain kinds and often courser-grained
% variability~\cite{KAK08}.


%As can be seen in \secref{db} and \secref{q}, treating variability as an
%orthogonal concern and encoding it directly in databases and their queries 
%
%It is general in the sense
%that any set of variant databases can be encoded as a VDB, and it enables
%directly associating variation in the databases to variation in software.
%
%Variational typing techniques developed in the SPL community.

%\textbf{Is it a good idea to explicitly encode variability in databases?}

\begin{comment}

\textbf{What is the role of the use cases in advancing variation research?}
%
They  illustrate the process of explicitly encoding
variation in the database and queries in \secref{db} and \secref{q}, respectively. 
%
The first step 
is recognizing features involved in the variational scenario. 
A feature is a trait that is essential for convey some information
need. 
%
Thus, for each feature, one should (1) enumerate the operations that must be supported both
to implement the feature itself and to resolve undesirable feature
conflicts, (2) identified the information needs to implement these
operations, and (3) map features to database elements accordingly.
%extended the variational schema to satisfy these
%information needs.
%
This process outlines generating the variational schema for a variational
scenario as show cased for our \edit{use cases} in 
\secref{emp-vsch} and \secref{enron-vsch}.
%
%
It is easy to generate a VDB from a set of 
database variants with their corresponding configuration.
However, we generated our VDBs from scratch, that is,
we did not have any database variants and their corresponding
configuration. \edit{Our attempts at generating the use cases are 
a sample of the burden that developers and DBAs have undergo
in order to manage variability in a database that cannot represent
variation, however, our attempts result in building a database that
can represent variation and thus reduces any future burden as 
compared to a traditional database where DBAs have to deal with 
variation constantly.}

\end{comment}


%\textbf{How can the introduced case studies help evaluating VDBMS?}
%what can these case studies be used for to evaluate VDBs?
%--> the merit of using them to evaluate our tool?
%     what are the question we intend to answer with our tool?
%Studying the cases helped us understand the flow of
%data and the query through VDBMS better and allowed us to refactor
%our data types to achieve a more useable and understandable encoding. 
%%
%We are using the introduced \edit{use cases} to debug and evaluate the
%performance of different approaches implemented in VDBMS 
%for running a variational query on a VDB. Profiling the result
%of said evaluation will help us recognize where we can optimize
%our system. 
%%
%Additionally, we expected our language to be more useable, readable, and
%easier to understand. However, we realize that there is a buy-in cost
%for programmers to pay when they first learn the language. Furthermore,
%the variation in queries may get complicated, yet the programmer can 
%still express their information need the way they are most comfortable
%since variability flows in different parts of a query as shown in \secref{emp-qs}, 
%that is, the overall query
%can be wrapped in a choice, attributes can be projected variationally and
%conditions can be encoded variationally. 
%Still, we believe that the expressiveness of our
%language allows for type checking queries (with variation encoded in them)
%even though it adds more complexity 
%to the language.

%\textbf{How can other researchers benefit from the introduced case studies?}
%
%
%The \edit{use cases} provide an accessible example of how one can encode different 
%kinds of variability in a database using a VDB. This allows one to have a 
%starting point for attacking a new kind of variability appearing in databases,
%e.g., one could have used the employee \edit{use case} to attack database 
%evolution without developing an entirely new system. For example, 
%we suspect that VDBs can be used to also model database 
%versioning~\cite{datasetVersioning,dbVersioning} and
%data provenance~\cite{bt07sigmod}. Such applications can be pursued in future.
%%
%Additionally, the distribution of our databases and queries in a generic
%manner allows others to parse them easily to fit their research.
% %
%Furthermore, these \edit{use cases} can be used to emulate the encoding of 
% variation within other sorts of data such as spreadsheets.
 
 
%\textbf{What are the lessons learned while generating these case studies?}
%do they illustrate common challenges/problems when handling variational data?
%
%Beyond the use of these studies for development of our system,
%%The case studies 
%they illustrate that the most critical step of
%generating a VDB is recognizing the features and configurations
%in order to correctly encode variability of each variant.
%
%Encoding variation explicitly in both the database and the query
%provides developers with tracing variation and connecting
%the variation in data to variation in software development.
%For example,
%the ability to check properties over data as mentioned in
%\secref{prop} is really useful for debugging the
%human mistakes. Similarly, the type system ensures that 
%queries do not contain undesired behavior. 
%%
%Thus, we believe that explicitly
%encoding variability in the data and the queries gives developers the 
%power of testing properties that otherwise would not be possible. 
%This is arguably the strongest advantage of VDBs. Recently
%lots of attempts have been done
%on statically analysis, theorem proving, model checking, and program verification
%of variable programs and 
%variable structure~\cite{brkts20vamos, bks11fvoos, ldl07jss, tmbhvs14}, especially 
%for safety-critical systems,
%\eric{Eric, feel free to add other related work to support this argument.}
% and since databases are part of 
%software system the ability to analyze, check, and verify 
%properties over variational databases provides programmers
%with a seamless framework where they can verify software
%and its artifacts.
%


%%\textbf{What are the future research directions?}
%The \edit{use cases} can be used to develop further research directions
%to make programming with variational databases easier and safer.
%%
%%The first direct future work following these case studies is generating
%%a VDB automatically from a set of databases with their corresponding
%%configuration. 
%%\parisa{remove this if we have a semi-auto gen vdb approach.}
%%
%%considering the new complexity and expressiveness added 
%%to the language (explicit encoding of variability), 
%For example, one can extend the query language with holes and even suggestions for 
%the holes would make programmers job easier in both writing queries
%and debugging them.
%%
%It will be very interesting to see the application of VDBs and 
%variational queries in software testing.
%
%%\end{comment}


%\chapter{Proof of Variation-Preserving}
%\label{sec:proof}

\end{document}

\begin{figure}
%[ht]

\textbf{Relational database objects:}
\begin{syntax}
\synDef \vAtt \attnametype &&&\textit{Attribute Name}\\
\synDef \vRel \relnametype &&& \textit{Relation Name}\\
\synDef {\underline{\vTuple}} { \tupletype} &\eqq& {(\vi v \numAtts)} & \textit{Tuple}\\
\synDef {\underline{\vAttList}} {\settype \attnametype} &\eqq& 
\setDef {\vAtt_1,  \vAtt_2, \ldots,  \vAtt_\numAtts} & \textit{Set of Attributes}\\
\synDef \pRelSch \pRelSchSet &\eqq& \vRel\left(\underline{\vAttList}\right) & \textit{Relation Schema}\\
\synDef \pRelCont \pRelContSet &\eqq& \setDef {\pTuple_1, \pTuple_2, \ldots, \pTuple_k} & \textit{Relation Content}\\
\synDef \pTab \pTabSet &\eqq& \left(\pRelSch, \pRelCont\right) & \textit{Table}\\
\synDef \pSch \pSchSet  &\eqq& \vSchDef & \textit{Schema}\\
%\synDef \pInst \pInstSet &\eqq& \setDef {(\pRelSch_1,\pRelCont_1), (\pRelSch_2,\pRelCont_2), \ldots, (\pRelSch_n,\pRelCont_n) } & \textit{Database Instance}\\
\synDef \pInst \pInstSet &\eqq& \setDef {\pTab_1, \pTab_2, \ldots, \pTab_n } & \textit{Database Instance}\\
\end{syntax}

\medskip
\textbf{Relational database type synonyms:}
\begin{alignat*}{1}
\pRelContSet &= {\settype \tupletype}\\
\pTabSet &= \typepair{\pRelSchSet} {\pRelContSet}\\
\pSchSet &= \settype {\underline \relschtype}\\
{\underline \dbinsttype} &= \settype {\typepair \pRelSchSet {\underline \relconttype}}
\end{alignat*}
\caption[Relational databases definition]{Relational database objects and type synonyms.}
\label{fig:rdb-def}
\end{figure} 

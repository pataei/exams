\begin{figure}
%\textbf{Configuration selection semantics of \vqsTxt:}
\begin{alignat*}{1}
\qGroup . &: \qSet \totype \vartype \pQSet\\
%
\qGroup \vRel &= \annot [\t] \pRel\\
\qGroup {\vSel \vQ}  &=  
\setDef {\annot [\dimMeta \wedge \dimMeta_\vCond] {\left(\sigma_{\pCond} \pQ\right)} \myOR
\annot \pQ \in \qGroup \vQ, \annot [\dimMeta_\vCond] \pCond \in \cGroup}
\\
%
\qGroup {\vPrj [\vAttList] \vQ} &= 
\setDef {\annot [\dimMeta \wedge \dimMeta_\vAttList] {\left(\pi_{\pAttList} \pQ \right)} \myOR
\annot \pQ \in \qGroup \vQ, \annot [\dimMeta_\vAttList] \pAttList \in \aGroup}
\\
%
\qGroup {{\vQ_1} \times {\vQ_2}} &= 
\setDef {\annot [\dimMeta_1 \wedge \dimMeta_2] {\left(\pQ_1 \times \pQ_2\right)} \myOR
\annot [\dimMeta_1] \pQ_1 \in \qGroup {\vQ_1}, \annot [\dimMeta_2] \pQ_2 \in \qGroup {\vQ_2} }
\\
%
\qGroup {{\vQ_1} \Join_\vCond {\vQ_2}} &= 
\setDef {\annot [\dimMeta_1 \wedge \dimMeta_2 \wedge \dimMeta_\vCond] {\left(\pQ_1 \Join_{\pCond} \pQ_2 \right)} \myOR 
\annot [\dimMeta_1] \pQ_1 \in \qGroup {\vQ_1}, \annot [\dimMeta_2] \pQ_2 \in \qGroup {\vQ_2}
%& \hspace{104pt}
,\annot [\dimMeta_\vCond] \pCond \in \cGroup  }
\\
%
\qGroup {\chc {\vQ_1, \vQ_2}} &= 
\setDef {\annot [\dimMeta \wedge \dimMeta_1] \pQ_1 \myOR  \annot [\dimMeta_1] \pQ_1 \in \qGroup {\vQ_1} }
\cup 
\setDef {\annot [\neg \dimMeta \wedge \dimMeta_2] \pQ_2 \myOR  \annot [\dimMeta_2] \pQ_2 \in \qGroup {\vQ_2}}  \\
%
\qGroup {{\vQ_1} \circ {\vQ_2}} &= 
\setDef {\annot [\dimMeta_1 \wedge \dimMeta_2] {\left(\pQ_1 \circ \pQ_2\right)} \myOR
\annot [\dimMeta_1] \pQ_1 \in \qGroup {\vQ_1}, \annot [\dimMeta_2] \pQ_2 \in \qGroup {\vQ_2} }\\
%
\qGroup {\empRel} &= \annot [\t] { \empRel}
\end{alignat*}
\caption[Unique configuration of variational queries]{Unique configuration of variational queries. 
The unique configuration function assumes that the input is well-typed.
}
\label{fig:vq-group}
\end{figure}
\maybeAdd{see if you have more efficient \aGroup\ and \cGroup}

%\begin{landscape}
\begin{figure}
%\begin{minipage}[t]{0.5\textwidth}
\caption[Example of derivation tree to determine the type of a query]{Derivation tree for \ensuremath{\vQ_1} in \exref{type}.}
%\textbf{Derivation tree for \ensuremath{\vQ_1} in \exref{type}:}

%\footnotesize
\scriptsize

\textbf{Assumption 1:}
\begin{prooftree*}
\hypo {\empbio  {(\empno, \sex, \birthdate, \annot [\vFour] \name, \annot [\vFive] \fname, \annot [\vFive] \lname) }\in \vSch_2}
\hypo {\sat {\dimMeta_2 \wedge \dimMeta_2}}
\Infer2 [\textsc{Relation-E}] {\env [\dimMeta_2] [\vSch_2] {\empbio} {\annot [\dimMeta_2] {(\empno, \sex, \birthdate, \annot [\vFour] \name, \annot [\vFive] \fname, \annot [\vFive] \lname)}}}
\end{prooftree*}

\medskip
%\begin{prooftree*}
\textbf{Assumption 2:}
\begin{alignat*}{1}
%\hypo {
&  { \{\annot [\vFour \vee \vFive] \empno, \name, \fname, \lname  \} }\\ 
&\qquad \prec
\pushIn{\{ \empno, \sex,  \birthdate, \annot [\vFour] \name, \annot [\vFive] \fname, \annot [\vFive] \lname\}^{\dimMeta_2} } 
%}
%\end{prooftree*}
\end{alignat*}

\medskip
\textbf{Assumption 3:}
\begin{alignat*}{1}
|\pushIn { \{\annot [\vFour \vee \vFive] \empno, \name, \fname, \lname  \}^{\dimMeta_2} }|
= |{ \{\annot [\vFour \vee \vFive] \empno, \name, \fname, \lname  \} }|
\end{alignat*}

\medskip
\textbf{Final derivation tree:}
\begin{prooftree*}
\hypo {\textit{\bf Assumption 1}}
%\hypo {\empbio \annot [\dimMeta_2] {(\empno, \sex, \birthdate, \annot [\vFour] \name, \annot [\vFive] \fname, \annot [\vFive] \lname) }\in \vSch_2}
%\hypo {\sat {\dimMeta_2 \wedge \dimMeta_2}}
%\Infer2 [\relationE] {\env [\dimMeta_2] [\vSch_2] {\empbio} {\annot [\dimMeta_2] {(\empno, \sex, \birthdate, \annot [\vFour] \name, \annot [\vFive] \fname, \annot [\vFive] \lname)}}}
\hypo {\textit{\bf Assumption 2}}
%\hypo {\subsume { \{\annot [\vFour \vee \vFive] \empno, \name, \fname, \lname  \} } 
%{\{\annot [\dimMeta_2] \empno, \annot [\dimMeta_2] \sex, \annot [\dimMeta_2] \birthdate, \annot [\vFour] \name, \annot [\vFive] \fname, \annot [\vFive] \lname\} } }
\hypo {\textit{\bf Assumption 3}}
\Infer3 [\textsc{Project-E}] {\env [\dimMeta_2] [\vSch_2] {\vQ_1} {\annot [\dimMeta_2] {( \annot [(\vFour \vee \vFive) ] \empno, \annot [\vFour] \name, \annot [\vFive] \fname, \annot [\vFive] \lname)} }}
\end{prooftree*}
\label{fig:vq1-type}
\end{figure}

\begin{figure}
\caption[Example of derivation tree to determine the type of a query]{Derivation tree for \ensuremath{\vQ_2} in \exref{type}.}
%\medskip
%\textbf{Derivation tree for \ensuremath{\vQ_2} in \exref{type}:}

\scriptsize
\textbf{Assumption 1:}
\begin{prooftree*}
\hypo {\empbio \annot [\dimMeta_2] {(\empno, \sex, \birthdate, \annot [\vFour] \name, \annot [\vFive] \fname, \annot [\vFive] \lname) }\in \vSch_2}
\end{prooftree*}

\medskip
\textbf{Assumption 2:}
\begin{prooftree*}
\hypo {\textit{\bf Assumption 1}}
\hypo {\sat {(\dimMeta_2 \wedge (\neg \vThree)) \wedge \dimMeta_2}}
\Infer2 [\textsc{Relation-E}] {\env [\dimMeta_2 \wedge (\neg \vThree)] [\vSch_2] {\empbio} {\annot [\dimMeta_2 \wedge (\neg \vThree)] {(\empno, \sex, \birthdate, \annot [\vFour] \name, \annot [\vFive] \fname, \annot [\vFive] \lname)}}}
\end{prooftree*}

\medskip
\textbf{Assumption 3:}
\begin{alignat*}{1}
|\pushIn { \{ \empno, \name, \fname, \lname  \}^{\dimMeta_2 \wedge \neg \vThree} } |
=|{ \{ \empno, \name, \fname, \lname  \} } |
\end{alignat*}

%\begin{prooftree*}
\medskip
\textbf{Assumption 4:}
\begin{alignat*}{1}
%\hypo {
%\subsume
& { \{ \empno, \name, \fname, \lname  \} } \\
&\qquad \prec {\{\annot [\dimMeta_2 \wedge (\neg \vThree)] \empno, \annot [\dimMeta_2 \wedge (\neg \vThree)] \sex, \annot [\dimMeta_2 \wedge (\neg \vThree)] \birthdate, \annot [\vFour] \name, \annot [\vFive] \fname, \annot [\vFive] \lname\} } 
%}
%\end{prooftree*}
\end{alignat*}

\medskip
\textbf{Assumption 5 (derivation tree for $\mathit{left} = {\pi_{\empno, \name, \fname, \lname} (\empbio)}$):}
\begin{prooftree*}
\hypo {\textit{\bf Assumption 2}}
\hypo {\textit{\bf Assumption 3}}
\hypo {\textit{\bf Assumption 4}}
\Infer3 [\textsc{Project-E}] {\env [\dimMeta_2 \wedge (\neg \vThree)] [\vSch_2] {\mathit{left}} {\annot [\dimMeta_2 \wedge (\neg \vThree)] {( \empno, \annot [\vFour] \name, \annot [\vFive] \fname, \annot [\vFour] \lname)}}}
\end{prooftree*}

\medskip
\textbf{Final derivation tree for $\vQ_2$:}
\begin{prooftree*}
%\hypo {Assumption 2}
\hypo {\textit{\bf Assumption 5}}
\Infer0 [\textsc{EmptyRelation-E}] {\env [\dimMeta_2 \wedge \neg (\neg \vThree)] [\vSch_2] {\empRel} {\annot [\texttt{false}] {\setDef \ }}}
\Infer2 [\textsc{Choice-E}] {\env [\dimMeta_2] [\vSch_2] {\vQ_2} {\annot [(\dimMeta_2 \wedge (\neg \vThree)) \vee \texttt{false}] {( \empno, \annot [\vFour] \name, \annot [\vFive] \fname, \annot [\vFive] \lname)}}}
\end{prooftree*}


%\begin{prooftree*}
%\hypo {\empbio \annot [\dimMeta_2] {(\empno, \sex, \birthdate, \annot [\vFour] \name, \annot [\vFive] \fname, \annot [\vFive] \lname) }\in \vSch_2}
%\hypo {\sat {(\dimMeta_2 \wedge (\vFour \vee \vFive)) \wedge \dimMeta_2}}
%\Infer2 [\relationE] {\env [\dimMeta_2 \wedge (\vFour \vee \vFive)] [\vSch_2] {\empbio} {\annot [\dimMeta_2 \wedge (\vFour \vee \vFive)] {(\empno, \sex, \birthdate, \annot [\vFour] \name, \annot [\vFive] \fname, \annot [\vFive] \lname)}}}
%\hypo {\subsume { \{ \empno, \name, \fname, \lname  \} } {\{\annot [\dimMeta_2 \wedge (\vFour \vee \vFive)] \empno, \annot [\dimMeta_2 \wedge (\vFour \vee \vFive)] \sex, \annot [\dimMeta_2 \wedge (\vFour \vee \vFive)] \birthdate, \annot [\vFour] \name, \annot [\vFive] \fname, \annot [\vFive] \lname\} } }
%\Infer2 [\prjE] {\env [\dimMeta_2 \wedge (\vFour \vee \vFive)] [\vSch_2] {\prj {\empno, \name, \fname, \lname} \empbio} {\annot [\dimMeta_2 \wedge (\vFour \vee \vFive)] {(\annot [\dimMeta_2 \wedge (\vFour \vee \vFive)] \empno, \annot [\vFour] \name, \annot [\vFive] \fname, \annot [\vFour] \lname)}}}
%\Infer0 [\empRelE] {\env [\dimMeta_2 \wedge \neg (\vFour \vee \vFive)] [\vSch_2] {\empRel} {\annot [\f] {\setDef \ }}}
%\Infer2 [\choiceE] {\env [\dimMeta_2] [\vSch_2] {\vQ_2} {\annot [(\dimMeta_2 \wedge (\vFour \vee \vFive)) \vee (\f \wedge \neg (\vFour \vee \vFive))] {(\annot [(\vFour \vee \vFive) \wedge \dimMeta_2] \empno, \annot [\vFour] \name, \annot [\vFive] \fname, \annot [\vFive] \lname)}}}
%\end{prooftree*}

%  \begin{mathpar}
%  \small
%  
%  \inferrule[\prjE]
%  {\inferrule [\relationE] 
%  {blah}
%  {\env [\dimMeta_2] [\vSch_2] {\empbio} {\annot [\dimMeta_2] {(\empno, \sex, \birthdate, \annot [\vFour] \name, \annot [\vFive] \fname, \annot [\vFive] \lname)}}}\\
%\subsume { \{\annot [\vFour \vee \vFive] \empno, \name, \fname, \lname  \} } 
%{\{\annot [\dimMeta_2] \empno, \annot [\dimMeta_2] \sex, \annot [\dimMeta_2] \birthdate, \annot [\vFour] \name, \annot [\vFive] \fname, \annot [\vFive] \lname\}}}
%  %
%  {\env [\dimMeta_2] [\vSch_2] {\vQ_1} {\annot [\dimMeta_2] {\setDef {\annot [(\vFour \vee \vFive) \wedge \dimMeta_2] {(\empno, \annot [\vFour] \name, \annot [\vFive] \fname, \annot [\vFive] \lname)} }}}}
%
%%implicitly-typed lang:
%    \inferrule[\relationE]
%  	{\vRel (\vType)^{\VVal \dimMeta} \in \vSch \\
%	\sat{\vctx \wedge \VVal \dimMeta} }
%     {\envWithSchema{\envInContext [\vctx \wedge \VVal \dimMeta] {\vType}}}
%
%% explicitly-typed vra:  
%%  \inferrule[\prjE]
%%  	{\envPrime \\
%%    	\subsume {\annot \vType}  {\annot [\VVal \vctx] {\VVal \vType}}}
%%    {\env{\vPrj[\vType] \vQ} {\envInContext [\vctx] \vType}}
%
%%implicitly-typed lang:
%  \inferrule[\prjE]
%  	{\envPrime \\
%    	\subsume {\pushIn {\annot \vType}}  {\pushIn {\annot [\VVal \vctx] {\VVal \vType}}}}
%    {\env{\vPrj[\vType] \vQ} {\envInContext [\VVal \vctx] {\left(\pushIn{\annot {\vType}} \cap {\VVal \vType} \right)}}}
%
%
%  \inferrule[\selE]
%  	{\env \vQ {\envInContext [\VVal \vctx] \vType} \\
%    	\envCondAnnot \vCond}
%    {\env{\vSel \vQ}{\envInContext [\VVal \vctx] \vType}}
%    
%  \inferrule[\choiceE]
%  	{\envOne[\vctx \wedge \VVal \dimMeta] \\
%    	\envTwo[\vctx \wedge \neg \VVal \dimMeta]}
%    {\env{\chc[\VVal \dimMeta]{\vQ_1, \vQ_2}}{
%     \envInContext [(\vctx_1 \wedge \VVal \dimMeta) \vee (\vctx_2 \wedge \neg \VVal \dimMeta)] 
%     {\left({\pushIn {\envInContext [\vctx_1 \wedge \VVal \dimMeta] \vType_1}} \cup
%    							{\pushIn {\envInContext [\vctx_2 \wedge \neg \VVal \dimMeta] \vType_2}}\right)}}}
%    
%  \inferrule[\productE]
%  	{\envOne \\
%    	\envTwo\\
%	\pushIn {\annot [\vctx_1] \vType_1} \cap \pushIn {\annot [\vctx_2] \vType_2} = \{\}}
%    {\env{\vQ_1 \times \vQ_2}{\envInContext [\vctx_1 \wedge \vctx_2] 
%      {\left(\pushIn {\annot [\vctx_1] \vType_1} \cup \pushIn {\annot [\vctx_2] \vType_2} \right)}}}
%
%
%  \inferrule[\setopE]
%  	{\envOne \\
%    	\envTwo \\
%	\envEval {\pushIn {\annot [\vctx_1] \vType_1}} {\pushIn {\annot [\vctx_2] \vType_2}}}
%%        \envEval{\envInContext{\vType_1}} \vType \\
%%        \envEval{\envInContext{\vType_2}} \vType}
%    {\env{\vQ_1 \circ \vQ_2} {\envInContext [\vctx_1] \vType_1} }
%
%%  \inferrule[\diffE]
%%  	{\envOne \\
%%    	\envTwo \\
%%        \envEval{\envInContext{\vType_1}} \vType \\
%%        \envEval{\envInContext{\vType_2}} \vType}
%%    {\env{\vQ_1 \setminus \vQ_2} \vType}
%  \end{mathpar}
%  
%\medskip
%\textbf{V-condition typing rules (b: boolean tag, \pAtt: plain attribute, k: constant value):}
%%(b: boolean tag, A: plain attribute, k: constant value)}
%  \begin{mathpar}
%  \small    
%
%  \inferrule[\conjC]
%  	{\envCond \vCond_1\\
%    	\envCond \vCond_2}
%    {\envCond{\vCond_1 \wedge \vCond_2}}
%    
%  \inferrule[\disjC]
%  	{\envCond \vCond_1\\
%    	\envCond \vCond_2}
%    {\envCond{\vCond_1 \vee \vCond_2}}
%    
%
%
%  \inferrule[\choiceC]
%%  	{\defType{\relInContext{\vContext''}}\in \vSch \\
%    	{\envCond[\vctx \wedge \VVal \dimMeta, \vType]{\vCond_1} \\
%        \envCond[\vctx \wedge \neg \VVal \dimMeta, \vType]{\vCond_2}}
%    {\envCond{\chc[\VVal \dimMeta]{\vCond_1, \vCond_2}}}
%    
%
%  \inferrule[\notC]
%  	{\envCond \vCond}
%    {\envCond \neg \vCond}
%        
%%  \inferrule[]
%%  	{\envCond[\vContext \wedge \dimMeta]{\vCond_1} \\
%%    	\envCond[\vContext \wedge \neg\dimMeta]{\vCond_2}}
%%    {\envCond{\chc{\vCond_1, \vCond_2}}}
%    
%
%    
%  \inferrule[\attValC]
%  	{
%	%\defType{\relInContext{\vContext'}}\in \vSch \\
%    	\optAtt [\VVal \dimMeta] \in \vType \\
%%	\taut{{\VVal \dimMeta} \imply \vctx} \\
%        \sat {\VVal \dimMeta \wedge \vctx}\\
%        \cte \in \dom \vAtt}
%    {\envCond{\op \pAtt \cte}}
%    
%  \inferrule[\boolC]
%  	{}
%    {\envCond \bTag}
%    
%
%    
%  \inferrule[\attAttC]
%  	{
%	%\defType{\relInContext{\vContext'}}\in \vSch \\
%    	\optAtt [\dimMeta_1] [\vAtt_1]\in \vType \\
%         {\optAtt [\dimMeta_2] [\vAtt_2]} \in \vType \\
%%         \taut{\dimMeta_1 \imply \vctx} \\
%%         \taut{\dimMeta_2 \imply \vctx} \\
%        \sat { \dimMeta_1 \wedge \dimMeta_2 \wedge \vctx}\\
%        \type[\vAtt_1] = \type[ \vAtt_2]}
%    {\envCond{\op{\pAtt_1}{\pAtt_2}}}
%    
%  \end{mathpar}
%
%%\caption{V-condition typing relation. A v-condition \vCond\ is well-typed if 
%%it is valid in the variational context \vctx\ and type environment \vType, i.e., 
%%\envCond \vCond. Note that the type rules for v-conditions return a boolean, if
%%the v-condition is type-correct the rules return \t, otherwise they return \f.}
%\caption{VRA and v-condition typing relation. The typing rule of a join query is the combination
%of rules \selE\ and \productE.}
%\label{fig:vq-stat-sem}
%\end{minipage}
\label{fig:vq2-type}
\end{figure}
%\end{landscape}

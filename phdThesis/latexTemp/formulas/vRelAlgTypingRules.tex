\begin{figure}
%\begin{minipage}[t]{0.5\textwidth}
\textbf{V-queries typing rules:}

  \begin{mathpar}
  \small
  
  \inferrule[\empRelE]
  {}
  {\env {\empRel} {\annot [\f] {\setDef \ }}}
%  \inferrule[\judge]
 % 	{\env{\vQ}{\vType}}
 %   {}
%
% explicitly-typed vra:
%    \inferrule[\relationE]
%  	{\vRel (\vType)^{\VVal \dimMeta} \in \vSch \\
%	\neg \sat{\vctx \wedge \neg \VVal \dimMeta} }
%     {\envWithSchema{\envInContext [\vctx ] {\vType}}}

%implicitly-typed lang:
    \inferrule[\relationE]
  	{ \vRel (\vType)^{\VVal \dimMeta} \in \vSch \\
	\sat {\vctx \wedge \getPCfrom \vRelSch \vSch}}
%	\sat{\vctx \wedge \VVal \dimMeta} }
     {\envWithSchema{\envInContext [\vctx \wedge \VVal \dimMeta] {\vType}}}

% explicitly-typed vra:  
%  \inferrule[\prjE]
%  	{\envPrime \\
%    	\subsume {\annot \vType}  {\annot [\VVal \vctx] {\VVal \vType}}}
%    {\env{\vPrj[\vType] \vQ} {\envInContext [\vctx] \vType}}

%implicitly-typed lang:
  \inferrule[\prjE]
  	{\envPrime \\
	|\pushIn {\annot \vType}| = | \vType | \\
%	\pushIn {\annot \vType} \neq \setDef \ \\
%	\pushIn {\annot [\VVal \dimMeta] {\VVal \vType}} \neq \setDef \ \\
    	\subsume { \vType}  {\pushIn {\annot [\VVal \vctx] {\VVal \vType}}}}
    {\env{\vPrj[\vType] \vQ} {\envInContext [\VVal \vctx] {\left(\vType \cap {\VVal \vType} \right)}}}
%    the older version 8/8/20:
%    	\subsume {\pushIn {\annot \vType}}  {\pushIn {\annot [\VVal \vctx] {\VVal \vType}}}}
%    {\env{\vPrj[\vType] \vQ} {\envInContext [\VVal \vctx] {\left(\pushIn{\annot {\vType}} \cap {\VVal \vType} \right)}}}


  \inferrule[\selE]
  	{\env \vQ {\envInContext [\VVal \vctx] \vType} \\
    	\envCondAnnot \vCond}
    {\env{\vSel \vQ}{\envInContext [\VVal \vctx] \vType}}
    
  \inferrule[\choiceE]
  	{\envOne[\vctx \wedge \VVal \dimMeta] \\
    	\envTwo[\vctx \wedge \neg \VVal \dimMeta]}
    {\env{\chc[\VVal \dimMeta]{\vQ_1, \vQ_2}}{
     \envInContext [(\vctx_1 \wedge \VVal \dimMeta) \vee (\vctx_2 \wedge \neg \VVal \dimMeta)] 
     {\left({\pushIn {\envInContext [\vctx_1] \vType_1}} \cup
    							{\pushIn {\envInContext [\vctx_2] \vType_2}}\right)}}}
% older version 8/8/20:
%    {\env{\chc[\VVal \dimMeta]{\vQ_1, \vQ_2}}{
%     \envInContext [(\vctx_1 \wedge \VVal \dimMeta) \vee (\vctx_2 \wedge \neg \VVal \dimMeta)] 
%     {\left({\pushIn {\envInContext [\vctx_1 \wedge \VVal \dimMeta] \vType_1}} \cup
%    							{\pushIn {\envInContext [\vctx_2 \wedge \neg \VVal \dimMeta] \vType_2}}\right)}}}
    
  \inferrule[\productE]
  	{\envOne \\
    	\envTwo\\
	\pushIn {\annot [\vctx_1] \vType_1} \cap \pushIn {\annot [\vctx_2] \vType_2} = \{\}}
    {\env{\vQ_1 \times \vQ_2}{\envInContext [\vctx_1 \wedge \vctx_2] 
      {\left(\pushIn {\annot [\vctx_1] \vType_1} \cup \pushIn {\annot [\vctx_2] \vType_2} \right)}}}


  \inferrule[\setopE]
  	{\envOne \\
    	\envTwo \\
	\envEval {\pushIn {\annot [\vctx_1] \vType_1}} {\pushIn {\annot [\vctx_2] \vType_2}}}
%        \envEval{\envInContext{\vType_1}} \vType \\
%        \envEval{\envInContext{\vType_2}} \vType}
    {\env{\vQ_1 \circ \vQ_2} {\envInContext [\vctx_1] \vType_1} }

%  \inferrule[\diffE]
%  	{\envOne \\
%    	\envTwo \\
%        \envEval{\envInContext{\vType_1}} \vType \\
%        \envEval{\envInContext{\vType_2}} \vType}
%    {\env{\vQ_1 \setminus \vQ_2} \vType}
  \end{mathpar}
  
\medskip
\textbf{V-condition typing rules:}
% (b: boolean tag, \pAtt: plain attribute, k: constant value):}
%(b: boolean tag, A: plain attribute, k: constant value)}
  \begin{mathpar}
  \small    

  \inferrule[\conjC]
  	{\envCond \vCond_1\\
    	\envCond \vCond_2}
    {\envCond{\vCond_1 \wedge \vCond_2}}
    
  \inferrule[\disjC]
  	{\envCond \vCond_1\\
    	\envCond \vCond_2}
    {\envCond{\vCond_1 \vee \vCond_2}}
    


  \inferrule[\choiceC]
%  	{\defType{\relInContext{\vContext''}}\in \vSch \\
    	{\envCond[\vctx \wedge \VVal \dimMeta, \vType]{\vCond_1} \\
        \envCond[\vctx \wedge \neg \VVal \dimMeta, \vType]{\vCond_2}}
    {\envCond{\chc[\VVal \dimMeta]{\vCond_1, \vCond_2}}}
    

  \inferrule[\notC]
  	{\envCond \vCond}
    {\envCond \neg \vCond}
        
%  \inferrule[]
%  	{\envCond[\vContext \wedge \dimMeta]{\vCond_1} \\
%    	\envCond[\vContext \wedge \neg\dimMeta]{\vCond_2}}
%    {\envCond{\chc{\vCond_1, \vCond_2}}}
    

    
  \inferrule[\attValC]
  	{
	%\defType{\relInContext{\vContext'}}\in \vSch \\
    	\optAtt [\VVal \dimMeta] \in \vType \\
%	\taut{{\VVal \dimMeta} \imply \vctx} \\
        \sat {\VVal \dimMeta \wedge \vctx}}
%        \\
%        \cte \in \dom \vAtt}
    {\envCond{\op \pAtt \cte}}
    
  \inferrule[\boolC]
  	{}
    {\envCond \bTag}
    

    
  \inferrule[\attAttC]
  	{
	%\defType{\relInContext{\vContext'}}\in \vSch \\
    	\optAtt [\dimMeta_1] [\vAtt_1]\in \vType \\
         {\optAtt [\dimMeta_2] [\vAtt_2]} \in \vType \\
%         \taut{\dimMeta_1 \imply \vctx} \\
%         \taut{\dimMeta_2 \imply \vctx} \\
        \sat { \dimMeta_1 \wedge \dimMeta_2 \wedge \vctx}}
%        \\
%        \type[\vAtt_1] = \type[ \vAtt_2]}
    {\envCond{\op{\pAtt_1}{\pAtt_2}}}
    
  \end{mathpar}

%\caption{V-condition typing relation. A v-condition \vCond\ is well-typed if 
%it is valid in the variational context \vctx\ and type environment \vType, i.e., 
%\envCond \vCond. Note that the type rules for v-conditions return a boolean, if
%the v-condition is type-correct the rules return \t, otherwise they return \f.}
\caption[\TODO{shortcaption}]{VRA and v-condition typing relation. 
The rules assume that the underlying VDB is well-formed. 
Remember that our theory assumes all attributes have the same type
and all constants belong to attributes' domain. 
%The typing rule of a join query is the combination
%of rules \selE\ and \productE.
}
\label{fig:vq-stat-sem}
%\end{minipage}
\end{figure}


\begin{figure}
%[ht]
%
%\begin{comment}
%\textbf{Relational model generic objects:}
%\begin{syntax}
%%OLD
%D\in \mathbf{Dom} &&& \textit{Domain}\\
%A\in \mathbf{Att} &&& \textit{Attribute Name}\\
%R\in \mathbf{R} &&& \textit{Relation Name}\\
%t \in \mathbf{T} &&& \textit{Tuple}
%\end{syntax}
%
%\medskip
%\textbf{Relational model definition:}
%\begin{syntax}
%l\in \mathbf{L} &=& \vn{A} &\textit{Attribute set}\\
%s \in \mathbb{S} &=& R(A_1, \ldots , A_n ) & \textit{Relation specification}\\
%S \in \mathcal{\mathbf{S}} &\Coloneqq& {\vn{s}} & \textit{Schema}\\
%T \in \mathbf{T} &\Coloneqq& \{\llangle t(1), \ldots, t(k)\rrangle \myOR \\
%&&t(i) \in D_i,
%1 \leq i \leq k ,\\
%&&k = \mathit{arity}(R) \} 
%%v_1^1\in D_1, \ldots, v_n^1\in D_n\rrangle, \ldots, \llangle v_1^m\in D_1, \ldots, v_n^m\in D_n\rrangle|\\
%%&& \hspace{0.5cm} m = \textit{number of } R_I\textit{'s tuples}\} 
%&\textit{Relation Instance (Table)}\\
%%I \in \mathbf{Inst} &\Coloneqq& R_{1_I}, \cdots, R_{n_I} & \textit{Database Instance}
%\end{syntax}
%
%
%\medskip
%\textbf{Variational relational algebra objects:}
%\begin{syntax}
%\synDef \dimMeta \ffSet &&&\textit{Presence condition}\\
%\synDef \vAtt \vAttSet &&&\textit{Variational attribute}\\
%\synDef \vAttList \vAttSet &\eqq& \vAtt, \vAttList \myOR \empAtt &\textit{Variational attribute list}\\
%\synDef \vRelSch \vRelSet &\eqq& \vRelDef &\textit{Variational relation schema}\\
% \vRel &&&\textit{Variational relation}\\
%\synDef \vSch \vSchSet &\eqq& \vSchDef &\textit{Variational schema}\\
% &&&\textit{Variational database instance}
%\end{syntax}
%\end{comment}
%
%%%%%%%%%%%%%%%%%%%%%%%%%%%%%%%%%%%%%%%%%%%%%%%%%%%%
\textbf{Variational database objects:}
\begin{syntax}
%\synDef \vAtt \attNames &&&\textit{Attribute Name}\\
%\synDef \vRel \relNames &&& \textit{Relation Name}\\
%\synDef \vAttList {\boldmth{\mathbf{V} \mathbf{A}}} &\eqq& 
%\setDef {\annot [\dimMeta_1] \vAtt_1, \annot [\dimMeta_2] \vAtt_2, \ldots, \annot [\dimMeta_k] \vAtt_k} & \textit{Variational Set of Attributes}\\
%\synDef \vRelSch \vRelSchSet &\eqq& \vRelDef & \textit{Variational Relation Schema}\\
%\synDef \vSch \vSchSet &\eqq& \vSchDef & \textit{Variational Schema}
%
\synDef \vTuple {\vartype \tupletype} &\eqq& \annot[\dimMeta]{(\vi v \numAtts)} & \textit{Variational Tuple}\\
%\vTuple\in\vRelCont \eqq \annot[\dimMeta_\vTuple]{(\vi v \numAtts)}
\synDef \vRelCont \vRelContSet &\eqq& \setDef {\vi \vTuple \numTuples} & \textit{Variational Relation Content}\\
%\vRelCont \in \vRelContSet \eqq \setDef {\vi \vTuple \numTuples}
\synDef \vTab \tabletype &\eqq& (\vRelSch, \vRelCont) & \textit{Variational Table}\\
\synDef \vdbInst \vdbInstSet &\eqq& \annot [\dimMeta] {\setDef {\vi \vTab \numRels} } & \textit{Variational Database Instance}
%\synDef \vdbInst  \vdbInstSet \eqq \annot [\dimMeta] {\setDef {\vi \vTab \numRels} }
\end{syntax}

\medskip
\textbf{Variational database type synonyms:}
\begin{alignat*}{1}
\vRelContSet &= \settype {(\vartype \tupletype)}\\
\tabletype &= \typepair{\vRelSchSet, \vRelContSet}\\
\vdbInstSet &= \vartype {\left( \settype {\left(\left(\relschtype,\relconttype\right)\right)}\right)}
\end{alignat*}

\medskip
\textbf{Variational tuple configuration:}
%
\begin{alignat*}{1}%\raggedleft
\ouSemType [] . &: {\vartype \tupletype} \to \vRelSchSet \to \confSet \to \maybe \tupletype\\
%\end{flalign*}
%
%\begin{flalign*}%\raggedleft
\ouSem{\vRelSch} {\annot  {\left( {\vi v \numAtts}\right)}}  &\\
& \hspace{-50pt} = \begin{cases}
(v_i, \cdots, v_j), &\If \forall k. 1 \leq i \leq k \leq j \leq l, \fSem {\getPCfrom {\getAtt {k}} \vRelSch \wedge \dimMeta} = \t\\
\bot, &\Otherwise
\end{cases}
%\left( \ovSem {v_1}, \hdots, \ovSem {v_\numAtts} \right) &\\
%& \textit{ where } \forall 1 \leq i \leq \numAtts: \\
%&\hspace{5pt} \ovSem {v_i} = 
%\begin{cases}
%v_i, & \If \fSem {\fModel \wedge \getPC{\getRel{\getAtt{v_i}}} \wedge \getPC {\getAtt {v_i}} \wedge \dimMeta_\tuple} \\
%\varepsilon, & \Otherwise
%\end{cases}
\end{alignat*}

%\medskip
\textbf{Variational relation content configuration:}
%
\begin{alignat*}{1}%\raggedleft
\otSemType [] . &: \vRelContSet \to \vRelSchSet \to \confSet \to \pRelContSet\\
%\end{flalign*}
%
%\begin{flalign*}%\raggedleft
\otSem {\vRelSch} {\setDef {\vi \tuple \numTuples}} &= \setDef {\ouSem {\vRelSch}{\tuple_1}, \hdots, \ouSem{\vRelSch} {\tuple_\numTuples}}
\end{alignat*}

%\medskip
\textbf{VDB instance configuration:}
%
\begin{alignat*}{1}%\raggedleft
\odbSem [] . &: \vdbInstSet \to \confSet \to \pInstSet\\
%\end{flalign*}
%
%\begin{flalign*}%\raggedleft
\odbSem { \annot  {\setDef {\vi \vTab \numRels}}} 
&=\odbSem { \annot  { \setDef {\left( \vRelSch_1, \vRelCont_1\right), \ldots, 
\left( \vRelSch_\numRels, \vRelCont_\numRels\right)}}}\\
& = \begin{cases}
\setDef{\left( \orSem {\vRel_1 \annot [\dimMeta_1 {\wedge \dimMeta}] {\left( \vAttList_1 \right)} }, 
\otSem {\vRelSch_1} {\pushIn {\annot [\dimMeta_1 {\wedge \dimMeta}] \vRelCont_1}} \right), \ldots}, &\If \fSem \dimMeta = \t \\
\setDef {}, \Otherwise
\end{cases}
%&= \setDef {(\orSem {\vRelSch_1}, \otSem {\vRelCont_1}), \hdots, (\orSem {\vRelSch_\numRels}, \otSem {\vRelCont_\numRels} )}&
\end{alignat*}

\caption{VDB instance syntax and configuration.
%The input to all configuration functions assumes a well-formed input,
%either a v-cond (see \secref{type-sys} or a (part of a) VDB.
%\TODO{you need to define well-formedness for vdb and mention it 
%for vcond somewhere in the paper.}
%%V-cond configuration only accepts conditions that are type correct. 
%%All the configuration functions are defined over a given database
%%with v-schema \vSch. 
%A set with question mark at the end, e.g., \maybe \pRelSchSet, 
%denotes an optional type, meaning that the original set is extended
%with a non-value, \ensuremath{\bot}.
%\revised{
Note that the schema of a relation must be passed to the configuration function
for its content,
however, the variational schema does not need to be passed to configuration 
functions of smaller parts of the variational schema such as \orSem .  or \olSem .
since all needed information for configuring a part of a variational schema
is propagated. 
%Note that $\vRelContSet = \settype {(\vartype \tupletype)}$.
%}
% of variational set of attributes, v-relations, and v-schema.
%$\varepsilon$ denotes a non-existent relation and value.
%Note that the feature model and 
%relation presence condition are passed all the way to attributes due to the 
%hierarchal structure of presence conditions within a v-schema.
}
\label{fig:vdb-conf}
\end{figure} 

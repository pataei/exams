\begin{figure}
%[ht]

\begin{comment}
\textbf{Relational model generic objects:}
\begin{syntax}
%OLD
D\in \mathbf{Dom} &&& \textit{Domain}\\
A\in \mathbf{Att} &&& \textit{Attribute Name}\\
R\in \mathbf{R} &&& \textit{Relation Name}\\
t \in \mathbf{T} &&& \textit{Tuple}
\end{syntax}

\medskip
\textbf{Relational model definition:}
\begin{syntax}
l\in \mathbf{L} &=& \vn{A} &\textit{Attribute set}\\
s \in \mathbb{S} &=& R(A_1, \ldots , A_n ) & \textit{Relation specification}\\
S \in \mathcal{\mathbf{S}} &\Coloneqq& {\vn{s}} & \textit{Schema}\\
T \in \mathbf{T} &\Coloneqq& \{\llangle t(1), \ldots, t(k)\rrangle \myOR \\
&&t(i) \in D_i,
1 \leq i \leq k ,\\
&&k = \mathit{arity}(R) \} 
%v_1^1\in D_1, \ldots, v_n^1\in D_n\rrangle, \ldots, \llangle v_1^m\in D_1, \ldots, v_n^m\in D_n\rrangle|\\
%&& \hspace{0.5cm} m = \textit{number of } R_I\textit{'s tuples}\} 
&\textit{Relation Instance (Table)}\\
%I \in \mathbf{Inst} &\Coloneqq& R_{1_I}, \cdots, R_{n_I} & \textit{Database Instance}
\end{syntax}


\medskip
\textbf{Variational relational algebra objects:}
\begin{syntax}
\synDef \dimMeta \ffSet &&&\textit{Presence condition}\\
\synDef \vAtt \vAttSet &&&\textit{Variational attribute}\\
\synDef \vAttList \vAttSet &\eqq& \vAtt, \vAttList \myOR \empAtt &\textit{Variational attribute list}\\
\synDef \vRelSch \vRelSet &\eqq& \vRelDef &\textit{Variational relation schema}\\
 \vRel &&&\textit{Variational relation}\\
\synDef \vSch \vSchSet &\eqq& \vSchDef &\textit{Variational schema}\\
 &&&\textit{Variational database instance}
\end{syntax}
\end{comment}

%%%%%%%%%%%%%%%%%%%%%%%%%%%%%%%%%%%%%%%%%%%%%%%%%%%
\textbf{Variational Condition Configuration:}
\begin{alignat*}{1}
\ecSem [] . &: \vCondSet \to \confSet \to \pCondSet\\
%
\ecSem \bTag &= \bTag \\
%
\ecSem \vAttOpCte &= 
    \vAttOpCte\\
%	\begin{cases}
%		\vAttOpCte, &\text{ if } \pAtt \in \attr [\eeSem \vRel]\\
%		\f, &\text{ otherwise}
%	\end{cases}\\
%
\ecSem \vAttOpAtt &= 
       \vAttOpAtt\\
%	\begin{cases}
%		\pAttOpAtt, &\text{ if } \pAtt_1 \in \attr [\eeSem \vRel] \&\ 
%		                                   \pAtt_2 \in \attr [\eeSem \vRel] \\
%		\f,  &\text{ otherwise}
%	\end{cases}\\
%
\ecSem {\neg \vCond} &= \neg \ecSem \vCond\\
%
\ecSem {\orr \vCond} &= \ecSem {\vCond_1} \vee \ecSem {\vCond_2}\\
%
\ecSem {\annd \vCond} &= \ecSem {\vCond_1} \wedge \ecSem {\vCond_2}\\
%
\ecSem {\chc {\vCond_1, \vCond_2}} &=
	\begin{cases}
		\ecSem {\vCond_1}, &\text{ if } \fSem \dimMeta  \\
		\ecSem {\vCond_2}, &\text{ otherwise}
	\end{cases}
\end{alignat*}

%\medskip
\textbf{Variational Set of Attributes Configuration:}
\begin{flalign*}
& \olSem [] . &: &\ \vAttSet \to \confSet \to \pAttSet\\
%\end{flalign*}
%
%\begin{flalign*}
& \olSem {\{\optAtt\} \cup \vAttList} &\spcEq &\ 
    \begin{cases}
        \{\pAtt\} \cup \olSem{\vAttList},
        &\If \fSem {\dimMeta}\\
%         \wedge \getPCfrom \vAtt \vRel } \\
%                            & \If \fSem {\dimMeta \wedge \getPC{\getRel \vAtt} \wedge \fModel} \\
        \olSem{\vAttList} , & \Otherwise
     \end{cases} \\
% & \olSem {\{\optAtt\} \cup \vAttList} &\spcEq &\  \olSem {\{\optAtt\}} \cup \olSem {\vAttList}\\
& \olSem {\setDef{}} &\spcEq & \setDef{}
\end{flalign*}

%
%\medskip
\textbf{V-Relation Schema Configuration:}
\begin{flalign*}%\raggedleft
&\orSem [] . : \vRelSchSet \to \confSet  \to \maybe \pRelSchSet&\\
%\end{flalign*}
%
%\begin{flalign*}
&\orSem {\vRelDef [\vRel] [\dimMeta_\vAttList]} = 
	\begin{cases}
		\vRel\left({\olSem {\pushInBold { \annot [\dimMeta_\vAttList] \vAttList}}}\right), &\If \fSem {\dimMeta_\vAttList} \\
%		&\If \fSem {\getPCfrom \vRel \vSch} \\
%		&\If \fSem {\dimMeta \wedge \fModel}) \\
		\bot, &\Otherwise
	\end{cases}&
\end{flalign*}

%
%\medskip
\textbf{V-Schema Configuration:}
\begin{flalign*}%\raggedleft
&\osSem [] . : \vSchSet \to \confSet \to \pSchSet&\\
%\end{flalign*}
%
%\begin{flalign*}
&\osSem {\annot [\fModel] {\setDef {\vRelDefNum 1, \ldots, \vRelDefNum \numRels}}}\\
&\hspace{0.3cm}= \begin{cases}
%		\setDef {\orSem {\vRelDefNumF 1}, \ldots, \orSem {\vRelDefNumF n}},
                 \setDef {\orSem {\vRel_1( \vAttList_1 )^{\dimMeta_1 \boldmth{\wedge \fModel}} }, \ldots, 
                 \orSem {\vRel_\numRels( \vAttList_\numRels)^{\dimMeta_\numRels \boldmth{\wedge \fModel}} }},	
                         & \If \fSem {\fModel } \\	
%        & \If \fSem \fModel \\
        \setDef{}, & \text{otherwise}
	\end{cases}&
\end{flalign*}

%\medskip
\textbf{V-Tuple Configuration:}
%
\begin{flalign*}%\raggedleft
&\ouSemType [] . : \vRelCont \to \vRelSchSet \to \confSet \to \maybe \pRelCont&\\
%\end{flalign*}
%
%\begin{flalign*}%\raggedleft
&\ouSem{\vRelSch} {\annot [ \dimMeta_\tuple] {\left( {\vi v \numAtts}\right)}}  \\
& = \begin{cases}
(v_i, \cdots, v_j), &\If \forall k, 1 \leq i \leq k \leq j, \fSem {\getPCfrom {\getAtt {k}} \vRelSch \wedge \dimMeta_\tuple}\\
\bot, &\Otherwise
\end{cases}
%\left( \ovSem {v_1}, \hdots, \ovSem {v_\numAtts} \right) &\\
%& \textit{ where } \forall 1 \leq i \leq \numAtts: \\
%&\hspace{5pt} \ovSem {v_i} = 
%\begin{cases}
%v_i, & \If \fSem {\fModel \wedge \getPC{\getRel{\getAtt{v_i}}} \wedge \getPC {\getAtt {v_i}} \wedge \dimMeta_\tuple} \\
%\varepsilon, & \Otherwise
%\end{cases}
\end{flalign*}

%\medskip
\textbf{V-Relation Content Configuration:}
%
\begin{flalign*}%\raggedleft
&\otSemType [] . : \vRelContSet \to \vRelSchSet \to \confSet \to \pRelContSet&\\
%\end{flalign*}
%
%\begin{flalign*}%\raggedleft
&\otSem {\vRelSch} {\setDef {\vi \tuple \numTuples}} = \setDef {\ouSem {\vRelSch}{\tuple_1}, \hdots, \ouSem{\vRelSch} {\tuple_\numTuples}}&
\end{flalign*}

%\medskip
\textbf{VDB Instance Configuration:}
%
\begin{flalign*}%\raggedleft
&\odbSem [] . : \vInstSet \to \confSet \to \pInstSet&\\
%\end{flalign*}
%
%\begin{flalign*}%\raggedleft
&\odbSem { \annot [\fModel] {\setDef {\vi \vTab \numRels}}} 
=\odbSem { \annot [\fModel] { \setDef {\left( \vRelSch_1, \vRelCont_1\right), \ldots, 
\left( \vRelSch_\numRels, \vRelCont_\numRels\right)}}}&\\
& = \begin{cases}
\setDef{\left( \orSem {\vRel_1 \annot [\dimMeta_1 \boldmth{\wedge \fModel}] {\left( \vAttList_1 \right)} }, 
\otSem {\vRelSch_1} {\pushInBold {\annot [\dimMeta_1 \boldmth{\wedge \fModel}] \vRelCont_1}} \right), \ldots}, &\If \fSem \fModel \\
\setDef {}, \Otherwise
\end{cases}
%&= \setDef {(\orSem {\vRelSch_1}, \otSem {\vRelCont_1}), \hdots, (\orSem {\vRelSch_\numRels}, \otSem {\vRelCont_\numRels} )}&
\end{flalign*}

\caption{
V-cond and VDB instance configurations.
%The input to all configuration functions assumes a well-formed input,
%either a v-cond (see \secref{type-sys} or a (part of a) VDB.
%\TODO{you need to define well-formedness for vdb and mention it 
%for vcond somewhere in the paper.}
%%V-cond configuration only accepts conditions that are type correct. 
%%All the configuration functions are defined over a given database
%%with v-schema \vSch. 
%A set with question mark at the end, e.g., \maybe \pRelSchSet, 
%denotes an optional type, meaning that the original set is extended
%with a non-value, \ensuremath{\bot}.
%\revised{
Note that the schema of a relation must be passed to the configuration function
for its content,
however, the v-schema does not need to be passed to configuration 
functions of smaller parts of the v-schema such as \orSem .  or \olSem .
since all needed infromation for configuring a part of a v-schema
is propagated, \textcolor {green(munsell)}{parts written in green}.
%}
% of variational set of attributes, v-relations, and v-schema.
%$\varepsilon$ denotes a non-existent relation and value.
%Note that the feature model and 
%relation presence condition are passed all the way to attributes due to the 
%hierarchal structure of presence conditions within a v-schema.
}
\label{fig:vdb-conf}
\end{figure} 

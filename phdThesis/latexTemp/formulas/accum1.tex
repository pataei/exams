\begin{figure}

\textbf{Table accumulation function:}
\begin{alignat*}{1}
\mathit{accum} &: \settype \fSet \totype \settype {\typepair \confSet \pTabSet} \totype \tabletype\\
\mathit{accum} \  \mathit{fs} \ \mathit{ts} &= \mathit{mkTable} \ \mathit{vsch} \ \mathit{tables}\\
%&\hspace{60pt} (\mathit{addPresCondToConfTables} \ \mathit{fs} \\
%&\hspace{140pt} (\mathit{fitConfTablesToVsch} \ \mathit{ts} \ \mathit{vsch}))\\
&\hspace{-40pt}\textit{where }
\mathit{vsch} = \mathit{tablesToVsch} \ \mathit{fs} \ \mathit{ts}\\
&\hspace{-6pt} \mathit{tables} = \mathit{addPresCondToConfTables} \ \mathit{fs} \ \mathit {fitted}\\
&\hspace{-6pt} \mathit{fitted} = \mathit{fitConfTablesToVsch} \ \mathit{ts} \ \mathit{vsch}
\end{alignat*}


\medskip 
\textbf{Auxiliary functions for table accumulation:}
\footnotesize
\begin{alignat*}{1}
\mathit{schToVsch} &: \settype \fSet \totype \settype {\typepair \confSet \pRelSchSet} \totype \vRelSchSet\\
\mathit{tablesToVsch} &: \settype \fSet \totype \settype {\typepair \confSet \pTabSet} \totype \vRelSchSet\\
\mathit{fitTableToVsch} &: \pTabSet \totype \vRelSchSet \totype \pTabSet\\
\mathit{fitConfTablesToVsch} &: \settype {\typepair \confSet \pTabSet} \totype \vRelSchSet \totype \settype {\typepair \confSet \pTabSet}\\
\mathit{addPresCondToConfContent} &: \settype \fSet \totype \typepair \confSet \pRelContSet \totype \vRelContSet\\
\mathit{addPresCondToConfTables} &: \settype \fSet \totype \settype {\typepair \confSet \pTabSet} \totype \settype \vRelContSet\\
\mathit{mkTable} &: \vRelSchSet \totype \settype \vRelContSet \totype \tabletype
\end{alignat*}


\caption[Accumulation function of a set of relational tables with their attached configuration into a variational table]{Accumulation function of a set of relational tables with their attached configuration into a variational table and its auxiliary functions. The definition uses spaces to pass parameters. For example, $f \ x$ states that the parameter $x$ is passed to the function $x$ and $f\ x\ y$ states that
parameters $x$ and $y$ are passed to $f$ as the first and second arguments, respectively.
}
\label{fig:accum1}
\end{figure}


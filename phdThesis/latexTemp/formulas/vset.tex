\begin{figure}
%\textbf{Feature expression generic object:}
%\begin{syntax}
%\synDef \fName \fSet &\textit{Feature Name}
%%c \in \mathbf{C} &\textit{Configuration}
%\end{syntax}
%
%\medskip
\textbf{Configuration of an annotated element:}
\begin{alignat*}{1}
\xeSem [] . &: \elemSet \to \confSet \to \maybe \pelemSet\\
\xeSem {\annot \elem} &=
  \begin{cases}
    \elem, & \If \fSem {\dimMeta} = \t\\
    \bot, &\Otherwise    
%		\vRel\left({\olSem {\pushInBold { \annot [\dimMeta_\vAttList] \vAttList}}}\right), &\If \fSem {\dimMeta_\vAttList} \\
%%		&\If \fSem {\getPCfrom \vRel \vSch} \\
%%		&\If \fSem {\dimMeta \wedge \fModel}) \\
%		\bot, &\Otherwise
  \end{cases}
\end{alignat*}

\medskip
\textbf{Configuration of (annotated) variational sets:}
\begin{alignat*}{1}
\osetSem [] . &: \vsetSet \to \confSet \to \psetSet\\
\osetSem {\setDef{\annot [\dimMeta_1] {\elem_1}, \annot [\dimMeta_2] {\elem_2}, \ldots, \annot [\dimMeta_n] {\elem_n}}} &= \dropbot {\setDef{\xeSem {\annot [\dimMeta_1] {\elem_1}}, 
\xeSem {\annot [\dimMeta_2] {\elem_2}}, \ldots, \xeSem {\annot [\dimMeta_n] {\elem_n}} }}\\
\osetSem {\setDef \ } &= \setDef \ \\
%
\osetSem {\annot {\vset}} &= \osetSem {\pushIn {\annot \vset}}\\
\osetSem {\annot [\t] {\setDef \ }} & = \setDef \ \\
\osetSem {\annot [\f] {\setDef \ }} & = \setDef \ 
\end{alignat*}

\medskip
\textbf{Dropping bots from a plain set:}
\begin{alignat*}{1}
%\dropbot {} &: \psetSet \to \psetSet\\
\dropbot {\setDef {\elem_1, \elem_2, \ldots, \elem_n}} &= \setDef {\elem_i \myOR \elem_i \neq \bot}
\end{alignat*}

\medskip 
\textbf{Normalization of (annotated) variational sets:}
\begin{alignat*}{1}
\pushIn {\setDef {\annot [\dimMeta_1] {\elem_1}, \annot [\dimMeta_2] {\elem_2}, \ldots,
\annot [\dimMeta_n] {\elem_n}}} &= \setDef {
\annot [\dimMeta_i] \elem_i \myOR 1 \leq i \leq n, \sat {\dimMeta_i}}\\
%\pushIn {\annot {\setDef {\annot [\dimMeta_1] {\elem_1}, \annot [\dimMeta_2] {\elem_2}, \ldots,
%\annot [\dimMeta_n] {\elem_n}}}} &= 
\pushIn {\annot \vset} &=
\setDef {\annot [\dimMeta_i \wedge \dimMeta] {\elem_i} \myOR 
\annot [\dimMeta_i] \elem_i \in \annot \vset, \sat {\dimMeta_i \wedge \dimMeta}}
\end{alignat*}

\medskip 
\textbf{Operations over (annotated) variational sets:}
\begin{alignat*}{1}
\vset_1 \cup \vset_2 &= \setDef {\annot [\dimMeta_1] \elem \myOR \annot [\dimMeta_1] \elem \in \pushIn {\vset_1}, \exists \dimMeta_2. \annot [\dimMeta_2] \elem \not \in \pushIn{\vset_2}}\\
&\hspace{15pt} \cup \setDef {\annot [\dimMeta_2] \elem \myOR \annot [\dimMeta_2] \elem \in \pushIn {\vset_2}, 
\exists \dimMeta_1. \annot [\dimMeta_1] \elem  \not \in \pushIn {\vset_1}}\\
&\hspace{15pt} \cup \setDef {\annot [\dimMeta_1 \vee \dimMeta_2] \elem \myOR 
\annot [\dimMeta_1] \elem \in \pushIn {\vset_1}, \annot [\dimMeta_2] \elem \in \pushIn {\vset_2}}\\
\vset_1 \cap \vset_2 &= \setDef {
\annot [\dimMeta_1 \wedge \dimMeta_2 ]\elem \myOR
\annot [\dimMeta_1] \elem \in \vset_1, \annot [\dimMeta_2] \elem \in \vset_2,
\sat {\dimMeta_1 \wedge \dimMeta_2}}\\
\vset_1 \times \vset_2 &= \setDef{
\annot [\dimMeta_1 \wedge \dimMeta_2] {(\elem_1, \elem_2)} \myOR
\annot [\dimMeta_1] \elem_1 \in \vset_1, \annot [\dimMeta_2] \elem_2 \in \vset_2}\\
\vset_1 \equiv \vset_2 &\textit{ iff }
\forall \annot  \elem \in (\filterSet {\vset_1} \cup \filterSet {\vset_2}),
\annot [\dimMeta_1] \elem \in \filterSet {\vset_1}, \annot [\dimMeta_2] \elem \in \filterSet {\vset_2}.
\dimMeta_1 \equiv \dimMeta_2, \dimMeta \equiv \dimMeta_1
\end{alignat*}

%%\begin{syntax}
%%\synDef \fName \fSet &&&\textit{Feature Name}\\
%%\synDef \bTag \bSet &\eqq& \t \myOR \f & \textit{Boolean Value}\\
%%\synDef \dimMeta \ffSet &\eqq& \bTag \myOR \fName \myOR \neg \dimMeta \myOR \dimMeta \wedge \dimMeta \myOR \dimMeta \vee \dimMeta & \textit{Feature Expression}\\
%%\synDef \config \confSet &:& \fSet \to \bSet &\textit{Configuration}
%%\end{syntax}
%
%\medskip
%\textbf{Evaluation of feature expressions:}
%\begin{alignat*}{1}
%%\fSem [] . &: \ffSet \to \confSet \to \bSet\\
%%\fSem \bTag &= \bTag\\
%%\fSem \fName &= \config \ \fName\\
%%\fSem {\neg \fName} &= \neg \fSem \fName\\
%%\fSem {\annd \dimMeta} &= \fSem {\dimMeta_1} \wedge \fSem {\dimMeta_2}\\
%%\fSem {\orr \dimMeta} &= \fSem {\dimMeta_1} \vee \fSem {\dimMeta_2}\\
%\end{alignat*}
%
%\medskip
%\textbf{Relations over feature expressions:}
%%\begin{alignat*}{1}
%%\dimMeta_1 \equiv \dimMeta_2 \textit{ iff \ } \forall \config \in \confSet. \fSem {\dimMeta_1} = \fSem {\dimMeta_2}\\
%%\sat {\dimMeta} \textit{ iff \ } \exists \config \in \confSet. \fSem {\dimMeta} = \t\\
%%\unsat {\dimMeta} \textit{ iff \ } \forall \config \in \confSet. \fSem {\dimMeta} = \f
%%%\dimMeta_1 \oplus \dimMeta_2 = (\dimMeta_1 \wedge \neg \dimMeta_2) \vee (\dimMeta_2 \wedge \neg \dimMeta_1)
%\end{alignat*}
%
%\medskip
%\textbf{Syntactic sugar for mutually exclusive features:}
%\begin{alignat*}{1}
%%\A \oplus \B \oplus \ldots \oplus \fName_n
%%= (\A \wedge \neg \B \wedge \ldots \wedge \neg \fName_n)\\
%%\vee (\neg \A \wedge \B \wedge \ldots \wedge \neg \fName_n)\\
%%\vee (\neg \A \wedge \neg \B \wedge \ldots \wedge \fName_n)
%\end{alignat*}


\caption{Configuration of variational set and annotated variational set, normalization of variational sets and annotated variational sets, and operations over variational sets. The operations on variational sets
are overloaded, that is, $\cup$ is used to both denote the union of variational sets and plain sets.
}
\label{fig:vset}
\end{figure}


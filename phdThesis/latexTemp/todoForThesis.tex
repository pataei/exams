\TODO{define bold face uppercase set symbols. follow the alice book.}

\TODO{define semantics of VRA.
define var-pres at sem level and prove it. }

\TODO{formally define RA type system.}

\TODO{have an example that shows an empty list of attributes projected from a subq is valid in both vra and ra.}

\TODO{define annot vset config formally}

\TODO{define vset config formally}

\TODO{include grouping of vq in impl}

intro:
\TODO{add that DB community misses the bigger picture s.t. it conveys that
they missed the concept of variation. also remember to add instances that 
are combination of other instances or are unknown and more recent. 
plus, mention that the needs even in the same situation/instance may 
vary and the especialized systems don't provide all the information needs.
e.g., I may have a data integration system that I want to know where the data
is coming from. I cannot do this with the current status quo (data integration
and data provenance).}

prelim:
\TODO{make sure all the \t\ and \f\ are in the same font and style across the paper and thesis.}

subsection rel db:
 \TODO{add definition of values. 
add constraints of the database definition. e.g., attributes of a relation schema are pairwise disjoint.
also, att(v) isn't unique. look into where you're using att(v) and either use index instead of value.
also, mention that attribute names are unique in our theory although we don't enforce this in our 
implementation. values are from domain of attribute.}

\TODO{define cells both here and in thesis.}

\TODO{for thesis remember to mention that set operations must have the same 
relation schema and not just set of attributes. }

\TODO{look into the variables you're using, esp, for the micro example and make sure
there's no conflict with the variable names that you're using.}

subsection encode variability:
\TODO{ make sure that you have the formal definition in thesis. and make sure that
your description of this function in the paper suffices (i.e., the reader will understand
what it actually does.)}

\TODO{define \ensuremath{\oplus} formally in thesis}

\TODO{pc(sth) the sth should be annotate, i.e., not just 2 but 2 with its annotation.
correct this across the paper. make sure you stay consistent in thesis.}

subsection vset:
\TODO{remind readers that f is feature name and not feature expression.}

\TODO{have an explicit function filter that drops elements with pres cond false.}

vdb:
\TODO{fig 2. define the type of V[] (thesis + vldb). define bold R underline and bold U underline 
with empty. also define empty mathematically. more on how to do this after discussing with
Eric. }

subsection vsch:
\TODO{define the v-schema in a figure in thesis.}

\TODO{define bold face A, and other sets as well, both here and in thesis}

vq:
\TODO{make sure all queries are clear. both thesis and vldb.
ex of unclear query is q7 in prelim. }

subsection vra:
\TODO{we need to address what is projection empty from r? is it type-ill or is it empty? I think we agreed on empty in my prelim!}

\TODO{state that C[[q]] has the assumption that the query is type-correct. }

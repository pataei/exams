\subsection{Motivating Example}
\label{sec:motex}


%%%\point{Database-backed software produced by SPL is an instance
%%%of variation in databases that is poorly managed in practice.}
%%%Explains SPL briefly and how variation appears in databases used to
%%%store data for software produced by SPL. As well as how poorly it is 
%%%managed in practice.
%Database-backed software produced by software product line (SPL) 
%is an example variation in databases in space and is \revised{partially}
%supported. 
%SPL is an
%approach to developing and maintaining software-intensive systems 
%in a cost-effective, easy to maintain manner by accommodating variation
%in the software that is being reused~\cite{splBook}. 
%An SPL produces multiple software products for different business requirements and environments.
%%%The products of a SPL pertain to a
%%%common application domain or business goal. 
%%%They also have a common
%%%managed set of features that describe the specific need for a product. 
%%%They share 
%%%a common codebase which is used to produce a product with respect to its set of 
%%%selected (enabled) features
%%In SPL, a common codebase is shared and used to produce products w.r.t.
%%a set of selected (enabled) features~\cite{splBook}. 
%%% from the SPL feature set designed to describe
%%%specific needs for products~\cite{splBook}. 
%%Different products of an SPL typically have different
%%sets of enabled features or are tailored to run in different environments. 
%These
%differences impose different data requirements which creates variation in space 
%in the shared database used in the common codebase. 
%%Hence, 
%%each product has its own database variant.
%%For example, different legal
%%requirements often require tracking different data in products tailored for use
%%in different countries or regions~\cite{splBook}.
%%Different data requirements results in different desired data which 
%%creates database variants. 
%%
%%
%%
%\revised{
%Since the existing data variation models developed in the databases community
%do not align with all of the variation scenarios that arise in software, SPL
%researchers have identified the need for more general encodings of variation in
%data models and database schema.
%%
%%Since the variation in this case is mainly exclusion/inclusion of attributes/relations
%%for different variants of the software~\cite{ATW18poly}, researchers
%%%the database variants corresponding to software products differ mainly 
%%in their schema, a relation/attribute can either be included or excluded for a 
%%specific software product~\cite{ATW18poly}.
%%
% To this end, they 
% have developed
%encodings of data models that allow for arbitrary variation by annotating
%different elements of the model with features from the
%SPL~\cite{skrhas09DBIS,slrs12CAiSE,ad11varDataModel}.
%%
%However, these solutions address \emph{only} variation in the data model but do not
%extend to the level of the data or queries. The lack of variation support in
%queries leads to unsafe techniques such as encoding different variants of query
%through string munging, while the lack of variation support in data precludes
%testing with multiple variants of a database at once.
%}
%%The variation is in the form of exclusion/inclusion of tables/attributes based on
%%selected features for a product~\cite{ATW18poly}.
%%%
%%In practice, software systems produced by a SPL are accommodated with a database that
%%has all attributes and tables available for all variants-- a database with universal schema~\cite{ATW18poly}. 
%%Unfortunately, this approach is
%%inefficient, error-prone, and filled with lots of null values since not all attributes and tables
%%are valid for all variant products. A possible solution to this could be defining views on 
%%the universal database per software variant and write queries for each variant against its 
%%view~\cite{ATW18poly}.
%%However, this is burdensome, expensive, and costly to maintain since it 
%%requires developers to generate and maintain numerous view definitions
%%in addition to manually generating
%%and managing the mappings between views and the universal schema for each product.


\begin{table}[htbp]
\caption{Schema variants of the employee database developed for multiple software variants by an SPL. Note that an \educational\ database variant must contain a \basic\ database variant too. 
%Employee schema evolution of a database for a SPL.
%%(evolution features \vOne -- \vFive\ for employee part of the schema (the left schema column), evolution features \tOne -- \tFive\ for education part
%%of the schema (the right schema column), and \edu\ feature representing the education feature of SPL). 
%A feature (a boolean variable) represents 
%inclusion/exclusion of tables/attributes.  
%The dash-underlined attributes in the basic schema 
%have \emph{variational} present in their relation, i.e.,
%they only exist in their relation when \edu\ = \t.
%%in left schema column only exist when the \edu\ feature of SPL is enabled. 
%%This table represents 1056 possible schema variants:
%%\ensuremath{2^5} schema variants when \edu\ is disabled 
%%(by enabling any combination of \vOne--\vFive) 
%%in addition to \ensuremath{2^5 \times 2^5} when \edu\ is enabled.
}
\label{tab:mot}
\centering
\small
%\footnotesize
%\scriptsize
\begin{subtable}[t]{\textwidth}
\centering
\caption{Database schema variants for \basic\ software variants.}
\label{tab:mot-basic}
\begin{tabular} {| c | l |}
\hline
\textbf{Temporal Features} & \textbf{\basic\ Database Schema Variants}\\
\hline
\multirow{3}{*}{\vOne} &  \engemp\ (\empno, \name, \hiredate, \titleatt, \deptname) \\
& \othemp\ (\empno, \name, \hiredate, \titleatt, \deptname) \\
& \job\ (\titleatt, \salary) \\
\hline
\multirow{2}{*}{\vTwo} & \cellcolor{yellow}{\empacct\ (\empno, \name, \hiredate, \titleatt, \deptname)}\\
& \cellcolor{yellow}{\job\ (\titleatt, \salary)} \\
\hline
\multirow{4}{*}{\vThree} & \empacct\ (\empno, \name, \hiredate, \titleatt, \deptno)\\
& \job\ (\titleatt, \salary) \\
& \dept\ (\deptname, \deptno, \managerno) \\
& \empbio\ (\empno, \sex, \birthdate)\\
\hline
\multirow{4}{*}{\vFour} & \empacct\ (\empno, \hiredate, \titleatt, \deptno, \dashuline{\isstudent}, \dashuline{\isteacher}) \\
& \job\ (\titleatt, \salary) \\
& \dept\ (\deptname, \deptno, \managerno) \\
& \empbio\ (\empno, \sex, \birthdate, \name) \\
\hline
\multirow{4}{*}{\vFive} & \empacct\ (\empno, \hiredate, \titleatt, \deptno,  \dashuline{\isstudent}, \dashuline{\isteacher}, \salary)\\
& \dept\ (\deptname, \deptno, \managerno,  \dashuline{\studentnum}, \dashuline{\teachernum}) \\
& \empbio\ (\empno, \sex, \birthdate, \fname, \lname) \\
\hline
\end{tabular}
\end{subtable}

\medskip
\medskip
\medskip
\begin{subtable}[t]{\textwidth}
\centering
\caption{Database schema variants for \educational\ software variants.}
\label{tab:mot-edu}
\begin{tabular} {| c | l |}
\hline
\textbf{Temporal Feature} & \textbf{\educational\ Database Schema Variants }\\
%\cline{2-3}
%\textbf{\tiny Features} & \multicolumn{1}{ c ||} {\basic} & \multicolumn{1}{ c |} {\educational} & \textbf{\tiny Features}\\
\hline
%\cline{2-3}
%\cline{2-3}
%\hhline{-==}
\multirow{2}{*}{\tOne} & \course\ (\cname, \tno)\\
& \student\ (\sno, \cname) \\
\hline
\multirow{2}{*}{\tTwo} & \course\ (\cno, \cname, \tno)\\
%\cdashline{2-3}
& \student\ (\sno, \cno)\\
\hline
\multirow{3}{*}{\tThree} & \cellcolor{yellow}{\course\ (\cno, \cname)} \\
& \cellcolor{yellow}{\teach\ (\tno, \cno)} \\
& \cellcolor{yellow}{\student\ (\sno, \cno, \grade)} \\
\hline
\multirow{4}{*}{\tFour} & \ecourse\ (\cno, \cname) \\
& \course\ (\cno, \cname, \timeatt, \class) \\
& \teach\ (\tno, \cno) \\
& \student\ (\sno, \cno, \grade) \\
\hline
\multirow{4}{*}{\tFive}  & \ecourse\ (\cno, \cname, \deptno) \\
& \course\ (\cno, \cname, \timeatt, \class, \deptno) \\
& \teach\ (\tno, \cno)\\
& \take\ (\sno, \cno, \grade)\\
\hline
\end{tabular}
\end{subtable}
\end{table}



In this section, we motivate VDBMS by illustrating the scenario described in
\secref{mot}, where two kinds of database variation interact, producing a
scenario that is not well-supported by any existing tools. The scenario
involves the evolution over time of a database-backed SPL.

% the interaction of two kinds of variation in databases 
% resulting in a new kind:
% database-backed software produced by an SPL and schema evolution.

An SPL uses a set of boolean variables called \emph{features} 
to indicate functionality that may be included or not in each software
variant.
Consider an SPL that generates management software for companies. 
It has a feature \edu\ that 
indicates whether a company
provides educational resources, such as courses for its 
employees.
%\footnote{In practice, such a SPL would have two features: \base\ and \edu,
%where \base\ is an arbitrary feature, i.e., for all variants it must be enabled, and it 
%indicates software variants that provide just basic functionalities. For simplicity
%and without loss of generality, we drop this feature.}. 
%NOTE: YOU CAN OMIT THE FOOTNOTE IF YOU'RE RUNNING OUT OF SPACE!!!!!!!
Software variants in which \edu\ is enabled (i.e., $\edu=\t$) provide this
additional functionality while variants where it is disabled provide only the
basic functionality.


If \edu\ is the only optional feature, then at any point in time, this SPL has
two variants: \basic\ and \educational. However, each variant will also evolve
as the SPL evolves, leading to several different \basic\ and \educational\
variants over time.


%\point{Database of this SPL.}
Each variant of the SPL needs a database to store information
about employees, and the selection of features impacts the database: While
\basic\ variants do not need to store any education-related records,
\educational\ variants do. 
% In practice, SPL developers use 
%only one database for both variant categories~\cite{ATW18poly}, 
%which can easily be separated
%by using the \edu\ feature.
We visualize the impact of both features and evolution on the schema in
\tabref{mot}, where feature variation is captured in the columns and variation
over time is captured in the rows.
%
% It has two schema types: \basic\ and \educational.
%
Each \basic\ schema variant contains only the schema in a cell in column \basic\
of \tabref{mot-basic} while an \educational\ schema variant consists of two sub-schemas: one from the \basic\
column of \tabref{mot-basic} and another from the \educational\ column of \tabref{mot-edu}.
%
The \basic\ sub-schema and the \educational\ sub-schema may vary over time
independently. For example, an extension to the \educational\ sub-schema may
use an older version of the \basic\ schema. This is reflected by the
highlighted cells in \tabref{mot}, which describes a complete schema for a
particular \educational\ variant.
%
To describe variation over time of each sub-schema, we introduce two disjoint
sets of temporal features (boolean variables, like any other feature), shown in
the left columns of \tabref{mot-basic} and \tabref{mot-edu}.
%the leftmost and rightmost columns.


%Note that these sub-schemas do not 
%have to be adjunct cells.

% Rows of \tabref{mot} indicate the evolution of schema variants over time.
% To capture the evolution of the software and its database we add 
% two disjoint sets of features which again are boolean variables.
% % and call them
% %\emph{temporal features}. 
% The temporal feature sets are disjoint to allow 
% individual paces for the evolution of each type of schema, e.g.
% yellow cells of \tabref{mot} show a valid schema variant even though
% the \basic\ and \educational\ sub-schemas are not in the same row. 
%
%For example, a valid software variant 
%can have the \basic\ schema associated with \vThree\ in addition
%to the \educational\ schema associated with \tFour. 
%The following scenario reiterates 
%these concepts through an example.
%Hence, we introduce \emph{temporal features} to tag schemas when they change over time.
%A \basic\ schema is associated with a temporal feature of \vOne\ -- \vFive\ while 
%an \educational\ schema is associated with a temporal feature of \tOne\ -- \tFive.
%We have two sets of temporal features because when \edu\ is enabled
%any \educational\ and \basic\ schemas can be grouped to form a complete
%schema.
%For example, a valid software variant 
%can have the \basic\ schema associated with \vThree\ and
%the \educational\ schema associated with \tFour. 
%%\point{schemas could evolve independent from each other.}
%The \educational\ schema evolves independently from the 
%\basic\ schema. For example, due to new requirements SPL DBAs need to 
%add online courses to the schema. Schema variant associated with
%\tFour\ demonstrate this change. That is why we need to have
%two sets of temporal features.
%For example, a valid software variant 
%can have the \basic\ schema associated with \vThree\ in addition
%to the \educational\ schema associated with \tFour. 
%
%the \educational\ category 
%includes tables from the \basic\ schema in addition to ones from the \educational\ schema 
%while the \basic\
%category only consists of its own tables). Separating these database
%variants using SPL features results in much cleaner databases as
%opposed to databases created for SPL in practice.

%\point{Component evolution.}
Now, consider the following scenario:
In the initial design of the \basic\ database, the DBA
settles on three tables \engemp, \othemp, and \job, shown in \tabref{mot-basic} and associated
with feature \vOne. 
After some time, they decide
to refactor the schema to remove redundant tables,
combining the two
relations \engemp\ and \othemp\ into one, \empacct, associated with feature \vTwo. 
Since some clients' 
software relies on a previous design, the two schemas have to coexist in parallel.
%However, to capture what schema a software variant is using we introduce
%temporal features, e.g., \vOne\ and \vTwo, and tag corresponding schemas 
%with these features. 
Therefore, the existence (presence) of \engemp\ and \othemp\
relations is \emph{variational}, they only exist in the \basic\
schema when $\vOne=\t$.
This scenario is an example of \emph{component evolution} in SPLs, where
%database evolution in SPL resulted from 
developers
update, refactor, and improve components of the SPL, including the database~\cite{dbSPLevolve}.
% is called 
%\emph{component evolution}~\cite{dbSPLevolve}.

%\point{Product evolution.}
Now, consider the case where a client that previously requested a \basic\ variant of the
software has added courses to educate its
employees in specific subjects. The SPL developer needs to enable
the \edu\ feature for this client, forcing the adjustment of the schema variant
to include the \educational\ sub-schema.
This case describes \emph{product evolution}, where
database evolution in an SPL results
from clients adding/removing features/components~\cite{dbSPLevolve}. 
%is called 
%\emph{product evolution}~\cite{dbSPLevolve}. 

%\point{\edu\ affects basic too.}
The situation is further complicated, since the \basic\ and \educational\
schemas are interdependent.
%not independent of each other: 
Consider
%enabling the \edu\ feature not only adds the relations in the \educational\
%schema but it also affects the \basic\ schema. 
the \basic\ schema 
variant for feature \vFour. Attributes \isstudent\ and \isteacher\ only exist
in the \empacct\ relation when $\edu=\t$, represented by a \dashuline{dash-underline},
otherwise the \empacct\
relation has only the attributes \empno, \hiredate, \titleatt, and \deptno.
Hence, the attributes \isstudent\ and \isteacher\ in \empacct\ relation are
\emph{variational}, that is, they only exist in \empacct\ relation
when $\edu=\t$.
%similar to the presence of \engemp\ which
%is variational depending on whether \vOne is enabled or not. 

%%\point{schemas could evolve independent from each other.}
%The \educational\ schema evolves independently from the 
%\basic\ schema. For example, due to new requirements SPL DBAs need to 
%add online courses to the schema. Schema variant associated with
%\tFour\ demonstrate this change. That is why we need to have
%two sets of temporal features.
%For example, a valid software variant 
%can have the \basic\ schema associated with \vThree\ in addition
%to the \educational\ schema associated with \tFour. 

Our example demonstrates how different kinds of variation
% in time and space
interact with each other, an inevitable consequence of modern software
development.
The described interaction is similar to a recent scenario we discussed with an
industry contact in \secref{industryex}.

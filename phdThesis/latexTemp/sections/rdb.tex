\section{Relational Databases}
\label{sec:rdb}

\TODO{add examples of tables.}
\TODO{add figure of all definitions}

%\point{database schema.}
A relational database \pDB\ stores information in a structured manner by forcing
data to conform to a \emph{schema} \pSch\ that is a finite set 
$\setDef {{\pRelSch}_1, \ldots, {\pRelSch}_n}$ of \emph{relation schemas}.
A relation schema is defined as
$\pRelSch = \vRel \paran {\vi \pAtt k}$ where each $\pAtt_i \in \pAttSet$ is an
\emph{attribute} contained in a relation named \vRel. 
Note that there is a
total order $\leq_{\pAttSet}$ on \pAttSet\ and every listed set of attributes is written according
to $\leq_{\pAttSet}$.
%unless otherwise specified. 
For theoretical development, it suffices to use the same domain of values
for all of the attributes. Thus, we fix a countably infinite set \emph{domain} \domSet.
%(disjoint from \pAttSet).
A \emph{value} and a \emph{constant} are elements of \domSet.

%Then, $\getRel {\pAtt}$ returns the relation that contains the 
%attribute.

%\type [\pAtt]\ returns the \emph{type} of values associated with attribute \pAtt.

%\point{database content.}
The content of database \pDB\ is stored in the form of \emph{tuples}. A tuple \pTuple\
is a mapping between an ordered set of attributes and their values,
%\footnote{\TODO{\*we can probably remove this if run out of space! \*Note that we only have
%pure relational values, thus, values do not have variational counterparts. Hence,
%we do not use our convention of underlying them to show they are variational.}}, 
i.e.,
$\pTuple = \paran {\vi {\underline v} k}$ for the relation schema \vRel \paran {\vi \pAtt k}.
Hence a \emph{relation content}, \pRelCont, is a set of tuples \setDef {\vi \pTuple m}.
$\getAtt {i}$ returns the attribute corresponding to
index $i$ in a tuple.
A \emph{table} \ensuremath{\pTab = \left(\pRelSch, \pRelCont\right)} is a pair of relation schema and relation content.
A \emph{database instance}, \pInst, of the database \pDB\ with the
schema \pSch, is a set of tables $\setDef {\pTab_1, \ldots, \pTab_n}$.
%relation contents 
%$\setDef  {{\pRelCont}_1, \ldots, {\pRelCont}_n}$ corresponding
%to a set of relation schemas $\setDef {{\pRelSch}_1, \ldots, {\pRelSch}_n}$ 
%defined in \pSch. 
For brevity, when it is clear from the context we refer to a database instance
by \emph{database}.


%\point{database schema.}
A relational database \pDB\ stores information in a structured manner by forcing
data to conform to a \emph{schema} \pSch\ that is a finite set 
$\setDef {{\pRelSch}_1, \ldots, {\pRelSch}_n}$ of \emph{relation schemas}.
A relation schema is defined as
$\pRelSch = \vRel \paran {\vi \pAtt k}$ where each $\pAtt_i$ is an
\emph{attribute} contained in a relation named \vRel. 
$\getRel {\pAtt}$ returns the relation that contains the attribute.
\type [\pAtt]\ returns the \emph{type} of values associated with attribute \pAtt.

%\point{database content.}
The content of database \pDB\ is stored in the form of \emph{tuples}. A tuple \pTuple\
is a mapping between a list of relation schema attributes and their values,
%\footnote{\TODO{\*we can probably remove this if run out of space! \*Note that we only have
%pure relational values, thus, values do not have variational counterparts. Hence,
%we do not use our convention of underlying them to show they are variational.}}, 
i.e.,
$\pTuple = \paran {\vi {\underline v} k}$ for the relation schema \vRel \paran {\vi \pAtt k}.
Hence a \emph{relation content}, \pRelCont, is a set of tuples \setDef {\vi \pTuple m}.
$\getAtt {\underline v}$ returns the attribute the value corresponds to.
A \emph{table} \pTab\ is a pair of relation content and relation schema.
A \emph{database instance}, \pInst, of the database \pDB\ with the
schema \pSch, is a set of tables $\setDef {\pTab_1, \ldots, \pTab_n}$.
%relation contents 
%$\setDef  {{\pRelCont}_1, \ldots, {\pRelCont}_n}$ corresponding
%to a set of relation schemas $\setDef {{\pRelSch}_1, \ldots, {\pRelSch}_n}$ 
%defined in \pSch. 
For brevity, when it is clear from the context we refer to a database instance
by \emph{database}.



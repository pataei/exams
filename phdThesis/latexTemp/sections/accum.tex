\subsection{Accumulation of Relational Tables to a Variational Table}
\label{sec:accum}

\eric{pls read this entire section. thx!}
After connecting variational queries to relational queries, to define the 
semantics of VRA we need to connect
the results of multiple relational queries to the result of a single variational 
query. 
%
Since we have two approaches to connect a variational query to relational queries 
we define two \emph{accumulation} functions that generate a 
variational table from a set of relational tables. 
%
The first accumulation function $\mathit{accum} : \settype \fSet \totype \settype {\typepair \confSet \pTabSet} \totype \tabletype$ takes the feature space of a database and a set of relational
tables with their attached configurations and generates a variational table. \figref{accum1} 
defines this function in terms of some auxiliary functions. 
%
The $\mathit{mkTable}$ function takes a variational relation schema and a set of 
variational relation contents and generates a variational table that has the given schema
and the variational tuples in the input tables. 
%
The $\mathit{addPresCondToConfTables}$ function maps the $\mathit{addPresCondToConfContent}$
over a set of tables and their attached configuration and the  $\mathit{addPresCondToConfContent}$
function adds the \pcatt\ attribute to a relational table and its corresponding value which is 
a feature expression associated with the given configuration using the closed set of
features.
%
The $\mathit{fitConfTablesToVsch}$ maps the function $\mathit{fitTableToVsch}$ to tables of a set of 
relational tables and their attached configuration.
The $\mathit{fitTableToVsch}$ function adjusts a table, both its schema and content, 
to a variational relation schema.
%
The $\mathit{tablesToVsch}$ maps the function $\mathit{schToVsch}$ to a set of 
relational tables and their attached configuration. 
The $\mathit{schToVsch}$ generates a variational relation schema from a set of
plain relation schema and their attached configuration given the closed set of 
features of the database's feature space.
%
Note that to generate a feature expression from a configuration it is essential to
pass the closed set of features.
%
\exref{acc-table-from-conf} illustrates the behavior of these auxiliary functions and the
table accumulation function over the relational tables in \tabref{vq-conf-res}.


\begin{figure}

\textbf{Table accumulation function:}
\begin{alignat*}{1}
\mathit{accum} &: \settype \fSet \totype \settype {\typepair \confSet \pTabSet} \totype \tabletype\\
\mathit{accum} \  \mathit{fs} \ \mathit{ts} &= \mathit{mkTable} \ \mathit{vsch} \ \mathit{tables}\\
%&\hspace{60pt} (\mathit{addPresCondToConfTables} \ \mathit{fs} \\
%&\hspace{140pt} (\mathit{fitConfTablesToVsch} \ \mathit{ts} \ \mathit{vsch}))\\
&\hspace{-40pt}\textit{where }
\mathit{vsch} = \mathit{tablesToVsch} \ \mathit{fs} \ \mathit{ts}\\
&\hspace{-6pt} \mathit{tables} = \mathit{addPresCondToConfTables} \ \mathit{fs} \ \mathit {fitted}\\
&\hspace{-6pt} \mathit{fitted} = \mathit{fitConfTablesToVsch} \ \mathit{ts} \ \mathit{vsch}
\end{alignat*}


\medskip 
\textbf{Auxiliary functions for table accumulation:}
\footnotesize
\begin{alignat*}{1}
\mathit{schToVsch} &: \settype \fSet \totype \settype {\typepair \confSet \pRelSchSet} \totype \vRelSchSet\\
\mathit{tablesToVsch} &: \settype \fSet \totype \settype {\typepair \confSet \pTabSet} \totype \vRelSchSet\\
\mathit{fitTableToVsch} &: \pTabSet \totype \vRelSchSet \totype \pTabSet\\
\mathit{fitConfTablesToVsch} &: \settype {\typepair \confSet \pTabSet} \totype \vRelSchSet \totype \settype {\typepair \confSet \pTabSet}\\
\mathit{addPresCondToConfContent} &: \settype \fSet \totype \typepair \confSet \pRelContSet \totype \vRelContSet\\
\mathit{addPresCondToConfTables} &: \settype \fSet \totype \settype {\typepair \confSet \pTabSet} \totype \settype \vRelContSet\\
\mathit{mkTable} &: \vRelSchSet \totype \settype \vRelContSet \totype \tabletype
\end{alignat*}


\caption[Accumulation function of a set of relational tables with their attached configuration into a variational table]{Accumulation function of a set of relational tables with their attached configuration into a variational table and its auxiliary functions. The definition uses spaces to pass parameters. For example, $f \ x$ states that the parameter $x$ is passed to the function $x$ and $f\ x\ y$ states that
parameters $x$ and $y$ are passed to $f$ as the first and second arguments, respectively.
}
\label{fig:accum1}
\end{figure}



\begin{example}
\label{eg:acc-table-from-conf}
Consider the query $\VVal {\vQ_1}$ written over the VDB with variational schema $\vSch_2$ and 
feature space $\fSet = \setDef {\vThree, \vFour, \vFive}$, all given in \exref{vq-specific}. 
%
All configured relational queries of $\VVal {\vQ_1}$ for VDDB's valid configurations and their
corresponding results in form of a relational table are given
in \exref{vq-sem} and \tabref{vq-conf-res}, respectively. 
%
Now we show how the relational tables of the configured queries, shown in \tabref{}vq-conf-res, are accumulated 
to the variational table, shown in \tabref{vq1-res}, as the result of the variational query $\VVal \vQ_1$ by 
using the table accumulation function $\mathit{accum}$.
%
%\tabref{}
\end{example}

The second accumulation function
 $\VVal {\mathit{accum}} :  \settype {\bm{(}\vartype \pTabSet\bm{)}} \totype \tabletype$ 
 takes a set of relational tables that are annotated with
a feature expression instead of their attached configuration. \figref{accum2} defines
this function and its auxiliary functions. The auxiliary functions are similar to the ones
defined in \figref{accum2} except that they do not need to generate a feature expression
from a configuration and a set of closed features.
%
\exref{acc-table-from-group} illustrates the behavior of these auxiliary functions and the second accumulation
function over the tables in \tabref{vq-conf-res}.

\begin{figure}

\textbf{Table accumulation function:}
\begin{alignat*}{1}
\VVal {\mathit{accum}} &:  \settype {\bm{(}\vartype \pTabSet\bm{)}} \totype \tabletype\\
\VVal {\mathit{accum}} \  \mathit{fs} \ \mathit{ts} &= \mathit{mkTable} \ \mathit{vsch} \ \mathit{tables}\\
&\hspace{-40pt}\textit{where }
\mathit{vsch} = \mathit{annotTablesToVsch}  \ \mathit{ts}\\
&\hspace{-6pt} \mathit{tables} = \mathit{addPresCondToVarTables} \ \mathit {fitted}\\
&\hspace{-6pt} \mathit{fitted} = \mathit{fitVarTablesToVsch} \ \mathit{ts} \ \mathit{vsch}
\end{alignat*}


\medskip 
\textbf{Auxiliary functions for table accumulation:}
\footnotesize
\begin{alignat*}{1}
\mathit{annotSchToVsch} &:  \settype {\bm{(}\vartype \pRelSchSet\bm{)}} \totype \vRelSchSet\\
\mathit{annotTablesToVsch} &:  \settype {\bm{(}\vartype \pTabSet\bm{)}} \totype \vRelSchSet\\
%\mathit{fitTableToVsch} &: \pTabSet \totype \vRelSchSet \totype \pTabSet\\
\mathit{fitVarTablesToVsch} &: \settype {\bm{(}\vartype \pTabSet\bm{)}} \totype \vRelSchSet \totype \settype {\bm{(}\vartype \pTabSet\bm{)}}\\
\mathit{addPresCondToVarContent} &:  \vartype \pRelContSet \totype \vRelContSet\\
\mathit{addPresCondToVarTables} &:  \settype {\bm{(}\vartype \pTabSet\bm{)}} \totype \settype \vRelContSet
%\mathit{mkTable} &: \vRelSchSet \totype \settype \vRelContSet \totype \tabletype
\end{alignat*}


\caption[Accumulation function of a set of relational tables annotated with a feature expression into a variational table]{Accumulation function of a set of relational tables annotated with a feature expression into a variational table and its auxiliary functions. The definition uses spaces to pass parameters, e.g., $f \ x = f(x)$ and $f \ x \ y = f(x,y)$.
}
\label{fig:accum1}
\end{figure}



\begin{example}
\label{eg:acc-table-from-group}
\wrrite{write this}
\end{example}
\chapter{Variational Queries}
\label{ch:vql}

\eric{start reading here pls}
Now that we have introduced the variational database framework 
we need a query language to extract information from a VDB instance
and unlike relational query languages (like SQL and relational algebra)
it needs to account for the new aspect of our database: variation. 
%
\wrrite{to be written}
we use ... that has these differences with blah
%
we introduce these differences through building up a query to extract info
required by the variational intent. 
\eric{stop reading here. the rest is from vldb.}

%\point{queries need to be able to express variability encoded in vdb.}
%The variational nature of a VDB requires a query language that
%accounts for variation directly.
%To express and represent variation in queries,
%we incorporate choice calculus~\cite{Walk13thesis, EW11tosem}  into a 
%structured query language. 
\fromppr{vldb. have to read through and edit}
We now embark on the formal definition of our query language. 
We formally define 
\emph{variational relational algebra} (VRA) in \secref{vrel-alg}
as our algebraic query language.
A query written in VRA is called a \emph{variational query} (\emph{v-query});
we use query and variational query interchangeably when it is clear from context. 
Unlike relational queries that convey an intent over a single database, 
a variational query typically conveys the same intent over several 
relational database variants. However, a single variational query is also capable of capturing different 
intents over different database variants.
%Consequently, the expressiveness of variational queries may cause them to be 
%more complicated than relational queries, discussed in \secref{type-sys}. 
%Hence, 

Due to the expressiveness of variational queries, we define a type system for VRA that statically checks a
variational query against the underlying v-schema in \secref{type-sys}.
%
To make variational queries more useable we relieve the user from repeating 
the v-schema's variation in their variational queries. This is achieved by 
explicitly annotating queries in \secref{constrain}.
%In \secref{constrain}, we define an operation that explicitly annotates a
%variational query with information contained in a v-schema. 
%This operation is useful to
%define the \emph{variation-preservation} property for VRA and its type system,
%which is discussed in \secref{var-pres}, and demonstrates how our framework
%satisfies the information need \nTwo.
We then define the \emph{variation-preservation} property for VRA at
the type level in \secref{var-pres}.
% which proves that our framework
%satisfies the requirement \nTwo.
%
Finally, we provide 
%we close out this section by providing 
a set of syntactic rules that are semantic-preserving 
in \secref{var-min} that enable factoring and distributing
variation points within a variational query, which enables syntactic refactorings
including maximizing sharing within a variational query.
%for reducing a query's variation.

%\subsubsection{Variational Relational Algebra}
\label{sec:vrel-alg}


%\point{vra = cc + ra}
%Considering the variational nature of a VDB, to satisfy a user's information 
%need when extracting information, 
%we need a query language that not only considers the structure of 
%relational databases (such as SQL and relational algebra (RA)) but also 
%accounts for the variation encoded in the VDB. We achieve this by:
%1) picking relational algebra as our main query language and
%2) using \emph{choices}~\cite{Walk13thesis, EW11tosem} 
%and presence conditions to account for variation. 

To account for variation, VRA combines relational algebra (RA) with 
\emph{choices}~\cite{Walk13thesis, EW11tosem}.
%\point{choice.}
A choice $\chc{x,y}$ consists of a feature expression \dimMeta\ and 
two alternatives $x$ and $y$. For a given configuration \config, 
the choice $\chc{x, y}$ can be replaced by $x$ if \dimMeta\
evaluates to \t\ under configuration \config, (i.e., \fSem{\dimMeta}),
or $y$ otherwise. 
In essence, choices allow a v-queries
to encode variation in a structured and systematic manner. 

\begin{figure}
\begin{syntax}

\multicolumn{4}{l}{\textbf{Operators:}} \\[1ex]
\bullet
  &\eqq& \multicolumn{2}{l}{< \myOR \leq \myOR = \myOR \neq \myOR > \myOR \geq} \\
\circ
  &\eqq& \cup \myOR \cap \\[2ex]

\multicolumn{4}{l}{\textbf{Variational conditions:}} \\[1ex]
\vCond\in\vCondSet
  &\eqq&  \multicolumn{2}{l}{
          \bTag
   \myOR  \pAtt \bullet \cte
   \myOR  \pAtt \bullet \pAtt
   \myOR  \neg \vCond
   \myOR  \vCond \vee \vCond} \\
  &\myOR& \multicolumn{2}{l}{
          \vCond \wedge \vCond
   \myOR \chc{\vCond,\vCond}} \\[2ex]

\multicolumn{4}{l}{\textbf{Variational queries:}} \\[1ex]
\vQ\in\qSet
  &\eqq&  \vRel     & \textit{Relation}\\
  &\myOR& \vSel \vQ & \textit{Selection}\\
  &\myOR& \vPrj[\vAttList]{\vQ} & \textit{Projection}\\
  &\myOR& \chc{\vQ,\vQ} & \textit{Choice}\\
% &\myOR& \vQ \Join_\vCond \vQ & \textit{Variational Join}\\
  &\myOR& \vQ \times \vQ & \textit{Cartesian Product}\\
  &\myOR& \vQ \circ \vQ  & \textit{Set Operation}\\
% &\myOR& \vQ \backslash \vQ &\textit{Variational Set Difference}\\
  &\myOR& \empRel & \textit{Empty Relation}

\end{syntax}

\caption[Syntax of variational relational algebra]{Syntax of variational relational algebra.}
%\TODO{remember that
%you removed join (also removed it from query config def and constrain query 
%by schema). if you want use it just say it's a syntactic sugar.}
%$<, \leq, =, \neq, >, \geq$.
%$\circ$ denotes set operators: union and difference.}
\label{fig:v-alg-def}
\end{figure}



%\point{explain notation and VRA operations.}
The syntax of VRA is given in \figref{v-alg-def}.
%
In VRA, the selection operation is similar to standard RA selection except
that the condition parameter is \emph{variational} meaning that it may contain
choices.
For example, the query 
\ensuremath{\sigma_{\chc {\vAtt_1=\vAtt_2,\vAtt_1=\vAtt_3}} \vRel}
selects a v-tuple \vTuple\ if it satisfies
the condition \ensuremath{\vAtt_1 = \vAtt_2} 
and  \ensuremath{\sat {\dimMeta \wedge \getPC \vTuple}}
or
if \ensuremath{\vAtt_1 = \vAtt_3} 
and \ensuremath{\sat {\neg \dimMeta \wedge \getPC \vTuple }}.
%
The projection operation is parameterized by a v-set of attributes. For
example,
the query $\pi_{\vAtt_1, \optAtt [\dimMeta] [\vAtt_2]} \vRel$
projects $\vAtt_1$ from relation \vRel\ unconditionally, and $\vAtt_2$ 
when \sat{\dimMeta}.
%
The choice operation enables combining two v-queries to be used in different
variants based on a given feature expression. In practice,
it is often useful to return information in some variants and nothing at all in
others. We introduce an explicit \emph{empty} query \empRel\ to facilitate
this. The empty query is used, for example, in 
\ensuremath{\vQ_2} in \exref{vq-specific}. 
%The set operations between queries are v-set operations defined in \secref{vset}.
The rest of VRA's operations are similar to RA, where all set operations
(union, intersection, and cross product) are changed to the corresponding
variational set operations defined in \secref{vlist-vset}.
%
In examples, we also use a join operation with a variational condition,
$\vQ_1\bowtie_\vCond\vQ_2$, which is syntactic sugar for
$\sigma_\vCond(\vQ_1\times\vQ_2)$.


Our implementation of VRA also provides mechanisms for renaming queries and
qualifying attributes with relation/subquery names. These features are needed
to support self joins and projecting attributes with the same name in different
relations. However, for simplicity, we omit these features from the formal
definition in this proposal.


%A query can simply 
%refer to a relation, filter tuples based on a variational condition 
%(which is a relational condition with choices of two conditions), and
%project a variational list of attributes. Besides production of two queries and
%set operations, VRA allows for a choice of two v-queries. This demands an
%\emph{empty} query since an alternative of a choice can very well inquire 
%no information at all. 
%For example, the query $\chc {\vQ_1, \}$
%

%\subsubsection{Running a V-Query on a VDB Results in a V-Table}
%\label{sec:run-vq-get-vtab}
%A v-query systematically represents a set of relational query variants associated to their
%corresponding database variants. Hence, intuitively the user expects to 
%get such variation in their result as well. 

The result of a v-query is a v-table with the relation name $\mathit{result}$.
%
For example, assume that v-tuples $\annot[\fOne]{(1,2)}$ and $\annot[\neg
f_3]{(3,4)}$ belong to a v-relation $\vRel(\vAtt_1,\vAtt_2)$, which is the only
relation in a VDB with the trivial feature model \t.
%
The query $\chc[f_3]{\pi_{\optAtt[f_2][\vAtt_1]}\vRel,\empRel}$ returns a
v-table with relation schema $\annot[f_3]{\mathit{result}(\annot[f_2]{a_1})}$,
which indicates that the result is only non-empty when $f_3$ is true and that the
result includes attribute $a_1$ when $f_2$ is true. \secref{type-sys} defines a
type system that yields the relation schema for any well-formed query.
%
The content of the result relation is a single v-tuple $\annot[f_1]{(1)}$. The
tuple $\annot[\neg f_3]{(3)}$ is not included since the projection occurs in
the context of the choice in $f_3$, which is incompatible with the presence
condition of the tuple (i.e., $\unsat{f_3 \wedge\neg f_3}$). This illustrates
how choices can effectively filter the tuples in a VDB based on their presence
conditions.

% Although there is no need to update the presence condition of the returned
% tuples, yet choices can filter the returned v-tuples.

% Note that here the value \ensuremath{1}
% of attribute \ensuremath{\vAtt_1} is present in VDB variants where 
% \ensuremath{\sat {\A \wedge \B \wedge \C}} although the presence 
% condition of the returned v-tuple does not have to state this condition
% since 
%
% overall presence 
%condition and the presence conditions of attributes and tuples are
%restricted by the variation enforced by the query.
%
%Note that the presence condition of tuples, attributes, and the return relation
%is restricted by the variation enforced by the query. 
% correct this so that you don't conjunct the pc of relation and clarify that it's relations's pc and not the attributes. although the conjunction should be satisfiable.}
%




%%Hence, VRA is more expressive than RA 
%%because it can encode variational queries.
%The variational nature allows users to write interesting queries in many ways:
%1) to express their variational information need or to filter returned tuples
%they can use annotations or 
%choices, \exref{vq-specific},
%2) to express the same intent over several database variants they can 
%use choices in queries or conditions, \exref{vq-same-intent-mult-vars},
%and 
%3) they can also use choices to express different intents over database variants.
%\TODO{Eric, should we drop the last since it creates messy results and isn't really useful?}.
%%The expressiveness of VRA satisfies \textbf{N1}, this is illustrated in 
%%\exref{vq-specific} and \exref{vq-same-intent-mult-vars}.
%%Interestingly, VRA's expressiveness enables users to express 
%%their information need more specifically by stating the exact condition
%%under which an information need is inquired. \exref{vq-specific} illustrates this.
%%It also allows users to express the same intent over several database 
%%variants

% \NOTE{
% To express the variational information need or to filter returned tuples
% users can use annotations or choices. \exref{vq-specific} illustrates this.
% }

The following example illustrates, in the context of our running example, how
a v-query can be used to express variational information needs.

\begin{example}
\label{eg:vq-specific}
%VRA's expressiveness consequently facilitates expressing exactly the condition
%under which an information need is inquired. 
Assume a VDB with
features \vThree, \vFour, and \vFive, and
the corresponding \empbio\ schema variants in \tabref{mot}. 
The v-schema for this VDB is:
%
\begin{align*}
\vSch_2 &=
\{\empbio (\empno, \sex, \birthdate,
\optAtt [\vFour] [\name], \optAtt [\vFive] [\fname],
%& \hspace{16pt} 
\optAtt [\vFive] [\lname] )\}^{\fModel_2}\\
& \hspace{4pt} \textit{where } \fModel_2 = {\vThree \oplus \vFour \oplus \vFive}.
%\left(\vThree \wedge \neg \vFour \wedge \neg \vFive\right)
%  \vee \left(\vFour \wedge \neg \vThree \wedge \neg \vFive\right) 
%   \vee \left(\vFive \wedge \neg \vThree \wedge \neg \vFour\right)}.
\end{align*}
%
The user wants the employee ID numbers (\empno) and names for variants 
\vFour\ and \vFive.
The user needs to project the \name\ attribute 
for variant \vFour, the \fname\ and \lname\ attributes for variant \vFive,
and \empno\ attribute for both variants.
This can be expressed with the following v-query.
\[
\vQ_1 = \pi_{\optAtt [\vFour \vee \vFive] [\empno], \name, \fname, \lname} \empbio
\]
\end{example}

Note that the user does not need to repeat the variability information encoded
in the v-schema in their query, that is, they do not need to annotate \name,
\fname, and \lname\ with \vFour, \vFive, and \vFive, respectively. We discuss
this in more detail in \secref{type-sys} and \secref{constrain}. $\vQ_1$
queries all three variants simultaneously although the returned results are
only associated with variants \vFour\ and \vFive\ due to the annotation of the
attribute \empno\ in the query and the presence conditions of the rest of the
projected attributes in the schema.
%
Yet, selecting only two out of the three variants can be written more
explicitly in a query by using a choice:
$\vQ_2=\chc[\vFour\vee\vFive]{\pi_{\empno,\name,\fname,\lname}\empbio,\empRel}$. 
%
Note that queries $\vQ_1$ and $\vQ_2$ return the same set of v-tuples since
neither returns tuples associated with variant \vThree, but their returned
v-tables have different presence conditions, as will be discussed later in
\exref{type}. 
%

%\NOTE{
VRA has semantic-preserving equivalence rules that allow users 
to incorporate their taste and preference of where and how
 they want to encode variation in their queries. These rules
 factor out the commonality of subqueries and generate queries
 with less variation. \secref{var-min} expands on these rules.
%}

The next example illustrates how a v-query can be used to express the same
intent over several database variants using choices and conditions. Expressing
the same intent over several instances by a single query relieves the DBA from
maintaining separate queries for different versions or configurations of the
schema.

\begin{example}
\label{eg:vq-same-intent-mult-vars}
Assume a VDB with features \vOne--\vFive\ and the corresponding \basic\ schema
variants in \tabref{mot}. The user wants to get all employee names across all
variants, which they can express by the following v-query.
%
\begin{align*}
\vQ_3 &= 
  \vOne\chcL
    \pi_{\name}\engemp \cup \pi_{\name}\othemp, \\
 & \hspace{18pt}
    (\vTwo\vee\vThree)\chcL
      \pi_{\name}\empacct, 
%      \\
% & \hspace{24pt}
      \chc[(\vFour\vee\vFive)]{\pi_{\name,\fname,\lname}\empbio, \emp}\chcR\chcR
\end{align*}
%
Since the v-schema enforces that exactly one of \vOne--\vFive\ be enabled, we
can simplify the query by omitting the final choice.
%
\begin{align*}
\vQ_4 &= 
  \vOne\chcL
    \pi_{\name}\engemp \cup \pi_{\name}\othemp, 
%    \\
% & \hspace{18pt}
    \chc[(\vTwo\vee\vThree)]{
      \pi_{\name}\empacct,
      \pi_{\name,\fname,\lname}\empbio}
\end{align*}
%
\end{example}

In principle, v-queries can also express arbitrarily different intents over
different database variants. However, we expect that v-queries are best used to
capture single (or at least related) intents that vary in their realization
since this is easier to understand and increases the potential for sharing in
both the representation and execution of a v-query.


% \subsubsection{VRA Semantics}
% \label{sec:vra-sem}

\begin{figure}
%\textbf{Configuration selection semantics of \vqsTxt:}
\begin{alignat*}{1}
\eeSem [] . &: \qSet \to \confSet \to \pQSet\\
%
\eeSem \vRel &= \orSem \vRel = \pRel\\
\eeSem {\vSel \vQ}  &= \vSel [\ecSem \vCond] {\eeSem \vQ}\\
%
\eeSem {\vPrj [\vAttList] \vQ} &= \vPrj [\olSem \vAttList] {\eeSem \vQ}\\
%
\eeSem {{\vQ_1} \times {\vQ_2}} &= \eeSem {\vQ_1} \times \eeSem {\vQ_2}\\
%
%\eeSem {{\vQ_1} \Join_\vCond {\vQ_2}} &= \eeSem {\vQ_1} \Join_{\ecSem \vCond} \eeSem {\vQ_2}\\
%
\eeSem {\chc {\vQ_1, \vQ_2}} &= 
	\begin{cases}
		\eeSem {\vQ_1}, \text{ if } \fSem \dimMeta = \t\\
		\eeSem {\vQ_2}, \text{ otherwise}
	\end{cases}\\
%
\eeSem {{\vQ_1} \circ {\vQ_2}} &= \eeSem {\vQ_1} \circ \eeSem {\vQ_2}\\
%
\eeSem {\empRel} &= \underline {\empRel}
\end{alignat*}
\caption{Configuration of VRA which assumes that the given v-query
is well-typed. 
%\orSem ., \ecSem ., and \olSem . are
%configuration of v-relation, v-condition, and variational attribute
%set, respectively, defined in \figref{vdb-conf}, 
%\figref{vcond-conf-sem}, \figref{vdb-conf}.
Note that we have extended RA with an empty relation $\underline {\empRel}$.}
\label{fig:v-alg-conf-sem}
\end{figure}


The semantics of VRA can be understood as a combination of the
\emph{configuration semantics} of VRA, defined in \figref{v-alg-conf-sem}, the
configuration semantics of VDBs, defined in \figref{vdb-conf}, and the
semantics of plain RA.
%
%\TODO{Make the following a more precise description of how these three
%semantics work together, i.e.\ for every valid configuration of the feature
%model, we can configure the v-query and VDB in the same way to yield a plain RA
%query that is then executed over the corresponding plain RDB.}
%
Thus, the v-query
semantics is the set of semantics of its configured relational queries over
their corresponding configured relational database variant for every valid
configuration of the feature model of the VDB.
%
The configuration function maps a v-query under
a given configuration
to a pure relational query, defined in \figref{v-alg-conf-sem}.
%Configuring a v-query
%for all valid configurations, accessible from VDB's feature model,
%provides the complete meaning of a v-query in terms of RA semantics.
%
Users can deploy queries for a specific variant by configuring 
them,
%The configuration of a query allows users to deploy queries for a
%specific variant when they desire, 
satisfying query part of \nThree\ need stated in \secref{mot}.

%To define VRA semantics we map 
%a v-query to a pure relational query to re-use RA's semantics.
%However, to avoid losing the variation encoded 
%in the v-query, 
%we need to determine the variant under which such a
%mapping is valid. Thus, we introduce the semantic functions that 
%relate a v-query to a relational query.

%
%\textbf{Configuring a v-query:} 
%It maps a v-query under a 
%given configuration to a relational query, denoted by \eeSem . 
%and defined in \figref{v-alg-conf-sem}. Configuring a v-query
%for all valid configurations, accessible from VDB's feature model,
%provides the complete meaning of a v-query in terms of RA semantics.
%Users can deploy queries for a specific variant by configuring 
%them,
%%The configuration of a query allows users to deploy queries for a
%%specific variant when they desire, 
%satisfying query part of \nThree.

\begin{example}
\label{eg:conf-vq}
Assume the underlying VDB has the v-schema
% \t\ feature model and the v-relation
\ensuremath{
\vSch_3 = \{ \vRel \left( \optAtt [\fOne] [\vAtt_1], \vAtt_2, \vAtt_3 \right)^{\fOne \vee \fTwo}
\}} 
and only two features \fOne\ and \fTwo.
The v-query 
\ensuremath{
\vQ_5 = \vPrj [{\vAtt_1, \optAtt [\fOne \wedge \fTwo] [\vAtt_2], \optAtt [\fTwo] [\vAtt_3]}] \vRel
}
is configured to the following relational queries:
\ensuremath{\eeSem [\setDef \fOne] {\vQ_5} = \eeSem [\setDef \ ] {\vQ_5} = \pi_{\pAtt_1} \pRel},
\ensuremath{\eeSem [\setDef \fTwo] {\vQ_5} =
 \pi_{\pAtt_1, \pAtt_3} \pRel},
\ensuremath{\eeSem [\setDef {\fOne, \fTwo}] {\vQ_5} = \pi_{\pAtt_1, \pAtt_2, \pAtt_3} \pRel}.
\end{example}




%\textbf{Expressiveness of VRA:} \point{VRA is more expressive than RA since it
%can express all queries in RA and a couple of them all together in one query.}

VRA enables querying multiple database variants encoded as a singled VDB
simultaneously and selectively, satisfying the query need state in \secref{mot}
(\textbf{N1}).
%
More precisely, VRA is \emph{maximally expressive} in the sense that it can
express any set of plain RA queries over any subset of relational database
variants encoded as a VDB. This claim is captured by the following theorem.

\begin{theorem}
\label{thm:max-expr}
Given a set of plain RA queries $\pQ_1,\ldots,\pQ_n$ where each query $\pQ_i$
is to be executed over a disjoint subset $\vdbInst_i$ of variants of the VDB
instance \vdbInst, there exists a v-query $q$ such that
$\forall c.\; \odbSem[c]{\vdbInst} = \vdbInst_i \implies \eeSem[c]{q} = \pQ_i$.
\end{theorem}

\begin{proof}
By construction. Let $f_i$ be the feature expression that uniquely
characterizes the variants in each $\vdbInst_i$.
Then $q =$ \\
\(
\chc[(f_1\wedge\neg f_2\wedge\ldots\wedge\neg f_n)]{\pQ_1,
  \chc[(f_2\wedge\ldots\wedge\neg f_n)]{\pQ_2,\ldots
    \chc[f_n]{\pQ_n,\emp}\ldots}}.
\)\\
\end{proof}

\noindent
%
The above construction relies on the fact that every RA query is a valid VRA
(sub)query in which every presence condition is true.
%
Of course, in most realistic scenarios, we expect that v-queries can be encoded
more efficiently by sharing commonalities and embedding relevant choices and
presence conditions within the v-query.


%\textbf{Grouping a v-query:} 
%maps a v-query to a set of
%relational queries annotated with feature expressions, denoted by \qGroup .
%and defined in \figref{vq-group}. The presence condition of relational queries 
%indicate the group of configurations where the mapping holds. In essence, 
%grouping of v-query \vQ\ groups together all configurations with the same relational
%query produced from configuring \vQ. 
%Hence, the generated set
%%\dropit{could drop this if it's confusing!}
%of relational queries from grouping a v-query contains distinct (unique) queries.
%For example, consider the query \ensuremath {\vQ_5} in \exref{conf-vq}.
%Grouping \ensuremath{\vQ_5} results in the set:
%\ensuremath{
%\setDef{
%\left( \pi_{\pAtt_1, \pAtt_2, \pAtt_3} \pRel \right)^{\fOne \wedge \fTwo},
%\left(\pi_{\pAtt_1, \pAtt_3} \pRel \right)^{\neg \fOne \wedge \fTwo},
%\left(  \pi_{\pAtt_1} \pRel \right)^{( \fOne \wedge \neg \fTwo) \vee (\neg \fOne \wedge \neg \fTwo)}
%}
%}.
%
%
%
%\begin{figure}
%\textbf{Configuration selection semantics of \vqsTxt:}
\begin{alignat*}{1}
\qGroup . &: \qSet \totype \settype {\bm{(} \vartype \pQSet \bm{)}}\\
%
\qGroup \vRel &= \setDef {\annot [\t] \pRel}\\
\qGroup {\vSel \vQ}  &=  
\setDef {\annot [\dimMeta \wedge \dimMeta_\vCond] {\left(\sigma_{\pCond} \pQ\right)} \myOR
\annot \pQ \in \qGroup \vQ, \annot [\dimMeta_\vCond] \pCond \in \cGroup}
\\
%
\qGroup {\vPrj [\vAttList] \vQ} &= 
\setDef {\annot [\dimMeta \wedge \dimMeta_\vAttList] {\left(\pi_{\pAttList} \pQ \right)} \myOR
\annot \pQ \in \qGroup \vQ, \annot [\dimMeta_\vAttList] \pAttList \in \aGroup}
\\
%
\qGroup {{\vQ_1} \times {\vQ_2}} &= 
\setDef {\annot [\dimMeta_1 \wedge \dimMeta_2] {\left(\pQ_1 \times \pQ_2\right)} \myOR
\annot [\dimMeta_1] \pQ_1 \in \qGroup {\vQ_1}, \annot [\dimMeta_2] \pQ_2 \in \qGroup {\vQ_2} }
\\
%
\qGroup {{\vQ_1} \Join_\vCond {\vQ_2}} &= 
\setDef {\annot [\dimMeta_1 \wedge \dimMeta_2 \wedge \dimMeta_\vCond] {\left(\pQ_1 \Join_{\pCond} \pQ_2 \right)} \myOR 
\annot [\dimMeta_1] \pQ_1 \in \qGroup {\vQ_1}, \annot [\dimMeta_2] \pQ_2 \in \qGroup {\vQ_2}
%& \hspace{104pt}
,\annot [\dimMeta_\vCond] \pCond \in \cGroup  }
\\
%
\qGroup {\chc {\vQ_1, \vQ_2}} &= 
\setDef {\annot [\dimMeta \wedge \dimMeta_1] \pQ_1 \myOR  \annot [\dimMeta_1] \pQ_1 \in \qGroup {\vQ_1} }
\cup 
\setDef {\annot [\neg \dimMeta \wedge \dimMeta_2] \pQ_2 \myOR  \annot [\dimMeta_2] \pQ_2 \in \qGroup {\vQ_2}}  \\
%
\qGroup {{\vQ_1} \circ {\vQ_2}} &= 
\setDef {\annot [\dimMeta_1 \wedge \dimMeta_2] {\left(\pQ_1 \circ \pQ_2\right)} \myOR
\annot [\dimMeta_1] \pQ_1 \in \qGroup {\vQ_1}, \annot [\dimMeta_2] \pQ_2 \in \qGroup {\vQ_2} }\\
%
\qGroup {\empRel} &= \annot [\t] { \empRel}
\end{alignat*}
\caption[Unique configuration of variational queries]{Unique configuration of variational queries. 
The unique configuration function assumes that the input is well-typed.
}
\label{fig:vq-group}
\end{figure}




%%\input{sections/validVqs}
%\subsubsection{Well-Typed (Valid) Query}
\label{sec:type-sys}

%\begin{figure}

\textbf{V-queries typing rules:}

  \begin{mathpar}
  \small
  
%  \inferrule[\judge]
 % 	{\env{\vQ}{\vType}}
 %   {}
    \inferrule[\relationE]
  	{\vRel (\vType)^{\VVal \dimMeta} \in \vSch \\
	\neg \sat{\vctx \wedge \neg \VVal \dimMeta} }
     {\envWithSchema{\envInContext [\vctx ] {\vType}}}

%implicitly-typed lang:
%    \inferrule[\relationE]
%  	{\vRel (\vType)^{\VVal \dimMeta} \in \vSch \\
%	\sat{\vctx \wedge \VVal \dimMeta} }
%     {\envWithSchema{\envInContext [\vctx \wedge \VVal \dimMeta] {\vType}}}
  
  \inferrule[\prjE]
  	{\envPrime \\
    	\subsumeExpl {\annot \vType}  {\annot [\VVal \vctx] {\VVal \vType}}}
    {\env{\vPrj[\vType] \vQ} {\envInContext [\vctx] \vType}}

%implicitly-typed lang:
%  \inferrule[\prjE]
%  	{\envPrime \\
%    	\subsume {\annot \vType}  {\annot [\VVal \vctx] {\VVal \vType}}}
%    {\env{\vPrj[\vType] \vQ} {\envInContext [\VVal \vctx] {\left(\annot {\vType} \cap {\VVal \vType} \right)}}}


  \inferrule[\selE]
  	{\env \vQ {\envInContext [\VVal \vctx] \vType} \\
    	\envCondAnnot \vCond}
    {\env{\vSel \vQ}{\envInContext [\VVal \vctx] \vType}}
    
  \inferrule[\choiceE]
  	{\envOne[\vctx \wedge \VVal \dimMeta] \\
    	\envTwo[\vctx \wedge \neg \VVal \dimMeta]}
    {\env{\chc[\VVal \dimMeta]{\vQ_1, \vQ_2}}{
     \envInContext [\vctx_1 \vee \vctx_2] {\left({\envInContext [\vctx_1] \vType_1} \cup
    							{\envInContext [\vctx_2] \vType_2}\right)}}}
    
  \inferrule[\productE]
  	{\envOne \\
    	\envTwo\\
	\annot [\vctx_1] \vType_1 \cap \annot [\vctx_2] \vType_2 = \{\}}
    {\env{\vQ_1 \times \vQ_2}{\envInContext [\vctx_1 \wedge \vctx_2] {\left(\vType_1 \cup \vType_2\right)}}}


  \inferrule[\setopE]
  	{\envOne \\
    	\envTwo \\
	\envEval {\annot [\vctx_1] \vType_1} {\annot [\vctx_2] \vType_2}}
%        \envEval{\envInContext{\vType_1}} \vType \\
%        \envEval{\envInContext{\vType_2}} \vType}
    {\env{\vQ_1 \circ \vQ_2} {\envInContext [\vctx_1] \vType_1} }

%  \inferrule[\diffE]
%  	{\envOne \\
%    	\envTwo \\
%        \envEval{\envInContext{\vType_1}} \vType \\
%        \envEval{\envInContext{\vType_2}} \vType}
%    {\env{\vQ_1 \setminus \vQ_2} \vType}
  \end{mathpar}
  
\medskip
\textbf{V-condition typing rules (b: boolean tag, \pAtt: plain attribute, k: constant value):}
%(b: boolean tag, A: plain attribute, k: constant value)}
  \begin{mathpar}
  \small    

  \inferrule[\conjC]
  	{\envCond \vCond_1\\
    	\envCond \vCond_2}
    {\envCond{\vCond_1 \wedge \vCond_2}}
    
  \inferrule[\disjC]
  	{\envCond \vCond_1\\
    	\envCond \vCond_2}
    {\envCond{\vCond_1 \vee \vCond_2}}
    


  \inferrule[\choiceC]
%  	{\defType{\relInContext{\vContext''}}\in \vSch \\
    	{\envCond[\vctx \wedge \VVal \dimMeta, \vType]{\vCond_1} \\
        \envCond[\vctx \wedge \neg \VVal \dimMeta, \vType]{\vCond_2}}
    {\envCond{\chc[\VVal \dimMeta]{\vCond_1, \vCond_2}}}
    

  \inferrule[\notC]
  	{\envCond \vCond}
    {\envCond \neg \vCond}
        
%  \inferrule[]
%  	{\envCond[\vContext \wedge \dimMeta]{\vCond_1} \\
%    	\envCond[\vContext \wedge \neg\dimMeta]{\vCond_2}}
%    {\envCond{\chc{\vCond_1, \vCond_2}}}
    

    
  \inferrule[\attValC]
  	{
	%\defType{\relInContext{\vContext'}}\in \vSch \\
    	\optAtt [\VVal \dimMeta] \in \vType \\
	\taut{{\VVal \dimMeta} \imply \vctx} \\
        \cte \in \dom \vAtt}
    {\envCond{\op \pAtt \cte}}
    
  \inferrule[\boolC]
  	{}
    {\envCond \bTag}
    

    
  \inferrule[\attAttC]
  	{
	%\defType{\relInContext{\vContext'}}\in \vSch \\
    	\optAtt [\dimMeta_1] [\vAtt_1]\in \vType \\
         {\optAtt [\dimMeta_2] [\vAtt_2]} \in \vType \\
         \taut{\dimMeta_1 \imply \vctx} \\
         \taut{\dimMeta_2 \imply \vctx} \\
        \type[\vAtt_1] = \type[ \vAtt_2]}
    {\envCond{\op{\pAtt_1}{\pAtt_2}}}
    
  \end{mathpar}

%\caption{V-condition typing relation. A v-condition \vCond\ is well-typed if 
%it is valid in the variational context \vctx\ and type environment \vType, i.e., 
%\envCond \vCond. Note that the type rules for v-conditions return a boolean, if
%the v-condition is type-correct the rules return \t, otherwise they return \f.}
\caption{Explicitly-annotated VRA and v-condition typing relation. The explicitly-annotated
query is generated by pushing the schema to the implicitly-annotated query. Hence,
all annotations are explicit and included in the query, i.e., they are at least as specific as
corresponding presence conditions encoded in the v-schema. We use 
\ensuremath{\annot \vRel} as syntactic sugar for \chc {\vRel, \empRel}.
}
\label{fig:vq-stat-sem}
\end{figure}



%\point{aspects we need to type check v-queries.}
To prevent running v-queries that have errors
we implement a \emph{static type system} for VRA. The 
type system ensures queries are \emph{well-typed}, i.e., they comply
with the underlying v-schema, both w.r.t. the traditional structure of 
the database and the variability encoded in the database. 
%
Assume we have the VDB given in \exref{conf-vq} with the only 
relation
\ensuremath{
\vRel \left( \optAtt [\fOne] [\vAtt_1], \vAtt_2, \vAtt_3 \right)^{\fOne \vee \fTwo}
}. 
Attribute \ensuremath{\vAtt_4} cannot be projected from \vRel\ because
it is not present in \vRel, thus,
the query \ensuremath{\vPrj [\vAtt_4] \vRel} is invalid.
Similarly, the query 
\ensuremath{\vPrj [{\optAtt [\neg \fOne] [\vAtt_1]}] \vRel
} has an error because \ensuremath{\vAtt_1} is not present in 
\vRel\ for 
\ensuremath{
\forall \config \in \confSet. \fSem {\neg \fOne} = \t}, but
these are the only configurations where the query desires to project attribute
\ensuremath{\vAtt_1} from \vRel.
%is invalid because the variation encoded in the query is violating the 
%corresponding variation in v-schema, i.e., the feature expression
%\ensuremath{\neg \fOne \wedge \getPC{\vAtt_1}} is not satisfiable.
%For example, while projecting an annotated attribute \optAtt\ from a 
%v-relation \vRel\ not only the attribute must belong to the v-relation, i.e., 
%$\vRel \annot [\dimMeta_\vRel] {\paran {\optAtt [\dimMeta_1], \vAttList}}$, but 
%the feature expression $\dimMeta \wedge \dimMeta_\vRel \wedge \dimMeta_1$ 
%must also be satisfiable, i.e., 
%%a database variant in the intersection of variability
%%encoded in the v-schema and the query exists s.t. attribute \vAtt\ is present in 
%%relation \vRel.
%the attribute \vAtt\ must be present in the relation \vRel\
%under the condition imposed by the query and the v-schema. 
%
%
%A v-query that is not well-typed is \emph{ill-typed}.

\begin{figure}
%\begin{minipage}[t]{0.5\textwidth}
\textbf{Variational queries typing rules:}

  \begin{mathpar}
  \small
  
  \inferrule[\empRelE]
  {}
  {\env {\empRel} {\annot [\f] {\setDef \ }}}
%  \inferrule[\judge]
 % 	{\env{\vQ}{\vType}}
 %   {}
%
% explicitly-typed vra:
%    \inferrule[\relationE]
%  	{\vRel (\vType)^{\VVal \dimMeta} \in \vSch \\
%	\neg \sat{\vctx \wedge \neg \VVal \dimMeta} }
%     {\envWithSchema{\envInContext [\vctx ] {\vType}}}

%implicitly-typed lang:
    \inferrule[\relationE]
  	{ \vRel (\vType)^{\VVal \dimMeta} \in \vSch \\
	\sat {\vctx \wedge \getPCfrom \vRelSch \vSch}}
%	\sat{\vctx \wedge \VVal \dimMeta} }
     {\envWithSchema{\envInContext [\vctx \wedge \VVal \dimMeta] {\vType}}}

% explicitly-typed vra:  
%  \inferrule[\prjE]
%  	{\envPrime \\
%    	\subsume {\annot \vType}  {\annot [\VVal \vctx] {\VVal \vType}}}
%    {\env{\vPrj[\vType] \vQ} {\envInContext [\vctx] \vType}}

%implicitly-typed lang:
  \inferrule[\prjE]
  	{\envPrime \\
	|\pushIn {\annot \vType}| = | \vType | \\
%	\pushIn {\annot \vType} \neq \setDef \ \\
%	\pushIn {\annot [\VVal \dimMeta] {\VVal \vType}} \neq \setDef \ \\
    	\subsume { \vType}  {\pushIn {\annot [\VVal \vctx] {\VVal \vType}}}}
    {\env{\vPrj[\vType] \vQ} {\envInContext [\VVal \vctx] {\left(\vType \cap {\VVal \vType} \right)}}}
%    the older version 8/8/20:
%    	\subsume {\pushIn {\annot \vType}}  {\pushIn {\annot [\VVal \vctx] {\VVal \vType}}}}
%    {\env{\vPrj[\vType] \vQ} {\envInContext [\VVal \vctx] {\left(\pushIn{\annot {\vType}} \cap {\VVal \vType} \right)}}}


  \inferrule[\selE]
  	{\env \vQ {\envInContext [\VVal \vctx] \vType} \\
    	\envCondAnnot \vCond}
    {\env{\vSel \vQ}{\envInContext [\VVal \vctx] \vType}}
    
  \inferrule[\choiceE]
  	{\envOne[\vctx \wedge \VVal \dimMeta] \\
    	\envTwo[\vctx \wedge \neg \VVal \dimMeta]}
    {\env{\chc[\VVal \dimMeta]{\vQ_1, \vQ_2}}{
     \envInContext [(\vctx_1) \vee (\vctx_2)] 
     {\left({\pushIn {\envInContext [\vctx_1] \vType_1}} \cup
    							{\pushIn {\envInContext [\vctx_2] \vType_2}}\right)}}}
% older version 8/8/20:
%    {\env{\chc[\VVal \dimMeta]{\vQ_1, \vQ_2}}{
%     \envInContext [(\vctx_1 \wedge \VVal \dimMeta) \vee (\vctx_2 \wedge \neg \VVal \dimMeta)] 
%     {\left({\pushIn {\envInContext [\vctx_1 \wedge \VVal \dimMeta] \vType_1}} \cup
%    							{\pushIn {\envInContext [\vctx_2 \wedge \neg \VVal \dimMeta] \vType_2}}\right)}}}
    
  \inferrule[\productE]
  	{\envOne \\
    	\envTwo\\
	\pushIn {\annot [\vctx_1] \vType_1} \cap \pushIn {\annot [\vctx_2] \vType_2} = \{\}}
    {\env{\vQ_1 \times \vQ_2}{\envInContext [\vctx_1 \wedge \vctx_2] 
      {\left(\pushIn {\annot [\vctx_1] \vType_1} \cup \pushIn {\annot [\vctx_2] \vType_2} \right)}}}


  \inferrule[\setopE]
  	{\envOne \\
    	\envTwo \\
	\envEval {\pushIn {\annot [\vctx_1] \vType_1}} {\pushIn {\annot [\vctx_2] \vType_2}}}
%        \envEval{\envInContext{\vType_1}} \vType \\
%        \envEval{\envInContext{\vType_2}} \vType}
    {\env{\vQ_1 \circ \vQ_2} {\envInContext [\vctx_1] \vType_1} }

%  \inferrule[\diffE]
%  	{\envOne \\
%    	\envTwo \\
%        \envEval{\envInContext{\vType_1}} \vType \\
%        \envEval{\envInContext{\vType_2}} \vType}
%    {\env{\vQ_1 \setminus \vQ_2} \vType}
  \end{mathpar}
  
\medskip
\textbf{Variational condition typing rules:}
% (b: boolean tag, \pAtt: plain attribute, k: constant value):}
%(b: boolean tag, A: plain attribute, k: constant value)}
  \begin{mathpar}
  \small    

  \inferrule[\boolC]
  	{}
    {\envCond \bTag}
    
  \inferrule[\conjC]
  	{\envCond \vCond_1\\
    	\envCond \vCond_2}
    {\envCond{\vCond_1 \wedge \vCond_2}}
    
  \inferrule[\disjC]
  	{\envCond \vCond_1\\
    	\envCond \vCond_2}
    {\envCond{\vCond_1 \vee \vCond_2}}
   

  \inferrule[\choiceC]
%  	{\defType{\relInContext{\vContext''}}\in \vSch \\
    	{\envCond[\vctx \wedge \VVal \dimMeta, \vType]{\vCond_1} \\
        \envCond[\vctx \wedge \neg \VVal \dimMeta, \vType]{\vCond_2}}
    {\envCond{\chc[\VVal \dimMeta]{\vCond_1, \vCond_2}}}
    

  \inferrule[\notC]
  	{\envCond \vCond}
    {\envCond \neg \vCond}
        
%  \inferrule[]
%  	{\envCond[\vContext \wedge \dimMeta]{\vCond_1} \\
%    	\envCond[\vContext \wedge \neg\dimMeta]{\vCond_2}}
%    {\envCond{\chc{\vCond_1, \vCond_2}}}
    

    
  \inferrule[\attValC]
  	{
	%\defType{\relInContext{\vContext'}}\in \vSch \\
    	\optAtt [\VVal \dimMeta] \in \vType \\
%	\taut{{\VVal \dimMeta} \imply \vctx} \\
        \sat {\VVal \dimMeta \wedge \vctx}}
%        \\
%        \cte \in \dom \vAtt}
    {\envCond{\op \pAtt \cte}}

    
  \inferrule[\attAttC]
  	{
	%\defType{\relInContext{\vContext'}}\in \vSch \\
    	\optAtt [\dimMeta_1] [\vAtt_1]\in \vType \\
         {\optAtt [\dimMeta_2] [\vAtt_2]} \in \vType \\
%         \taut{\dimMeta_1 \imply \vctx} \\
%         \taut{\dimMeta_2 \imply \vctx} \\
        \sat { \dimMeta_1 \wedge \dimMeta_2 \wedge \vctx}}
%        \\
%        \type[\vAtt_1] = \type[ \vAtt_2]}
    {\envCond{\op{\pAtt_1}{\pAtt_2}}}
    
  \end{mathpar}

%\caption{V-condition typing relation. A v-condition \vCond\ is well-typed if 
%it is valid in the variational context \vctx\ and type environment \vType, i.e., 
%\envCond \vCond. Note that the type rules for v-conditions return a boolean, if
%the v-condition is type-correct the rules return \t, otherwise they return \f.}
\caption[Typing rules of variational relational algebra and variational condition]{VRA and variational condition typing relation. 
The rules assume that the underlying VDB is well-formed. 
Remember that our theory assumes all attributes have the same type
and all constants belong to attributes' domain. 
%The typing rule of a join query is the combination
%of rules \selE\ and \productE.
}
\label{fig:vq-stat-sem}
%\end{minipage}
\end{figure}


%\begin{figure}

\textbf{V-queries typing rules:}

  \begin{mathpar}
  \small
  
%  \inferrule[\judge]
 % 	{\env{\vQ}{\vType}}
 %   {}
    \inferrule[\relationE]
  	{\vRel (\vType)^{\VVal \dimMeta} \in \vSch \\
	\neg \sat{\vctx \wedge \neg \VVal \dimMeta} }
     {\envWithSchema{\envInContext [\vctx ] {\vType}}}

%implicitly-typed lang:
%    \inferrule[\relationE]
%  	{\vRel (\vType)^{\VVal \dimMeta} \in \vSch \\
%	\sat{\vctx \wedge \VVal \dimMeta} }
%     {\envWithSchema{\envInContext [\vctx \wedge \VVal \dimMeta] {\vType}}}
  
  \inferrule[\prjE]
  	{\envPrime \\
    	\subsumeExpl {\annot \vType}  {\annot [\VVal \vctx] {\VVal \vType}}}
    {\env{\vPrj[\vType] \vQ} {\envInContext [\vctx] \vType}}

%implicitly-typed lang:
%  \inferrule[\prjE]
%  	{\envPrime \\
%    	\subsume {\annot \vType}  {\annot [\VVal \vctx] {\VVal \vType}}}
%    {\env{\vPrj[\vType] \vQ} {\envInContext [\VVal \vctx] {\left(\annot {\vType} \cap {\VVal \vType} \right)}}}


  \inferrule[\selE]
  	{\env \vQ {\envInContext [\VVal \vctx] \vType} \\
    	\envCondAnnot \vCond}
    {\env{\vSel \vQ}{\envInContext [\VVal \vctx] \vType}}
    
  \inferrule[\choiceE]
  	{\envOne[\vctx \wedge \VVal \dimMeta] \\
    	\envTwo[\vctx \wedge \neg \VVal \dimMeta]}
    {\env{\chc[\VVal \dimMeta]{\vQ_1, \vQ_2}}{
     \envInContext [\vctx_1 \vee \vctx_2] {\left({\envInContext [\vctx_1] \vType_1} \cup
    							{\envInContext [\vctx_2] \vType_2}\right)}}}
    
  \inferrule[\productE]
  	{\envOne \\
    	\envTwo\\
	\annot [\vctx_1] \vType_1 \cap \annot [\vctx_2] \vType_2 = \{\}}
    {\env{\vQ_1 \times \vQ_2}{\envInContext [\vctx_1 \wedge \vctx_2] {\left(\vType_1 \cup \vType_2\right)}}}


  \inferrule[\setopE]
  	{\envOne \\
    	\envTwo \\
	\envEval {\annot [\vctx_1] \vType_1} {\annot [\vctx_2] \vType_2}}
%        \envEval{\envInContext{\vType_1}} \vType \\
%        \envEval{\envInContext{\vType_2}} \vType}
    {\env{\vQ_1 \circ \vQ_2} {\envInContext [\vctx_1] \vType_1} }

%  \inferrule[\diffE]
%  	{\envOne \\
%    	\envTwo \\
%        \envEval{\envInContext{\vType_1}} \vType \\
%        \envEval{\envInContext{\vType_2}} \vType}
%    {\env{\vQ_1 \setminus \vQ_2} \vType}
  \end{mathpar}
  
\medskip
\textbf{V-condition typing rules (b: boolean tag, \pAtt: plain attribute, k: constant value):}
%(b: boolean tag, A: plain attribute, k: constant value)}
  \begin{mathpar}
  \small    

  \inferrule[\conjC]
  	{\envCond \vCond_1\\
    	\envCond \vCond_2}
    {\envCond{\vCond_1 \wedge \vCond_2}}
    
  \inferrule[\disjC]
  	{\envCond \vCond_1\\
    	\envCond \vCond_2}
    {\envCond{\vCond_1 \vee \vCond_2}}
    


  \inferrule[\choiceC]
%  	{\defType{\relInContext{\vContext''}}\in \vSch \\
    	{\envCond[\vctx \wedge \VVal \dimMeta, \vType]{\vCond_1} \\
        \envCond[\vctx \wedge \neg \VVal \dimMeta, \vType]{\vCond_2}}
    {\envCond{\chc[\VVal \dimMeta]{\vCond_1, \vCond_2}}}
    

  \inferrule[\notC]
  	{\envCond \vCond}
    {\envCond \neg \vCond}
        
%  \inferrule[]
%  	{\envCond[\vContext \wedge \dimMeta]{\vCond_1} \\
%    	\envCond[\vContext \wedge \neg\dimMeta]{\vCond_2}}
%    {\envCond{\chc{\vCond_1, \vCond_2}}}
    

    
  \inferrule[\attValC]
  	{
	%\defType{\relInContext{\vContext'}}\in \vSch \\
    	\optAtt [\VVal \dimMeta] \in \vType \\
	\taut{{\VVal \dimMeta} \imply \vctx} \\
        \cte \in \dom \vAtt}
    {\envCond{\op \pAtt \cte}}
    
  \inferrule[\boolC]
  	{}
    {\envCond \bTag}
    

    
  \inferrule[\attAttC]
  	{
	%\defType{\relInContext{\vContext'}}\in \vSch \\
    	\optAtt [\dimMeta_1] [\vAtt_1]\in \vType \\
         {\optAtt [\dimMeta_2] [\vAtt_2]} \in \vType \\
         \taut{\dimMeta_1 \imply \vctx} \\
         \taut{\dimMeta_2 \imply \vctx} \\
        \type[\vAtt_1] = \type[ \vAtt_2]}
    {\envCond{\op{\pAtt_1}{\pAtt_2}}}
    
  \end{mathpar}

%\caption{V-condition typing relation. A v-condition \vCond\ is well-typed if 
%it is valid in the variational context \vctx\ and type environment \vType, i.e., 
%\envCond \vCond. Note that the type rules for v-conditions return a boolean, if
%the v-condition is type-correct the rules return \t, otherwise they return \f.}
\caption{Explicitly-annotated VRA and v-condition typing relation. The explicitly-annotated
query is generated by pushing the schema to the implicitly-annotated query. Hence,
all annotations are explicit and included in the query, i.e., they are at least as specific as
corresponding presence conditions encoded in the v-schema. We use 
\ensuremath{\annot \vRel} as syntactic sugar for \chc {\vRel, \empRel}.
}
\label{fig:vq-stat-sem}
\end{figure}



\figref{vq-stat-sem} defines VRA's \emph {typing relation}
as a set of inference rules assigning \emph{types} to queries. 
%
The type of a query is a v-relation schema \ensuremath{\mathit{result}\annot {(\vAttList)}},
however, for brevity and since the relation name is the same for all 
queries we consider the type of a query an annotated v-set of attributes
where attributes are projected by the query from the VDB
and their presence conditions determine their valid variants.
The presence condition of attributes in the type of a query may differ 
from their presence conditions in v-schema due to variation constraints
imposed by the query.
For example, continuing with relation
\ensuremath{
\vRel \left( \optAtt [\fOne] [\vAtt_1], \vAtt_2, \vAtt_3 \right)^{\fOne \vee \fTwo}
},
the query \ensuremath{\vPrj [{\optAtt [\fOne] [\vAtt_2]}] \vRel} has 
the type \ensuremath{\{\annot [\fOne] {\vAtt_2} \}^{\fOne \vee \fTwo}} while 
according to \vRel's schema 
\ensuremath{\getPC {\vAtt_2} = \fOne \vee \fTwo}, i.e., the presence
condition of attribute \ensuremath{{\vAtt_2}} changes through the query.
The presence condition of the entire set determines the condition under
which the entire table (i.e., attributes and tuples) are valid. 
Note that it is essential to consider the type of a query an \emph{annotated}
v-set to account for the presence condition of the entire table.
%similar to how we encode v-relation schemas. 
%If we
%consider the type of a query a variational attribute set we lose information (i.e.,
%the condition under which tuples are valid).
%The final variation context, after running a v-query, is the
%presence condition of the returned v-table. That is why we 
%consider the type environment as a variational set of attributes instead of 
%a relation schema. 

%\begin{example}
%\label{eg:vq-affect-vctx}
%Assume we have the VDB defined in \exref{vsch}. 
%Consider the query $\pi_{\name} \empbio$. The returned 
%table has the type: $(\optAtt [\vFour] [\name])^\fModel$. 
%However, if we change the query to: 
%$\pi_{\optAtt [\edu] [\name]} \empbio$, the returned table has the
%type: $(\optAtt [\vFour \wedge \edu] [\name])^\fModel$. 
%\end{example}

%
%\eric{Eric, feel free to summarize rule explanations. I basically
%explained them how I'd read them.}
VRA's typing relation, as defined in \figref{vq-stat-sem}, 
has the judgement form \env \vQ {\envInContext [\VVal \vctx] \vType}. 
This states
that in \emph \vctxTxt\ \vctx\ within v-schema \vSch, 
v-query \vQ\ has type \envInContext [\VVal \vctx] \vType. 
If a query does not have a type, it is \emph{ill-typed}.
\emph{Variation context} is a feature expression that the type system 
keeps and refines to keep track of variation encoded by a query.
%To capture the variation encoded in a query,
%we keep and refine a feature expression, called a \emph{variation context}.
The variation context is initiated by the feature model. For brevity,
we use the judgment form \envWithoutVctx \vQ {\envInContext [\VVal \vctx] \vType}\
for  \env [\getPC \vSch] \vQ {\envInContext [\VVal \vctx] \vType}
i.e., the variation context is initialized. Note that attributes with an
unsatisfiable presence condition are not present in any 
database variant, i.e., they are not present for any configuration.
Thus, the existence of such attribute in a type does not change
the type semantically, based on the defined equivalence rule for 
v-sets, given in \defref{vset-eq}. Hence, we do not filter out such attributes
explicitly in \figref{vq-stat-sem}, however, for simplicity, 
the implemented type system
drops the attributes with an unsatisfiable presence condition.
 

%
The rule \relationE\ states that, in \vctxTxt\ \vctx\ with
underlying \vschTxt\ \vSch, assuming that
1) \vSch\ contains
the relation \vRel\ with presence condition $\VVal \dimMeta$
and v-set of attributes \vType\ 
and
2) there exists a valid variant in the intersection of variation context \vctx\
and \vRel's presence condition \ensuremath{\VVal \dimMeta}, i.e., 
\ensuremath {\sat {\vctx \wedge \VVal \dimMeta}},
%\vctxTxt\ \vctx\ of query \vRel\
%is more specific than the \presCondTxt\ of the relation $\VVal \dimMeta$, denoted by
%$\taut {\vctx \to \VVal \dimMeta}$, 
then query \vRel\ has type \vType\ annotated with \ensuremath { \vctx \wedge \VVal \dimMeta}.
%denoted by \envInContext \vType, formally defined in \defref{vCtxtAppliedType}.

% 
The rule \prjE\ states that, in \vctxTxt\ \vctx\ within v-schema \vSch, assuming 
that the subquery \vQ\ has type $\envInContext [\VVal \vctx] {\VVal \vType}$,
v-query $\pi_\vType \vQ$
has type \ensuremath {\envInContext [\VVal \vctx] {\left( \vType \cap \VVal \vType\right)}}, 
if  all attributes in \vType\ are present in \vctx\
%\pushIn {\annot \vType} is not an empty v-set
and
 \ensuremath {\pushIn {\envInContext [\VVal \vctx] {\VVal \vType}}} subsumes \vType.
%the variational 
%attribute set \vType\ constrained (annotated) with variational context \vctx, i.e., \annot \vType.
The subsumption, defined in \defref{vset-subsumption},
ensures that the subquery \vQ\ does not have an empty type
and it includes all attributes in 
the projected attribute set and attributes' presence conditions do not 
contradict each other. Returning the intersection of types, defined in 
\defref{vset-intersect}, filters both 
attributes and their presence conditions.
\exref{type} illustrates generating the type of a query step by step.


\begin{example}
\label{eg:type}
We illustrate how a query enforces variation encoded within it to the result.
We do this by illustrating how the type system generates two different 
types for queries
\ensuremath{\vQ_1} and \ensuremath{\vQ_2}
given in \exref{vq-specific}.
%\footnote{Derivation trees of these examples can be fine here! 
%%\TODO{derive the trees using the new rules.}
%\TODO{Eric, do we need them? If yes, where should we put them?}}. 
For brevity, we simplify feature expressions when possible.
%
For \ensuremath{\vQ_1}, it applies the \prjE\ rule under
the variation context initiated to 
\ensuremath{\fModel_2 = \vThree \oplus \vFour \oplus \vFive}
and schema \ensuremath{\vSch_2}.
It now has to apply the
\relationE\ rule to the subquery \empbio\ under the same variation context and schema,
resulting in the type
\ensuremath{
\vAttList_\empbio =  \{\empno, \sex, \birthdate,}
\ensuremath{ 
\optAtt [\vFour] [\name], \optAtt [\vFive] [\fname], \optAtt [\vFive] [\lname]\}^{\fModel_2}}.
Now that it has the type of the subquery \empbio\ 
it verifies that the projected attribute v-set
% annotated with the variation context, i.e.,
\ensuremath{
\vAttList_{\mathit{prj}} =
 \{\optAtt [\vFour \vee \vFive] [\empno],
\name,}
\ensuremath{ \fname, \lname\}^{\fModel_2}},
%^{\vThree \oplus \vFour \oplus \vFive}},
is subsumed by \ensuremath{\vAttList_\empbio}. 
Thus, it generates the type of query \ensuremath{\vQ_1} by
intersecting \ensuremath{\vAttList_{\mathit{prj}}} and \ensuremath{\vAttList_\empbio}
annotated with \ensuremath{\vAttList_\empbio}'s presence condition resulting in the type
\ensuremath{
\vAttList_{\vQ_1} = 
\{\optAtt [\vFour \vee \vFive] [\empno],
\optAtt [\vFour] [\name], }
\ensuremath{
\optAtt [\vFive] [\fname], \optAtt [\vFive] [\lname]\}^{\fModel_2}}.
%
This type demonstrates the structure of the result of query \ensuremath{\vQ_1}.
%
As for \ensuremath{\vQ_2}, the type system applies the \choiceE\ rule
under the variation 
context initiated to \ensuremath{\fModel_2} and schema \ensuremath{\vSch_2}.
It then applies the \prjE\ and \empRelE\ rules to the left and right
alternatives of the choice, respectively, which generates the types
\ensuremath{
\vType_\mathit{left} = \annot [\fModel_2 \wedge (\vFour \vee \vFive)] {(\empno, \annot [\vFour] \name,
 \annot [\vFive] \fname,\annot [\vFive] \lname)}}
and \ensuremath{\vType_\mathit{right} = \annot [\f] {\setDef \ }}, respectively.
Finally, it generates the type of \ensuremath{\vQ_2} by 
annotating the union of \ensuremath{\vType_\mathit{left}} and \ensuremath{\vType_\mathit{right}}
with \ensuremath{\fModel_2 \wedge (\vFour \vee \vFive)}, resulting in the 
final type of \\
\ensuremath{\vType_{\vQ_2} = 
\annot [\fModel_2 \wedge (\vFour \vee \vFive)] {(\empno, \annot [\vFour] \name,
 \annot [\vFive] \fname,\annot [\vFive] \lname)}}.
 Note that \ensuremath{\vType_{\vQ_2}}'s presence condition 
 explicitly accounts for only two variants
 while \ensuremath{\vType_1} does not do so even though \ensuremath{\vQ_1}
 does not return any tuple that belong to variant \ensuremath{\setDef \vThree} because
 of its attributes presence condition. 
\end{example}


%
The rule \selE\ states that, in \vctxTxt\ \vctx\ within v-schema \vSch, assuming 
that the subquery \vQ\ has type {\envInContext [\VVal \vctx] \vType}, 
the v-query $\sigma_{\vCond} \vQ$
has type {\envInContext [\VVal \vctx] \vType},
if the \vCondTxt\ \vCond\ is well-formed w.r.t.
 \vctxTxt\ \vctx\ and \tenvTxt\ {\envInContext [\VVal \vctx] \vType}, 
denoted by v-condition's typing relation 
\envCondAnnot \vCond.
Note that in variational condition typing rules, 
the presence condition of the query type is pushed in.
% applied to the 
%variational attribute set, thus, they have the form \envCond \vCond\ 
%instead of \envCondAnnot \vCond. 
The rules state that attributes used in a
\vCondTxt\ must be valid in \vType\ and 
attribute's \presCondTxt\ \ensuremath {\VVal \dimMeta} 
in type \vType\ must exists within \vctxTxt\ \vctx,
denoted by \ensuremath{\sat {\VVal \dimMeta \wedge \vctx}}.
%be more specific than the \vctxTxt\ \vctx,
%denoted by \ensuremath{\taut {\VVal \dimMeta \to \vctx}}, 
%since \vType\ is the exact type and specification of the subquery within
%a selection query which is at least as specific as the \vctxTxt\ under which
%the selection query is written. 
%the \fexpTxt\ attached to an attribute in a 
%\vCondTxt\ must be more specific than its \presCondTxt\ in type \vType. 
They also
check the constraints of traditional relational databases, such as the type of two 
compared attributes must be the same.

%
The rule \choiceE\ states that, in \vctxTxt\ \vctx\ within v-schema \vSch, the type of 
a choice of two subqueries is the \emph{union of types}, defined in 
\defref{vset-union}, of its subqueries annotated with the disjunction of their presence
conditions conjuncted with the corresponding condition of the choice's dimension.
%annotated 
%with their corresponding \vctxTxt s, which depends on the feature expression of the choice
%query $\VVal \dimMeta$. 
A choice query is well-typed iff both of 
its subqueries $\vQ_1$ and $\vQ_2$ are well-typed.
%Note that we do not simplify 
%\ensuremath{
%\annot [\vctx_1 \vee \vctx_2] {\left(\envInContext [\vctx_1] \vType_1 \cup \envInContext [\vctx_2] \vType_2\right)}
%}
%to 
%\ensuremath{
%\envInContext [\vctx_1] \vType_1 \cup \envInContext [\vctx_2] \vType_2
%}
%because we need to know the presence condition of the entire type, i.e., \ensuremath{\vctx_1 \vee \vctx_2},
%to know the condition under which tuples are valid\footnote{The simplification holds because
%\ensuremath{
%\annot [\vctx_1 \vee \vctx_2] {\left(\envInContext [\vctx_1] \vType_1 \cup \envInContext [\vctx_2] \vType_2\right)}
%\equiv
%\envInContext [\vctx_1\wedge (\vctx_1 \vee \vctx_2)] \vType_1 \cup \envInContext [\vctx_2\wedge (\vctx_1 \vee \vctx_2)] \vType_2
%\equiv
%\envInContext [\vctx_1] \vType_1 \cup \envInContext [\vctx_2] \vType_2
%}
%}.
Note that \choiceE\ is the only rule that refines the variation context. 


% 
The rule \productE\ states that the type of a product query in \vctxTxt\
\vctx\ is the union of the type of its subqueries annotated with the 
conjunction of their presence conditions, assuming that 
they are disjoint. 
%Note that 
%\ensuremath{
%\annot [\vctx_1 \wedge \vctx_2] {\left(\envInContext [\vctx_1] \vType_1 \cup \envInContext [\vctx_2] \vType_2\right)}
%\equiv 
%\envInContext [\vctx_1 \wedge \vctx_2] \vType_1 \cup \envInContext [\vctx_1 \wedge \vctx_2] \vType_2
%\equiv
%\annot [\vctx_1 \wedge \vctx_2] {\left(\vType_1 \cup \vType_2\right)}
%}.

% 
The rule \setopE\ denotes the typing rule for set operation queries such as 
union and difference. It states that, if the subqueries $\vQ_1$ and $\vQ_2$
have types $\envInContext [\vctx_1] \vType_1$ and 
$\envInContext [\vctx_2] \vType_2$, respectively, in \vctxTxt\ \vctx,
then the v-query of their set operation has type $\envInContext [\vctx_1] \vType_1$, iff 
$\pushIn {\envInContext [\vctx_1] \vType_1}$ and $\pushIn {\envInContext [\vctx_2] \vType_2}$ are \emph{equivalent}.
%their annotated type with
%\vctx\ are \emph{equivalent} to \vType. 
The \emph{type equivalence} is v-set equivalence, defined in \defref{vset-eq},
for v-sets of attributes.


%9-19-18 notes:
%* type soundness theorem: 
%    * ideally we want to prove this, but at least we should formulate it
%    * the way you do this usually, is that you have a semantics and you want to show 
%        * progress: if your program is well-typed your either done evaluating it or you can keep evaluate it -> in our case is kind of given, because we don't have a turing complete language!!
%        * preservation: when we evaluate sth it doesn't change its type! -> what we want to show is the relation that semantics gives us back fits the schema that the type system gives us. so there is consistency between type system and semantics. 
%    * so in the context of variational queries what that would mean is that when we evaluate a query, the relation that we get back actually has the type that we said it has via the type system
%    * PROBLEM: our semantics is via SQL so proving this will be kind of hairy but we should at least write this down and try to convince ourselves that it's correct, and we'll figure out what we can say in the context of the paper.
%    * [TODO] writing test cases that it?s correct!! in terms of mini database at two levels, haskell and database. the test cases will be queries against this mini database with lots of variability in it. you can write a quick check property. so you basically are testing the commuting diagram, so you either:
%        1. configure the query first and then give its type
%        2. give its variational type first and then configure it
%        3. and you should get the same thing from both
 


%The purpose of establishing \emph{type safety} is to ensure that the
%static and dynamic semantics are consistent with each other. 
%@Eric do we need to say anything about why we don't take the standard 
%approach?
%\NEED{Since 
%we define \vqTxt\ dynamic semantics in terms of relational algebra,
%i.e. we translate a \vqTxt\ into a set of relational algebra expression 
%and then combine the result of them into a \vrelTxt, we do not take 
%the standard approach of defining and proving \emph{progress} and 
%\emph{preservation} properties.
%}

%We 
%follow the approach developed by \TODO{fill in later!!}, which distinguishes
%two type safety properties, preservation and progress. The preservation
%theorem establishes that \vqsTxt\ preserve type assignments, i.e. that the
%type of a \vqTxt\ accurately predicates the type of the result of evaluating 
%that \vqTxt. \TODO{state the context and type you start out with}


%pierce def:
%progress: a well-typed term is not stuck (either it is a value or it can take a step 
%according to the evaluation rules)
%\begin{theorem}
%\label{thm:progress}
%Suppose \vQ\ is a closed, well-typed \vqTxt\ (that is, \env {\vQ} {\vType} for some 
%\vType). Then either \vQ\ is a value (a \vrelTxt ) or else there is some 
%\TODO{in order to define this we need to define:\\
%small step of relational algebra\\
%translation rules to rel alg\\
%combining the set of queries resulted from translation to output a var table\\
%canonical forms\\
%
%\end{theorem}


%pierce def:
%preservation: if a well-typed term takes a step of evaluation, then the resulting 
%term is also well typed
%\begin{theorem}
%\label{thm:type-pres}
%
%\end{theorem}

%\TODO{double check the following ph and write the properties based on it:}
%Conceptually, a variational query describes a query that can
%be executed over any database instance consistent with the 
%variational schema. The property that must hold between a 
%variational query \vQ\ and a variational schema \vSch
%is that for every plain query $\pQ_\config$ obtained from \vQ 
%by configuring with a function $\config: \fSet \to \bSet$, $\pQ_\config$ 
%is consistent with the corresponding plain schema $\pSch_\config$ 
%obtained by $\osSem  \vSch$ with the same function \config. That 
%is, every variant query matches the corresponding variant schema.
%%\REMEMBER{In \secref{prop-q-lang}, we prove that our query language,
%%variational relational algebra, can encode this idea and also
%%recover any of the conceptually potential results for any instance
%%of the variational database.}



%\subsubsection{Variation Minimization}
\label{sec:var-min}

\TODO {add more rules}
%
%\TODO{Eric, I kept this here and I just point out this property of VRA in 
%\secref{vrel-alg} in a note box (could you please review that too?). 
%How do you feel about moving this subsection to appendix?}
VRA is flexible since an information need can be represented via multiple
v-queries as demonstrated in \exref{vq-specific} and \exref{vq-same-intent-mult-vars}.
It allows users to incorporate their personal taste and task requirements
into v-queries they write by 
having different levels of variation. For example, consider the explicitly annotated query
\ensuremath{\vQ_5} 
in \secref{constrain}:\\
\ensuremath {
\vQ_5 =
\pi_{\optAtt [\vFour \vee \vFive] [\empno], \optAtt [\vFour] [\name], \optAtt [\vFive] [\fname], \optAtt [\vFive] [\lname]  } \chc [\fModel_2] {\empbio, \empRel}}
%\vQ_5 =  \pi_{\optAtt [\vFour \vee \vFive] [\empno], \optAtt [\vFour] [\name], \optAtt [\vFive] [\fname], \optAtt [\vFive] [\lname]  } \empbio}.
%from \exref{vq-specific}. 
To be explicit about the exact query that will be run for 
each variant 
%and knowing that 
%\ensuremath{
%\getPC \empbio = \vThree \vee \vFour \vee \vFive
%},
the user can \emph{lift up} the variation and rewrite the query as\\
\ensuremath{
\small
\VVal \vQ_5 = \chc [\vFour] {\pi_{\empno, \name} \empbio, 
\chc [\vFive] {\pi_{\empno, \fname, \lname} \empbio, \emp}} 
}.
While \ensuremath{\vQ_5} contains less redundancy \ensuremath{\VVal \vQ_5}
is more comprehensible. 
Thus, \emph{supporting multiple levels of variation 
creates a tension between reducing redundancy and maintaining comprehensibility.}

We define \emph{variation minimization} rules, \figref{var-min}.
% and include 
%interesting ones in \secref{var-min}.
Pushing in variation into a query, i.e., applying rules left-to-right, 
reduces redundancy
% and improves performance
while lifting them up, i.e., applying rules right-to-left, 
makes a query more understandable. 
When applied left-to-right, the rules are terminating since the scope of variation 
%always gets smaller.
monotonically decreases in size.


\begin{figure}
\textbf{Choice Distributive Rules:}
\begin{alignat*}{1}
\small
%-- f<? l? q?, ? l? q?> ? ? (f<l?, l?>) f<q?, q?>
%\inferrule
%{}
%\chc {\pi_{\vAttList_1} \vQ_1, \pi_{\vAttList_2} \vQ_2 } 
%&\equiv
%\pi_{\chc {\vAttList_1, \vAttList_2}} \chc {\vQ_1, \vQ_2}\\
%-- f<? l? q?, ? l? q?> ? ? ((l??), (l? \^�f )) f<q?, q?>
%\inferrule
%{}
\chc {\pi_{\vAttList_1} \vQ_1, \pi_{\vAttList_2} \vQ_2}
&\equiv
\pi_{\annot \vAttList_1, \annot [\neg \dimMeta] \vAttList_2} \chc {\vQ_1, \vQ_2}\\
%-- f<? c? q?, ? c? q?> ? ? f<c?, c?> f<q?, q?>
%\inferrule
%{}
\chc {\sigma_{\vCond_1} \vQ_1, \sigma_{\vCond_2} \vQ_2} 
&\equiv
\sigma_{\chc {\vCond_1, \vCond_2}} \chc {\vQ_1, \vQ_2}\\
%-- f<q? � q?, q? � q?> ? f<q?, q?> � f<q?, q?>
%\inferrule
%{}
\chc {\vQ_1 \times \vQ_2, \vQ_3 \times \vQ_4}
&\equiv
\chc {\vQ_1, \vQ_3} \times \chc {\vQ_2, \vQ_4}\\
%-- f<q? ?\_c? q?, q? ?\_c? q?> ? f<q?, q?> ?\_(f<c?, c?>) f<q?, q?>
%\inferrule
%{}
\chc {\vQ_1 \Join_{\vCond_1} \vQ_2, \vQ_3 \Join_{\vCond_2} \vQ_4}
&\equiv
\chc {\vQ_1, \vQ_3} \Join_{\chc {\vCond_1, \vCond_2}} \chc {\vQ_2, \vQ_4}\\
%-- f<q? ? q?, q? ? q?> ? f<q?, q?> ? f<q?, q?>
%\inferrule
%{}
\chc {\vQ_1 \circ \vQ_2, \vQ_3 \circ \vQ_4}
&\equiv
\chc {\vQ_1, \vQ_3} \circ \chc {\vQ_2, \vQ_4}
%-- f<q? ? q?, q? ? q?> ? f<q?, q?> ? f<q?, q?>
%\inferrule
%{}
%{-}
\end{alignat*}

\medskip
\textbf{CC and RA Optimization Rules Combined:}
\begin{alignat*}{1}
\small
%-- f<? (c? ? c?) q?, ? (c? ? c?) q?> ? ? (c? ? f<c?, c?>) f<q?, q?>
%\inferrule
%{}
\chc {\sigma_{\vCond_1 \wedge \vCond_2} \vQ_1, \sigma_{\vCond_1 \wedge \vCond_3} \vQ_2}
&\equiv
\sigma_{\vCond_1 \wedge \chc {\vCond_2, \vCond_3}} \chc {\vQ_1, \vQ_2}\\
%-- ? c? (f<? c? q?, ? c? q?>) ? ? (c? ? f<c?, c?>) f<q?, q?>
%\inferrule
%{}
\sigma_{\vCond_1} \chc {\sigma_{\vCond_2} \vQ_1, \sigma_{\vCond_3} \vQ_2}
&\equiv
\sigma_{\vCond_1 \wedge \chc {\vCond_2, \vCond_3}} \chc {\vQ_1, \vQ_2}\\
%-- f<q? ?\_(c? ? c?) q?, q? ?\_(c? ? c?) q?> ? ? (f<c?, c?>) (f<q?, q?> ?\_c? f<q?, q?>)
%\inferrule
%{}
\chc {\vQ_1 \Join_{\vCond_1 \wedge \vCond_2} \vQ_2, \vQ_3 \Join_{\vCond_1 \wedge \vCond_3} \vQ_4}
&\equiv
\sigma_{\chc {\vCond_2, \vCond_3}} \left( \chc {\vQ_1, \vQ_3} \Join_{\vCond_1} \chc {\vQ_2, \vQ_4} \right)
\end{alignat*}

\caption{Some of variation minimization rules.}
\label{fig:var-min}
\end{figure}


\section{Variational Relational Algebra}
\label{sec:vrel-alg}

%
%\wrrite{you want to have an example here that doesn't need to be explicitly annotated. 
%then use the same example to illustrate the semantics via ra sem.}
%\TODO{we use ... that has these differences with blah
%%
%we introduce these differences through building up a query to extract info
%required by the variational intent. }

%\point{vra = cc + ra}
%Considering the variational nature of a VDB, to satisfy a user's information 
%need when extracting information, 
%we need a query language that not only considers the structure of 
%relational databases (such as SQL and relational algebra (RA)) but also 
%accounts for the variation encoded in the VDB. We achieve this by:
%1) picking relational algebra as our main query language and
%2) using \emph{choices}~\cite{Walk13thesis, EW11tosem} 
%and presence conditions to account for variation. 

To account for variation, VRA combines relational algebra (RA) with 
\emph{choices}~\cite{EW11tosem,HW16fosd,Walk13thesis}.
%\point{choice.}
Remember that a choice $\chc{\elem_1,\elem_2}$ consists of a feature expression \dimMeta, called
the \emph{dimension} of the choice, and 
two \emph{alternatives} $\elem_1$ and $\elem_2$. For a given configuration \config, 
the choice $\chc{\elem_1, \elem_2}$ can be replaced by $\elem_1$ if \dimMeta\
evaluates to \t\ under configuration \config, (i.e., \fSem{\dimMeta}),
or $\elem_2$ otherwise. 
% Choices allow a variational queries
% to encode variation in a structured and systematic manner. 

\begin{figure}
\begin{syntax}

\multicolumn{4}{l}{\textbf{Operators:}} \\[1ex]
\bullet
  &\eqq& \multicolumn{2}{l}{< \myOR \leq \myOR = \myOR \neq \myOR > \myOR \geq} \\
\circ
  &\eqq& \cup \myOR \cap \\[2ex]

\multicolumn{4}{l}{\textbf{Variational conditions:}} \\[1ex]
\vCond\in\vCondSet
  &\eqq&  \multicolumn{2}{l}{
          \bTag
   \myOR  \pAtt \bullet \cte
   \myOR  \pAtt \bullet \pAtt
   \myOR  \neg \vCond
   \myOR  \vCond \vee \vCond} \\
  &\myOR& \multicolumn{2}{l}{
          \vCond \wedge \vCond
   \myOR \chc{\vCond,\vCond}} \\[2ex]

\multicolumn{4}{l}{\textbf{Variational queries:}} \\[1ex]
\vQ\in\qSet
  &\eqq&  \vRel     & \textit{Relation}\\
  &\myOR& \vSel \vQ & \textit{Selection}\\
  &\myOR& \vPrj[\vAttList]{\vQ} & \textit{Projection}\\
  &\myOR& \chc{\vQ,\vQ} & \textit{Choice}\\
% &\myOR& \vQ \Join_\vCond \vQ & \textit{Variational Join}\\
  &\myOR& \vQ \times \vQ & \textit{Cartesian Product}\\
  &\myOR& \vQ \circ \vQ  & \textit{Set Operation}\\
% &\myOR& \vQ \backslash \vQ &\textit{Variational Set Difference}\\
  &\myOR& \empRel & \textit{Empty Relation}

\end{syntax}

\caption[Syntax of variational relational algebra]{Syntax of variational relational algebra.}
%\TODO{remember that
%you removed join (also removed it from query config def and constrain query 
%by schema). if you want use it just say it's a syntactic sugar.}
%$<, \leq, =, \neq, >, \geq$.
%$\circ$ denotes set operators: union and difference.}
\label{fig:v-alg-def}
\end{figure}



%\point{explain notation and VRA operations.}
The syntax of VRA is given in \figref{v-alg-def}.
%
The selection operation is similar to standard RA selection except
that the condition parameter is \emph{variational} meaning that it may contain
choices.
For example, the query 
\ensuremath{\sigma_{\chc {\vAtt_1=\vAtt_2,\vAtt_1=\vAtt_3}} (\vRel)}
selects a variational tuple \vTuple\ if it satisfies
the condition \ensuremath{\vAtt_1 = \vAtt_2} 
and  \ensuremath{\sat {\dimMeta \wedge \getPC \vTuple}}
or
if \ensuremath{\vAtt_1 = \vAtt_3} 
and \ensuremath{\sat {\neg \dimMeta \wedge \getPC \vTuple }}.
%
The projection operation is parameterized by a variational set of attributes, \vAttList. For
example,
the query $\pi_{\vAtt_1, \optAtt [\dimMeta] [\vAtt_2]} (\vRel)$
projects $\vAtt_1$ from relation \vRel\ unconditionally, and $\vAtt_2$ 
when \sat{\dimMeta}.
%
The choice operation enables combining two variational queries to be used in different
variants based on the dimension. In practice,
it is often useful to return information in some variants and nothing at all in
others. We introduce an explicit \emph{empty} query \empRel\ to facilitate
this. 
Similar to our definition of the empty query for relational algebra, for VRA we
also have: $\empRel=\vPrj[\set{}]{\vQ}$.
The empty query is used, for example, in 
\ensuremath{\vQ_2} in \exref{vq-specific}. 
%The set operations between queries are v-set operations defined in \secref{vset}.
The rest of VRA's operations are similar to RA, where all set operations
(union, intersection, and product) are changed to the corresponding
variational set operations defined in \secref{vset}.
%\secref{vlist-vset}.
%
%\remember{
%In examples, we also use a join operation with a variational condition,
%$\vQ_1\bowtie_\vCond\vQ_2$, which is syntactic sugar for
%$\sigma_\vCond(\vQ_1\times\vQ_2)$.}


Our implementation of VRA also provides mechanisms for renaming queries and
qualifying attributes with relation/sub\-query names. These features are needed
to support self joins and to project attributes with the same name in different
relations. However, for simplicity, we omit these features from the formal
definition in this thesis.


%A query can simply 
%refer to a relation, filter tuples based on a variational condition 
%(which is a relational condition with choices of two conditions), and
%project a variational list of attributes. Besides production of two queries and
%set operations, VRA allows for a choice of two variational queries. This demands an
%\emph{empty} query since an alternative of a choice can very well inquire 
%no information at all. 
%For example, the query $\chc {\vQ_1, \}$
%

%\subsubsection{Running a Variational query on a VDB Results in a Variational table}
%\label{sec:run-vq-get-vtab}
%A variational query systematically represents a set of relational query variants associated to their
%corresponding database variants. Hence, intuitively the user expects to 
%get such variation in their result as well. 

The result of a variational query is a variational table with the reserved relation name $\mathit{result}$.
%
For example, assume that variational tuples $\annot[\fOne]{(1,2)}$ and $\annot[\neg
f_3]{(3,4)}$ belong to a variational relation $\vRel(\vAtt_1,\vAtt_2)$, which is the only
relation in a VDB with the trivial feature model \t.
%
The query $\chc[f_3]{\pi_{\optAtt[f_2][\vAtt_1]}(\vRel),\empRel}$ returns a
variational table with relation schema $\annot[f_3]{\mathit{result}(\annot[f_2]{a_1})}$,
which indicates that the result is only non-empty when $f_3$ is \t\ and that the
result includes attribute $a_1$ when $f_2$ is \t. 
%\secref{type-sys} defines a
%type system that yields the relation schema for any well-formed query.
%
The content of the result relation for the example query is a single variational tuple
$\annot[f_1]{(1)}$. The tuple $\annot[\neg f_3]{(3)}$ is not included since the
projection occurs in the context of a choice in $f_3$, which is incompatible
with the presence condition of the tuple, i.e., $\unsat{f_3 \wedge\neg f_3}$.
This illustrates how choices can effectively filter the tuples in a VDB based
on the dimension.
%, satisfying the second part of \nOne.
%
% Although there is no need to update the presence condition of the returned
% tuples, yet choices can filter the returned variational tuples.
%
% Note that here the value \ensuremath{1}
% of attribute \ensuremath{\vAtt_1} is present in VDB variants where 
% \ensuremath{\sat {\A \wedge \B \wedge \C}} although the presence 
% condition of the returned variational tuple does not have to state this condition
% since 
%
% overall presence 
%condition and the presence conditions of attributes and tuples are
%restricted by the variation enforced by the query.
%
%Note that the presence condition of tuples, attributes, and the return relation
%is restricted by the variation enforced by the query. 
% correct this so that you don't conjunct the pc of relation and clarify that it's relations's pc and not the attributes. although the conjunction should be satisfiable.}
%
%
%%Hence, VRA is more expressive than RA 
%%because it can encode variational queries.
%The variational nature allows users to write interesting queries in many ways:
%1) to express their variational information need or to filter returned tuples
%they can use annotations or 
%choices, \exref{vq-specific},
%2) to express the same intent over several database variants they can 
%use choices in queries or conditions, \exref{vq-same-intent-mult-vars},
%and 
%3) they can also use choices to express different intents over database variants.
%\TODO{Eric, should we drop the last since it creates messy results and isn't really useful?}.
%%The expressiveness of VRA satisfies \textbf{N1}, this is illustrated in 
%%\exref{vq-specific} and \exref{vq-same-intent-mult-vars}.
%%Interestingly, VRA's expressiveness enables users to express 
%%their information need more specifically by stating the exact condition
%%under which an information need is inquired. \exref{vq-specific} illustrates this.
%%It also allows users to express the same intent over several database 
%%variants
% \NOTE{
% To express the variational information need or to filter returned tuples
% users can use annotations or choices. \exref{vq-specific} illustrates this.
% }
%
%The following example
\exref{vq-specific} illustrates
%, in the context of our running example, 
how
a variational query can be used to express variational information needs.

\begin{example}
\label{eg:vq-specific}
%VRA's expressiveness consequently facilitates expressing exactly the condition
%under which an information need is inquired. 
%\wrrite{build up this example. and show the result tables.}
Assume a VDB with
\ensuremath{\features = \setDef {\vThree, \vFour, \vFive}}, 
and the only variational table \empbio\ shown in \tabref{empbio-vtab}.
The VDB has the feature model $\dimMeta_2 = \oneof {\vThree, \vFour, \vFive}$
which states that the three \vThree--\vFive\ are mutually exclusive. 
Note that $\dimMeta_2$ is different from the feature model 
$\dimMeta_{\mathit{mot}}$ of the \empbio\ variational table
shown in \tabref{empbio-vtab} .
%the corresponding \empbio\ schema variants in \tabref{mot}. 
The variational schema for this VDB is:\\
%
\centerline{\ensuremath{
\vSch_2 =
\{\empbio (\empno, \sex, \birthdate,
\optAtt [\vFour] [\name], \optAtt [\vFive] [\fname],
 \optAtt [\vFive] [\lname] )\}^{\dimMeta_2}
% \\
%& \hspace{-38pt} \textit{where } \dimMeta_2 = \oneof {\vThree, \vFour, \vFive}
%{\vThree \oplus \vFour \oplus \vFive}.
%\left(\vThree \wedge \neg \vFour \wedge \neg \vFive\right)
%  \vee \left(\vFour \wedge \neg \vThree \wedge \neg \vFive\right) 
%   \vee \left(\vFive \wedge \neg \vThree \wedge \neg \vFour\right)}.
%\end{align*}
}}.
%
Now, the user wants the employee ID numbers (\empno) and their names for variants 
that enable either \vFour\ or \vFive\ but not \vThree.
%\set{\vFour} and \set{\vFive}.
We show the steps to build up multiple queries that can extract this information. 
First, to extract the required attributes we write the query $\vQ_0$ to project all the needed
attributes without considering the variational aspect of projection. \\
\centerline{\ensuremath{
\vQ_0 = \pi_{\empno, \name, \fname, \lname} (\empbio)
}}
Note that the presence condition attribute (\pcatt) does not need to be projected. In fact, 
the presence condition attribute is returned for every variational query since that is the only
way to keep track of variation at the content level. 
%
\tabref{vq0-res} shows
the result of query $\vQ_0$ over the described VDB.
%
Note that the presence condition of the result is $\getPCfrom \empbio {\vSch_2} = \oneof {\vThree, \vFour, \vFive} \wedge (\vThree \vee \vFour \vee \vFive)$ which can be simplified to
$\oneof {\vThree, \vFour, \vFive}$. We discuss how the 
presence conditions of the returned result and its attributes are generated in \secref{type-sys}.
%

\begin{table}[ht!]
%\caption[Results of some variational queries]{Results of some variational queries over the VDB instance described in \exref{vq-specific}.}
%\label{tab:vq-res}
%\centering
%\begin{subtable}[t]{\textwidth}
\centering
\caption[Result of a variational query]{Result of the v-query $\vQ_0 = \pi_{\empno, \name, \fname, \lname} (\empbio)$.}
\label{tab:vq0-res}
\footnotesize
\arrayrulecolor{blue}
%!{\color{black}\vrule}
\begin{tabular} {c !{\color{black}\vrule} l l l l : l }
 {\textcolor{blue}{$\oneof {\vThree, \vFour,\vFive}$} }& {\textcolor{blue}{\texttt{true}}}&  {\textcolor{blue}{$\vFour$}} &  {\textcolor{blue}{$\vFive$}} &  {\textcolor{blue}{$\vFive $}} & {\textcolor{blue}{\texttt{true}}}\\
\arrayrulecolor{blue}\hdashline
\multirow{2}{*}{$\mathit{result}$}  & \empno & \name & \fname & \lname & \pcatt \\
\arrayrulecolor{black}\cline{2-6}
& 12001 & Ulf Hofstetter & Ulf & Hofstetter  & $\textcolor{blue}{\vThree \vee \vFour \vee \vFive}$\\
& 12002 & Luise McFarlan & Luise & McFarlan  & $\textcolor{blue}{\vThree \vee \vFour \vee \vFive}$\\
& 12003 & Shir DuCasse & Shir & DuCasse  & $\textcolor{blue}{\vThree \vee \vFour \vee \vFive}$\\
 &80001  & Nagui Merli & Nagui & Merli & $\textcolor{blue}{ \vFour \vee \vFive}$\\
 & 80002 & Mayuko Meszaros & Mayuko & Meszaros & $\textcolor{blue}{ \vFour \vee \vFive}$\\
 & 80003 & Theirry Viele & Theirry & Viele & $\textcolor{blue}{ \vFour \vee \vFive}$\\
 & 200001  & Selwyn Koshiba & Selwyn & Koshiba & \textcolor{blue}{\vFive}\\
 & 200002  & Bedrich Markovitch & Bedrich & Markovitch & \textcolor{blue}{\vFive}\\
 & 200003  & Pascal Benzmuller & Pascal & Benzmuller  & \textcolor{blue}{\vFive}\\
 & \ldots  & \ldots & \ldots & \ldots & \textcolor{blue}{\ldots} \\
\arrayrulecolor{white}\hline
\end{tabular}
%\end{subtable}
%
%\medskip
%\medskip
%\medskip
%\begin{subtable}[t]{\textwidth}
%\centering
%\caption{Result of the variational queries $\vQ_1 = \pi_{\optAtt [\vFour \vee \vFive] [\empno], \name, \fname, \lname} (\empbio)$ and 
%$\VVal {\vQ_1} = \pi_{\optAtt [(\vFour \vee \vFive) \wedge \neg \vThree] [\empno], 
%\optAtt [\vFour \wedge \neg \vThree \wedge \neg \vFive] [\name], 
%\optAtt [\vFive \wedge \neg \vThree \wedge \neg \vFour] [\fname], 
%\optAtt [\vFive \wedge \neg \vThree \wedge \neg \vFour] [\lname]} (\empbio)
%$.}
%\label{tab:vq1-res}
%\footnotesize
%\arrayrulecolor{blue}
%%!{\color{black}\vrule}
%\begin{tabular} {c !{\color{black}\vrule} l l l l : l }
% {\textcolor{blue}{$\oneof {\vThree, \vFour,\vFive}$} }& {\textcolor{blue}{$\vFour \vee \vFive$}}&  {\textcolor{blue}{$\vFour $}} &  {\textcolor{blue}{$\vFive $}} &  {\textcolor{blue}{$\vFive$}} & {\textcolor{blue}{\texttt{true}}}\\
%\arrayrulecolor{blue}\hdashline
%\multirow{2}{*}{$\mathit{result}$}  & \empno & \name & \fname & \lname & \pcatt \\
%\arrayrulecolor{black}\cline{2-6}
%%& 12001 & & & & \textcolor{blue}{\vThree}\\
%%& 12002 & & & & \textcolor{blue}{\vThree}\\
%%& 12003 & & & & \textcolor{blue}{\vThree}\\
% &80001  & Nagui Merli & & & \textcolor{blue}{\vFour}\\
% & 80002 & Mayuko Meszaros & & & \textcolor{blue}{\vFour}\\
% & 80003 & Theirry Viele & & & \textcolor{blue}{\vFour}\\
% & 200001  & & Selwyn & Koshiba & \textcolor{blue}{\vFive}\\
% & 200002  & & Bedrich & Markovitch & \textcolor{blue}{\vFive}\\
% & 200003  & & Pascal & Benzmuller  & \textcolor{blue}{\vFive}\\
% & \ldots  & \ldots & \ldots & \ldots& \textcolor{blue}{\ldots} \\
%\arrayrulecolor{white}\hline
%\end{tabular}
%\end{subtable}
%
%\medskip
%\medskip
%\medskip
%\begin{subtable}[t]{\textwidth}
%\centering
%\caption{Result of the variational query $\vQ_2 = \chc[\neg \vThree]{\pi_{\empno,\name,\fname,\lname}(\empbio),\empRel}$.}
%\label{tab:vq2-res}
%\footnotesize
%\arrayrulecolor{blue}
%%!{\color{black}\vrule}
%\begin{tabular} {c !{\color{black}\vrule} l l l l : l }
% {\textcolor{blue}{$\oneof {\vThree, \vFour,\vFive} \wedge \neg \vThree$} }& {\textcolor{blue}{\t}}&  {\textcolor{blue}{$\vFour $}} &  {\textcolor{blue}{$\vFive $}} &  {\textcolor{blue}{$\vFive$}} & {\textcolor{blue}{\texttt{true}}}\\
%\arrayrulecolor{blue}\hdashline
%\multirow{2}{*}{$\mathit{result}$}  & \empno & \name & \fname & \lname & \pcatt \\
%\arrayrulecolor{black}\cline{2-6}
%%& 12001 & & & & \textcolor{blue}{\vThree}\\
%%& 12002 & & & & \textcolor{blue}{\vThree}\\
%%& 12003 & & & & \textcolor{blue}{\vThree}\\
% &80001  & Nagui Merli & & & \textcolor{blue}{\vFour}\\
% & 80002 & Mayuko Meszaros & & & \textcolor{blue}{\vFour}\\
% & 80003 & Theirry Viele & & & \textcolor{blue}{\vFour}\\
% & 200001  & & Selwyn & Koshiba & \textcolor{blue}{\vFive}\\
% & 200002  & & Bedrich & Markovitch & \textcolor{blue}{\vFive}\\
% & 200003  & & Pascal & Benzmuller  & \textcolor{blue}{\vFive}\\
% & \ldots  & \ldots & \ldots & \ldots& \textcolor{blue}{\ldots} \\
%\arrayrulecolor{white}\hline
%\end{tabular}
%\end{subtable}
%
\end{table}


Now we pay attention to the variational aspect of the query. Knowing that the variation encoded
in the VDB can be inferred (that is, the VDB exists if and only if exactly
 one of the features \vThree--\vFive\ is enabled, the \name\ attribute only exists for variants
that enable \vFour\ and the \fname\ and \lname\ attributes only exist for variants that
enable \vFive) and since we only want the
projected attributes for variants that enable \vFour\ or \vFive\ we can write the
query $\vQ_1$.\\
%
\centerline{\ensuremath{
\vQ_1 = \pi_{\optAtt [\vFour \vee \vFive] [\empno], \name, \fname, \lname} (\empbio)
}}
%
\tabref{vq1-res} shows the result of this query over the described VDB.
Note that the first three tuples from \tabref{vq0-res} are not returned since the query
does not project the \empno\
attribute for variants that enable \vThree\ and  attributes 
\name, \fname, and \lname\ do not exist for these variants in the VDB. 
Thus, the tuple will just be empty and so is dropped. 
%The user needs to project the \name\ attribute 
%for variant \set{\vFour}, the \fname\ and \lname\ attributes for variant
%\set{\vFive}, and \empno\ attribute for both variants.
%This can be expressed with the following variational query.
%If we did not know that the database enforces the variation encoded in itself
%we had to repeat that variation. 

\begin{table}[ht!]
%\caption[Results of some variational queries]{Results of some variational queries over the VDB instance described in \exref{vq-specific}.}
%\label{tab:vq-res}
%\centering
%\begin{subtable}[t]{\textwidth}
%\centering
%\caption{Result of the variational query $\vQ_0 = \pi_{\empno, \name, \fname, \lname} (\empbio)$.}
%\label{tab:vq0-res}
%\footnotesize
%\arrayrulecolor{blue}
%%!{\color{black}\vrule}
%\begin{tabular} {c !{\color{black}\vrule} l l l l : l }
% {\textcolor{blue}{$\oneof {\vThree, \vFour,\vFive}$} }& {\textcolor{blue}{\texttt{true}}}&  {\textcolor{blue}{$\vFour$}} &  {\textcolor{blue}{$\vFive$}} &  {\textcolor{blue}{$\vFive $}} & {\textcolor{blue}{\texttt{true}}}\\
%\arrayrulecolor{blue}\hdashline
%\multirow{2}{*}{$\mathit{result}$}  & \empno & \name & \fname & \lname & \pcatt \\
%\arrayrulecolor{black}\cline{2-6}
%& 12001 & & & & \textcolor{blue}{\vThree}\\
%& 12002 & & & & \textcolor{blue}{\vThree}\\
%& 12003 & & & & \textcolor{blue}{\vThree}\\
% &80001  & Nagui Merli & & & \textcolor{blue}{\vFour}\\
% & 80002 & Mayuko Meszaros & & & \textcolor{blue}{\vFour}\\
% & 80003 & Theirry Viele & & & \textcolor{blue}{\vFour}\\
% & 200001  & & Selwyn & Koshiba & \textcolor{blue}{\vFive}\\
% & 200002  & & Bedrich & Markovitch & \textcolor{blue}{\vFive}\\
% & 200003  & & Pascal & Benzmuller  & \textcolor{blue}{\vFive}\\
% & \ldots  & \ldots & \ldots & \ldots & \textcolor{blue}{\ldots} \\
%\arrayrulecolor{white}\hline
%\end{tabular}
%\end{subtable}
%
%\medskip
%\medskip
%\medskip
%\begin{subtable}[t]{\textwidth}
\centering
\caption[Result of a variational query]{Result of the v-queries $\vQ_1 = \pi_{\optAtt [\vFour \vee \vFive] [\empno], \name, \fname, \lname} (\empbio)$ and 
$\VVal {\vQ_1} = \pi_{\optAtt [(\vFour \vee \vFive) \wedge \neg \vThree] [\empno], 
\optAtt [\vFour \wedge \neg \vThree \wedge \neg \vFive] [\name], 
\optAtt [\vFive \wedge \neg \vThree \wedge \neg \vFour] [\fname], 
\optAtt [\vFive \wedge \neg \vThree \wedge \neg \vFour] [\lname]} (\empbio)
$.}
\label{tab:vq1-res}
\footnotesize
\arrayrulecolor{blue}
%!{\color{black}\vrule}
\begin{tabular} {c !{\color{black}\vrule} l l l l : l }
 {\textcolor{blue}{$\oneof {\vThree, \vFour,\vFive}$} }& {\textcolor{blue}{$\vFour \vee \vFive$}}&  {\textcolor{blue}{$\vFour $}} &  {\textcolor{blue}{$\vFive $}} &  {\textcolor{blue}{$\vFive$}} & {\textcolor{blue}{\texttt{true}}}\\
\arrayrulecolor{blue}\hdashline
\multirow{2}{*}{$\mathit{result}$}  & \empno & \name & \fname & \lname & \pcatt \\
\arrayrulecolor{black}\cline{2-6}
%%& 12001 & & & & \textcolor{blue}{\vThree}\\
%%& 12002 & & & & \textcolor{blue}{\vThree}\\
%%& 12003 & & & & \textcolor{blue}{\vThree}\\
& 12001 & Ulf Hofstetter & Ulf & Hofstetter  & $\textcolor{blue}{\vThree \vee \vFour \vee \vFive}$\\
& 12002 & Luise McFarlan & Luise & McFarlan  & $\textcolor{blue}{\vThree \vee \vFour \vee \vFive}$\\
& 12003 & Shir DuCasse & Shir & DuCasse  & $\textcolor{blue}{\vThree \vee \vFour \vee \vFive}$\\
 &80001  & Nagui Merli & Nagui & Merli & $\textcolor{blue}{\vFour \vee \vFive}$\\
 & 80002 & Mayuko Meszaros & Mayuko & Meszaros & $\textcolor{blue}{ \vFour \vee \vFive}$\\
 & 80003 & Theirry Viele & Theirry & Viele & $\textcolor{blue}{ \vFour \vee \vFive}$\\
 & 200001  & Selwyn Koshiba & Selwyn & Koshiba & \textcolor{blue}{\vFive}\\
 & 200002  & Bedrich Markovitch & Bedrich & Markovitch & \textcolor{blue}{\vFive}\\
 & 200003  & Pascal Benzmuller & Pascal & Benzmuller  & \textcolor{blue}{\vFive}\\
 & \ldots  & \ldots & \ldots & \ldots & \textcolor{blue}{\ldots} \\
% &80001  & Nagui Merli & & & \textcolor{blue}{\vFour}\\
% & 80002 & Mayuko Meszaros & & & \textcolor{blue}{\vFour}\\
% & 80003 & Theirry Viele & & & \textcolor{blue}{\vFour}\\
% & 200001  & & Selwyn & Koshiba & \textcolor{blue}{\vFive}\\
% & 200002  & & Bedrich & Markovitch & \textcolor{blue}{\vFive}\\
% & 200003  & & Pascal & Benzmuller  & \textcolor{blue}{\vFive}\\
% & \ldots  & \ldots & \ldots & \ldots& \textcolor{blue}{\ldots} \\
\arrayrulecolor{white}\hline
\end{tabular}
%\end{subtable}
%
%\medskip
%\medskip
%\medskip
%\begin{subtable}[t]{\textwidth}
%\centering
%\caption{Result of the variational query $\vQ_2 = \chc[\neg \vThree]{\pi_{\empno,\name,\fname,\lname}(\empbio),\empRel}$.}
%\label{tab:vq2-res}
%\footnotesize
%\arrayrulecolor{blue}
%%!{\color{black}\vrule}
%\begin{tabular} {c !{\color{black}\vrule} l l l l : l }
% {\textcolor{blue}{$\oneof {\vThree, \vFour,\vFive} \wedge \neg \vThree$} }& {\textcolor{blue}{\t}}&  {\textcolor{blue}{$\vFour $}} &  {\textcolor{blue}{$\vFive $}} &  {\textcolor{blue}{$\vFive$}} & {\textcolor{blue}{\texttt{true}}}\\
%\arrayrulecolor{blue}\hdashline
%\multirow{2}{*}{$\mathit{result}$}  & \empno & \name & \fname & \lname & \pcatt \\
%\arrayrulecolor{black}\cline{2-6}
%%& 12001 & & & & \textcolor{blue}{\vThree}\\
%%& 12002 & & & & \textcolor{blue}{\vThree}\\
%%& 12003 & & & & \textcolor{blue}{\vThree}\\
% &80001  & Nagui Merli & & & \textcolor{blue}{\vFour}\\
% & 80002 & Mayuko Meszaros & & & \textcolor{blue}{\vFour}\\
% & 80003 & Theirry Viele & & & \textcolor{blue}{\vFour}\\
% & 200001  & & Selwyn & Koshiba & \textcolor{blue}{\vFive}\\
% & 200002  & & Bedrich & Markovitch & \textcolor{blue}{\vFive}\\
% & 200003  & & Pascal & Benzmuller  & \textcolor{blue}{\vFive}\\
% & \ldots  & \ldots & \ldots & \ldots& \textcolor{blue}{\ldots} \\
%\arrayrulecolor{white}\hline
%\end{tabular}
%\end{subtable}
%
\end{table}


If desired, we can also make the inferred presence conditions explicit, as 
demonstrated in the following query $\VVal {\vQ_1}$.\\
%This is expressed in $\VVal {\vQ_1}$. 
\centerline{\ensuremath{
\VVal {\vQ_1} = 
\pi_{\optAtt [(\vFour \vee \vFive) \wedge \neg \vThree] [\empno], 
\optAtt [\vFour \wedge \neg \vThree \wedge \neg \vFive] [\name], 
\optAtt [\vFive \wedge \neg \vThree \wedge \neg \vFour] [\fname], 
\optAtt [\vFive \wedge \neg \vThree \wedge \neg \vFour] [\lname]} (\empbio)
}}
%
The result of the query $\VVal {\vQ_1}$ is still \tabref{vq1-res}.
%\eric{
%%isnt this a moot point? since the feature model has already been applied, the result is unchanged by whether the feature model restricts it or not? 
%Yeah, I guess. but that was the whole point of saying that fm has been applied.}
Note that all the variation encoded in the VDB is applied to the result of
a query. Thus,
the result of a variational query stands on its own, that is,
it is not part of a bigger structure like the variational tables in a VDB.
%Note that unlike the variational table \empbio\ shown in \tabref{empbio-vtab},
%%which is restricted not only by its presence condition but also by the feature model,
%the result of a variational query is not part of a bigger variational structure (the VDB).
%Thus, it is only restricted by its presence condition and not the feature model although
%the feature model has already been applied to its presence condition.
\end{example}


In the example, note that the user does not need to repeat the variability  encoded
in the variational schema in their query, that is, they do not need to annotate \name,
\fname, and \lname\ with \vFour, \vFive, and \vFive, respectively. We discuss
this in more detail in \secref{constrain}. $\vQ_1$
queries all three variants simultaneously although the returned results are
only associated with variants \vFour\ and \vFive\ due to the annotation of the
attribute \empno\ in the query and the presence conditions of the rest of the
projected attributes in the schema.
%
Yet, the query can be further simplified with a choice. $\vQ_2$ selects only two
out of the three variants explicitly:\\
%selecting only two out of the three variants can be written more
%explicitly in a query by using a choice:
\centerline{\ensuremath{
\vQ_2=\chc[\neg \vThree]{\pi_{\empno,\name,\fname,\lname}(\empbio),\empRel}}}. 
%
\tabref{vq2-res} shows the result of this query over the VDB described in \exref{vq-specific}.
%

\begin{table}[ht!]
%\caption[Results of some variational queries]{Results of some variational queries over the VDB instance described in \exref{vq-specific}.}
%\label{tab:vq-res}
%\centering
%\begin{subtable}[t]{\textwidth}
%\centering
%\caption{Result of the variational query $\vQ_0 = \pi_{\empno, \name, \fname, \lname} (\empbio)$.}
%\label{tab:vq0-res}
%\footnotesize
%\arrayrulecolor{blue}
%%!{\color{black}\vrule}
%\begin{tabular} {c !{\color{black}\vrule} l l l l : l }
% {\textcolor{blue}{$\oneof {\vThree, \vFour,\vFive}$} }& {\textcolor{blue}{\texttt{true}}}&  {\textcolor{blue}{$\vFour$}} &  {\textcolor{blue}{$\vFive$}} &  {\textcolor{blue}{$\vFive $}} & {\textcolor{blue}{\texttt{true}}}\\
%\arrayrulecolor{blue}\hdashline
%\multirow{2}{*}{$\mathit{result}$}  & \empno & \name & \fname & \lname & \pcatt \\
%\arrayrulecolor{black}\cline{2-6}
%& 12001 & & & & \textcolor{blue}{\vThree}\\
%& 12002 & & & & \textcolor{blue}{\vThree}\\
%& 12003 & & & & \textcolor{blue}{\vThree}\\
% &80001  & Nagui Merli & & & \textcolor{blue}{\vFour}\\
% & 80002 & Mayuko Meszaros & & & \textcolor{blue}{\vFour}\\
% & 80003 & Theirry Viele & & & \textcolor{blue}{\vFour}\\
% & 200001  & & Selwyn & Koshiba & \textcolor{blue}{\vFive}\\
% & 200002  & & Bedrich & Markovitch & \textcolor{blue}{\vFive}\\
% & 200003  & & Pascal & Benzmuller  & \textcolor{blue}{\vFive}\\
% & \ldots  & \ldots & \ldots & \ldots & \textcolor{blue}{\ldots} \\
%\arrayrulecolor{white}\hline
%\end{tabular}
%\end{subtable}
%
%\medskip
%\medskip
%\medskip
%\begin{subtable}[t]{\textwidth}
%\centering
%\caption{Result of the variational queries $\vQ_1 = \pi_{\optAtt [\vFour \vee \vFive] [\empno], \name, \fname, \lname} (\empbio)$ and 
%$\VVal {\vQ_1} = \pi_{\optAtt [(\vFour \vee \vFive) \wedge \neg \vThree] [\empno], 
%\optAtt [\vFour \wedge \neg \vThree \wedge \neg \vFive] [\name], 
%\optAtt [\vFive \wedge \neg \vThree \wedge \neg \vFour] [\fname], 
%\optAtt [\vFive \wedge \neg \vThree \wedge \neg \vFour] [\lname]} (\empbio)
%$.}
%\label{tab:vq1-res}
%\footnotesize
%\arrayrulecolor{blue}
%%!{\color{black}\vrule}
%\begin{tabular} {c !{\color{black}\vrule} l l l l : l }
% {\textcolor{blue}{$\oneof {\vThree, \vFour,\vFive}$} }& {\textcolor{blue}{$\vFour \vee \vFive$}}&  {\textcolor{blue}{$\vFour $}} &  {\textcolor{blue}{$\vFive $}} &  {\textcolor{blue}{$\vFive$}} & {\textcolor{blue}{\texttt{true}}}\\
%\arrayrulecolor{blue}\hdashline
%\multirow{2}{*}{$\mathit{result}$}  & \empno & \name & \fname & \lname & \pcatt \\
%\arrayrulecolor{black}\cline{2-6}
%%& 12001 & & & & \textcolor{blue}{\vThree}\\
%%& 12002 & & & & \textcolor{blue}{\vThree}\\
%%& 12003 & & & & \textcolor{blue}{\vThree}\\
% &80001  & Nagui Merli & & & \textcolor{blue}{\vFour}\\
% & 80002 & Mayuko Meszaros & & & \textcolor{blue}{\vFour}\\
% & 80003 & Theirry Viele & & & \textcolor{blue}{\vFour}\\
% & 200001  & & Selwyn & Koshiba & \textcolor{blue}{\vFive}\\
% & 200002  & & Bedrich & Markovitch & \textcolor{blue}{\vFive}\\
% & 200003  & & Pascal & Benzmuller  & \textcolor{blue}{\vFive}\\
% & \ldots  & \ldots & \ldots & \ldots& \textcolor{blue}{\ldots} \\
%\arrayrulecolor{white}\hline
%\end{tabular}
%\end{subtable}
%
%\medskip
%\medskip
%\medskip
%\begin{subtable}[t]{\textwidth}
\centering
\caption[Result of a variational query]{Result of the v-query $\vQ_2 = \chc[\neg \vThree]{\pi_{\empno,\name,\fname,\lname}(\empbio),\empRel}$.}
\label{tab:vq2-res}
\footnotesize
\arrayrulecolor{blue}
%!{\color{black}\vrule}
\begin{tabular} {c !{\color{black}\vrule} l l l l : l }
 {\textcolor{blue}{$\oneof {\vThree, \vFour,\vFive} \wedge \neg \vThree$} }& {\textcolor{blue}{\t}}&  {\textcolor{blue}{$\vFour $}} &  {\textcolor{blue}{$\vFive $}} &  {\textcolor{blue}{$\vFive$}} & {\textcolor{blue}{\texttt{true}}}\\
\arrayrulecolor{blue}\hdashline
\multirow{2}{*}{$\mathit{result}$}  & \empno & \name & \fname & \lname & \pcatt \\
\arrayrulecolor{black}\cline{2-6}
%%& 12001 & & & & \textcolor{blue}{\vThree}\\
%%& 12002 & & & & \textcolor{blue}{\vThree}\\
%%& 12003 & & & & \textcolor{blue}{\vThree}\\
& 12001 & Ulf Hofstetter & Ulf & Hofstetter  & $\textcolor{blue}{ \vFour \vee \vFive}$\\
& 12002 & Luise McFarlan & Luise & McFarlan  & $\textcolor{blue}{ \vFour \vee \vFive}$\\
& 12003 & Shir DuCasse & Shir & DuCasse  & $\textcolor{blue}{ \vFour \vee \vFive}$\\
 &80001  & Nagui Merli & Nagui & Merli & $\textcolor{blue}{ \vFour \vee \vFive}$\\
 & 80002 & Mayuko Meszaros & Mayuko & Meszaros & $\textcolor{blue}{ \vFour \vee \vFive}$\\
 & 80003 & Theirry Viele & Theirry & Viele & $\textcolor{blue}{ \vFour \vee \vFive}$\\
 & 200001  & Selwyn Koshiba & Selwyn & Koshiba & \textcolor{blue}{\vFive}\\
 & 200002  & Bedrich Markovitch & Bedrich & Markovitch & \textcolor{blue}{\vFive}\\
 & 200003  & Pascal Benzmuller & Pascal & Benzmuller  & \textcolor{blue}{\vFive}\\
 & \ldots  & \ldots & \ldots & \ldots & \textcolor{blue}{\ldots} \\
% &80001  & Nagui Merli & & & \textcolor{blue}{\vFour}\\
% & 80002 & Mayuko Meszaros & & & \textcolor{blue}{\vFour}\\
% & 80003 & Theirry Viele & & & \textcolor{blue}{\vFour}\\
% & 200001  & & Selwyn & Koshiba & \textcolor{blue}{\vFive}\\
% & 200002  & & Bedrich & Markovitch & \textcolor{blue}{\vFive}\\
% & 200003  & & Pascal & Benzmuller  & \textcolor{blue}{\vFive}\\
% & \ldots  & \ldots & \ldots & \ldots& \textcolor{blue}{\ldots} \\
\arrayrulecolor{white}\hline
\end{tabular}
%\end{subtable}
%
\end{table}



Note that, as shown in \tabref{vq1-res} and \tabref{vq2-res}, 
queries $\vQ_1$ and $\vQ_2$ return the same set of variational tuples.
However, the first three tuples in \tabref{vq1-res} could belong to a variant that 
enables any of \vThree--\vFive\ whereas the first three tuples in \tabref{vq2-res}
could only belong to variants that either enable \vFour\ or \vFive. 
This difference is due to the difference in their tables' presence conditions, 
that is, $\vQ_2$ filters out tuples that belong to variant \vThree\ at the schema 
level while $\vQ_1$ does not. We discuss this more in \exref{type}. 
More importantly, even though the first three tuples in \tabref{vq1-res} could 
belong to a variant that enables \vThree, configuring \tabref{vq1-res}
for such a variant drops the first three tuples since all their attributes would 
be \nul. We illustrate how configuring \tabref{vq1-res} for variant \setDef \vThree\
drops the first three tuples in \exref{conf-vq}.
% since
%neither returns tuples associated with variant \vThree, but their returned
%variational tables have different presence conditions, thus, $\vQ_2$ filters out
%tuples that belong to variant \vThree\ at the schema level while $\vQ_1$ does not. We discuss this
%more in \exref{type}. 
%

%\NOTE{
%\revised{VRA has \revised{syntactic} equivalence rules, described in
%\secref{var-min}, that enable semantics-preserving transformations of queries
%similar to the transformation of $\vQ_1$ into $\vQ_2$ (and vice versa). These
%rules enable factoring commonality out of subqueries, among other
%transformations.}

%The next example 
 Expressing
the same intent over several database variants by a single query relieves the DBA from
maintaining separate queries for different variants or configurations of the
schema.
\exref{vq-same-intent-mult-vars} 
illustrates this point.
% by using choices.
%how a variational query can be used to express the same
%intent over several database variants using choices and conditions.

\begin{example}
\label{eg:vq-same-intent-mult-vars}
Assume a VDB with  \ensuremath{\features = \setDef{\vOne, \ldots, \vFive}}
and the corresponding \basic\ schema
variants in \tabref{mot}. The user wants to get all employee names across all
variants. They express this intent by the query $\vQ_3$:
%
\begin{align*}
\vQ_3 &= 
  \vOne\chcL
    (\pi_{\name}(\engemp)) \cup (\pi_{\name}(\othemp)) \\
 & \hspace{32pt},
    (\vTwo\vee\vThree)\chcL
      \pi_{\name}(\empacct) \\
 & \hspace{88pt},
      \chc[(\vFour\vee\vFive)]{\pi_{\name,\fname,\lname}\empbio, \emp}\chcR\chcR
\end{align*}
%
Since the variational schema enforces that exactly one of \vOne--\ \vFive\ be enabled, we
can simplify the query by omitting the final choice.
%
\begin{align*}
\vQ_4 &= 
  \vOne\chcL
    (\pi_{\name}(\engemp)) \cup (\pi_{\name}(\othemp)) \\
 & \hspace{32pt},
    \chc[(\vTwo\vee\vThree)]{
      \pi_{\name}(\empacct),
      \pi_{\name,\fname,\lname}(\empbio)}
\end{align*}
%
\end{example}

In principle, variational queries can also express arbitrarily different intents over
different database variants. However, we expect that variational queries are best used to
capture single (or at least related) intents that vary in their realization
since this is easier to understand and increases the potential for sharing in
both the representation and execution of a variational query.







%\subsection{VRA Type System}
\label{sec:typesys}

\TODO{type sys}


\section{Variational Query Language Properties}
\label{sec:vqlprop}

\TODO{prop. show for vra.}





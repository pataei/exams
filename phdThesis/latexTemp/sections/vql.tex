\chapter{Variational Queries}
\label{ch:vql}

\rewrite{change all $\vFour \vee \vFive$ to $\neg \vThree$ in all sections and
update the derivation tree examples to the final type system. }

\eric{pls read the following paragraph}
Now that we have introduced the variational database framework 
we need a query language to extract information from a VDB instance
and unlike relational query languages (like SQL and relational algebra)
it needs to account for the new aspect of our database: variation. 


%\point{queries need to be able to express variability encoded in vdb.}
%The variational nature of a VDB requires a query language that
%accounts for variation directly.
%To express and represent variation in queries,
%we incorporate choice calculus~\cite{Walk13thesis, EW11tosem}  into a 
%structured query language. 
We formally define 
\emph{variational relational algebra} (VRA) in \secref{vrel-alg}
as our algebraic query language.
A query written in VRA is called a \emph{variational query} (\emph{v-query});
we use query and variational query interchangeably when it is clear from context. 
Unlike relational queries that convey an intent over a single database, 
a variational query typically conveys the same intent over several 
relational database variants. However, a single variational query is also capable of capturing different 
intents over different database variants.
%Consequently, the expressiveness of variational queries may cause them to be 
%more complicated than relational queries, discussed in \secref{type-sys}. 
%Hence, 

\eric{pls read the following paragraph}
To understand the meaning of variational queries
we define the semantics of variational queries via the
semantics of relational queries in \secref{vrasem}. We define
approaches to configure a variational query to a relational query
in \secref{vraconf}. Then, we use the results of multiple relational
queries to accumulate the result of the original variational query 
in \secref{accum}.
%\maybeAdd{add direct sem of VRA if time allows and equiv in vql prop.}

Due to the expressiveness of variational queries, 
we define a type system for VRA that statically checks a
variational query against the underlying variational schema in \secref{type-sys}.
%
To make variational queries more useable we relieve the user from repeating 
the variational schema's variation in their variational queries. This is achieved by 
explicitly annotating queries in \secref{constrain}.
%In \secref{constrain}, we define an operation that explicitly annotates a
%variational query with information contained in a v-schema. 
%This operation is useful to
%define the \emph{variation-preservation} property for VRA and its type system,
%which is discussed in \secref{var-pres}, and demonstrates how our framework
%satisfies the information need \nTwo.
We then define the \emph{variation-preservation} property for VRA at
the type level in \secref{var-pres}.
% which proves that our framework
%satisfies the requirement \nTwo.
%
Finally, we provide 
%we close out this section by providing 
a set of syntactic rules that are semantic-preserving 
in \secref{var-min} that enable factoring and distributing
variation points within a variational query, which enables syntactic refactorings
including maximizing sharing within a variational query.
%for reducing a query's variation.


\section{Variational Relational Algebra}
\label{sec:vra}

\TODO{vra}

\subsection{VRA Configuration}
\label{sec:vraconf}

\TODO{vra configuration}


\section{VRA Semantics }
\label{sec:vrasem}

%\TODO{vra semantics. we understand it through RA sem + accumulation}

We use the semantics of relational queries to define the semantics of 
variational queries. In \secref{vraconf}, we define the configuration function
for variational queries which takes a configuration and a variational query
and returns a relational query. We also define another version of the
variational query configuration function that generates unique relational
query variants. Then, in \secref{accum}, we define an accumulation function that accumulates
multiple (annotated) relational tables into a variational table. Finally, in \secref{vradensem}, we  
define the denotational semantics of VRA using the defined configuration and
accumulation functions.
%
%\maybeAdd{if have time add VRA sem + equiv}
%dentoational semantics of VRA
%equivalence of dent sem and config and accumulation  --> in properties section



\section{VRA Type System}
\label{sec:type-sys}

\TODO{type sys}
\TODO{work on the derivation tree example separate from the text since youre waiting for eric on the text}

\section{Variation-Minimization Rules}
\label{sec:var-min}

%
%\maybeAdd{add $\vQ_6$ is simplified of  $\VVal \vQ_6$ because of rule application blah blah.}
%\maybeAdd{add example + more rules + point out interesting ones}
%
VRA is flexible since an information need can be represented via multiple
variational queries as demonstrated in \exref{vq-specific} and \exref{vq-same-intent-mult-vars}.
It allows users to incorporate their personal taste and task requirements
into variational queries they write by 
having different levels of variation. For example, consider the explicitly annotated query
\ensuremath{\vQ_6} 
in \secref{constrain}.
%\ensuremath {
\[
\vQ_6 =
\pi_{\optAtt [\vFour \vee \vFive] [\empno], \optAtt [\vFour] [\name], \optAtt [\vFive] [\fname], \optAtt [\vFive] [\lname]  } \left( \chc [\fModel_2] {\empbio, \empRel}\right)
\]
%}.
%\vQ_5 =  \pi_{\optAtt [\vFour \vee \vFive] [\empno], \optAtt [\vFour] [\name], \optAtt [\vFive] [\fname], \optAtt [\vFive] [\lname]  } \empbio}.
%from \exref{vq-specific}. 
To be explicit about the exact query that will be run for 
each variant 
%and knowing that 
%\ensuremath{
%\getPC \empbio = \vThree \vee \vFour \vee \vFive
%},
the query $\vQ_6$'s variation can be \emph{lifted up} by using choices, resulting in the query $\VVVal \vQ_6$.
%\ensuremath{
%\small
\[
\VVVal \vQ_6 = \chc [\vFour] {\pi_{\empno, \name} \empbio, 
\chc [\vFive] {\pi_{\empno, \fname, \lname} \empbio, \emp}} 
\]
%}.
While \ensuremath{\vQ_6} contains less redundancy \ensuremath{\VVVal \vQ_6}
is more comprehensible since the variants are explicitly stated in the dimension of the choice. 
Thus, \emph{supporting multiple levels of variation 
creates a tension between reducing redundancy and maintaining comprehensibility.}

We define \emph{variation minimization} rules in \figref{var-min} that are syntactic and 
preserve the semantics.
% and include 
%interesting ones in \secref{var-min}.
Pushing in variation into a query, i.e., applying rules left-to-right, 
reduces redundancy
% and improves performance
while lifting them up, i.e., applying rules right-to-left, 
makes a query more understandable. 
When applied left-to-right, the rules are terminating since the scope of variation 
%always gets smaller.
monotonically decreases in size.
%
%\revised{
%Additionally, these rules can be used to simplify queries after
%explicitly annotating them with a schema. For example, the first rule in \figref{var-min}
%is used to simplify the query \ensuremath{\constrain [\vSch_2] {\vQ_1}}, introduced in \secref{var-pres},
% which resulted
%in \ensuremath{\vQ_6}.}


\begin{figure}
\textbf{Choice Distributive Rules:}
\begin{alignat*}{1}
\small
%-- f<? l? q?, ? l? q?> ? ? (f<l?, l?>) f<q?, q?>
%\inferrule
%{}
%\chc {\pi_{\vAttList_1} \vQ_1, \pi_{\vAttList_2} \vQ_2 } 
%&\equiv
%\pi_{\chc {\vAttList_1, \vAttList_2}} \chc {\vQ_1, \vQ_2}\\
%-- f<? l? q?, ? l? q?> ? ? ((l??), (l? \^�f )) f<q?, q?>
%\inferrule
%{}
\chc {\pi_{\vAttList_1} \vQ_1, \pi_{\vAttList_2} \vQ_2}
&\equiv
\pi_{\annot \vAttList_1, \annot [\neg \dimMeta] \vAttList_2} \chc {\vQ_1, \vQ_2}\\
%-- f<? c? q?, ? c? q?> ? ? f<c?, c?> f<q?, q?>
%\inferrule
%{}
\chc {\sigma_{\vCond_1} \vQ_1, \sigma_{\vCond_2} \vQ_2} 
&\equiv
\sigma_{\chc {\vCond_1, \vCond_2}} \chc {\vQ_1, \vQ_2}\\
%-- f<q? � q?, q? � q?> ? f<q?, q?> � f<q?, q?>
%\inferrule
%{}
\chc {\vQ_1 \times \vQ_2, \vQ_3 \times \vQ_4}
&\equiv
\chc {\vQ_1, \vQ_3} \times \chc {\vQ_2, \vQ_4}\\
%-- f<q? ?\_c? q?, q? ?\_c? q?> ? f<q?, q?> ?\_(f<c?, c?>) f<q?, q?>
%\inferrule
%{}
\chc {\vQ_1 \Join_{\vCond_1} \vQ_2, \vQ_3 \Join_{\vCond_2} \vQ_4}
&\equiv
\chc {\vQ_1, \vQ_3} \Join_{\chc {\vCond_1, \vCond_2}} \chc {\vQ_2, \vQ_4}\\
%-- f<q? ? q?, q? ? q?> ? f<q?, q?> ? f<q?, q?>
%\inferrule
%{}
\chc {\vQ_1 \circ \vQ_2, \vQ_3 \circ \vQ_4}
&\equiv
\chc {\vQ_1, \vQ_3} \circ \chc {\vQ_2, \vQ_4}
%-- f<q? ? q?, q? ? q?> ? f<q?, q?> ? f<q?, q?>
%\inferrule
%{}
%{-}
\end{alignat*}

\medskip
\textbf{CC and RA Optimization Rules Combined:}
\begin{alignat*}{1}
\small
%-- f<? (c? ? c?) q?, ? (c? ? c?) q?> ? ? (c? ? f<c?, c?>) f<q?, q?>
%\inferrule
%{}
\chc {\sigma_{\vCond_1 \wedge \vCond_2} \vQ_1, \sigma_{\vCond_1 \wedge \vCond_3} \vQ_2}
&\equiv
\sigma_{\vCond_1 \wedge \chc {\vCond_2, \vCond_3}} \chc {\vQ_1, \vQ_2}\\
%-- ? c? (f<? c? q?, ? c? q?>) ? ? (c? ? f<c?, c?>) f<q?, q?>
%\inferrule
%{}
\sigma_{\vCond_1} \chc {\sigma_{\vCond_2} \vQ_1, \sigma_{\vCond_3} \vQ_2}
&\equiv
\sigma_{\vCond_1 \wedge \chc {\vCond_2, \vCond_3}} \chc {\vQ_1, \vQ_2}\\
%-- f<q? ?\_(c? ? c?) q?, q? ?\_(c? ? c?) q?> ? ? (f<c?, c?>) (f<q?, q?> ?\_c? f<q?, q?>)
%\inferrule
%{}
\chc {\vQ_1 \Join_{\vCond_1 \wedge \vCond_2} \vQ_2, \vQ_3 \Join_{\vCond_1 \wedge \vCond_3} \vQ_4}
&\equiv
\sigma_{\chc {\vCond_2, \vCond_3}} \left( \chc {\vQ_1, \vQ_3} \Join_{\vCond_1} \chc {\vQ_2, \vQ_4} \right)
\end{alignat*}

\caption{Some of variation minimization rules.}
\label{fig:var-min}
\end{figure}

%\section{VRA Type System}
\label{sec:type-sys}

\TODO{type sys}
\TODO{work on the derivation tree example separate from the text since youre waiting for eric on the text}

\section{VRA Semantics }
\label{sec:vrasem}

%\TODO{vra semantics. we understand it through RA sem + accumulation}

We use the semantics of relational queries to define the semantics of 
variational queries. In \secref{vraconf}, we define the configuration function
for variational queries which takes a configuration and a variational query
and returns a relational query. We also define another version of the
variational query configuration function that generates unique relational
query variants. Then, in \secref{accum}, we define an accumulation function that accumulates
multiple (annotated) relational tables into a variational table. Finally, in \secref{vradensem}, we  
define the denotational semantics of VRA using the defined configuration and
accumulation functions.
%
%\maybeAdd{if have time add VRA sem + equiv}
%dentoational semantics of VRA
%equivalence of dent sem and config and accumulation  --> in properties section



\subsection{VRA Configuration}
\label{sec:vraconf}


\wrrite{add the example from vra and show config and tables.}
\TODO{example following the vra example}

%\NOTE{
%Also, the following definition of the semantics contradicts with the
%description earlier in the section about producing a \emph{result}
%relation.
%
%\medskip
%Also also, maybe we should move the discussion of the semantics before the
%examples? It's a bit surprising to come across it here.}

%The semantics of VRA can be understood as a combination of the
%\emph{configuration semantics} of VRA, defined in \figref{v-alg-conf-sem}, the
%configuration semantics of VDBs, defined in \figref{vdb-conf}, and the
%semantics of plain RA.
%%
%%\TODO{Make the following a more precise description of how these three
%%semantics work together, i.e.\ for every valid configuration of the feature
%%model, we can configure the variational query and VDB in the same way to yield a plain RA
%%query that is then executed over the corresponding plain RDB.}
%%
%Thus, the variational query
%semantics is the set of semantics of its configured relational queries over
%their corresponding configured relational database variant for every valid
%configuration of the feature model of the VDB.
%
We now embark on the formal definition of variational queries configuration.
The \emph{configuration} function maps a variational query under
a configuration
to a relational query, defined in \figref{v-alg-conf-sem}. Thus, a variational query 
can be understood as a set of relational queries that their results is gathered
in a single table and tagged with the feature expression stating their variant.
%Configuring a variational query
%for all valid configurations, accessible from VDB's feature model,
%provides the complete meaning of a variational query in terms of RA semantics.
%
Users can deploy queries for a specific variant by configuring 
them.
%The configuration of a query allows users to deploy queries for a
%specific variant when they desire, 
%satisfying query part of \nThree\ requirement. 
\exref{conf-vq} illustrates configuring a query.
% stated in \secref{mot}.

%To define VRA semantics we map 
%a variational query to a pure relational query to re-use RA's semantics.
%However, to avoid losing the variation encoded 
%in the variational query, 
%we need to determine the variant under which such a
%mapping is valid. Thus, we introduce the semantic functions that 
%relate a variational query to a relational query.

%
%\textbf{Configuring a variational query:} 
%It maps a variational query under a 
%given configuration to a relational query, denoted by \eeSem . 
%and defined in \figref{v-alg-conf-sem}. Configuring a variational query
%for all valid configurations, accessible from VDB's feature model,
%provides the complete meaning of a variational query in terms of RA semantics.
%Users can deploy queries for a specific variant by configuring 
%them,
%%The configuration of a query allows users to deploy queries for a
%%specific variant when they desire, 
%satisfying query part of \nThree.

\begin{example}
\label{eg:conf-vq}
Assume the underlying VDB has the variational schema
% \t\ feature model and the variational relation
\ensuremath{
\vSch_3 = \{ \vRel \left( \optAtt [\fOne] [\vAtt_1], \vAtt_2, \vAtt_3 \right)^{\fOne \vee \fTwo}
\}} 
and the feature space 
\ensuremath{
\fSet = \setDef{ \fOne, \fTwo}}.
The variational query 
\ensuremath{
\vQ_5 = \vPrj [{\vAtt_1, \optAtt [\fOne \wedge \fTwo] [\vAtt_2], \optAtt [\fTwo] [\vAtt_3]}] (\vRel)
}
is configured to the following relational queries:
\ensuremath{\eeSem [\setDef \fOne] {\vQ_5} = \eeSem [\setDef \ ] {\vQ_5} = \pi_{\pAtt_1} \pRel},
\ensuremath{\eeSem [\setDef \fTwo] {\vQ_5} =
 \pi_{\pAtt_1, \pAtt_3} \pRel},
\ensuremath{\eeSem [\setDef {\fOne, \fTwo}] {\vQ_5} = \pi_{\pAtt_1, \pAtt_2, \pAtt_3} \pRel}.
\end{example}



\begin{figure}
\textbf{Variational condition configuration:}
\begin{alignat*}{1}
\ecSem [] . &: \vCondSet \totype \confSet \totype \pCondSet\\
%
\ecSem \bTag &= \bTag \\
%
\ecSem \vAttOpCte &= 
    \vAttOpCte\\
%	\begin{cases}
%		\vAttOpCte, &\text{ if } \pAtt \in \attr [\eeSem \vRel]\\
%		\f, &\text{ otherwise}
%	\end{cases}\\
%
\ecSem \vAttOpAtt &= 
       \vAttOpAtt\\
%	\begin{cases}
%		\pAttOpAtt, &\text{ if } \pAtt_1 \in \attr [\eeSem \vRel] \&\ 
%		                                   \pAtt_2 \in \attr [\eeSem \vRel] \\
%		\f,  &\text{ otherwise}
%	\end{cases}\\
%
\ecSem {\neg \vCond} &= \neg \ecSem \vCond\\
%
\ecSem {\orr \vCond} &= \ecSem {\vCond_1} \vee \ecSem {\vCond_2}\\
%
\ecSem {\annd \vCond} &= \ecSem {\vCond_1} \wedge \ecSem {\vCond_2}\\
%
\ecSem {\chc {\vCond_1, \vCond_2}} &=
	\begin{cases}
		\ecSem {\vCond_1}, &\text{ if } \fSem \dimMeta  = \t \\
		\ecSem {\vCond_2}, &\text{ otherwise}
	\end{cases}
\end{alignat*}
%\caption[Configuration of variational conditions]{Configuration of variational conditions. 
%The configuration function assumes that the input is well-typed. 
%%\orSem ., \ecSem ., and \olSem . are
%%configuration of v-relation, v-condition, and variational attribute
%%set, defined in \figref{vdb-conf} of \appref{vdb-conf}.
%%set, respectively, defined in \figref{vdb-conf}, 
%%\figref{vcond-conf-sem}, \figref{vdb-conf}.
%%Note that we have extended RA with an empty relation $\underline {\empRel}$.
%}
%\label{fig:v-cond-conf-sem}
%\end{figure}
%
%\begin{figure}

\medskip
\textbf{Variational query configuration:}
\begin{alignat*}{1}
\eeSem [] . &: \qSet \totype \confSet \totype \pQSet\\
%
\eeSem \vRel &= \orSem \vRel = \pRel\\
\eeSem {\vSel \vQ}  &= \vSel [\ecSem \vCond] {\eeSem \vQ}\\
%
\eeSem {\vPrj [\vAttList] \vQ} &= \vPrj [\olSem \vAttList] {\eeSem \vQ}\\
%
%\eeSem {\vPrj [\{\}] \vQ} &= \empPRel\\
%
\eeSem {{\vQ_1} \times {\vQ_2}} &= \eeSem {\vQ_1} \times \eeSem {\vQ_2}\\
%
%\eeSem {{\vQ_1} \Join_\vCond {\vQ_2}} &= \eeSem {\vQ_1} \Join_{\ecSem \vCond} \eeSem {\vQ_2}\\
%
\eeSem {\chc {\vQ_1, \vQ_2}} &= 
	\begin{cases}
		\eeSem {\vQ_1}, \text{ if } \fSem \dimMeta = \t\\
		\eeSem {\vQ_2}, \text{ otherwise}
	\end{cases}\\
%
\eeSem {{\vQ_1} \circ {\vQ_2}} &= \eeSem {\vQ_1} \circ \eeSem {\vQ_2}\\
%
\eeSem {\empRel} &= \empPRel
\end{alignat*}
\caption[Configuration of variational queries and conditions]{Configuration of variational queries and conditions. 
The configuration function assumes that the input, either the variational query or the variational condition,
is well-typed. 
%\orSem ., \ecSem ., and \olSem . are
%configuration of v-relation, v-condition, and variational attribute
%set, defined in \figref{vdb-conf} of \appref{vdb-conf}.
%set, respectively, defined in \figref{vdb-conf}, 
%\figref{vcond-conf-sem}, \figref{vdb-conf}.
%Note that we have extended RA with an empty relation $\underline {\empRel}$.
}
\label{fig:v-alg-conf-sem}
\end{figure}


%\textbf{Grouping a variational query:} 
%maps a variational query to a set of
%relational queries annotated with feature expressions, denoted by \qGroup .
%and defined in \figref{vq-group}. The presence condition of relational queries 
%indicate the group of configurations where the mapping holds. In essence, 
%grouping of variational query \vQ\ groups together all configurations with the same relational
%query produced from configuring \vQ. 
%Hence, the generated set
%%\dropit{could drop this if it's confusing!}
%of relational queries from grouping a variational query contains distinct (unique) queries.
%For example, consider the query \ensuremath {\vQ_5} in \exref{conf-vq}.
%Grouping \ensuremath{\vQ_5} results in the set:
%\ensuremath{
%\setDef{
%\left( \pi_{\pAtt_1, \pAtt_2, \pAtt_3} \pRel \right)^{\fOne \wedge \fTwo},
%\left(\pi_{\pAtt_1, \pAtt_3} \pRel \right)^{\neg \fOne \wedge \fTwo},
%\left(  \pi_{\pAtt_1} \pRel \right)^{( \fOne \wedge \neg \fTwo) \vee (\neg \fOne \wedge \neg \fTwo)}
%}
%}.
%
%
%

\eric{pls read the following paragraph and the \figref{vq-group}}
\TODO{write this + correct formulas + give an example.
The trick is to attach the condition under which a 
RA query is valid. This condition is a feature expression generated from
the set of configurations that configured the v-query into the same RA query, i.e.,
\ensuremath{
\qGroup \vQ = \setDef {\annot \pQ \myOR \forall \config \in \confSet: \fSem \dimMeta = \t,
\eeSem \vQ = \pQ}
}.
}

\begin{figure}
%\textbf{Configuration selection semantics of \vqsTxt:}
\begin{alignat*}{1}
\qGroup . &: \qSet \to \ddot \pQSet\\
%
\qGroup \vRel &= \annot [\t] \pRel\\
\qGroup {\vSel \vQ}  &=  
\setDef {\annot [\dimMeta \wedge \dimMeta_\vCond] {\left(\sigma_{\pCond} \pQ\right)} \myOR
\annot \pQ \in \qGroup \vQ, \annot [\dimMeta_\vCond] \pCond \in \cGroup}
\\
%
\qGroup {\vPrj [\vAttList] \vQ} &= 
\setDef {\annot [\dimMeta \wedge \dimMeta_\vAttList] {\left(\pi_{\pAttList} \pQ \right)} \myOR
\annot \pQ \in \qGroup \vQ, \annot [\dimMeta_\vAttList] \pAttList \in \aGroup}
\\
%
\qGroup {{\vQ_1} \times {\vQ_2}} &= 
\setDef {\annot [\dimMeta_1 \wedge \dimMeta_2] {\left(\pQ_1 \times \pQ_2\right)} \myOR
\annot [\dimMeta_1] \pQ_1 \in \qGroup {\vQ_1}, \annot [\dimMeta_2] \pQ_2 \in \qGroup {\vQ_2} }
\\
%
\qGroup {{\vQ_1} \Join_\vCond {\vQ_2}} &= 
\setDef {\annot [\dimMeta_1 \wedge \dimMeta_2 \wedge \dimMeta_\vCond] {\left(\pQ_1 \Join_{\pCond} \pQ_2 \right)} \myOR 
\annot [\dimMeta_1] \pQ_1 \in \qGroup {\vQ_1}, \annot [\dimMeta_2] \pQ_2 \in \qGroup {\vQ_2}\\
& \hspace{104pt}
,\annot [\dimMeta_\vCond] \pCond \in \cGroup  }
\\
%
\qGroup {\chc {\vQ_1, \vQ_2}} &= 
\setDef {\annot [\dimMeta \wedge \dimMeta_1] \pQ_1 \myOR  \annot [\dimMeta_1] \pQ_1 \in \qGroup {\vQ_1} }
\cup 
\setDef {\annot [\neg \dimMeta \wedge \dimMeta_2] \pQ_2 \myOR  \annot [\dimMeta_2] \pQ_2 \in \qGroup {\vQ_2}}  \\
%
\qGroup {{\vQ_1} \circ {\vQ_2}} &= 
\setDef {\annot [\dimMeta_1 \wedge \dimMeta_2] {\left(\pQ_1 \circ \pQ_2\right)} \myOR
\annot [\dimMeta_1] \pQ_1 \in \qGroup {\vQ_1}, \annot [\dimMeta_2] \pQ_2 \in \qGroup {\vQ_2} }\\
%
\qGroup {\empRel} &= \annot [\t] {\underline \empRel}
\end{alignat*}
\caption[\TODO{shortcaption}]{Grouping of v-queries. \cGroup\ and \aGroup\ indicate the grouping of v-conditions and
variational attribute lists, respectively. They both follow the definition provided in 
\secref{impl}.
\TODO{define $\ddot \pQSet$}
}
\label{fig:vq-group}
\end{figure}

\subsection{Accumulation of Relational Tables to a Variational Table}
\label{sec:accum}

\eric{pls read this entire section. thx!}
After connecting variational queries to relational queries, to define the 
semantics of VRA we need to connect
the results of multiple relational queries to the result of a single variational 
query. 
%
Since we have two approaches to connect a variational query to relational queries 
we define two \emph{accumulation} functions that generate a 
variational table from a set of relational tables. 
%
The first accumulation function $\mathit{accum} : \settype \fSet \totype \settype {\typepair \confSet \pTabSet} \totype \tabletype$ takes the feature space of a database and a set of relational
tables with their attached configurations and generates a variational table. \figref{accum1} 
defines this function in terms of some auxiliary functions. 
%
The $\mathit{mkTable}$ function takes a variational relation schema and a set of 
variational relation contents and generates a variational table that has the given schema
and the variational tuples in the input tables. 
%
The $\mathit{addPresCondToConfTables}$ function maps the $\mathit{addPresCondToConfContent}$
over a set of tables and their attached configuration and the  $\mathit{addPresCondToConfContent}$
function adds the \pcatt\ attribute to a relational table and its corresponding value which is 
a feature expression associated with the given configuration using the closed set of
features.
%
The $\mathit{fitConfTablesToVsch}$ maps the function $\mathit{fitTableToVsch}$ to tables of a set of 
relational tables and their attached configuration.
The $\mathit{fitTableToVsch}$ function adjusts a table, both its schema and content, 
to a variational relation schema.
%
The $\mathit{tablesToVsch}$ maps the function $\mathit{schToVsch}$ to a set of 
relational tables and their attached configuration. 
The $\mathit{schToVsch}$ generates a variational relation schema from a set of
plain relation schema and their attached configuration given the closed set of 
features of the database's feature space.
%
Note that to generate a feature expression from a configuration it is essential to
pass the closed set of features.
%
\exref{acc-table-from-conf} illustrates the behavior of these auxiliary functions and the
table accumulation function over the relational tables in \tabref{vq-conf-res}.


\begin{figure}

\textbf{Table accumulation function:}
\begin{alignat*}{1}
\mathit{accum} &: \settype \fSet \totype \settype {\typepair \confSet \pTabSet} \totype \tabletype\\
\mathit{accum} \  \mathit{fs} \ \mathit{ts} &= \mathit{mkTable} \ \mathit{vsch} \ \mathit{tables}\\
%&\hspace{60pt} (\mathit{addPresCondToConfTables} \ \mathit{fs} \\
%&\hspace{140pt} (\mathit{fitConfTablesToVsch} \ \mathit{ts} \ \mathit{vsch}))\\
&\hspace{-40pt}\textit{where }
\mathit{vsch} = \mathit{tablesToVsch} \ \mathit{fs} \ \mathit{ts}\\
&\hspace{-6pt} \mathit{tables} = \mathit{addPresCondToConfTables} \ \mathit{fs} \ \mathit {fitted}\\
&\hspace{-6pt} \mathit{fitted} = \mathit{fitConfTablesToVsch} \ \mathit{ts} \ \mathit{vsch}
\end{alignat*}


\medskip 
\textbf{Auxiliary functions for table accumulation:}
\footnotesize
\begin{alignat*}{1}
\mathit{schToVsch} &: \settype \fSet \totype \settype {\typepair \confSet \pRelSchSet} \totype \vRelSchSet\\
\mathit{tablesToVsch} &: \settype \fSet \totype \settype {\typepair \confSet \pTabSet} \totype \vRelSchSet\\
\mathit{fitTableToVsch} &: \pTabSet \totype \vRelSchSet \totype \pTabSet\\
\mathit{fitConfTablesToVsch} &: \settype {\typepair \confSet \pTabSet} \totype \vRelSchSet \totype \settype {\typepair \confSet \pTabSet}\\
\mathit{addPresCondToConfContent} &: \settype \fSet \totype \typepair \confSet \pRelContSet \totype \vRelContSet\\
\mathit{addPresCondToConfTables} &: \settype \fSet \totype \settype {\typepair \confSet \pTabSet} \totype \settype \vRelContSet\\
\mathit{mkTable} &: \vRelSchSet \totype \settype \vRelContSet \totype \tabletype
\end{alignat*}


\caption[Accumulation function of a set of relational tables with their attached configuration into a variational table]{Accumulation function of a set of relational tables with their attached configuration into a variational table and its auxiliary functions. The definition uses spaces to pass parameters. For example, $f \ x$ states that the parameter $x$ is passed to the function $x$ and $f\ x\ y$ states that
parameters $x$ and $y$ are passed to $f$ as the first and second arguments, respectively.
}
\label{fig:accum1}
\end{figure}



\begin{example}
\label{eg:acc-table-from-conf}
Consider the query $\VVal {\vQ_1}$ written over the VDB with variational schema $\vSch_2$ and 
feature space $\fSet = \setDef {\vThree, \vFour, \vFive}$, all given in \exref{vq-specific}. 
%
All configured relational queries of $\VVal {\vQ_1}$ for VDDB's valid configurations and their
corresponding results in form of a relational table are given
in \exref{vq-sem} and \tabref{vq-conf-res}, respectively. 
%
Now we show how the relational tables of the configured queries, shown in \tabref{}vq-conf-res, are accumulated 
to the variational table, shown in \tabref{vq1-res}, as the result of the variational query $\VVal \vQ_1$ by 
using the table accumulation function $\mathit{accum}$.
%
%\tabref{}
\end{example}

The second accumulation function
 $\VVal {\mathit{accum}} :  \settype {\bm{(}\vartype \pTabSet\bm{)}} \totype \tabletype$ 
 takes a set of relational tables that are annotated with
a feature expression instead of their attached configuration. \figref{accum2} defines
this function and its auxiliary functions. The auxiliary functions are similar to the ones
defined in \figref{accum2} except that they do not need to generate a feature expression
from a configuration and a set of closed features.
%
\exref{acc-table-from-group} illustrates the behavior of these auxiliary functions and the second accumulation
function over the tables in \tabref{vq-conf-res}.

\begin{figure}

\textbf{Table accumulation function:}
\begin{alignat*}{1}
\VVal {\mathit{accum}} &:  \settype {\bm{(}\vartype \pTabSet\bm{)}} \totype \tabletype\\
\VVal {\mathit{accum}} \  \mathit{fs} \ \mathit{ts} &= \mathit{mkTable} \ \mathit{vsch} \ \mathit{tables}\\
&\hspace{-40pt}\textit{where }
\mathit{vsch} = \mathit{annotTablesToVsch}  \ \mathit{ts}\\
&\hspace{-6pt} \mathit{tables} = \mathit{addPresCondToVarTables} \ \mathit {fitted}\\
&\hspace{-6pt} \mathit{fitted} = \mathit{fitVarTablesToVsch} \ \mathit{ts} \ \mathit{vsch}
\end{alignat*}


\medskip 
\textbf{Auxiliary functions for table accumulation:}
\footnotesize
\begin{alignat*}{1}
\mathit{annotSchToVsch} &:  \settype {\bm{(}\vartype \pRelSchSet\bm{)}} \totype \vRelSchSet\\
\mathit{annotTablesToVsch} &:  \settype {\bm{(}\vartype \pTabSet\bm{)}} \totype \vRelSchSet\\
%\mathit{fitTableToVsch} &: \pTabSet \totype \vRelSchSet \totype \pTabSet\\
\mathit{fitVarTablesToVsch} &: \settype {\bm{(}\vartype \pTabSet\bm{)}} \totype \vRelSchSet \totype \settype {\bm{(}\vartype \pTabSet\bm{)}}\\
\mathit{addPresCondToVarContent} &:  \vartype \pRelContSet \totype \vRelContSet\\
\mathit{addPresCondToVarTables} &:  \settype {\bm{(}\vartype \pTabSet\bm{)}} \totype \settype \vRelContSet
%\mathit{mkTable} &: \vRelSchSet \totype \settype \vRelContSet \totype \tabletype
\end{alignat*}


\caption[Accumulation function of a set of relational tables annotated with a feature expression into a variational table]{Accumulation function of a set of relational tables annotated with a feature expression into a variational table and its auxiliary functions. The definition uses spaces to pass parameters, e.g., $f \ x = f(x)$ and $f \ x \ y = f(x,y)$.
}
\label{fig:accum1}
\end{figure}



\begin{example}
\label{eg:acc-table-from-group}
\wrrite{write this}
\end{example}
\subsection{VRA Denotational Semantics }
\label{sec:vradensem}

\eric{pls read this entire section. thx!}
Now that we have all required parts we define the denotational semantics of 
variational relational algebra using the denotational semantics of relational 
algebra. The denotational semantics of relational queries 
$\mathit{rqSem} : \pQSet \totype \pInstSet \totype \pTabSet$ takes a plain query and
a plain database and returns a table named $\mathit{result}$. We do not give a formal 
definition of $\mathit{rqSem}$, however, examples of the semantics
of a query are given in \tabref{vq-conf-res}.
%
We then define the VRA denotational semantics 
$\mathit{vqSem} : \qSet \totype \vdbInstSet \totype \tabletype$ as the 
accumulation of relational tables resulted from the semantics of its
configured queries over their corresponding configured databases for all 
valid configurations of a variational database. 
%
The $\mathit{mapRQSem}$ takes a set of relational queries with their attached
configurations and a variational database instance and returns the set of query 
semantics over their configured database
with their attached configurations, that is, it maps $\mathit{rqSem}$ on the 
relational queries over their corresponding relational database.%
\footnote{In implementation, the closed set of features and valid configurations
of a VDB are contained within, instead of extracting them from the database. However,
we keep the formalization simple and assume that they can also be retrieved from
the VDB.}

%\TODO{think if you want to have example or more explanation of helpers.}

\begin{figure}

\textbf{VRA denotational semantics:}
\begin{alignat*}{1}
\mathit{vqSem} &: \qSet \totype \vdbInstSet \totype \tabletype\\
\mathit{vqSem} \  \vQ \ \vdbInst &= \mathit{accum} \ \mathit{fs} \ \mathit{tabs}\\
&\hspace{-30pt}\textit{where } \mathit{fs} =\mathit{featues} \ \vdbInst\\
&\hspace{2pt} \mathit{rqs} = \mathit{qToConfRelQs} \ \vQ \ (\mathit{validConfigs} \ \vdbInst)\\
&\hspace{2pt} \mathit{tabs} = \mathit{mapRQSem} \ \mathit{rqs} \ \vdbInst
\end{alignat*}


\medskip 
\textbf{Auxiliary functions for VRA denotational semantics:}
%\footnotesize
\begin{alignat*}{1}
\mathit{rqSem} &: \pQSet \totype \pInstSet \totype \pTabSet\\
\mathit{mapRQSem} &: \settype {\typepair \confSet \pQSet} \totype \vdbInstSet \totype \settype {\typepair \confSet \pTabSet}\\
\mathit{features} &: \vdbInstSet \totype \settype \fSet\\
\mathit{validConfigs} &: \vdbInstSet \totype \settype \confSet\\
\mathit{qToConfRelQs} &: \qSet \totype \settype \confSet \totype \settype {\typepair \confSet \pQSet}\\
\end{alignat*}


\caption[VRA denotational semantics]{Denotational semantics of variational relational algebra.
Note that the query \vQ\ is well-typed and explicitly annotated by the schema of the VDB instance \vdbInst.
}
\label{fig:densem}
\end{figure}



%\begin{example}
%\label{eg:sem}
%\wrrite{write the damn thing}
%\end{example}


\section{VRA Type System}
\label{sec:type-sys}

\TODO{type sys}
\TODO{work on the derivation tree example separate from the text since youre waiting for eric on the text}

\section{Explicitly Annotating Queries}
\label{sec:constrain}

%\point{type system allows the ql to be flexible and usable.}
%The type system is designed s.t. it relieves the user from necessarily incorporating
%the v-schema variability into their queries as long as the variational queries variability
%does not violate the v-schema, 
Variational queries do not need to repeat information that can be inferred from the variational schema
or the type of a query.
%
For example, the query \ensuremath{\vQ_1} shown in \exref{vq-specific} 
does not contradict the schema and
thus is type correct. However,
 it does not include the presence conditions of attributes and the relation encoded in
the schema while \ensuremath{\vQ_6} repeats this information:\\
%
\centerline{
\ensuremath{
\vQ_6 =
\pi_{\optAtt [(\vFour \vee \vFive) \wedge \neg \vThree] [\empno], \optAtt [\neg \vThree \wedge \vFour \wedge \neg \vFive] [\name], \optAtt [\neg \vThree \wedge \neg \vFour \wedge \vFive] [\fname], \optAtt [\neg \vThree \wedge \neg \vFour \wedge \vFive] [\lname]  } \left(\chc [\dimMeta_2] {\empbio, \empRel} \right)}}.%
%
\footnote{
The query $\vQ_6$ is the simplified version of
\[\constrain [\vSch_2] {\vQ_1} = 
\pi_{\optAtt [(\vFour \vee \vFive) \wedge \neg \vThree] [\empno], \optAtt [\neg \vThree \wedge \vFour \wedge \neg \vFive] [\name], \optAtt [\neg \vThree \wedge \neg \vFour \wedge \vFive] [\fname], \optAtt [\neg \vThree \wedge \neg \vFour \wedge \vFive] [\lname]  } \left(\constrain [\vSch_2] {\empbio}\right)\]
where $\constrain [\vSch_2] {\empbio} = \chc [\dimMeta_2] {\pi_{\empno, \annot [\vFour] \name, \annot [\vFive] \fname, \annot [\vFive] \lname} (\empbio),\empRel} $.
}

%\pi_{\optAtt [(\vFour \vee \vFive) \wedge \fModel_2] [\empno], \optAtt [\vFour \wedge \fModel_2] [\name], \optAtt [\vFive \wedge \fModel_2] [\fname], \optAtt [\vFive \wedge \fModel_2] [\lname]  } \empbio}}.
%

%\NOTE{
%This is the unsimplified version:
%\begin{align*}
%\VVal {\vQ_5} &= 
%\pi_{\optAtt [\vFour \vee \vFive] [\empno], \optAtt [\vFour] [\name], \optAtt [\vFive] [\fname], \optAtt [\vFive] [\lname]  } \\
%&(\chc [ \fModel_2 ] {\pi_{\empno, \sex, \birthdate, \optAtt [\vFour ] [\name], \optAtt [\vFive] [\fname], \optAtt [\vFive] [\lname]} \empbio, \empRel  })
%\end{align*}
%}
Similarly, the projection in the query 
\ensuremath{\vQ_7 = \pi_{\name, \fname} (\mathit{subq}_7)}
where 
\ensuremath{
\mathit{subq}_7 = \chc [ \vFour] {\pi_\name (\vQ_6), \pi_\fname (\vQ_6)}
}
is written over 
\ensuremath{\vSch_2} and it 
%\centerline{
%\ensuremath{
%\vQ_6 =
%\pi_{\name, \fname} \mathit{subq}_6
%} 
%}}
does not repeat the presence conditions of attributes from its \ensuremath{\mathit{subq}_7}'s type.
The query
%\centerline{
\ensuremath{
\vQ_8 =
\pi_{\optAtt [\vFour ] [\name],\optAtt [\neg \vFour] [\fname]} (\mathit{subq}_7)
%\chc [ \vFour] {\pi_\name \vQ_5, \pi_\fname \vQ_5}
}
%}
makes the annotations of projected attributes \emph{explicit} with respect to both 
the variational schema \ensuremath{\vSch_2} and its subquery's type.
%\TODO {give an example, schema: R(A,B), query: $\pi_{A,B} (F<\pi_A R, \pi_B R>)$
%becomes $\pi_{A^F, B^{\neg F}} ...$}
%The variation encoded in variational queries can
%be more restrictive or more loose than v-schema variation without violating them.
Although relieving the user from explicitly repeating variation makes VRA easier to use, 
queries still have to state variation explicitly to avoid losing information when 
decoupled from the schema.
%We do this by defining a function, 
%\ensuremath {\constrain \vQ}, with type \ensuremath{ \qSet \to \vSchSet \to \qSet
%},
%that \emph{explicitly annotates a query \vQ\ given the underlying schema \vSch}.
We do this by defining the function 
\ensuremath {\constrain \vQ : \qSet \totype \vSchSet \totype \qSet
},
that \emph{explicitly annotates a query \vQ\ with the  schema \vSch}.
%Note that \ensuremath {\constrain \vQ} needs to take the underlying schema as
%an input since it is using the type system (which relies on the schema) as a helper function.
The explicitly annotating query function, 
formally defined in \figref{constrain}, 
conjoins attributes and relations
presence conditions with the corresponding annotations in the query 
and wraps subqueries in a choice when needed. 
Note that, $\vQ_8$ and $\vQ_6$ are the result of $\constrain [\vSch_2] {\vQ_7}$
and $\constrain [\vSch_2] {\vQ_1}$, respectively, after simplification~\footnote{More specifically,
they are simpilified using rules defined in \figref{var-min}}.
%Queries $\vQ_7$ and $\vQ_5$ are examples of applying the 
%explicitly annotation function to queries $\vQ_6$ and $\vQ_1$, respectively,
%after simplifying them.
%\exref{constrain} illustrates how the constrain function transforms queries
%and allows users to be more flexible with their queries. 

\section{Explicitly Annotating Queries}
\label{sec:constrain}

%\point{type system allows the ql to be flexible and usable.}
%The type system is designed s.t. it relieves the user from necessarily incorporating
%the v-schema variability into their queries as long as the variational queries variability
%does not violate the v-schema, 
Variational queries do not need to repeat information that can be inferred from the variational schema
or the type of a query.
%
For example, the query \ensuremath{\vQ_1} shown in \exref{vq-specific} 
does not contradict the schema and
thus is type correct. However,
 it does not include the presence conditions of attributes and the relation encoded in
the schema while \ensuremath{\vQ_6} repeats this information:\\
%
\centerline{
\ensuremath{
\vQ_6 =
\pi_{\optAtt [(\vFour \vee \vFive) \wedge \neg \vThree] [\empno], \optAtt [\neg \vThree \wedge \vFour \wedge \neg \vFive] [\name], \optAtt [\neg \vThree \wedge \neg \vFour \wedge \vFive] [\fname], \optAtt [\neg \vThree \wedge \neg \vFour \wedge \vFive] [\lname]  } \left(\chc [\dimMeta_2] {\empbio, \empRel} \right)}}.%
%
\footnote{
The query $\vQ_6$ is the simplified version of
\[\constrain [\vSch_2] {\vQ_1} = 
\pi_{\optAtt [(\vFour \vee \vFive) \wedge \neg \vThree] [\empno], \optAtt [\neg \vThree \wedge \vFour \wedge \neg \vFive] [\name], \optAtt [\neg \vThree \wedge \neg \vFour \wedge \vFive] [\fname], \optAtt [\neg \vThree \wedge \neg \vFour \wedge \vFive] [\lname]  } \left(\constrain [\vSch_2] {\empbio}\right)\]
where $\constrain [\vSch_2] {\empbio} = \chc [\dimMeta_2] {\pi_{\empno, \annot [\vFour] \name, \annot [\vFive] \fname, \annot [\vFive] \lname} (\empbio),\empRel} $.
}

%\pi_{\optAtt [(\vFour \vee \vFive) \wedge \fModel_2] [\empno], \optAtt [\vFour \wedge \fModel_2] [\name], \optAtt [\vFive \wedge \fModel_2] [\fname], \optAtt [\vFive \wedge \fModel_2] [\lname]  } \empbio}}.
%

%\NOTE{
%This is the unsimplified version:
%\begin{align*}
%\VVal {\vQ_5} &= 
%\pi_{\optAtt [\vFour \vee \vFive] [\empno], \optAtt [\vFour] [\name], \optAtt [\vFive] [\fname], \optAtt [\vFive] [\lname]  } \\
%&(\chc [ \fModel_2 ] {\pi_{\empno, \sex, \birthdate, \optAtt [\vFour ] [\name], \optAtt [\vFive] [\fname], \optAtt [\vFive] [\lname]} \empbio, \empRel  })
%\end{align*}
%}
Similarly, the projection in the query 
\ensuremath{\vQ_7 = \pi_{\name, \fname} (\mathit{subq}_7)}
where 
\ensuremath{
\mathit{subq}_7 = \chc [ \vFour] {\pi_\name (\vQ_6), \pi_\fname (\vQ_6)}
}
is written over 
\ensuremath{\vSch_2} and it 
%\centerline{
%\ensuremath{
%\vQ_6 =
%\pi_{\name, \fname} \mathit{subq}_6
%} 
%}}
does not repeat the presence conditions of attributes from its \ensuremath{\mathit{subq}_7}'s type.
The query
%\centerline{
\ensuremath{
\vQ_8 =
\pi_{\optAtt [\vFour ] [\name],\optAtt [\neg \vFour] [\fname]} (\mathit{subq}_7)
%\chc [ \vFour] {\pi_\name \vQ_5, \pi_\fname \vQ_5}
}
%}
makes the annotations of projected attributes \emph{explicit} with respect to both 
the variational schema \ensuremath{\vSch_2} and its subquery's type.
%\TODO {give an example, schema: R(A,B), query: $\pi_{A,B} (F<\pi_A R, \pi_B R>)$
%becomes $\pi_{A^F, B^{\neg F}} ...$}
%The variation encoded in variational queries can
%be more restrictive or more loose than v-schema variation without violating them.
Although relieving the user from explicitly repeating variation makes VRA easier to use, 
queries still have to state variation explicitly to avoid losing information when 
decoupled from the schema.
%We do this by defining a function, 
%\ensuremath {\constrain \vQ}, with type \ensuremath{ \qSet \to \vSchSet \to \qSet
%},
%that \emph{explicitly annotates a query \vQ\ given the underlying schema \vSch}.
We do this by defining the function 
\ensuremath {\constrain \vQ : \qSet \totype \vSchSet \totype \qSet
},
that \emph{explicitly annotates a query \vQ\ with the  schema \vSch}.
%Note that \ensuremath {\constrain \vQ} needs to take the underlying schema as
%an input since it is using the type system (which relies on the schema) as a helper function.
The explicitly annotating query function, 
formally defined in \figref{constrain}, 
conjoins attributes and relations
presence conditions with the corresponding annotations in the query 
and wraps subqueries in a choice when needed. 
Note that, $\vQ_8$ and $\vQ_6$ are the result of $\constrain [\vSch_2] {\vQ_7}$
and $\constrain [\vSch_2] {\vQ_1}$, respectively, after simplification~\footnote{More specifically,
they are simpilified using rules defined in \figref{var-min}}.
%Queries $\vQ_7$ and $\vQ_5$ are examples of applying the 
%explicitly annotation function to queries $\vQ_6$ and $\vQ_1$, respectively,
%after simplifying them.
%\exref{constrain} illustrates how the constrain function transforms queries
%and allows users to be more flexible with their queries. 

\section{Explicitly Annotating Queries}
\label{sec:constrain}

%\point{type system allows the ql to be flexible and usable.}
%The type system is designed s.t. it relieves the user from necessarily incorporating
%the v-schema variability into their queries as long as the variational queries variability
%does not violate the v-schema, 
Variational queries do not need to repeat information that can be inferred from the variational schema
or the type of a query.
%
For example, the query \ensuremath{\vQ_1} shown in \exref{vq-specific} 
does not contradict the schema and
thus is type correct. However,
 it does not include the presence conditions of attributes and the relation encoded in
the schema while \ensuremath{\vQ_6} repeats this information:\\
%
\centerline{
\ensuremath{
\vQ_6 =
\pi_{\optAtt [(\vFour \vee \vFive) \wedge \neg \vThree] [\empno], \optAtt [\neg \vThree \wedge \vFour \wedge \neg \vFive] [\name], \optAtt [\neg \vThree \wedge \neg \vFour \wedge \vFive] [\fname], \optAtt [\neg \vThree \wedge \neg \vFour \wedge \vFive] [\lname]  } \left(\chc [\dimMeta_2] {\empbio, \empRel} \right)}}.%
%
\footnote{
The query $\vQ_6$ is the simplified version of
\[\constrain [\vSch_2] {\vQ_1} = 
\pi_{\optAtt [(\vFour \vee \vFive) \wedge \neg \vThree] [\empno], \optAtt [\neg \vThree \wedge \vFour \wedge \neg \vFive] [\name], \optAtt [\neg \vThree \wedge \neg \vFour \wedge \vFive] [\fname], \optAtt [\neg \vThree \wedge \neg \vFour \wedge \vFive] [\lname]  } \left(\constrain [\vSch_2] {\empbio}\right)\]
where $\constrain [\vSch_2] {\empbio} = \chc [\dimMeta_2] {\pi_{\empno, \annot [\vFour] \name, \annot [\vFive] \fname, \annot [\vFive] \lname} (\empbio),\empRel} $.
}

%\pi_{\optAtt [(\vFour \vee \vFive) \wedge \fModel_2] [\empno], \optAtt [\vFour \wedge \fModel_2] [\name], \optAtt [\vFive \wedge \fModel_2] [\fname], \optAtt [\vFive \wedge \fModel_2] [\lname]  } \empbio}}.
%

%\NOTE{
%This is the unsimplified version:
%\begin{align*}
%\VVal {\vQ_5} &= 
%\pi_{\optAtt [\vFour \vee \vFive] [\empno], \optAtt [\vFour] [\name], \optAtt [\vFive] [\fname], \optAtt [\vFive] [\lname]  } \\
%&(\chc [ \fModel_2 ] {\pi_{\empno, \sex, \birthdate, \optAtt [\vFour ] [\name], \optAtt [\vFive] [\fname], \optAtt [\vFive] [\lname]} \empbio, \empRel  })
%\end{align*}
%}
Similarly, the projection in the query 
\ensuremath{\vQ_7 = \pi_{\name, \fname} (\mathit{subq}_7)}
where 
\ensuremath{
\mathit{subq}_7 = \chc [ \vFour] {\pi_\name (\vQ_6), \pi_\fname (\vQ_6)}
}
is written over 
\ensuremath{\vSch_2} and it 
%\centerline{
%\ensuremath{
%\vQ_6 =
%\pi_{\name, \fname} \mathit{subq}_6
%} 
%}}
does not repeat the presence conditions of attributes from its \ensuremath{\mathit{subq}_7}'s type.
The query
%\centerline{
\ensuremath{
\vQ_8 =
\pi_{\optAtt [\vFour ] [\name],\optAtt [\neg \vFour] [\fname]} (\mathit{subq}_7)
%\chc [ \vFour] {\pi_\name \vQ_5, \pi_\fname \vQ_5}
}
%}
makes the annotations of projected attributes \emph{explicit} with respect to both 
the variational schema \ensuremath{\vSch_2} and its subquery's type.
%\TODO {give an example, schema: R(A,B), query: $\pi_{A,B} (F<\pi_A R, \pi_B R>)$
%becomes $\pi_{A^F, B^{\neg F}} ...$}
%The variation encoded in variational queries can
%be more restrictive or more loose than v-schema variation without violating them.
Although relieving the user from explicitly repeating variation makes VRA easier to use, 
queries still have to state variation explicitly to avoid losing information when 
decoupled from the schema.
%We do this by defining a function, 
%\ensuremath {\constrain \vQ}, with type \ensuremath{ \qSet \to \vSchSet \to \qSet
%},
%that \emph{explicitly annotates a query \vQ\ given the underlying schema \vSch}.
We do this by defining the function 
\ensuremath {\constrain \vQ : \qSet \totype \vSchSet \totype \qSet
},
that \emph{explicitly annotates a query \vQ\ with the  schema \vSch}.
%Note that \ensuremath {\constrain \vQ} needs to take the underlying schema as
%an input since it is using the type system (which relies on the schema) as a helper function.
The explicitly annotating query function, 
formally defined in \figref{constrain}, 
conjoins attributes and relations
presence conditions with the corresponding annotations in the query 
and wraps subqueries in a choice when needed. 
Note that, $\vQ_8$ and $\vQ_6$ are the result of $\constrain [\vSch_2] {\vQ_7}$
and $\constrain [\vSch_2] {\vQ_1}$, respectively, after simplification~\footnote{More specifically,
they are simpilified using rules defined in \figref{var-min}}.
%Queries $\vQ_7$ and $\vQ_5$ are examples of applying the 
%explicitly annotation function to queries $\vQ_6$ and $\vQ_1$, respectively,
%after simplifying them.
%\exref{constrain} illustrates how the constrain function transforms queries
%and allows users to be more flexible with their queries. 

\input{formulas/constrainVQbySch}

\begin{theorem}
\label{thm:expl-same-type}
If the query \vQ\ has the type \annot \vType\ then its explicitly annotated counterpart has an equivalent type, that is: \\
%
\centerline{
\ensuremath{%\raggedleft
\envWithoutVctx {\vQ} {\annot \vType} \Rightarrow \envWithoutVctx {\constrain \vQ} {\annot [\VVal \dimMeta] {\VVal \vType}} \textit{ and } \annot \vType \equiv \annot [\VVal \dimMeta] {\VVal \vType}
}}
%
\end{theorem}

\begin{proof}
By structural induction. We encoded and proved this theorem in the Coq proof assistant~\cite{Khan21}.
\end{proof}

This theorem shows that the type system applies the schema to the type of a query although it does not apply it to the query. 
The \emph{type equivalence} is variational set equivalence, defined 
in \figref{vset}, for normalized variational sets of attributes.
%\footnote{We proved this theorem in the Coq proof assistant. The encoding of the theorem and the proof can be found in second author's MS thesis~\cite{FaribaThesis}.}.

%We encode and prove \thmref{expl-same-type} in the Coq proof assistant~\cite{Khan21}.
We illustrate the application of \thmref{expl-same-type} to queries
\ensuremath{\vQ_1} and \ensuremath{\vQ_6}.
%
\exref{type} explained how \ensuremath{\vQ_1}'s type is generated step-by-step.
The variation context and underlying schema are
the same and the subquery \empbio\ has the same type. 
The projected attribute set annotated with the variation context is:\\
\ensuremath{
\vAttList_2 =  \{\annot [(\vFour \vee \vFive) \wedge \neg \vThree] \empno, 
%\ensuremath{ 
\optAtt [\neg \vThree \wedge \vFour \wedge \neg \vFive] [\name], \optAtt [\neg \vThree \wedge \neg \vFour \wedge \vFive] [\fname], \optAtt [\neg \vThree \wedge \neg \vFour \wedge \vFive] [\lname]\}^{\dimMeta_2}}, which is clearly subsumed by \ensuremath{\vAttList_\empbio}, thus, 
%the type of \empbio, \vAttList, and
its intersection with \ensuremath{\vAttList_\empbio} annotated
with the presence condition of \ensuremath{\vAttList_\empbio} is itself,
hence, \ensuremath{\vAttList_{\vQ_1} \equiv \vAttList_{\vQ_6}}.
%which makes it obvious that \ensuremath{\vAttList_{\vQ_1} \equiv \vAttList_{\vQ_6}}.
%\end{example}


%
Explicitly annotating variational queries not only relieves the user from repeating the
database's variation in their queries but it is also necessary for the functions that 
take a query without taking the schema, such as the query configuration function 
which is explained in \secref{vraconf}.
This is contra to other functions that have to take both the query and the 
schema, such as the type system. 
We explain this in more details in \secref{vraconf}.
%\exref{exp-annot-nec} illustrates why a query passed to the configuration function 
%must be explicitly annotated.
%
%\begin{example}
%\label{eg:exp-annot-nec}
%Consider the variational query $\vQ_5= \vPrj [{\vAtt_1, \optAtt [\fOne \wedge \fTwo] [\vAtt_2], \optAtt [\fTwo] [\vAtt_3]}] (\vRel)$ given in
%\exref{conf-vq}. This query is not explicitly annotated since attribute $\vAtt_1$ does not
%carry its variational encoding from the database, that is, it does not have the presence
%condition $\A$. Explicitly annotating this query gives us query $\VVal {\vQ_5} =  \vPrj [{\optAtt [\A][\vAtt_1], \optAtt [\fOne \wedge \fTwo] [\vAtt_2], \optAtt [\fTwo] [\vAtt_3]}] (\vRel)$.
%Configuring $\VVal {\vQ_5}$ results in the same query as configuring $\vQ_5$ except for 
%configuration \setDef {\ }, that is, $\eeSem [\setDef \ ] {\VVal {\vQ_5}} = \pi_{\setDef {\ }} \pRel = \empRel$. The reason why $\eeSem [\setDef \ ] {\vQ_5} $ is incorrect is that $\vQ_5$ is missing
%the variation attached to attribute $\vAtt_1$ and the configuration function does not consider
%the schema of a database while configuring variational queries written over that database. 
%\end{example}


\begin{theorem}
\label{thm:expl-same-type}
If the query \vQ\ has the type \annot \vType\ then its explicitly annotated counterpart has an equivalent type, that is: \\
%
\centerline{
\ensuremath{%\raggedleft
\envWithoutVctx {\vQ} {\annot \vType} \Rightarrow \envWithoutVctx {\constrain \vQ} {\annot [\VVal \dimMeta] {\VVal \vType}} \textit{ and } \annot \vType \equiv \annot [\VVal \dimMeta] {\VVal \vType}
}}
%
\end{theorem}

\begin{proof}
By structural induction. We encoded and proved this theorem in the Coq proof assistant~\cite{Khan21}.
\end{proof}

This theorem shows that the type system applies the schema to the type of a query although it does not apply it to the query. 
The \emph{type equivalence} is variational set equivalence, defined 
in \figref{vset}, for normalized variational sets of attributes.
%\footnote{We proved this theorem in the Coq proof assistant. The encoding of the theorem and the proof can be found in second author's MS thesis~\cite{FaribaThesis}.}.

%We encode and prove \thmref{expl-same-type} in the Coq proof assistant~\cite{Khan21}.
We illustrate the application of \thmref{expl-same-type} to queries
\ensuremath{\vQ_1} and \ensuremath{\vQ_6}.
%
\exref{type} explained how \ensuremath{\vQ_1}'s type is generated step-by-step.
The variation context and underlying schema are
the same and the subquery \empbio\ has the same type. 
The projected attribute set annotated with the variation context is:\\
\ensuremath{
\vAttList_2 =  \{\annot [(\vFour \vee \vFive) \wedge \neg \vThree] \empno, 
%\ensuremath{ 
\optAtt [\neg \vThree \wedge \vFour \wedge \neg \vFive] [\name], \optAtt [\neg \vThree \wedge \neg \vFour \wedge \vFive] [\fname], \optAtt [\neg \vThree \wedge \neg \vFour \wedge \vFive] [\lname]\}^{\dimMeta_2}}, which is clearly subsumed by \ensuremath{\vAttList_\empbio}, thus, 
%the type of \empbio, \vAttList, and
its intersection with \ensuremath{\vAttList_\empbio} annotated
with the presence condition of \ensuremath{\vAttList_\empbio} is itself,
hence, \ensuremath{\vAttList_{\vQ_1} \equiv \vAttList_{\vQ_6}}.
%which makes it obvious that \ensuremath{\vAttList_{\vQ_1} \equiv \vAttList_{\vQ_6}}.
%\end{example}


%
Explicitly annotating variational queries not only relieves the user from repeating the
database's variation in their queries but it is also necessary for the functions that 
take a query without taking the schema, such as the query configuration function 
which is explained in \secref{vraconf}.
This is contra to other functions that have to take both the query and the 
schema, such as the type system. 
We explain this in more details in \secref{vraconf}.
%\exref{exp-annot-nec} illustrates why a query passed to the configuration function 
%must be explicitly annotated.
%
%\begin{example}
%\label{eg:exp-annot-nec}
%Consider the variational query $\vQ_5= \vPrj [{\vAtt_1, \optAtt [\fOne \wedge \fTwo] [\vAtt_2], \optAtt [\fTwo] [\vAtt_3]}] (\vRel)$ given in
%\exref{conf-vq}. This query is not explicitly annotated since attribute $\vAtt_1$ does not
%carry its variational encoding from the database, that is, it does not have the presence
%condition $\A$. Explicitly annotating this query gives us query $\VVal {\vQ_5} =  \vPrj [{\optAtt [\A][\vAtt_1], \optAtt [\fOne \wedge \fTwo] [\vAtt_2], \optAtt [\fTwo] [\vAtt_3]}] (\vRel)$.
%Configuring $\VVal {\vQ_5}$ results in the same query as configuring $\vQ_5$ except for 
%configuration \setDef {\ }, that is, $\eeSem [\setDef \ ] {\VVal {\vQ_5}} = \pi_{\setDef {\ }} \pRel = \empRel$. The reason why $\eeSem [\setDef \ ] {\vQ_5} $ is incorrect is that $\vQ_5$ is missing
%the variation attached to attribute $\vAtt_1$ and the configuration function does not consider
%the schema of a database while configuring variational queries written over that database. 
%\end{example}


\begin{theorem}
\label{thm:expl-same-type}
If the query \vQ\ has the type \annot \vType\ then its explicitly annotated counterpart has an equivalent type, that is: \\
%
\centerline{
\ensuremath{%\raggedleft
\envWithoutVctx {\vQ} {\annot \vType} \Rightarrow \envWithoutVctx {\constrain \vQ} {\annot [\VVal \dimMeta] {\VVal \vType}} \textit{ and } \annot \vType \equiv \annot [\VVal \dimMeta] {\VVal \vType}
}}
%
\end{theorem}

\begin{proof}
By structural induction. We encoded and proved this theorem in the Coq proof assistant~\cite{Khan21}.
\end{proof}

This theorem shows that the type system applies the schema to the type of a query although it does not apply it to the query. 
The \emph{type equivalence} is variational set equivalence, defined 
in \figref{vset}, for normalized variational sets of attributes.
%\footnote{We proved this theorem in the Coq proof assistant. The encoding of the theorem and the proof can be found in second author's MS thesis~\cite{FaribaThesis}.}.

%We encode and prove \thmref{expl-same-type} in the Coq proof assistant~\cite{Khan21}.
We illustrate the application of \thmref{expl-same-type} to queries
\ensuremath{\vQ_1} and \ensuremath{\vQ_6}.
%
\exref{type} explained how \ensuremath{\vQ_1}'s type is generated step-by-step.
The variation context and underlying schema are
the same and the subquery \empbio\ has the same type. 
The projected attribute set annotated with the variation context is:\\
\ensuremath{
\vAttList_2 =  \{\annot [(\vFour \vee \vFive) \wedge \neg \vThree] \empno, 
%\ensuremath{ 
\optAtt [\neg \vThree \wedge \vFour \wedge \neg \vFive] [\name], \optAtt [\neg \vThree \wedge \neg \vFour \wedge \vFive] [\fname], \optAtt [\neg \vThree \wedge \neg \vFour \wedge \vFive] [\lname]\}^{\dimMeta_2}}, which is clearly subsumed by \ensuremath{\vAttList_\empbio}, thus, 
%the type of \empbio, \vAttList, and
its intersection with \ensuremath{\vAttList_\empbio} annotated
with the presence condition of \ensuremath{\vAttList_\empbio} is itself,
hence, \ensuremath{\vAttList_{\vQ_1} \equiv \vAttList_{\vQ_6}}.
%which makes it obvious that \ensuremath{\vAttList_{\vQ_1} \equiv \vAttList_{\vQ_6}}.
%\end{example}


%
Explicitly annotating variational queries not only relieves the user from repeating the
database's variation in their queries but it is also necessary for the functions that 
take a query without taking the schema, such as the query configuration function 
which is explained in \secref{vraconf}.
This is contra to other functions that have to take both the query and the 
schema, such as the type system. 
We explain this in more details in \secref{vraconf}.
%\exref{exp-annot-nec} illustrates why a query passed to the configuration function 
%must be explicitly annotated.
%
%\begin{example}
%\label{eg:exp-annot-nec}
%Consider the variational query $\vQ_5= \vPrj [{\vAtt_1, \optAtt [\fOne \wedge \fTwo] [\vAtt_2], \optAtt [\fTwo] [\vAtt_3]}] (\vRel)$ given in
%\exref{conf-vq}. This query is not explicitly annotated since attribute $\vAtt_1$ does not
%carry its variational encoding from the database, that is, it does not have the presence
%condition $\A$. Explicitly annotating this query gives us query $\VVal {\vQ_5} =  \vPrj [{\optAtt [\A][\vAtt_1], \optAtt [\fOne \wedge \fTwo] [\vAtt_2], \optAtt [\fTwo] [\vAtt_3]}] (\vRel)$.
%Configuring $\VVal {\vQ_5}$ results in the same query as configuring $\vQ_5$ except for 
%configuration \setDef {\ }, that is, $\eeSem [\setDef \ ] {\VVal {\vQ_5}} = \pi_{\setDef {\ }} \pRel = \empRel$. The reason why $\eeSem [\setDef \ ] {\vQ_5} $ is incorrect is that $\vQ_5$ is missing
%the variation attached to attribute $\vAtt_1$ and the configuration function does not consider
%the schema of a database while configuring variational queries written over that database. 
%\end{example}

\section{Variation-Minimization Rules}
\label{sec:var-min}

%
%\maybeAdd{add $\vQ_6$ is simplified of  $\VVal \vQ_6$ because of rule application blah blah.}
%\maybeAdd{add example + more rules + point out interesting ones}
%
VRA is flexible since an information need can be represented via multiple
variational queries as demonstrated in \exref{vq-specific} and \exref{vq-same-intent-mult-vars}.
It allows users to incorporate their personal taste and task requirements
into variational queries they write by 
having different levels of variation. For example, consider the explicitly annotated query
\ensuremath{\vQ_6} 
in \secref{constrain}.
%\ensuremath {
\[
\vQ_6 =
\pi_{\optAtt [\vFour \vee \vFive] [\empno], \optAtt [\vFour] [\name], \optAtt [\vFive] [\fname], \optAtt [\vFive] [\lname]  } \left( \chc [\fModel_2] {\empbio, \empRel}\right)
\]
%}.
%\vQ_5 =  \pi_{\optAtt [\vFour \vee \vFive] [\empno], \optAtt [\vFour] [\name], \optAtt [\vFive] [\fname], \optAtt [\vFive] [\lname]  } \empbio}.
%from \exref{vq-specific}. 
To be explicit about the exact query that will be run for 
each variant 
%and knowing that 
%\ensuremath{
%\getPC \empbio = \vThree \vee \vFour \vee \vFive
%},
the query $\vQ_6$'s variation can be \emph{lifted up} by using choices, resulting in the query $\VVVal \vQ_6$.
%\ensuremath{
%\small
\[
\VVVal \vQ_6 = \chc [\vFour] {\pi_{\empno, \name} \empbio, 
\chc [\vFive] {\pi_{\empno, \fname, \lname} \empbio, \emp}} 
\]
%}.
While \ensuremath{\vQ_6} contains less redundancy \ensuremath{\VVVal \vQ_6}
is more comprehensible since the variants are explicitly stated in the dimension of the choice. 
Thus, \emph{supporting multiple levels of variation 
creates a tension between reducing redundancy and maintaining comprehensibility.}

We define \emph{variation minimization} rules in \figref{var-min} that are syntactic and 
preserve the semantics.
% and include 
%interesting ones in \secref{var-min}.
Pushing in variation into a query, i.e., applying rules left-to-right, 
reduces redundancy
% and improves performance
while lifting them up, i.e., applying rules right-to-left, 
makes a query more understandable. 
When applied left-to-right, the rules are terminating since the scope of variation 
%always gets smaller.
monotonically decreases in size.
%
%\revised{
%Additionally, these rules can be used to simplify queries after
%explicitly annotating them with a schema. For example, the first rule in \figref{var-min}
%is used to simplify the query \ensuremath{\constrain [\vSch_2] {\vQ_1}}, introduced in \secref{var-pres},
% which resulted
%in \ensuremath{\vQ_6}.}


\begin{figure}
\textbf{Choice Distributive Rules:}
\begin{alignat*}{1}
\small
%-- f<? l? q?, ? l? q?> ? ? (f<l?, l?>) f<q?, q?>
%\inferrule
%{}
%\chc {\pi_{\vAttList_1} \vQ_1, \pi_{\vAttList_2} \vQ_2 } 
%&\equiv
%\pi_{\chc {\vAttList_1, \vAttList_2}} \chc {\vQ_1, \vQ_2}\\
%-- f<? l? q?, ? l? q?> ? ? ((l??), (l? \^�f )) f<q?, q?>
%\inferrule
%{}
\chc {\pi_{\vAttList_1} \vQ_1, \pi_{\vAttList_2} \vQ_2}
&\equiv
\pi_{\annot \vAttList_1, \annot [\neg \dimMeta] \vAttList_2} \chc {\vQ_1, \vQ_2}\\
%-- f<? c? q?, ? c? q?> ? ? f<c?, c?> f<q?, q?>
%\inferrule
%{}
\chc {\sigma_{\vCond_1} \vQ_1, \sigma_{\vCond_2} \vQ_2} 
&\equiv
\sigma_{\chc {\vCond_1, \vCond_2}} \chc {\vQ_1, \vQ_2}\\
%-- f<q? � q?, q? � q?> ? f<q?, q?> � f<q?, q?>
%\inferrule
%{}
\chc {\vQ_1 \times \vQ_2, \vQ_3 \times \vQ_4}
&\equiv
\chc {\vQ_1, \vQ_3} \times \chc {\vQ_2, \vQ_4}\\
%-- f<q? ?\_c? q?, q? ?\_c? q?> ? f<q?, q?> ?\_(f<c?, c?>) f<q?, q?>
%\inferrule
%{}
\chc {\vQ_1 \Join_{\vCond_1} \vQ_2, \vQ_3 \Join_{\vCond_2} \vQ_4}
&\equiv
\chc {\vQ_1, \vQ_3} \Join_{\chc {\vCond_1, \vCond_2}} \chc {\vQ_2, \vQ_4}\\
%-- f<q? ? q?, q? ? q?> ? f<q?, q?> ? f<q?, q?>
%\inferrule
%{}
\chc {\vQ_1 \circ \vQ_2, \vQ_3 \circ \vQ_4}
&\equiv
\chc {\vQ_1, \vQ_3} \circ \chc {\vQ_2, \vQ_4}
%-- f<q? ? q?, q? ? q?> ? f<q?, q?> ? f<q?, q?>
%\inferrule
%{}
%{-}
\end{alignat*}

\medskip
\textbf{CC and RA Optimization Rules Combined:}
\begin{alignat*}{1}
\small
%-- f<? (c? ? c?) q?, ? (c? ? c?) q?> ? ? (c? ? f<c?, c?>) f<q?, q?>
%\inferrule
%{}
\chc {\sigma_{\vCond_1 \wedge \vCond_2} \vQ_1, \sigma_{\vCond_1 \wedge \vCond_3} \vQ_2}
&\equiv
\sigma_{\vCond_1 \wedge \chc {\vCond_2, \vCond_3}} \chc {\vQ_1, \vQ_2}\\
%-- ? c? (f<? c? q?, ? c? q?>) ? ? (c? ? f<c?, c?>) f<q?, q?>
%\inferrule
%{}
\sigma_{\vCond_1} \chc {\sigma_{\vCond_2} \vQ_1, \sigma_{\vCond_3} \vQ_2}
&\equiv
\sigma_{\vCond_1 \wedge \chc {\vCond_2, \vCond_3}} \chc {\vQ_1, \vQ_2}\\
%-- f<q? ?\_(c? ? c?) q?, q? ?\_(c? ? c?) q?> ? ? (f<c?, c?>) (f<q?, q?> ?\_c? f<q?, q?>)
%\inferrule
%{}
\chc {\vQ_1 \Join_{\vCond_1 \wedge \vCond_2} \vQ_2, \vQ_3 \Join_{\vCond_1 \wedge \vCond_3} \vQ_4}
&\equiv
\sigma_{\chc {\vCond_2, \vCond_3}} \left( \chc {\vQ_1, \vQ_3} \Join_{\vCond_1} \chc {\vQ_2, \vQ_4} \right)
\end{alignat*}

\caption{Some of variation minimization rules.}
\label{fig:var-min}
\end{figure}

\section{Variational Relational Algebra Properties}
\label{sec:vqlprop}

\eric{pls read the following paragraph}
In this section, we discuss important properties of VRA. We first discuss its expressiveness
with regards to the relational algebra in \secref{express}.
Then, we discuss VRA's type safety in \secref{var-pres} by taking advantage of the
relational algebra's type safety and defining a property that connect VRA's type 
system to relational algebra's type system, called \emph{variation-preserving property}.

\subsection{Expressiveness}
\label{sec:express}

VRA enables querying multiple database variants encoded as a singled VDB
simultaneously and selectively.
%, satisfying the query need \nOne\ stated in \secref{mot}.
%(\textbf{N1}).
%
More precisely, VRA is \emph{maximally expressive} in the sense that it can
express any set of plain RA queries over any subset of relational database
variants encoded as a VDB. 
We prove this claim in \thmref{max-expr}.
%This claim is captured by the following theorem.

\begin{theorem}
\label{thm:max-expr}
Given a set of plain RA queries $\pQ_1,\ldots,\pQ_n$ where each query $\pQ_i$
is to be executed over a disjoint subset $\vdbInst_i$ of variants of the VDB
instance \vdbInst, there exists a variational query $q$ such that
$\forall \config \in \confSet.\; \odbSem{\vdbInst} = \vdbInst_i \implies \eeSem{\vQ} = \pQ_i$.
\end{theorem}

\begin{proof}
By construction. Let $f_i$ be the feature expression that uniquely
characterizes the variants in each $\vdbInst_i$.
Then 
\[
\vQ =
%\small{\vQ =
%\(
\chc[(\fName_1\wedge\neg \fName_2\wedge\ldots\wedge\neg \fName_n)]{\pQ_1,
  \chc[(\fName_2\wedge\ldots\wedge\neg \fName_n)]{\pQ_2,\ldots
    \chc[\fName_n]{\pQ_n,\emp}\ldots}}.
\]
\end{proof}

\noindent
%
The above construction relies on the fact that every RA query is a valid VRA
(sub)query in which every presence condition is \t.
%
Of course, in most realistic scenarios, we expect that variational queries can be encoded
more efficiently by sharing commonalities and embedding relevant choices and
presence conditions within the variational query.



\subsection{Type Safety}
\label{sec:var-pres}

%\arashComment{ The property of variation preservation is too short and it is not clear why it is important. I also did not find the proofs in Appendix D.}
%\resp{I established why such a property is important in the variational context where we are putting all variants together throughout the paper. here we just define the property and say that it holds at the type level. We also decided to just introduced this as a property for VLDB submission. We'll have the proofs and other properties in another paper.}
%\responded
To show that VRA is type safe we benefit from RA's type safety~\cite{RAtypeSys}
by defining the \emph{variation-preserving} property for VRA which connects VRA to RA.
%Variation-preserving property of VRA's type system and RA's type safety~\cite{RAtypeSys} 
%implies that VRA's type system is also type safe.
%Similar to other applications of variational research~\cite{CEW16ecoop,CEW14toplas},
%the type system must preserve
% the variation encoded in a variational query.
%
The 
\emph{variation-preserving property with respect to variational schema} states that
if a query \vQ\ has type \vType\ then 
configuring the type of a valid explicitly annotated query
is the same as the type of its configured
corresponding query. 
%
\thmref{var-pres} proves this property.

%, 
%i.e., no matter which path the constrained query takes in the diagram it will results
%to the same set of attributes.
%
% the code that produces the diagram
%\hspace{-2cm}
\begin{wrapfigure}{r}{0.24\textwidth}
\begin{center}
\begin{tikzcd}[column sep=2.3em]
  \constrain \vQ   \rar{\mathit{type}}  \dar[swap,dashed]{\eeSem . }
& {\annot \vType}  \dar[dashed]{\olSem . } \\
  \pQ \rar{\underline{type}}
& \pAttList
\end{tikzcd}
\end{center}
\end{wrapfigure}
%
\thmref{var-pres} is visualized in the diagram below, where 
the vertical arrows indicate corresponding configure functions,
\ensuremath{\mathit{type}} indicates VRA's type system, 
that is, \ensuremath{\mathit{type}(\vQ, \vSch) = \annot \vAttList} is 
\ensuremath{\envWithoutVctx \vQ {\envInContext [ \vctx] \vType}},
% of variational query \vQ\
%generated by VRA's type system and 
and
\ensuremath{\underline{\mathit{type}} (\pQ, \underline {\vSch})} indicates RA's type system
for the relational query \pQ\ over the relational database schema $\underline {\vSch}$,
that is, \ensuremath{\pEnv {\pQ} {\pAttList}}.
%Note that for simplicity, w
We assume that corresponding variation schema and schema is
passed to type systems.
% of relational query \pQ.
Simply put, 
the relational type of the configured variational query \vQ\ with configuration \config, 
that is, \ensuremath{\underline {\mathit{type}} (\eeSem {\vQ}, \osSem \vSch)},
must be the same as the configured variational type 
of the variational query \vQ\ with configuration \config, 
that is, \ensuremath{\olSem {\mathit{type} (\vQ, \vSch)}}.
\emph{Clearly the diagram commutes}: taking either path of 1) configuring \constrain \vQ\ first and 
then obtaining the relational type of it or 
2) obtaining the variational type of \constrain \vQ\ first and then configuring it results
in the same set of attributes. 
The variation-preserving property enforces the maintenance of variants that a tuple
belongs to through running a query at the schema level.%
%, partially satisfying second part of 
%\nTwo~
\footnote{
We define this property as a test at the semantics level and show that
%We have not proved this property at the semantics level, however, 
all our experimental
queries passed it.}.
%the test for variation-preserving property at the semantics level.}.
%configuring a variational query \vQ\ for configuration \config\ first and then 
%if we configure variational query \vQ\ for a given configuration \config\ its type (a set of attributes)
%must be the same as if we generate the variational attribute set for
%\vQ\ by VRA's type system and then configure it with \config,
%
%\appref{type-sys-prop-proof} sketches the proof of 
%VRA's type system being variation-preserving.
\exref{var-pres} illustrates why the query must be constrained by the variation schema
in the variation-preserving diagram.

\begin{theorem}
\label{thm:var-pres}
For all configurations \config, if a query \vQ\ has type \annot \vType\ 
then its configured query \ensuremath{\eeSem {\constrain \vQ}}
has type \ensuremath{\olSem {\annot \vType}}, i.e., \\
\centerline{
\ensuremath{
\forall \config \in \confSet. \envWithoutVctx { \vQ} {\annot \vType} \Rightarrow 
\pEnv [\osSem {\vSch}] {\eeSem {\constrain \vQ}} {\olSem {\annot \vType}}
}}.
\end{theorem}

\begin{proof}
By structural induction. We proved this theorem in the Coq proof assistant~\cite{Khan21}.
\end{proof}


\thmref{var-pres} implies that for all valid configurations of a VDB, any variational
query is correlated to a relational query and since RA is type safe its queries are
type safe. Thus, variational queries are type safe. 

\begin{example}
\label{eg:var-pres}
Consider the variational query 
\ensuremath{\vQ_5 = \vPrj [{\vAtt_1, \optAtt [\fOne \wedge \fTwo] [\vAtt_2], \optAtt [\fTwo] [\vAtt_3]}] \vRel} 
given in \exref{conf-vq}. It is well-typed
and  it has the type
\ensuremath{\vAttList =
\setDef {\optAtt [\fOne] [\vAtt_1], 
\optAtt [\fOne \wedge \fTwo] [\vAtt_2], 
\optAtt [\fTwo] [\vAtt_3]}
}.
Configuring \vAttList\ for the variant that both \fOne\ and \fTwo\ are disabled
results is an empty attribute set. However, the type of its configured query
for this variant, i.e., \ensuremath{\eeSem [\setDef \ ] {\vQ_5} =  \pi_{\pAtt_1} \pRel}, is the 
attribute set \ensuremath{\setDef {\pAtt_1}}. This violates the
variation-preserving property. A similar problem happens for the variant of
\setDef {\fTwo}, i.e., \ensuremath{
\underline{\mathit{type}} \left( \eeSem [\setDef \fTwo] {\vQ_5} \right) = 
\underline{\mathit{type}} \left( \pi_{\pAtt_1, \pAtt_3} \pRel \right) = 
\setDef{\pAtt_1, \pAtt_3} \not = \setDef{\pAtt_3} 
= \olSem [\setDef \fTwo] {\vAttList}
= \olSem [\setDef \fTwo] {\mathit{type} \left( \vQ_5 \right)}
}. However, the variation-preserving property holds for the 
constrained query by variation schema, i.e., 
\ensuremath{
\constrain [\vSch_3] {\vQ_5} = 
\vPrj [{\optAtt [\fOne] [\vAtt_1], \optAtt [\fOne \wedge \fTwo] [\vAtt_2], \optAtt [\fTwo] [\vAtt_3]}] \vRel
}.
Thus, the input query to the configuration function \eeSem . \emph{must} be explicitly
annotated by the underlying variation schema for the configured query to match the underlying 
configured schema.
%We can restrict VRA's type system to enforce users to incorporate the
%variation schema into their queries, e.g., \ensuremath{\vQ_5} becomes
%\ensuremath{\VVal \vQ_5 = 
%\vPrj [{\optAtt [\fOne] [\vAtt_1], 
%\optAtt [\fOne \wedge \fTwo] [\vAtt_2], 
%\optAtt [\fTwo] [\vAtt_3]}] \vRel
%}. However, one of the purposes of our type system is to relieve the users 
%from having to encode the VDB's variability into their queries.
%% this burdens the user to know the exact variation encoded in
%%the database in addition to the original variation they want to encode in their query.
%To avoid this violation without requiring users to repeat VDB's variability in their queries,
%after type checking a query we push the variation schema onto the variational query,
%e.g., doing so for \ensuremath{\vQ_5} results in \ensuremath{\VVal \vQ_5}.
\end{example}





%if time allows have a subsection for properties of the equivalnece of dent sem and ra + accum


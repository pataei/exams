\chapter{Variational Queries}
\label{ch:vql}

\eric{pls read the following paragraph}
Now that we have introduced the variational database framework 
we need a query language to extract information from a VDB instance
and unlike relational query languages (like SQL and relational algebra)
it needs to account for the new aspect of our database: variation. 


%\point{queries need to be able to express variability encoded in vdb.}
%The variational nature of a VDB requires a query language that
%accounts for variation directly.
%To express and represent variation in queries,
%we incorporate choice calculus~\cite{Walk13thesis, EW11tosem}  into a 
%structured query language. 
We formally define 
\emph{variational relational algebra} (VRA) in \secref{vrel-alg}
as our algebraic query language.
A query written in VRA is called a \emph{variational query} (\emph{v-query});
we use query and variational query interchangeably when it is clear from context. 
Unlike relational queries that convey an intent over a single database, 
a variational query typically conveys the same intent over several 
relational database variants. However, a single variational query is also capable of capturing different 
intents over different database variants.
%Consequently, the expressiveness of variational queries may cause them to be 
%more complicated than relational queries, discussed in \secref{type-sys}. 
%Hence, 

\eric{pls read the following paragraph}
To understand the meaning of variational queries
we define the semantics of variational queries via the
semantics of relational queries in \secref{vrasem}. We define
approaches to configure a variational query to a relational query
in \secref{vraconf}. Then, we use the results of multiple relational
queries to accumulate the result of the original variational query 
in \secref{accum}.
\maybeAdd{add direct sem of VRA if time allows and equiv in vql prop.}

Due to the expressiveness of variational queries, 
we define a type system for VRA that statically checks a
variational query against the underlying variational schema in \secref{type-sys}.
%
To make variational queries more useable we relieve the user from repeating 
the variational schema's variation in their variational queries. This is achieved by 
explicitly annotating queries in \secref{constrain}.
%In \secref{constrain}, we define an operation that explicitly annotates a
%variational query with information contained in a v-schema. 
%This operation is useful to
%define the \emph{variation-preservation} property for VRA and its type system,
%which is discussed in \secref{var-pres}, and demonstrates how our framework
%satisfies the information need \nTwo.
We then define the \emph{variation-preservation} property for VRA at
the type level in \secref{var-pres}.
% which proves that our framework
%satisfies the requirement \nTwo.
%
Finally, we provide 
%we close out this section by providing 
a set of syntactic rules that are semantic-preserving 
in \secref{var-min} that enable factoring and distributing
variation points within a variational query, which enables syntactic refactorings
including maximizing sharing within a variational query.
%for reducing a query's variation.


\section{Variational Relational Algebra}
\label{sec:vrel-alg}

%
%\wrrite{you want to have an example here that doesn't need to be explicitly annotated. 
%then use the same example to illustrate the semantics via ra sem.}
%\TODO{we use ... that has these differences with blah
%%
%we introduce these differences through building up a query to extract info
%required by the variational intent. }

%\point{vra = cc + ra}
%Considering the variational nature of a VDB, to satisfy a user's information 
%need when extracting information, 
%we need a query language that not only considers the structure of 
%relational databases (such as SQL and relational algebra (RA)) but also 
%accounts for the variation encoded in the VDB. We achieve this by:
%1) picking relational algebra as our main query language and
%2) using \emph{choices}~\cite{Walk13thesis, EW11tosem} 
%and presence conditions to account for variation. 

To account for variation, VRA combines relational algebra (RA) with 
\emph{choices}~\cite{EW11tosem,HW16fosd,Walk13thesis}.
%\point{choice.}
Remember that a choice $\chc{\elem_1,\elem_2}$ consists of a feature expression \dimMeta, called
the \emph{dimension} of the choice, and 
two \emph{alternatives} $\elem_1$ and $\elem_2$. For a given configuration \config, 
the choice $\chc{\elem_1, \elem_2}$ can be replaced by $\elem_1$ if \dimMeta\
evaluates to \t\ under configuration \config, (i.e., \fSem{\dimMeta}),
or $\elem_2$ otherwise. 
% Choices allow a variational queries
% to encode variation in a structured and systematic manner. 

\begin{figure}
\begin{syntax}

\multicolumn{4}{l}{\textbf{Operators:}} \\[1ex]
\bullet
  &\eqq& \multicolumn{2}{l}{< \myOR \leq \myOR = \myOR \neq \myOR > \myOR \geq} \\
\circ
  &\eqq& \cup \myOR \cap \\[2ex]

\multicolumn{4}{l}{\textbf{Variational conditions:}} \\[1ex]
\vCond\in\vCondSet
  &\eqq&  \multicolumn{2}{l}{
          \bTag
   \myOR  \pAtt \bullet \cte
   \myOR  \pAtt \bullet \pAtt
   \myOR  \neg \vCond
   \myOR  \vCond \vee \vCond} \\
  &\myOR& \multicolumn{2}{l}{
          \vCond \wedge \vCond
   \myOR \chc{\vCond,\vCond}} \\[2ex]

\multicolumn{4}{l}{\textbf{Variational queries:}} \\[1ex]
\vQ\in\qSet
  &\eqq&  \vRel     & \textit{Relation}\\
  &\myOR& \vSel \vQ & \textit{Selection}\\
  &\myOR& \vPrj[\vAttList]{\vQ} & \textit{Projection}\\
  &\myOR& \chc{\vQ,\vQ} & \textit{Choice}\\
% &\myOR& \vQ \Join_\vCond \vQ & \textit{Variational Join}\\
  &\myOR& \vQ \times \vQ & \textit{Cartesian Product}\\
  &\myOR& \vQ \circ \vQ  & \textit{Set Operation}\\
% &\myOR& \vQ \backslash \vQ &\textit{Variational Set Difference}\\
  &\myOR& \empRel & \textit{Empty Relation}

\end{syntax}

\caption[Syntax of variational relational algebra]{Syntax of variational relational algebra.}
%\TODO{remember that
%you removed join (also removed it from query config def and constrain query 
%by schema). if you want use it just say it's a syntactic sugar.}
%$<, \leq, =, \neq, >, \geq$.
%$\circ$ denotes set operators: union and difference.}
\label{fig:v-alg-def}
\end{figure}



%\point{explain notation and VRA operations.}
The syntax of VRA is given in \figref{v-alg-def}.
%
The selection operation is similar to standard RA selection except
that the condition parameter is \emph{variational} meaning that it may contain
choices.
For example, the query 
\ensuremath{\sigma_{\chc {\vAtt_1=\vAtt_2,\vAtt_1=\vAtt_3}} (\vRel)}
selects a variational tuple \vTuple\ if it satisfies
the condition \ensuremath{\vAtt_1 = \vAtt_2} 
and  \ensuremath{\sat {\dimMeta \wedge \getPC \vTuple}}
or
if \ensuremath{\vAtt_1 = \vAtt_3} 
and \ensuremath{\sat {\neg \dimMeta \wedge \getPC \vTuple }}.
%
The projection operation is parameterized by a variational set of attributes, \vAttList. For
example,
the query $\pi_{\vAtt_1, \optAtt [\dimMeta] [\vAtt_2]} (\vRel)$
projects $\vAtt_1$ from relation \vRel\ unconditionally, and $\vAtt_2$ 
when \sat{\dimMeta}.
%
The choice operation enables combining two variational queries to be used in different
variants based on the dimension. In practice,
it is often useful to return information in some variants and nothing at all in
others. We introduce an explicit \emph{empty} query \empRel\ to facilitate
this. 
Similar to our definition of the empty query for relational algebra, for VRA we
also have: $\empRel=\vPrj[\set{}]{\vQ}$.
The empty query is used, for example, in 
\ensuremath{\vQ_2} in \exref{vq-specific}. 
%The set operations between queries are v-set operations defined in \secref{vset}.
The rest of VRA's operations are similar to RA, where all set operations
(union, intersection, and product) are changed to the corresponding
variational set operations defined in \secref{vset}.
%\secref{vlist-vset}.
%
%\remember{
%In examples, we also use a join operation with a variational condition,
%$\vQ_1\bowtie_\vCond\vQ_2$, which is syntactic sugar for
%$\sigma_\vCond(\vQ_1\times\vQ_2)$.}


Our implementation of VRA also provides mechanisms for renaming queries and
qualifying attributes with relation/sub\-query names. These features are needed
to support self joins and to project attributes with the same name in different
relations. However, for simplicity, we omit these features from the formal
definition in this thesis.


%A query can simply 
%refer to a relation, filter tuples based on a variational condition 
%(which is a relational condition with choices of two conditions), and
%project a variational list of attributes. Besides production of two queries and
%set operations, VRA allows for a choice of two variational queries. This demands an
%\emph{empty} query since an alternative of a choice can very well inquire 
%no information at all. 
%For example, the query $\chc {\vQ_1, \}$
%

%\subsubsection{Running a Variational query on a VDB Results in a Variational table}
%\label{sec:run-vq-get-vtab}
%A variational query systematically represents a set of relational query variants associated to their
%corresponding database variants. Hence, intuitively the user expects to 
%get such variation in their result as well. 

The result of a variational query is a variational table with the reserved relation name $\mathit{result}$.
%
For example, assume that variational tuples $\annot[\fOne]{(1,2)}$ and $\annot[\neg
f_3]{(3,4)}$ belong to a variational relation $\vRel(\vAtt_1,\vAtt_2)$, which is the only
relation in a VDB with the trivial feature model \t.
%
The query $\chc[f_3]{\pi_{\optAtt[f_2][\vAtt_1]}(\vRel),\empRel}$ returns a
variational table with relation schema $\annot[f_3]{\mathit{result}(\annot[f_2]{a_1})}$,
which indicates that the result is only non-empty when $f_3$ is \t\ and that the
result includes attribute $a_1$ when $f_2$ is \t. 
%\secref{type-sys} defines a
%type system that yields the relation schema for any well-formed query.
%
The content of the result relation for the example query is a single variational tuple
$\annot[f_1]{(1)}$. The tuple $\annot[\neg f_3]{(3)}$ is not included since the
projection occurs in the context of a choice in $f_3$, which is incompatible
with the presence condition of the tuple, i.e., $\unsat{f_3 \wedge\neg f_3}$.
This illustrates how choices can effectively filter the tuples in a VDB based
on the dimension.
%, satisfying the second part of \nOne.
%
% Although there is no need to update the presence condition of the returned
% tuples, yet choices can filter the returned variational tuples.
%
% Note that here the value \ensuremath{1}
% of attribute \ensuremath{\vAtt_1} is present in VDB variants where 
% \ensuremath{\sat {\A \wedge \B \wedge \C}} although the presence 
% condition of the returned variational tuple does not have to state this condition
% since 
%
% overall presence 
%condition and the presence conditions of attributes and tuples are
%restricted by the variation enforced by the query.
%
%Note that the presence condition of tuples, attributes, and the return relation
%is restricted by the variation enforced by the query. 
% correct this so that you don't conjunct the pc of relation and clarify that it's relations's pc and not the attributes. although the conjunction should be satisfiable.}
%
%
%%Hence, VRA is more expressive than RA 
%%because it can encode variational queries.
%The variational nature allows users to write interesting queries in many ways:
%1) to express their variational information need or to filter returned tuples
%they can use annotations or 
%choices, \exref{vq-specific},
%2) to express the same intent over several database variants they can 
%use choices in queries or conditions, \exref{vq-same-intent-mult-vars},
%and 
%3) they can also use choices to express different intents over database variants.
%\TODO{Eric, should we drop the last since it creates messy results and isn't really useful?}.
%%The expressiveness of VRA satisfies \textbf{N1}, this is illustrated in 
%%\exref{vq-specific} and \exref{vq-same-intent-mult-vars}.
%%Interestingly, VRA's expressiveness enables users to express 
%%their information need more specifically by stating the exact condition
%%under which an information need is inquired. \exref{vq-specific} illustrates this.
%%It also allows users to express the same intent over several database 
%%variants
% \NOTE{
% To express the variational information need or to filter returned tuples
% users can use annotations or choices. \exref{vq-specific} illustrates this.
% }
%
%The following example
\exref{vq-specific} illustrates
%, in the context of our running example, 
how
a variational query can be used to express variational information needs.

\begin{example}
\label{eg:vq-specific}
%VRA's expressiveness consequently facilitates expressing exactly the condition
%under which an information need is inquired. 
%\wrrite{build up this example. and show the result tables.}
Assume a VDB with
\ensuremath{\features = \setDef {\vThree, \vFour, \vFive}}, 
and the only variational table \empbio\ shown in \tabref{empbio-vtab}.
The VDB has the feature model $\dimMeta_2 = \oneof {\vThree, \vFour, \vFive}$
which states that the three \vThree--\vFive\ are mutually exclusive. 
Note that $\dimMeta_2$ is different from the feature model 
$\dimMeta_{\mathit{mot}}$ of the \empbio\ variational table
shown in \tabref{empbio-vtab} .
%the corresponding \empbio\ schema variants in \tabref{mot}. 
The variational schema for this VDB is:\\
%
\centerline{\ensuremath{
\vSch_2 =
\{\empbio (\empno, \sex, \birthdate,
\optAtt [\vFour] [\name], \optAtt [\vFive] [\fname],
 \optAtt [\vFive] [\lname] )\}^{\dimMeta_2}
% \\
%& \hspace{-38pt} \textit{where } \dimMeta_2 = \oneof {\vThree, \vFour, \vFive}
%{\vThree \oplus \vFour \oplus \vFive}.
%\left(\vThree \wedge \neg \vFour \wedge \neg \vFive\right)
%  \vee \left(\vFour \wedge \neg \vThree \wedge \neg \vFive\right) 
%   \vee \left(\vFive \wedge \neg \vThree \wedge \neg \vFour\right)}.
%\end{align*}
}}.
%
Now, the user wants the employee ID numbers (\empno) and their names for variants 
that enable either \vFour\ or \vFive\ but not \vThree.
%\set{\vFour} and \set{\vFive}.
We show the steps to build up multiple queries that can extract this information. 
First, to extract the required attributes we write the query $\vQ_0$ to project all the needed
attributes without considering the variational aspect of projection. \\
\centerline{\ensuremath{
\vQ_0 = \pi_{\empno, \name, \fname, \lname} (\empbio)
}}
Note that the presence condition attribute (\pcatt) does not need to be projected. In fact, 
the presence condition attribute is returned for every variational query since that is the only
way to keep track of variation at the content level. 
%
\tabref{vq0-res} shows
the result of query $\vQ_0$ over the described VDB.
%
Note that the presence condition of the result is $\getPCfrom \empbio {\vSch_2} = \oneof {\vThree, \vFour, \vFive} \wedge (\vThree \vee \vFour \vee \vFive)$ which can be simplified to
$\oneof {\vThree, \vFour, \vFive}$. We discuss how the 
presence conditions of the returned result and its attributes are generated in \secref{type-sys}.
%

\begin{table}[ht!]
%\caption[Results of some variational queries]{Results of some variational queries over the VDB instance described in \exref{vq-specific}.}
%\label{tab:vq-res}
%\centering
%\begin{subtable}[t]{\textwidth}
\centering
\caption[Result of a variational query]{Result of the v-query $\vQ_0 = \pi_{\empno, \name, \fname, \lname} (\empbio)$.}
\label{tab:vq0-res}
\footnotesize
\arrayrulecolor{blue}
%!{\color{black}\vrule}
\begin{tabular} {c !{\color{black}\vrule} l l l l : l }
 {\textcolor{blue}{$\oneof {\vThree, \vFour,\vFive}$} }& {\textcolor{blue}{\texttt{true}}}&  {\textcolor{blue}{$\vFour$}} &  {\textcolor{blue}{$\vFive$}} &  {\textcolor{blue}{$\vFive $}} & {\textcolor{blue}{\texttt{true}}}\\
\arrayrulecolor{blue}\hdashline
\multirow{2}{*}{$\mathit{result}$}  & \empno & \name & \fname & \lname & \pcatt \\
\arrayrulecolor{black}\cline{2-6}
& 12001 & Ulf Hofstetter & Ulf & Hofstetter  & $\textcolor{blue}{\vThree \vee \vFour \vee \vFive}$\\
& 12002 & Luise McFarlan & Luise & McFarlan  & $\textcolor{blue}{\vThree \vee \vFour \vee \vFive}$\\
& 12003 & Shir DuCasse & Shir & DuCasse  & $\textcolor{blue}{\vThree \vee \vFour \vee \vFive}$\\
 &80001  & Nagui Merli & Nagui & Merli & $\textcolor{blue}{ \vFour \vee \vFive}$\\
 & 80002 & Mayuko Meszaros & Mayuko & Meszaros & $\textcolor{blue}{ \vFour \vee \vFive}$\\
 & 80003 & Theirry Viele & Theirry & Viele & $\textcolor{blue}{ \vFour \vee \vFive}$\\
 & 200001  & Selwyn Koshiba & Selwyn & Koshiba & \textcolor{blue}{\vFive}\\
 & 200002  & Bedrich Markovitch & Bedrich & Markovitch & \textcolor{blue}{\vFive}\\
 & 200003  & Pascal Benzmuller & Pascal & Benzmuller  & \textcolor{blue}{\vFive}\\
 & \ldots  & \ldots & \ldots & \ldots & \textcolor{blue}{\ldots} \\
\arrayrulecolor{white}\hline
\end{tabular}
%\end{subtable}
%
%\medskip
%\medskip
%\medskip
%\begin{subtable}[t]{\textwidth}
%\centering
%\caption{Result of the variational queries $\vQ_1 = \pi_{\optAtt [\vFour \vee \vFive] [\empno], \name, \fname, \lname} (\empbio)$ and 
%$\VVal {\vQ_1} = \pi_{\optAtt [(\vFour \vee \vFive) \wedge \neg \vThree] [\empno], 
%\optAtt [\vFour \wedge \neg \vThree \wedge \neg \vFive] [\name], 
%\optAtt [\vFive \wedge \neg \vThree \wedge \neg \vFour] [\fname], 
%\optAtt [\vFive \wedge \neg \vThree \wedge \neg \vFour] [\lname]} (\empbio)
%$.}
%\label{tab:vq1-res}
%\footnotesize
%\arrayrulecolor{blue}
%%!{\color{black}\vrule}
%\begin{tabular} {c !{\color{black}\vrule} l l l l : l }
% {\textcolor{blue}{$\oneof {\vThree, \vFour,\vFive}$} }& {\textcolor{blue}{$\vFour \vee \vFive$}}&  {\textcolor{blue}{$\vFour $}} &  {\textcolor{blue}{$\vFive $}} &  {\textcolor{blue}{$\vFive$}} & {\textcolor{blue}{\texttt{true}}}\\
%\arrayrulecolor{blue}\hdashline
%\multirow{2}{*}{$\mathit{result}$}  & \empno & \name & \fname & \lname & \pcatt \\
%\arrayrulecolor{black}\cline{2-6}
%%& 12001 & & & & \textcolor{blue}{\vThree}\\
%%& 12002 & & & & \textcolor{blue}{\vThree}\\
%%& 12003 & & & & \textcolor{blue}{\vThree}\\
% &80001  & Nagui Merli & & & \textcolor{blue}{\vFour}\\
% & 80002 & Mayuko Meszaros & & & \textcolor{blue}{\vFour}\\
% & 80003 & Theirry Viele & & & \textcolor{blue}{\vFour}\\
% & 200001  & & Selwyn & Koshiba & \textcolor{blue}{\vFive}\\
% & 200002  & & Bedrich & Markovitch & \textcolor{blue}{\vFive}\\
% & 200003  & & Pascal & Benzmuller  & \textcolor{blue}{\vFive}\\
% & \ldots  & \ldots & \ldots & \ldots& \textcolor{blue}{\ldots} \\
%\arrayrulecolor{white}\hline
%\end{tabular}
%\end{subtable}
%
%\medskip
%\medskip
%\medskip
%\begin{subtable}[t]{\textwidth}
%\centering
%\caption{Result of the variational query $\vQ_2 = \chc[\neg \vThree]{\pi_{\empno,\name,\fname,\lname}(\empbio),\empRel}$.}
%\label{tab:vq2-res}
%\footnotesize
%\arrayrulecolor{blue}
%%!{\color{black}\vrule}
%\begin{tabular} {c !{\color{black}\vrule} l l l l : l }
% {\textcolor{blue}{$\oneof {\vThree, \vFour,\vFive} \wedge \neg \vThree$} }& {\textcolor{blue}{\t}}&  {\textcolor{blue}{$\vFour $}} &  {\textcolor{blue}{$\vFive $}} &  {\textcolor{blue}{$\vFive$}} & {\textcolor{blue}{\texttt{true}}}\\
%\arrayrulecolor{blue}\hdashline
%\multirow{2}{*}{$\mathit{result}$}  & \empno & \name & \fname & \lname & \pcatt \\
%\arrayrulecolor{black}\cline{2-6}
%%& 12001 & & & & \textcolor{blue}{\vThree}\\
%%& 12002 & & & & \textcolor{blue}{\vThree}\\
%%& 12003 & & & & \textcolor{blue}{\vThree}\\
% &80001  & Nagui Merli & & & \textcolor{blue}{\vFour}\\
% & 80002 & Mayuko Meszaros & & & \textcolor{blue}{\vFour}\\
% & 80003 & Theirry Viele & & & \textcolor{blue}{\vFour}\\
% & 200001  & & Selwyn & Koshiba & \textcolor{blue}{\vFive}\\
% & 200002  & & Bedrich & Markovitch & \textcolor{blue}{\vFive}\\
% & 200003  & & Pascal & Benzmuller  & \textcolor{blue}{\vFive}\\
% & \ldots  & \ldots & \ldots & \ldots& \textcolor{blue}{\ldots} \\
%\arrayrulecolor{white}\hline
%\end{tabular}
%\end{subtable}
%
\end{table}


Now we pay attention to the variational aspect of the query. Knowing that the variation encoded
in the VDB can be inferred (that is, the VDB exists if and only if exactly
 one of the features \vThree--\vFive\ is enabled, the \name\ attribute only exists for variants
that enable \vFour\ and the \fname\ and \lname\ attributes only exist for variants that
enable \vFive) and since we only want the
projected attributes for variants that enable \vFour\ or \vFive\ we can write the
query $\vQ_1$.\\
%
\centerline{\ensuremath{
\vQ_1 = \pi_{\optAtt [\vFour \vee \vFive] [\empno], \name, \fname, \lname} (\empbio)
}}
%
\tabref{vq1-res} shows the result of this query over the described VDB.
Note that the first three tuples from \tabref{vq0-res} are not returned since the query
does not project the \empno\
attribute for variants that enable \vThree\ and  attributes 
\name, \fname, and \lname\ do not exist for these variants in the VDB. 
Thus, the tuple will just be empty and so is dropped. 
%The user needs to project the \name\ attribute 
%for variant \set{\vFour}, the \fname\ and \lname\ attributes for variant
%\set{\vFive}, and \empno\ attribute for both variants.
%This can be expressed with the following variational query.
%If we did not know that the database enforces the variation encoded in itself
%we had to repeat that variation. 

\begin{table}[ht!]
%\caption[Results of some variational queries]{Results of some variational queries over the VDB instance described in \exref{vq-specific}.}
%\label{tab:vq-res}
%\centering
%\begin{subtable}[t]{\textwidth}
%\centering
%\caption{Result of the variational query $\vQ_0 = \pi_{\empno, \name, \fname, \lname} (\empbio)$.}
%\label{tab:vq0-res}
%\footnotesize
%\arrayrulecolor{blue}
%%!{\color{black}\vrule}
%\begin{tabular} {c !{\color{black}\vrule} l l l l : l }
% {\textcolor{blue}{$\oneof {\vThree, \vFour,\vFive}$} }& {\textcolor{blue}{\texttt{true}}}&  {\textcolor{blue}{$\vFour$}} &  {\textcolor{blue}{$\vFive$}} &  {\textcolor{blue}{$\vFive $}} & {\textcolor{blue}{\texttt{true}}}\\
%\arrayrulecolor{blue}\hdashline
%\multirow{2}{*}{$\mathit{result}$}  & \empno & \name & \fname & \lname & \pcatt \\
%\arrayrulecolor{black}\cline{2-6}
%& 12001 & & & & \textcolor{blue}{\vThree}\\
%& 12002 & & & & \textcolor{blue}{\vThree}\\
%& 12003 & & & & \textcolor{blue}{\vThree}\\
% &80001  & Nagui Merli & & & \textcolor{blue}{\vFour}\\
% & 80002 & Mayuko Meszaros & & & \textcolor{blue}{\vFour}\\
% & 80003 & Theirry Viele & & & \textcolor{blue}{\vFour}\\
% & 200001  & & Selwyn & Koshiba & \textcolor{blue}{\vFive}\\
% & 200002  & & Bedrich & Markovitch & \textcolor{blue}{\vFive}\\
% & 200003  & & Pascal & Benzmuller  & \textcolor{blue}{\vFive}\\
% & \ldots  & \ldots & \ldots & \ldots & \textcolor{blue}{\ldots} \\
%\arrayrulecolor{white}\hline
%\end{tabular}
%\end{subtable}
%
%\medskip
%\medskip
%\medskip
%\begin{subtable}[t]{\textwidth}
\centering
\caption[Result of a variational query]{Result of the v-queries $\vQ_1 = \pi_{\optAtt [\vFour \vee \vFive] [\empno], \name, \fname, \lname} (\empbio)$ and 
$\VVal {\vQ_1} = \pi_{\optAtt [(\vFour \vee \vFive) \wedge \neg \vThree] [\empno], 
\optAtt [\vFour \wedge \neg \vThree \wedge \neg \vFive] [\name], 
\optAtt [\vFive \wedge \neg \vThree \wedge \neg \vFour] [\fname], 
\optAtt [\vFive \wedge \neg \vThree \wedge \neg \vFour] [\lname]} (\empbio)
$.}
\label{tab:vq1-res}
\footnotesize
\arrayrulecolor{blue}
%!{\color{black}\vrule}
\begin{tabular} {c !{\color{black}\vrule} l l l l : l }
 {\textcolor{blue}{$\oneof {\vThree, \vFour,\vFive}$} }& {\textcolor{blue}{$\vFour \vee \vFive$}}&  {\textcolor{blue}{$\vFour $}} &  {\textcolor{blue}{$\vFive $}} &  {\textcolor{blue}{$\vFive$}} & {\textcolor{blue}{\texttt{true}}}\\
\arrayrulecolor{blue}\hdashline
\multirow{2}{*}{$\mathit{result}$}  & \empno & \name & \fname & \lname & \pcatt \\
\arrayrulecolor{black}\cline{2-6}
%%& 12001 & & & & \textcolor{blue}{\vThree}\\
%%& 12002 & & & & \textcolor{blue}{\vThree}\\
%%& 12003 & & & & \textcolor{blue}{\vThree}\\
& 12001 & Ulf Hofstetter & Ulf & Hofstetter  & $\textcolor{blue}{\vThree \vee \vFour \vee \vFive}$\\
& 12002 & Luise McFarlan & Luise & McFarlan  & $\textcolor{blue}{\vThree \vee \vFour \vee \vFive}$\\
& 12003 & Shir DuCasse & Shir & DuCasse  & $\textcolor{blue}{\vThree \vee \vFour \vee \vFive}$\\
 &80001  & Nagui Merli & Nagui & Merli & $\textcolor{blue}{\vFour \vee \vFive}$\\
 & 80002 & Mayuko Meszaros & Mayuko & Meszaros & $\textcolor{blue}{ \vFour \vee \vFive}$\\
 & 80003 & Theirry Viele & Theirry & Viele & $\textcolor{blue}{ \vFour \vee \vFive}$\\
 & 200001  & Selwyn Koshiba & Selwyn & Koshiba & \textcolor{blue}{\vFive}\\
 & 200002  & Bedrich Markovitch & Bedrich & Markovitch & \textcolor{blue}{\vFive}\\
 & 200003  & Pascal Benzmuller & Pascal & Benzmuller  & \textcolor{blue}{\vFive}\\
 & \ldots  & \ldots & \ldots & \ldots & \textcolor{blue}{\ldots} \\
% &80001  & Nagui Merli & & & \textcolor{blue}{\vFour}\\
% & 80002 & Mayuko Meszaros & & & \textcolor{blue}{\vFour}\\
% & 80003 & Theirry Viele & & & \textcolor{blue}{\vFour}\\
% & 200001  & & Selwyn & Koshiba & \textcolor{blue}{\vFive}\\
% & 200002  & & Bedrich & Markovitch & \textcolor{blue}{\vFive}\\
% & 200003  & & Pascal & Benzmuller  & \textcolor{blue}{\vFive}\\
% & \ldots  & \ldots & \ldots & \ldots& \textcolor{blue}{\ldots} \\
\arrayrulecolor{white}\hline
\end{tabular}
%\end{subtable}
%
%\medskip
%\medskip
%\medskip
%\begin{subtable}[t]{\textwidth}
%\centering
%\caption{Result of the variational query $\vQ_2 = \chc[\neg \vThree]{\pi_{\empno,\name,\fname,\lname}(\empbio),\empRel}$.}
%\label{tab:vq2-res}
%\footnotesize
%\arrayrulecolor{blue}
%%!{\color{black}\vrule}
%\begin{tabular} {c !{\color{black}\vrule} l l l l : l }
% {\textcolor{blue}{$\oneof {\vThree, \vFour,\vFive} \wedge \neg \vThree$} }& {\textcolor{blue}{\t}}&  {\textcolor{blue}{$\vFour $}} &  {\textcolor{blue}{$\vFive $}} &  {\textcolor{blue}{$\vFive$}} & {\textcolor{blue}{\texttt{true}}}\\
%\arrayrulecolor{blue}\hdashline
%\multirow{2}{*}{$\mathit{result}$}  & \empno & \name & \fname & \lname & \pcatt \\
%\arrayrulecolor{black}\cline{2-6}
%%& 12001 & & & & \textcolor{blue}{\vThree}\\
%%& 12002 & & & & \textcolor{blue}{\vThree}\\
%%& 12003 & & & & \textcolor{blue}{\vThree}\\
% &80001  & Nagui Merli & & & \textcolor{blue}{\vFour}\\
% & 80002 & Mayuko Meszaros & & & \textcolor{blue}{\vFour}\\
% & 80003 & Theirry Viele & & & \textcolor{blue}{\vFour}\\
% & 200001  & & Selwyn & Koshiba & \textcolor{blue}{\vFive}\\
% & 200002  & & Bedrich & Markovitch & \textcolor{blue}{\vFive}\\
% & 200003  & & Pascal & Benzmuller  & \textcolor{blue}{\vFive}\\
% & \ldots  & \ldots & \ldots & \ldots& \textcolor{blue}{\ldots} \\
%\arrayrulecolor{white}\hline
%\end{tabular}
%\end{subtable}
%
\end{table}


If desired, we can also make the inferred presence conditions explicit, as 
demonstrated in the following query $\VVal {\vQ_1}$.\\
%This is expressed in $\VVal {\vQ_1}$. 
\centerline{\ensuremath{
\VVal {\vQ_1} = 
\pi_{\optAtt [(\vFour \vee \vFive) \wedge \neg \vThree] [\empno], 
\optAtt [\vFour \wedge \neg \vThree \wedge \neg \vFive] [\name], 
\optAtt [\vFive \wedge \neg \vThree \wedge \neg \vFour] [\fname], 
\optAtt [\vFive \wedge \neg \vThree \wedge \neg \vFour] [\lname]} (\empbio)
}}
%
The result of the query $\VVal {\vQ_1}$ is still \tabref{vq1-res}.
%\eric{
%%isnt this a moot point? since the feature model has already been applied, the result is unchanged by whether the feature model restricts it or not? 
%Yeah, I guess. but that was the whole point of saying that fm has been applied.}
Note that all the variation encoded in the VDB is applied to the result of
a query. Thus,
the result of a variational query stands on its own, that is,
it is not part of a bigger structure like the variational tables in a VDB.
%Note that unlike the variational table \empbio\ shown in \tabref{empbio-vtab},
%%which is restricted not only by its presence condition but also by the feature model,
%the result of a variational query is not part of a bigger variational structure (the VDB).
%Thus, it is only restricted by its presence condition and not the feature model although
%the feature model has already been applied to its presence condition.
\end{example}


In the example, note that the user does not need to repeat the variability  encoded
in the variational schema in their query, that is, they do not need to annotate \name,
\fname, and \lname\ with \vFour, \vFive, and \vFive, respectively. We discuss
this in more detail in \secref{constrain}. $\vQ_1$
queries all three variants simultaneously although the returned results are
only associated with variants \vFour\ and \vFive\ due to the annotation of the
attribute \empno\ in the query and the presence conditions of the rest of the
projected attributes in the schema.
%
Yet, the query can be further simplified with a choice. $\vQ_2$ selects only two
out of the three variants explicitly:\\
%selecting only two out of the three variants can be written more
%explicitly in a query by using a choice:
\centerline{\ensuremath{
\vQ_2=\chc[\neg \vThree]{\pi_{\empno,\name,\fname,\lname}(\empbio),\empRel}}}. 
%
\tabref{vq2-res} shows the result of this query over the VDB described in \exref{vq-specific}.
%

\begin{table}[ht!]
%\caption[Results of some variational queries]{Results of some variational queries over the VDB instance described in \exref{vq-specific}.}
%\label{tab:vq-res}
%\centering
%\begin{subtable}[t]{\textwidth}
%\centering
%\caption{Result of the variational query $\vQ_0 = \pi_{\empno, \name, \fname, \lname} (\empbio)$.}
%\label{tab:vq0-res}
%\footnotesize
%\arrayrulecolor{blue}
%%!{\color{black}\vrule}
%\begin{tabular} {c !{\color{black}\vrule} l l l l : l }
% {\textcolor{blue}{$\oneof {\vThree, \vFour,\vFive}$} }& {\textcolor{blue}{\texttt{true}}}&  {\textcolor{blue}{$\vFour$}} &  {\textcolor{blue}{$\vFive$}} &  {\textcolor{blue}{$\vFive $}} & {\textcolor{blue}{\texttt{true}}}\\
%\arrayrulecolor{blue}\hdashline
%\multirow{2}{*}{$\mathit{result}$}  & \empno & \name & \fname & \lname & \pcatt \\
%\arrayrulecolor{black}\cline{2-6}
%& 12001 & & & & \textcolor{blue}{\vThree}\\
%& 12002 & & & & \textcolor{blue}{\vThree}\\
%& 12003 & & & & \textcolor{blue}{\vThree}\\
% &80001  & Nagui Merli & & & \textcolor{blue}{\vFour}\\
% & 80002 & Mayuko Meszaros & & & \textcolor{blue}{\vFour}\\
% & 80003 & Theirry Viele & & & \textcolor{blue}{\vFour}\\
% & 200001  & & Selwyn & Koshiba & \textcolor{blue}{\vFive}\\
% & 200002  & & Bedrich & Markovitch & \textcolor{blue}{\vFive}\\
% & 200003  & & Pascal & Benzmuller  & \textcolor{blue}{\vFive}\\
% & \ldots  & \ldots & \ldots & \ldots & \textcolor{blue}{\ldots} \\
%\arrayrulecolor{white}\hline
%\end{tabular}
%\end{subtable}
%
%\medskip
%\medskip
%\medskip
%\begin{subtable}[t]{\textwidth}
%\centering
%\caption{Result of the variational queries $\vQ_1 = \pi_{\optAtt [\vFour \vee \vFive] [\empno], \name, \fname, \lname} (\empbio)$ and 
%$\VVal {\vQ_1} = \pi_{\optAtt [(\vFour \vee \vFive) \wedge \neg \vThree] [\empno], 
%\optAtt [\vFour \wedge \neg \vThree \wedge \neg \vFive] [\name], 
%\optAtt [\vFive \wedge \neg \vThree \wedge \neg \vFour] [\fname], 
%\optAtt [\vFive \wedge \neg \vThree \wedge \neg \vFour] [\lname]} (\empbio)
%$.}
%\label{tab:vq1-res}
%\footnotesize
%\arrayrulecolor{blue}
%%!{\color{black}\vrule}
%\begin{tabular} {c !{\color{black}\vrule} l l l l : l }
% {\textcolor{blue}{$\oneof {\vThree, \vFour,\vFive}$} }& {\textcolor{blue}{$\vFour \vee \vFive$}}&  {\textcolor{blue}{$\vFour $}} &  {\textcolor{blue}{$\vFive $}} &  {\textcolor{blue}{$\vFive$}} & {\textcolor{blue}{\texttt{true}}}\\
%\arrayrulecolor{blue}\hdashline
%\multirow{2}{*}{$\mathit{result}$}  & \empno & \name & \fname & \lname & \pcatt \\
%\arrayrulecolor{black}\cline{2-6}
%%& 12001 & & & & \textcolor{blue}{\vThree}\\
%%& 12002 & & & & \textcolor{blue}{\vThree}\\
%%& 12003 & & & & \textcolor{blue}{\vThree}\\
% &80001  & Nagui Merli & & & \textcolor{blue}{\vFour}\\
% & 80002 & Mayuko Meszaros & & & \textcolor{blue}{\vFour}\\
% & 80003 & Theirry Viele & & & \textcolor{blue}{\vFour}\\
% & 200001  & & Selwyn & Koshiba & \textcolor{blue}{\vFive}\\
% & 200002  & & Bedrich & Markovitch & \textcolor{blue}{\vFive}\\
% & 200003  & & Pascal & Benzmuller  & \textcolor{blue}{\vFive}\\
% & \ldots  & \ldots & \ldots & \ldots& \textcolor{blue}{\ldots} \\
%\arrayrulecolor{white}\hline
%\end{tabular}
%\end{subtable}
%
%\medskip
%\medskip
%\medskip
%\begin{subtable}[t]{\textwidth}
\centering
\caption[Result of a variational query]{Result of the v-query $\vQ_2 = \chc[\neg \vThree]{\pi_{\empno,\name,\fname,\lname}(\empbio),\empRel}$.}
\label{tab:vq2-res}
\footnotesize
\arrayrulecolor{blue}
%!{\color{black}\vrule}
\begin{tabular} {c !{\color{black}\vrule} l l l l : l }
 {\textcolor{blue}{$\oneof {\vThree, \vFour,\vFive} \wedge \neg \vThree$} }& {\textcolor{blue}{\t}}&  {\textcolor{blue}{$\vFour $}} &  {\textcolor{blue}{$\vFive $}} &  {\textcolor{blue}{$\vFive$}} & {\textcolor{blue}{\texttt{true}}}\\
\arrayrulecolor{blue}\hdashline
\multirow{2}{*}{$\mathit{result}$}  & \empno & \name & \fname & \lname & \pcatt \\
\arrayrulecolor{black}\cline{2-6}
%%& 12001 & & & & \textcolor{blue}{\vThree}\\
%%& 12002 & & & & \textcolor{blue}{\vThree}\\
%%& 12003 & & & & \textcolor{blue}{\vThree}\\
& 12001 & Ulf Hofstetter & Ulf & Hofstetter  & $\textcolor{blue}{ \vFour \vee \vFive}$\\
& 12002 & Luise McFarlan & Luise & McFarlan  & $\textcolor{blue}{ \vFour \vee \vFive}$\\
& 12003 & Shir DuCasse & Shir & DuCasse  & $\textcolor{blue}{ \vFour \vee \vFive}$\\
 &80001  & Nagui Merli & Nagui & Merli & $\textcolor{blue}{ \vFour \vee \vFive}$\\
 & 80002 & Mayuko Meszaros & Mayuko & Meszaros & $\textcolor{blue}{ \vFour \vee \vFive}$\\
 & 80003 & Theirry Viele & Theirry & Viele & $\textcolor{blue}{ \vFour \vee \vFive}$\\
 & 200001  & Selwyn Koshiba & Selwyn & Koshiba & \textcolor{blue}{\vFive}\\
 & 200002  & Bedrich Markovitch & Bedrich & Markovitch & \textcolor{blue}{\vFive}\\
 & 200003  & Pascal Benzmuller & Pascal & Benzmuller  & \textcolor{blue}{\vFive}\\
 & \ldots  & \ldots & \ldots & \ldots & \textcolor{blue}{\ldots} \\
% &80001  & Nagui Merli & & & \textcolor{blue}{\vFour}\\
% & 80002 & Mayuko Meszaros & & & \textcolor{blue}{\vFour}\\
% & 80003 & Theirry Viele & & & \textcolor{blue}{\vFour}\\
% & 200001  & & Selwyn & Koshiba & \textcolor{blue}{\vFive}\\
% & 200002  & & Bedrich & Markovitch & \textcolor{blue}{\vFive}\\
% & 200003  & & Pascal & Benzmuller  & \textcolor{blue}{\vFive}\\
% & \ldots  & \ldots & \ldots & \ldots& \textcolor{blue}{\ldots} \\
\arrayrulecolor{white}\hline
\end{tabular}
%\end{subtable}
%
\end{table}



Note that, as shown in \tabref{vq1-res} and \tabref{vq2-res}, 
queries $\vQ_1$ and $\vQ_2$ return the same set of variational tuples.
However, the first three tuples in \tabref{vq1-res} could belong to a variant that 
enables any of \vThree--\vFive\ whereas the first three tuples in \tabref{vq2-res}
could only belong to variants that either enable \vFour\ or \vFive. 
This difference is due to the difference in their tables' presence conditions, 
that is, $\vQ_2$ filters out tuples that belong to variant \vThree\ at the schema 
level while $\vQ_1$ does not. We discuss this more in \exref{type}. 
More importantly, even though the first three tuples in \tabref{vq1-res} could 
belong to a variant that enables \vThree, configuring \tabref{vq1-res}
for such a variant drops the first three tuples since all their attributes would 
be \nul. We illustrate how configuring \tabref{vq1-res} for variant \setDef \vThree\
drops the first three tuples in \exref{conf-vq}.
% since
%neither returns tuples associated with variant \vThree, but their returned
%variational tables have different presence conditions, thus, $\vQ_2$ filters out
%tuples that belong to variant \vThree\ at the schema level while $\vQ_1$ does not. We discuss this
%more in \exref{type}. 
%

%\NOTE{
%\revised{VRA has \revised{syntactic} equivalence rules, described in
%\secref{var-min}, that enable semantics-preserving transformations of queries
%similar to the transformation of $\vQ_1$ into $\vQ_2$ (and vice versa). These
%rules enable factoring commonality out of subqueries, among other
%transformations.}

%The next example 
 Expressing
the same intent over several database variants by a single query relieves the DBA from
maintaining separate queries for different variants or configurations of the
schema.
\exref{vq-same-intent-mult-vars} 
illustrates this point.
% by using choices.
%how a variational query can be used to express the same
%intent over several database variants using choices and conditions.

\begin{example}
\label{eg:vq-same-intent-mult-vars}
Assume a VDB with  \ensuremath{\features = \setDef{\vOne, \ldots, \vFive}}
and the corresponding \basic\ schema
variants in \tabref{mot}. The user wants to get all employee names across all
variants. They express this intent by the query $\vQ_3$:
%
\begin{align*}
\vQ_3 &= 
  \vOne\chcL
    (\pi_{\name}(\engemp)) \cup (\pi_{\name}(\othemp)) \\
 & \hspace{32pt},
    (\vTwo\vee\vThree)\chcL
      \pi_{\name}(\empacct) \\
 & \hspace{88pt},
      \chc[(\vFour\vee\vFive)]{\pi_{\name,\fname,\lname}\empbio, \emp}\chcR\chcR
\end{align*}
%
Since the variational schema enforces that exactly one of \vOne--\ \vFive\ be enabled, we
can simplify the query by omitting the final choice.
%
\begin{align*}
\vQ_4 &= 
  \vOne\chcL
    (\pi_{\name}(\engemp)) \cup (\pi_{\name}(\othemp)) \\
 & \hspace{32pt},
    \chc[(\vTwo\vee\vThree)]{
      \pi_{\name}(\empacct),
      \pi_{\name,\fname,\lname}(\empbio)}
\end{align*}
%
\end{example}

In principle, variational queries can also express arbitrarily different intents over
different database variants. However, we expect that variational queries are best used to
capture single (or at least related) intents that vary in their realization
since this is easier to understand and increases the potential for sharing in
both the representation and execution of a variational query.







%\subsection{VRA Type System}
\label{sec:typesys}

\TODO{type sys}


\section{VRA Semantics }
\label{sec:vrasem}

%\TODO{vra semantics. we understand it through RA sem + accumulation}

We use the semantics of relational queries to define the semantics of 
variational queries. We first define the configuration function
for variational queries which takes a configuration and a variational query
and returns a relational query, \secref{vraconf}. We also define another version of the
variational query configuration function that generates unique relational
query variants, \secref{vraconf}. Then, we define an accumulation function that accumulates
multiple (annotated) relational tables into a variational table, \secref{accum}. Finally, we  
define the denotational semantics of VRA using the defined configuration and
accumulation functions, \secref{vradensem}.
%
%\maybeAdd{if have time add VRA sem + equiv}
%dentoational semantics of VRA
%equivalence of dent sem and config and accumulation  --> in properties section



\subsection{VRA Configuration}
\label{sec:vraconf}

\begin{figure}
%\textbf{Configuration selection semantics of \vqsTxt:}
\begin{alignat*}{1}
\eeSem [] . &: \qSet \to \confSet \to \pQSet\\
%
\eeSem \vRel &= \orSem \vRel = \pRel\\
\eeSem {\vSel \vQ}  &= \vSel [\ecSem \vCond] {\eeSem \vQ}\\
%
\eeSem {\vPrj [\vAttList] \vQ} &= \vPrj [\olSem \vAttList] {\eeSem \vQ}\\
%
\eeSem {{\vQ_1} \times {\vQ_2}} &= \eeSem {\vQ_1} \times \eeSem {\vQ_2}\\
%
%\eeSem {{\vQ_1} \Join_\vCond {\vQ_2}} &= \eeSem {\vQ_1} \Join_{\ecSem \vCond} \eeSem {\vQ_2}\\
%
\eeSem {\chc {\vQ_1, \vQ_2}} &= 
	\begin{cases}
		\eeSem {\vQ_1}, \text{ if } \fSem \dimMeta = \t\\
		\eeSem {\vQ_2}, \text{ otherwise}
	\end{cases}\\
%
\eeSem {{\vQ_1} \circ {\vQ_2}} &= \eeSem {\vQ_1} \circ \eeSem {\vQ_2}\\
%
\eeSem {\empRel} &= \underline {\empRel}
\end{alignat*}
\caption{Configuration of VRA which assumes that the given v-query
is well-typed. 
%\orSem ., \ecSem ., and \olSem . are
%configuration of v-relation, v-condition, and variational attribute
%set, respectively, defined in \figref{vdb-conf}, 
%\figref{vcond-conf-sem}, \figref{vdb-conf}.
Note that we have extended RA with an empty relation $\underline {\empRel}$.}
\label{fig:v-alg-conf-sem}
\end{figure}


%\NOTE{
%Also, the following definition of the semantics contradicts with the
%description earlier in the section about producing a \emph{result}
%relation.
%
%\medskip
%Also also, maybe we should move the discussion of the semantics before the
%examples? It's a bit surprising to come across it here.}

%The semantics of VRA can be understood as a combination of the
%\emph{configuration semantics} of VRA, defined in \figref{v-alg-conf-sem}, the
%configuration semantics of VDBs, defined in \figref{vdb-conf}, and the
%semantics of plain RA.
%%
%%\TODO{Make the following a more precise description of how these three
%%semantics work together, i.e.\ for every valid configuration of the feature
%%model, we can configure the variational query and VDB in the same way to yield a plain RA
%%query that is then executed over the corresponding plain RDB.}
%%
%Thus, the variational query
%semantics is the set of semantics of its configured relational queries over
%their corresponding configured relational database variant for every valid
%configuration of the feature model of the VDB.
%
%We now embark on the formal definition of variational queries configuration.
The \emph{configuration} function maps a variational query under
a configuration
to a relational query, defined in \figref{v-alg-conf-sem}. Thus, a variational query 
can be understood as a set of relational queries, the results of which are gathered
in a single table and tagged with the feature expression stating their variants.
%Configuring a variational query
%for all valid configurations, accessible from VDB's feature model,
%provides the complete meaning of a variational query in terms of RA semantics.
%
Users can deploy queries for a specific variant by configuring 
the variational query.
%
%The configuration of a query allows users to deploy queries for a
%specific variant when they desire, 
%satisfying query part of \nThree\ requirement. 
\exref{conf-vq} illustrates configuring a query
and \exref{vq-sem} illustrates the configuration of query $\VVal {\vQ_1}$ from \exref{vq-specific} and the corresponding relational results table.

%\begin{figure}
%\textbf{Configuration selection semantics of variational conditions:}
\begin{alignat*}{1}
\ecSem [] . &: \vCondSet \to \confSet \to \pCondSet\\
%
\ecSem \bTag &= \bTag \\
%
\ecSem \vAttOpCte &= 
    \vAttOpCte\\
%	\begin{cases}
%		\vAttOpCte, &\text{ if } \pAtt \in \attr [\eeSem \vRel]\\
%		\f, &\text{ otherwise}
%	\end{cases}\\
%
\ecSem \vAttOpAtt &= 
       \vAttOpAtt\\
%	\begin{cases}
%		\pAttOpAtt, &\text{ if } \pAtt_1 \in \attr [\eeSem \vRel] \&\ 
%		                                   \pAtt_2 \in \attr [\eeSem \vRel] \\
%		\f,  &\text{ otherwise}
%	\end{cases}\\
%
\ecSem {\neg \vCond} &= \neg \ecSem \vCond\\
%
\ecSem {\orr \vCond} &= \ecSem {\vCond_1} \vee \ecSem {\vCond_2}\\
%
\ecSem {\annd \vCond} &= \ecSem {\vCond_1} \wedge \ecSem {\vCond_2}\\
%
\ecSem {\chc {\vCond_1, \vCond_2}} &=
	\begin{cases}
		\ecSem {\vCond_1}, &\text{ if } \fSem \dimMeta = \t \\
		\ecSem {\vCond_2}, &\text{ otherwise}
	\end{cases}
\end{alignat*}
\caption{V-condition configuration.
%which assumes v-conditions
%are well-typed.
}
\label{fig:vcond-conf-sem}
\end{figure}


%To define VRA semantics we map 
%a variational query to a pure relational query to re-use RA's semantics.
%However, to avoid losing the variation encoded 
%in the variational query, 
%we need to determine the variant under which such a
%mapping is valid. Thus, we introduce the semantic functions that 
%relate a variational query to a relational query.

%
%\textbf{Configuring a variational query:} 
%It maps a variational query under a 
%given configuration to a relational query, denoted by \eeSem . 
%and defined in \figref{v-alg-conf-sem}. Configuring a variational query
%for all valid configurations, accessible from VDB's feature model,
%provides the complete meaning of a variational query in terms of RA semantics.
%Users can deploy queries for a specific variant by configuring 
%them,
%%The configuration of a query allows users to deploy queries for a
%%specific variant when they desire, 
%satisfying query part of \nThree.

\begin{example}
\label{eg:conf-vq}
Assume the underlying VDB has the variational schema
% \t\ feature model and the variational relation
\ensuremath{
\vSch_3 = \{ \vRel \left( \optAtt [\fOne] [\vAtt_1], \vAtt_2, \vAtt_3 \right)^{\fOne \vee \fTwo}
\}} 
and the feature space 
\ensuremath{
\features = \setDef{ \fOne, \fTwo}}.
For valid configurations of this VDB (that is, \setDef {\ }, \setDef \A, \setDef \B, and \setDef {\A, \B}), 
the variational query 
\ensuremath{
\vQ_5 = \vPrj [{\vAtt_1, \optAtt [\fOne \wedge \fTwo] [\vAtt_2], \optAtt [\fTwo] [\vAtt_3]}] (\vRel)
}
is configured to the following relational queries:
\begin{alignat*}{1}
\eeSem [\setDef \ ] {\vQ_5} &= \pi_{\pAtt_1} \pRel\\
\eeSem [\setDef \fOne] {\vQ_5} &=  \pi_{\pAtt_1} \pRel\\
\eeSem [\setDef \fTwo] {\vQ_5} &= \pi_{\pAtt_1, \pAtt_3} \pRel\\
\eeSem [\setDef {\fOne, \fTwo}] {\vQ_5} &= \pi_{\pAtt_1, \pAtt_2, \pAtt_3} \pRel
\end{alignat*}
\end{example}





%\textbf{Grouping a variational query:} 
%maps a variational query to a set of
%relational queries annotated with feature expressions, denoted by \qGroup .
%and defined in \figref{vq-group}. The presence condition of relational queries 
%indicate the group of configurations where the mapping holds. In essence, 
%grouping of variational query \vQ\ groups together all configurations with the same relational
%query produced from configuring \vQ. 
%Hence, the generated set
%%\dropit{could drop this if it's confusing!}
%of relational queries from grouping a variational query contains distinct (unique) queries.
%For example, consider the query \ensuremath {\vQ_5} in \exref{conf-vq}.
%Grouping \ensuremath{\vQ_5} results in the set:
%\ensuremath{
%\setDef{
%\left( \pi_{\pAtt_1, \pAtt_2, \pAtt_3} \pRel \right)^{\fOne \wedge \fTwo},
%\left(\pi_{\pAtt_1, \pAtt_3} \pRel \right)^{\neg \fOne \wedge \fTwo},
%\left(  \pi_{\pAtt_1} \pRel \right)^{( \fOne \wedge \neg \fTwo) \vee (\neg \fOne \wedge \neg \fTwo)}
%}
%}.
%
%
%

\begin{example}
\label{eg:vq-sem}
Consider the query $\VVal {\vQ_1}$ given in \exref{vq-specific}: \\
\centerline{
$\VVal {\vQ_1} = 
\pi_{\optAtt [(\vFour \vee \vFive) \wedge \neg \vThree] [\empno], 
\optAtt [\vFour \wedge \neg \vThree \wedge \neg \vFive] [\name], 
\optAtt [\vFive \wedge \neg \vThree \wedge \neg \vFour] [\fname], 
\optAtt [\vFive \wedge \neg \vThree \wedge \neg \vFour] [\lname]} (\empbio)
$.}  
Configuring $\VVal {\vQ_1}$ for all valid configurations 
(\setDef \vThree, \setDef \vFour, \setDef \vFive) of the given VDB
results in three relational queries:
%
\begin{alignat*}{1}
%
\eeSem [\setDef {\vThree}] {\VVal {\vQ_1}} &= \empRel\\
%
\eeSem [\setDef {\vFour}] {\VVal {\vQ_1}} &= \pi_{\empno, \name} (\empbio)\\
%
\eeSem [\setDef {\vFive}] {\VVal {\vQ_1}} &= \pi_{\empno, \fname, \lname} (\empbio)
%
%\eeSem [\setDef {\ }] {\VVal {\vQ_1}} &= \empRel 
\end{alignat*}
%
\noindent
\tabref{vq-conf-res} shows the result of these relational queries.
\end{example}

\begin{table}[!htbp]
\caption[Results of relational queries from configuring a variational query]{Results of relational queries from configuring the variational query $\VVal {\vQ_1}$.}
\label{tab:vq-conf-res}
\centering
\small
%\footnotesize
%\scriptsize
\begin{subtable}[t]{\textwidth}
\centering
\caption{Result of the query $\eeSem [\setDef {\vThree}] {\VVal {\vQ_1}} = \empRel$.}
\label{tab:vq-conf1}
\arrayrulecolor{black}
\begin{tabular} {c | l }
\multirow{2}{*}{$\mathit{result}$} & \textcolor{white}{blah blah}\\
\cline{2-2}
&  \\
\arrayrulecolor{white}\hline
\end{tabular}
\end{subtable}

%\medskip
%\medskip
\medskip
\begin{subtable}[t]{\textwidth}
%\begin{center}
\centering
%\tiny
\caption{Result of the query $\eeSem [\setDef {\vFour}] {\VVal {\vQ_1}} = \pi_{\empno, \name} (\empbio)$.}
\label{tab:vq-conf2}
\arrayrulecolor{black}
\begin{tabular} {c | l l }
%\hline
%\hhline{-==}
\multirow{2}{*}{$\mathit{result}$}  & \empno & \name\\
\cline{2-3}
 &80001 & Nagui Merli\\
 & 80002 & Mayuko Meszaros\\
 & 80003 & Theirry Viele\\
&\ldots & \ldots \\
\arrayrulecolor{white}\hline
\end{tabular}
%\end{center}
\end{subtable}

%\medskip
%\medskip
\medskip
\begin{subtable}[t]{\textwidth}
%\begin{center}
\centering
%\footnotesize
%\tiny
\caption{Result of the query \ensuremath{\eeSem [\setDef {\vFive}] {\VVal {\vQ_1}} = \pi_{\empno, \fname, \lname} (\empbio)}.}
\label{tab:vq-conf3}
\arrayrulecolor{black}
\begin{tabular} {c | l l l}
%\hline
%\hhline{-==}
\multirow{2}{*}{$\mathit{result}$}  & \empno &\fname &\lname\\
\cline{2-4}
 & 200001 & Selwyn & Koshiba \\
 & 200002 & Bedrich & Markovitch \\
 & 200003 & Pascal & Benzmuller  \\
 & \ldots & \ldots & \ldots \\
 \arrayrulecolor{white}\hline
\end{tabular}
%\end{center}
\end{subtable}

\end{table}




Often a variational query will yield the same plain query for multiple configurations.
For our semantics, it is useful to get the set of unique variants of a variational query.
%Unfortunately, the configuration of variational queries may result in
%duplicate relational queries. In practice, this is not very efficient, as discussed
%later in \secref{exp}. 
Thus, we define the \emph{unique variants} (unique configuration) function, whose type is given below.
\[
\qGroup[\cdot]{\cdot} : \qSet \totype \settype \fSet \totype \settype {\bm{(} \vartype \pQSet \bm{)}}
\]
This function takes a variational query and VDB's set of features
and returns a set of configured relational queries annotated with
a presence condition. The presence condition is a feature expression generated from
the set of configurations that configured the variational query into the same relational query.
To generate this presence condition from configurations we need to know the closed 
set of VDB's features.
%
This is done by the $\mathit{genFexp} (\config,\features)$ that takes a configuration and a closed set of 
features and generates the feature expression \dimMeta\ that is only satisfiable by the configuration
\config. For example, $\mathit{genFexp} (\setDef {\A},\setDef {\A, \B}) = \A \wedge \neg \B$ and
$\mathit{genFexp} (\setDef {\A, \B},\setDef {\A, \B}) = \A \wedge \B$.
%
Remember that the set of enabled features of a configuration denote the said configuration,
for example, $\setDef {\A}$ denotes the configuration in which only feature \A\ has been 
enabled.

%
In essence, the unique variants function can be defined for all data types that encode variation.
For example, the unique configuration function for 
variational queries can be defined as follows.
\begin{alignat*}{1}
\qGroup{\vQ} &=
  \{ \pQ^{e_1 \vee\ldots\vee e_n}
     \myOR \pQ^{e_1}, \ldots, \pQ^{e_n}
       \in \{ (\eeSem{\vQ})^{\mathit{genFexp}(\config,\features)}
         \myOR \config\in\confSet \} \}
% \qGroup \vQ &= \{ \annot \pQ \myOR \dimMeta = \bigvee_{\dimMeta_i \in \mathit{es}} \dimMeta_i,
% \mathit{es} = \{\dimMeta_i \myOR \forall \config \in \confSet. \eeSem \vQ = \pQ, 
% \dimMeta_i = \mathit{genFexp} (\config,\features) \}\\
% &\hspace{50pt}, \exists \config \in \confSet. \fSem \dimMeta = \t, \eeSem \vQ = \pQ
% \}
\end{alignat*}
%\centerline{
%\ensuremath{
%\qGroup \vQ = \setDef {\annot \pQ \myOR \forall \config \in \confSet: \fSem \dimMeta = \t,
%\eeSem \vQ = \pQ}
%}.}
%\]
The unique configuration for variational sets of attributes
($\aG(\cdot,\cdot)$) and variational conditions ($\cG(\cdot,\cdot)$) are
defined similarly; their types are given below.
\begin{alignat*}{1}
\aG(\cdot,\cdot) &:
  \vAttSet \totype \settype \fSet \totype {\vartype {\bm{(}\settype \attnametype \bm{)}}} \\
\cG(\cdot,\cdot) &:
  \vCondSet \totype \settype \fSet \totype \vartype \pCondSet
\end{alignat*}
%
However, the definition of $\qGroup[\cdot]{\cdot}$ is not efficient since it
still enumerates all possible 
configurations. Thus, we define the more efficient unique configuration function
for variational queries in \figref{vq-group}.
%
%\exref{group-vq} and \exref{vq-group} provide the unique configuration of the queries 
%given in \exref{conf-vq} and \exref{vq-sem}, respectively.

\begin{figure}
%\textbf{Configuration selection semantics of \vqsTxt:}
\begin{alignat*}{1}
\qGroup . &: \qSet \totype \settype {\bm{(} \vartype \pQSet \bm{)}}\\
%
\qGroup \vRel &= \setDef {\annot [\t] \pRel}\\
\qGroup {\vSel \vQ}  &=  
\setDef {\annot [\dimMeta \wedge \dimMeta_\vCond] {\left(\sigma_{\pCond} \pQ\right)} \myOR
\annot \pQ \in \qGroup \vQ, \annot [\dimMeta_\vCond] \pCond \in \cGroup}
\\
%
\qGroup {\vPrj [\vAttList] \vQ} &= 
\setDef {\annot [\dimMeta \wedge \dimMeta_\vAttList] {\left(\pi_{\pAttList} \pQ \right)} \myOR
\annot \pQ \in \qGroup \vQ, \annot [\dimMeta_\vAttList] \pAttList \in \aGroup}
\\
%
\qGroup {{\vQ_1} \times {\vQ_2}} &= 
\setDef {\annot [\dimMeta_1 \wedge \dimMeta_2] {\left(\pQ_1 \times \pQ_2\right)} \myOR
\annot [\dimMeta_1] \pQ_1 \in \qGroup {\vQ_1}, \annot [\dimMeta_2] \pQ_2 \in \qGroup {\vQ_2} }
\\
%
\qGroup {{\vQ_1} \Join_\vCond {\vQ_2}} &= 
\setDef {\annot [\dimMeta_1 \wedge \dimMeta_2 \wedge \dimMeta_\vCond] {\left(\pQ_1 \Join_{\pCond} \pQ_2 \right)} \myOR 
\annot [\dimMeta_1] \pQ_1 \in \qGroup {\vQ_1}, \annot [\dimMeta_2] \pQ_2 \in \qGroup {\vQ_2}
%& \hspace{104pt}
,\annot [\dimMeta_\vCond] \pCond \in \cGroup  }
\\
%
\qGroup {\chc {\vQ_1, \vQ_2}} &= 
\setDef {\annot [\dimMeta \wedge \dimMeta_1] \pQ_1 \myOR  \annot [\dimMeta_1] \pQ_1 \in \qGroup {\vQ_1} }
\cup 
\setDef {\annot [\neg \dimMeta \wedge \dimMeta_2] \pQ_2 \myOR  \annot [\dimMeta_2] \pQ_2 \in \qGroup {\vQ_2}}  \\
%
\qGroup {{\vQ_1} \circ {\vQ_2}} &= 
\setDef {\annot [\dimMeta_1 \wedge \dimMeta_2] {\left(\pQ_1 \circ \pQ_2\right)} \myOR
\annot [\dimMeta_1] \pQ_1 \in \qGroup {\vQ_1}, \annot [\dimMeta_2] \pQ_2 \in \qGroup {\vQ_2} }\\
%
\qGroup {\empRel} &= \annot [\t] { \empRel}
\end{alignat*}
\caption[Unique configuration of variational queries]{Unique configuration of variational queries. 
The unique configuration function assumes that the input is well-typed.
}
\label{fig:vq-group}
\end{figure}


\begin{example}
\label{eg:group-vq}
Consider the query \ensuremath{
\vQ_5 = \vPrj [{\vAtt_1, \optAtt [\fOne \wedge \fTwo] [\vAtt_2], \optAtt [\fTwo] [\vAtt_3]}] (\vRel)
}
given in \exref{conf-vq}. The unique configuration of this query results in the following set of queries:
%
\[
\qGroup [\{\A,\B\}] {\vQ_5} = \setDef {
\annot [(\A \wedge \neg \B) \vee (\neg \A \wedge \neg \B)] {(\pi_{\pAtt_1} (\pRel))},
\annot [\neg \A \wedge \B] {( \pi_{\pAtt_1, \pAtt_3} (\pRel))},
\annot [\A \wedge \B] {(\pi_{\pAtt_1, \pAtt_2, \pAtt_3} (\pRel))}
}.
\]
\end{example}

\begin{example}
\label{eg:vq-group}
Consider the query $\VVal {\vQ_1}$ configured in \exref{vq-sem}:\\
\centerline{
$\VVal {\vQ_1} = 
\pi_{\optAtt [(\vFour \vee \vFive) \wedge \neg \vThree] [\empno], 
\optAtt [\vFour \wedge \neg \vThree \wedge \neg \vFive] [\name], 
\optAtt [\vFive \wedge \neg \vThree \wedge \neg \vFour] [\fname], 
\optAtt [\vFive \wedge \neg \vThree \wedge \neg \vFour] [\lname]} (\empbio)
$.}  
The unique configuration of it results in:
\begin{alignat*}{1}
\qGroup [\{\vThree, \vFour, \vFive \}] {\VVal {\vQ_1}} &= \{
\annot [\vThree \wedge \neg \vFour \wedge \neg \vFive] {\empRel},
\annot [\neg \vThree \wedge \vFour \wedge \neg \vFive] {\left(\pi_{\empno, \name} (\empbio)\right)}\\
&\qquad ,\annot [\neg \vThree \wedge \vFour \wedge \neg \vFive] {\left(\pi_{\empno, \fname, \lname} (\empbio) \right)}\}.
\end{alignat*}
\end{example}



\subsection{Accumulation of Relational Tables to a Variational Table}
\label{sec:accum}

After connecting variational queries to relational queries, to define the 
semantics of VRA we need to connect
the results of multiple relational queries to the result of a single variational 
query. 
%
Since we have two approaches to connect a variational query to relational queries 
we define two \emph{accumulation} functions that generate a 
variational table from a set of relational tables. 

%
The first accumulation function $\mathit{accum} : \settype \fSet \totype \settype {\typepair \confSet \pTabSet} \totype \tabletype$ takes the feature space of a database and a set of relational
tables with their attached configurations and generates a variational table. \figref{accum1} 
defines this function in terms of some auxiliary functions. 
%
The $\mathit{mkTable}$ function takes a variational relation schema and a set of 
variational relation contents and generates a variational table that has the given schema
and the variational tuples in the input tables. 
%
The $\mathit{addPresCondToConfTables}$ function maps the $\mathit{addPresCondToConfContent}$
over a set of tables and their attached configuration and the  $\mathit{addPresCondToConfContent}$
function adds the \pcatt\ attribute to a relational table and its corresponding value which is 
a feature expression associated with the given configuration using the closed set of
features.
%
The $\mathit{fitConfTablesToVsch}$ maps the function $\mathit{fitTableToVsch}$ to tables of a set of 
relational tables and their attached configuration.
The $\mathit{fitTableToVsch}$ function adjusts a table, both its schema and content, 
to a variational relation schema.
%
The $\mathit{tablesToVsch}$ maps the function $\mathit{schToVsch}$ to a set of 
relational tables and their attached configuration. 
The $\mathit{schToVsch}$ generates a variational relation schema from a set of
plain relation schema and their attached configuration given the closed set of 
features of the database's feature space.%
\footnote{In the implementation, for efficiency, we pass the type of the query from VRA's type system
as the variational relation schema that is generated by the $\mathit{tablesToVsch}$ function.}
%
Note that to generate a feature expression from a configuration it is essential to
pass the closed set of features.
%
\exref{acc-table-from-conf} illustrates the behavior of these auxiliary functions and the
table accumulation function over the relational tables in \tabref{vq-conf-res}.


\begin{figure}

\textbf{Table accumulation function:}
\begin{alignat*}{1}
\mathit{accum} &: \settype \fSet \totype \settype {\typepair \confSet \pTabSet} \totype \tabletype\\
\mathit{accum} \  \mathit{fs} \ \mathit{ts} &= \mathit{mkTable} \ \mathit{vsch} \ \mathit{tables}\\
%&\hspace{60pt} (\mathit{addPresCondToConfTables} \ \mathit{fs} \\
%&\hspace{140pt} (\mathit{fitConfTablesToVsch} \ \mathit{ts} \ \mathit{vsch}))\\
&\hspace{-40pt}\textit{where }
\mathit{vsch} = \mathit{tablesToVsch} \ \mathit{fs} \ \mathit{ts}\\
&\hspace{-6pt} \mathit{tables} = \mathit{addPresCondToConfTables} \ \mathit{fs} \ \mathit {fitted}\\
&\hspace{-6pt} \mathit{fitted} = \mathit{fitConfTablesToVsch} \ \mathit{ts} \ \mathit{vsch}
\end{alignat*}


\medskip 
\textbf{Auxiliary functions for table accumulation:}
\footnotesize
\begin{alignat*}{1}
\mathit{schToVsch} &: \settype \fSet \totype \settype {\typepair \confSet \pRelSchSet} \totype \vRelSchSet\\
\mathit{tablesToVsch} &: \settype \fSet \totype \settype {\typepair \confSet \pTabSet} \totype \vRelSchSet\\
\mathit{fitTableToVsch} &: \pTabSet \totype \vRelSchSet \totype \pTabSet\\
\mathit{fitConfTablesToVsch} &: \settype {\typepair \confSet \pTabSet} \totype \vRelSchSet \totype \settype {\typepair \confSet \pTabSet}\\
\mathit{addPresCondToConfContent} &: \settype \fSet \totype \typepair \confSet \pRelContSet \totype \vRelContSet\\
\mathit{addPresCondToConfTables} &: \settype \fSet \totype \settype {\typepair \confSet \pTabSet} \totype \settype \vRelContSet\\
\mathit{mkTable} &: \vRelSchSet \totype \settype \vRelContSet \totype \tabletype
\end{alignat*}


\caption[Accumulation function of a set of relational tables with their attached configuration into a variational table]{Accumulation function of a set of relational tables with their attached configuration into a variational table and its auxiliary functions. The definition uses spaces to pass parameters. For example, $f \ x$ states that the parameter $x$ is passed to the function $x$ and $f\ x\ y$ states that
parameters $x$ and $y$ are passed to $f$ as the first and second arguments, respectively.
}
\label{fig:accum1}
\end{figure}



\begin{example}
\label{eg:acc-table-from-conf}
Consider the query $\VVal {\vQ_1}$ written over the VDB with variational schema $\vSch_2$ and 
feature space $\features = \setDef {\vThree, \vFour, \vFive}$, all given in \exref{vq-specific}. 
%
All configured relational queries of $\VVal {\vQ_1}$ for VDB's valid configurations and their
corresponding results in form of a relational table are given
in \exref{vq-sem} and \tabref{vq-conf-res}, respectively. 
%
Now we show how the relational tables of the configured queries, shown in \tabref{vq-conf-res}, are accumulated 
to the variational table, shown in \tabref{vq1-res}, as the result of the variational query $\VVal \vQ_1$ by 
using the table accumulation function $\mathit{accum}$.
%
As the first step of accumulation, we generate the variational relation schema by
applying $\mathit{tablesToVsch}$ to tables in \tabref{vq-conf-res}. 
This results in the variational relation schema $\vRelSch_{\mathit{accum}}$
%
\begin{alignat*}{1}
\vRelSch_{\mathit{accum}} &= \mathit{result} (\annot [(\neg \vThree \wedge \vFour \wedge \neg \vFive)\vee(\neg \vThree \wedge \neg \vFour \wedge \vFive)] \empno, \annot [\neg \vThree \wedge \vFour \wedge \neg \vFive] \name,\\
&\hspace{30pt}\annot [\neg \vThree \wedge \neg \vFour \wedge \vFive] \fname, \annot [\neg \vThree \wedge \neg \vFour \wedge \vFive] \lname)^{\oneof {\vThree,\vFour,\vFive}}
\end{alignat*}
%
\noindent
Note that the presence conditions are generated based on the configurations attached to
the tables. For example, the presence condition $(\neg \vThree \wedge \vFour \wedge \neg \vFive)\vee(\neg \vThree \wedge \neg \vFour \wedge \vFive)$ associated with the attribute \empno\
is the disjunction of  two feature expressions $(\neg \vThree \wedge \vFour \wedge \neg \vFive)$
and $(\neg \vThree \wedge \neg \vFour \wedge \vFive)$ where they represent the configuration
\setDef \vFour\ (associated to \tabref{vq-conf2}) and \setDef \vFive\ (associated to \tabref{vq-conf3}),
respectively. That is, the configuration \setDef \vFour\ represents the variants that only enable the
feature \vFour\ from \vThree--\vFive, thus, its corresponding feature expression is $(\neg \vThree \wedge \vFour \wedge \neg \vFive)$. That is why we need to pass the closed set of features 
to the auxiliary functions (to generate feature expression corresponding to configurations).

%
In the next step, the tables in \tabref{vq-conf-res} are adjusted so that they all match a certain
relation schema. This is achieved by the $\mathit{fitConfTablesToVsch}$ which gets all the 
tables in \tabref{vq-conf-res} with their associated configurations and the variational relation
schema generated by passing them to the $\mathit{tablesToVsch}$. This is done by
mapping the  $\mathit{fitTableToVsch}$ to all the tables in 
\tabref{vq-conf-res} with their associated configurations. This function simply adds
 attributes of the variational relation schema to the table that do not exists in the table 
 and puts \nul\ as values (indicated by the white space) in the tuples for those attributes. 
%
\tabref{fitting1}--\tabref{fitting3} illustrate the application of 
$\mathit{fitTableToVsch}$ to \tabref{vq-conf1}--\tabref{vq-conf3} and variational relation 
schema $\vRelSch_{\mathit{accum}}$.
%
\begin{table}[!htbp]
%\caption[Example of step two of table accumulation]{Step two of table accumulation applies the 
%$\mathit{fitConfTablesToVsch}$ function to all tables of \tabref{vq-conf-res} and their corresponding 
%configurations and the variational relation schema $\vRelSch_{\mathit{accum}}$.}
%\label{tab:fitting}
%\centering
%\small
%%\footnotesize
%%\scriptsize
%\begin{subtable}[t]{\textwidth}
\centering
\caption[Example of step two of table accumulation]{Result of the $\mathit{fitTableToVsch}$ applied to 
\tabref{vq-conf1} and  variational relation schema $\vRelSch_{\mathit{accum}}$.}
\label{tab:fitting1}
\arrayrulecolor{black}
\begin{tabular} {c | l l l l  }
% {\textcolor{blue}{$\oneof {\vThree, \vFour,\vFive}$} }& {\textcolor{blue}{$\vFour \vee \vFive$}}&  {\textcolor{blue}{$\vFour $}} &  {\textcolor{blue}{$\vFive $}} &  {\textcolor{blue}{$\vFive$}} & {\textcolor{blue}{\texttt{true}}}\\
%\arrayrulecolor{blue}\hdashline
\multirow{2}{*}{$\mathit{result}$}  & \empno & \name & \fname & \lname\\
\arrayrulecolor{black}\cline{2-5} 
& & & & \\
\arrayrulecolor{white}\hline
\end{tabular}
%\end{subtable}
%
%%\medskip
%%\medskip
%\medskip
%\begin{subtable}[t]{\textwidth}
%%\begin{center}
%\centering
%%\tiny
%\caption{Result of the $\mathit{fitTableToVsch}$ applied to 
%\tabref{vq-conf2} and  variational relation schema $\vRelSch_{\mathit{accum}}$.}
%\label{tab:fitting2}
%\arrayrulecolor{black}
%\begin{tabular} {c | l l l l  }
%% {\textcolor{blue}{$\oneof {\vThree, \vFour,\vFive}$} }& {\textcolor{blue}{$\vFour \vee \vFive$}}&  {\textcolor{blue}{$\vFour $}} &  {\textcolor{blue}{$\vFive $}} &  {\textcolor{blue}{$\vFive$}} & {\textcolor{blue}{\texttt{true}}}\\
%%\arrayrulecolor{blue}\hdashline
%\multirow{2}{*}{$\mathit{result}$}  & \empno & \name & \fname & \lname\\
%\arrayrulecolor{black}\cline{2-5}  
%& 12001 & Ulf Hofstetter & & \\
%& 12002 & Luise McFarlan & & \\
%& 12003 & Shir DuCasse & & \\
% &80001 & Nagui Merli & & \\
% & 80002 & Mayuko Meszaros & & \\
% & 80003 & Theirry Viele & & \\
%% &80001 & Nagui Merli & & \\
%% & 80002 & Mayuko Meszaros & & \\
%% & 80003 & Theirry Viele & & \\
%&\ldots & \ldots  & \ldots & \ldots \\
%\arrayrulecolor{white}\hline
%\end{tabular}
%%\end{center}
%\end{subtable}
%
%%\medskip
%%\medskip
%\medskip
%\begin{subtable}[t]{\textwidth}
%%\begin{center}
%\centering
%%\footnotesize
%%\tiny
%\caption{Result of the $\mathit{fitTableToVsch}$ applied to 
%\tabref{vq-conf3} and  variational relation schema $\vRelSch_{\mathit{accum}}$.}
%\label{tab:fitting3}
%\arrayrulecolor{black}
%\begin{tabular} {c | l l l l  }
%% {\textcolor{blue}{$\oneof {\vThree, \vFour,\vFive}$} }& {\textcolor{blue}{$\vFour \vee \vFive$}}&  {\textcolor{blue}{$\vFour $}} &  {\textcolor{blue}{$\vFive $}} &  {\textcolor{blue}{$\vFive$}} & {\textcolor{blue}{\texttt{true}}}\\
%%\arrayrulecolor{blue}\hdashline
%\multirow{2}{*}{$\mathit{result}$}  & \empno & \name & \fname & \lname\\
%\arrayrulecolor{black}\cline{2-5}  
% & 12001 & & Ulf & Hofstetter \\
%& 12002 & & Luise & McFarlan\\
%& 12003 & & Shir & DuCasse\\
% &80001 & & Nagui & Merli\\
% & 80002 & & Mayuko & Meszaros\\
% & 80003 & & Theirry & Viele\\
% & 200001 & & Selwyn & Koshiba \\
% & 200002 & & Bedrich & Markovitch \\
% & 200003 & & Pascal & Benzmuller  \\
%% & 200001 & & Selwyn & Koshiba \\
%% & 200002 & & Bedrich & Markovitch \\
%% & 200003 & & Pascal & Benzmuller  \\
% & \ldots & \ldots & \ldots & \ldots \\
% \arrayrulecolor{white}\hline
%\end{tabular}
%%\end{center}
%\end{subtable}
%
\end{table}


\begin{table}[!htbp]
%\caption[Example of step two of table accumulation]{Step two of table accumulation applies the 
%$\mathit{fitConfTablesToVsch}$ function to all tables of \tabref{vq-conf-res} and their corresponding 
%configurations and the variational relation schema $\vRelSch_{\mathit{accum}}$.}
%\label{tab:fitting}
%\centering
%\small
%%\footnotesize
%%\scriptsize
%\begin{subtable}[t]{\textwidth}
%\centering
%\caption{Result of the $\mathit{fitTableToVsch}$ applied to 
%\tabref{vq-conf1} and  variational relation schema $\vRelSch_{\mathit{accum}}$.}
%\label{tab:fitting1}
%\arrayrulecolor{black}
%\begin{tabular} {c | l l l l  }
%% {\textcolor{blue}{$\oneof {\vThree, \vFour,\vFive}$} }& {\textcolor{blue}{$\vFour \vee \vFive$}}&  {\textcolor{blue}{$\vFour $}} &  {\textcolor{blue}{$\vFive $}} &  {\textcolor{blue}{$\vFive$}} & {\textcolor{blue}{\texttt{true}}}\\
%%\arrayrulecolor{blue}\hdashline
%\multirow{2}{*}{$\mathit{result}$}  & \empno & \name & \fname & \lname\\
%\arrayrulecolor{black}\cline{2-5} 
%& & & & \\
%\arrayrulecolor{white}\hline
%\end{tabular}
%\end{subtable}
%
%%\medskip
%%\medskip
%\medskip
%\begin{subtable}[t]{\textwidth}
%\begin{center}
\centering
%\tiny
\caption[Example of step two of table accumulation]{Result of the $\mathit{fitTableToVsch}$ applied to 
\tabref{vq-conf2} and  variational relation schema $\vRelSch_{\mathit{accum}}$.}
\label{tab:fitting2}
\arrayrulecolor{black}
\begin{tabular} {c | l l l l  }
% {\textcolor{blue}{$\oneof {\vThree, \vFour,\vFive}$} }& {\textcolor{blue}{$\vFour \vee \vFive$}}&  {\textcolor{blue}{$\vFour $}} &  {\textcolor{blue}{$\vFive $}} &  {\textcolor{blue}{$\vFive$}} & {\textcolor{blue}{\texttt{true}}}\\
%\arrayrulecolor{blue}\hdashline
\multirow{2}{*}{$\mathit{result}$}  & \empno & \name & \fname & \lname\\
\arrayrulecolor{black}\cline{2-5}  
& 12001 & Ulf Hofstetter & & \\
& 12002 & Luise McFarlan & & \\
& 12003 & Shir DuCasse & & \\
 &80001 & Nagui Merli & & \\
 & 80002 & Mayuko Meszaros & & \\
 & 80003 & Theirry Viele & & \\
% &80001 & Nagui Merli & & \\
% & 80002 & Mayuko Meszaros & & \\
% & 80003 & Theirry Viele & & \\
&\ldots & \ldots  & \ldots & \ldots \\
\arrayrulecolor{white}\hline
\end{tabular}
%\end{center}
%\end{subtable}
%
%%\medskip
%%\medskip
%\medskip
%\begin{subtable}[t]{\textwidth}
%%\begin{center}
%\centering
%%\footnotesize
%%\tiny
%\caption{Result of the $\mathit{fitTableToVsch}$ applied to 
%\tabref{vq-conf3} and  variational relation schema $\vRelSch_{\mathit{accum}}$.}
%\label{tab:fitting3}
%\arrayrulecolor{black}
%\begin{tabular} {c | l l l l  }
%% {\textcolor{blue}{$\oneof {\vThree, \vFour,\vFive}$} }& {\textcolor{blue}{$\vFour \vee \vFive$}}&  {\textcolor{blue}{$\vFour $}} &  {\textcolor{blue}{$\vFive $}} &  {\textcolor{blue}{$\vFive$}} & {\textcolor{blue}{\texttt{true}}}\\
%%\arrayrulecolor{blue}\hdashline
%\multirow{2}{*}{$\mathit{result}$}  & \empno & \name & \fname & \lname\\
%\arrayrulecolor{black}\cline{2-5}  
% & 12001 & & Ulf & Hofstetter \\
%& 12002 & & Luise & McFarlan\\
%& 12003 & & Shir & DuCasse\\
% &80001 & & Nagui & Merli\\
% & 80002 & & Mayuko & Meszaros\\
% & 80003 & & Theirry & Viele\\
% & 200001 & & Selwyn & Koshiba \\
% & 200002 & & Bedrich & Markovitch \\
% & 200003 & & Pascal & Benzmuller  \\
%% & 200001 & & Selwyn & Koshiba \\
%% & 200002 & & Bedrich & Markovitch \\
%% & 200003 & & Pascal & Benzmuller  \\
% & \ldots & \ldots & \ldots & \ldots \\
% \arrayrulecolor{white}\hline
%\end{tabular}
%%\end{center}
%\end{subtable}
%
\end{table}


\begin{table}[!htbp]
%\caption[Example of step two of table accumulation]{Step two of table accumulation applies the 
%$\mathit{fitConfTablesToVsch}$ function to all tables of \tabref{vq-conf-res} and their corresponding 
%configurations and the variational relation schema $\vRelSch_{\mathit{accum}}$.}
%\label{tab:fitting}
%\centering
%\small
%%\footnotesize
%%\scriptsize
%\begin{subtable}[t]{\textwidth}
%\centering
%\caption{Result of the $\mathit{fitTableToVsch}$ applied to 
%\tabref{vq-conf1} and  variational relation schema $\vRelSch_{\mathit{accum}}$.}
%\label{tab:fitting1}
%\arrayrulecolor{black}
%\begin{tabular} {c | l l l l  }
%% {\textcolor{blue}{$\oneof {\vThree, \vFour,\vFive}$} }& {\textcolor{blue}{$\vFour \vee \vFive$}}&  {\textcolor{blue}{$\vFour $}} &  {\textcolor{blue}{$\vFive $}} &  {\textcolor{blue}{$\vFive$}} & {\textcolor{blue}{\texttt{true}}}\\
%%\arrayrulecolor{blue}\hdashline
%\multirow{2}{*}{$\mathit{result}$}  & \empno & \name & \fname & \lname\\
%\arrayrulecolor{black}\cline{2-5} 
%& & & & \\
%\arrayrulecolor{white}\hline
%\end{tabular}
%\end{subtable}
%
%%\medskip
%%\medskip
%\medskip
%\begin{subtable}[t]{\textwidth}
%%\begin{center}
%\centering
%%\tiny
%\caption{Result of the $\mathit{fitTableToVsch}$ applied to 
%\tabref{vq-conf2} and  variational relation schema $\vRelSch_{\mathit{accum}}$.}
%\label{tab:fitting2}
%\arrayrulecolor{black}
%\begin{tabular} {c | l l l l  }
%% {\textcolor{blue}{$\oneof {\vThree, \vFour,\vFive}$} }& {\textcolor{blue}{$\vFour \vee \vFive$}}&  {\textcolor{blue}{$\vFour $}} &  {\textcolor{blue}{$\vFive $}} &  {\textcolor{blue}{$\vFive$}} & {\textcolor{blue}{\texttt{true}}}\\
%%\arrayrulecolor{blue}\hdashline
%\multirow{2}{*}{$\mathit{result}$}  & \empno & \name & \fname & \lname\\
%\arrayrulecolor{black}\cline{2-5}  
%& 12001 & Ulf Hofstetter & & \\
%& 12002 & Luise McFarlan & & \\
%& 12003 & Shir DuCasse & & \\
% &80001 & Nagui Merli & & \\
% & 80002 & Mayuko Meszaros & & \\
% & 80003 & Theirry Viele & & \\
%% &80001 & Nagui Merli & & \\
%% & 80002 & Mayuko Meszaros & & \\
%% & 80003 & Theirry Viele & & \\
%&\ldots & \ldots  & \ldots & \ldots \\
%\arrayrulecolor{white}\hline
%\end{tabular}
%%\end{center}
%\end{subtable}
%
%%\medskip
%%\medskip
%\medskip
%\begin{subtable}[t]{\textwidth}
%\begin{center}
\centering
%\footnotesize
%\tiny
\caption[Example of step two of table accumulation]{Result of the $\mathit{fitTableToVsch}$ applied to 
\tabref{vq-conf3} and  variational relation schema $\vRelSch_{\mathit{accum}}$.}
\label{tab:fitting3}
\arrayrulecolor{black}
\begin{tabular} {c | l l l l  }
% {\textcolor{blue}{$\oneof {\vThree, \vFour,\vFive}$} }& {\textcolor{blue}{$\vFour \vee \vFive$}}&  {\textcolor{blue}{$\vFour $}} &  {\textcolor{blue}{$\vFive $}} &  {\textcolor{blue}{$\vFive$}} & {\textcolor{blue}{\texttt{true}}}\\
%\arrayrulecolor{blue}\hdashline
\multirow{2}{*}{$\mathit{result}$}  & \empno & \name & \fname & \lname\\
\arrayrulecolor{black}\cline{2-5}  
 & 12001 & & Ulf & Hofstetter \\
& 12002 & & Luise & McFarlan\\
& 12003 & & Shir & DuCasse\\
 &80001 & & Nagui & Merli\\
 & 80002 & & Mayuko & Meszaros\\
 & 80003 & & Theirry & Viele\\
 & 200001 & & Selwyn & Koshiba \\
 & 200002 & & Bedrich & Markovitch \\
 & 200003 & & Pascal & Benzmuller  \\
% & 200001 & & Selwyn & Koshiba \\
% & 200002 & & Bedrich & Markovitch \\
% & 200003 & & Pascal & Benzmuller  \\
 & \ldots & \ldots & \ldots & \ldots \\
 \arrayrulecolor{white}\hline
\end{tabular}
%\end{center}
%\end{subtable}
%
\end{table}


%
Then, the $\mathit{addPresCondToConfContent}$ 
function adds the presence condition attribute and its values 
to relation contents of \tabref{fitting1}--\tabref{fitting3}, resulting in \tabref{pcadded} which illustrates a set of 
relation contents that are separated by the red bold line. Note that since \tabref{fitting1}
does not have any tuples, \tabref{pcadded} does not have any tuples associated with
the variant \setDef \vThree.
%
\begin{table}[!htbp]
\caption[Example of step three of table accumulation]{Step three of table accumulation adds the 
presence condition values to relation contents. The table illustrates a set of relation contents that 
are separated by the red bold line between them. The tuples follow the order of attributes in the
relation schema.}
\label{tab:pcadded}
\centering
\small
%\footnotesize
%\scriptsize
\arrayrulecolor{blue}
\begin{tabular} {c !{\color{black}\vrule} l l l l : l }
%\multirow{2}{*}{$\mathit{result}$}  & \empno & \name & \fname & \lname & \pcatt \\
\multirow{2}{*}{\textcolor{white}{result}} & & & & &  \\
\arrayrulecolor{black}\cline{2-6}
& & & & &  \textcolor{blue}{$\vThree \wedge \neg \vFour \wedge \neg \vFive$}\\
\arrayrulecolor{red}\specialrule{.2em}{.1em}{.1em}
 &80001 & Nagui Merli & & & \textcolor{blue}{$\neg \vThree \wedge \vFour \wedge \neg \vFive$}\\
 & 80002 & Mayuko Meszaros & & & \textcolor{blue}{$\neg \vThree \wedge \vFour \wedge \neg \vFive$}\\
 & 80003 & Theirry Viele & & & \textcolor{blue}{$\neg \vThree \wedge \vFour \wedge \neg \vFive$}\\
&\ldots & \ldots  & \ldots & \ldots & \textcolor{blue}{\ldots}\\
\arrayrulecolor{red}\specialrule{.2em}{.1em}{.1em}
 & 200001 & & Selwyn & Koshiba & \textcolor{blue}{$\neg \vThree \wedge \neg \vFour \wedge \vFive$}\\
 & 200002 & & Bedrich & Markovitch &\textcolor{blue}{$\neg \vThree \wedge \neg \vFour \wedge \vFive$} \\
 & 200003 & & Pascal & Benzmuller &\textcolor{blue}{$\neg \vThree \wedge \neg \vFour \wedge \vFive$} \\
 & \ldots & \ldots & \ldots & \ldots & \textcolor{blue}{\ldots}\\
\arrayrulecolor{white}\hline
\end{tabular}

\end{table}


%
Finally, the $\mathit{mkTable}$ function takes the variational relation schema $\vRelSch_{\mathit{accum}}$
and \tabref{pcadded}. Note that the values in tuples of \tabref{pcadded} follow the order of the
attributes in the variational relation schema. This results in \tabref{mktab} which is equivalent to
the result of $\VVal \vQ_1$ given in \tabref{vq1-res}.
%
\begin{table}
\caption[Example of the final step of table accumulation]{Final step of table accumulation passes the
variational relation schema $\vRelSch_{\mathit{accum}}$ and relation contents in \tabref{pcadded} to the $\mathit{mkTable}$ function.}
\label{tab:mktab}
\centering
%\footnotesize
%\scriptsize
\tiny
\arrayrulecolor{blue}
%!{\color{black}\vrule}
\begin{tabular} {c !{\color{black}\vrule} l l l l : l }
& {\textcolor{blue}{$(\neg \vThree \wedge \vFour \wedge \neg \vFive)$}}&   &  &  & \\
 {\textcolor{blue}{$\oneof {\vThree, \vFour,\vFive}$} }& {\textcolor{blue}{$(\vee(\neg \vThree \wedge \neg \vFour \wedge \vFive)$}}&  {\textcolor{blue}{$\neg \vThree \wedge \vFour \wedge \neg \vFive $}} &  {\textcolor{blue}{$\neg \vThree \wedge \neg \vFour \wedge \vFive $}} &  {\textcolor{blue}{$\neg \vThree \wedge \neg \vFour \wedge \vFive$}} & {\textcolor{blue}{\texttt{true}}}\\
\arrayrulecolor{blue}\hdashline
\multirow{2}{*}{$\mathit{result}$}  & \empno & \name & \fname & \lname & \pcatt \\
\arrayrulecolor{black}\cline{2-6}
& & & & &  \textcolor{blue}{$\vThree \wedge \neg \vFour \wedge \neg \vFive$}\\
%& 12001 & & & & \textcolor{blue}{\vThree}\\
%& 12002 & & & & \textcolor{blue}{\vThree}\\
%& 12003 & & & & \textcolor{blue}{\vThree}\\
 &80001  & Nagui Merli & & & \textcolor{blue}{$\neg \vThree \wedge \vFour \wedge \neg \vFive$}\\
 & 80002 & Mayuko Meszaros & & & \textcolor{blue}{$\neg \vThree \wedge \vFour \wedge \neg \vFive$}\\
 & 80003 & Theirry Viele & & & \textcolor{blue}{$\neg \vThree \wedge \vFour \wedge \neg \vFive$}\\
 & 200001  & & Selwyn & Koshiba & \textcolor{blue}{$\neg \vThree \wedge \neg \vFour \wedge \vFive$} \\
 & 200002  & & Bedrich & Markovitch & \textcolor{blue}{$\neg \vThree \wedge \neg \vFour \wedge \vFive$} \\
 & 200003  & & Pascal & Benzmuller  & \textcolor{blue}{$\neg \vThree \wedge \neg \vFour \wedge \vFive$} \\
 & \ldots  & \ldots & \ldots & \ldots& \textcolor{blue}{\ldots} \\
\arrayrulecolor{white}\hline
\end{tabular}

\end{table}

\end{example}

The second accumulation function
 $\VVal {\mathit{accum}} :  \settype {\bm{(}\vartype \pTabSet\bm{)}} \totype \tabletype$ 
 takes a set of relational tables that are annotated with
a feature expression instead of their attached configuration. \figref{accum2} defines
this function and its auxiliary functions. The auxiliary functions are similar to the ones
defined in \figref{accum2} except that they do not need to generate a feature expression
from a configuration and a set of closed features.
%
%\exref{acc-table-from-group} illustrates the behavior of these auxiliary functions and the second accumulation
%function over the tables in \tabref{vq-conf-res}.

\begin{figure}[ht!]

\textbf{Table accumulation function:}
\begin{alignat*}{1}
\VVal {\mathit{accum}} &:  \settype {\bm{(}\vartype \pTabSet\bm{)}} \totype \tabletype\\
\VVal {\mathit{accum}} \  \mathit{fs} \ \mathit{ts} &= \mathit{mkTable} \ \mathit{vsch} \ \mathit{tables}\\
&\hspace{-40pt}\textit{where }
\mathit{vsch} = \mathit{annotTablesToVsch}  \ \mathit{ts}\\
&\hspace{-6pt} \mathit{tables} = \mathit{addPresCondToVarTables} \ \mathit {fitted}\\
&\hspace{-6pt} \mathit{fitted} = \mathit{fitVarTablesToVsch} \ \mathit{ts} \ \mathit{vsch}
\end{alignat*}


\medskip 
\textbf{Auxiliary functions for table accumulation:}
\footnotesize
\begin{alignat*}{1}
\mathit{annotSchToVsch} &:  \settype {\bm{(}\vartype \pRelSchSet\bm{)}} \totype \vRelSchSet\\
\mathit{annotTablesToVsch} &:  \settype {\bm{(}\vartype \pTabSet\bm{)}} \totype \vRelSchSet\\
%\mathit{fitTableToVsch} &: \pTabSet \totype \vRelSchSet \totype \pTabSet\\
\mathit{fitVarTablesToVsch} &: \settype {\bm{(}\vartype \pTabSet\bm{)}} \totype \vRelSchSet \totype \settype {\bm{(}\vartype \pTabSet\bm{)}}\\
\mathit{addPresCondToVarContent} &:  \vartype \pRelContSet \totype \vRelContSet\\
\mathit{addPresCondToVarTables} &:  \settype {\bm{(}\vartype \pTabSet\bm{)}} \totype \settype \vRelContSet
%\mathit{mkTable} &: \vRelSchSet \totype \settype \vRelContSet \totype \tabletype
\end{alignat*}


\caption[Accumulation function of a set of relational tables annotated with a feature expression into a variational table]{Accumulation function of a set of relational tables annotated with a feature expression into a variational table and its auxiliary functions. The definition uses spaces to pass parameters, e.g., $f \ x = f(x)$ and $f \ x \ y = f(x,y)$.
}
\label{fig:accum2}
\end{figure}



%\begin{example}
%\label{eg:acc-table-from-group}
%\wrrite{write this}
%\end{example}
\subsection{VRA Type System}
\label{sec:typesys}

\TODO{type sys}


\section{Explicitly Annotating Queries}
\label{sec:constrain}

%\point{type system allows the ql to be flexible and usable.}
%The type system is designed s.t. it relieves the user from necessarily incorporating
%the v-schema variability into their queries as long as the variational queries variability
%does not violate the v-schema, 
Variational queries do not need to repeat information that can be inferred from the v-schema
or the type of a query.
%
For example, the query \ensuremath{\vQ_1} shown in \exref{vq-specific} 
does not contradict the schema and
thus is type correct. However,
 it does not include the presence conditions of attributes and the relation encoded in
the schema while \ensuremath{\vQ_6} repeats this information:\\
%
\centerline{
\ensuremath{
\vQ_6 =
\pi_{\optAtt [\vFour \vee \vFive] [\empno], \optAtt [\vFour] [\name], \optAtt [\vFive] [\fname], \optAtt [\vFive] [\lname]  } \left(\chc [\dimMeta_2] {\empbio, \empRel} \right)}}.

%\pi_{\optAtt [(\vFour \vee \vFive) \wedge \fModel_2] [\empno], \optAtt [\vFour \wedge \fModel_2] [\name], \optAtt [\vFive \wedge \fModel_2] [\fname], \optAtt [\vFive \wedge \fModel_2] [\lname]  } \empbio}}.
%

%\NOTE{
%This is the unsimplified version:
%\begin{align*}
%\VVal {\vQ_5} &= 
%\pi_{\optAtt [\vFour \vee \vFive] [\empno], \optAtt [\vFour] [\name], \optAtt [\vFive] [\fname], \optAtt [\vFive] [\lname]  } \\
%&(\chc [ \fModel_2 ] {\pi_{\empno, \sex, \birthdate, \optAtt [\vFour ] [\name], \optAtt [\vFive] [\fname], \optAtt [\vFive] [\lname]} \empbio, \empRel  })
%\end{align*}
%}
Similarly, the projection in the query 
\ensuremath{\vQ_7 = \pi_{\name, \fname} (\mathit{subq}_7)}
where 
\ensuremath{
\mathit{subq}_7 = \chc [ \vFour] {\pi_\name (\vQ_6), \pi_\fname (\vQ_6)}
}
is written over 
\ensuremath{\vSch_2} and it 
%\centerline{
%\ensuremath{
%\vQ_6 =
%\pi_{\name, \fname} \mathit{subq}_6
%} 
%}}
does not repeat the presence conditions of attributes from its \ensuremath{\mathit{subq}_7}'s type.
The query
%\centerline{
\ensuremath{
\vQ_8 =
\pi_{\optAtt [\vFour ] [\name],\optAtt [\neg \vFour] [\fname]} (\mathit{subq}_7)
%\chc [ \vFour] {\pi_\name \vQ_5, \pi_\fname \vQ_5}
}
%}
makes the annotations of projected attributes \emph{explicit} with respect to both 
the v-schema \ensuremath{\vSch_2} and its subquery's type.
%\TODO {give an example, schema: R(A,B), query: $\pi_{A,B} (F<\pi_A R, \pi_B R>)$
%becomes $\pi_{A^F, B^{\neg F}} ...$}
%The variation encoded in variational queries can
%be more restrictive or more loose than v-schema variation without violating them.
Although relieving the user from explicitly repeating variation makes VRA easier to use, 
queries still have to state variation explicitly to avoid losing information when 
decoupled from the schema.
%We do this by defining a function, 
%\ensuremath {\constrain \vQ}, with type \ensuremath{ \qSet \to \vSchSet \to \qSet
%},
%that \emph{explicitly annotates a query \vQ\ given the underlying schema \vSch}.
We do this by defining the function 
\ensuremath {\constrain \vQ : \qSet \totype \vSchSet \totype \qSet
},
that \emph{explicitly annotates a query \vQ\ with the  schema \vSch}.
%Note that \ensuremath {\constrain \vQ} needs to take the underlying schema as
%an input since it is using the type system (which relies on the schema) as a helper function.
The explicitly annotating query function, 
formally defined in \figref{constrain}, 
conjoins attributes and relations
presence conditions with the corresponding annotations in the query 
and wraps subqueries in a choice when needed. 
Note that, $\vQ_8$ and $\vQ_6$ are the result of $\constrain [\vSch_2] {\vQ_7}$
and $\constrain [\vSch_2] {\vQ_1}$, respectively, after simplification~\footnote{More specifically,
they are simpilified using rules defined in \figref{var-min}}.
%Queries $\vQ_7$ and $\vQ_5$ are examples of applying the 
%explicitly annotation function to queries $\vQ_6$ and $\vQ_1$, respectively,
%after simplifying them.
%\exref{constrain} illustrates how the constrain function transforms queries
%and allows users to be more flexible with their queries. 

\section{Explicitly Annotating Queries}
\label{sec:constrain}

%\point{type system allows the ql to be flexible and usable.}
%The type system is designed s.t. it relieves the user from necessarily incorporating
%the v-schema variability into their queries as long as the variational queries variability
%does not violate the v-schema, 
Variational queries do not need to repeat information that can be inferred from the v-schema
or the type of a query.
%
For example, the query \ensuremath{\vQ_1} shown in \exref{vq-specific} 
does not contradict the schema and
thus is type correct. However,
 it does not include the presence conditions of attributes and the relation encoded in
the schema while \ensuremath{\vQ_6} repeats this information:\\
%
\centerline{
\ensuremath{
\vQ_6 =
\pi_{\optAtt [\vFour \vee \vFive] [\empno], \optAtt [\vFour] [\name], \optAtt [\vFive] [\fname], \optAtt [\vFive] [\lname]  } \left(\chc [\dimMeta_2] {\empbio, \empRel} \right)}}.

%\pi_{\optAtt [(\vFour \vee \vFive) \wedge \fModel_2] [\empno], \optAtt [\vFour \wedge \fModel_2] [\name], \optAtt [\vFive \wedge \fModel_2] [\fname], \optAtt [\vFive \wedge \fModel_2] [\lname]  } \empbio}}.
%

%\NOTE{
%This is the unsimplified version:
%\begin{align*}
%\VVal {\vQ_5} &= 
%\pi_{\optAtt [\vFour \vee \vFive] [\empno], \optAtt [\vFour] [\name], \optAtt [\vFive] [\fname], \optAtt [\vFive] [\lname]  } \\
%&(\chc [ \fModel_2 ] {\pi_{\empno, \sex, \birthdate, \optAtt [\vFour ] [\name], \optAtt [\vFive] [\fname], \optAtt [\vFive] [\lname]} \empbio, \empRel  })
%\end{align*}
%}
Similarly, the projection in the query 
\ensuremath{\vQ_7 = \pi_{\name, \fname} (\mathit{subq}_7)}
where 
\ensuremath{
\mathit{subq}_7 = \chc [ \vFour] {\pi_\name (\vQ_6), \pi_\fname (\vQ_6)}
}
is written over 
\ensuremath{\vSch_2} and it 
%\centerline{
%\ensuremath{
%\vQ_6 =
%\pi_{\name, \fname} \mathit{subq}_6
%} 
%}}
does not repeat the presence conditions of attributes from its \ensuremath{\mathit{subq}_7}'s type.
The query
%\centerline{
\ensuremath{
\vQ_8 =
\pi_{\optAtt [\vFour ] [\name],\optAtt [\neg \vFour] [\fname]} (\mathit{subq}_7)
%\chc [ \vFour] {\pi_\name \vQ_5, \pi_\fname \vQ_5}
}
%}
makes the annotations of projected attributes \emph{explicit} with respect to both 
the v-schema \ensuremath{\vSch_2} and its subquery's type.
%\TODO {give an example, schema: R(A,B), query: $\pi_{A,B} (F<\pi_A R, \pi_B R>)$
%becomes $\pi_{A^F, B^{\neg F}} ...$}
%The variation encoded in variational queries can
%be more restrictive or more loose than v-schema variation without violating them.
Although relieving the user from explicitly repeating variation makes VRA easier to use, 
queries still have to state variation explicitly to avoid losing information when 
decoupled from the schema.
%We do this by defining a function, 
%\ensuremath {\constrain \vQ}, with type \ensuremath{ \qSet \to \vSchSet \to \qSet
%},
%that \emph{explicitly annotates a query \vQ\ given the underlying schema \vSch}.
We do this by defining the function 
\ensuremath {\constrain \vQ : \qSet \totype \vSchSet \totype \qSet
},
that \emph{explicitly annotates a query \vQ\ with the  schema \vSch}.
%Note that \ensuremath {\constrain \vQ} needs to take the underlying schema as
%an input since it is using the type system (which relies on the schema) as a helper function.
The explicitly annotating query function, 
formally defined in \figref{constrain}, 
conjoins attributes and relations
presence conditions with the corresponding annotations in the query 
and wraps subqueries in a choice when needed. 
Note that, $\vQ_8$ and $\vQ_6$ are the result of $\constrain [\vSch_2] {\vQ_7}$
and $\constrain [\vSch_2] {\vQ_1}$, respectively, after simplification~\footnote{More specifically,
they are simpilified using rules defined in \figref{var-min}}.
%Queries $\vQ_7$ and $\vQ_5$ are examples of applying the 
%explicitly annotation function to queries $\vQ_6$ and $\vQ_1$, respectively,
%after simplifying them.
%\exref{constrain} illustrates how the constrain function transforms queries
%and allows users to be more flexible with their queries. 

\section{Explicitly Annotating Queries}
\label{sec:constrain}

%\point{type system allows the ql to be flexible and usable.}
%The type system is designed s.t. it relieves the user from necessarily incorporating
%the v-schema variability into their queries as long as the variational queries variability
%does not violate the v-schema, 
Variational queries do not need to repeat information that can be inferred from the v-schema
or the type of a query.
%
For example, the query \ensuremath{\vQ_1} shown in \exref{vq-specific} 
does not contradict the schema and
thus is type correct. However,
 it does not include the presence conditions of attributes and the relation encoded in
the schema while \ensuremath{\vQ_6} repeats this information:\\
%
\centerline{
\ensuremath{
\vQ_6 =
\pi_{\optAtt [\vFour \vee \vFive] [\empno], \optAtt [\vFour] [\name], \optAtt [\vFive] [\fname], \optAtt [\vFive] [\lname]  } \left(\chc [\dimMeta_2] {\empbio, \empRel} \right)}}.

%\pi_{\optAtt [(\vFour \vee \vFive) \wedge \fModel_2] [\empno], \optAtt [\vFour \wedge \fModel_2] [\name], \optAtt [\vFive \wedge \fModel_2] [\fname], \optAtt [\vFive \wedge \fModel_2] [\lname]  } \empbio}}.
%

%\NOTE{
%This is the unsimplified version:
%\begin{align*}
%\VVal {\vQ_5} &= 
%\pi_{\optAtt [\vFour \vee \vFive] [\empno], \optAtt [\vFour] [\name], \optAtt [\vFive] [\fname], \optAtt [\vFive] [\lname]  } \\
%&(\chc [ \fModel_2 ] {\pi_{\empno, \sex, \birthdate, \optAtt [\vFour ] [\name], \optAtt [\vFive] [\fname], \optAtt [\vFive] [\lname]} \empbio, \empRel  })
%\end{align*}
%}
Similarly, the projection in the query 
\ensuremath{\vQ_7 = \pi_{\name, \fname} (\mathit{subq}_7)}
where 
\ensuremath{
\mathit{subq}_7 = \chc [ \vFour] {\pi_\name (\vQ_6), \pi_\fname (\vQ_6)}
}
is written over 
\ensuremath{\vSch_2} and it 
%\centerline{
%\ensuremath{
%\vQ_6 =
%\pi_{\name, \fname} \mathit{subq}_6
%} 
%}}
does not repeat the presence conditions of attributes from its \ensuremath{\mathit{subq}_7}'s type.
The query
%\centerline{
\ensuremath{
\vQ_8 =
\pi_{\optAtt [\vFour ] [\name],\optAtt [\neg \vFour] [\fname]} (\mathit{subq}_7)
%\chc [ \vFour] {\pi_\name \vQ_5, \pi_\fname \vQ_5}
}
%}
makes the annotations of projected attributes \emph{explicit} with respect to both 
the v-schema \ensuremath{\vSch_2} and its subquery's type.
%\TODO {give an example, schema: R(A,B), query: $\pi_{A,B} (F<\pi_A R, \pi_B R>)$
%becomes $\pi_{A^F, B^{\neg F}} ...$}
%The variation encoded in variational queries can
%be more restrictive or more loose than v-schema variation without violating them.
Although relieving the user from explicitly repeating variation makes VRA easier to use, 
queries still have to state variation explicitly to avoid losing information when 
decoupled from the schema.
%We do this by defining a function, 
%\ensuremath {\constrain \vQ}, with type \ensuremath{ \qSet \to \vSchSet \to \qSet
%},
%that \emph{explicitly annotates a query \vQ\ given the underlying schema \vSch}.
We do this by defining the function 
\ensuremath {\constrain \vQ : \qSet \totype \vSchSet \totype \qSet
},
that \emph{explicitly annotates a query \vQ\ with the  schema \vSch}.
%Note that \ensuremath {\constrain \vQ} needs to take the underlying schema as
%an input since it is using the type system (which relies on the schema) as a helper function.
The explicitly annotating query function, 
formally defined in \figref{constrain}, 
conjoins attributes and relations
presence conditions with the corresponding annotations in the query 
and wraps subqueries in a choice when needed. 
Note that, $\vQ_8$ and $\vQ_6$ are the result of $\constrain [\vSch_2] {\vQ_7}$
and $\constrain [\vSch_2] {\vQ_1}$, respectively, after simplification~\footnote{More specifically,
they are simpilified using rules defined in \figref{var-min}}.
%Queries $\vQ_7$ and $\vQ_5$ are examples of applying the 
%explicitly annotation function to queries $\vQ_6$ and $\vQ_1$, respectively,
%after simplifying them.
%\exref{constrain} illustrates how the constrain function transforms queries
%and allows users to be more flexible with their queries. 

\input{formulas/constrainVQbySch}

\begin{theorem}
\label{thm:expl-same-type}
If the query \vQ\ has the type \vType\ then its explicitly annotated counterpart has the same type \vType, i.e.: \\
%
\centerline{
\ensuremath{%\raggedleft
\envWithoutVctx {\vQ} {\vType} \Rightarrow \envWithoutVctx {\constrain \vQ} {\VVal \vType} \textit{ and } \vType \equiv {\VVal \vType}
}}
%
This shows that the type system applies the schema to the type of a query although it does not apply it to the query. 
The \emph{type equivalence} is variational set equivalence, defined 
in \figref{vset}, for normalized variational sets of attributes.
%\footnote{We proved this theorem in the Coq proof assistant. The encoding of the theorem and the proof can be found in second author's MS thesis~\cite{FaribaThesis}.}.
\end{theorem}

We encode and prove \thmref{expl-same-type} in the Coq proof assistant~\cite{FaribaThesis}.
We also illustrate the application of \thmref{expl-same-type} to queries
\ensuremath{\vQ_1} and \ensuremath{\vQ_6}.
%
\exref{type} explained how \ensuremath{\vQ_1}'s type is generated step-by-step.
The variation context and underlying schema are
the same and the subquery \empbio\ has the same type. 
The projected attribute set annotated with the variation context is:
\ensuremath{
\vAttList_2 =  \{\annot [\vFour \vee \vFive] \empno, }
\ensuremath{ 
\optAtt [\vFour] [\name], \optAtt [\vFive] [\fname], \optAtt [\vFive] [\lname]\}^{\dimMeta_2}}, which is clearly subsumed by \ensuremath{\vAttList_\empbio}, thus, 
%the type of \empbio, \vAttList, and
its intersection with \ensuremath{\vAttList_\empbio} annotated
with the presence condition of \ensuremath{\vAttList_\empbio} is itself,
hence, \ensuremath{\vAttList_{\vQ_1} \equiv \vAttList_{\vQ_6}}.
%which makes it obvious that \ensuremath{\vAttList_{\vQ_1} \equiv \vAttList_{\vQ_6}}.
%\end{example}

\begin{theorem}
\label{thm:expl-same-type}
If the query \vQ\ has the type \vType\ then its explicitly annotated counterpart has the same type \vType, i.e.: \\
%
\centerline{
\ensuremath{%\raggedleft
\envWithoutVctx {\vQ} {\vType} \Rightarrow \envWithoutVctx {\constrain \vQ} {\VVal \vType} \textit{ and } \vType \equiv {\VVal \vType}
}}
%
This shows that the type system applies the schema to the type of a query although it does not apply it to the query. 
The \emph{type equivalence} is variational set equivalence, defined 
in \figref{vset}, for normalized variational sets of attributes.
%\footnote{We proved this theorem in the Coq proof assistant. The encoding of the theorem and the proof can be found in second author's MS thesis~\cite{FaribaThesis}.}.
\end{theorem}

We encode and prove \thmref{expl-same-type} in the Coq proof assistant~\cite{FaribaThesis}.
We also illustrate the application of \thmref{expl-same-type} to queries
\ensuremath{\vQ_1} and \ensuremath{\vQ_6}.
%
\exref{type} explained how \ensuremath{\vQ_1}'s type is generated step-by-step.
The variation context and underlying schema are
the same and the subquery \empbio\ has the same type. 
The projected attribute set annotated with the variation context is:
\ensuremath{
\vAttList_2 =  \{\annot [\vFour \vee \vFive] \empno, }
\ensuremath{ 
\optAtt [\vFour] [\name], \optAtt [\vFive] [\fname], \optAtt [\vFive] [\lname]\}^{\dimMeta_2}}, which is clearly subsumed by \ensuremath{\vAttList_\empbio}, thus, 
%the type of \empbio, \vAttList, and
its intersection with \ensuremath{\vAttList_\empbio} annotated
with the presence condition of \ensuremath{\vAttList_\empbio} is itself,
hence, \ensuremath{\vAttList_{\vQ_1} \equiv \vAttList_{\vQ_6}}.
%which makes it obvious that \ensuremath{\vAttList_{\vQ_1} \equiv \vAttList_{\vQ_6}}.
%\end{example}

\begin{theorem}
\label{thm:expl-same-type}
If the query \vQ\ has the type \vType\ then its explicitly annotated counterpart has the same type \vType, i.e.: \\
%
\centerline{
\ensuremath{%\raggedleft
\envWithoutVctx {\vQ} {\vType} \Rightarrow \envWithoutVctx {\constrain \vQ} {\VVal \vType} \textit{ and } \vType \equiv {\VVal \vType}
}}
%
This shows that the type system applies the schema to the type of a query although it does not apply it to the query. 
The \emph{type equivalence} is variational set equivalence, defined 
in \figref{vset}, for normalized variational sets of attributes.
%\footnote{We proved this theorem in the Coq proof assistant. The encoding of the theorem and the proof can be found in second author's MS thesis~\cite{FaribaThesis}.}.
\end{theorem}

We encode and prove \thmref{expl-same-type} in the Coq proof assistant~\cite{FaribaThesis}.
We also illustrate the application of \thmref{expl-same-type} to queries
\ensuremath{\vQ_1} and \ensuremath{\vQ_6}.
%
\exref{type} explained how \ensuremath{\vQ_1}'s type is generated step-by-step.
The variation context and underlying schema are
the same and the subquery \empbio\ has the same type. 
The projected attribute set annotated with the variation context is:
\ensuremath{
\vAttList_2 =  \{\annot [\vFour \vee \vFive] \empno, }
\ensuremath{ 
\optAtt [\vFour] [\name], \optAtt [\vFive] [\fname], \optAtt [\vFive] [\lname]\}^{\dimMeta_2}}, which is clearly subsumed by \ensuremath{\vAttList_\empbio}, thus, 
%the type of \empbio, \vAttList, and
its intersection with \ensuremath{\vAttList_\empbio} annotated
with the presence condition of \ensuremath{\vAttList_\empbio} is itself,
hence, \ensuremath{\vAttList_{\vQ_1} \equiv \vAttList_{\vQ_6}}.
%which makes it obvious that \ensuremath{\vAttList_{\vQ_1} \equiv \vAttList_{\vQ_6}}.
%\end{example}
\section{Variation-Minimization Rules}
\label{sec:var-min}

%
%\maybeAdd{add $\vQ_6$ is simplified of  $\VVal \vQ_6$ because of rule application blah blah.}
%\maybeAdd{add example + more rules + point out interesting ones}
%
VRA is flexible since an information need can be represented via multiple
variational queries as demonstrated in \exref{vq-specific} and \exref{vq-same-intent-mult-vars}.
It allows users to incorporate their personal taste and task requirements
into variational queries they write by 
having different levels of variation. For example, consider the explicitly annotated query
\ensuremath{\vQ_6} 
in \secref{constrain}.
%\ensuremath {
\[
\vQ_6 =
\pi_{\optAtt [\vFour \vee \vFive] [\empno], \optAtt [\vFour] [\name], \optAtt [\vFive] [\fname], \optAtt [\vFive] [\lname]  } \left( \chc [\fModel_2] {\empbio, \empRel}\right)
\]
%}.
%\vQ_5 =  \pi_{\optAtt [\vFour \vee \vFive] [\empno], \optAtt [\vFour] [\name], \optAtt [\vFive] [\fname], \optAtt [\vFive] [\lname]  } \empbio}.
%from \exref{vq-specific}. 
To be explicit about the exact query that will be run for 
each variant 
%and knowing that 
%\ensuremath{
%\getPC \empbio = \vThree \vee \vFour \vee \vFive
%},
the query $\vQ_6$'s variation can be \emph{lifted up} by using choices, resulting in the query $\VVVal \vQ_6$.
%\ensuremath{
%\small
\[
\VVVal \vQ_6 = \chc [\vFour] {\pi_{\empno, \name} \empbio, 
\chc [\vFive] {\pi_{\empno, \fname, \lname} \empbio, \emp}} 
\]
%}.
While \ensuremath{\vQ_6} contains less redundancy \ensuremath{\VVVal \vQ_6}
is more comprehensible since the variants are explicitly stated in the dimension of the choice. 
Thus, \emph{supporting multiple levels of variation 
creates a tension between reducing redundancy and maintaining comprehensibility.}

We define \emph{variation minimization} rules in \figref{var-min} that are syntactic and 
preserve the semantics.
% and include 
%interesting ones in \secref{var-min}.
Pushing in variation into a query, i.e., applying rules left-to-right, 
reduces redundancy
% and improves performance
while lifting them up, i.e., applying rules right-to-left, 
makes a query more understandable. 
When applied left-to-right, the rules are terminating since the scope of variation 
%always gets smaller.
monotonically decreases in size.
%
%\revised{
%Additionally, these rules can be used to simplify queries after
%explicitly annotating them with a schema. For example, the first rule in \figref{var-min}
%is used to simplify the query \ensuremath{\constrain [\vSch_2] {\vQ_1}}, introduced in \secref{var-pres},
% which resulted
%in \ensuremath{\vQ_6}.}


\begin{figure}
\textbf{Choice Distributive Rules:}
\begin{alignat*}{1}
\small
%-- f<? l? q?, ? l? q?> ? ? (f<l?, l?>) f<q?, q?>
%\inferrule
%{}
%\chc {\pi_{\vAttList_1} \vQ_1, \pi_{\vAttList_2} \vQ_2 } 
%&\equiv
%\pi_{\chc {\vAttList_1, \vAttList_2}} \chc {\vQ_1, \vQ_2}\\
%-- f<? l? q?, ? l? q?> ? ? ((l??), (l? \^�f )) f<q?, q?>
%\inferrule
%{}
\chc {\pi_{\vAttList_1} \vQ_1, \pi_{\vAttList_2} \vQ_2}
&\equiv
\pi_{\annot \vAttList_1, \annot [\neg \dimMeta] \vAttList_2} \chc {\vQ_1, \vQ_2}\\
%-- f<? c? q?, ? c? q?> ? ? f<c?, c?> f<q?, q?>
%\inferrule
%{}
\chc {\sigma_{\vCond_1} \vQ_1, \sigma_{\vCond_2} \vQ_2} 
&\equiv
\sigma_{\chc {\vCond_1, \vCond_2}} \chc {\vQ_1, \vQ_2}\\
%-- f<q? � q?, q? � q?> ? f<q?, q?> � f<q?, q?>
%\inferrule
%{}
\chc {\vQ_1 \times \vQ_2, \vQ_3 \times \vQ_4}
&\equiv
\chc {\vQ_1, \vQ_3} \times \chc {\vQ_2, \vQ_4}\\
%-- f<q? ?\_c? q?, q? ?\_c? q?> ? f<q?, q?> ?\_(f<c?, c?>) f<q?, q?>
%\inferrule
%{}
\chc {\vQ_1 \Join_{\vCond_1} \vQ_2, \vQ_3 \Join_{\vCond_2} \vQ_4}
&\equiv
\chc {\vQ_1, \vQ_3} \Join_{\chc {\vCond_1, \vCond_2}} \chc {\vQ_2, \vQ_4}\\
%-- f<q? ? q?, q? ? q?> ? f<q?, q?> ? f<q?, q?>
%\inferrule
%{}
\chc {\vQ_1 \circ \vQ_2, \vQ_3 \circ \vQ_4}
&\equiv
\chc {\vQ_1, \vQ_3} \circ \chc {\vQ_2, \vQ_4}
%-- f<q? ? q?, q? ? q?> ? f<q?, q?> ? f<q?, q?>
%\inferrule
%{}
%{-}
\end{alignat*}

\medskip
\textbf{CC and RA Optimization Rules Combined:}
\begin{alignat*}{1}
\small
%-- f<? (c? ? c?) q?, ? (c? ? c?) q?> ? ? (c? ? f<c?, c?>) f<q?, q?>
%\inferrule
%{}
\chc {\sigma_{\vCond_1 \wedge \vCond_2} \vQ_1, \sigma_{\vCond_1 \wedge \vCond_3} \vQ_2}
&\equiv
\sigma_{\vCond_1 \wedge \chc {\vCond_2, \vCond_3}} \chc {\vQ_1, \vQ_2}\\
%-- ? c? (f<? c? q?, ? c? q?>) ? ? (c? ? f<c?, c?>) f<q?, q?>
%\inferrule
%{}
\sigma_{\vCond_1} \chc {\sigma_{\vCond_2} \vQ_1, \sigma_{\vCond_3} \vQ_2}
&\equiv
\sigma_{\vCond_1 \wedge \chc {\vCond_2, \vCond_3}} \chc {\vQ_1, \vQ_2}\\
%-- f<q? ?\_(c? ? c?) q?, q? ?\_(c? ? c?) q?> ? ? (f<c?, c?>) (f<q?, q?> ?\_c? f<q?, q?>)
%\inferrule
%{}
\chc {\vQ_1 \Join_{\vCond_1 \wedge \vCond_2} \vQ_2, \vQ_3 \Join_{\vCond_1 \wedge \vCond_3} \vQ_4}
&\equiv
\sigma_{\chc {\vCond_2, \vCond_3}} \left( \chc {\vQ_1, \vQ_3} \Join_{\vCond_1} \chc {\vQ_2, \vQ_4} \right)
\end{alignat*}

\caption{Some of variation minimization rules.}
\label{fig:var-min}
\end{figure}

\section{Variational Query Language Properties}
\label{sec:vqlprop}

\TODO{prop. show for vra.}



%if time allows have a subsection for properties of the equivalnece of dent sem and ra + accum


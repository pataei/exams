%In this thesis I present the variational database, a formal framework for representing variation 
%in relational databases, the variational relational algebra, a query language for extracting
%information from the variational database that accounts explicitly for variation in the query
%language.
In this thesis, I present the variational database management system, a formal framework and its
implementation for representing variation in relational databases and managing variational
information needs. 
%
A variational database is intended to support any kind of variation in a database.
Specific kinds of variation in databases have already been studied and are well-supported,
for example, schema evolution systems address the variation of a database's schema
 over time and data integration systems address  variation caused by accessing data 
from multiple data sources simultaneously. 
%, both
%the existing instances and more importantly newly arisen instances which could be a 
%combination of existing instances.
%
However, many other kinds of variation in databases arise in practice, and different 
kinds of variation often interact, but these scenarios are not well-supported by the existing work.
%However, due to new business requirements new kinds of variation in databases raises and
%they are not addressed by existing work. Even though some of the new kinds of variation
%in databases are the result of combining well-studied kinds of variation, their solutions
%cannot address the new kind of variation.
%
For example, neither the schema evolution systems nor the database integration systems
can address variation that arises when data sources combined in one database
evolve over time.
%For example, existing instances of variation in databases are when a single database evolves
%over time and when data from different sources is combined. While there are context-specific 
%solutions to both problems neither solution can address when data sources combined in one
%database evolves over time. 

%
This thesis collects a large amount of work: 
%
It defines the variational database framework and the syntax and denotational
semantics of the variational relational algebra, a query language for variational databases.
% along with its type system and 
%semantic-preserving syntax-based equivalence rules. 
%
%It also defines
%the requirements of a generic variational database framework
%% that must be met 
%that makes the framework expressive enough to encode any kind of variation in databases.
%% in order to satisfy 
%%the variety of information needs users when variation appears in databases and 
%%variational information needs that must be satisfied in a generic variational database 
%%framework and 
%Additionally, it  \TODO{shows/proves} that the introduced framework satisfies all these
%needs. 
%
It presents two use cases of the variational database framework that are based on existing 
data sets and scenarios that are partially supported by existing techniques.
%
It presents the variational database management system which is the implementation of 
 variational databases and variational relational algebra as an abstract layer written in Haskell
on top of a traditional RDBMS. 
%
It also presents several theoretical results related to the framework and the query language, such 
as syntax-based equivalence rules that preserve the semantics of a query, 
a type system for ensuring that a variational query is well-formed with respect to the 
underlying variational schema, and a confluence property of the variational relational
algebra type system with respect to the relational algebra type system.
%and semantics. 
%\ensure{drop the var pres prop at semantics if you don't have it}
%Variation occurs in databases in many different forms and contexts. For example, a single database schema evolves over time, data from different sources may be combined, and the various configurations of a software product line (SPL) may have different data needs. 
%%
%While approaches have been developed to deal with many such scenarios, particularly in the fields of database evolution and data integration, there is no solution that treats variation as a general and orthogonal concern in databases. This is a problem when various kinds of variation intersect, such as during the evolution of a SPL.
%%
%Databases used in the same context that are intended to satisfy the same information need
%share commonalities while varying in some aspects as well. 
%% 
%For example, databases used to store information for a software system are intended to 
%satisfy the client's information need and they share some parts of the schema, however, 
%they still vary in some parts of the schema and most of the content.
%%
%That is, they are \emph{variants} of a conceptual hypothetical database that captures all 
%the \emph{variation} among them.
%%
%This pattern appears repeatedly in databases. Instances of variation occurring in databases
%are: schema evolution, database integration, database versioning, data extraction, and 
%software development either using software product lines approaches or not. 
%%
%While there are specialized approaches to some instances of variation in databases
%there are no generic solutions to manage variation in databases.
%
%In this thesis, we answer the question: ``can we abstract out this repeating pattern of 
%variation appearing in databases and bring the hypothetical database that captures all
%the variation among a number of databases to life?''
%%
%To this end, we formalize the database that captures all variation of a number of databases,
%called a \emph{variational database}, and a query language that allows interaction with
%the said database, called \emph{variational relational algebra}. 
%\TODO{maybe two languages, another: variational SQL, depends on my timing!}
%%
%We implement these concepts in \emph{Variational Database Management System (VDBMS)},
%demonstrate the feasibility of our concepts by developing two real-world use cases, and 
%examine the performance of VDBMS over our two use cases. 

\section{Variational Research}
\label{sec:varresearch}

In this section, we discuss some of the related variational research and other applications of it. 
%
The representation of variational schemas and variational tables is based on previous
work on variational sets~\cite{EWC13fosd}, which is part of a larger effort
toward developing safe and efficient variational data
structures~\cite{Walk14onward,MMWWK17vamos}. 
%
The representation of variation in variational queries is based on 
formula choice calculus~\cite{Walk13thesis, HW16fosd}.
%
The central motivation of work on
variational data structures is that many applications can benefit from
maintaining and computing with variation at
runtime~\cite{EW11gttse,CEW16ecoop}. Implementing SPL analyses
are an example of such an application, but there are many
more~\cite{Walk14onward}.The ability to maintain and query several
variants of a database at once extends the idea of computing with variation to
relational databases.

VDBMS is not the only system that extends an existing system with variation. 
\citet{Grasley18} expands on interpreters for variational imperative
languages by providing a formal operational semantics for the variational imperative
language VIMP.
\citet{Alkubaish20} investigates the use of 
algebraic effects to resolve the conflict between variation and side effects.
\citet{young20} add variation to SAT solvers and argue that the variational SAT solver
automates the interaction with the incremental solver.



%We introduce
%variational SAT solving, which differs from incremental SAT solving by accepting all related problems as a single variational input
%and returning all results as a single variational output. Our central
%idea is to make explicit the operations of incremental SAT solving, thereby encoding differences between related SAT problems as
%local points of variation. Our approach automates the interaction
%with the incremental solver and enables methods to automatically
%optimize sharing of the input. To evaluate our methods we construct a prototype variational SAT solver and perform an empirical
%analysis on two real-world datasets that applied incremental solvers
%to software evolution scenarios. We show, assuming a variational
%input, that the prototype solver scales better for these problems
%than naive incremental solving while also removing the need to
%track individual results


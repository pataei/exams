\section{Motivation and Impact}
\label{sec:mot}


%\structure{openning} Variation in databases appears abundantly in different
%forms and contexts.
Managing variation in databases is a perennial problem that appears in
different forms and
contexts~\cite{curateVdata,ALW21vamos,ready17cidr,clams16sigmod,datahub15cidr}.
%
In databases, variation arises when several database instances, which may
differ in schema, content, or constraints, conceptually represent the same
database.
%
%The problem of managing variation within a database is not new.
%
Existing work on database variation focuses on specific kinds of variation such
as schema evolution~\cite{SchEvolRA90McKenzie, schVersioning97Castro,
tempSchEvol91Ariav, tsql95Snodgrass, prima08Moon}, data
integration~\cite{dataIntegBook}, temporal database~\cite{tempDataMng, tempDBSurv, tempDBbook},
and database versioning~\cite{datasetVersioning,dbVersioning}. These works provide solutions
specific to the kinds of variation they address, but do not provide a
general-purpose solution to managing variation in databases.
%
This is a problem because new kinds of variation frequently arise and different
kinds of variation often interact, and the existing solutions are often
ill-suited to these scenarios.


% Unfortunately, some of these tools do not address all their user's needs.
% Furthermore, they are all \emph{variation-unaware}, i.e., they dismiss that
% they are addressing a specific kind of a bigger problem. Thus, not only they 
% cannot address a new kind of database variataion but they also cannot 
% address the interaction of different kinds of database variation since they
% assume that each kind of variation is \emph{isolated} from another
% kinds. This costs database researchers
% to develop a new system for every individual kind of data variation
% and forces developers and DBAs to use manually unsafe workaround.

%Specific kinds of this problem have been extensively studied including 
%schema evolution~\cite{SchEvolRA90McKenzie, 
%schVersioning97Castro, tempSchEvol91Ariav, tsql95Snodgrass, 
%prima08Moon}, 
%data integration~\cite{dataIntegBook}, 
%and database versioning~\cite{datasetVersioning,dbVersioning},
%where each instance has a context-specific tool.
%Lots of research has
%been done on some cases of this problem such as schema evolution,
%data integration, and database versioning; 
%resulting in well-supported context-specific systems.

%\point{Schema evolution is an instance of variation in databases
%that is well-supported.}
%Explains how schema evolution (which is unavoidable) 
%is an instance of variation in databases.
%And mentions some of the current solutions, emphasizing that
%they cannot address other instances of the problem.

Schema evolution is an example of a kind of variation in databases that is
well-supported~\cite{SchEvolRA90McKenzie, schVersioning97Castro,
tempSchEvol91Ariav, tsql95Snodgrass, prima08Moon}.
%
In the schema evolution scenario, a database's schema changes over time as the
database is extended or refactored to support new information needs, and the
goal is to automatically migrate data and queries to new versions of the
database. Thus, schema evolution is a kind of \emph{variation} over the
dimension of ``time'', and each version of the database can be viewed as a
\emph{variant} of the same database.
%
% Current solutions addressing schema
% evolution dismiss that it is a kind of variation, thus,
% they \emph{simulate} the effect of variation by
% relying on temporal nature of schema evolution and using
% timestamps~\cite{SchEvolRA90McKenzie, schVersioning97Castro, 
% tempSchEvol91Ariav, tsql95Snodgrass} 
% or keeping an external file of time-line history of 
% changes applied to the database~\cite{prima08Moon}. 


However, other kinds of database variation are less well supported. One example
% However, other scenarios require multiple variants of a database to be
% managed simultaneously (by analogy, we refer to this as variation in the
% dimension of \emph{space}).
arises in the context of \emph{software product lines} (SPLs)~\cite{fospl}. 
SPL approaches are used when the same code base is used to produce multiple different
variants of a software system, customized with different sets of features or
tailored for different clients. Naturally, the data requirements of each
variant of an SPL may differ~\cite{skrhas09DBIS}.
%
SPL researchers have developed various encodings that allow describing
variation in the data model among variants by annotating elements of the model
with features from the SPL~\cite{skrhas09DBIS,slrs12CAiSE,ad11varDataModel}.
These solutions can generate a database schema variant for each software
variant of the SPL. 
%
Unfortunately, these solutions address only variation in the data model but do not
extend to the level of data or queries. The lack of variation support in
queries leads to unsafe techniques such as encoding different variants of query
through string munging, while the lack of variation support in data precludes
testing with multiple variants of a database at once.

%\revised{
%However, these context-specific tools do not address all information
%needs of their users which costs database administrators and developers to use 
%manually unsafe workarounds.
%%
%%there are instances of the general problem that are
%%not well-studied, resulting in using manual approaches that burden database experts.
%Moreover, these tools consider instances \emph{isolated} from each other.
%Consequently, they 
%cannot address the same instance when it interacts with another instance and creates 
%a new instance of data variation.
%%For example, the variation of an evolving integrated database cannot be managed by
%%either a data integration system or schema evolution system.
%The lack of a \emph{variation-aware} database system costs database researchers
%to develop a new system for every individual instance of data variation. 
%In this paper, we provide a variation-aware database framework that explicitly expresses 
%variation in both the database and queries. Such a system can be instantiated for
%any instance of data variation or combination of them and it addresses all variational information needs safely.
%%each instance of data variation and manage any combination of data variation instances.
%%Additionally, it can potentially be optimized for an instance of data variation instead of developing a new system.
%}
%Moreover, different kinds of variation can interact\revised{, creating new kinds of data variation, }
%which cannot be addressed
%by current approaches due to lack of a general solution to managing 
%different kinds of variation in databases.
%In this paper, we provide a fundamental solution to managing
%variation in databases by considering variation 
%explicitly 
%%as 
%%a \emph{first-class citizen}
%in the framework, allowing for encoding different kinds of variation
%\revised{
%in both the database and queries.}

%\structure{the following paragraphs are funnel}
%\point{Schematic and content-level variation in DBs that conceptually 
%represent the same data.}
%\safespace{
%Variation in databases arises when multiple database instances 
%conceptually represent the same database, but, differ
%slightly either in their schema and/or content. 
%%These database
%%instances coexist in parallel.
%The variation in schema and/or content occurs in two \emph{dimensions}:
%time and space. Variation in space refers to different variants of database that
%coexist in parallel while variation in time refers to the evolution of 
%database, similar to variation observed in software~\cite{Thu19vv}. 
%Note that variation in a database can occur due to both dimensions
%at the same time.
%}



%%\point{Database-backed software produced by SPL is an instance
%%of variation in databases that is poorly managed in practice.}
%%Explains SPL briefly and how variation appears in databases used to
%%store data for software produced by SPL. As well as how poorly it is 
%%managed in practice.
%Database-backed software produced by software product line (SPL) 
%is an example variation in databases in space and is \revised{partially}
%supported. 
%SPL is an
%approach to developing and maintaining software-intensive systems 
%in a cost-effective, easy to maintain manner by accommodating variation
%in the software that is being reused~\cite{splBook}. 
%An SPL produces multiple software products for different business requirements and environments.
%%%The products of a SPL pertain to a
%%%common application domain or business goal. 
%%%They also have a common
%%%managed set of features that describe the specific need for a product. 
%%%They share 
%%%a common codebase which is used to produce a product with respect to its set of 
%%%selected (enabled) features
%%In SPL, a common codebase is shared and used to produce products w.r.t.
%%a set of selected (enabled) features~\cite{splBook}. 
%%% from the SPL feature set designed to describe
%%%specific needs for products~\cite{splBook}. 
%%Different products of an SPL typically have different
%%sets of enabled features or are tailored to run in different environments. 
%These
%differences impose different data requirements which creates variation in space 
%in the shared database used in the common codebase. 
%%Hence, 
%%each product has its own database variant.
%%For example, different legal
%%requirements often require tracking different data in products tailored for use
%%in different countries or regions~\cite{splBook}.
%%Different data requirements results in different desired data which 
%%creates database variants. 
%%
%%
%%
%\revised{
%Since the existing data variation models developed in the databases community
%do not align with all of the variation scenarios that arise in software, SPL
%researchers have identified the need for more general encodings of variation in
%data models and database schema.
%%
%%Since the variation in this case is mainly exclusion/inclusion of attributes/relations
%%for different variants of the software~\cite{ATW18poly}, researchers
%%%the database variants corresponding to software products differ mainly 
%%in their schema, a relation/attribute can either be included or excluded for a 
%%specific software product~\cite{ATW18poly}.
%%
% To this end, they 
% have developed
%encodings of data models that allow for arbitrary variation by annotating
%different elements of the model with features from the
%SPL~\cite{skrhas09DBIS,slrs12CAiSE,ad11varDataModel}.
%%
%However, these solutions address \emph{only} variation in the data model but do not
%extend to the level of the data or queries. The lack of variation support in
%queries leads to unsafe techniques such as encoding different variants of query
%through string munging, while the lack of variation support in data precludes
%testing with multiple variants of a database at once.
%}
%%The variation is in the form of exclusion/inclusion of tables/attributes based on
%%selected features for a product~\cite{ATW18poly}.
%%%
%%In practice, software systems produced by a SPL are accommodated with a database that
%%has all attributes and tables available for all variants-- a database with universal schema~\cite{ATW18poly}. 
%%Unfortunately, this approach is
%%inefficient, error-prone, and filled with lots of null values since not all attributes and tables
%%are valid for all variant products. A possible solution to this could be defining views on 
%%the universal database per software variant and write queries for each variant against its 
%%view~\cite{ATW18poly}.
%%However, this is burdensome, expensive, and costly to maintain since it 
%%requires developers to generate and maintain numerous view definitions
%%in addition to manually generating
%%and managing the mappings between views and the universal schema for each product.


%\point{Schema evolution meets SPL evolution and results in databases
%used in SPL that their schemas evolve over time.}Thu19vv, splEvolveBP14
%Mentions that SPL evolution is hot topic. Part of this evolution is database
%evolution. This is where two instances of managing variation in databases
%interact and even the well-supported systems for schema evolution cannot
%address it. 

Worse, still, is when multiple kinds of variation interact. Although structural
variation over time is well-supported via schema evolution, and structural
variation in ``space'' is partially supported by recent SPL research, there is
no support for the inevitable evolution of an SPL's variational data
model~\cite{dbSPLevolve}. Nor do existing approaches support variation across
all levels of a relational database: schema, queries, and content.
%
%In our previous work, 
In this thesis, we argue that schema evolution, SPL-like variation,
and other forms of database variation, such as data integration and database
versioning, are all facets of a similar problem that can be addressed by
\emph{treating variation as a general and orthogonal concern} in relational
databases~\cite{ATW17dbpl,ATW18poly,ALW21vamos, vldbArXiv}.
%
A significant advantage of treating different kinds of variation uniformly is
that it is easy to support the interaction of multiple kinds of variation, and
to coordinate variation in structure with corresponding variation in queries
and content. We illustrate these aspects throughout the thesis using a
motivating example described in \secref{motex}.

% The situation exacerbates even more when two kinds of variation
% interact and create a new kind of database variation: the evolution of
% the database used in development of software by an SPL.
% %in simultaneous development of multiple software variants.
% This is due to the evolution of software and its
% artifacts; an inevitable phenomena~\cite{dbSPLevolve}. 
% This is where even previous context-specific solutions like schema evolution
% tools fail.
%The lack of support for variation in databases exacerbates even more when
%software evolves. 
%Software evolution is inevitable, so is its artifacts evolution, 
%including databases~\cite{dbSPLevolve}.
%Variation in software development is unavoidable and 
%This is where two kinds of database  of managing 
%variation in databases (schema evolution and database-backed software 
%developed by SPL) interact \revised{and in fact they create a new kind of variation}.
% and variation occurs both in time and space. 
%While there are solutions to schema evolution
%they cannot adapt to a new situation because they only provide a solution to
%variation of databases in time and cannot encode the interaction of database variation in
%time and space.
%dismiss variation of databases in space, 
%i.e., they dismiss that database variants that have been created due to variation
%in time must coexist in parallel (a property of variation in 
%space~\cite{Thu19vv}). 
%We use this case as our motivating example in \secref{mot} and explain it 
%in more details.

%\structure{knowledge gap + challenge}
%\point{Reiterate the knowledge gap. Introduce the challenge.} 
%Reiterates the knowledge gap. Explains the challenge:
%The challenge then becomes encoding variation in databases
%that can model different instances and satisfy different specialists' needs 
%at different stages (like development, deployment, information extraction, etc).
%%Explains the challenge as incorporating variation in databases s.t. all 
%%database variants are gathered in one place. 
%Also, itemizes the users' 
%needs in such an environment: 
%1) Query some/all variants simultaneously and selectively.
%2) Track the original variant of a piece of data and the variation applied
%to it through a query.
%3) Deploy the database and its queries to a variant of it.

% As we have shown, 

% This is a good summary, but I don't think we need it here:
%\begin{comment}
To summarize, variation in databases is abundant and
inexorable~\cite{dbDecay16Stonebraker} and arises in many different contexts.
Existing solutions are specialized to address specific kinds of variation but
do not support the interaction of multiple kinds of variation, nor many
variation scenarios that cut across different levels of a database. The lack of
support for these scenarios negatively impact database administrators, data
analysts, and software developers~\cite{dbSPLevolve}. 
%\end{comment}


% are
% not variation-aware and 
%address specific kinds of 
%variation in databases and they 
%in time and space, however, they all consider only one of these dimensions and
% are all extremely tailored to a specific context. Consequently, they fail to address
% the interaction of their specific kind of database variation with other kinds.
%a new instance of variation appears.
%the two dimensions and different contexts interact. 

% Hence, the challenge becomes defining a variation-aware database that can 
% %offering a fundamental generic framework that 
% %incorporating variation in databases s.t. it 
% can model different kinds of variation
% in different contexts such that it satisfies different specialists' needs at different stages,
% e.g., development, information extraction, deployment, and testing. 
%
%We categorize these needs and explain them throughout the paper:
%\begin{enumerate}[leftmargin=*]
%\item [\textbf{N0}]Have access to all database variants at a given time
%\item [\textbf{N1}]Query multiple database variants simultaneously and selectively
%\item [\textbf{N2}]Keep track of which variants a piece of data belongs to and ensuring that 
%         it is maintained throughout a query
%\item [\textbf{N3}]Deploy one variant of the database and its associated queries.
%\end{enumerate}
%\structure{proposed solution}
%%\point{challenge: capturing/representing variability in the database + 
%%(put-all-together origin-tracking framework) + solution (contributions).}
%%\structure{point-last.}
%Therefore, 
%\emph{
%how can we incorporate variability in databases to allow for
%expressive queries over multiple
%database variants selectively and simultaneously while keeping
%track of the original variation of data and variation applied to it by
%a query
%%without violating data provenance
%%and 
%s.t. it is applicable in different contexts?}
%
%\revised{To account for variation explicitly, we use a \emph{variation space} 
%and propositional formulas of features, called \emph{feature expression}, 
%to refer to a subset of the space (\secref{encode-var})~\cite{ATW17dbpl}.}
%

\begin{comment}
In previous work, we have proposed the idea of \emph{variational databases},
which incorporate variation as a general and orthogonal concern in relational
databases~\cite{ATW17dbpl,ATW18poly}. In this paper, we significantly expand on
and realize this idea through the formalization and implementation of a
\emph{variational database management system} (VDBMS).
%
Specifically, this paper makes the following contributions:
%
\begin{itemize}[leftmargin=*]
\itemsep0em
% \item We define the requirements of a variational database framework, \secref{req}.

%[wide, labelwidth=!, labelindent=0pt, topsep=1pt]
%[leftmargin=*]
%\item To account for variation explicitly, we use a \emph{variation space} 
%and propositional formulas of features to refer to a subset of the space (\secref{encode-var})~\cite{ATW17dbpl}.
%\item We provide a framework to capture variation within a database using
%propositional formulas over  
%sets of features, called \emph{feature expressions}, following~\cite{ATW17dbpl}.
\item We provide a formal model of \emph{variational databases} (VDB), whose
structure are given by a \emph{variational schema} and whose content are given
by \emph{variational tables} (\secref{vdb}).
%both the structure (schema) and
%content (tuples) of the database, introducing \emph{variational schemas}, \secref{vsch}, 
%and \emph{variational tables}, \secref{vtab}, and together \emph{VDBs},
%\secref{vdb}.
%, satisfying \textbf{N0} and first part of \textbf{N2}.
\item We define \emph{variational relational algebra} as a query language for
VDB, a \emph{static type system} for ensuring that all variants of a query are
compatible with the corresponding variants of the VDB, and a set of
\emph{minimization rules} for reducing the size of queries (\secref{vq}).
%(\secref{vrel-alg}), its static type system, \secref{type-sys}, and
% \emph{variation-minimzation} rules, \secref{var-min}.
%a combination of relational algebra and 
%choice calculus~\cite{EW11tosem,Walk13thesis}, \secref{vrel-alg}.
%Users query a VDB by a \emph{variational query}, \secref{vq}.
%\item 
%%To make variational 
%\revised{
%%For more usability, w
%We define VRA's static type system, \secref{type-sys},
%and \emph{variation-minimzation} rules, \secref{var-min}.}
% queries more useable and easier to understand, respectively,
%by defining 
%a static type system, \secref{type-sys},
%and \emph{variation-minimiztion} rules, \secref{var-min}.
%to make it easier to understand and more useable. 
%rules to minimize 
%variation in v-query, \secref{var-min}, to provide better
%efficiency and usability. 
%This completes satisfiability of \textbf{N2}.
\item 
%To query a 
%variational database and receive clear results
%\revised{To interact with a VDB}
We implement a prototype of VDBMS as a layer on top of a traditional DBMS
(\secref{impl}) and evaluate this implementation on previously
developed~\cite{ALW21vamos} use cases (\secref{exp-disc}).
%,
%satisfying all four needs: \textbf{N0}-\textbf{N3}.
%\textbf{N1}, \textbf{N2}, and \textbf{N3}.
\end{itemize}
\end{comment}

%\point{evaluate VDBMS + its pay off}
%To evaluate VDBMS, 
%VDBMS provides new functionality to databases by accounting for 
%schematic and content-level variation. 
%\revised{ We use two previously developed use cases that 
%show the feasibility of VDB framework and illustrate that}
%\TODO{idk what yet!!\secref{exp-disc}}
%%We provide two use cases of how VDBMS can
%%model the schema evolution and SPL instances of variation in databases
%%for two real-world databases,
%%(employee database\footnote{\url{https://github.com/datacharmer/test_db}} 
%%and Enron email data corpus\footnote{\url{http://www.ahschulz.de/enron-email-data/}}),
%%\secref{exp-disc}. 
%%We also adopt practical queries from~\cite{prima08Moon} and~\cite{emailSPL},
%%respectively,
%%for our use cases to assess VDBMS, \secref{exp-qs}.
%\TODO{We hypothesize that it especially pays
%off when there is lots of variation in the database and the user needs to query
%many of the variants simultaneously, \secref{exp-disc}. }
%To examine our hypothesis,  
%we adopt two real-world databases 
%from two different contexts (schema evolution and SPL) and
%generate their counterpart VDBs (based on the changes applied to them).
%, assess
%the performance of VDBMS using a set of queries within each context, and
%finally analyze and discuss the results, \secref{exp-disc}.
%Our experiments show ...
%, elucidated more in \secref{exp-disc}.


\begin{comment}
\fromppr{vamos. incorporate these into mot.}
Just as variation is ubiquitous in software, it is also ubiquitous in the
relational databases that software systems rely on.
%
Database researchers have long studied different kinds of variation in
databases, such as through work on database
evolution~\cite{schVersioning97Castro,SchEvolRA90McKenzie,prima08Moon},
database versioning~\cite{datasetVersioning,dbVersioning}, and data
integration~\cite{dataIntegBook}.
%
However, work in the databases community does not identify \emph{variation} as
a general, orthogonal concern that arises in many different contexts.
%
This is a problem since it means that tools and techniques developed for one
kind of database variation cannot be easily reused for another. More concretely
for software developers, and especially software product line (SPL)
practitioners, the lack of a general representation of database variation means
that many kinds of variation that arise in software do not map cleanly onto
corresponding variation encodings in the databases they use.


% while the software product line (SPL) community has identified \emph{variation}
% or \emph{variability} as a general, orthogonal concern that arises in many
% different contexts, work on different flavors of data variation remains
% distinct.

In contrast, SPL researchers have invested significant effort in studying
\emph{variation} (or \emph{variability}) as a general phenomenon and concern in
software.
%
Although many kinds of software variation are possible, most can be roughly
organized into variation in \emph{time} or \emph{space}~\cite{Thu19vv}.
Variation in \emph{time} refers to the evolution of a system and is addressed
by revision control systems and configuration management~\cite{Dart91}, while
variation in \emph{space} refers to the simultaneous development and
maintenance of related systems with different feature sets and is the
traditional focus of research on SPLs~\cite{FOSPL16}.
%
Multiple lines of work in the SPL community have sought to develop general
purpose representations of variation in software, such as delta-oriented
programming~\cite{Schaefer10dop} and the choice calculus~\cite{EW11tosem},
among others. These can be used to unify both kinds of variation in software,
enabling reuse of analyses across dimensions and enabling new kinds of analyses
that consider variation in both time and space simultaneously~\cite{Thu19vv}.

Because the existing data variation models developed in the databases community
do not align with all of the variation scenarios that arise in software, SPL
researchers have identified the need for more general encodings of variation in
data models and database schema. To this end, they have developed
encodings of data models that allow for arbitrary variation by annotating
different elements of the model with features from the
SPL~\cite{skrhas09DBIS,slrs12CAiSE,ad11varDataModel}.
%
However, these solutions address only variation in the data model but do not
extend to the level of the data or queries. The lack of variation support in
queries leads to unsafe techniques such as encoding different variants of query
through string munging, while the lack of variation support in data precludes
testing with multiple variants of a database at once.
% This impacts DBAs, data scientists, and developers
% significantly~\cite{dbSPLevolve}.

%\begin{comment}
%Effectively managing variation is a fundamental challenge of software
%engineering. Research on CM and SPLs
%have developed numerous representations and strategies for effectively managing
%different kinds of variation in software. However, these solutions typically do
%not extend to managing variation in the artifacts and systems that software
%uses.
%%An especially tricky aspect is managing corresponding variation in
%%the external artifacts and services that a software system interacts with.
%%
%Here, we focus on \emph{relational databases}.
%%one kind of external artifact that is ubiquitous in
%%software but not well-supported by current approaches to managing software
%%variation: \emph{relational databases}.
%%
%Different variants of a software system have different information 
%needs~\cite{skrhas09DBIS}, which
%implies a corresponding need for variation in the structure and content of the
%relational databases that these systems use and rely on. This clearly results in 
%having variation in the database used to develop a SPL.
%%
%%While research on software product lines (SPLs) \cite{fospl} 
%%has led to a variety of representations and techniques for safely working with
%%many variants of a software system, these solutions don't extend to relational
%%databases.
%
%In a database-supported SPL, 
%typically a number of strategies are employed to
%accommodate the different information needs of different variants.
%%
%The first is that a different relational database may be \emph{specified and
%created per-variant}, according to the information needs of each
%variant~\cite{marco13featureAdaptSch}. 
%This approach is labor-intensive and difficult to maintain
%since changes need to be propagated across variants manually.
%%
%The second strategy is to define a single \emph{global schema that applies
%to all variants}~\cite{batini86dbSchIntegAnalysis}. 
%This strategy is more efficient to maintain compared to the previous approach
%but is still hard to maintain,
%especially in face of SPL evolution. Due to lack of separation of concerns
%and suboptimal traceability of requirements to database elements~\cite{skrhas09DBIS}
%it is also complex, hard to understand, and unscalable~\cite{slrs12CAiSE}. 
%Additionally, it suffers from design limitation and 
%error-proneness since parts of the schema will be irrelevant to each variant,
%resulting in losing database's integrity constraints~\cite{slrs12CAiSE}.
%%Irrelevant attributes are typically populated by NULL-values, which may later
%%be referenced since it is impossible to check or enforce that queries in each
%%variant use the database in a safe and consistent way.
%%
%The third strategy is to define a \emph{variable data model}~\cite{skrhas09DBIS, 
%slrs12CAiSE, ad11varDataModel} which models a database schema 
%(usually as an Entity-Relation model) with
%annotations of features from SPL to indicate their variable existence. 
%This approach addresses problems of the previous approach, however,
%it does not address the variation that appears in queries and data. 
%Thus, developers have to write the required information need as a
%query encoded as a string per variant. Not only this is labor-some but
%also due to the nature of queries being encoded as strings there is no
%static check to ensure that queries are type correct. 
%\end{comment}

\structure{use the following paragraph and items for contribution maybe.}
In previous work, we have developed \emph{variational databases (VDBs)} and
\emph{variational queries}~\cite{ATW18poly,ATW17dbpl} to encode general
variation in the representation and use of relational databases. Our work
extends ideas developed in the SPL community to 
% the creation, management, and querying of
relational databases.
%
Conceptually, a VDB represents potentially many different plain relational
databases at the same time. Similarly, a variational query represents
potentially many different queries, each one corresponding to a variant of the
VDB.
%
Together, VDBs and variational queries enable safely and efficiently working
with many variants of a relational database at once, and reliably integrating
the variants of a database with the corresponding variants of an SPL.
%
We are currently implementing these ideas in \emph{VDBMS}, a practical
implementation of variational databases as a lightweight wrapper on top of a
traditional relational database management system.


However, the generic and expressive approach of VDB in dealing with database
variation creates new complexity and costs which raises the question: Is
explicitly encoding variation in databases actually a good idea?
%
With this question in mind, in this paper, we:
%
\begin{itemize}
%
\item Show the feasibility of VDB by systematically generating two VDBs from
realistic scenarios of database variation in time and space (\secref{db}).
%
\item Illustrate the applicability of variational queries by encoding
information needs for the developed VDBs using scenarios described in the
literature (\secref{q}).
%
\item Discuss the tradeoffs of explicitly encoding variation in databases
(\secref{dis}).
% and future research directions (\secref{dis}).
%
\end{itemize}
\fromppr{end of from vamos. remember to read the commented out part too.}
\end{comment}

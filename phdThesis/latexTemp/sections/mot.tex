\section{Motivation and Impact}
\label{sec:mot}

\TODO{include a more elaborate version of vldb intro with maybe parts of vamos. 
also include the requirements here and elaborate more. you can also refer to sections
that address each requirement. finally introduce the mot ex as a concrete running
example throughout the thesis. also industry ex as an example that came up 
in conversation with industry contact.
}

%\structure{openning}
%Variation in databases appears abundantly in different forms and contexts.
Managing variation in databases is a perennial problem in database literature
and appears in different forms and 
contexts~\cite{curateVdata,ALW21vamos,ready17cidr,clams16sigmod,datahub15cidr}.
%
Variation in databases mainly arises when multiple database instances 
conceptually represent the same database, but, differ
in their schema, content, or constraints. 
%The problem
%of managing variation within a database is not new. 
Existing work on variation in databases addresses specific  kinds of 
variation in a context and proposes solutions specific to the context of the
variation, such as  
schema evolution~\cite{SchEvolRA90McKenzie, 
schVersioning97Castro, tempSchEvol91Ariav, tsql95Snodgrass, 
prima08Moon}, 
data integration~\cite{dataIntegBook}, 
and database versioning~\cite{datasetVersioning,dbVersioning}.

%
Unfortunately, some of these tools do not address all their user's needs.
Furthermore, they are all \emph{variation-unaware}, i.e., they dismiss that
they are addressing a specific kind of a bigger problem. Thus, not only they 
cannot address a new kind of database variataion but they also cannot 
address the interaction of different kinds of database variation since they
assume that each kind of variation is \emph{isolated} from another
kinds. This costs database researchers
to develop a new system for every individual kind of data variation
and forces developers and DBAs to use manually unsafe workaround.
%Specific kinds of this problem have been extensively studied including 
%schema evolution~\cite{SchEvolRA90McKenzie, 
%schVersioning97Castro, tempSchEvol91Ariav, tsql95Snodgrass, 
%prima08Moon}, 
%data integration~\cite{dataIntegBook}, 
%and database versioning~\cite{datasetVersioning,dbVersioning},
%where each instance has a context-specific tool.
%Lots of research has
%been done on some cases of this problem such as schema evolution,
%data integration, and database versioning; 
%resulting in well-supported context-specific systems.

%\point{Schema evolution is an instance of variation in databases
%that is well-supported.}
%Explains how schema evolution (which is unavoidable) 
%is an instance of variation in databases.
%And mentions some of the current solutions, emphasizing that
%they cannot address other instances of the problem.
For example, schema evolution is a kind of schematic variation in databases
over time
that is well-supported~\cite{SchEvolRA90McKenzie, 
schVersioning97Castro, tempSchEvol91Ariav, tsql95Snodgrass, 
prima08Moon}.
Changes applied to the schema over time are \emph{variation} 
in the database and every time the database evolves, a new
\emph{variant} is generated.
%which needs to coexists in parallel
%with other variants.
Current solutions addressing schema
evolution dismiss that it is a kind of variation, thus,
they \emph{simulate} the effect of variation by
relying on temporal nature of schema evolution and using
timestamps~\cite{SchEvolRA90McKenzie, schVersioning97Castro, 
tempSchEvol91Ariav, tsql95Snodgrass} 
or keeping an external file of time-line history of 
changes applied to the database~\cite{prima08Moon}. 
%These approaches only consider variation in time and do not
%%However, none of them 
%incorporate the time-based changes into
%the database directly, rather they \emph{simulate} the effect of these changes.
%,
%resulting in brittle systems.

Unlike schema evolution, some kinds of database variation are
partially supported. For example, while developing software for 
multiple clients simultaneously, an approach called software 
product line (SPL)~\cite{splBook}, a different database schema is 
required for each client  due to 
client's different business requirements and environments~\cite{skrhas09DBIS}.
% 
SPL researchers have developed
encodings of data models that allow for arbitrary variation by annotating
different elements of the model with features from the
SPL~\cite{skrhas09DBIS,slrs12CAiSE,ad11varDataModel}.
Thus, they can generate a database schema  \emph{variant} for 
each software \emph{variant} requested by a client and generated by the SPL. 
%
However, these solutions address \emph{only} variation in the data model but do not
extend to the level of the data or queries. The lack of variation support in
queries leads to unsafe techniques such as encoding different variants of query
through string munging, while the lack of variation support in data precludes
testing with multiple variants of a database at once.

%\revised{
%However, these context-specific tools do not address all information
%needs of their users which costs database administrators and developers to use 
%manually unsafe workarounds.
%%
%%there are instances of the general problem that are
%%not well-studied, resulting in using manual approaches that burden database experts.
%Moreover, these tools consider instances \emph{isolated} from each other.
%Consequently, they 
%cannot address the same instance when it interacts with another instance and creates 
%a new instance of data variation.
%%For example, the variation of an evolving integrated database cannot be managed by
%%either a data integration system or schema evolution system.
%The lack of a \emph{variation-aware} database system costs database researchers
%to develop a new system for every individual instance of data variation. 
%In this paper, we provide a variation-aware database framework that explicitly expresses 
%variation in both the database and queries. Such a system can be instantiated for
%any instance of data variation or combination of them and it addresses all variational information needs safely.
%%each instance of data variation and manage any combination of data variation instances.
%%Additionally, it can potentially be optimized for an instance of data variation instead of developing a new system.
%}
%Moreover, different kinds of variation can interact\revised{, creating new kinds of data variation, }
%which cannot be addressed
%by current approaches due to lack of a general solution to managing 
%different kinds of variation in databases.
%In this paper, we provide a fundamental solution to managing
%variation in databases by considering variation 
%explicitly 
%%as 
%%a \emph{first-class citizen}
%in the framework, allowing for encoding different kinds of variation
%\revised{
%in both the database and queries.}

%\structure{the following paragraphs are funnel}
%\point{Schematic and content-level variation in DBs that conceptually 
%represent the same data.}
%\safespace{
%Variation in databases arises when multiple database instances 
%conceptually represent the same database, but, differ
%slightly either in their schema and/or content. 
%%These database
%%instances coexist in parallel.
%The variation in schema and/or content occurs in two \emph{dimensions}:
%time and space. Variation in space refers to different variants of database that
%coexist in parallel while variation in time refers to the evolution of 
%database, similar to variation observed in software~\cite{Thu19vv}. 
%Note that variation in a database can occur due to both dimensions
%at the same time.
%}



%%\point{Database-backed software produced by SPL is an instance
%%of variation in databases that is poorly managed in practice.}
%%Explains SPL briefly and how variation appears in databases used to
%%store data for software produced by SPL. As well as how poorly it is 
%%managed in practice.
%Database-backed software produced by software product line (SPL) 
%is an example variation in databases in space and is \revised{partially}
%supported. 
%SPL is an
%approach to developing and maintaining software-intensive systems 
%in a cost-effective, easy to maintain manner by accommodating variation
%in the software that is being reused~\cite{splBook}. 
%An SPL produces multiple software products for different business requirements and environments.
%%%The products of a SPL pertain to a
%%%common application domain or business goal. 
%%%They also have a common
%%%managed set of features that describe the specific need for a product. 
%%%They share 
%%%a common codebase which is used to produce a product with respect to its set of 
%%%selected (enabled) features
%%In SPL, a common codebase is shared and used to produce products w.r.t.
%%a set of selected (enabled) features~\cite{splBook}. 
%%% from the SPL feature set designed to describe
%%%specific needs for products~\cite{splBook}. 
%%Different products of an SPL typically have different
%%sets of enabled features or are tailored to run in different environments. 
%These
%differences impose different data requirements which creates variation in space 
%in the shared database used in the common codebase. 
%%Hence, 
%%each product has its own database variant.
%%For example, different legal
%%requirements often require tracking different data in products tailored for use
%%in different countries or regions~\cite{splBook}.
%%Different data requirements results in different desired data which 
%%creates database variants. 
%%
%%
%%
%\revised{
%Since the existing data variation models developed in the databases community
%do not align with all of the variation scenarios that arise in software, SPL
%researchers have identified the need for more general encodings of variation in
%data models and database schema.
%%
%%Since the variation in this case is mainly exclusion/inclusion of attributes/relations
%%for different variants of the software~\cite{ATW18poly}, researchers
%%%the database variants corresponding to software products differ mainly 
%%in their schema, a relation/attribute can either be included or excluded for a 
%%specific software product~\cite{ATW18poly}.
%%
% To this end, they 
% have developed
%encodings of data models that allow for arbitrary variation by annotating
%different elements of the model with features from the
%SPL~\cite{skrhas09DBIS,slrs12CAiSE,ad11varDataModel}.
%%
%However, these solutions address \emph{only} variation in the data model but do not
%extend to the level of the data or queries. The lack of variation support in
%queries leads to unsafe techniques such as encoding different variants of query
%through string munging, while the lack of variation support in data precludes
%testing with multiple variants of a database at once.
%}
%%The variation is in the form of exclusion/inclusion of tables/attributes based on
%%selected features for a product~\cite{ATW18poly}.
%%%
%%In practice, software systems produced by a SPL are accommodated with a database that
%%has all attributes and tables available for all variants-- a database with universal schema~\cite{ATW18poly}. 
%%Unfortunately, this approach is
%%inefficient, error-prone, and filled with lots of null values since not all attributes and tables
%%are valid for all variant products. A possible solution to this could be defining views on 
%%the universal database per software variant and write queries for each variant against its 
%%view~\cite{ATW18poly}.
%%However, this is burdensome, expensive, and costly to maintain since it 
%%requires developers to generate and maintain numerous view definitions
%%in addition to manually generating
%%and managing the mappings between views and the universal schema for each product.


%\point{Schema evolution meets SPL evolution and results in databases
%used in SPL that their schemas evolve over time.}Thu19vv, splEvolveBP14
%Mentions that SPL evolution is hot topic. Part of this evolution is database
%evolution. This is where two instances of managing variation in databases
%interact and even the well-supported systems for schema evolution cannot
%address it. 
The situation exacerbates even more when two kinds of variation
interact and create a new kind of database variation: the evolution of
the database used in development of software by an SPL.
%in simultaneous development of multiple software variants.
This is due to the evolution of software and its
artifacts; an inevitable phenomena~\cite{dbSPLevolve}. 
This is where even previous context-specific solutions like schema evolution
tools fail.
%The lack of support for variation in databases exacerbates even more when
%software evolves. 
%Software evolution is inevitable, so is its artifacts evolution, 
%including databases~\cite{dbSPLevolve}.
%Variation in software development is unavoidable and 
%This is where two kinds of database  of managing 
%variation in databases (schema evolution and database-backed software 
%developed by SPL) interact \revised{and in fact they create a new kind of variation}.
% and variation occurs both in time and space. 
%While there are solutions to schema evolution
%they cannot adapt to a new situation because they only provide a solution to
%variation of databases in time and cannot encode the interaction of database variation in
%time and space.
%dismiss variation of databases in space, 
%i.e., they dismiss that database variants that have been created due to variation
%in time must coexist in parallel (a property of variation in 
%space~\cite{Thu19vv}). 
We motivate this case through an example in \secref{mot}.
%We use this case as our motivating example in \secref{mot} and explain it 
%in more details.

%\structure{knowledge gap + challenge}
%\point{Reiterate the knowledge gap. Introduce the challenge.} 
%Reiterates the knowledge gap. Explains the challenge:
%The challenge then becomes encoding variation in databases
%that can model different instances and satisfy different specialists' needs 
%at different stages (like development, deployment, information extraction, etc).
%%Explains the challenge as incorporating variation in databases s.t. all 
%%database variants are gathered in one place. 
%Also, itemizes the users' 
%needs in such an environment: 
%1) Query some/all variants simultaneously and selectively.
%2) Track the original variant of a piece of data and the variation applied
%to it through a query.
%3) Deploy the database and its queries to a variant of it.
As we have shown, 
variation in databases is abundant and inexorable~\cite{dbDecay16Stonebraker};
impacts DBAs, data analysts, and developers significantly~\cite{dbSPLevolve}; 
and appears in different contexts. Current methods
% are
% not variation-aware and 
%address specific kinds of 
%variation in databases and they 
%in time and space, however, they all consider only one of these dimensions and
are all extremely tailored to a specific context. Consequently, they fail to address
the interaction of their specific kind of database variation with other kinds.
%a new instance of variation appears.
%the two dimensions and different contexts interact. 
Hence, the challenge becomes defining a variation-aware database that can 
%offering a fundamental generic framework that 
%incorporating variation in databases s.t. it 
can model different kinds of variation
in different contexts such that it satisfies different specialists' needs at different stages,
e.g., development, information extraction, deployment, and testing. 
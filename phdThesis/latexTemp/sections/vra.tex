\section{Variational Relational Algebra}
\label{sec:vrel-alg}

%\point{vra = cc + ra}
%Considering the variational nature of a VDB, to satisfy a user's information 
%need when extracting information, 
%we need a query language that not only considers the structure of 
%relational databases (such as SQL and relational algebra (RA)) but also 
%accounts for the variation encoded in the VDB. We achieve this by:
%1) picking relational algebra as our main query language and
%2) using \emph{choices}~\cite{Walk13thesis, EW11tosem} 
%and presence conditions to account for variation. 

To account for variation, VRA combines relational algebra (RA) with 
\emph{choices}~\cite{EW11tosem,HW16fosd,Walk13thesis}.
%\point{choice.}
Remember that a choice $\chc{\elem_1,\elem_2}$ consists of a feature expression \dimMeta, called
the \emph{dimension} of the choice, and 
two \emph{alternatives} $\elem_1$ and $\elem_2$. For a given configuration \config, 
the choice $\chc{\elem_1, \elem_2}$ can be replaced by $\elem_1$ if \dimMeta\
evaluates to \t\ under configuration \config, (i.e., \fSem{\dimMeta}),
or $\elem_2$ otherwise. 
% Choices allow a variational queries
% to encode variation in a structured and systematic manner. 

\begin{figure}
\begin{syntax}

\multicolumn{4}{l}{\textbf{Operators:}} \\[1ex]
\bullet
  &\eqq& \multicolumn{2}{l}{< \myOR \leq \myOR = \myOR \neq \myOR > \myOR \geq} \\
\circ
  &\eqq& \cup \myOR \cap \\[2ex]

\multicolumn{4}{l}{\textbf{Variational conditions:}} \\[1ex]
\vCond\in\vCondSet
  &\eqq&  \multicolumn{2}{l}{
          \bTag
   \myOR  \pAtt \bullet \cte
   \myOR  \pAtt \bullet \pAtt
   \myOR  \neg \vCond
   \myOR  \vCond \vee \vCond} \\
  &\myOR& \multicolumn{2}{l}{
          \vCond \wedge \vCond
   \myOR \chc{\vCond,\vCond}} \\[2ex]

\multicolumn{4}{l}{\textbf{Variational queries:}} \\[1ex]
\vQ\in\qSet
  &\eqq&  \vRel     & \textit{Relation}\\
  &\myOR& \vSel \vQ & \textit{Selection}\\
  &\myOR& \vPrj[\vAttList]{\vQ} & \textit{Projection}\\
  &\myOR& \chc{\vQ,\vQ} & \textit{Choice}\\
% &\myOR& \vQ \Join_\vCond \vQ & \textit{Variational Join}\\
  &\myOR& \vQ \times \vQ & \textit{Cartesian Product}\\
  &\myOR& \vQ \circ \vQ  & \textit{Set Operation}\\
% &\myOR& \vQ \backslash \vQ &\textit{Variational Set Difference}\\
  &\myOR& \empRel & \textit{Empty Relation}

\end{syntax}

\caption[Syntax of variational relational algebra]{Syntax of variational relational algebra.}
%\TODO{remember that
%you removed join (also removed it from query config def and constrain query 
%by schema). if you want use it just say it's a syntactic sugar.}
%$<, \leq, =, \neq, >, \geq$.
%$\circ$ denotes set operators: union and difference.}
\label{fig:v-alg-def}
\end{figure}



%\point{explain notation and VRA operations.}
The syntax of VRA is given in \figref{v-alg-def}.
%
The selection operation is similar to standard RA selection except
that the condition parameter is \emph{variational} meaning that it may contain
choices.
For example, the query 
\ensuremath{\sigma_{\chc {\vAtt_1=\vAtt_2,\vAtt_1=\vAtt_3}} (\vRel)}
selects a variational tuple \vTuple\ if it satisfies
the condition \ensuremath{\vAtt_1 = \vAtt_2} 
and  \ensuremath{\sat {\dimMeta \wedge \getPC \vTuple}}
or
if \ensuremath{\vAtt_1 = \vAtt_3} 
and \ensuremath{\sat {\neg \dimMeta \wedge \getPC \vTuple }}.
%
The projection operation is parameterized by a v-set of attributes, \vAttList. For
example,
the query $\pi_{\vAtt_1, \optAtt [\dimMeta] [\vAtt_2]} (\vRel)$
projects $\vAtt_1$ from relation \vRel\ unconditionally, and $\vAtt_2$ 
when \sat{\dimMeta}.
%
The choice operation enables combining two variational queries to be used in different
variants based on the dimension. In practice,
it is often useful to return information in some variants and nothing at all in
others. We introduce an explicit \emph{empty} query \empRel\ to facilitate
this. 
Similar to our definition of the empty query for relational algebra, for VRA we
also have: $\empRel=\vPrj[\set{}]{\vQ}$.
The empty query is used, for example, in 
\ensuremath{\vQ_2} in \exref{vq-specific}. 
%The set operations between queries are v-set operations defined in \secref{vset}.
The rest of VRA's operations are similar to RA, where all set operations
(union, intersection, and product) are changed to the corresponding
variational set operations defined in \secref{vset}.
%\secref{vlist-vset}.
%
%\remember{
%In examples, we also use a join operation with a variational condition,
%$\vQ_1\bowtie_\vCond\vQ_2$, which is syntactic sugar for
%$\sigma_\vCond(\vQ_1\times\vQ_2)$.}


Our implementation of VRA also provides mechanisms for renaming queries and
qualifying attributes with relation/sub\-query names. These features are needed
to support self joins and to project attributes with the same name in different
relations. However, for simplicity, we omit these features from the formal
definition in this paper.


%A query can simply 
%refer to a relation, filter tuples based on a variational condition 
%(which is a relational condition with choices of two conditions), and
%project a variational list of attributes. Besides production of two queries and
%set operations, VRA allows for a choice of two variational queries. This demands an
%\emph{empty} query since an alternative of a choice can very well inquire 
%no information at all. 
%For example, the query $\chc {\vQ_1, \}$
%

%\subsubsection{Running a Variational query on a VDB Results in a Variational table}
%\label{sec:run-vq-get-vtab}
%A variational query systematically represents a set of relational query variants associated to their
%corresponding database variants. Hence, intuitively the user expects to 
%get such variation in their result as well. 

The result of a variational query is a variational table with the reserved relation name $\mathit{result}$.
%
For example, assume that variational tuples $\annot[\fOne]{(1,2)}$ and $\annot[\neg
f_3]{(3,4)}$ belong to a variational relation $\vRel(\vAtt_1,\vAtt_2)$, which is the only
relation in a VDB with the trivial feature model \t.
%
The query $\chc[f_3]{\pi_{\optAtt[f_2][\vAtt_1]}(\vRel),\empRel}$ returns a
variational table with relation schema $\annot[f_3]{\mathit{result}(\annot[f_2]{a_1})}$,
which indicates that the result is only non-empty when $f_3$ is \t\ and that the
result includes attribute $a_1$ when $f_2$ is \t. 
%\secref{type-sys} defines a
%type system that yields the relation schema for any well-formed query.
%
The content of the result relation for the example query is a single variational tuple
$\annot[f_1]{(1)}$. The tuple $\annot[\neg f_3]{(3)}$ is not included since the
projection occurs in the context of a choice in $f_3$, which is incompatible
with the presence condition of the tuple, i.e., $\unsat{f_3 \wedge\neg f_3}$.
This illustrates how choices can effectively filter the tuples in a VDB based
on the dimension.
%, satisfying the second part of \nOne.
%
% Although there is no need to update the presence condition of the returned
% tuples, yet choices can filter the returned variational tuples.
%
% Note that here the value \ensuremath{1}
% of attribute \ensuremath{\vAtt_1} is present in VDB variants where 
% \ensuremath{\sat {\A \wedge \B \wedge \C}} although the presence 
% condition of the returned variational tuple does not have to state this condition
% since 
%
% overall presence 
%condition and the presence conditions of attributes and tuples are
%restricted by the variation enforced by the query.
%
%Note that the presence condition of tuples, attributes, and the return relation
%is restricted by the variation enforced by the query. 
% correct this so that you don't conjunct the pc of relation and clarify that it's relations's pc and not the attributes. although the conjunction should be satisfiable.}
%
%
%%Hence, VRA is more expressive than RA 
%%because it can encode variational queries.
%The variational nature allows users to write interesting queries in many ways:
%1) to express their variational information need or to filter returned tuples
%they can use annotations or 
%choices, \exref{vq-specific},
%2) to express the same intent over several database variants they can 
%use choices in queries or conditions, \exref{vq-same-intent-mult-vars},
%and 
%3) they can also use choices to express different intents over database variants.
%\TODO{Eric, should we drop the last since it creates messy results and isn't really useful?}.
%%The expressiveness of VRA satisfies \textbf{N1}, this is illustrated in 
%%\exref{vq-specific} and \exref{vq-same-intent-mult-vars}.
%%Interestingly, VRA's expressiveness enables users to express 
%%their information need more specifically by stating the exact condition
%%under which an information need is inquired. \exref{vq-specific} illustrates this.
%%It also allows users to express the same intent over several database 
%%variants
% \NOTE{
% To express the variational information need or to filter returned tuples
% users can use annotations or choices. \exref{vq-specific} illustrates this.
% }
%
%The following example
\exref{vq-specific} illustrates
%, in the context of our running example, 
how
a variational query can be used to express variational information needs.

\begin{example}
\label{eg:vq-specific}
%VRA's expressiveness consequently facilitates expressing exactly the condition
%under which an information need is inquired. 
Assume a VDB with
\ensuremath{\fSet = \setDef {\vThree, \vFour, \vFive}}, 
and
the corresponding \empbio\ schema variants in \tabref{mot}. 
The variational schema for this VDB is:
%
\begin{align*}
\vSch_2 &=
\{\empbio (\empno, \sex, \birthdate,
\optAtt [\vFour] [\name], \optAtt [\vFive] [\fname],
 \optAtt [\vFive] [\lname] )\}^{\dimMeta_2}\\
& \hspace{-38pt} \textit{where } \dimMeta_2 = {\vThree \oplus \vFour \oplus \vFive}.
%\left(\vThree \wedge \neg \vFour \wedge \neg \vFive\right)
%  \vee \left(\vFour \wedge \neg \vThree \wedge \neg \vFive\right) 
%   \vee \left(\vFive \wedge \neg \vThree \wedge \neg \vFour\right)}.
\end{align*}
%
The user wants the employee ID numbers (\empno) and names for variants 
\set{\vFour} and \set{\vFive}.
The user needs to project the \name\ attribute 
for variant \set{\vFour}, the \fname\ and \lname\ attributes for variant
\set{\vFive}, and \empno\ attribute for both variants.
This can be expressed with the following variational query.
\[
\vQ_1 = \pi_{\optAtt [\vFour \vee \vFive] [\empno], \name, \fname, \lname} (\empbio)
\]
\end{example}

In the example, note that the user does not need to repeat the variability  encoded
in the variational schema in their query, that is, they do not need to annotate \name,
\fname, and \lname\ with \vFour, \vFive, and \vFive, respectively. We discuss
this in more detail in \secref{constrain}. $\vQ_1$
queries all three variants simultaneously although the returned results are
only associated with variants \vFour\ and \vFive\ due to the annotation of the
attribute \empno\ in the query and the presence conditions of the rest of the
projected attributes in the schema.
%
Yet, the query can be more simplified with a choice. $\vQ_2$ selects only two
out of the three variants explicitly:
%selecting only two out of the three variants can be written more
%explicitly in a query by using a choice:
$\vQ_2=\chc[\vFour\vee\vFive]{\pi_{\empno,\name,\fname,\lname}(\empbio),\empRel}$. 
%
Note that queries $\vQ_1$ and $\vQ_2$ return the same set of variational tuples since
neither returns tuples associated with variant \vThree, but their returned
variational tables have different presence conditions, thus, $\vQ_2$ filters out
tuples that belong to variant \vThree\ while $\vQ_1$ does not. We discuss this
\rewrite{rewrite ref}
more \exref{type}. 
%

%\NOTE{
%\revised{VRA has \revised{syntactic} equivalence rules, described in
%\secref{var-min}, that enable semantics-preserving transformations of queries
%similar to the transformation of $\vQ_1$ into $\vQ_2$ (and vice versa). These
%rules enable factoring commonality out of subqueries, among other
%transformations.}

%The next example 
 Expressing
the same intent over several database variants by a single query relieves the DBA from
maintaining separate queries for different variants or configurations of the
schema.
\exref{vq-same-intent-mult-vars} 
illustrates this point.
% by using choices.
%how a variational query can be used to express the same
%intent over several database variants using choices and conditions.

\begin{example}
\label{eg:vq-same-intent-mult-vars}
Assume a VDB with  \ensuremath{\fSet = \setDef{\vOne, \ldots, \vFive}}
and the corresponding \basic\ schema
variants in \tabref{mot}. The user wants to get all employee names across all
variants. They express this intent by the query $\vQ_3$:
%
\begin{align*}
\vQ_3 &= 
  \vOne\chcL
    (\pi_{\name}(\engemp)) \cup (\pi_{\name}(\othemp)) \\
 & \qquad,
    (\vTwo\vee\vThree)\chcL
      \pi_{\name}(\empacct) \\
 & \qquad\qquad,
      \chc[(\vFour\vee\vFive)]{\pi_{\name,\fname,\lname}\empbio, \emp}\chcR\chcR
\end{align*}
%
Since the variational schema enforces that exactly one of \vOne--\ \vFive\ be enabled, we
can simplify the query by omitting the final choice.
%
\begin{align*}
\vQ_4 &= 
  \vOne\chcL
    (\pi_{\name}(\engemp)) \cup (\pi_{\name}(\othemp)) \\
 & \qquad,
    \chc[(\vTwo\vee\vThree)]{
      \pi_{\name}(\empacct),
      \pi_{\name,\fname,\lname}(\empbio)}
\end{align*}
%
\end{example}

In principle, variational queries can also express arbitrarily different intents over
different database variants. However, we expect that variational queries are best used to
capture single (or at least related) intents that vary in their realization
since this is easier to understand and increases the potential for sharing in
both the representation and execution of a variational query.


% \subsubsection{VRA Semantics}
% \label{sec:vra-sem}

\begin{figure}
%\textbf{Configuration selection semantics of \vqsTxt:}
\begin{alignat*}{1}
\eeSem [] . &: \qSet \to \confSet \to \pQSet\\
%
\eeSem \vRel &= \orSem \vRel = \pRel\\
\eeSem {\vSel \vQ}  &= \vSel [\ecSem \vCond] {\eeSem \vQ}\\
%
\eeSem {\vPrj [\vAttList] \vQ} &= \vPrj [\olSem \vAttList] {\eeSem \vQ}\\
%
\eeSem {{\vQ_1} \times {\vQ_2}} &= \eeSem {\vQ_1} \times \eeSem {\vQ_2}\\
%
%\eeSem {{\vQ_1} \Join_\vCond {\vQ_2}} &= \eeSem {\vQ_1} \Join_{\ecSem \vCond} \eeSem {\vQ_2}\\
%
\eeSem {\chc {\vQ_1, \vQ_2}} &= 
	\begin{cases}
		\eeSem {\vQ_1}, \text{ if } \fSem \dimMeta = \t\\
		\eeSem {\vQ_2}, \text{ otherwise}
	\end{cases}\\
%
\eeSem {{\vQ_1} \circ {\vQ_2}} &= \eeSem {\vQ_1} \circ \eeSem {\vQ_2}\\
%
\eeSem {\empRel} &= \underline {\empRel}
\end{alignat*}
\caption{Configuration of VRA which assumes that the given v-query
is well-typed. 
%\orSem ., \ecSem ., and \olSem . are
%configuration of v-relation, v-condition, and variational attribute
%set, respectively, defined in \figref{vdb-conf}, 
%\figref{vcond-conf-sem}, \figref{vdb-conf}.
Note that we have extended RA with an empty relation $\underline {\empRel}$.}
\label{fig:v-alg-conf-sem}
\end{figure}


%\NOTE{
%Also, the following definition of the semantics contradicts with the
%description earlier in the section about producing a \emph{result}
%relation.
%
%\medskip
%Also also, maybe we should move the discussion of the semantics before the
%examples? It's a bit surprising to come across it here.}

%The semantics of VRA can be understood as a combination of the
%\emph{configuration semantics} of VRA, defined in \figref{v-alg-conf-sem}, the
%configuration semantics of VDBs, defined in \figref{vdb-conf}, and the
%semantics of plain RA.
%%
%%\TODO{Make the following a more precise description of how these three
%%semantics work together, i.e.\ for every valid configuration of the feature
%%model, we can configure the variational query and VDB in the same way to yield a plain RA
%%query that is then executed over the corresponding plain RDB.}
%%
%Thus, the variational query
%semantics is the set of semantics of its configured relational queries over
%their corresponding configured relational database variant for every valid
%configuration of the feature model of the VDB.
%
The \emph{configuration} function maps a variational query under
a configuration
to a relational query, defined in \figref{v-alg-conf-sem}. Thus, a variational query 
can be understood as a set of relational queries that their results is gathered
in a single table and tagged with the feature expression stating their variant.
%Configuring a variational query
%for all valid configurations, accessible from VDB's feature model,
%provides the complete meaning of a variational query in terms of RA semantics.
%
Users can deploy queries for a specific variant by configuring 
them.
%The configuration of a query allows users to deploy queries for a
%specific variant when they desire, 
%satisfying query part of \nThree\ requirement. 
\exref{conf-vq} illustrates configuring a query.
% stated in \secref{mot}.

%To define VRA semantics we map 
%a variational query to a pure relational query to re-use RA's semantics.
%However, to avoid losing the variation encoded 
%in the variational query, 
%we need to determine the variant under which such a
%mapping is valid. Thus, we introduce the semantic functions that 
%relate a variational query to a relational query.

%
%\textbf{Configuring a variational query:} 
%It maps a variational query under a 
%given configuration to a relational query, denoted by \eeSem . 
%and defined in \figref{v-alg-conf-sem}. Configuring a variational query
%for all valid configurations, accessible from VDB's feature model,
%provides the complete meaning of a variational query in terms of RA semantics.
%Users can deploy queries for a specific variant by configuring 
%them,
%%The configuration of a query allows users to deploy queries for a
%%specific variant when they desire, 
%satisfying query part of \nThree.

\begin{example}
\label{eg:conf-vq}
Assume the underlying VDB has the variational schema
% \t\ feature model and the variational relation
\ensuremath{
\vSch_3 = \{ \vRel \left( \optAtt [\fOne] [\vAtt_1], \vAtt_2, \vAtt_3 \right)^{\fOne \vee \fTwo}
\}} 
and 
\ensuremath{
\fSet = \setDef{ \fOne, \fTwo}}.
The variational query 
\ensuremath{
\vQ_5 = \vPrj [{\vAtt_1, \optAtt [\fOne \wedge \fTwo] [\vAtt_2], \optAtt [\fTwo] [\vAtt_3]}] (\vRel)
}
is configured to the following relational queries:
\ensuremath{\eeSem [\setDef \fOne] {\vQ_5} = \eeSem [\setDef \ ] {\vQ_5} = \pi_{\pAtt_1} \pRel},
\ensuremath{\eeSem [\setDef \fTwo] {\vQ_5} =
 \pi_{\pAtt_1, \pAtt_3} \pRel},
\ensuremath{\eeSem [\setDef {\fOne, \fTwo}] {\vQ_5} = \pi_{\pAtt_1, \pAtt_2, \pAtt_3} \pRel}.
\end{example}




%\textbf{Expressiveness of VRA:} \point{VRA is more expressive than RA since it
%can express all queries in RA and a couple of them all together in one query.}

VRA enables querying multiple database variants encoded as a singled VDB
simultaneously and selectively.
%, satisfying the query need \nOne\ stated in \secref{mot}.
%(\textbf{N1}).
%
More precisely, VRA is \emph{maximally expressive} in the sense that it can
express any set of plain RA queries over any subset of relational database
variants encoded as a VDB. 
We prove this claim in \thmref{max-expr}.
%This claim is captured by the following theorem.

\begin{theorem}
\label{thm:max-expr}
Given a set of plain RA queries $\pQ_1,\ldots,\pQ_n$ where each query $\pQ_i$
is to be executed over a disjoint subset $\vdbInst_i$ of variants of the VDB
instance \vdbInst, there exists a variational query $q$ such that
$\forall \config \in \confSet.\; \odbSem{\vdbInst} = \vdbInst_i \implies \eeSem{\vQ} = \pQ_i$.
\end{theorem}

\begin{proof}
By construction. Let $f_i$ be the feature expression that uniquely
characterizes the variants in each $\vdbInst_i$.
Then 
\ensuremath{\vQ =}
%\small{\vQ =
\(
\chc[(\fName_1\wedge\neg \fName_2\wedge\ldots\wedge\neg \fName_n)]{\pQ_1,
  \chc[(\fName_2\wedge\ldots\wedge\neg \fName_n)]{\pQ_2,\ldots
    \chc[\fName_n]{\pQ_n,\emp}\ldots}}.
\)
\end{proof}

\noindent
%
The above construction relies on the fact that every RA query is a valid VRA
(sub)query in which every presence condition is \t.
%
Of course, in most realistic scenarios, we expect that variational queries can be encoded
more efficiently by sharing commonalities and embedding relevant choices and
presence conditions within the variational query.


%\textbf{Grouping a variational query:} 
%maps a variational query to a set of
%relational queries annotated with feature expressions, denoted by \qGroup .
%and defined in \figref{vq-group}. The presence condition of relational queries 
%indicate the group of configurations where the mapping holds. In essence, 
%grouping of variational query \vQ\ groups together all configurations with the same relational
%query produced from configuring \vQ. 
%Hence, the generated set
%%\dropit{could drop this if it's confusing!}
%of relational queries from grouping a variational query contains distinct (unique) queries.
%For example, consider the query \ensuremath {\vQ_5} in \exref{conf-vq}.
%Grouping \ensuremath{\vQ_5} results in the set:
%\ensuremath{
%\setDef{
%\left( \pi_{\pAtt_1, \pAtt_2, \pAtt_3} \pRel \right)^{\fOne \wedge \fTwo},
%\left(\pi_{\pAtt_1, \pAtt_3} \pRel \right)^{\neg \fOne \wedge \fTwo},
%\left(  \pi_{\pAtt_1} \pRel \right)^{( \fOne \wedge \neg \fTwo) \vee (\neg \fOne \wedge \neg \fTwo)}
%}
%}.
%
%
%
\wrrite{do i want to write this?}
\begin{figure}
%\textbf{Configuration selection semantics of \vqsTxt:}
\begin{alignat*}{1}
\qGroup . &: \qSet \totype \settype {\bm{(} \vartype \pQSet \bm{)}}\\
%
\qGroup \vRel &= \setDef {\annot [\t] \pRel}\\
\qGroup {\vSel \vQ}  &=  
\setDef {\annot [\dimMeta \wedge \dimMeta_\vCond] {\left(\sigma_{\pCond} \pQ\right)} \myOR
\annot \pQ \in \qGroup \vQ, \annot [\dimMeta_\vCond] \pCond \in \cGroup}
\\
%
\qGroup {\vPrj [\vAttList] \vQ} &= 
\setDef {\annot [\dimMeta \wedge \dimMeta_\vAttList] {\left(\pi_{\pAttList} \pQ \right)} \myOR
\annot \pQ \in \qGroup \vQ, \annot [\dimMeta_\vAttList] \pAttList \in \aGroup}
\\
%
\qGroup {{\vQ_1} \times {\vQ_2}} &= 
\setDef {\annot [\dimMeta_1 \wedge \dimMeta_2] {\left(\pQ_1 \times \pQ_2\right)} \myOR
\annot [\dimMeta_1] \pQ_1 \in \qGroup {\vQ_1}, \annot [\dimMeta_2] \pQ_2 \in \qGroup {\vQ_2} }
\\
%
\qGroup {{\vQ_1} \Join_\vCond {\vQ_2}} &= 
\setDef {\annot [\dimMeta_1 \wedge \dimMeta_2 \wedge \dimMeta_\vCond] {\left(\pQ_1 \Join_{\pCond} \pQ_2 \right)} \myOR 
\annot [\dimMeta_1] \pQ_1 \in \qGroup {\vQ_1}, \annot [\dimMeta_2] \pQ_2 \in \qGroup {\vQ_2}
%& \hspace{104pt}
,\annot [\dimMeta_\vCond] \pCond \in \cGroup  }
\\
%
\qGroup {\chc {\vQ_1, \vQ_2}} &= 
\setDef {\annot [\dimMeta \wedge \dimMeta_1] \pQ_1 \myOR  \annot [\dimMeta_1] \pQ_1 \in \qGroup {\vQ_1} }
\cup 
\setDef {\annot [\neg \dimMeta \wedge \dimMeta_2] \pQ_2 \myOR  \annot [\dimMeta_2] \pQ_2 \in \qGroup {\vQ_2}}  \\
%
\qGroup {{\vQ_1} \circ {\vQ_2}} &= 
\setDef {\annot [\dimMeta_1 \wedge \dimMeta_2] {\left(\pQ_1 \circ \pQ_2\right)} \myOR
\annot [\dimMeta_1] \pQ_1 \in \qGroup {\vQ_1}, \annot [\dimMeta_2] \pQ_2 \in \qGroup {\vQ_2} }\\
%
\qGroup {\empRel} &= \annot [\t] { \empRel}
\end{alignat*}
\caption[Unique configuration of variational queries]{Unique configuration of variational queries. 
The unique configuration function assumes that the input is well-typed.
}
\label{fig:vq-group}
\end{figure}



%\subsection{VRA Configuration}
\label{sec:vraconf}

\TODO{vra configuration}


\section{VRA Semantics }
\label{sec:vrasem}

%\TODO{vra semantics. we understand it through RA sem + accumulation}

We use the semantics of relational queries to define the semantics of 
variational queries. We first define the configuration function
for variational queries which takes a configuration and a variational query
and returns a relational query, \secref{vraconf}. We also define another version of the
variational query configuration function that generates unique relational
query variants, \secref{vraconf}. Then, we define an accumulation function that accumulates
multiple (annotated) relational tables into a variational table, \secref{accum}. Finally, we  
define the denotational semantics of VRA using the defined configuration and
accumulation functions, \secref{vradensem}.
%
%\maybeAdd{if have time add VRA sem + equiv}
%dentoational semantics of VRA
%equivalence of dent sem and config and accumulation  --> in properties section



\wrrite{should add this if i have time}
\section{Explicitly Annotating Queries}
\label{sec:constrain}

%\point{type system allows the ql to be flexible and usable.}
%The type system is designed s.t. it relieves the user from necessarily incorporating
%the v-schema variability into their queries as long as the variational queries variability
%does not violate the v-schema, 
Variational queries do not need to repeat information that can be inferred from the v-schema
or the type of a query.
%
For example, the query \ensuremath{\vQ_1} shown in \exref{vq-specific} 
does not contradict the schema and
thus is type correct. However,
 it does not include the presence conditions of attributes and the relation encoded in
the schema while \ensuremath{\vQ_6} repeats this information:\\
%
\centerline{
\ensuremath{
\vQ_6 =
\pi_{\optAtt [\vFour \vee \vFive] [\empno], \optAtt [\vFour] [\name], \optAtt [\vFive] [\fname], \optAtt [\vFive] [\lname]  } \left(\chc [\dimMeta_2] {\empbio, \empRel} \right)}}.

%\pi_{\optAtt [(\vFour \vee \vFive) \wedge \fModel_2] [\empno], \optAtt [\vFour \wedge \fModel_2] [\name], \optAtt [\vFive \wedge \fModel_2] [\fname], \optAtt [\vFive \wedge \fModel_2] [\lname]  } \empbio}}.
%

%\NOTE{
%This is the unsimplified version:
%\begin{align*}
%\VVal {\vQ_5} &= 
%\pi_{\optAtt [\vFour \vee \vFive] [\empno], \optAtt [\vFour] [\name], \optAtt [\vFive] [\fname], \optAtt [\vFive] [\lname]  } \\
%&(\chc [ \fModel_2 ] {\pi_{\empno, \sex, \birthdate, \optAtt [\vFour ] [\name], \optAtt [\vFive] [\fname], \optAtt [\vFive] [\lname]} \empbio, \empRel  })
%\end{align*}
%}
Similarly, the projection in the query 
\ensuremath{\vQ_7 = \pi_{\name, \fname} (\mathit{subq}_7)}
where 
\ensuremath{
\mathit{subq}_7 = \chc [ \vFour] {\pi_\name (\vQ_6), \pi_\fname (\vQ_6)}
}
is written over 
\ensuremath{\vSch_2} and it 
%\centerline{
%\ensuremath{
%\vQ_6 =
%\pi_{\name, \fname} \mathit{subq}_6
%} 
%}}
does not repeat the presence conditions of attributes from its \ensuremath{\mathit{subq}_7}'s type.
The query
%\centerline{
\ensuremath{
\vQ_8 =
\pi_{\optAtt [\vFour ] [\name],\optAtt [\neg \vFour] [\fname]} (\mathit{subq}_7)
%\chc [ \vFour] {\pi_\name \vQ_5, \pi_\fname \vQ_5}
}
%}
makes the annotations of projected attributes \emph{explicit} with respect to both 
the v-schema \ensuremath{\vSch_2} and its subquery's type.
%\TODO {give an example, schema: R(A,B), query: $\pi_{A,B} (F<\pi_A R, \pi_B R>)$
%becomes $\pi_{A^F, B^{\neg F}} ...$}
%The variation encoded in variational queries can
%be more restrictive or more loose than v-schema variation without violating them.
Although relieving the user from explicitly repeating variation makes VRA easier to use, 
queries still have to state variation explicitly to avoid losing information when 
decoupled from the schema.
%We do this by defining a function, 
%\ensuremath {\constrain \vQ}, with type \ensuremath{ \qSet \to \vSchSet \to \qSet
%},
%that \emph{explicitly annotates a query \vQ\ given the underlying schema \vSch}.
We do this by defining the function 
\ensuremath {\constrain \vQ : \qSet \totype \vSchSet \totype \qSet
},
that \emph{explicitly annotates a query \vQ\ with the  schema \vSch}.
%Note that \ensuremath {\constrain \vQ} needs to take the underlying schema as
%an input since it is using the type system (which relies on the schema) as a helper function.
The explicitly annotating query function, 
formally defined in \figref{constrain}, 
conjoins attributes and relations
presence conditions with the corresponding annotations in the query 
and wraps subqueries in a choice when needed. 
Note that, $\vQ_8$ and $\vQ_6$ are the result of $\constrain [\vSch_2] {\vQ_7}$
and $\constrain [\vSch_2] {\vQ_1}$, respectively, after simplification~\footnote{More specifically,
they are simpilified using rules defined in \figref{var-min}}.
%Queries $\vQ_7$ and $\vQ_5$ are examples of applying the 
%explicitly annotation function to queries $\vQ_6$ and $\vQ_1$, respectively,
%after simplifying them.
%\exref{constrain} illustrates how the constrain function transforms queries
%and allows users to be more flexible with their queries. 

\section{Explicitly Annotating Queries}
\label{sec:constrain}

%\point{type system allows the ql to be flexible and usable.}
%The type system is designed s.t. it relieves the user from necessarily incorporating
%the v-schema variability into their queries as long as the variational queries variability
%does not violate the v-schema, 
Variational queries do not need to repeat information that can be inferred from the v-schema
or the type of a query.
%
For example, the query \ensuremath{\vQ_1} shown in \exref{vq-specific} 
does not contradict the schema and
thus is type correct. However,
 it does not include the presence conditions of attributes and the relation encoded in
the schema while \ensuremath{\vQ_6} repeats this information:\\
%
\centerline{
\ensuremath{
\vQ_6 =
\pi_{\optAtt [\vFour \vee \vFive] [\empno], \optAtt [\vFour] [\name], \optAtt [\vFive] [\fname], \optAtt [\vFive] [\lname]  } \left(\chc [\dimMeta_2] {\empbio, \empRel} \right)}}.

%\pi_{\optAtt [(\vFour \vee \vFive) \wedge \fModel_2] [\empno], \optAtt [\vFour \wedge \fModel_2] [\name], \optAtt [\vFive \wedge \fModel_2] [\fname], \optAtt [\vFive \wedge \fModel_2] [\lname]  } \empbio}}.
%

%\NOTE{
%This is the unsimplified version:
%\begin{align*}
%\VVal {\vQ_5} &= 
%\pi_{\optAtt [\vFour \vee \vFive] [\empno], \optAtt [\vFour] [\name], \optAtt [\vFive] [\fname], \optAtt [\vFive] [\lname]  } \\
%&(\chc [ \fModel_2 ] {\pi_{\empno, \sex, \birthdate, \optAtt [\vFour ] [\name], \optAtt [\vFive] [\fname], \optAtt [\vFive] [\lname]} \empbio, \empRel  })
%\end{align*}
%}
Similarly, the projection in the query 
\ensuremath{\vQ_7 = \pi_{\name, \fname} (\mathit{subq}_7)}
where 
\ensuremath{
\mathit{subq}_7 = \chc [ \vFour] {\pi_\name (\vQ_6), \pi_\fname (\vQ_6)}
}
is written over 
\ensuremath{\vSch_2} and it 
%\centerline{
%\ensuremath{
%\vQ_6 =
%\pi_{\name, \fname} \mathit{subq}_6
%} 
%}}
does not repeat the presence conditions of attributes from its \ensuremath{\mathit{subq}_7}'s type.
The query
%\centerline{
\ensuremath{
\vQ_8 =
\pi_{\optAtt [\vFour ] [\name],\optAtt [\neg \vFour] [\fname]} (\mathit{subq}_7)
%\chc [ \vFour] {\pi_\name \vQ_5, \pi_\fname \vQ_5}
}
%}
makes the annotations of projected attributes \emph{explicit} with respect to both 
the v-schema \ensuremath{\vSch_2} and its subquery's type.
%\TODO {give an example, schema: R(A,B), query: $\pi_{A,B} (F<\pi_A R, \pi_B R>)$
%becomes $\pi_{A^F, B^{\neg F}} ...$}
%The variation encoded in variational queries can
%be more restrictive or more loose than v-schema variation without violating them.
Although relieving the user from explicitly repeating variation makes VRA easier to use, 
queries still have to state variation explicitly to avoid losing information when 
decoupled from the schema.
%We do this by defining a function, 
%\ensuremath {\constrain \vQ}, with type \ensuremath{ \qSet \to \vSchSet \to \qSet
%},
%that \emph{explicitly annotates a query \vQ\ given the underlying schema \vSch}.
We do this by defining the function 
\ensuremath {\constrain \vQ : \qSet \totype \vSchSet \totype \qSet
},
that \emph{explicitly annotates a query \vQ\ with the  schema \vSch}.
%Note that \ensuremath {\constrain \vQ} needs to take the underlying schema as
%an input since it is using the type system (which relies on the schema) as a helper function.
The explicitly annotating query function, 
formally defined in \figref{constrain}, 
conjoins attributes and relations
presence conditions with the corresponding annotations in the query 
and wraps subqueries in a choice when needed. 
Note that, $\vQ_8$ and $\vQ_6$ are the result of $\constrain [\vSch_2] {\vQ_7}$
and $\constrain [\vSch_2] {\vQ_1}$, respectively, after simplification~\footnote{More specifically,
they are simpilified using rules defined in \figref{var-min}}.
%Queries $\vQ_7$ and $\vQ_5$ are examples of applying the 
%explicitly annotation function to queries $\vQ_6$ and $\vQ_1$, respectively,
%after simplifying them.
%\exref{constrain} illustrates how the constrain function transforms queries
%and allows users to be more flexible with their queries. 

\section{Explicitly Annotating Queries}
\label{sec:constrain}

%\point{type system allows the ql to be flexible and usable.}
%The type system is designed s.t. it relieves the user from necessarily incorporating
%the v-schema variability into their queries as long as the variational queries variability
%does not violate the v-schema, 
Variational queries do not need to repeat information that can be inferred from the v-schema
or the type of a query.
%
For example, the query \ensuremath{\vQ_1} shown in \exref{vq-specific} 
does not contradict the schema and
thus is type correct. However,
 it does not include the presence conditions of attributes and the relation encoded in
the schema while \ensuremath{\vQ_6} repeats this information:\\
%
\centerline{
\ensuremath{
\vQ_6 =
\pi_{\optAtt [\vFour \vee \vFive] [\empno], \optAtt [\vFour] [\name], \optAtt [\vFive] [\fname], \optAtt [\vFive] [\lname]  } \left(\chc [\dimMeta_2] {\empbio, \empRel} \right)}}.

%\pi_{\optAtt [(\vFour \vee \vFive) \wedge \fModel_2] [\empno], \optAtt [\vFour \wedge \fModel_2] [\name], \optAtt [\vFive \wedge \fModel_2] [\fname], \optAtt [\vFive \wedge \fModel_2] [\lname]  } \empbio}}.
%

%\NOTE{
%This is the unsimplified version:
%\begin{align*}
%\VVal {\vQ_5} &= 
%\pi_{\optAtt [\vFour \vee \vFive] [\empno], \optAtt [\vFour] [\name], \optAtt [\vFive] [\fname], \optAtt [\vFive] [\lname]  } \\
%&(\chc [ \fModel_2 ] {\pi_{\empno, \sex, \birthdate, \optAtt [\vFour ] [\name], \optAtt [\vFive] [\fname], \optAtt [\vFive] [\lname]} \empbio, \empRel  })
%\end{align*}
%}
Similarly, the projection in the query 
\ensuremath{\vQ_7 = \pi_{\name, \fname} (\mathit{subq}_7)}
where 
\ensuremath{
\mathit{subq}_7 = \chc [ \vFour] {\pi_\name (\vQ_6), \pi_\fname (\vQ_6)}
}
is written over 
\ensuremath{\vSch_2} and it 
%\centerline{
%\ensuremath{
%\vQ_6 =
%\pi_{\name, \fname} \mathit{subq}_6
%} 
%}}
does not repeat the presence conditions of attributes from its \ensuremath{\mathit{subq}_7}'s type.
The query
%\centerline{
\ensuremath{
\vQ_8 =
\pi_{\optAtt [\vFour ] [\name],\optAtt [\neg \vFour] [\fname]} (\mathit{subq}_7)
%\chc [ \vFour] {\pi_\name \vQ_5, \pi_\fname \vQ_5}
}
%}
makes the annotations of projected attributes \emph{explicit} with respect to both 
the v-schema \ensuremath{\vSch_2} and its subquery's type.
%\TODO {give an example, schema: R(A,B), query: $\pi_{A,B} (F<\pi_A R, \pi_B R>)$
%becomes $\pi_{A^F, B^{\neg F}} ...$}
%The variation encoded in variational queries can
%be more restrictive or more loose than v-schema variation without violating them.
Although relieving the user from explicitly repeating variation makes VRA easier to use, 
queries still have to state variation explicitly to avoid losing information when 
decoupled from the schema.
%We do this by defining a function, 
%\ensuremath {\constrain \vQ}, with type \ensuremath{ \qSet \to \vSchSet \to \qSet
%},
%that \emph{explicitly annotates a query \vQ\ given the underlying schema \vSch}.
We do this by defining the function 
\ensuremath {\constrain \vQ : \qSet \totype \vSchSet \totype \qSet
},
that \emph{explicitly annotates a query \vQ\ with the  schema \vSch}.
%Note that \ensuremath {\constrain \vQ} needs to take the underlying schema as
%an input since it is using the type system (which relies on the schema) as a helper function.
The explicitly annotating query function, 
formally defined in \figref{constrain}, 
conjoins attributes and relations
presence conditions with the corresponding annotations in the query 
and wraps subqueries in a choice when needed. 
Note that, $\vQ_8$ and $\vQ_6$ are the result of $\constrain [\vSch_2] {\vQ_7}$
and $\constrain [\vSch_2] {\vQ_1}$, respectively, after simplification~\footnote{More specifically,
they are simpilified using rules defined in \figref{var-min}}.
%Queries $\vQ_7$ and $\vQ_5$ are examples of applying the 
%explicitly annotation function to queries $\vQ_6$ and $\vQ_1$, respectively,
%after simplifying them.
%\exref{constrain} illustrates how the constrain function transforms queries
%and allows users to be more flexible with their queries. 

\input{formulas/constrainVQbySch}

\begin{theorem}
\label{thm:expl-same-type}
If the query \vQ\ has the type \vType\ then its explicitly annotated counterpart has the same type \vType, i.e.: \\
%
\centerline{
\ensuremath{%\raggedleft
\envWithoutVctx {\vQ} {\vType} \Rightarrow \envWithoutVctx {\constrain \vQ} {\VVal \vType} \textit{ and } \vType \equiv {\VVal \vType}
}}
%
This shows that the type system applies the schema to the type of a query although it does not apply it to the query. 
The \emph{type equivalence} is variational set equivalence, defined 
in \figref{vset}, for normalized variational sets of attributes.
%\footnote{We proved this theorem in the Coq proof assistant. The encoding of the theorem and the proof can be found in second author's MS thesis~\cite{FaribaThesis}.}.
\end{theorem}

We encode and prove \thmref{expl-same-type} in the Coq proof assistant~\cite{FaribaThesis}.
We also illustrate the application of \thmref{expl-same-type} to queries
\ensuremath{\vQ_1} and \ensuremath{\vQ_6}.
%
\exref{type} explained how \ensuremath{\vQ_1}'s type is generated step-by-step.
The variation context and underlying schema are
the same and the subquery \empbio\ has the same type. 
The projected attribute set annotated with the variation context is:
\ensuremath{
\vAttList_2 =  \{\annot [\vFour \vee \vFive] \empno, }
\ensuremath{ 
\optAtt [\vFour] [\name], \optAtt [\vFive] [\fname], \optAtt [\vFive] [\lname]\}^{\dimMeta_2}}, which is clearly subsumed by \ensuremath{\vAttList_\empbio}, thus, 
%the type of \empbio, \vAttList, and
its intersection with \ensuremath{\vAttList_\empbio} annotated
with the presence condition of \ensuremath{\vAttList_\empbio} is itself,
hence, \ensuremath{\vAttList_{\vQ_1} \equiv \vAttList_{\vQ_6}}.
%which makes it obvious that \ensuremath{\vAttList_{\vQ_1} \equiv \vAttList_{\vQ_6}}.
%\end{example}

\begin{theorem}
\label{thm:expl-same-type}
If the query \vQ\ has the type \vType\ then its explicitly annotated counterpart has the same type \vType, i.e.: \\
%
\centerline{
\ensuremath{%\raggedleft
\envWithoutVctx {\vQ} {\vType} \Rightarrow \envWithoutVctx {\constrain \vQ} {\VVal \vType} \textit{ and } \vType \equiv {\VVal \vType}
}}
%
This shows that the type system applies the schema to the type of a query although it does not apply it to the query. 
The \emph{type equivalence} is variational set equivalence, defined 
in \figref{vset}, for normalized variational sets of attributes.
%\footnote{We proved this theorem in the Coq proof assistant. The encoding of the theorem and the proof can be found in second author's MS thesis~\cite{FaribaThesis}.}.
\end{theorem}

We encode and prove \thmref{expl-same-type} in the Coq proof assistant~\cite{FaribaThesis}.
We also illustrate the application of \thmref{expl-same-type} to queries
\ensuremath{\vQ_1} and \ensuremath{\vQ_6}.
%
\exref{type} explained how \ensuremath{\vQ_1}'s type is generated step-by-step.
The variation context and underlying schema are
the same and the subquery \empbio\ has the same type. 
The projected attribute set annotated with the variation context is:
\ensuremath{
\vAttList_2 =  \{\annot [\vFour \vee \vFive] \empno, }
\ensuremath{ 
\optAtt [\vFour] [\name], \optAtt [\vFive] [\fname], \optAtt [\vFive] [\lname]\}^{\dimMeta_2}}, which is clearly subsumed by \ensuremath{\vAttList_\empbio}, thus, 
%the type of \empbio, \vAttList, and
its intersection with \ensuremath{\vAttList_\empbio} annotated
with the presence condition of \ensuremath{\vAttList_\empbio} is itself,
hence, \ensuremath{\vAttList_{\vQ_1} \equiv \vAttList_{\vQ_6}}.
%which makes it obvious that \ensuremath{\vAttList_{\vQ_1} \equiv \vAttList_{\vQ_6}}.
%\end{example}

\begin{theorem}
\label{thm:expl-same-type}
If the query \vQ\ has the type \vType\ then its explicitly annotated counterpart has the same type \vType, i.e.: \\
%
\centerline{
\ensuremath{%\raggedleft
\envWithoutVctx {\vQ} {\vType} \Rightarrow \envWithoutVctx {\constrain \vQ} {\VVal \vType} \textit{ and } \vType \equiv {\VVal \vType}
}}
%
This shows that the type system applies the schema to the type of a query although it does not apply it to the query. 
The \emph{type equivalence} is variational set equivalence, defined 
in \figref{vset}, for normalized variational sets of attributes.
%\footnote{We proved this theorem in the Coq proof assistant. The encoding of the theorem and the proof can be found in second author's MS thesis~\cite{FaribaThesis}.}.
\end{theorem}

We encode and prove \thmref{expl-same-type} in the Coq proof assistant~\cite{FaribaThesis}.
We also illustrate the application of \thmref{expl-same-type} to queries
\ensuremath{\vQ_1} and \ensuremath{\vQ_6}.
%
\exref{type} explained how \ensuremath{\vQ_1}'s type is generated step-by-step.
The variation context and underlying schema are
the same and the subquery \empbio\ has the same type. 
The projected attribute set annotated with the variation context is:
\ensuremath{
\vAttList_2 =  \{\annot [\vFour \vee \vFive] \empno, }
\ensuremath{ 
\optAtt [\vFour] [\name], \optAtt [\vFive] [\fname], \optAtt [\vFive] [\lname]\}^{\dimMeta_2}}, which is clearly subsumed by \ensuremath{\vAttList_\empbio}, thus, 
%the type of \empbio, \vAttList, and
its intersection with \ensuremath{\vAttList_\empbio} annotated
with the presence condition of \ensuremath{\vAttList_\empbio} is itself,
hence, \ensuremath{\vAttList_{\vQ_1} \equiv \vAttList_{\vQ_6}}.
%which makes it obvious that \ensuremath{\vAttList_{\vQ_1} \equiv \vAttList_{\vQ_6}}.
%\end{example}
\section{Variation-Minimization Rules}
\label{sec:var-min}

%
%\maybeAdd{add $\vQ_6$ is simplified of  $\VVal \vQ_6$ because of rule application blah blah.}
%\maybeAdd{add example + more rules + point out interesting ones}
%
VRA is flexible since an information need can be represented via multiple
variational queries as demonstrated in \exref{vq-specific} and \exref{vq-same-intent-mult-vars}.
It allows users to incorporate their personal taste and task requirements
into variational queries they write by 
having different levels of variation. For example, consider the explicitly annotated query
\ensuremath{\vQ_6} 
in \secref{constrain}.
%\ensuremath {
\[
\vQ_6 =
\pi_{\optAtt [\vFour \vee \vFive] [\empno], \optAtt [\vFour] [\name], \optAtt [\vFive] [\fname], \optAtt [\vFive] [\lname]  } \left( \chc [\fModel_2] {\empbio, \empRel}\right)
\]
%}.
%\vQ_5 =  \pi_{\optAtt [\vFour \vee \vFive] [\empno], \optAtt [\vFour] [\name], \optAtt [\vFive] [\fname], \optAtt [\vFive] [\lname]  } \empbio}.
%from \exref{vq-specific}. 
To be explicit about the exact query that will be run for 
each variant 
%and knowing that 
%\ensuremath{
%\getPC \empbio = \vThree \vee \vFour \vee \vFive
%},
the query $\vQ_6$'s variation can be \emph{lifted up} by using choices, resulting in the query $\VVVal \vQ_6$.
%\ensuremath{
%\small
\[
\VVVal \vQ_6 = \chc [\vFour] {\pi_{\empno, \name} \empbio, 
\chc [\vFive] {\pi_{\empno, \fname, \lname} \empbio, \emp}} 
\]
%}.
While \ensuremath{\vQ_6} contains less redundancy \ensuremath{\VVVal \vQ_6}
is more comprehensible since the variants are explicitly stated in the dimension of the choice. 
Thus, \emph{supporting multiple levels of variation 
creates a tension between reducing redundancy and maintaining comprehensibility.}

We define \emph{variation minimization} rules in \figref{var-min} that are syntactic and 
preserve the semantics.
% and include 
%interesting ones in \secref{var-min}.
Pushing in variation into a query, i.e., applying rules left-to-right, 
reduces redundancy
% and improves performance
while lifting them up, i.e., applying rules right-to-left, 
makes a query more understandable. 
When applied left-to-right, the rules are terminating since the scope of variation 
%always gets smaller.
monotonically decreases in size.
%
%\revised{
%Additionally, these rules can be used to simplify queries after
%explicitly annotating them with a schema. For example, the first rule in \figref{var-min}
%is used to simplify the query \ensuremath{\constrain [\vSch_2] {\vQ_1}}, introduced in \secref{var-pres},
% which resulted
%in \ensuremath{\vQ_6}.}


\begin{figure}
\textbf{Choice Distributive Rules:}
\begin{alignat*}{1}
\small
%-- f<? l? q?, ? l? q?> ? ? (f<l?, l?>) f<q?, q?>
%\inferrule
%{}
%\chc {\pi_{\vAttList_1} \vQ_1, \pi_{\vAttList_2} \vQ_2 } 
%&\equiv
%\pi_{\chc {\vAttList_1, \vAttList_2}} \chc {\vQ_1, \vQ_2}\\
%-- f<? l? q?, ? l? q?> ? ? ((l??), (l? \^�f )) f<q?, q?>
%\inferrule
%{}
\chc {\pi_{\vAttList_1} \vQ_1, \pi_{\vAttList_2} \vQ_2}
&\equiv
\pi_{\annot \vAttList_1, \annot [\neg \dimMeta] \vAttList_2} \chc {\vQ_1, \vQ_2}\\
%-- f<? c? q?, ? c? q?> ? ? f<c?, c?> f<q?, q?>
%\inferrule
%{}
\chc {\sigma_{\vCond_1} \vQ_1, \sigma_{\vCond_2} \vQ_2} 
&\equiv
\sigma_{\chc {\vCond_1, \vCond_2}} \chc {\vQ_1, \vQ_2}\\
%-- f<q? � q?, q? � q?> ? f<q?, q?> � f<q?, q?>
%\inferrule
%{}
\chc {\vQ_1 \times \vQ_2, \vQ_3 \times \vQ_4}
&\equiv
\chc {\vQ_1, \vQ_3} \times \chc {\vQ_2, \vQ_4}\\
%-- f<q? ?\_c? q?, q? ?\_c? q?> ? f<q?, q?> ?\_(f<c?, c?>) f<q?, q?>
%\inferrule
%{}
\chc {\vQ_1 \Join_{\vCond_1} \vQ_2, \vQ_3 \Join_{\vCond_2} \vQ_4}
&\equiv
\chc {\vQ_1, \vQ_3} \Join_{\chc {\vCond_1, \vCond_2}} \chc {\vQ_2, \vQ_4}\\
%-- f<q? ? q?, q? ? q?> ? f<q?, q?> ? f<q?, q?>
%\inferrule
%{}
\chc {\vQ_1 \circ \vQ_2, \vQ_3 \circ \vQ_4}
&\equiv
\chc {\vQ_1, \vQ_3} \circ \chc {\vQ_2, \vQ_4}
%-- f<q? ? q?, q? ? q?> ? f<q?, q?> ? f<q?, q?>
%\inferrule
%{}
%{-}
\end{alignat*}

\medskip
\textbf{CC and RA Optimization Rules Combined:}
\begin{alignat*}{1}
\small
%-- f<? (c? ? c?) q?, ? (c? ? c?) q?> ? ? (c? ? f<c?, c?>) f<q?, q?>
%\inferrule
%{}
\chc {\sigma_{\vCond_1 \wedge \vCond_2} \vQ_1, \sigma_{\vCond_1 \wedge \vCond_3} \vQ_2}
&\equiv
\sigma_{\vCond_1 \wedge \chc {\vCond_2, \vCond_3}} \chc {\vQ_1, \vQ_2}\\
%-- ? c? (f<? c? q?, ? c? q?>) ? ? (c? ? f<c?, c?>) f<q?, q?>
%\inferrule
%{}
\sigma_{\vCond_1} \chc {\sigma_{\vCond_2} \vQ_1, \sigma_{\vCond_3} \vQ_2}
&\equiv
\sigma_{\vCond_1 \wedge \chc {\vCond_2, \vCond_3}} \chc {\vQ_1, \vQ_2}\\
%-- f<q? ?\_(c? ? c?) q?, q? ?\_(c? ? c?) q?> ? ? (f<c?, c?>) (f<q?, q?> ?\_c? f<q?, q?>)
%\inferrule
%{}
\chc {\vQ_1 \Join_{\vCond_1 \wedge \vCond_2} \vQ_2, \vQ_3 \Join_{\vCond_1 \wedge \vCond_3} \vQ_4}
&\equiv
\sigma_{\chc {\vCond_2, \vCond_3}} \left( \chc {\vQ_1, \vQ_3} \Join_{\vCond_1} \chc {\vQ_2, \vQ_4} \right)
\end{alignat*}

\caption{Some of variation minimization rules.}
\label{fig:var-min}
\end{figure}

\subsection{VRA Type System}
\label{sec:typesys}

\TODO{type sys}


%\section{Variation-Minimization Rules}
\label{sec:var-min}

%
%\maybeAdd{add $\vQ_6$ is simplified of  $\VVal \vQ_6$ because of rule application blah blah.}
%\maybeAdd{add example + more rules + point out interesting ones}
%
VRA is flexible since an information need can be represented via multiple
variational queries as demonstrated in \exref{vq-specific} and \exref{vq-same-intent-mult-vars}.
It allows users to incorporate their personal taste and task requirements
into variational queries they write by 
having different levels of variation. For example, consider the explicitly annotated query
\ensuremath{\vQ_6} 
in \secref{constrain}.
%\ensuremath {
\[
\vQ_6 =
\pi_{\optAtt [\vFour \vee \vFive] [\empno], \optAtt [\vFour] [\name], \optAtt [\vFive] [\fname], \optAtt [\vFive] [\lname]  } \left( \chc [\fModel_2] {\empbio, \empRel}\right)
\]
%}.
%\vQ_5 =  \pi_{\optAtt [\vFour \vee \vFive] [\empno], \optAtt [\vFour] [\name], \optAtt [\vFive] [\fname], \optAtt [\vFive] [\lname]  } \empbio}.
%from \exref{vq-specific}. 
To be explicit about the exact query that will be run for 
each variant 
%and knowing that 
%\ensuremath{
%\getPC \empbio = \vThree \vee \vFour \vee \vFive
%},
the query $\vQ_6$'s variation can be \emph{lifted up} by using choices, resulting in the query $\VVVal \vQ_6$.
%\ensuremath{
%\small
\[
\VVVal \vQ_6 = \chc [\vFour] {\pi_{\empno, \name} \empbio, 
\chc [\vFive] {\pi_{\empno, \fname, \lname} \empbio, \emp}} 
\]
%}.
While \ensuremath{\vQ_6} contains less redundancy \ensuremath{\VVVal \vQ_6}
is more comprehensible since the variants are explicitly stated in the dimension of the choice. 
Thus, \emph{supporting multiple levels of variation 
creates a tension between reducing redundancy and maintaining comprehensibility.}

We define \emph{variation minimization} rules in \figref{var-min} that are syntactic and 
preserve the semantics.
% and include 
%interesting ones in \secref{var-min}.
Pushing in variation into a query, i.e., applying rules left-to-right, 
reduces redundancy
% and improves performance
while lifting them up, i.e., applying rules right-to-left, 
makes a query more understandable. 
When applied left-to-right, the rules are terminating since the scope of variation 
%always gets smaller.
monotonically decreases in size.
%
%\revised{
%Additionally, these rules can be used to simplify queries after
%explicitly annotating them with a schema. For example, the first rule in \figref{var-min}
%is used to simplify the query \ensuremath{\constrain [\vSch_2] {\vQ_1}}, introduced in \secref{var-pres},
% which resulted
%in \ensuremath{\vQ_6}.}


\begin{figure}
\textbf{Choice Distributive Rules:}
\begin{alignat*}{1}
\small
%-- f<? l? q?, ? l? q?> ? ? (f<l?, l?>) f<q?, q?>
%\inferrule
%{}
%\chc {\pi_{\vAttList_1} \vQ_1, \pi_{\vAttList_2} \vQ_2 } 
%&\equiv
%\pi_{\chc {\vAttList_1, \vAttList_2}} \chc {\vQ_1, \vQ_2}\\
%-- f<? l? q?, ? l? q?> ? ? ((l??), (l? \^�f )) f<q?, q?>
%\inferrule
%{}
\chc {\pi_{\vAttList_1} \vQ_1, \pi_{\vAttList_2} \vQ_2}
&\equiv
\pi_{\annot \vAttList_1, \annot [\neg \dimMeta] \vAttList_2} \chc {\vQ_1, \vQ_2}\\
%-- f<? c? q?, ? c? q?> ? ? f<c?, c?> f<q?, q?>
%\inferrule
%{}
\chc {\sigma_{\vCond_1} \vQ_1, \sigma_{\vCond_2} \vQ_2} 
&\equiv
\sigma_{\chc {\vCond_1, \vCond_2}} \chc {\vQ_1, \vQ_2}\\
%-- f<q? � q?, q? � q?> ? f<q?, q?> � f<q?, q?>
%\inferrule
%{}
\chc {\vQ_1 \times \vQ_2, \vQ_3 \times \vQ_4}
&\equiv
\chc {\vQ_1, \vQ_3} \times \chc {\vQ_2, \vQ_4}\\
%-- f<q? ?\_c? q?, q? ?\_c? q?> ? f<q?, q?> ?\_(f<c?, c?>) f<q?, q?>
%\inferrule
%{}
\chc {\vQ_1 \Join_{\vCond_1} \vQ_2, \vQ_3 \Join_{\vCond_2} \vQ_4}
&\equiv
\chc {\vQ_1, \vQ_3} \Join_{\chc {\vCond_1, \vCond_2}} \chc {\vQ_2, \vQ_4}\\
%-- f<q? ? q?, q? ? q?> ? f<q?, q?> ? f<q?, q?>
%\inferrule
%{}
\chc {\vQ_1 \circ \vQ_2, \vQ_3 \circ \vQ_4}
&\equiv
\chc {\vQ_1, \vQ_3} \circ \chc {\vQ_2, \vQ_4}
%-- f<q? ? q?, q? ? q?> ? f<q?, q?> ? f<q?, q?>
%\inferrule
%{}
%{-}
\end{alignat*}

\medskip
\textbf{CC and RA Optimization Rules Combined:}
\begin{alignat*}{1}
\small
%-- f<? (c? ? c?) q?, ? (c? ? c?) q?> ? ? (c? ? f<c?, c?>) f<q?, q?>
%\inferrule
%{}
\chc {\sigma_{\vCond_1 \wedge \vCond_2} \vQ_1, \sigma_{\vCond_1 \wedge \vCond_3} \vQ_2}
&\equiv
\sigma_{\vCond_1 \wedge \chc {\vCond_2, \vCond_3}} \chc {\vQ_1, \vQ_2}\\
%-- ? c? (f<? c? q?, ? c? q?>) ? ? (c? ? f<c?, c?>) f<q?, q?>
%\inferrule
%{}
\sigma_{\vCond_1} \chc {\sigma_{\vCond_2} \vQ_1, \sigma_{\vCond_3} \vQ_2}
&\equiv
\sigma_{\vCond_1 \wedge \chc {\vCond_2, \vCond_3}} \chc {\vQ_1, \vQ_2}\\
%-- f<q? ?\_(c? ? c?) q?, q? ?\_(c? ? c?) q?> ? ? (f<c?, c?>) (f<q?, q?> ?\_c? f<q?, q?>)
%\inferrule
%{}
\chc {\vQ_1 \Join_{\vCond_1 \wedge \vCond_2} \vQ_2, \vQ_3 \Join_{\vCond_1 \wedge \vCond_3} \vQ_4}
&\equiv
\sigma_{\chc {\vCond_2, \vCond_3}} \left( \chc {\vQ_1, \vQ_3} \Join_{\vCond_1} \chc {\vQ_2, \vQ_4} \right)
\end{alignat*}

\caption{Some of variation minimization rules.}
\label{fig:var-min}
\end{figure}

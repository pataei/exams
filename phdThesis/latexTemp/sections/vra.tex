\section{Variational Relational Algebra}
\label{sec:vrel-alg}

%
%\wrrite{you want to have an example here that doesn't need to be explicitly annotated. 
%then use the same example to illustrate the semantics via ra sem.}
%\TODO{we use ... that has these differences with blah
%%
%we introduce these differences through building up a query to extract info
%required by the variational intent. }

%\point{vra = cc + ra}
%Considering the variational nature of a VDB, to satisfy a user's information 
%need when extracting information, 
%we need a query language that not only considers the structure of 
%relational databases (such as SQL and relational algebra (RA)) but also 
%accounts for the variation encoded in the VDB. We achieve this by:
%1) picking relational algebra as our main query language and
%2) using \emph{choices}~\cite{Walk13thesis, EW11tosem} 
%and presence conditions to account for variation. 

To account for variation, VRA combines relational algebra (RA) with 
\emph{choices}~\cite{EW11tosem,HW16fosd,Walk13thesis}.
%\point{choice.}
Remember that a choice $\chc{\elem_1,\elem_2}$ consists of a feature expression \dimMeta, called
the \emph{dimension} of the choice, and 
two \emph{alternatives} $\elem_1$ and $\elem_2$. For a given configuration \config, 
the choice $\chc{\elem_1, \elem_2}$ can be replaced by $\elem_1$ if \dimMeta\
evaluates to \t\ under configuration \config, (i.e., \fSem{\dimMeta}),
or $\elem_2$ otherwise. 
% Choices allow a variational queries
% to encode variation in a structured and systematic manner. 

\begin{figure}
\begin{syntax}

\multicolumn{4}{l}{\textbf{Operators:}} \\[1ex]
\bullet
  &\eqq& \multicolumn{2}{l}{< \myOR \leq \myOR = \myOR \neq \myOR > \myOR \geq} \\
\circ
  &\eqq& \cup \myOR \cap \\[2ex]

\multicolumn{4}{l}{\textbf{Variational conditions:}} \\[1ex]
\vCond\in\vCondSet
  &\eqq&  \multicolumn{2}{l}{
          \bTag
   \myOR  \pAtt \bullet \cte
   \myOR  \pAtt \bullet \pAtt
   \myOR  \neg \vCond
   \myOR  \vCond \vee \vCond} \\
  &\myOR& \multicolumn{2}{l}{
          \vCond \wedge \vCond
   \myOR \chc{\vCond,\vCond}} \\[2ex]

\multicolumn{4}{l}{\textbf{Variational queries:}} \\[1ex]
\vQ\in\qSet
  &\eqq&  \vRel     & \textit{Relation}\\
  &\myOR& \vSel \vQ & \textit{Selection}\\
  &\myOR& \vPrj[\vAttList]{\vQ} & \textit{Projection}\\
  &\myOR& \chc{\vQ,\vQ} & \textit{Choice}\\
% &\myOR& \vQ \Join_\vCond \vQ & \textit{Variational Join}\\
  &\myOR& \vQ \times \vQ & \textit{Cartesian Product}\\
  &\myOR& \vQ \circ \vQ  & \textit{Set Operation}\\
% &\myOR& \vQ \backslash \vQ &\textit{Variational Set Difference}\\
  &\myOR& \empRel & \textit{Empty Relation}

\end{syntax}

\caption[Syntax of variational relational algebra]{Syntax of variational relational algebra.}
%\TODO{remember that
%you removed join (also removed it from query config def and constrain query 
%by schema). if you want use it just say it's a syntactic sugar.}
%$<, \leq, =, \neq, >, \geq$.
%$\circ$ denotes set operators: union and difference.}
\label{fig:v-alg-def}
\end{figure}



%\point{explain notation and VRA operations.}
The syntax of VRA is given in \figref{v-alg-def}.
%
The selection operation is similar to standard RA selection except
that the condition parameter is \emph{variational} meaning that it may contain
choices.
For example, the query 
\ensuremath{\sigma_{\chc {\vAtt_1=\vAtt_2,\vAtt_1=\vAtt_3}} (\vRel)}
selects a variational tuple \vTuple\ if it satisfies
the condition \ensuremath{\vAtt_1 = \vAtt_2} 
and  \ensuremath{\sat {\dimMeta \wedge \getPC \vTuple}}
or
if \ensuremath{\vAtt_1 = \vAtt_3} 
and \ensuremath{\sat {\neg \dimMeta \wedge \getPC \vTuple }}.
%
The projection operation is parameterized by a variational set of attributes, \vAttList. For
example,
the query $\pi_{\vAtt_1, \optAtt [\dimMeta] [\vAtt_2]} (\vRel)$
projects $\vAtt_1$ from relation \vRel\ unconditionally, and $\vAtt_2$ 
when \sat{\dimMeta}.
%
The choice operation enables combining two variational queries to be used in different
variants based on the dimension. In practice,
it is often useful to return information in some variants and nothing at all in
others. We introduce an explicit \emph{empty} query \empRel\ to facilitate
this. 
Similar to our definition of the empty query for relational algebra, for VRA we
also have: $\empRel=\vPrj[\set{}]{\vQ}$.
The empty query is used, for example, in 
\ensuremath{\vQ_2} in \exref{vq-specific}. 
%The set operations between queries are v-set operations defined in \secref{vset}.
The rest of VRA's operations are similar to RA, where all set operations
(union, intersection, and product) are changed to the corresponding
variational set operations defined in \secref{vset}.
%\secref{vlist-vset}.
%
%\remember{
%In examples, we also use a join operation with a variational condition,
%$\vQ_1\bowtie_\vCond\vQ_2$, which is syntactic sugar for
%$\sigma_\vCond(\vQ_1\times\vQ_2)$.}


Our implementation of VRA also provides mechanisms for renaming queries and
qualifying attributes with relation/sub\-query names. These features are needed
to support self joins and to project attributes with the same name in different
relations. However, for simplicity, we omit these features from the formal
definition in this thesis.


%A query can simply 
%refer to a relation, filter tuples based on a variational condition 
%(which is a relational condition with choices of two conditions), and
%project a variational list of attributes. Besides production of two queries and
%set operations, VRA allows for a choice of two variational queries. This demands an
%\emph{empty} query since an alternative of a choice can very well inquire 
%no information at all. 
%For example, the query $\chc {\vQ_1, \}$
%

%\subsubsection{Running a Variational query on a VDB Results in a Variational table}
%\label{sec:run-vq-get-vtab}
%A variational query systematically represents a set of relational query variants associated to their
%corresponding database variants. Hence, intuitively the user expects to 
%get such variation in their result as well. 

The result of a variational query is a variational table with the reserved relation name $\mathit{result}$.
%
For example, assume that variational tuples $\annot[\fOne]{(1,2)}$ and $\annot[\neg
f_3]{(3,4)}$ belong to a variational relation $\vRel(\vAtt_1,\vAtt_2)$, which is the only
relation in a VDB with the trivial feature model \t.
%
The query $\chc[f_3]{\pi_{\optAtt[f_2][\vAtt_1]}(\vRel),\empRel}$ returns a
variational table with relation schema $\annot[f_3]{\mathit{result}(\annot[f_2]{a_1})}$,
which indicates that the result is only non-empty when $f_3$ is \t\ and that the
result includes attribute $a_1$ when $f_2$ is \t. 
%\secref{type-sys} defines a
%type system that yields the relation schema for any well-formed query.
%
The content of the result relation for the example query is a single variational tuple
$\annot[f_1]{(1)}$. The tuple $\annot[\neg f_3]{(3)}$ is not included since the
projection occurs in the context of a choice in $f_3$, which is incompatible
with the presence condition of the tuple, i.e., $\unsat{f_3 \wedge\neg f_3}$.
This illustrates how choices can effectively filter the tuples in a VDB based
on the dimension.
%, satisfying the second part of \nOne.
%
% Although there is no need to update the presence condition of the returned
% tuples, yet choices can filter the returned variational tuples.
%
% Note that here the value \ensuremath{1}
% of attribute \ensuremath{\vAtt_1} is present in VDB variants where 
% \ensuremath{\sat {\A \wedge \B \wedge \C}} although the presence 
% condition of the returned variational tuple does not have to state this condition
% since 
%
% overall presence 
%condition and the presence conditions of attributes and tuples are
%restricted by the variation enforced by the query.
%
%Note that the presence condition of tuples, attributes, and the return relation
%is restricted by the variation enforced by the query. 
% correct this so that you don't conjunct the pc of relation and clarify that it's relations's pc and not the attributes. although the conjunction should be satisfiable.}
%
%
%%Hence, VRA is more expressive than RA 
%%because it can encode variational queries.
%The variational nature allows users to write interesting queries in many ways:
%1) to express their variational information need or to filter returned tuples
%they can use annotations or 
%choices, \exref{vq-specific},
%2) to express the same intent over several database variants they can 
%use choices in queries or conditions, \exref{vq-same-intent-mult-vars},
%and 
%3) they can also use choices to express different intents over database variants.
%\TODO{Eric, should we drop the last since it creates messy results and isn't really useful?}.
%%The expressiveness of VRA satisfies \textbf{N1}, this is illustrated in 
%%\exref{vq-specific} and \exref{vq-same-intent-mult-vars}.
%%Interestingly, VRA's expressiveness enables users to express 
%%their information need more specifically by stating the exact condition
%%under which an information need is inquired. \exref{vq-specific} illustrates this.
%%It also allows users to express the same intent over several database 
%%variants
% \NOTE{
% To express the variational information need or to filter returned tuples
% users can use annotations or choices. \exref{vq-specific} illustrates this.
% }
%
%The following example
\exref{vq-specific} illustrates
%, in the context of our running example, 
how
a variational query can be used to express variational information needs.

\begin{example}
\label{eg:vq-specific}
%VRA's expressiveness consequently facilitates expressing exactly the condition
%under which an information need is inquired. 
%\wrrite{build up this example. and show the result tables.}
\eric{pls read this example. it has changed a bit}
Assume a VDB with
\ensuremath{\fSet = \setDef {\vThree, \vFour, \vFive}}, 
and the only variational table \empbio\ shown in \tabref{empbio-vtab}
with the feature model $\dimMeta_2 = \oneof {\vThree, \vFour, \vFive}$
which is different from the feature model $\dimMeta_{\mathit{mot}}$ of the \empbio\ variational table
shown in \tabref{empbio-vtab} .
%the corresponding \empbio\ schema variants in \tabref{mot}. 
The variational schema for this VDB is:
%
\begin{align*}
\vSch_2 &=
\{\empbio (\empno, \sex, \birthdate,
\optAtt [\vFour] [\name], \optAtt [\vFive] [\fname],
 \optAtt [\vFive] [\lname] )\}^{\dimMeta_2}
% \\
%& \hspace{-38pt} \textit{where } \dimMeta_2 = \oneof {\vThree, \vFour, \vFive}
%{\vThree \oplus \vFour \oplus \vFive}.
%\left(\vThree \wedge \neg \vFour \wedge \neg \vFive\right)
%  \vee \left(\vFour \wedge \neg \vThree \wedge \neg \vFive\right) 
%   \vee \left(\vFive \wedge \neg \vThree \wedge \neg \vFour\right)}.
\end{align*}
%
Now, the user wants the employee ID numbers (\empno) and their names for variants 
that enable either \vFour\ or \vFive\ but not \vThree.
%\set{\vFour} and \set{\vFive}.
We show the steps to build up multiple queries that can extract this information. 
First, to extract the required attributes we write the query $\vQ_0$ to project all the needed
attributes without considering the variational aspect of projection. 
Note that the presence condition attribute (\pcatt) does not need to be projected. In fact, 
the presence condition attribute is returned for every variational query since that is the only
way to keep track of variation at the content level. 
%
\tabref{vq0-res} shows
the result of this query over the described VDB.
%
Note that the presence condition of the result is $\getPCfrom \empbio {\vSch_2} = \oneof {\vThree, \vFour, \vFive} \wedge (\vThree \vee \vFour \vee \vFive)$ which can be simplified to
$\oneof {\vThree, \vFour, \vFive}$. We discuss how the 
presence conditions of the returned result and its attributes are generated in \secref{type-sys}.
%
\[
\vQ_0 = \pi_{\empno, \name, \fname, \lname} (\empbio)
\]
Now we pay attention to the variational aspect of the query. Knowing that the database
enforces the variation encoded in itself (that is, the VDB exists if and only if exactly
 one of the features \vThree--\vFive\ is enabled, the \name\ attribute only exists for variants
that enable \vFour\ and the \fname\ and \lname\ attributes only exist for variants that
enable \vFive) and since we only want the
projected attributes for variants that enable \vFour\ or \vFive\ we can write the
query $\vQ_1$. \tabref{vq1-res} shows the result of this query over the described VDB.
%
Note that the first three tuples from \tabref{vq0-res} are not returned since the \empno\
attribute does not exist for them in addition to the previously non-existing attributes 
\name, \fname, and \lname. Thus, the tuple will just be empty and so is dropped. 
%The user needs to project the \name\ attribute 
%for variant \set{\vFour}, the \fname\ and \lname\ attributes for variant
%\set{\vFive}, and \empno\ attribute for both variants.
%This can be expressed with the following variational query.
\[
\vQ_1 = \pi_{\optAtt [\vFour \vee \vFive] [\empno], \name, \fname, \lname} (\empbio)
\]
If we did not know that the database enforces the variation encoded in itself
we had to repeat that variation. This is expressed in $\VVal {\vQ_1}$. The
result of this query is still \tabref{vq1-res}.
\[
\VVal {\vQ_1} = 
\pi_{\optAtt [(\vFour \vee \vFive) \wedge \neg \vThree] [\empno], 
\optAtt [\vFour \wedge \neg \vThree \wedge \neg \vFive] [\name], 
\optAtt [\vFive \wedge \neg \vThree \wedge \neg \vFour] [\fname], 
\optAtt [\vFive \wedge \neg \vThree \wedge \neg \vFour] [\lname]} (\empbio)
\]
%
Note that unlike the variational table \empbio\ shown in \tabref{empbio-vtab},
%which is restricted not only by its presence condition but also by the feature model,
the result of a variational query is not part of a bigger variational structure (the VDB).
Thus, it is only restricted by its presence condition and not the feature model although
the feature model has already been applied to its presence condition.
\end{example}

\begin{table}
\caption[Results of some variational queries]{Results of some variational queries over the VDB instance described in \exref{vq-specific}.}
\label{tab:vq-res}
\centering
\begin{subtable}[t]{\textwidth}
\centering
\caption{Result of the variational query $\pi_{\empno, \name, \fname, \lname} (\empbio)$.}
\label{tab:vq0-res}
\footnotesize
\arrayrulecolor{blue}
%!{\color{black}\vrule}
\begin{tabular} {c !{\color{black}\vrule} l l l l : l }
 {\textcolor{blue}{$\oneof {\vThree, \vFour,\vFive}$} }& {\textcolor{blue}{\texttt{true}}}& \tiny {\textcolor{blue}{$\vFour$}} &  {\textcolor{blue}{$\vFive$}} &  {\textcolor{blue}{$\vFive $}} & {\textcolor{blue}{\texttt{true}}}\\
\arrayrulecolor{blue}\hdashline
\multirow{2}{*}{$\mathit{result}$}  & \empno & \name & \fname & \lname & \pcatt \\
\arrayrulecolor{black}\cline{2-6}
& 12001 & & & & \textcolor{blue}{\vThree}\\
& 12002 & & & & \textcolor{blue}{\vThree}\\
& 12003 & & & & \textcolor{blue}{\vThree}\\
 &80001  & Nagui Merli & & & \textcolor{blue}{\vFour}\\
 & 80002 & Mayuko Meszaros & & & \textcolor{blue}{\vFour}\\
 & 80003 & Theirry Viele & & & \textcolor{blue}{\vFour}\\
 & 200001  & & Selwyn & Koshiba & \textcolor{blue}{\vFive}\\
 & 200002  & & Bedrich & Markovitch & \textcolor{blue}{\vFive}\\
 & 200003  & & Pascal & Benzmuller  & \textcolor{blue}{\vFive}\\
 & \ldots  & \ldots & \ldots & \ldots & \textcolor{blue}{\ldots} \\
\arrayrulecolor{white}\hline
\end{tabular}
\end{subtable}

\medskip
\medskip
\medskip
\begin{subtable}[t]{\textwidth}
\centering
\caption{Result of the variational queries $\pi_{\optAtt [\vFour \vee \vFive] [\empno], \name, \fname, \lname} (\empbio)$ and 
$\pi_{\optAtt [(\vFour \vee \vFive) \wedge \neg \vThree] [\empno], 
\optAtt [\vFour \wedge \neg \vThree \wedge \neg \vFive] [\name], 
\optAtt [\vFive \wedge \neg \vThree \wedge \neg \vFour] [\fname], 
\optAtt [\vFive \wedge \neg \vThree \wedge \neg \vFour] [\lname]} (\empbio)
$.}
\label{tab:vq1-res}
\footnotesize
\arrayrulecolor{blue}
%!{\color{black}\vrule}
\begin{tabular} {c !{\color{black}\vrule} l l l l : l }
 {\textcolor{blue}{$\oneof {\vThree, \vFour,\vFive}$} }& {\textcolor{blue}{$\vFour \vee \vFive$}}&  {\textcolor{blue}{$\vFour $}} &  {\textcolor{blue}{$\vFive $}} &  {\textcolor{blue}{$\vFive$}} & {\textcolor{blue}{\texttt{true}}}\\
\arrayrulecolor{blue}\hdashline
\multirow{2}{*}{$\mathit{result}$}  & \empno & \name & \fname & \lname & \pcatt \\
\arrayrulecolor{black}\cline{2-6}
%& 12001 & & & & \textcolor{blue}{\vThree}\\
%& 12002 & & & & \textcolor{blue}{\vThree}\\
%& 12003 & & & & \textcolor{blue}{\vThree}\\
 &80001  & Nagui Merli & & & \textcolor{blue}{\vFour}\\
 & 80002 & Mayuko Meszaros & & & \textcolor{blue}{\vFour}\\
 & 80003 & Theirry Viele & & & \textcolor{blue}{\vFour}\\
 & 200001  & & Selwyn & Koshiba & \textcolor{blue}{\vFive}\\
 & 200002  & & Bedrich & Markovitch & \textcolor{blue}{\vFive}\\
 & 200003  & & Pascal & Benzmuller  & \textcolor{blue}{\vFive}\\
 & \ldots  & \ldots & \ldots & \ldots& \textcolor{blue}{\ldots} \\
\arrayrulecolor{white}\hline
\end{tabular}
\end{subtable}

\medskip
\medskip
\medskip
\begin{subtable}[t]{\textwidth}
\centering
\caption{Result of the variational query $\chc[\vFour\vee\vFive]{\pi_{\empno,\name,\fname,\lname}(\empbio),\empRel}$.}
\label{tab:vq2-res}
\footnotesize
\arrayrulecolor{blue}
%!{\color{black}\vrule}
\begin{tabular} {c !{\color{black}\vrule} l l l l : l }
 {\textcolor{blue}{$\oneof {\vThree, \vFour,\vFive} \wedge \neg \vThree$} }& {\textcolor{blue}{$\vFour \vee \vFive$}}&  {\textcolor{blue}{$\vFour $}} &  {\textcolor{blue}{$\vFive $}} &  {\textcolor{blue}{$\vFive$}} & {\textcolor{blue}{\texttt{true}}}\\
\arrayrulecolor{blue}\hdashline
\multirow{2}{*}{$\mathit{result}$}  & \empno & \name & \fname & \lname & \pcatt \\
\arrayrulecolor{black}\cline{2-6}
%& 12001 & & & & \textcolor{blue}{\vThree}\\
%& 12002 & & & & \textcolor{blue}{\vThree}\\
%& 12003 & & & & \textcolor{blue}{\vThree}\\
 &80001  & Nagui Merli & & & \textcolor{blue}{\vFour}\\
 & 80002 & Mayuko Meszaros & & & \textcolor{blue}{\vFour}\\
 & 80003 & Theirry Viele & & & \textcolor{blue}{\vFour}\\
 & 200001  & & Selwyn & Koshiba & \textcolor{blue}{\vFive}\\
 & 200002  & & Bedrich & Markovitch & \textcolor{blue}{\vFive}\\
 & 200003  & & Pascal & Benzmuller  & \textcolor{blue}{\vFive}\\
 & \ldots  & \ldots & \ldots & \ldots& \textcolor{blue}{\ldots} \\
\arrayrulecolor{white}\hline
\end{tabular}
\end{subtable}

\end{table}


In the example, note that the user does not need to repeat the variability  encoded
in the variational schema in their query, that is, they do not need to annotate \name,
\fname, and \lname\ with \vFour, \vFive, and \vFive, respectively. We discuss
this in more detail in \secref{constrain}. $\vQ_1$
queries all three variants simultaneously although the returned results are
only associated with variants \vFour\ and \vFive\ due to the annotation of the
attribute \empno\ in the query and the presence conditions of the rest of the
projected attributes in the schema.
%
Yet, the query can be more simplified with a choice. $\vQ_2$ selects only two
out of the three variants explicitly:
%selecting only two out of the three variants can be written more
%explicitly in a query by using a choice:
$\vQ_2=\chc[\neg \vThree]{\pi_{\empno,\name,\fname,\lname}(\empbio),\empRel}$. 
%
\tabref{vq2-res} shows the result fo this query over the VDB described in \exref{vq-specific}.
%
Note that, as shown in \tabref{vq-res}, 
queries $\vQ_1$ and $\vQ_2$ return the same set of variational tuples since
neither returns tuples associated with variant \vThree, but their returned
variational tables have different presence conditions, thus, $\vQ_2$ filters out
tuples that belong to variant \vThree\ at the schema level while $\vQ_1$ does not. We discuss this
more in \exref{type}. 
%

%\NOTE{
%\revised{VRA has \revised{syntactic} equivalence rules, described in
%\secref{var-min}, that enable semantics-preserving transformations of queries
%similar to the transformation of $\vQ_1$ into $\vQ_2$ (and vice versa). These
%rules enable factoring commonality out of subqueries, among other
%transformations.}

%The next example 
 Expressing
the same intent over several database variants by a single query relieves the DBA from
maintaining separate queries for different variants or configurations of the
schema.
\exref{vq-same-intent-mult-vars} 
illustrates this point.
% by using choices.
%how a variational query can be used to express the same
%intent over several database variants using choices and conditions.

\begin{example}
\label{eg:vq-same-intent-mult-vars}
Assume a VDB with  \ensuremath{\fSet = \setDef{\vOne, \ldots, \vFive}}
and the corresponding \basic\ schema
variants in \tabref{mot}. The user wants to get all employee names across all
variants. They express this intent by the query $\vQ_3$:
%
\begin{align*}
\vQ_3 &= 
  \vOne\chcL
    (\pi_{\name}(\engemp)) \cup (\pi_{\name}(\othemp)) \\
 & \hspace{32pt},
    (\vTwo\vee\vThree)\chcL
      \pi_{\name}(\empacct) \\
 & \hspace{88pt},
      \chc[(\vFour\vee\vFive)]{\pi_{\name,\fname,\lname}\empbio, \emp}\chcR\chcR
\end{align*}
%
Since the variational schema enforces that exactly one of \vOne--\ \vFive\ be enabled, we
can simplify the query by omitting the final choice.
%
\begin{align*}
\vQ_4 &= 
  \vOne\chcL
    (\pi_{\name}(\engemp)) \cup (\pi_{\name}(\othemp)) \\
 & \hspace{32pt},
    \chc[(\vTwo\vee\vThree)]{
      \pi_{\name}(\empacct),
      \pi_{\name,\fname,\lname}(\empbio)}
\end{align*}
%
\end{example}

In principle, variational queries can also express arbitrarily different intents over
different database variants. However, we expect that variational queries are best used to
capture single (or at least related) intents that vary in their realization
since this is easier to understand and increases the potential for sharing in
both the representation and execution of a variational query.







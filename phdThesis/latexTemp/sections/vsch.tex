\section{Variational Schema}
\label{sec:vsch}

\TODO{change vdb conf to vsch.}
\rewrite{read and revise}
\fromppr{vldb}
\TODO{remember to use definitions of config elem and vset}
\TODO{add vsch of mot ex}
%\begin{figure}
%[ht]

\textbf{Variational schema objects:}
\begin{syntax}
\synDef \vAtt \attnametype &&&\textit{Attribute Name}\\
\synDef \vRel \relnametype &&& \textit{Relation Name}\\
\synDef \vAttList \vAttSet &\eqq& 
\setDef {\annot [\dimMeta_1] \vAtt_1, \annot [\dimMeta_2] \vAtt_2, \ldots, \annot [\dimMeta_k] \vAtt_k} & \textit{Variational Set of Attributes}\\
\synDef \vRelSch \vRelSchSet &\eqq& \vRelDef & \textit{Variational Relation Schema}\\
\synDef \vSch \vSchSet &\eqq& \vSchDef & \textit{Variational Schema}
\end{syntax}

\medskip
\textbf{Variational schema type synonyms:}
\begin{alignat*}{1}
\vSchSet &= \vartype {\bm{(} \settype \relschtype\bm{)}}
\end{alignat*}

\medskip
\textbf{Presence condition of attributes and relations:}
\begin{alignat*}{1}
\getPCfrom {\vAtt} {\vSch} &= \getPCfrom {\vAtt} {\vRel \annot [\dimMeta_\vRel] {(\ldots, \annot [\dimMeta_\vAtt] \vAtt, \ldots)}} \wedge \getPCfrom \vRel \vSch = \getPC {\annot [\dimMeta_\vAtt] \vAtt} \wedge \getPCfrom \vRel \vSch=
\dimMeta_\vAtt \wedge \dimMeta_\vRel \wedge \getPC \vSch\\
\getPCfrom {\vRel} \vSch &= \getPC {\vRel \annot [\dimMeta_\vRel] {(\vAttList)}} \wedge \getPC {\vSch} = \dimMeta_\vRel \wedge \getPC \vSch
\end{alignat*}

\medskip
\textbf{Variational set of attributes configuration:}
\begin{alignat*}{1}
 \olSem [] . &: \ \vAttSet \totype \confSet \totype \pAttSet\\
 \olSem \vAttList &= \elemSem \vAttList
 \end{alignat*}
%& \olSem {\{\optAtt\} \cup \vAttList} &\spcEq &\ 
%    \begin{cases}
%        \{\pAtt\} \cup \olSem{\vAttList},
%        &\If \fSem {\dimMeta}\\
%%         \wedge \getPCfrom \vAtt \vRel } \\
%%                            & \If \fSem {\dimMeta \wedge \getPC{\getRel \vAtt} \wedge \fModel} \\
%        \olSem{\vAttList} , & \Otherwise
%     \end{cases} \\
%% & \olSem {\{\optAtt\} \cup \vAttList} &\spcEq &\  \olSem {\{\optAtt\}} \cup \olSem {\vAttList}\\
%& \olSem {\setDef{}} &\spcEq & \setDef{}
%\end{flalign*}

%
\medskip
\textbf{Variational relation schema configuration:}
\begin{alignat*}{1}%\raggedleft
\orSem [] . &: \vRelSchSet \totype \confSet  \totype \maybe \pRelSchSet\\
%\end{flalign*}
%
%\begin{flalign*}
\orSem {\vRelDef [\vRel] [\dimMeta_\vAttList]} &= 
	\begin{cases}
		\vRel\left({\olSem {\pushIn { \annot [\dimMeta_\vAttList] \vAttList}}}\right), &\If \fSem {\dimMeta_\vAttList} = \t\\
%		&\If \fSem {\getPCfrom \vRel \vSch} \\
%		&\If \fSem {\dimMeta \wedge \fModel}) \\
		\bot, &\Otherwise
	\end{cases}
\end{alignat*}

%
\medskip
\textbf{Variational schema configuration:}
\begin{alignat*}{1}%\raggedleft
\osSem [] . &: \vSchSet \totype \confSet \totype \pSchSet\\
%\end{flalign*}
%
%\begin{flalign*}
\osSem {\annot [\dimMeta] {\setDef {\vRelDefNum 1, \ldots, \vRelDefNum \numRels}}}\\
&\hspace{-70pt}= \begin{cases}
%		\setDef {\orSem {\vRelDefNumF 1}, \ldots, \orSem {\vRelDefNumF n}},
                 \setDef {\orSem {\vRel_1( \vAttList_1 )^{\dimMeta_1 {\wedge \dimMeta}} }, \ldots, 
                 \orSem {\vRel_\numRels( \vAttList_\numRels)^{\dimMeta_\numRels {\wedge \dimMeta}} }},	
                         & \If \fSem {\dimMeta } =\t \\	
%        & \If \fSem \fModel \\
        \setDef{}, & \text{otherwise}
	\end{cases}
\end{alignat*}


\caption[Variational schema definition and configuration]{Variational schema definition, presence condition of attributes and relations, and variational schema configuration.}
\label{fig:vsch}
\end{figure} 


A variational schema captures variation in the structure of a database by indicating which attributes and relations are included or excluded in which variants.
%
%\point{using annotated elements to show variability in schema.}
%Variation can exist in the structure of data, i.e., the schema.
%As motivated in \secref{mot}, schema variations include/exclude relations/attributes. 
To this end, we annotate attributes, relations, and the schema itself with
feature expressions,
which describe the conditions under which each is present.
%
A \emph{\vrelTxt\ schema} (\emph{v-relation schema}), \vRelSch, is a relation name
with an annotated variational set of attributes,
%\centerline{
$\synDef {\vRelSch} \vRelSchSet \eqq \vRelDef$,
%}
where \ensuremath{\vAttList \in \vAttSet} is a variational attribute set.
The presence condition of the v-relation schema, \dimMeta, determines in what
variants of the database the relation itself is present in.
%A \emph{variational attribute set}, \vAttList, is a variational set of attributes.
%%i.e.,
%%$\synDef \vAttList \vAttSet \eqq \optAtt, \vAttList \myOR \empAtt$,
%%where \empAtt\ denotes an empty attribute. 
%%\exref{vsch} illustrates creating
%%a v-schema.
%
A \emph{\vschTxt} (\emph{v-schema}) is an annotated set of v-relation 
schemas,
%\centerline{ 
$\synDef \vSch \vSchSet \eqq \vSchDef$.
%} 
The presence condition of the entire v-schema, \fModel, is the VDB's feature
model, which provides a top-level constraint on the set of valid
configurations, as described in \secref{encode-var}.
%We call such configurations
%\emph{valid} configurations. The v-schema's presence condition 
%is the VDB feature model since it captures the relationship
%between features of the underlying application and their constraints,
%as explained in \secref{encode-var} and introduced as the VDB feature model. 
Hence, the v-schema defines all valid schema variants of a VDB. 


\begin{example}
\label{eg:vsch}
$\vSch_1$ is the v-schema of a VDB that only includes relations \empacct\ and \ecourse\ in the last two rows
of \tabref{mot} and has the
feature space 
\ensuremath{\fSet = \setDef{ \vFour, \vFive, \edu, \tFour, \tFive}}.
Note that attributes that exist conditionally are annotated with a feature expression
to account for such a condition, e.g., the \salary\ attribute only exists when \vFive\ = \t.
%
\begin{align*}
\vSch_1 &=
\{ \empacct ( \empno, \hiredate, \titleatt, \deptno, \annot [\vFive] \salary, \\
&\hspace{55pt} \annot [\edu] \isstudent,
\annot [\edu] \isteacher )^{\vFour \vee \vFive},\\
%
&\hspace{17pt} \ecourse ( \cno, \cname, \annot [\tFive] \deptno )^{\tFour \vee \tFive} \}^{\fModel_1}\\
\fModel_1 &= (\vFour\oplus\vFive) \wedge
  ((\edu\wedge(\tFour\oplus\tFive)) \vee (\neg\edu\wedge\neg(\tFour\vee\tFive)))
\end{align*}
where \ensuremath{\fModel_1} allows only one temporal feature for the basic
schema to be enabled at a time, and either one temporal feature for the education
extension, if \edu\ is enabled, or else no temporal feature for the education
extension.
\end{example}

\NOTE{Double-check my simplification of $\fModel_1$.}

%\begin{table*}
\caption[Variational schema of the motivating example]{Variational schema $\vSch_\mot$ 
%of the motivating example given in \secref{mot} 
with feature model $\dimMeta_\mot$.
This variational schema encompasses 30 relational schemas: five schemas when \edu\ = \f\ and 25 schemas otherwise. 
%A v-schema encoding the variation of the employee schema introduced in \tabref{mot}. The feature model \fModel\ only allows one temporal feature to be true from a set of temporal features at the time: 
%\ensuremath{
%\fModel = \edu \vee 
%\left( 
%\vOne \oplus \vTwo \oplus \vThree \oplus \vFour \oplus \vFive
%\right) 
%\vee 
%\left( 
%\tOne \oplus \tTwo \oplus \tThree \oplus \tFour \oplus \tFive
%\right)
%}. 
%%where \ensuremath{\fName_1 \oplus \fName_2 = (\fName_1 \wedge \neg \fName_2) \vee (\fName_2 \wedge \neg \fName_1)}.
%However, the feature model can be encoded differently to allow more than one temporal feature to be true at the time. Hence, 
%this is not the only v-schema capturing variation of the employee schema. 
%Additionally, the encoding can change by formulating presence conditions differently while representing the same v-schema, 
%e.g., the presence condition of the \job\ relation can be changed to \ensuremath{\neg \vFive}. 
%This v-schema encompasses 30 relational schemas: five schemas when \edu\ = \f\ and 25 schemas otherwise. 
%%Note that the feature model restricts the schema s.t. only one variant of a sub-schema can exists in the schema, e.g., both \vOne\ and \vTwo cannot be enabled at the same time.
}
\label{tab:mot-vsch}
\arrayrulecolor{black}
\begin{center}
\small
\begin{tabular} {| l |}
\hline
\ensuremath{
\engemp (\empno, \name, \hiredate,\titleatt,\deptname )^{\textcolor{blue}{\vOne}}
}\\
\ensuremath{
\othemp (\empno, \name, \hiredate,\titleatt,\deptname )^{\textcolor{blue}{\vOne}}
}\\
\ensuremath{
\empacct (\empno, \optAtt [{\textcolor{blue}{\vTwo \vee \vThree}}] [\name], \hiredate, \titleatt, \optAtt [{\textcolor{blue}{\vTwo}}] [\deptname], \optAtt [{\textcolor{blue}{\vThree \vee \vFour \vee \vFive}}] [\deptno], \optAtt [{\textcolor{blue}{\vFive}}] [\salary],} \\
\hspace{40pt} \ensuremath{\optAtt [{\textcolor{blue}{\vFour \vee \vFive}}] [\isstudent], \optAtt [{\textcolor{blue}{\vFour \vee \vFive}}] [\isteacher] )^{\textcolor{blue}{\vTwo \vee \vThree \vee \vFour \vFive}}
}\\
\ensuremath{
\job \left(\titleatt, \salary  \right)^{\textcolor{blue}{\vOne \vee \vTwo \vee \vThree \vee \vFour}}
}\\
\ensuremath{
\dept \left(\deptname, \deptno, \managerno, \optAtt [{\textcolor{blue}{\vFive}}] [\studentnum], \optAtt [{\textcolor{blue}{\vFive}}] [\teachernum] \right)^{\textcolor{blue}{\vThree \vee \vFour \vee \vFive}}
}\\
\ensuremath{
\empbio \left(\empno, \sex, \birthdate, \optAtt [{\textcolor{blue}{\vFour}}] [\name], \optAtt [{\textcolor{blue}{\vFive}}] [\fname], \optAtt [{\textcolor{blue}{\vFive}}] [\lname] \right)^{\textcolor{blue}{\vThree \vee \vFour \vee \vFive}}
}\\
%\hdashline
\ensuremath{
\course \left(\optAtt [{\textcolor{blue}{\neg \tOne}}] [\cno], \cname, \optAtt [{\textcolor{blue}{\tOne \vee \tTwo}}] [\tno], \optAtt [{\textcolor{blue}{\tFour \vee \tFive}}] [\timeatt], \optAtt [{\textcolor{blue}{\tFour \vee \tFive}}] [\class], \optAtt [{\textcolor{blue}{\tFive}}] [\deptno] \right)^{\textcolor{blue}{\edu}}
}\\
\ensuremath{
\student \left(\sno, \optAtt [{\textcolor{blue}{\tOne}}] [\cname], \optAtt [{\textcolor{blue}{\neg \tOne}}] [\cno], \optAtt [{\textcolor{blue}{\tThree \vee \tFour}}] [\grade] \right)^{\textcolor{blue}{\edu \wedge \neg \tFive}}
}\\
\ensuremath{
\teach \left(\tno, \cno \right)^{\textcolor{blue}{\edu \wedge \left(\tThree \vee \tFour \vee \tFive\right)}}
}\\
\ensuremath{
\ecourse \left(\cno, \cname, \optAtt [{\textcolor{blue}{\tFive}}] [\deptno] \right)^{\textcolor{blue}{\edu \wedge \left(\tFour \vee \tFive\right)}}
}\\
\ensuremath{
\take \left(\sno, \cno, \grade \right)^{\textcolor{blue}{\edu \wedge \tFive}}
}\\
%\multirow{3}{*}{\vOne} &  \engemp\ (\empno, \name, \hiredate,\titleatt,\deptname) & 
%\course\ (\cname, \tno) & \multirow{3}{*}{\tOne}\\
%& \othemp\ (\empno, \name, \hiredate, \title, \deptname)  & \student\ (\sno, \cname) &\\
%& \job\ (\titleatt, \salary) &  &\\
%\hline
%\multirow{2}{*}{\vTwo} & \empacct\ (\empno, \name, \hiredate, \titleatt, \deptname) & \course\ (\cno, \cname, \tno) & \multirow{2}{*}{\tTwo}\\
%%\cdashline{2-3}
%& \job\ (\titleatt, \salary) & \student\ (\sno, \cno) & \\
%\hline
%\multirow{4}{*}{\vThree} & \empacct\ (\empno, \name, \hiredate, \titleatt, \deptno) & \course\ (\cno, \cname) & \multirow{4}{*}{\tThree}\\
%& \job\ (\titleatt, \salary) & \teach\ (\tno, \cno) &\\
%& \dept\ (\deptname, \deptno, \managerno) & \student\ (\sno, \cno, \grade) &\\
%& \empbio\ (\empno, \sex, \birthdate) & &\\
%\hline
%\multirow{4}{*}{\vFour} & \empacct\ (\empno, \hiredate, \titleatt, \deptno, \dashuline{\isstudent}, \dashuline{\isteacher}) & \ecourse\ (\cno, \cname) & \multirow{4}{*}{\tFour}\\
%& \job\ (\titleatt, \salary) & \course\ (\cno, \cname, \timeatt, \class) & \\
%& \dept\ (\deptname, \deptno, \managerno) & \teach\ (\tno, \cno) & \\
%& \empbio\ (\empno, \sex, \birthdate, \name) & \student\ (\sno, \cno, \grade) & \\
%\hline
%\multirow{4}{*}{\vFive} & \empacct\ (\empno, \hiredate, \titleatt, \deptno,  \dashuline{\isstudent}, \dashuline{\isteacher}, \salary) & \ecourse\ (\cno, \cname, \deptno) & \multirow{4}{*}{\tFive}\\
%& \dept\ (\deptname, \deptno, \managerno,  \dashuline{\studentnum}, \dashuline{\teachernum}) & \course\ (\cno, \cname, \timeatt, \class, \deptno) & \\
%& \empbio\ (\empno, \sex, \birthdate, \fname, \lname) & \teach\ (\tno, \cno) & \\
%&& \take\ (\sno, \cno, \grade) & \\
\hline
\end{tabular}
\end{center}
\end{table*}



%\textbf{Hierarchal structure of feature expressions in a v-schema:}
The presence of an attribute follows the hierarchal layout of information in a database:
an attribute's presence depends on the presence of its parent v-relation, which in turn 
depends on the presence of the v-schema. 
%relation presence condition and the feature model.
% presence of the v-schema. 
%
Thus, the complete presence condition of the attribute
$\annot[\dimMeta_\vAtt]{\vAtt}$ in v-relation
$\vRel\annot[\dimMeta_\vRel]{(\ldots)}$ defined in v-schema \vSch\ with feature
model \fModel\ is
$\getPCfrom{\vAtt}{\vSch}=\dimMeta_\vAtt\wedge\dimMeta_\vRel\wedge\fModel$.

%and it is present iff this presence condition is satisfiable, i.e., 
%\ensuremath{\sat {\getPCfrom \vAtt \vSch}}.
Similarly, the presence condition of v-relation $\vRel$ is
\ensuremath{\getPCfrom \vRel \vSch = \dimMeta_\vRel \wedge \fModel}.
For example, in \exref{vsch} we have
\ensuremath{\getPCfrom \empacct {\vSch_1} = (\vFour \vee \vFive) \wedge \fModel_1}.
%the conjunction of its \revised{annotation}
%%presence condition 
%with its v-relation's \revised{annotation}
%%presence condition
%and the feature model.
%
%Similarly,
%the presence condition of a v-relation is the conjunction of its
%\revised{annotation}
%%presence condition 
%and the feature model.
%%
%\revised{
%In other words, the annotated attribute \optAtt\ of v-relation 
%\ensuremath{\vRel \annot [\dimMeta_\vRel] {\left(\annot \vAtt, \ldots\right)}} 
%%with 
%%$\dimMeta_r = \getPC \vRel$
%defined in the v-schema \vSch\ with feature model \fModel\
%is valid if: $\sat {\getPCfrom \vAtt \vSch} = \sat {\dimMeta_\vAtt \wedge \dimMeta_r \wedge \fModel }$.}
Furthermore, a database element is only present in variants for
which its presence condition satisfiable, e.g., in \exref{vsch}
 the \isstudent\ attribute 
%described in \exref{vsch} 
%is present in the variants that make its presence condition 
%\ensuremath {\edu \wedge (\vFour \vee \vFive) \wedge \fModel_1 }
%satisfiable.
%So \isstudent\ 
is present in the variant 
\ensuremath{\setDef {\edu, \vFour, \tFive}} since 
\ensuremath{\fSem [\setDef {\edu, \vFour, \tFive}] {\getPCfrom \isstudent {\vSch_1}}
= \fSem [\setDef {\edu, \vFour, \tFive}] {\edu \wedge (\vFour \vee \vFive) \wedge \fModel_1}
= \t \wedge (\t \vee \f) \wedge ((\neg \t \wedge (\t \oplus \f)) \vee (\t \wedge (\t \oplus \f) \wedge (\f \oplus \t))) = \t}
but it is not present in the variant
\ensuremath{\setDef{\vFour, \tFive}} since in this variant \edu = \f, thus,
\ensuremath{\fSem [\setDef {\edu, \vFour, \tFive}] {\getPCfrom \isstudent {\vSch_1}} = \f}.

\NOTE{Revise the above to be consistent with the new $m_1$.}

%\subsubsection{Configuring a V-Schema}
%\label{sec:conf-vsch}
Intuitively and similar to v-sets, a v-schema is a systematic  
compact representation of a set of plain schemas called variants.
%, e.g., SPL, that encodes 
%the variation effectively inside the database schema by means of 
%feature expressions.
%Consequently, v-schemas relieve the
%need to define an intermediate schema and state mappings 
%between it and source schemas, like the approach that 
%data integration systems employ. 
%One can still obtain the 
A schema variant can be obtained 
%A specific pure relational schema for
%a database variant can be obtained 
by \emph{configuring} the v-schema with that variant's configuration.
We define the configuration function for v-schemas and its elements in \figref{vdb-conf} in \appref{vdb-conf}.
%For example, consider the v-schema in \exref{vsch}.
%Configuring the variational attribute set of the \empacct\ v-relation for 
%the variant \setDef {\vFive}, i.e., 
%\ensuremath {\olSem [\setDef {\vFive}] {\empno, \hiredate, \titleatt, \deptno, \annot [\vFive] \salary,
%\annot [\edu] \isstudent,
%\annot [\edu] \isteacher}},
%yields the 
%%relational 
%attribute set of
%\ensuremath {\setDef {\underline \empno,\underline \hiredate,\underline \titleatt,\underline \deptno }}.



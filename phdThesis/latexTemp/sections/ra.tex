\section{The SPJR Relational Algebra}
\label{sec:ra}

\TODO{relational algebra}
\TODO{add syntax definition}
\TODO{add type system}
\TODO{add examples with tables}
\TODO{maybe add semantics later on}

We do not extend the notation of using underline for relational algebra operations.
Instead, relational algebra operations are overloaded and they are used as
 both plain relational and variational operations. It is clear from the context when
 an operation is variational or not. 
 %
We also extend relational algebra s.t. projection of an empty list from a query is valid
and it returns an empty set. In fact, we denote such query by the \emph{empty} query \empPRel, thus,
\ensuremath{
\empPRel = \pi_{\{\}} \pQ}.


\TODO{the following is from prelim. revise.}

\begin{figure}

\begin{syntax}

% feature expressions
%\synDef{\dimMeta}{\ffSet}
%  &\eqq& \multicolumn{2}{l}{%
%         \t \myOR \f \myOR \fName \myOR \neg\fName
%         \myOR \dimMeta\wedge\dimMeta \myOR \dimMeta\vee\dimMeta}
%\\[1.5ex]

% relation conditions
\synDef{\pCond}{\pCondSet}
  &\eqq& \multicolumn{2}{l}{%
         \t \myOR \f \myOR \att\bullet\cte \myOR \att\bullet\att
         \myOR \neg\pCond \myOR \pCond\vee\pCond} \\
%     &|& \multicolumn{2}{l}{\vCond\wedge\vCond \myOR \chc{\vCond,\vCond}}
\\[1.5ex]

% variational relational algebra
\synDef{\pQ}{\pQSet}
  &\eqq& \pRel                 & \textit{Relation reference} \\
     &|& \pRen[\pRel]{\pQ}     & \textit{Renaming} \\
     &|& \pPrj[\pAttList]{\pQ} & \textit{Projection} \\
     &|& \pSel\pQ              & \textit{Selection} \\
     &|& \pQ \Join_{\pCond} \pQ  & \textit{Join} \\
%     &|& \chc{\vQ,\vQ}         & \textit{Choice} \\
%     &|& \empRel               & \textit{Empty relation} \\
%    &|& \vQ \times \vQ        & \textit{Cartesian Product} \\
%    &|& \vQ \circ \vQ         & \textit{Set operation} \\
\end{syntax}

\caption{Syntax of  relational algebra, where $\bullet$ ranges over
comparison operators ($<, \leq, =, \neq, >, \geq$), \cte\ over constant values,
\att\ over attribute names, and \pAttList\ over lists of attributes.
The syntactic category
% \dimMeta\ represents feature expressions, 
 \pCond\
is relational conditions, and \pQ\ is  relational algebra terms.
}
%\vspace{-20pt}
\label{fig:v-alg-def}
\end{figure}
%\vspace{-20pt}

\TODO{add bullet and conditions and attribute list to the definition.}

\figref{rel-alg} defines the syntax of 
relational algebra which allows users to query a relational database~\cite{AliceBook}.
%
The first five constructs are adapted from relational algebra:
%
A query may simply \emph{reference} a relation \pRel\ in the schema.
\emph{Renaming} allows giving a name to an intermediate query to be referenced
 later. Note that \pRel\ is an overloaded symbol that indicates both a relation
 and a relation name. 
%
A \emph{projection} enables selecting a subset of attributes from the results
of a subquery, for example, \vPrj[\pAtt_1]{\pRel} would return only attribute $\pAtt_1$
from $\pRel$.
%; we extend the standard project operator to work with annotated lists
%of attributes, for example, \vPrj[a_1,a_2^e]{r} would include $a_1$ for all
%configurations and also $a_2$ for configurations where $e$ is true.
%
A \emph{selection} enables filtering the tuples returned by a subquery based on a
given condition \pCond, for example, \vSel[\pAtt_1 > 3]{\pRel} would return all tuples
from $\pRel$ where the value for $\pAtt_1$ is greater than 3.
%; these conditions may be
%variational to enable returning different tuples for different configurations
%of the VDB.
%
A \emph{Cartesian products} simply cross products every tuple from its
left subquery with every tuple from its right subquery. 
%
The \emph{join} operation joins two subqueries based on a condition and
omitting its condition implies it is a natural join (i.e., join on the
shared attribute of the two subqueries).
For example, $\pRel_1 \bowtie_{\pAtt_1 = \pAtt_2} \pRel_2$ joins tuples from $\pRel_1$ 
and $\pRel_2$ where the attribute $\pAtt_1$ from relation $\pRel_1$ is equal to
attribute $\pAtt_2$ from relation $\pRel_2$. However, if we have $\pRel_1(\pAtt_1, \pAtt_3)$
and $\pRel_2 (\pAtt_1, \pAtt_2)$ then
$\pRel_1 \bowtie \pRel_2$ joins tuples from $\pRel_1$ and $\pRel_2$ where
attribute $\pAtt_1$ has the same value in $\pRel_1$ and $\pRel_2$. 
%
Also, note that join is simply a syntactic sugar for selection of cross product,
that is $\pQ_1 \bowtie_{\pCond} \pQ_2 = \vSel [\pCond] {(\pQ_1 \times \pQ_2)}$.
%
The set operations, union and intersection, require two subqueries to have the same set of attributes
and simply apply the operation, either union or intersection, to the tuples returned by
the subqueries.
For example, if we have $\pRel_1(\pAtt_1, \pAtt_2)$ with 
tuples $\{(1,2)(3,4)\}$
and $\pRel_2(\pAtt_1, \pAtt_2)$ with tuples $\{(1,2),(5,6) \}$
then $\pRel_1 \cup \pRel_2 $ returns the tuples $\{(1,2), (3,4), (5,6)\}$.

%, except
%that again we allow conditions to be variational.
%
%A \emph{choice} encodes a variation point between two subquery alternatives based on a
%given feature expression, e.g., \chc[f_1\wedge f_2]{\vQ_l,\vQ_r} yields
%the results of $\vQ_l$ alternative for configurations where $f_1$ and $f_2$ are enabled,
%and in other configurations yields the results of $\vQ_r$ alternative. Note that the
%conditions $\vCond$ used by selections and joins also contain choices, and
%these behave similarly.
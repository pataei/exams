\section{Contributions and Outline of this Thesis}
\label{sec:contribution}

\TODO{contribution}

Our contributions in this paper address this challenge:
\begin{itemize}
%[leftmargin=*]
%\itemsep0em
\item We define the requirements of a a variational database framework, \secref{req}.
%[wide, labelwidth=!, labelindent=0pt, topsep=1pt]
%[leftmargin=*]
%\item To account for variation explicitly, we use a \emph{variation space} 
%and propositional formulas of features to refer to a subset of the space (\secref{encode-var})~\cite{ATW17dbpl}.
%\item We provide a framework to capture variation within a database using
%propositional formulas over  
%sets of features, called \emph{feature expressions}, following~\cite{ATW17dbpl}.
\item We define the \emph{variational database (VDB)} by incorporating 
variation directly in the database, \secref{vdb}.
%both the structure (schema) and
%content (tuples) of the database, introducing \emph{variational schemas}, \secref{vsch}, 
%and \emph{variational tables}, \secref{vtab}, and together \emph{VDBs},
%\secref{vdb}.
%, satisfying \textbf{N0} and first part of \textbf{N2}.
\item We define the 
%To express user information needs 
\emph{variational relational algebra} query language, \secref{vrel-alg}, 
its static type system, \secref{type-sys},
and \emph{variation-minimzation} rules, \secref{var-min}.
%a combination of relational algebra and 
%choice calculus~\cite{EW11tosem,Walk13thesis}, \secref{vrel-alg}.
%Users query a VDB by a \emph{variational query}, \secref{vq}.
%\item 
%%To make variational 
%\revised{
%%For more usability, w
%We define VRA's static type system, \secref{type-sys},
%and \emph{variation-minimzation} rules, \secref{var-min}.}
% queries more useable and easier to understand, respectively,
%by defining 
%a static type system, \secref{type-sys},
%and \emph{variation-minimiztion} rules, \secref{var-min}.
%to make it easier to understand and more useable. 
%rules to minimize 
%variation in v-query, \secref{var-min}, to provide better
%efficiency and usability. 
%This completes satisfiability of \textbf{N2}.
\item 
%To query a 
%variational database and receive clear results
%\revised{To interact with a VDB}
We implement a prototype of our framework, called  \emph{Variational Database Management System (VDBMS)}, on top a traditional DBMS, \secref{impl}.
We test VDBMS on previously developed use cases, \secref{exp-disc}.
%,
%satisfying all four needs: \textbf{N0}-\textbf{N3}.
%\textbf{N1}, \textbf{N2}, and \textbf{N3}.
\end{itemize}
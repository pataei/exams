\section{Summary of Contributions}
\label{sec:sum-contr}

The main contribution of this thesis is the variational database framework, 
a generic database framework that explicitly accounts for variation, 
and the variational database management system, a prototype of the VDB framework. 
%
We argued that variation is an orthogonal concern to databases
and while there are approaches that address it in specific contexts,
there is no fundamental technique to address it in every contexts, especially in 
intersection of contexts. Hence, we incorporated variation as 
propositional formulas (\secref{encode-var}) 
into the database while keeping
track of variation while querying the database by employing approaches such as annotation,
variational sets, and formula choice calculus, introduced in \secref{vset} and
\secref{fcc}, respectively.
%
%In \secref{encode-var}
%we presented how we encode the variation space and variants. Then, in
%\secref{vset} and \secref{fcc} we presented ways we incorporate variation into elements of
%a database. 


%vdb
In essence, VDB systematically places multiple relational database variants 
in a single database while tracking their corresponding configurations at both
the content and structure levels of a database (\chref{vdb}). 
%
This is a source of complexity that may impact understandability, as can be
observed in our use cases introduced in \chref{vdbusecase}. 
However, it also has several advantages: it
is generic, in the sense that any set of database variants and queries can be
encoded as a VDB and variational queries. Additionally, it enables direct association of variation
in a database to variation in other parts interacting with the database such as software.
%
 We discussed that explicitly encoding variation in databases
 allows  tracing variation between the program and data. It also empowers
 developers to check properties over a database to ensure that
their desired constraints over a database hold (\secref{vdbfprop}).
%Furthermore, such an encoding allows for 
%systematical checking the properties of database variants and ensuring
%the consistency of them .


%vq
We also defined a variational query language (VRA) to extract information from VDBs
(\secref{vrel-alg}) along with its denotational semantics in terms of the relational algebra 
semantics (\secref{vradensem}). 
%
VRA uses variational sets and formula choice calculus to incorporate
variation into the relational algebra. Although explicitly accounting for variation
in queries introduces more complexity in the language it provides a systematic
mechanism to ensure that the query follows the variation encoded in the database
as well as possibly imposing new variation (\secref{type-sys}). Still, this complexity
is alleviated in two ways: 1) the queries are not enforced to repeat the variation encoded in the 
database, since queries can automatically be explicitly annotated by schemas (\secref{constrain}),
and 2) the variation in queries are confluent, that is, the variation can move from one 
part of the query to another without changing its semantics due to syntactic equivalence
rules (\secref{var-min}). 
%
Importantly, VRA is variation-preserving both at the type and semantics level
(\secref{var-pres} and \secref{var-pres-sem}). 
That is, the corresponding variants of data is tracked throughout the execution of
the query.

%usecase
We provided two use cases that illustrate how software variation leads to
corresponding variation in relational databases (\chref{vdbusecase}). 
These use cases demonstrate
the feasibility of VDBs and variational queries to capture the data needs of variational
software systems.
%
We argued that effectively managing such variation is an open problem, and we
believe that these use cases will form a useful basis for evaluating
research that addresses it, such as our own VDBMS prototype.
%
The case studies were developed by systematically combining existing data
sources with software variation scenarios described in the literature. They
each consist of a variational schema describing the structural variation of
the database, the variational database itself containing the variational
content, and a set of variational queries that satisfy realistic information
needs over multiple variants of each database.
% 
These case studies can be used to 1) evaluate approaches and systems
 attempting to manage any kind of variation in databases, 2) learn how a
 variational database can be generated from a scenario that describes such
 variation, and 3) design a system that automatically generates a variational
 database from non-variational databases and their corresponding variant. In
 particular, we used these use cases to evaluate VDBMS (\secref{exp}). 
% It would be interesting to investigate 1)
% how database systems that manage a specific kind of variation deal with
% variational databases and 2) how database systems that account for different
% kinds of variation can be improved to manage more specific kinds of
% variation.


%vdbms
Finally, we implemented the VDB framework and VRA query language as
a variational database management system and compared different approaches
of running a variational query on an underlying relational database engine 
(\chref{vdbms}). Our experiments demonstrated that our different SQL generator
approaches are comparable while filtering out tuples with unsatisfiable presence
conditions takes a significantly long time, especially as the number of returned
tuples grows. 

%Finally, we illustrated VDBMS: the implementation of our framework
%and compared its performance against simulated approaches that
%could be used to extract information from a database in a variational
%scenario.
%The most important contribution of this paper is the expressiveness of
%our framework (VDB and VRA) and our system (VDBMS).



\section{Summary of Contributions}
\label{sec:sum-contr}

\fromppr{from vamos}
We provide two use cases that illustrate how software variation leads to
corresponding variation in relational databases. These use cases demonstrate
the feasibility of VDBs and v-queries to capture the data needs of variational
software systems.
%
We argue that effectively managing such variation is an open problem, and we
believe that these use cases will form a useful basis for evaluating
research that addresses it, such as our own VDBMS framework.
%
% VDB
%and VRA.
%
%The case studies were developed by systematically combining existing data
%sources with software variation scenarios described in the literature. They
%each consist of a variational schema describing the structural variation of
%the database, the variational database itself containing the variational
%content, and a set of variational queries that satisfy realistic information
%needs over multiple variants of each database.


VDBs encode variation explicitly in the structure and content of databases.
%
This is a source of complexity that may impact understandability, as can be
observed in our use cases. However, it also has several advantages: it
is general in the sense that any set of variant databases and queries can be
encoded as a VDB and v-queries, and it enables directly associating variation
in databases to variation in software.
%
By applying variational typing to variational queries, this generality does not
come at the cost of safety. Future work can explore how tooling can mitigate
the usability concerns using techniques that have been developed in the SPL
community.

% We discussed that explicitly encoding variation in databases
% allows  tracing variation between the program and data. It also empowers
% developer to check properties over their database and queries to ensure that
% constraints over a database holds and queries are well-behaved.

%p: taken from rel work
% Although we have focused on variational databases to support SPL development,
% the broader motivation of \emph{effectively computing with variability} is at
% the heart of our work. This is why VDBs support not only structural variation
% but also content-level variation. Also, while variational queries can be
% statically configured in the same way that SPLs typically are, our prototype

%%VDBMS implementation also supports directly executing variational queries on
%%variational databases to yield variational results.


% These case studies can be used to 1) evaluate approaches and systems
% attempting to manage any kind of variation in database, 2) learn how a
% variational database can be generated from a scenario that describes such
% variation, and 3) design a system that automatically generates a variational
% database from non-variational databases and their corresponding variant. In
% particular, we use these use cases to evaluate our Variational Database
% Management System~\cite{vldbArXiv}. It would be interesting to investigate 1)
% how database systems that manage a specific kind of variation deal with
% variational databases and 2) how database systems that account for different
% kinds of variation can be improved to manage more specific kinds of
% variation.

\fromppr{vldb}
%\point{briefly iterate contributions and how they solve needs.}
%\revised{
We argued that there is need to consider variation as an 
orthogonal concern to databases for two reasons:
1) Context-specific solutions do not provide their users with all 
variational information needs they may have, e.g., in \secref{intro}
we analyzed the lack of variational need support for database-back SPL.
As another example, imagine a user wants to know the data provenance of some
data they got from an integrated database 
 (i.e., what data source does the data belong to). This is a variational need that 
data integration systems cannot satisfy.
2) New variational scenarios could appear, often from combination of other data variation instances 
such as our motivating example in \secref{mot}.
%}


%We argued that variation is an orthogonal concern to databases
%and while there are approaches that address it in specific contexts
%there is no fundamental technique to address it in every contexts, especially in 
%intersection of contexts. Hence, we incorporated variation as 
%propositional formulas into database and its schema while keeping
%track of it while querying. 
%\revised{
We studied variational information needs, i.e., needs that appear in a 
variational context independent from the specific use case and 
we defined a generic framework that incorporates variation into all
parts of a database and we showed how our framework satisfies 
all variational information needs throughout the paper.
%}
%illustrated how we satisfy them throughout the paper. We also
%demonstrated how VDBMS can be applied to instances of 
%variation in databases.
%\revised{
Finally, we illustrated VDBMS: the implementation of our framework
and compared its performance against simulated approaches that
could be used to extract information from a database in a variational
scenario.
The most important contribution of this paper is the expressiveness of
our framework (VDB and VRA) and our system (VDBMS).
%}


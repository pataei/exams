\subsection{New Instance of Data Variation in Industry}
\label{sec:industry-ex}

\TODO{revise base on what you've explained so far in thesis.}

%\point{briefly iterate contributions and how they solve needs.}
New variational scenarios could appear, either from combination of other scenarios
or even a new scenario could reveal itself. For example, the following is
a scenario we recently discussed with an industry contact:
%
A software company develops software for different networking companies and
analyzes data from its clients to advise them accordingly. 
%
The company records information from each of its clients' networks in databases
customized to the particular hardware, operating systems, etc.\ that each
client uses.
%
The company analysts need to query information from all clients who agreed to
share their information, but the same information need will be represented
differently for each client.
%
This problem is essentially a combination of the SPL variation problem (the
company develops and maintains many databases that vary in structure and
content) and the data integration problem (querying over many databases that
vary in structure and content). However, neither the existing solutions from
the SPL community nor database integration address both sides of the problem.
%
Currently the company manually maintains variant schemas and queries, but this
does not take advantage of sharing and is a major maintenance challenge.
% , for the reasons SPL researchers are familiar with.
%
With a database encoding that supports explicit variation in schemas, content,
and queries, the company could maintain a single variational database that can
be configured for each client, import shared data into a VDB, and write
v-queries over the VDB to analyze the data, significantly reducing redundancy
across clients.
\chapter{Introduction}
\label{ch:intro}

%\TODO{points: 
%- def variation in db and is everywhere
%- instances but context specific solution don't suffice
%- thus a generic framework that addresses problems}
%
%\TODO{
%- dimension, behaviour
%- instances but context specific solution don't suffice
%- example of instances
%  - well-studied: schema evolution
%  - partial: SPL
%  - new instance either out of combination of other or new
%- generic framework to instantiate for each instance and address all variational needs.}


Managing variation in databases is a perennial problem in database literature
and appears in different forms and 
contexts~\cite{curateVdata,ALW21vamos,ready17cidr,clams16sigmod,datahub15cidr}
and it is unavoidable~\cite{dbDecay16Stonebraker}.
%
Variation in databases mainly arises when multiple database instances 
conceptually represent the same database but differ
in their schema, content, or constraints.
%
Specific kinds of variation in databases have been addressed by 
context-specific solutions, such as
%Existing work on variation in databases addresses specific  kinds of 
%variation in a context and proposes solutions specific to the context of the
%variation, such as  
schema evolution~\cite{SchEvolRA90McKenzie, 
schVersioning97Castro, tempSchEvol91Ariav, tsql95Snodgrass, 
prima08Moon}, 
data integration~\cite{dataIntegBook}, 
and database versioning~\cite{datasetVersioning,dbVersioning}.
%
However, there are no generic solution that addresses all kind of variation
in databases. We motivate the need for a generic solution to variation
in databases in \secref{mot}.
%
%These problems have a context-specific solution. However, different
%kinds of variation in databases behave differently. 
%For example, as illustrated in 
%\figref{behave}, the variation
%in schema evolution evolves linearly over time while the variation in data
%integration comes from multiple input databases creating one output database. 
%Thus, a context-specific solution for one kind of variation in databases cannot
%be used to address another kind of variation in databases. This becomes 
%problematic when different kinds of variation in databases interact, creating
%a new kind of variation in databases. 


The major contribution of this thesis is the \emph{variational database} framework,
a generic relational database framework that explicitly accounts for database variation,
and \emph{variational relational algebra}, a query language for our framework
that allows for information extraction from a variational database. 
The framework is generic because it can encode any kind of variation
in databases. Additionally and more importantly, it is designed such that it can satisfy
any information need that a user may have in a variational database scenario. 
Based on information needs in a variational database scenario, we define 
requirements of a variational database framework in \secref{mot} and throughout the
thesis, we show that our framework satisfies these requirements.

In addition to a formal description of the variational database framework and
variational relational algebra and some theoretical results,
this thesis distributes and presents two variational data sets (including both a variational
database and a set of queries) as well as a \emph{variational database management system}
that implements the variational database framework as an abstraction layer on top of a 
traditional relational database management system in Haskell.
\secref{contribution} enumerates the specific contributions in the contest of an outline
of the structure of the rest of the thesis. 

\fromppr{vamos. incorporate these into mot.}
Just as variation is ubiquitous in software, it is also ubiquitous in the
relational databases that software systems rely on.
%
Database researchers have long studied different kinds of variation in
databases, such as through work on database
evolution~\cite{schVersioning97Castro,SchEvolRA90McKenzie,prima08Moon},
database versioning~\cite{datasetVersioning,dbVersioning}, and data
integration~\cite{dataIntegBook}.
%
However, work in the databases community does not identify \emph{variation} as
a general, orthogonal concern that arises in many different contexts.
%
This is a problem since it means that tools and techniques developed for one
kind of database variation cannot be easily reused for another. More concretely
for software developers, and especially software product line (SPL)
practitioners, the lack of a general representation of database variation means
that many kinds of variation that arise in software do not map cleanly onto
corresponding variation encodings in the databases they use.


% while the software product line (SPL) community has identified \emph{variation}
% or \emph{variability} as a general, orthogonal concern that arises in many
% different contexts, work on different flavors of data variation remains
% distinct.

In contrast, SPL researchers have invested significant effort in studying
\emph{variation} (or \emph{variability}) as a general phenomenon and concern in
software.
%
Although many kinds of software variation are possible, most can be roughly
organized into variation in \emph{time} or \emph{space}~\cite{Thu19vv}.
Variation in \emph{time} refers to the evolution of a system and is addressed
by revision control systems and configuration management~\cite{Dart91}, while
variation in \emph{space} refers to the simultaneous development and
maintenance of related systems with different feature sets and is the
traditional focus of research on SPLs~\cite{FOSPL16}.
%
Multiple lines of work in the SPL community have sought to develop general
purpose representations of variation in software, such as delta-oriented
programming~\cite{Schaefer10dop} and the choice calculus~\cite{EW11tosem},
among others. These can be used to unify both kinds of variation in software,
enabling reuse of analyses across dimensions and enabling new kinds of analyses
that consider variation in both time and space simultaneously~\cite{Thu19vv}.

Because the existing data variation models developed in the databases community
do not align with all of the variation scenarios that arise in software, SPL
researchers have identified the need for more general encodings of variation in
data models and database schema. To this end, they have developed
encodings of data models that allow for arbitrary variation by annotating
different elements of the model with features from the
SPL~\cite{skrhas09DBIS,slrs12CAiSE,ad11varDataModel}.
%
However, these solutions address only variation in the data model but do not
extend to the level of the data or queries. The lack of variation support in
queries leads to unsafe techniques such as encoding different variants of query
through string munging, while the lack of variation support in data precludes
testing with multiple variants of a database at once.
% This impacts DBAs, data scientists, and developers
% significantly~\cite{dbSPLevolve}.

\begin{comment}
Effectively managing variation is a fundamental challenge of software
engineering. Research on CM and SPLs
have developed numerous representations and strategies for effectively managing
different kinds of variation in software. However, these solutions typically do
not extend to managing variation in the artifacts and systems that software
uses.
%An especially tricky aspect is managing corresponding variation in
%the external artifacts and services that a software system interacts with.
%
Here, we focus on \emph{relational databases}.
%one kind of external artifact that is ubiquitous in
%software but not well-supported by current approaches to managing software
%variation: \emph{relational databases}.
%
Different variants of a software system have different information 
needs~\cite{skrhas09DBIS}, which
implies a corresponding need for variation in the structure and content of the
relational databases that these systems use and rely on. This clearly results in 
having variation in the database used to develop a SPL.
%
%While research on software product lines (SPLs) \cite{fospl} 
%has led to a variety of representations and techniques for safely working with
%many variants of a software system, these solutions don't extend to relational
%databases.

In a database-supported SPL, 
typically a number of strategies are employed to
accommodate the different information needs of different variants.
%
The first is that a different relational database may be \emph{specified and
created per-variant}, according to the information needs of each
variant~\cite{marco13featureAdaptSch}. 
This approach is labor-intensive and difficult to maintain
since changes need to be propagated across variants manually.
%
The second strategy is to define a single \emph{global schema that applies
to all variants}~\cite{batini86dbSchIntegAnalysis}. 
This strategy is more efficient to maintain compared to the previous approach
but is still hard to maintain,
especially in face of SPL evolution. Due to lack of separation of concerns
and suboptimal traceability of requirements to database elements~\cite{skrhas09DBIS}
it is also complex, hard to understand, and unscalable~\cite{slrs12CAiSE}. 
Additionally, it suffers from design limitation and 
error-proneness since parts of the schema will be irrelevant to each variant,
resulting in losing database's integrity constraints~\cite{slrs12CAiSE}.
%Irrelevant attributes are typically populated by NULL-values, which may later
%be referenced since it is impossible to check or enforce that queries in each
%variant use the database in a safe and consistent way.
%
The third strategy is to define a \emph{variable data model}~\cite{skrhas09DBIS, 
slrs12CAiSE, ad11varDataModel} which models a database schema 
(usually as an Entity-Relation model) with
annotations of features from SPL to indicate their variable existence. 
This approach addresses problems of the previous approach, however,
it does not address the variation that appears in queries and data. 
Thus, developers have to write the required information need as a
query encoded as a string per variant. Not only this is labor-some but
also due to the nature of queries being encoded as strings there is no
static check to ensure that queries are type correct. 
\end{comment}

\structure{use the following paragraph and items for contribution maybe.}
In previous work, we have developed \emph{variational databases (VDBs)} and
\emph{variational queries}~\cite{ATW18poly,ATW17dbpl} to encode general
variation in the representation and use of relational databases. Our work
extends ideas developed in the SPL community to 
% the creation, management, and querying of
relational databases.
%
Conceptually, a VDB represents potentially many different plain relational
databases at the same time. Similarly, a variational query represents
potentially many different queries, each one corresponding to a variant of the
VDB.
%
Together, VDBs and variational queries enable safely and efficiently working
with many variants of a relational database at once, and reliably integrating
the variants of a database with the corresponding variants of an SPL.
%
We are currently implementing these ideas in \emph{VDBMS}, a practical
implementation of variational databases as a lightweight wrapper on top of a
traditional relational database management system.


However, the generic and expressive approach of VDB in dealing with database
variation creates new complexity and costs which raises the question: Is
explicitly encoding variation in databases actually a good idea?
%
With this question in mind, in this paper, we:
%
\begin{itemize}
%
\item Show the feasibility of VDB by systematically generating two VDBs from
realistic scenarios of database variation in time and space (\secref{db}).
%
\item Illustrate the applicability of variational queries by encoding
information needs for the developed VDBs using scenarios described in the
literature (\secref{q}).
%
\item Discuss the tradeoffs of explicitly encoding variation in databases
(\secref{dis}).
% and future research directions (\secref{dis}).
%
\end{itemize}
\fromppr{end of from vamos. remember to read the commented out part too.}


\section{Motivation and Impact}
\label{sec:mot}

\TODO{include a more elaborate version of vldb intro with maybe parts of vamos. 
also include the requirements here and elaborate more. you can also refer to sections
that address each requirement. finally introduce the mot ex as a concrete running
example throughout the thesis. also industry ex as an example that came up 
in conversation with industry contact.
}

%\structure{openning}
%Variation in databases appears abundantly in different forms and contexts.
Managing variation in databases is a perennial problem in database literature
and appears in different forms and 
contexts~\cite{curateVdata,ALW21vamos,ready17cidr,clams16sigmod,datahub15cidr}.
%
Variation in databases mainly arises when multiple database instances 
conceptually represent the same database, but, differ
in their schema, content, or constraints. 
%The problem
%of managing variation within a database is not new. 
Existing work on variation in databases addresses specific  kinds of 
variation in a context and proposes solutions specific to the context of the
variation, such as  
schema evolution~\cite{SchEvolRA90McKenzie, 
schVersioning97Castro, tempSchEvol91Ariav, tsql95Snodgrass, 
prima08Moon}, 
data integration~\cite{dataIntegBook}, 
and database versioning~\cite{datasetVersioning,dbVersioning}.

%
Unfortunately, some of these tools do not address all their user's needs.
Furthermore, they are all \emph{variation-unaware}, i.e., they dismiss that
they are addressing a specific kind of a bigger problem. Thus, not only they 
cannot address a new kind of database variataion but they also cannot 
address the interaction of different kinds of database variation since they
assume that each kind of variation is \emph{isolated} from another
kinds. This costs database researchers
to develop a new system for every individual kind of data variation
and forces developers and DBAs to use manually unsafe workaround.
%Specific kinds of this problem have been extensively studied including 
%schema evolution~\cite{SchEvolRA90McKenzie, 
%schVersioning97Castro, tempSchEvol91Ariav, tsql95Snodgrass, 
%prima08Moon}, 
%data integration~\cite{dataIntegBook}, 
%and database versioning~\cite{datasetVersioning,dbVersioning},
%where each instance has a context-specific tool.
%Lots of research has
%been done on some cases of this problem such as schema evolution,
%data integration, and database versioning; 
%resulting in well-supported context-specific systems.

%\point{Schema evolution is an instance of variation in databases
%that is well-supported.}
%Explains how schema evolution (which is unavoidable) 
%is an instance of variation in databases.
%And mentions some of the current solutions, emphasizing that
%they cannot address other instances of the problem.
For example, schema evolution is a kind of schematic variation in databases
over time
that is well-supported~\cite{SchEvolRA90McKenzie, 
schVersioning97Castro, tempSchEvol91Ariav, tsql95Snodgrass, 
prima08Moon}.
Changes applied to the schema over time are \emph{variation} 
in the database and every time the database evolves, a new
\emph{variant} is generated.
%which needs to coexists in parallel
%with other variants.
Current solutions addressing schema
evolution dismiss that it is a kind of variation, thus,
they \emph{simulate} the effect of variation by
relying on temporal nature of schema evolution and using
timestamps~\cite{SchEvolRA90McKenzie, schVersioning97Castro, 
tempSchEvol91Ariav, tsql95Snodgrass} 
or keeping an external file of time-line history of 
changes applied to the database~\cite{prima08Moon}. 
%These approaches only consider variation in time and do not
%%However, none of them 
%incorporate the time-based changes into
%the database directly, rather they \emph{simulate} the effect of these changes.
%,
%resulting in brittle systems.

Unlike schema evolution, some kinds of database variation are
partially supported. For example, while developing software for 
multiple clients simultaneously, an approach called software 
product line (SPL)~\cite{splBook}, a different database schema is 
required for each client  due to 
client's different business requirements and environments~\cite{skrhas09DBIS}.
% 
SPL researchers have developed
encodings of data models that allow for arbitrary variation by annotating
different elements of the model with features from the
SPL~\cite{skrhas09DBIS,slrs12CAiSE,ad11varDataModel}.
Thus, they can generate a database schema  \emph{variant} for 
each software \emph{variant} requested by a client and generated by the SPL. 
%
However, these solutions address \emph{only} variation in the data model but do not
extend to the level of the data or queries. The lack of variation support in
queries leads to unsafe techniques such as encoding different variants of query
through string munging, while the lack of variation support in data precludes
testing with multiple variants of a database at once.

%\revised{
%However, these context-specific tools do not address all information
%needs of their users which costs database administrators and developers to use 
%manually unsafe workarounds.
%%
%%there are instances of the general problem that are
%%not well-studied, resulting in using manual approaches that burden database experts.
%Moreover, these tools consider instances \emph{isolated} from each other.
%Consequently, they 
%cannot address the same instance when it interacts with another instance and creates 
%a new instance of data variation.
%%For example, the variation of an evolving integrated database cannot be managed by
%%either a data integration system or schema evolution system.
%The lack of a \emph{variation-aware} database system costs database researchers
%to develop a new system for every individual instance of data variation. 
%In this paper, we provide a variation-aware database framework that explicitly expresses 
%variation in both the database and queries. Such a system can be instantiated for
%any instance of data variation or combination of them and it addresses all variational information needs safely.
%%each instance of data variation and manage any combination of data variation instances.
%%Additionally, it can potentially be optimized for an instance of data variation instead of developing a new system.
%}
%Moreover, different kinds of variation can interact\revised{, creating new kinds of data variation, }
%which cannot be addressed
%by current approaches due to lack of a general solution to managing 
%different kinds of variation in databases.
%In this paper, we provide a fundamental solution to managing
%variation in databases by considering variation 
%explicitly 
%%as 
%%a \emph{first-class citizen}
%in the framework, allowing for encoding different kinds of variation
%\revised{
%in both the database and queries.}

%\structure{the following paragraphs are funnel}
%\point{Schematic and content-level variation in DBs that conceptually 
%represent the same data.}
%\safespace{
%Variation in databases arises when multiple database instances 
%conceptually represent the same database, but, differ
%slightly either in their schema and/or content. 
%%These database
%%instances coexist in parallel.
%The variation in schema and/or content occurs in two \emph{dimensions}:
%time and space. Variation in space refers to different variants of database that
%coexist in parallel while variation in time refers to the evolution of 
%database, similar to variation observed in software~\cite{Thu19vv}. 
%Note that variation in a database can occur due to both dimensions
%at the same time.
%}



%%\point{Database-backed software produced by SPL is an instance
%%of variation in databases that is poorly managed in practice.}
%%Explains SPL briefly and how variation appears in databases used to
%%store data for software produced by SPL. As well as how poorly it is 
%%managed in practice.
%Database-backed software produced by software product line (SPL) 
%is an example variation in databases in space and is \revised{partially}
%supported. 
%SPL is an
%approach to developing and maintaining software-intensive systems 
%in a cost-effective, easy to maintain manner by accommodating variation
%in the software that is being reused~\cite{splBook}. 
%An SPL produces multiple software products for different business requirements and environments.
%%%The products of a SPL pertain to a
%%%common application domain or business goal. 
%%%They also have a common
%%%managed set of features that describe the specific need for a product. 
%%%They share 
%%%a common codebase which is used to produce a product with respect to its set of 
%%%selected (enabled) features
%%In SPL, a common codebase is shared and used to produce products w.r.t.
%%a set of selected (enabled) features~\cite{splBook}. 
%%% from the SPL feature set designed to describe
%%%specific needs for products~\cite{splBook}. 
%%Different products of an SPL typically have different
%%sets of enabled features or are tailored to run in different environments. 
%These
%differences impose different data requirements which creates variation in space 
%in the shared database used in the common codebase. 
%%Hence, 
%%each product has its own database variant.
%%For example, different legal
%%requirements often require tracking different data in products tailored for use
%%in different countries or regions~\cite{splBook}.
%%Different data requirements results in different desired data which 
%%creates database variants. 
%%
%%
%%
%\revised{
%Since the existing data variation models developed in the databases community
%do not align with all of the variation scenarios that arise in software, SPL
%researchers have identified the need for more general encodings of variation in
%data models and database schema.
%%
%%Since the variation in this case is mainly exclusion/inclusion of attributes/relations
%%for different variants of the software~\cite{ATW18poly}, researchers
%%%the database variants corresponding to software products differ mainly 
%%in their schema, a relation/attribute can either be included or excluded for a 
%%specific software product~\cite{ATW18poly}.
%%
% To this end, they 
% have developed
%encodings of data models that allow for arbitrary variation by annotating
%different elements of the model with features from the
%SPL~\cite{skrhas09DBIS,slrs12CAiSE,ad11varDataModel}.
%%
%However, these solutions address \emph{only} variation in the data model but do not
%extend to the level of the data or queries. The lack of variation support in
%queries leads to unsafe techniques such as encoding different variants of query
%through string munging, while the lack of variation support in data precludes
%testing with multiple variants of a database at once.
%}
%%The variation is in the form of exclusion/inclusion of tables/attributes based on
%%selected features for a product~\cite{ATW18poly}.
%%%
%%In practice, software systems produced by a SPL are accommodated with a database that
%%has all attributes and tables available for all variants-- a database with universal schema~\cite{ATW18poly}. 
%%Unfortunately, this approach is
%%inefficient, error-prone, and filled with lots of null values since not all attributes and tables
%%are valid for all variant products. A possible solution to this could be defining views on 
%%the universal database per software variant and write queries for each variant against its 
%%view~\cite{ATW18poly}.
%%However, this is burdensome, expensive, and costly to maintain since it 
%%requires developers to generate and maintain numerous view definitions
%%in addition to manually generating
%%and managing the mappings between views and the universal schema for each product.


%\point{Schema evolution meets SPL evolution and results in databases
%used in SPL that their schemas evolve over time.}Thu19vv, splEvolveBP14
%Mentions that SPL evolution is hot topic. Part of this evolution is database
%evolution. This is where two instances of managing variation in databases
%interact and even the well-supported systems for schema evolution cannot
%address it. 
The situation exacerbates even more when two kinds of variation
interact and create a new kind of database variation: the evolution of
the database used in development of software by an SPL.
%in simultaneous development of multiple software variants.
This is due to the evolution of software and its
artifacts; an inevitable phenomena~\cite{dbSPLevolve}. 
This is where even previous context-specific solutions like schema evolution
tools fail.
%The lack of support for variation in databases exacerbates even more when
%software evolves. 
%Software evolution is inevitable, so is its artifacts evolution, 
%including databases~\cite{dbSPLevolve}.
%Variation in software development is unavoidable and 
%This is where two kinds of database  of managing 
%variation in databases (schema evolution and database-backed software 
%developed by SPL) interact \revised{and in fact they create a new kind of variation}.
% and variation occurs both in time and space. 
%While there are solutions to schema evolution
%they cannot adapt to a new situation because they only provide a solution to
%variation of databases in time and cannot encode the interaction of database variation in
%time and space.
%dismiss variation of databases in space, 
%i.e., they dismiss that database variants that have been created due to variation
%in time must coexist in parallel (a property of variation in 
%space~\cite{Thu19vv}). 
We motivate this case through an example in \secref{mot}.
%We use this case as our motivating example in \secref{mot} and explain it 
%in more details.

%\structure{knowledge gap + challenge}
%\point{Reiterate the knowledge gap. Introduce the challenge.} 
%Reiterates the knowledge gap. Explains the challenge:
%The challenge then becomes encoding variation in databases
%that can model different instances and satisfy different specialists' needs 
%at different stages (like development, deployment, information extraction, etc).
%%Explains the challenge as incorporating variation in databases s.t. all 
%%database variants are gathered in one place. 
%Also, itemizes the users' 
%needs in such an environment: 
%1) Query some/all variants simultaneously and selectively.
%2) Track the original variant of a piece of data and the variation applied
%to it through a query.
%3) Deploy the database and its queries to a variant of it.
As we have shown, 
variation in databases is abundant and inexorable~\cite{dbDecay16Stonebraker};
impacts DBAs, data analysts, and developers significantly~\cite{dbSPLevolve}; 
and appears in different contexts. Current methods
% are
% not variation-aware and 
%address specific kinds of 
%variation in databases and they 
%in time and space, however, they all consider only one of these dimensions and
are all extremely tailored to a specific context. Consequently, they fail to address
the interaction of their specific kind of database variation with other kinds.
%a new instance of variation appears.
%the two dimensions and different contexts interact. 
Hence, the challenge becomes defining a variation-aware database that can 
%offering a fundamental generic framework that 
%incorporating variation in databases s.t. it 
can model different kinds of variation
in different contexts such that it satisfies different specialists' needs at different stages,
e.g., development, information extraction, deployment, and testing. 

%In this section, we define the requirements that make a database framework
%variational. 
For a variational framework  to be expressive enough to encode
any kind of variation in databases, it must satisfy some requirements.
Thus, we define the requirements that make a database framework
variational through studying different kinds of variation in databases.  
%These requirements distinguish a database framework that 
%simulates the effect of variation compared to one that is variational. 
A variational artifact, including databases, encompasses multiple 
\emph{variants} of the artifact and provides a way to distinguishes 
between different variants that are all encoded in one place, the 
variational artifact. It also provides a way to get the variants from
the variational artifact, we call this \emph{configuration function}.
%
These requirements that help distinguish a database framework that 
simulates the effect of variation compared to one that is variational are
listed below:

%Our example demonstrates how instances of variation
%% in time and space
%arise and interact with each other, an unavoidable consequence of modern software
%development.
%We categorize the needs of a variational database framework at different stages
%such as development, test, and deployment and
%use our motivating example to illustrate them. Throughout the paper,
%we show how our framework achieves these needs via examples, proofs, and tests.
%%To be concrete throughout the paper we itemize
%%the needs of users working with a database with variation
%%through the needs of SPL developers and \revised{analysts and} DBAs:
%% for our motivating example.
%%These needs cover all general needs when variation appears in databases
%%because our motivating example considers variation in both possible 
%%dimensions; time and space:
%
\begin{enumerate}
%[wide, labelwidth=!, labelindent=0pt, topsep=1pt]
%[leftmargin=*]
\itemsep0em
\item [\textbf{(\nZero)}]
\emph{All database variants must be accessible at a given time}.
For example, in our motivating example, a company that started with 
\vOne\ of the \basic\ schema evolves over time but its different 
branches adopt the new schema at different paces, thus, it 
requires access to all variants of the \basic\ schema. 
%For example,
%SPL developers need to access all database variants while
%writing code to be able to 
% extract information for 
% all software variants
%they are developing. 
%
%Hence,
%\emph{users need to have access to all database variants at a given time}.
%
%\secref{vdb} explains how a VDB achieves this.
%%\tabref{mot-vsch} shows how v-schema achieves
%\exref{vsch} shows how v-schema achieves
%this at schema level for a part of our motivating example.
%%make sure the following is the case.
%%\TODO{and \secref{exp-disc} discusses two databases
%%that achieve this at both content and schema level.}
%
\item [\textbf{(\nOne)}]
\emph{The query language must allow for querying
%One must be able to query 
multiple database variants simultaneously.
Additionally, it must allow for filtering tuples to specific variants.}
That is, the framework must provide a query language that allows users to query multiple
database variants at the same time in addition to giving them the freedom to choose
the variants that they want to query. 
For example, an SPL tester that is testing a piece of code for the not highlighted 
variants of the software in \tabref{mot} needs to write queries that exclude the 
 variants associated with yellow cells of \tabref{mot}. 
%For example,
%depending on what component they are working on,
%\revised{SPL testers} need to be able to query all or some of the variants.
%
%Hence, 
%\emph{users need to query multiple database variants simultaneously and selectively}.
%
%\secref{vq} shows how our query language achieves this.
%% by introducing variation into queries, \secref{vq}.
%\exref{vq-specific} and \exref{vq-same-intent-mult-vars} illustrate this 
%for our motivating example.
%
\item [\textbf{(\nTwo)}]
\emph{Every piece of data must clearly state the variant it belongs to and 
this information must be kept throughout the entire framework.}
Continuing the example of the SPL tester, they need to know the variant that
some results belong to in order to be able to debug the software correctly and
accordingly. 
%The framework needs to 
%keep track of which variants a piece of data belongs to and ensuring that 
%it is maintained throughout a query}.
%For example,
%given that one can access all database variants and that they
%can query all variants simultaneously, for test purposes, 
%analysts need to know which variant a tuple belongs to.
%
%Hence,
%\emph{the framework needs to 
%keep track of which variants a piece of data belongs to and ensuring that 
%it is maintained throughout a query}.
%
%\revised{\secref{vdb} and \secref{type-sys} illustrate how our framework achieves this.}
%%We call this \emph{variation-preserving} property and }
%%Storing variants that a tuple belongs to in a VDB achieves the first part, \secref{vdb}, and
%%VRA's type system ensures the second part, \secref{type-sys}.
%\exref{var-pres} illustrates this for a given query.
%\TODO{we should show this for content level too. but for this submission we can say that 
%all approaches account for this and don't lose that info. discuss with Eric. say that it has been tested for all queries but not proved.}
%
\item [\textbf{(\nThree)}]
\emph{The variational database must provide a way for generating database
and query variants.}
For example, the SPL developers need to deploy the management software for each client,
thus, they need to configure the database schema and its queries in the code 
for each software variant.  
%Users need to deploy one variant of the database and its associated queries}.
%For example,
%SPL developers need to deploy the database and its queries
%to generate a specific software product for a client based on their
%requested features. 
%
%Hence,
%\emph{users need to deploy one variant of the database and its associated queries}.
%
%We define a \emph{configure} function for a VDB and its elements, \figref{vdb-conf}, 
%in addition to queries, \figref{v-alg-conf-sem}, that achieves this. 
%\exref{conf-vq} illustrate deploying an example query.
%
\end{enumerate}

Throughout the paper,
we show how our framework satisfies these requirements 
via examples, proofs, and tests.
\subsection{Motivating Example}
\label{sec:motex}

\TODO{combination of instances, behaviours, and dimensions}


\begin{table}[t]
\caption{Schema variants of the employee database developed for multiple software variants by an SPL.
%Employee schema evolution of a database for a SPL.
%%(evolution features \vOne -- \vFive\ for employee part of the schema (the left schema column), evolution features \tOne -- \tFive\ for education part
%%of the schema (the right schema column), and \edu\ feature representing the education feature of SPL). 
%A feature (a boolean variable) represents 
%inclusion/exclusion of tables/attributes.  
%The dash-underlined attributes in the basic schema 
%have \emph{variational} present in their relation, i.e.,
%they only exist in their relation when \edu\ = \t.
%%in left schema column only exist when the \edu\ feature of SPL is enabled. 
%%This table represents 1056 possible schema variants:
%%\ensuremath{2^5} schema variants when \edu\ is disabled 
%%(by enabling any combination of \vOne--\vFive) 
%%in addition to \ensuremath{2^5 \times 2^5} when \edu\ is enabled.
}
\label{tab:mot}
\begin{center}
\small
%\footnotesize
%\scriptsize
\begin{tabular} {| c | l || l | c |}
\hline
\textbf{\tiny Temporal} & \multicolumn{2}{ c |}{\textbf{Database Schema Variants for SPL Software Variants}} & \textbf{\tiny Temporal}\\
\cline{2-3}
\textbf{\tiny Features} & \multicolumn{1}{ c ||} {\basic} & \multicolumn{1}{ c |} {\educational} & \textbf{\tiny Features}\\
\hline
%\cline{2-3}
%\cline{2-3}
%\hhline{-==}
\multirow{3}{*}{\vOne} &  \engemp\ (\empno, \name, \hiredate, \titleatt, \deptname) & 
\course\ (\cname, \tno) & \multirow{3}{*}{\tOne}\\
& \othemp\ (\empno, \name, \hiredate, \titleatt, \deptname)  & \student\ (\sno, \cname) &\\
& \job\ (\titleatt, \salary) &  &\\
\hline
\multirow{2}{*}{\vTwo} & \cellcolor{yellow}{\empacct\ (\empno, \name, \hiredate, \titleatt, \deptname)} & \course\ (\cno, \cname, \tno) & \multirow{2}{*}{\tTwo}\\
%\cdashline{2-3}
& \cellcolor{yellow}{\job\ (\titleatt, \salary)} & \student\ (\sno, \cno) & \\
\hline
\multirow{4}{*}{\vThree} & \empacct\ (\empno, \name, \hiredate, \titleatt, \deptno) & \cellcolor{yellow}{\course\ (\cno, \cname)} & \multirow{4}{*}{\tThree}\\
& \job\ (\titleatt, \salary) & \cellcolor{yellow}{\teach\ (\tno, \cno)} &\\
& \dept\ (\deptname, \deptno, \managerno) & \cellcolor{yellow}{\student\ (\sno, \cno, \grade)} &\\
& \empbio\ (\empno, \sex, \birthdate) &\cellcolor{yellow}{} &\\
\hline
\multirow{4}{*}{\vFour} & \empacct\ (\empno, \hiredate, \titleatt, \deptno, \dashuline{\isstudent}, \dashuline{\isteacher}) & \ecourse\ (\cno, \cname) & \multirow{4}{*}{\tFour}\\
& \job\ (\titleatt, \salary) & \course\ (\cno, \cname, \timeatt, \class) & \\
& \dept\ (\deptname, \deptno, \managerno) & \teach\ (\tno, \cno) & \\
& \empbio\ (\empno, \sex, \birthdate, \name) & \student\ (\sno, \cno, \grade) & \\
\hline
\multirow{4}{*}{\vFive} & \empacct\ (\empno, \hiredate, \titleatt, \deptno,  \dashuline{\isstudent}, \dashuline{\isteacher}, \salary) & \ecourse\ (\cno, \cname, \deptno) & \multirow{4}{*}{\tFive}\\
& \dept\ (\deptname, \deptno, \managerno,  \dashuline{\studentnum}, \dashuline{\teachernum}) & \course\ (\cno, \cname, \timeatt, \class, \deptno) & \\
& \empbio\ (\empno, \sex, \birthdate, \fname, \lname) & \teach\ (\tno, \cno) & \\
&& \take\ (\sno, \cno, \grade) & \\
\hline
\end{tabular}
\end{center}
\end{table}

%\arashComment{Section 2 has a useful example but it is unnecessarily too complicated as combines SPL and schema evolution right off the bat.
%1) If you want to use both, you may start with one and then enrich the example with the other. 2) It may be useful to combine and summarize Sections 1 and 2 and use the example to ground the abstract description in Section 1. 3) You might have to make the example shorter to page limits for conferences.} 
%\resp{I numbered your concerns and responded to them one by one: 1) I introduced schema evolution first and then its happening in spl. 2) I still prefer separating motivating ex and introduction to avoid complicating the intro. I want the intro to be to the point and brief, explaining that variational databases have multiple application (instances), they have appeared in literature except that researchers have failed to recognize the commonality in all those instance which is the fact that they are all variation appearing in different contexts and applications. The purpose of motivating example is to draw the reader to continue reading the paper that is more structured. It respects the reader time (if they're interested they'll continue reading into motivating example) and is more organized. 3) I'll come back to this in the final version. However, I think it's important to put the time in and use the space to establish the functionality of VDB.}
%\responded

%\point{Overall scenario.}

%In practice, new instances of variation in databases arises. We discuss such a scenario that 
%we discussed with an industry contact in \appref{industry-ex}.
In this section, we motivate the interaction of two kinds of variation in databases 
resulting in a new kind:
database-backed software produced by an SPL and schema evolution.
An SPL uses a set of boolean variables called \emph{features} 
to indicate the functionalities that each software variant requires.
Consider an SPL that generates management software for companies. 
It has a feature \edu\ that 
indicates whether a company
provides educational means such as courses for its 
employees.
%\footnote{In practice, such a SPL would have two features: \base\ and \edu,
%where \base\ is an arbitrary feature, i.e., for all variants it must be enabled, and it 
%indicates software variants that provide just basic functionalities. For simplicity
%and without loss of generality, we drop this feature.}. 
%NOTE: YOU CAN OMIT THE FOOTNOTE IF YOU'RE RUNNING OUT OF SPACE!!!!!!!
Software variants in which \edu\ is disabled (i.e., \edu\ = \f) only provide basic 
functionalities while ones in which \edu\ is enabled provide educational functionalities
in addition to the basic ones. Thus, this SPL yields two types of variants:
\basic\ and \educational.

%\point{Database of this SPL.}
Each variant of this SPL needs a database to store information
about employees, but SPL features impact the database: While
\basic\ variants do not need to store any education-related records 
\educational\ variants do. 
% In practice, SPL developers use 
%only one database for both variant categories~\cite{ATW18poly}, 
%which can easily be separated
%by using the \edu\ feature.
We visualize the impact of features on the schema variants in \tabref{mot}:
It has two
schema types: \basic\ and \educational.
A \basic\ schema variant contains only the schema in one of the cells in column \basic\
while an \educational\ schema variant consists of two sub-schemas: one from the \basic\
column and another from the \educational\ one, e.g., 
the yellow highlighted cells include relation schemas for an \educational\ schema variant. 
%Note that these sub-schemas do not 
%have to be adjunct cells.

%
Rows of \tabref{mot} indicate the evolution of schema variants in time.
To capture the evolution of the software and its database we add 
two disjoint sets of features which again are boolean variables.
% and call them
%\emph{temporal features}. 
The temporal feature sets are disjoint to allow 
individual paces for the evolution of each type of schema, e.g.
yellow cells of \tabref{mot} show a valid schema variant even though
the \basic\ and \educational\ sub-schemas are not in the same row. 
%For example, a valid software variant 
%can have the \basic\ schema associated with \vThree\ in addition
%to the \educational\ schema associated with \tFour. 
%The following scenario reiterates 
%these concepts through an example.
%Hence, we introduce \emph{temporal features} to tag schemas when they change over time.
%A \basic\ schema is associated with a temporal feature of \vOne\ -- \vFive\ while 
%an \educational\ schema is associated with a temporal feature of \tOne\ -- \tFive.
%We have two sets of temporal features because when \edu\ is enabled
%any \educational\ and \basic\ schemas can be grouped to form a complete
%schema.
%For example, a valid software variant 
%can have the \basic\ schema associated with \vThree\ and
%the \educational\ schema associated with \tFour. 
%%\point{schemas could evolve independent from each other.}
%The \educational\ schema evolves independently from the 
%\basic\ schema. For example, due to new requirements SPL DBAs need to 
%add online courses to the schema. Schema variant associated with
%\tFour\ demonstrate this change. That is why we need to have
%two sets of temporal features.
%For example, a valid software variant 
%can have the \basic\ schema associated with \vThree\ in addition
%to the \educational\ schema associated with \tFour. 
%
%the \educational\ category 
%includes tables from the \basic\ schema in addition to ones from the \educational\ schema 
%while the \basic\
%category only consists of its own tables). Separating these database
%variants using SPL features results in much cleaner databases as
%opposed to databases created for SPL in practice.

%\point{Component evolution.}
Now, consider the following scenario:
In the initial design of the \basic\ database, SPL  DBAs
settle on three tables \engemp, \othemp, and \job; shown in \tabref{mot} and associated
with feature \vOne. 
After some time, they decide
to refactor the schema to remove redundant tables,
thus, they combine the two
relations \engemp\ and \othemp\ into one, \empacct; associated with  feature \vTwo. 
Since some clients' 
software relies on a previous design the two schemas have to coexist in parallel.
%However, to capture what schema a software variant is using we introduce
%temporal features, e.g., \vOne\ and \vTwo, and tag corresponding schemas 
%with these features. 
Therefore, the existence (presence) of \engemp\ and \othemp\
relations is \emph{variational}, i.e., they only exist in the \basic\
schema when \vOne\ = \t.
This scenario describes \emph{component evolution}:
%database evolution in SPL resulted from 
developers
update, refactor, and improve components including the database~\cite{dbSPLevolve}.
% is called 
%\emph{component evolution}~\cite{dbSPLevolve}.

%\point{Product evolution.}
Now, consider the case where a client that previously requested a \basic\ variant of the
management software has recently added courses to educate its
employees in specific subjects. Hence, an SPL developer needs to enable
the \edu\ feature for this client, forcing the adjustment of the schema variant to \educational. 
This case describes \emph{product evolution}:
database evolution in SPL resulted
from clients adding/removing features/components~\cite{dbSPLevolve}. 
%is called 
%\emph{product evolution}~\cite{dbSPLevolve}. 

%\point{\edu\ affects basic too.}
The situation is further complicated since the \basic\ and \educational\ schemas are interdependent:
%not independent of each other: 
Consider
%enabling the \edu\ feature not only adds the relations in the \educational\
%schema but it also affects the \basic\ schema. 
the \basic\ schema 
variant for feature \vFour. Attributes \isstudent\ and \isteacher\ only exists
in the \empacct\ relation when \edu\ = \t, represented by a \dashuline{dash-underline},
otherwise the \empacct\
relation has only four attributes: \empno, \hiredate, \titleatt, and \deptno.
Hence, the presence of attributes \isstudent\ and \isteacher\ in \empacct\ relation is
\emph{variational}, i.e., they only exist in \empacct\ relation
when \edu\ = \t.
%similar to the presence of \engemp\ which
%is variational depending on whether \vOne is enabled or not. 

%%\point{schemas could evolve independent from each other.}
%The \educational\ schema evolves independently from the 
%\basic\ schema. For example, due to new requirements SPL DBAs need to 
%add online courses to the schema. Schema variant associated with
%\tFour\ demonstrate this change. That is why we need to have
%two sets of temporal features.
%For example, a valid software variant 
%can have the \basic\ schema associated with \vThree\ in addition
%to the \educational\ schema associated with \tFour. 

Our example demonstrates how different kinds of variation
% in time and space
interact with each other, an indispensable consequence of modern software
development.
The described interaction is similar to a recent scenario we discussed with an
industry contact in \appref{industry-ex}.
%In practice, new instances of variation in databases arises. We discuss such a scenario that 
%we discussed with an industry contact in \appref{industry-ex}.


\subsection{New Instance of Data Variation in Industry}
\label{sec:industryex}



%\point{briefly iterate contributions and how they solve needs.}
New variational scenarios could appear, either from combination of other scenarios
or even a new scenario could reveal itself. For example, the following is
a scenario we recently discussed with an industry contact:
%
A software company develops software for different networking companies and
analyzes data from its clients to advise them accordingly. 
%
The company records information from each of its clients' networks in databases
customized to the particular hardware, operating systems, etc.\ that each
client uses.
%
The company analysts need to query information from all clients who agreed to
share their information, but the same information need will be represented
differently for each client.
%
This problem is essentially a combination of the SPL variation problem (the
company develops and maintains many databases that vary in structure and
content) and the data integration problem (querying over many databases that
vary in structure and content). However, neither the existing solutions from
the SPL community nor database integration address both sides of the problem.
%
Currently the company manually maintains variant schemas and queries, but this
does not take advantage of sharing and is a major maintenance challenge.
% , for the reasons SPL researchers are familiar with.
%
With a database encoding that supports explicit variation in schemas, content,
and queries, the company could maintain a single variational database that can
be configured for each client, import shared data into a variational database, and write
variational queries over the variational database to analyze the data, significantly reducing redundancy
across clients.
\subsection{Requirements of a Variation-Aware Database Framework}
\label{sec:req}

%In this section, we define the requirements that make a database framework
%variational. 
For a variation-aware framework  to be expressive enough to encode
any kind of variation in databases, it must satisfy some requirements.
Thus, we define the requirements that make a database framework
variational through studying different kinds of variation in databases.  
%These requirements distinguish a database framework that 
%simulates the effect of variation compared to one that is variational. 
A variational artifact, including databases, encompasses multiple 
\emph{variants} of the artifact and provides a way to distinguishes 
between different variants that are all encoded in one place, the 
variational artifact. It also provides a way to get the variants from
the variational artifact, we call this \emph{configuration function}.
%
These requirements that help distinguish a database framework that 
simulates the effect of variation compared to one that is variational are
listed below:

%Our example demonstrates how instances of variation
%% in time and space
%arise and interact with each other, an unavoidable consequence of modern software
%development.
%We categorize the needs of a variational database framework at different stages
%such as development, test, and deployment and
%use our motivating example to illustrate them. Throughout the paper,
%we show how our framework achieves these needs via examples, proofs, and tests.
%%To be concrete throughout the paper we itemize
%%the needs of users working with a database with variation
%%through the needs of SPL developers and \revised{analysts and} DBAs:
%% for our motivating example.
%%These needs cover all general needs when variation appears in databases
%%because our motivating example considers variation in both possible 
%%dimensions; time and space:
%
\begin{enumerate}
%[wide, labelwidth=!, labelindent=0pt, topsep=1pt]
%[leftmargin=*]
%\itemsep0em
\item [\textbf{(\nZero)}]
\emph{All database variants must be accessible at a given time}.
For example, in our motivating example, a company that started with 
\vOne\ of the \basic\ schema evolves over time but its different 
branches adopt the new schema at different paces, thus, it 
requires access to all variants of the \basic\ schema. 
%For example,
%SPL developers need to access all database variants while
%writing code to be able to 
% extract information for 
% all software variants
%they are developing. 
%
%Hence,
%\emph{users need to have access to all database variants at a given time}.
%
%\secref{vdb} explains how a VDB achieves this.
%%\tabref{mot-vsch} shows how v-schema achieves
%\exref{vsch} shows how v-schema achieves
%this at schema level for a part of our motivating example.
%%make sure the following is the case.
%%\TODO{and \secref{exp-disc} discusses two databases
%%that achieve this at both content and schema level.}
%
\item [\textbf{(\nOne)}]
\emph{The query language must allow for querying
%One must be able to query 
multiple database variants simultaneously.
Additionally, it must allow for filtering tuples to specific variants.}
That is, the framework must provide a query language that allows users to query multiple
database variants at the same time in addition to giving them the freedom to choose
the variants that they want to query. 
For example, an SPL tester that is testing a piece of code for the not highlighted 
variants of the software in \tabref{mot} needs to write queries that exclude the 
 variants associated with yellow cells of \tabref{mot}. 
%For example,
%depending on what component they are working on,
%\revised{SPL testers} need to be able to query all or some of the variants.
%
%Hence, 
%\emph{users need to query multiple database variants simultaneously and selectively}.
%
%\secref{vq} shows how our query language achieves this.
%% by introducing variation into queries, \secref{vq}.
%\exref{vq-specific} and \exref{vq-same-intent-mult-vars} illustrate this 
%for our motivating example.
%
\item [\textbf{(\nTwo)}]
\emph{Every piece of data must clearly state the variant it belongs to and 
this information must be kept throughout the entire framework.}
Continuing the example of the SPL tester, they need to know the variant that
some results belong to in order to be able to debug the software correctly and
accordingly. 
%The framework needs to 
%keep track of which variants a piece of data belongs to and ensuring that 
%it is maintained throughout a query}.
%For example,
%given that one can access all database variants and that they
%can query all variants simultaneously, for test purposes, 
%analysts need to know which variant a tuple belongs to.
%
%Hence,
%\emph{the framework needs to 
%keep track of which variants a piece of data belongs to and ensuring that 
%it is maintained throughout a query}.
%
%\revised{\secref{vdb} and \secref{type-sys} illustrate how our framework achieves this.}
%%We call this \emph{variation-preserving} property and }
%%Storing variants that a tuple belongs to in a VDB achieves the first part, \secref{vdb}, and
%%VRA's type system ensures the second part, \secref{type-sys}.
%\exref{var-pres} illustrates this for a given query.
%\TODO{we should show this for content level too. but for this submission we can say that 
%all approaches account for this and don't lose that info. discuss with Eric. say that it has been tested for all queries but not proved.}
%
\item [\textbf{(\nThree)}]
\emph{The variational database must provide a way for generating database
and query variants.}
For example, the SPL developers need to deploy the management software for each client,
thus, they need to configure the database schema and its queries in the code 
for each software variant.  
%Users need to deploy one variant of the database and its associated queries}.
%For example,
%SPL developers need to deploy the database and its queries
%to generate a specific software product for a client based on their
%requested features. 
%
%Hence,
%\emph{users need to deploy one variant of the database and its associated queries}.
%
%We define a \emph{configure} function for a VDB and its elements, \figref{vdb-conf}, 
%in addition to queries, \figref{v-alg-conf-sem}, that achieves this. 
%\exref{conf-vq} illustrate deploying an example query.
%
\end{enumerate}

Throughout the thesis,
we show how the proposed variational database framework satisfies these requirements 
via examples, proofs, and tests.


\section{Contributions and Outline of this Thesis}
\label{sec:contribution}

\TODO{to be filled out when I have the chapters.}

The high level goal of this thesis is to emphasize the need for a variation-aware database
framework and to present one such framework. Therefore, in addition to the formal 
definition of the framework and query language, it also provides variational data sets 
(including both the variational database and a set of queries) to illustrate the feasibility
of the proposed framework. Furthermore, it illustrates various approaches to implement
such a framework and compares their performance.

The rest of this section describes the structure of this thesis, enumerating the specific 
contribution that each chapter makes. 

\chref{bg} (\emph{Background}) introduces several concepts and terms that are the 
basis of this thesis. It describes types and how to interpret them. It explains relational
databases with assumptions that are held throughout the thesis and relational algebra. 
It also describes various ways of incorporating variation
into elements of a database. 

\TODO{\chref{frame}}

\TODO{\chref{vql}}

\chref{vdbusecase} aims at guiding an expert through generating a VDB and writing
variational queries for a variation scenario where unfortunately the database variants
do not exist and thus, generating the VDB requires the expert knowledge and cannot
be automated. 
%
It details two such variation scenario and introduces two use cases of VDB, one over space
(adopts the email SPL described by \citet{Hall05} and explained in \secref{enron-vdb})
 and another over time (adopts the evolution 
of an employee schema described by \citet{prima08Moon} and explained in \secref{emp-vdb}), 
in addition to how a VDB can store all database variants in a single database and
 how variational queries can capture various information needs over different database variants
in a single query. It also describes how  the VDBs were systematically 
generated and how the variational queries
were adapted and adjusted from papers describing the variation scenario. 
%
The last section of this chapter, \secref{usecase-disc}, discusses the question 
``should variation be encoded explicitly in databases?'' through the lessons learned 
while generating these use cases. 


\TODO{\chref{vdbms}}



\chref{rw} (\emph{Related Work}) collects research related to different kinds of variation 
in databases and other related variational research. 

Finally, \chref{conclusion} (\emph{Conclusion}) briefly presents ...
%\begin{itemize}
%%[leftmargin=*]
%%\itemsep0em
%\item We define the requirements of a a variational database framework, \secref{req}.
%%[wide, labelwidth=!, labelindent=0pt, topsep=1pt]
%%[leftmargin=*]
%%\item To account for variation explicitly, we use a \emph{variation space} 
%%and propositional formulas of features to refer to a subset of the space (\secref{encode-var})~\cite{ATW17dbpl}.
%%\item We provide a framework to capture variation within a database using
%%propositional formulas over  
%%sets of features, called \emph{feature expressions}, following~\cite{ATW17dbpl}.
%\item We define the \emph{variational database (VDB)} by incorporating 
%variation directly in the database, \secref{vdb}.
%%both the structure (schema) and
%%content (tuples) of the database, introducing \emph{variational schemas}, \secref{vsch}, 
%%and \emph{variational tables}, \secref{vtab}, and together \emph{VDBs},
%%\secref{vdb}.
%%, satisfying \textbf{N0} and first part of \textbf{N2}.
%\item We define the 
%%To express user information needs 
%\emph{variational relational algebra} query language, \secref{vrel-alg}, 
%its static type system, \secref{type-sys},
%and \emph{variation-minimzation} rules, \secref{var-min}.
%%a combination of relational algebra and 
%%choice calculus~\cite{EW11tosem,Walk13thesis}, \secref{vrel-alg}.
%%Users query a VDB by a \emph{variational query}, \secref{vq}.
%%\item 
%%%To make variational 
%%\revised{
%%%For more usability, w
%%We define VRA's static type system, \secref{type-sys},
%%and \emph{variation-minimzation} rules, \secref{var-min}.}
%% queries more useable and easier to understand, respectively,
%%by defining 
%%a static type system, \secref{type-sys},
%%and \emph{variation-minimiztion} rules, \secref{var-min}.
%%to make it easier to understand and more useable. 
%%rules to minimize 
%%variation in v-query, \secref{var-min}, to provide better
%%efficiency and usability. 
%%This completes satisfiability of \textbf{N2}.
%\item 
%%To query a 
%%variational database and receive clear results
%%\revised{To interact with a VDB}
%We implement a prototype of our framework, called  \emph{Variational Database Management System (VDBMS)}, on top a traditional DBMS, \secref{impl}.
%We test VDBMS on previously developed use cases, \secref{exp-disc}.
%%,
%%satisfying all four needs: \textbf{N0}-\textbf{N3}.
%%\textbf{N1}, \textbf{N2}, and \textbf{N3}.
%\end{itemize}


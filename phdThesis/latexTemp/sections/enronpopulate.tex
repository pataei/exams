\subsubsection{Populating the Email SPL VDB}
\label{sec:enron-pop}

The final step to create the email VDB is to populate the database with data
from the Enron email dataset, adapted to fit our v-schema~\cite{Shetty04}.
%
For evaluation purposes, we want the data from the dataset to be distributed
across multiple variants of the VDB. To simulate this, we identified five
plausible configurations of the email SPL, which we divide the data among. The
five configurations of the email SPL are:
% we considered are:
%
%\begin{itemize}
%
%\item 
\emph{basic email}, which includes only basic email functionality
and does not include any of the optional features;
% from the SPL.
%
%\item 
\emph{enhanced email}, which extends \emph{basic email} by
enabling two of the most commonly used email features, \forwardmessages\ and
\filtermessages;
%
%\item 
\emph{privacy-focused email}, which extends \emph{basic email}
with features that focus on privacy, specifically, the
\signature, \encryption, and \remailmessage\ features;
%
%\item 
\emph{business email}, which extends \emph{basic email} with
features tailored to an environment where most emails are expected to be among
users within the same business network, specifically, \addressbook,
\signature, \encryption, \autoresponder, and \mailhost; and
%
%\item 
\emph{premium email}, in which all of the optional features
in the SPL are enabled.
%
%\end{itemize}
%
For all variants, any features that are not enabled are disabled. 


The original Enron dataset has 150 employees with 252,759 email messages. 
%
We load this data into the \employees\ and \messages\ tables defined in
\secref{enron-vsch}, initializing all attributes that are not present in the
original dataset to \NULL.


For the \employees\ table, we construct five views corresponding to the five
variants of the email system described above. We allocate 30 employees to each
view based on their employee ID, that is, the first 30 employees sorted by
employee ID are associated with the basic email variant, the next 30 with the
enhanced email variant, and so on. The presence condition for each tuple is set
to the conjunction of features enabled in that view.
%
We then modify each of the views of the \employees\ table by adding randomly
generated values for attributes associated with the enabled features; 
e.g., in the view for the privacy-focused variant, we populate the
\verificationkey\ and \publickey\ attributes.
%
Any attribute that is not present in the given tuple due to a conflicting
presence condition will remain \NULL. For example, both the \verificationkey\
and \publickey\ attributes remain \NULL\ for employees in the enhanced variant
view since the presence condition does not include the corresponding features.


For the \messages\ table, we again create five views corresponding to each of
the variants. Each tuple is added to the view of the variant that contains the
message's sender, which updates the tuple's presence condition accordingly.
%
The \messages\ table also contains several additional attributes corresponding
to optional features, which we populate in a systematic way.
%
We set \issigned\ to \tru\ if the message sender has the \fsignature\ feature
enabled, and we set \isencrypted\ to \tru\ if \emph{both} the message sender
and recipient have \fencryption\ enabled.
%
We populate the \isforwardmsg, \isautoresponse, and \issystemnotification\
attributes by doing a lightweight analysis of message subjects to determine
whether the email is any of these special kinds of messages; for example, if
the subject begins with ``FWD'', we set the \isforwardmsg\ attribute to \tru.
%
If a forward or auto-reply message was sent by a user that does not have the
corresponding feature enabled, we filter it out of the dataset. After
filtering, the \messages\ relation contains 99,727 messages.
%
For each forward or auto-reply message, we also add a tuple with the relevant
information to the new \forwardmsg\ and \automsg\ tables.
%
For employees belonging to database variants that enable \remailmessage,
\autoresponder, \addressbook, or \mailhost\ we randomly generate tuples in the
tables that are specific to each of these features.
%
Finally, the \recipientinfo\ relation is imported directly from the dataset. We
set each tuple's presence condition to a conjunction of the presence conditions
of the sender and recipient.
% \eric{actually I conjuncted them, which now I'm not so sure about it!}


%We provide more detailed instructions for systematically constructing the email
%VDB in both MySQL and PostgreSQL in a wiki page associated with the
%repository.%
%\footnote{\url{https://github.com/lambda-land/VDBMS/wiki/Enron-Email-Database-UseCase-Doc\#steps-take-to-build-vdb-for-enron-case-study}}
%%
%Additionally, we provide SQL scripts to automate the creation of views for each
%variant%
%%
%\footnote{\href{https://github.com/lambda-land/VDBMS/blob/master/usecases/space-emailSPL/database/build/step1_build_email_variants.sql}{usecases/space-emailSPL/database/build/step1\_build\_email\_variants.sql}}
%%
%and to automate the population of these views with tuples from the original
%dataset,%
%%
%\footnote{\href{https://github.com/lambda-land/VDBMS/blob/master/usecases/space-emailSPL/database/build/step2_build_email_vdb.sql}{usecases/space-emailSPL/database/build/step2\_build\_email\_vdb.sql}} 
%%
%which also sets each tuple's presence condition.
%
%
%The resulting database is distributed in two forms, one with the embedded
%variational schema as described in \secref{enron-vsch},%
%%
%\footnote{\href{https://github.com/lambda-land/VDBMS/tree/master/usecases/space-emailSPL/database/withSchema}{usecases/space-emailSPL/database/withSchema}}
%%
%and one without the embedded schema%
%\footnote{\href{https://github.com/lambda-land/VDBMS/tree/master/usecases/space-emailSPL/database/withoutSchema}{usecases/space-emailSPL/database/withoutSchema}}
%%
%for use with our VDBMS tool in which the variational schema is provided
%separately.


%\point{VDBs are a good fit to encode variation in a database of a 
%database-backed SPL.}
%%variation over space
%
%
%\point{To showcase how a VDB captures different needs arising in SPLs 
%w.r.t. variation in their databases we generate a VDB for an email SPL.}\\
%- where data is coming from\\
%- where spl is coming from
%
%\point{We adopt the features and feature model from the SPL and 
%consider five database variants.}\\
%- describe 5 variants and their config.
%
%\point{Features are incorporated into both the schema and data.}\\
%- feature model\\
%- schema\\
%- data\\
%\rewrite{revise tables! taken from @Q!}
\begin{comment}
\begin{table*}
\caption{Schema variants for the email VDB.}
\label{tab:enron_sch}
\begin{center}
\small
\begin{tabular} {| l |l| c | l |}
\hline
\textbf{Product} & \textbf{Schema Variants} \\
% \cline{2-3}
% \textbf{Features} & \multicolumn{1}{ c ||} {$\edu \vee \neg \edu \equiv \t$} & \multicolumn{1}{ c |} {\edu} & \textbf{Features}\\
\hline 
\hline 
\multirow{4}{*}{\textbf{Basic Email}} &  \employees(\eid, \fname, \lname, \emailid, \folder, \status) \\
 &  \messages(\midatt, \sender, \dateatt, \messageid, \subject, \body, \folder\\
 & \phantom{invisible h}, \issystemnotification)  \\ 
 &  \recipientinfo(\rid, \midatt, \rtype, \rvalue)  \\
& \referenceinfo(\rid, \midatt, \reference)  \\
\hline
\multirow{6}{1cm}{\textbf{Enhanced Email}} &  \employees(\eid, \fname, \lname, \emailid, \folder, \status) \\
&  \messages(\midatt, \sender, \dateatt, \messageid, \subject, \body, \folder \\ 
& \phantom{invisible h}, \issystemnotification, \isforwardmsg) \\ 
& \recipientinfo(\rid, \midatt, \rtype, \rvalue)  \\
& \referenceinfo(\rid, \midatt, \reference)  \\
& \forwardmsg(\eid, \forwardaddr)  \\
& \filtermsg(\eid, \suffix)  \\
\hline
\multirow{5}{1cm}{\textbf{Privacy-focus Email}} &  \employees(\eid, \fname, \lname, \emailid, \folder, \status\\
& \phantom{invisible hhh}, \verificationkey, \publickey) \\
&  \messages(\midatt, \sender, \dateatt, \messageid, \subject, \body, \folder  \\  
& \phantom{invisible h}, \issystemnotification, \issigned, \isencrypted)   \\ 
&  \recipientinfo(\rid, \midatt, \rtype, \rvalue)   \\
& \referenceinfo(\rid, \midatt, \reference)  \\
& \remailmsg(\eid, \pseudonym)  \\
\hline
\multirow{7}{1cm}{\textbf{Group Email}} &  \employees(\eid, \fname, \lname, \emailid, \folder, \status\\
 & \phantom{invisible hhh}, \verificationkey, \publickey) \\
 &  \messages(\midatt, \sender, \dateatt, \messageid, \subject, \body, \folder   \\ 
 & \phantom{invisible h}, \issystemnotification, \isautoresponse, \issigned, \isencrypted)   \\ 
 &  \recipientinfo(\rid, \midatt, \rtype, \rvalue)   \\
& \referenceinfo(\rid, \midatt, \reference)  \\
& \automsg(\eid, \subject, \body)  \\
& \alias(\eid, \emailAtt, \nickname)  \\
& \mailhost(\eid, \username, \mailhost)  \\
\hline
\multirow{10}{1cm}{\textbf{Premium Email}} &  \employees(\eid, \fname, \lname, \emailid, \folder, \status\\
& \phantom{invisible hhh}, \veirficationkey, \publickey) \\
 &  \messages(\midatt, \sender, \dateatt, \messageid, \subject, \body, \folder \\  
 &  \phantom{invisible h}, \issystemnotification,  \isencrypted, \issigned, \isautoresponse \\
 & \phantom{invisible h}, \isforwardmsg)   \\ 
 & \recipientinfo(\rid, \midatt, \rtype, \rvalue)   \\
& \referenceinfo(\rid, \midatt, \reference)  \\
& \forwardmsg(\eid, \forwardaddr)  \\
& \filtermsg(\eid, \suffix)  \\
& \remailmsg(\eid, \pseudonym)  \\
& \automsg(\eid, \subject, \body)  \\
& \alias(\eid, \emailAtt, \nickname)  \\
& \mailhost(\eid, \username, \mailhost)  \\
\hline
\end{tabular}
\end{center}
\end{table*}
\end{comment}

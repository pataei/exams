\section{Variation in Time: Employee Use Case}
\label{sec:emp-vdb}



In our second case study, we focus on variation that occurs in ``time'', that
is, where the software variants are produced sequentially by incrementally
extending and modifying the previous variant in order to accommodate new
features or changing business requirements. Although new variants conceptually
replace older variants, in practice, older variants must often be maintained in
parallel; external dependencies, requirements, and other issues may prevent
clients from updating to the latest version.
%
Variation in software over time directly affects the databases such software
depends on~\cite{dbDecay16Stonebraker}, and dealing with such changes is a
well-studied problem in the database community known as \emph{database
evolution}~\cite{schVersioningSurvey95Roddick}.


Although research on database evolution has produced a variety of solutions for
managing database variation over time, these solutions do not treat variation
as an orthogonal property and so cannot also accommodate variation in space.
The goal of our work on variational databases is not to directly compete with
database evolution solutions for time-only variation scenarios, but rather to
present a more general model of database variation that can accommodate
variation in both time and space, and that integrates with related software via
feature annotations.

We demonstrate variation in time by 
using a VDB to encode an employee database evolution scenario
systematically adapted from
\citet{prima08Moon} and populated by a dataset that is widely used
in databases research.\footnote{\url{https://github.com/datacharmer/test_db}}


\subsection{Variation Scenario: An Evolving Employee Database}
\label{sec:emp-scenario}

\begin{table}
\caption{Evolution of an employee database schema~\cite{prima08Moon}.
%%from \citet{prima08Moon}.
}
\vspace{-8pt}
\label{tab:emp-sch}
\begin{center}
\small
\begin{tabular} {|l|l|}
\hline
\textbf{Version} & \textbf{Schema}\\
% & \multirow{1}{0.3cm}{\textbf{\sI}}&  \\
\hline 
\hline 
% \dashuline{\isstudent}
\multirow{3}{0.3cm}{\vOne} &  \engemp\ (\empno, \name, \hiredate, \titleatt, \deptname) \\
%& &  \multirow{3}{2cm}{hiredate $<$ 1988-01-01}\\
& \othemp\ (\empno, \name, \hiredate, \titleatt, \deptname) \\
%& \sOne &\\
& \job\ (\titleatt, \salary)\\
% & &\\
\hline
\multirow{2}{0.3cm}{\vTwo} & \empacct\ (\empno, \name, \hiredate, \titleatt, \deptname) \\
%& \multirow{2}{0.5cm}{\sTwo}&  \multirow{2}{2cm}{hiredate $<$ 1991-01-01}\\
& \job\ (\titleatt, \salary) \\
%&  &\\
\hline
\multirow{3}{0.3cm}{\vThree} & \empacct\ (\empno, \name, \hiredate, \titleatt, \deptno) \\
%&\multirow{3}{0.5cm}{\sThree}&  \multirow{3}{2cm}{hiredate $<$ 1994-01-01}\\
& \job\ (\titleatt, \salary)\\
% &  &\\
& \dept\ (\deptname, \deptno, \managerno) \\
%& &\\
\hline
\multirow{4}{0.3cm}{\vFour} & \empacct\ (\empno, \hiredate, \titleatt, \deptno) \\
%& \multirow{4}{0.5cm}{\sFour} &  \multirow{3}{2cm}{hiredate $<$ 1997-01-01}\\
& \job\ (\titleatt, \salary) \\
%& & \\
& \dept\ (\deptname, \deptno, \managerno)\\
%& & \\
& \empbio\ (\empno, \sex, \birthdate, \name)\\
%  & & \\
\hline
\multirow{3}{0.3cm}{\vFive} & \empacct\ (\empno, \hiredate, \titleatt, \deptno, \salary) \\
%& \multirow{3}{0.5cm}{\sFive} &  \multirow{3}{2.3cm}{hiredate $<$ 2000-01-28}\\
& \dept\ (\deptname, \deptno, \managerno) \\
%& & \\
& \empbio\ (\empno, \sex, \birthdate, \fname, \lname)\\
% & & \\
\hline
\end{tabular}
\vspace{-5pt}
\end{center}
\end{table}


\citet{prima08Moon} describe an evolution scenario in which the schema of a
company's employee management system changes over time, yielding the five
versions of the schema shown in \tabref{emp-sch}.
%
In \vOne, employees are split into two separate relations for
engineer and non-engineer personnel.
%
In \vTwo, these two tables are merged into one relation, \empacct.
%
In \vThree, departments are factored out of the \empacct\ relation and
into a new \dept\ relation to reduce redundancy in the database.
%
In \vFour, the company decides to start collecting more personal
information about their employees and stores all personal information in the
new relation \empbio.
%
Finally, in \vFive, the company decides to decouple salaries from
job titles and instead base salaries on individual employee's qualifications
and performance; this leads to dropping the \job\ relation and adding a new
\salary\ attribute to the \empacct\ relation. This version also separates the
\name\ attribute in \empbio\ into \fname\ and \lname\ attributes.


We associate a feature with each version of the schema, named 
$\vOne\ldots\vFive$.
%
These features are mutually exclusive since only one version of the
schema is valid at a time. This yields the  feature model
$\dimMeta_\employee$.
%
 Also, note that the feature model represent a restriction on the entire
 database.
%
\begin{align*}
\dimMeta_\employee
  &=   \oneof {\vOne, \vTwo, \vThree, \vFour, \vFive}
%  
%  \left(\vOne\wedge\neg\vTwo\wedge\neg\vThree\wedge\neg\vFour\wedge\neg\vFive\right)\\
%  &\quad
%  \vee\left(\neg\vOne\wedge\vTwo\wedge\neg\vThree\wedge\neg\vFour\wedge\neg\vFive\right)
%%  \\
%%  &
%  \vee\left(\neg\vOne\wedge\neg\vTwo\wedge\vThree\wedge\neg\vFour\wedge\neg\vFive\right)\\
%   &\quad
%   \vee\left(\neg\vOne\wedge\neg\vTwo\wedge\neg\vThree\wedge\vFour\wedge\neg\vFive\right)
%%  \\
%%  &
%  \vee\left(\neg\vOne\wedge\neg\vTwo\wedge\neg\vThree\wedge\neg\vFour\wedge\vFive\right)
\end{align*}

%As a reminder, based on the hierarchy of presence conditions, 
%the feature model $\fModel_\employee$ is used as the root presence condition of
%the variational schema for the employee VDB, implicitly applying it to all
%relations, attributes, and tuples in the database.


\subsection{Generating Variational Schema of the Employee VDB}
\label{sec:emp-vsch}

\begin{table}
\caption[short caption]{Employee v-schema with feature model.
 \ensuremath{\dimMeta_\employee}.}
\vspace{-8pt}
\label{tab:emp-vsch}
\begin{center}
\small
\begin{tabular} {|l|l|}
\hline
%\textbf{Variational Schema for Employee Evolution} \\
%\hline 
 % \annot [\vFour] \name
\rule{0pt}{3ex}%
$\engemp(\empno, \name, \hiredate,\titleatt,\deptname)^{\textcolor{blue}
\vOne}$ \\[1.1ex]
$\othemp(\empno, \name, \hiredate, \titleatt, \deptname)^{\textcolor{blue}
\vOne}$ \\[1.1ex]
$\empacct(\empno, \annot [\textcolor{blue}{\vTwo \vee \vThree}] \name,
\hiredate, \titleatt,$ \\
$\qquad\annot[\textcolor{blue} \vTwo]\deptname, \annot
[\textcolor{blue} {\vThree \vee \vFour \vee \vFive}]\deptno,\annot
[\textcolor{blue} \vFive]\salary)^{\textcolor{blue}{\vTwo \vee \vThree \vee
\vFour \vee \vFive}}$ \\[1.1ex]
$\job(\titleatt, \salary)^{\textcolor{blue}{\vTwo \vee \vThree \vee \vFour}}$
\\[1.1ex]
$\dept(\deptname, \deptno, \managerno)^{\textcolor{blue}{\vThree \vee \vFour
\vee \vFive}}$ \\[1.1ex]
$\empbio(\empno, \sex, \birthdate, \annot [\textcolor{blue} \vFour] \name,
\annot[\textcolor{blue}\vFive]{\fname}, \annot[\textcolor{blue}
\vFive]{\lname})^{\textcolor{blue} {\vFour \vee \vFive}}$ \\
\hline
\end{tabular}
\vspace{-12pt}
\end{center}
\end{table}


The variational schema for this scenario is given in \tabref{emp-vsch}. It
encodes all five of the schema versions in \tabref{emp-sch} and was
systematically generated by the following process. First, generate a universal
schema from all of the plain schema versions; the universal schema contains
every relation and attribute appearing in any of the five versions. Then,
annotate the attributes and relations in the universal schema according to the
versions they are present in.
%
For example, the \empacct\ relation is present in versions \vTwo--\vFive, so it
will be annotated by the feature expression
$\vTwo\vee\vThree\vee\vFour\vee\vFive$, while the \salary\ attribute within the
\empacct\ relation is present only in version \vFive, so it will be annotated
by simply \vFive.
%
 The overall variational schema will be annotated by the feature model
 $\dimMeta_\employee$, described in \secref{emp-scenario}.
%
Since the presence conditions of attributes are implicitly conjuncted with the
presence condition of their relation
 that contains them, 
 we can avoid redundant
annotations when an attribute is present in all instances of its parent
relation. For example, the \empbio\ relation is present in $\vFour\vee\vFive$,
and the \birthdate\ attribute is present in the same versions, so we do not
need to redundantly annotate 
 \birthdate.

Similar to the email SPL VDB, we distribute the variational schema for the
employee VDB in two formats:
%
First, we provide the schema in the encoding used by our prototype VDBMS tool.%
%\footnote{\href{https://github.com/lambda-land/VDBMS/blob/master/usecases/time-employee/schema/EmployeeSchema.hs}{usecases/time-employee/schema/EmployeeSchema.hs}}
%
Second, we provide a direct encoding in SQL that generates the universal schema
for the VDB in either MySQL or Postgres.%
%\footnote{\href{https://github.com/lambda-land/VDBMS/tree/master/usecases/time-employee/database/create}{usecases/time-employee/database/create}}
%
The variability of the schema is embedded within the employee VDB%
%\footnote{\href{https://github.com/lambda-land/VDBMS/tree/master/usecases/time-employee/database/withSchema}{usecases/time-employee/database/withSchema}}
%
using the same encoding as described at the end of \secref{enron-vsch}.%
\footnote{All encodings of the employee variational schema are available at: \url{https://zenodo.org/record/4321921}.} 

%\begin{comment}
\subsection{Populating the Employee VDB}
\label{sec:emp-pop}

Finally, we populate the employee VDB using data from the widely used employee
database linked to in this subsection's lede.
%
This database contains information for $240,124$ employees. To simulate the
evolution of the database over time, we divide the employees into five roughly
equal groups based on their hire date within the company. 
For example, the
first group consists of employees hired before $1988-01-01$, while the second
group contains employees hired from $1988-01-01$ to $1991-01-01$.
%
Each group is assumed to have been hired during the lifetime of a particular
version of the database, and is therefore added to that version of the database
and \emph{also} to all subsequent versions of the database. This simulates the
fact that as a database evolves, older records are typically forward propagated
to the new schema~\cite{schVersioningSurvey95Roddick}. Thus, \vFive\ contains
the records for all $240,124$ employees, while older versions will contain
progressively fewer records.
%
The final employee VDB has $954,762$ employee due to this forward propagation,
despite having the same number of employees as the original database.


The schema of the employee database used to populate the employee VDB
 is different from all versions of the
variational schema, yet it includes all required information. Thus,
%To populate
%the VDB, 
we manually mapped data from the original schema onto each version of
the variational schema.
%\end{comment}


We provide SQL scripts of required queries to automatically 
generate the employee VDB.
%\footnote{\href{https://github.com/lambda-land/VDBMS/blob/master/usecases/time-employee/database/build}{usecases/time-employee/database/build}}
We also provide SQL scripts to automate the separation of each group of employees
into views according to their hire date%
%\footnote{\href{https://github.com/lambda-land/VDBMS/blob/master/usecases/time-employee/database/build/step1_chop_employees.sql}{usecases/time-employee/database/build/step1\_chop\_employees.sql}}
%
and populating those views from data in the employee database.%
\footnote{All the scripts are available at: \url{https://zenodo.org/record/4321921}.} 
%\footnote{\href{https://github.com/lambda-land/VDBMS/blob/master/usecases/time-employee/database/build/step2_build_vdb.sql}{usecases/time-employee/database/build/step2\_build\_vdb.sql}}

As for any VDB, if an attribute is not present in any of the variants covered
by a tuple's presence condition, that attribute will be set to NULL in the
tuple. We do this even though the relevant information may be contained in the
original employee database to ensure that we have a consistent VDB. For
example, while inserting tuples into the \vFour\ view of the \empbio\ table, we
always insert NULL values attributes \fname\ and \lname.
%
We also provide the final employee VDB in four flavors: both with and without the
embedded schema, and in both cases, encoded in MySQL and PostgreSQL format.%
\footnote{Both formats are availabe at: \url{https://zenodo.org/record/4321921}.} 
%\footnote{\href{https://github.com/lambda-land/VDBMS/tree/master/usecases/time-employee/database}{usecases/time-employee/database}}
%
We have tested the employee  VDB for the properties described in \secref{vdbfprop} 
and all of them hold.


%\point{VDBs are a good fit to capture schema evolution of a database.}
%%, i.e., its variation over time.}
%
%\point{To showcase how a VDB captures variation over time we encode
%the schema evolution of an employee database as a VDB.}\\
%- data comes from blah. schema evolution comes from blah.\\ 
%- data stat. schema evolution. \\
%
%\point{We consider each version of the schema evolution as a database 
%variant and assign a feature to them.}
%
%\point{We incorporate the features into both the schema and data.}\\
%- how we incorporate features into schema. \\
%- how we incorporate features into tables.
%\rewrite{rewrite tables to your liking! taken from @Q}

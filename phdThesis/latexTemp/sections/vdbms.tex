\chapter{Variational Database Management System (VDBMS)}
\label{ch:vdbms}

%
We implement a prototype using the VDB and VRA frameworks as
%\revised{We implement our VDB and VRA framework as}
\emph{Variational Database Management System (VDBMS)}.
VDBMS is implemented in Haskell. VDBMS sits on 
top of any DBMS that the user used to store their data 
%\arashComment{I did not find any explanation on how variational tables are stored in an RDBMS.} 
%\resp{it is exactly implemented as formalized in variational table section.}
%\responded
in form of variational tables, explained in \secref{vtab}, with one exception:
%However, 
we encode variation at the content level by storing tuples presence condition 
as a presence condition attribute to all tables
since relational databases cannot represent annotations. 
Note that the rest of the presence conditions are stored in the Haskell side of the system.
The presence conditions stored in the database are encoded as strings, unlike 
the presence conditions in the Haskell side of VDBMS.
%To acquire an extensible system we implement 
To support running VDBMS with multiple different plain relational DBMS backends,
we provide
a shared interface
for communicating with the backend DBMS and
%connecting to and inquiring information from a DBMS and
instantiate it for different database engines such as PostgreSQL and
MySQL. 
%\rewrite{any dbms that has a library in haskell that has a function
%that returns the result to the user. eg that doesn't satisfy this is 
%database.sqlite3. } --> The following addresses this:
An expert can extend VDBMS to another database engine by
writing methods for connecting to and querying from the database.

%\input{sections/implVar}
%\point{vdb and vschema and config (bottom of fig).}
\textbf{VDBMS architecture:}
\figref{arch} shows the architecture of VDBMS and its modules.
For now, we assume a VDB and its v-schema are generated by an 
expert and are stored in a DBMS, we return to generation of VDBs in 
\secref{exp-disc}. A VDB can be \emph{configured} to its pure relational 
database variants, if desired by a user, by providing the configuration
of the desired variant, \figref{vdb-conf}.
For example, a SPL developer configures a VDB to produce 
software and its database for a client.
%To configure a VDB, VDBMS requires a list of valid configurations.
%Remember that the feature model is a feature expression that 
%encodes all valid configurations. Hence, solving the feature model
%by a SAT solver results in the list of valid configurations.

%\point{flow of vq in vdbms.}
Given a VDB and its v-schema, a user inputs a v-query \vQ\ to VDBMS.
%
First, \vQ\ is checked by the \emph{type system} to determine if it is invalid, explained in 
%First, \vQ\ is type-checked by the VRA type system introduced in 
\secref{type-sys}. 
If so, the user gets errors explaining what part of the 
query violated the v-schema.
%, shown in \exref{q-violate-sch}.
%\moredet{maybe give an ex of an error user will see! ref to ex of error given
%in \secref{type-sys}}
Otherwise, 
\vQ\ is constrained by the schema,
defined in \secref{constrain},
to ensure variation-preserving property w.r.t. v-schema throughout the execution flow of v-query 
in the system and then
%
it is passed to the \emph{variation minimization} module, introduced in 
\secref{var-min}, to minimize the variation of \vQ\ and apply
relational algebra optimization rules. 
%
The optimized query is then sent to the \emph{generator} module where
SQL queries are generated from v-queries, \secref{apps} provides three
approaches for this.
\exref{q-flow} in \appref{sql-gen} demonstrates the flow of a v-query through
VDBMS.

\begin{comment}
To generate runnable queries w.r.t. the underlying DBMS,
the minimized query \ensuremath {\VVal \vQ} is passed to 
the \emph{translate to RA} module that could use either 
configuring or grouping of v-queries, explained in \secref{vra-sem},
to generate RA queries. The generated 
queries are then sent to the \emph{SQL generator} module which generates
SQL queries in various ways from the relational algebra queries, explained
in \secref{sql-gen}.
%\moredet{in app have an ex of all this happening!}
\end{comment}

%\point{vtab builder.}
Having generated SQL queries, they are now run over the underlying 
VDB (stored in a DBMS desired by the user). The result could be either 
a v-table or a list of v-tables, depending on the approach chosen in 
the translator to RA and SQL generator modules. The v-table(s) is passed
to the \emph{v-table builder}
%\dropit{could drop \secref{vtab-build} and explain it here!}
%explained in \secref{vtab-build}, 
to create one v-table that filters out 
duplicate and invalid tuples, shrinks presence conditions, and 
eventually, returns the final v-table to the user.

\begin{figure}
\includegraphics[width = \linewidth] {figs/arch7.pdf}
\caption{VDBMS architecture and execution flow of a v-query. 
The dotted double-line from v-query to pushing v-schema module
indicates the dependency of passing the v-query to this module
only if it is valid. 
The dashed gray arrows with diamond heads demonstrate
an option for the flow of input. 
%We examine taking different routes
%to evaluate a v-query, resulting in various approaches in \secref{apps}.
The blue filled arrows track the data flow, the green hollow arrows 
indicate an input to a module.}
\label{fig:arch}
\end{figure}



%\input{sections/sqlGen}


\section{Implemented Approaches}
\label{sec:apps}

\TODO{apps}


\section{Experiments}
\label{sec:exp}

\wrrite{write the damn thing.}




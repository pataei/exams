\chapter{Variational Database Framework}
\label{ch:vdb}

%\point{introduce vdb and its parts.}
%A \emph{Variational Database (VDB)} is intuitively meaningful when a set of database 
%instances with variations in the schema and/or content exists and a user's information
%need requires accessing some or all of them simultaneously. 
\eric{just added the first sentence.}
\rewrite{define framework and vdb separetly}
In this chapter, we introduce the \emph{variational database} (VDB) framework and how it encodes
variation directly in relational databases.
To incorporate variation within a database, we annotate elements with feature expressions,
as introduced in \secref{vset}. We use annotated elements both in the schema and content.
Within a schema we allow attributes and relations to exist 
conditionally based on the feature expression assigned to them (\secref{vsch}).
At the content level, we annotate each tuple with a feature expression, indicating when the tuple 
is present (\secref{vtab}). 


%\point{define \vctxTxt.}
%\eric{could you pls review this?}
%Consequently, the existence of parts of a VDB is variational (conditional). 
%This condition potentially changes by running queries, explicated in \secref{vrel-alg}.
%It is captured by a feature expression, called \emph{variational context} and 
%denoted by \vctx. 
%In our definitions of VDB parts, we discuss the condition under which the part is
%valid and present. The condition holds when a feature expression, \dimMeta,
%is satisfiable.
%Such a condition is always considered under the variational context of VDB
%at the moment. To apply this restriction, we conjunct the variational context with
%\dimMeta. Hence, the condition is the satisfiability of $\dimMeta \wedge \vctx$.
%For brevity, we do not explain this over and over again. Rather,
%we specify it in the condition for validity of an element.

%\subsection{Variational Database Configuration}
\label{sec:vdbconf}

\TODO{vdb configuration}


\begin{table}
\caption[Examples of relation schemas and a variational relation schema]{The relation schema of \empbio\ for variants that enable one of the features \vThree, \vFour, or \vFive\ and the variational relation schema of \empbio\ encompassing 
the three variants of the plain relation \empbio.}
\label{tab:empbio-sch}
\centering
\small
%\footnotesize
%\scriptsize
\begin{subtable}[t]{\textwidth}
\centering
\caption{The relation schema of \empbio\ for variants that enable the feature \vThree.}
\label{tab:empbio-v3}
\begin{tabular} {c | l l l}
%\hline
\multirow{2}{*}{\empbio} & \empno & \sex & \birthdate\\
\cline{2-4}
%\hline
 &12001 & F& 1960-11-06\\
\arrayrulecolor{white}\hline
\end{tabular}
\end{subtable}

\medskip
\medskip
\medskip
\begin{subtable}[t]{\textwidth}
\centering
\caption{The relation schema of \empbio\ for variants that enable the feature \vFour.}
\label{tab:empbio-v4}
\begin{tabular} {c | l l l l}
\multirow{2}{*}{\empbio}  & \empno & \sex & \birthdate & \name\\
\cline{2-5}
 &80001 & M & 1956-09-30 & Nagui Merli \\
\arrayrulecolor{white}\hline
\end{tabular}
\end{subtable}

\medskip
\medskip
\medskip
\begin{subtable}[t]{\textwidth}
\centering
\caption{The relation schema of \empbio\ for variants that enable the feature \vFive.}
\label{tab:empbio-v5}
\begin{tabular} {c | l l l l l}
\multirow{2}{*}{\empbio}  & \empno & \sex & \birthdate & \fname & \lname\\
\cline{2-6}
 & 200000 & M & 1960-01-11 & Selwyn & Koshiba \\
\arrayrulecolor{white}\hline
\end{tabular}
\end{subtable}

\medskip
\medskip
\medskip
\begin{subtable}[t]{\textwidth}
\centering
\footnotesize
\caption{The variational relation schema of \empbio.}
\label{tab:empbio-vsch}
\begin{tabular} {c | l l l l l l l}
%\hline
%\hhline{-==}
\textcolor{blue}{$\vThree \vee \vFour \vee \vFive$} & \textcolor{blue}{\t} & \textcolor{blue}{\t} & \textcolor{blue}{\t} & \textcolor{blue}{$\vFour \wedge \neg \vThree \wedge \neg \vFive$} & \textcolor{blue}{$\vFive \wedge \neg \vThree \wedge \neg \vFour$} & \textcolor{blue}{$\vFive \wedge \neg \vThree \wedge \neg \vFour$}\\
\arrayrulecolor{blue}\hdashline
\multirow{2}{*}{\empbio}  & \empno & \sex & \birthdate & \name & \fname & \lname\\
\arrayrulecolor{black}\cline{2-7}
 &12001 & F& 1960-11-06 & & &  \\
  &80001 & M & 1956-09-30 & Nagui Merli & & \\
   & 200000 & M & 1960-01-11 & & Selwyn & Koshiba \\
\arrayrulecolor{white}\hline
%\job & \titleatt & \salary\\
%\cline{2-3}
%& Assistant Engineer & 61594\\
%& Senior Engineer & 96646\\
%& \ldots & \ldots \\
%& Staff & 77935\\
%& Technique Leader & 58345
\end{tabular}
\end{subtable}

\end{table}

\section{Variational Table}
\label{sec:vtab}

%\TODO{change vdb conf to vtab.}
%\rewrite{read and revise}
%\fromppr{vldb}
%\TODO{remember to use definitions of config elem and vset}
%\TODO{give example of vtab for mot ex}
%%\begin{figure}
%[ht]
%
%\begin{comment}
%\textbf{Relational model generic objects:}
%\begin{syntax}
%%OLD
%D\in \mathbf{Dom} &&& \textit{Domain}\\
%A\in \mathbf{Att} &&& \textit{Attribute Name}\\
%R\in \mathbf{R} &&& \textit{Relation Name}\\
%t \in \mathbf{T} &&& \textit{Tuple}
%\end{syntax}
%
%\medskip
%\textbf{Relational model definition:}
%\begin{syntax}
%l\in \mathbf{L} &=& \vn{A} &\textit{Attribute set}\\
%s \in \mathbb{S} &=& R(A_1, \ldots , A_n ) & \textit{Relation specification}\\
%S \in \mathcal{\mathbf{S}} &\Coloneqq& {\vn{s}} & \textit{Schema}\\
%T \in \mathbf{T} &\Coloneqq& \{\llangle t(1), \ldots, t(k)\rrangle \myOR \\
%&&t(i) \in D_i,
%1 \leq i \leq k ,\\
%&&k = \mathit{arity}(R) \} 
%%v_1^1\in D_1, \ldots, v_n^1\in D_n\rrangle, \ldots, \llangle v_1^m\in D_1, \ldots, v_n^m\in D_n\rrangle|\\
%%&& \hspace{0.5cm} m = \textit{number of } R_I\textit{'s tuples}\} 
%&\textit{Relation Instance (Table)}\\
%%I \in \mathbf{Inst} &\Coloneqq& R_{1_I}, \cdots, R_{n_I} & \textit{Database Instance}
%\end{syntax}
%
%
%\medskip
%\textbf{Variational relational algebra objects:}
%\begin{syntax}
%\synDef \dimMeta \ffSet &&&\textit{Presence condition}\\
%\synDef \vAtt \vAttSet &&&\textit{Variational attribute}\\
%\synDef \vAttList \vAttSet &\eqq& \vAtt, \vAttList \myOR \empAtt &\textit{Variational attribute list}\\
%\synDef \vRelSch \vRelSet &\eqq& \vRelDef &\textit{Variational relation schema}\\
% \vRel &&&\textit{Variational relation}\\
%\synDef \vSch \vSchSet &\eqq& \vSchDef &\textit{Variational schema}\\
% &&&\textit{Variational database instance}
%\end{syntax}
%\end{comment}
%
%%%%%%%%%%%%%%%%%%%%%%%%%%%%%%%%%%%%%%%%%%%%%%%%%%%%
\textbf{Variational database objects:}
\begin{syntax}
%\synDef \vAtt \attNames &&&\textit{Attribute Name}\\
%\synDef \vRel \relNames &&& \textit{Relation Name}\\
%\synDef \vAttList {\boldmth{\mathbf{V} \mathbf{A}}} &\eqq& 
%\setDef {\annot [\dimMeta_1] \vAtt_1, \annot [\dimMeta_2] \vAtt_2, \ldots, \annot [\dimMeta_k] \vAtt_k} & \textit{Variational Set of Attributes}\\
%\synDef \vRelSch \vRelSchSet &\eqq& \vRelDef & \textit{Variational Relation Schema}\\
%\synDef \vSch \vSchSet &\eqq& \vSchDef & \textit{Variational Schema}
%
\synDef \vTuple {\vartype \tupletype} &\eqq& \annot[\dimMeta]{(\vi v \numAtts)} & \textit{Variational Tuple}\\
%\vTuple\in\vRelCont \eqq \annot[\dimMeta_\vTuple]{(\vi v \numAtts)}
\synDef \vRelCont \vRelContSet &\eqq& \setDef {\vi \vTuple \numTuples} & \textit{Variational Relation Content}\\
%\vRelCont \in \vRelContSet \eqq \setDef {\vi \vTuple \numTuples}
\synDef \vTab \tabletype &\eqq& (\vRelSch, \vRelCont) & \textit{Variational Table}\\
\synDef \vdbInst \vdbInstSet &\eqq& \annot [\dimMeta] {\setDef {\vi \vTab \numRels} } & \textit{Variational Database Instance}
%\synDef \vdbInst  \vdbInstSet \eqq \annot [\dimMeta] {\setDef {\vi \vTab \numRels} }
\end{syntax}

\medskip
\textbf{Variational database type synonyms:}
\begin{alignat*}{1}
\vRelContSet &= \settype {(\vartype \tupletype)}\\
\tabletype &= \typepair{\vRelSchSet, \vRelContSet}\\
\vdbInstSet &= \vartype {\left( \settype {\left(\left(\relschtype,\relconttype\right)\right)}\right)}
\end{alignat*}

\medskip
\textbf{Variational tuple configuration:}
%
\begin{alignat*}{1}%\raggedleft
\ouSemType [] . &: {\vartype \tupletype} \to \vRelSchSet \to \confSet \to \maybe \tupletype\\
%\end{flalign*}
%
%\begin{flalign*}%\raggedleft
\ouSem{\vRelSch} {\annot  {\left( {\vi v \numAtts}\right)}}  &\\
& \hspace{-50pt} = \begin{cases}
(v_i, \cdots, v_j), &\If \forall k. 1 \leq i \leq k \leq j \leq l, \fSem {\getPCfrom {\getAtt {k}} \vRelSch \wedge \dimMeta} = \t\\
\bot, &\Otherwise
\end{cases}
%\left( \ovSem {v_1}, \hdots, \ovSem {v_\numAtts} \right) &\\
%& \textit{ where } \forall 1 \leq i \leq \numAtts: \\
%&\hspace{5pt} \ovSem {v_i} = 
%\begin{cases}
%v_i, & \If \fSem {\fModel \wedge \getPC{\getRel{\getAtt{v_i}}} \wedge \getPC {\getAtt {v_i}} \wedge \dimMeta_\tuple} \\
%\varepsilon, & \Otherwise
%\end{cases}
\end{alignat*}

%\medskip
\textbf{Variational relation content configuration:}
%
\begin{alignat*}{1}%\raggedleft
\otSemType [] . &: \vRelContSet \to \vRelSchSet \to \confSet \to \pRelContSet\\
%\end{flalign*}
%
%\begin{flalign*}%\raggedleft
\otSem {\vRelSch} {\setDef {\vi \tuple \numTuples}} &= \setDef {\ouSem {\vRelSch}{\tuple_1}, \hdots, \ouSem{\vRelSch} {\tuple_\numTuples}}
\end{alignat*}

%\medskip
\textbf{VDB instance configuration:}
%
\begin{alignat*}{1}%\raggedleft
\odbSem [] . &: \vdbInstSet \to \confSet \to \pInstSet\\
%\end{flalign*}
%
%\begin{flalign*}%\raggedleft
\odbSem { \annot  {\setDef {\vi \vTab \numRels}}} 
&=\odbSem { \annot  { \setDef {\left( \vRelSch_1, \vRelCont_1\right), \ldots, 
\left( \vRelSch_\numRels, \vRelCont_\numRels\right)}}}\\
& = \begin{cases}
\setDef{\left( \orSem {\vRel_1 \annot [\dimMeta_1 {\wedge \dimMeta}] {\left( \vAttList_1 \right)} }, 
\otSem {\vRelSch_1} {\pushIn {\annot [\dimMeta_1 {\wedge \dimMeta}] \vRelCont_1}} \right), \ldots}, &\If \fSem \dimMeta = \t \\
\setDef {}, \Otherwise
\end{cases}
%&= \setDef {(\orSem {\vRelSch_1}, \otSem {\vRelCont_1}), \hdots, (\orSem {\vRelSch_\numRels}, \otSem {\vRelCont_\numRels} )}&
\end{alignat*}

\caption{VDB instance syntax and configuration.
%The input to all configuration functions assumes a well-formed input,
%either a v-cond (see \secref{type-sys} or a (part of a) VDB.
%\TODO{you need to define well-formedness for vdb and mention it 
%for vcond somewhere in the paper.}
%%V-cond configuration only accepts conditions that are type correct. 
%%All the configuration functions are defined over a given database
%%with v-schema \vSch. 
%A set with question mark at the end, e.g., \maybe \pRelSchSet, 
%denotes an optional type, meaning that the original set is extended
%with a non-value, \ensuremath{\bot}.
%\revised{
Note that the schema of a relation must be passed to the configuration function
for its content,
however, the variational schema does not need to be passed to configuration 
functions of smaller parts of the variational schema such as \orSem .  or \olSem .
since all needed information for configuring a part of a variational schema
is propagated. 
%Note that $\vRelContSet = \settype {(\vartype \tupletype)}$.
%}
% of variational set of attributes, v-relations, and v-schema.
%$\varepsilon$ denotes a non-existent relation and value.
%Note that the feature model and 
%relation presence condition are passed all the way to attributes due to the 
%hierarchal structure of presence conditions within a v-schema.
}
\label{fig:vdb-conf}
\end{figure} 

%
%Variation also exists in database content. To account 
%for content variability, we tag tuples with 
%presence conditions. 
%%e.g., the tuple $(1,2)^{\A}$ only exists
%%when \A\  is enabled. 
%%
%Thus, a \emph{variational tuple} (\emph{v-tuple}) is an annotated tuple,
%$\vTuple\in\vRelCont \eqq \annot[\dimMeta_\vTuple]{(\vi v \numAtts)}$. A
%variational tuple corresponds to a variational relation,
%$\vRel\annot[\dimMeta_\vRel]{(\vi \vAtt \numAtts)}$,
%where each element $v_i$ is a value corresponding to attribute $\vAtt_i$
%(recall that attributes in a variational relation are ordered).
%%
%For example, $\annot[\tFive]{(38, PL, 678)}$ is a variational tuple that belongs to the
%\ecourse\ relation from \exref{vsch} and is only present when \tFive\ is
%enabled. 
%%
%The content of a variational relation
%%  \emph{variational relation content} 
%is a set of variational tuples,
%$\vRelCont \in \vRelContSet \eqq \setDef {\vi \vTuple \numTuples}$
%and 
%%
%a \emph{variational table} (\emph{v-table}) is the pair of its relation
%schema and content, $\vTab = (\vRelSch, \vRelCont)$.
%%
%A \emph{variational database instance}
%%of VDB \vDB\ with variational schema \vSch, 
%is a set of variational tables,
%$\synDef \vdbInst  \vdbInstSet \eqq \annot [\fModel] {\setDef {\vi \vTab \numRels} }$.
%%
%A VDB instance is \emph{well-formed} if its encoded variation at
%the schema and content level are consistent and satisfiable~\cite{ALW21vamos}.
%% We define properties that must hold for a VDB to be well-formed can be 
%% found in~\cite{ALW21vamos}.
%
%
%\NOTE{The explanation in the following two paragraphs is really hard to follow.
%Some ideas for improvement: (1) State the requirement that you're talking about
%first, then explain how VDB satisfies it; currently the requirement comes at
%the end, so for 1--3 sentences the reader is wondering why you're re-hashing
%this aspect of VDB. (2) The reader has forgotten by now what the requirements
%are, so a brief (few words) description is needed for each requirement as you
%discuss it. (3) It would help to follow the same order of requirements as
%\secref{mot} as closely as possible. 
%
%\medskip
%I recommend structuring this discussion more rigidly as, ``This encoding
%satisfies \nZero, which is about foo, by doing bar. It satisfies \nOne, which
%is about blah, by doing baz.'' Obviously there's plenty of room for making it
%read more nicely than that, but the lack of structure is making it hard to
%understand as-is.}
%
%
%This encoding of variational databases satisfies the requirements for a
%variational database described in \secref{mot}. Similar to a variational schema, a user
%can configure a variational table or a VDB for a specific variant, formally defined in
%\figref{vdb-conf} in \appref{vdb-conf}. This allows users to deploy a VDB for a
%specific configuration and generate the corresponding VDB variant, satisfying
%database part of \nThree\ need.
%%
%Additionally, 
%our VDB framework puts all variants of a database into
%one VDB (satisfying \nZero) 
%and it keep tracks of which variant a tuple belongs to by 
%annotating them with presence conditions. 
%For example, consider tuples
%\ensuremath{\annot [\tFive] {(38, PL, 678)}}
%and 
%\ensuremath{\annot [\tFour] {(23, DB, \nul)}}
%that belong to the \ecourse\ table. 
%The presence conditions \tFive\ and \tFour\ state that tuples belong to temporal
%variants four and five of this VDB, respectively.
%Hence, this framework tracks which variants a tuple belongs to 
%(first part of \nTwo).
%
%
%%As shown, o
%Our VDB framework encodes variation in databases 
%at two levels: schema and content.
%% We do not extend variation to 
%%the constraint level and only focus on variation at the schema 
%%and content levels. 
%In a database that is a variational artifact as defined in \secref{req},
%while content-level variation can stand on its own, such as
%frameworks used for database versioning and 
%experimental databases~\cite{dbVersioning},
%the schema level cannot, e.g., 
%\ensuremath{
%\ecourse \left(\cno, \cname, \optAtt [\tFive] [\deptno] \right)^{\edu \wedge \left(\tFour \vee \tFive\right)}
%} encodes variation at the schema level for relation \ecourse.
%Dropping the presence conditions of tuples leads to ambiguity, i.e.,
%it is unclear which variant each of the tuples
%\ensuremath{(38, PL, 678)}
%and 
%\ensuremath{(23, DB, \nul)} belongs to. We can only guess that
%they belong to variants where \tFour\ or \tFive\ are enabled, but, 
%we do not know for sure which one. Thus, it violates the \nTwo\ 
%requirement of a variational database framework.
%%where it is unclear which variant each tuple belongs to
%%and there is no way to recover such information.
%%Note that 
%%the VDB framework encodes both schema- and content-level
%%variation. A simpler framework could be used to encode 
%%only content-level variation (where tables consist of variational tuples but
%%have plain relational schema), similar to frameworks used for 
%%database versioning and experimental databases~\cite{dbVersioning}.
%%However, schema-level variation cannot be encoded without 
%%accounting for content-level variation in a framework where
%%variants coexist in parallel and they are all put into one database,
%%e.g., while 
%%\ensuremath{
%%\ecourse \left(\cno, \cname, \optAtt [\tFive] [\deptno] \right)^{\edu \wedge \left(\tFour \vee \tFive\right)}
%%} encodes variation at the schema-level for relation \ecourse,
%%dropping presence conditions of tuples results in tuples
%%\ensuremath{(38, PL, 678)}
%%and 
%%\ensuremath{(23, DB, \nul)}
%%where it is unclear which variant each tuple belongs to
%%and there is no way to recover such information.
%
%
%Note that we limit the granularity of variation in content to tuples, that is,
%the individual values within a tuple are not variational.
%%
%%Note that the value $v_i$ is present iff 
%%$\sat {\dimMeta_\vTuple \wedge \dimMeta_\vRel \wedge \dimMeta_\vAtt \wedge \fModel}$,
%%where, 
%%$\dimMeta_\vAtt = \getPC {\getAtt i}$ and
%%%,
%%%$\dimMeta_\vTuple = \getPC \vTuple$,
%%%\dimMeta = \getPC \vRel,
%%%and 
%%%\fModel\ is the feature model.
%%%
%%for simplicity, 
%%%Also, note that to avoid overcrowding the database with variation and feature 
%%%expressions
%%we only annotate tuples and not cells. 
%This design decision causes some redundancy.
%For example, the two tuples
%\ensuremath{\annot [\tFive] {(38, PL, 678)}} and 
%\ensuremath{\annot [\neg \tFive] {(38, PL, \nul)}}
%%\ensuremath{\annot [\fOne] {(1,2)}} and \ensuremath{\annot [\neg\fOne] {(1,3)}}
%cannot be represented as a single tuple
%% \ensuremath{(1, \chc [\fOne] {2,3})} 
%with variation in the third element. However, this design decision
%does not prevent us from distinguishing between a \nul\ value
%that represents a missing value and a \nul\ value that represents
%a cell that is not present. This distinction can be made by checking
%the satisfiability of 
%the presence condition of the value $v_i$ in tuple \vTuple\ of relation \vRel\ in schema \vSch:
%If $\sat{\getPCfrom {v_i} \vTuple}$ then the \nul\ indicates a missing value
%and otherwise it indicates a non-present cell, where 
%\ensuremath{\getPCfrom {v_i} \vTuple = \dimMeta_\vTuple \wedge \getPCfrom \vRel \vSch
%\wedge \getPCfrom {\getAtt i} \vRel}.
%% \dimMeta_\vRel \wedge \dimMeta_\vAtt \wedge \fModel
%%\revised{
%%where \ensuremath{\chc [\fOne] {2,3}} is a \emph{choice} of values $2$ and $3$
%%and it states that if \A\ is enabled the cell holds the value $2$ and otherwise it 
%%holds the value $3$. }
%%
%
%%\begin{figure}

%%%%%%%%%%%%%%%%%%%%%%%%%%%%%%%%%%%%%%%%%%%%%%%%%%%
\textbf{Variational Set of Attributes Configuration:}
\begin{flalign*}
& \olSem [] . : \ \vAttSet \to \confSet \to \pAttSet&\\
%\end{flalign*}
%
%\begin{flalign*}
& \olSem {\{\optAtt\} \cup \vAttList} \spcEq  
    \begin{cases}
        \{\pAtt\} \cup \olSem{\vAttList},
                            & \If \fSem {\dimMeta \wedge \getPC{\getRel \vAtt} \wedge \fModel} \\
        \olSem{\vAttList} , & \Otherwise
     \end{cases} &\\
% & \olSem {\{\optAtt\} \cup \vAttList} &\spcEq &\  \olSem {\{\optAtt\}} \cup \olSem {\vAttList}\\
& \olSem {\setDef{}} \spcEq  \setDef{}&
\end{flalign*}

%
\medskip
\textbf{Variational Relation Schema Configuration:}
\begin{flalign*}%\raggedleft
&\orSem [] . : \vRelSchSet \to \confSet \to \pRelSchSet&\\
%\end{flalign*}
%
%\begin{flalign*}
&\orSem \vRelDef = 
	\begin{cases}
		\vRel({\olSem {\vAttList}}, &\If \fSem {\dimMeta \wedge \fModel}) \\
		\empRel, &\Otherwise
	\end{cases}&
\end{flalign*}

%
\medskip
\textbf{Variational Schema Configuration:}
\begin{flalign*}%\raggedleft
&\osSem [] . : \vSchSet \to \confSet \to \pSchSet&\\
%\end{flalign*}
%
%\begin{flalign*}
&\osSem {\annot [\fModel] {\setDef {\vRelDefNum 1, \ldots, \vRelDefNum \numRels}}}
%&\hspace{0.3cm}
= \begin{cases}
%		\setDef {\orSem {\vRelDefNumF 1}, \ldots, \orSem {\vRelDefNumF n}},
                 \setDef {\orSem {\vRel_1( \vAttList_1 )^{\dimMeta_1 \wedge \fModel} }, \ldots, 
                 \orSem {\vRel_\numRels( \vAttList_\numRels)^{\dimMeta_\numRels \wedge \fModel} }},		
        & \If \fSem \fModel \\
        \setDef{}, & \text{otherwise}
	\end{cases}&
\end{flalign*}

\medskip
\textbf{Variational Tuple Configuration:}
%
\begin{flalign*}%\raggedleft
&\ouSem [] . : \vRelCont \to \confSet \to \pRelCont&\\
%\end{flalign*}
%
%\begin{flalign*}%\raggedleft
&\ouSem {\annot [ \dimMeta_\tuple] {\left( {\vi v \numAtts}\right)}} = \left( \ovSem {v_1}, \hdots, \ovSem {v_\numAtts} \right) 
%&\\
& \textit{ where } \forall 1 \leq i \leq \numAtts: 
%&\hspace{5pt} 
\ovSem {v_i} = 
\begin{cases}
v_i, & \If \fSem {\fModel \wedge \getPC{\getRel{\getAtt{v_i}}} \wedge \getPC {\getAtt {v_i}} \wedge \dimMeta_\tuple} \\
\varepsilon, & \Otherwise
\end{cases}
\end{flalign*}

\medskip
\textbf{V-Relation Content Configuration:}
%
\begin{flalign*}%\raggedleft
&\otSem [] . : \vRelContSet \to \confSet \to \pRelContSet&\\
%\end{flalign*}
%
%\begin{flalign*}%\raggedleft
&\otSem {\setDef {\vi \tuple \numTuples}} = \setDef {\ouSem {\tuple_1}, \hdots, \ouSem {\tuple_\numTuples}}&
\end{flalign*}

\medskip
\textbf{VDB Instance Configuration:}
%
\begin{flalign*}%\raggedleft
&\odbSem [] . : \vInstSet \to \confSet \to \pInstSet&\\
%\end{flalign*}
%
%\begin{flalign*}%\raggedleft
&\odbSem { {\setDef {\vi \vTab \numRels}}} = \setDef {(\orSem {\vRelSch_1}, \otSem {\vRelCont_1}), \hdots, (\orSem {\vRelSch_\numRels}, \otSem {\vRelCont_\numRels} )}&
\end{flalign*}

\caption{
V-cond and VDB instance configurations.
% of variational set of attributes, v-relations, and v-schema.
$\varepsilon$ denotes a non-existent relation and value.
%Note that the feature model and 
%relation presence condition are passed all the way to attributes due to the 
%hierarchal structure of presence conditions within a v-schema.
}
\label{fig:vdb-conf}
\end{figure} 


%\section{Variational Database}
\label{sec:vdb}

\TODO{vdb}

\section{Properties of a Variational Database Framework}
\label{sec:vdbfprop}


In this section, we describe a set of basic properties that a well-formed VDB
should satisfy.
%
These checks ensure that presence conditions are consistent and satisfiable,
which ensures that each element is present in at least one variant.
%
In the following, $\sat\dimMeta$ denotes a satisfiability check
that returns \t\ if the feature expression \dimMeta\ is satisfiable and \f\
otherwise.


A well-formed v-schema should have the following properties:
%
\begin{enumerate}
%
\item There is at least one valid configuration of the VDB feature model $\getPC \vSch$:\\
%
\centerline{
$\sat {\getPC \vSch}$}
%
\item Every relation \vRel\ is present in at least one configuration of the
variational schema \vSch:\\
%
\centerline{
$\forall\vRel\in\vSch. \sat{\getPCfrom \vRel \vSch}$}
%
\item Every attribute \vAtt\ in every relation \vRel\ is present in at least one
configuration of the variational schema \vSch:\\
%
\centerline{
$\forall\vAtt\in\vRel, \forall\vRel\in\vSch.
\sat{\getPCfrom \vRel \vSch \wedge\getPCfrom \vAtt \vRel}$}
%
\item If $\vSch_\config$ denotes the expected plain relational schema for
configuration $c$ of the variational schema \vSch, then configuring the
variational schema with that configuration, written $\sem[\config]{\vSch}$,
actually yields that variant:\\
%
\centerline{
$\forall\config\in\confSet. \osSem {\vSch} = \vSch_\config$}
%
\end{enumerate}


\noindent
%
At the data level, a well-formed VDB should have these properties:
%
\begin{enumerate}
%
\item Every tuple \vTuple\ in relation \vRel\ is present in at least one variant:\\
%
\centerline{
$\forall\vTuple\in\vRel, \forall\vRel\in\vSch.
\sat{\getPCfrom\vRel \vSch\wedge\getPCfrom \vTuple \vRel}$ }
%
\item For every tuple \vTuple\ in relation \vRel, if an attribute \vAtt\ in \vRel\ is
not present in any variants of the tuple, then the value of that attribute in
the tuple, written $\mathit{value}_\vTuple(\vAtt)$, should be NULL:\\
%\centerline{
$\forall\vTuple\in\vRel, \forall\vAtt\in\vRel,\forall \vRel\in\vSch.
\neg\sat{\getPCfrom \vRel \vSch \wedge\getPCfrom \vAtt \vRel\wedge\getPCfrom \vTuple \vRel}
\Rightarrow \mathit{value}_\vTuple(\vAtt) = \nul$
%
\end{enumerate}


\noindent
%
%
 Since a single VDB can supply data for many different database variants at the
 same time, encoding variation explicitly in a database allows the developers
 to check for different properties over all database variants.
%
Thus, depending on the context of the VDB, more specialized properties can be checked.
For example, if temporal variability in a database is accumulated over
variants (i.e.\ old data is included in more recent variants in addition to
newly added data), it is desirable to ensure that older variants are subsets of
newer variants.
%
This property should hold for our employee dataset, introduced in 
\secref{emp-vdb}. To check this, 
assume that configurations \ensuremath{\config_1, \config_2, \cdots}
represent time-ordered configurations, then check
\ensuremath{
\forall \config_i, \config_j \in \confSet, i \le j, \odbSem [\config_i] {\vdbInst} \subseteq \odbSem [\config_j]{\vdbInst}
}, 
where \ensuremath{\odbSem[\config]{\vdbInst}} denotes configuring the VDB instance
\vdbInst\ for configuration \config, defined in \figref{vdb-conf}.
%\parisa{note to myself, impl todo: actually check this for employee db when you got the time!}

%v-table checks:
%- \ensuremath{\forall tuple \in relation \in schema : sat (fm \wedge pc_relation \wedge pc_tuple)}\\
%- \ensuremath{\forall attribute \in relation \in schema, \forall val : if unsat (fm \wedge pc_relation \wedge pc_attribute \wedge pc_tuple)} then value must be null\\



\chapter{Variational Database Framework}
\label{ch:vdb}

%\point{introduce vdb and its parts.}
%A \emph{Variational Database (VDB)} is intuitively meaningful when a set of database 
%instances with variations in the schema and/or content exists and a user's information
%need requires accessing some or all of them simultaneously. 
\eric{just added the first sentence.}
\rewrite{define framework and vdb separetly}
In this chapter, we introduce the \emph{variational database} (VDB) framework and how it encodes
variation directly in relational databases.
To incorporate variation within a database, we annotate elements with feature expressions,
as introduced in \secref{vset}. We use annotated elements both in the schema and content.
Within a schema we allow attributes and relations to exist 
conditionally based on the feature expression assigned to them (\secref{vsch}).
At the content level, we annotate each tuple with a feature expression, indicating when the tuple 
is present (\secref{vtab}). 


%\point{define \vctxTxt.}
%\eric{could you pls review this?}
%Consequently, the existence of parts of a VDB is variational (conditional). 
%This condition potentially changes by running queries, explicated in \secref{vrel-alg}.
%It is captured by a feature expression, called \emph{variational context} and 
%denoted by \vctx. 
%In our definitions of VDB parts, we discuss the condition under which the part is
%valid and present. The condition holds when a feature expression, \dimMeta,
%is satisfiable.
%Such a condition is always considered under the variational context of VDB
%at the moment. To apply this restriction, we conjunct the variational context with
%\dimMeta. Hence, the condition is the satisfiability of $\dimMeta \wedge \vctx$.
%For brevity, we do not explain this over and over again. Rather,
%we specify it in the condition for validity of an element.

%\subsection{Variational Database Configuration}
\label{sec:vdbconf}

\TODO{vdb configuration}


\begin{table}
\caption[Examples of encoding variation at the schema level]{The relational tables of \empbio\ for variants that enable one of the features \vThree, \vFour, or \vFive\ and 
the variational relation \empbio\ that encompasses
the three variants of the plain table \empbio\ without accounting for variation at the content level.
Note that data from earlier variants like \setDef \vThree\ is propagated to the later variants like \setDef \vFour\ and \setDef \vFive.}
\label{tab:empbio-tab}
\centering
\small
%\footnotesize
%\scriptsize
\begin{subtable}[t]{\textwidth}
\centering
%\footnotesize
\scriptsize
\caption{The relational table of \empbio\ for variants that only enables the feature \vThree\ out of
the features \vOne--\vFive. The relation schema is captured by the name of the relation and its attributes.}
\label{tab:empbio-v3}
\begin{tabular} {c | l l l}
%\hline
\multirow{2}{*}{\empbio} & \empno & \sex & \birthdate\\
\cline{2-4}
%\hline
 &12001 & F& 1960-11-06\\
  &12002 & M& 1961-04-15\\
   &12003 & M& 1958-07-27\\
   &\ldots & \ldots & \ldots \\
\arrayrulecolor{white}\hline
\end{tabular}
\end{subtable}

\medskip
\medskip
\medskip
\begin{subtable}[t]{\textwidth}
\centering
%\footnotesize
\scriptsize
\caption{The relational table of \empbio\ for variants that only enables the feature \vFour\ out of
the features \vOne--\vFive.}
%The relation schema is captured by the name of the relation and attributes.}
\label{tab:empbio-v4}
\begin{tabular} {c | l l l l}
\multirow{2}{*}{\empbio}  & \empno & \sex & \birthdate & \name\\
\cline{2-5}
 &12001 & F& 1960-11-06 & Ulf Hofstetter\\
  &12002 & M& 1961-04-15 &Luise McFarlan \\
   &12003 & M& 1958-07-27 & Shir DuCasse \\
 &80001 & M & 1956-09-30 & Nagui Merli \\
 & 80002 & M & 1963-04-25 & Mayuko Meszaros\\
 & 80003 & F & 1960-10-26 & Theirry Viele\\
 & \ldots & \ldots & \ldots & \ldots \\
\arrayrulecolor{white}\hline
\end{tabular}
\end{subtable}

\medskip
\medskip
\medskip
\begin{subtable}[t]{\textwidth}
\centering
%\footnotesize
\scriptsize
\caption{The relational table of \empbio\ for variants that only enables the feature \vFive\ out of
the features \vOne--\vFive.}
%The relation schema is captured by the name of the relation and attributes.}
\label{tab:empbio-v5}
\begin{tabular} {c | l l l l l}
\multirow{2}{*}{\empbio}  & \empno & \sex & \birthdate & \fname & \lname\\
\cline{2-6}
 &12001 & F& 1960-11-06 &Ulf & Hofstetter\\
  &12002 & M& 1961-04-15 &Luise & McFarlan \\
   &12003 & M& 1958-07-27 & Shir & DuCasse \\
 &80001 & M & 1956-09-30 & Nagui & Merli \\
 & 80002 & M & 1963-04-25 & Mayuko & Meszaros\\
 & 80003 & F & 1960-10-26 & Theirry & Viele\\
 & 200001 & M & 1960-01-11 & Selwyn & Koshiba \\
 & 200002 & M & 1957-09-10 & Bedrich & Markovitch\\
 & 200003 & F & 1961-02-07 & Pascal & Benzmuller \\
 & \ldots & \ldots & \ldots & \ldots & \ldots\\
\arrayrulecolor{white}\hline
\end{tabular}
\end{subtable}

\medskip
\medskip
\medskip
\begin{subtable}[t]{\textwidth}
\centering
%\footnotesize
\tiny
\caption{The variational relation of \empbio\ without accounting for variation at the content level.
The relation schema is captured by the name of the relation and attributes in addition to their presence
conditions which are colored blue. }
%This table is present under a presence condition that applies
%to the entire database $\dimMeta_{\mathit{mot}}$ which is given in \exref{vsch-mot}.}
\label{tab:empbio-vsch}
\arrayrulecolor{blue}
\begin{tabular} {c !{\color{black}\vrule} l l l l l l }
%\hline
%\hhline{-==}
\tiny {\textcolor{blue}{$\vThree \vee \vFour \vee \vFive$} }& \tiny{\textcolor{blue}{\texttt{true}}} & \tiny{\textcolor{blue}{\texttt{true}}} & \tiny{\textcolor{blue}{\texttt{true}}} & \tiny {\textcolor{blue}{$\vFour \wedge \neg \vThree \wedge \neg \vFive$}} & \tiny {\textcolor{blue}{$\vFive \wedge \neg \vThree \wedge \neg \vFour$}} & \tiny {\textcolor{blue}{$\vFive \wedge \neg \vThree \wedge \neg \vFour$}}\\
\arrayrulecolor{blue}\hdashline
\multirow{2}{*}{\empbio}  & \empno & \sex & \birthdate & \name & \fname & \lname\\
\arrayrulecolor{black}\cline{2-7}
 &12001 & F& 1960-11-06 & Ulf Hofstetter & Ulf & Hofstetter \\
  &12002 & M& 1961-04-15 & Luise McFarlan & Luise & McFarlan \\
   &12003 & M& 1958-07-27 & Shir DuCasse & Shir & DuCasse \\
 &80001 & M & 1956-09-30 & Nagui Merli & Nagui & Merli \\
 & 80002 & M & 1963-04-25 & Mayuko Meszaros & Mayuko & Meszaros\\
 & 80003 & F & 1960-10-26 & Theirry Viele & Theirry & Viele \\
 & 200001 & M & 1960-01-11 & Selwyn Koshiba & Selwyn & Koshiba \\
 & 200002 & M & 1957-09-10 & Bedrich Markovitch & Bedrich & Markovitch\\
 & 200003 & F & 1961-02-07 & Pascal Benzmuller & Pascal & Benzmuller  \\
 & \ldots & \ldots & \ldots & \ldots & \ldots & \ldots\\
\arrayrulecolor{white}\hline
%\job & \titleatt & \salary\\
%\cline{2-3}
%& Assistant Engineer & 61594\\
%& Senior Engineer & 96646\\
%& \ldots & \ldots \\
%& Staff & 77935\\
%& Technique Leader & 58345
\end{tabular}
\end{subtable}
\end{table}

%\begin{table}
%\caption[Examples of encoding variation at the schema level]{The relational tables of \empbio\ for variants that enable one of the features \vThree, \vFour, or \vFive\ and 
%the variational relation \empbio\ that encompasses
%the three variants of the plain table \empbio\ without accounting for variation at the content level.
%}
%\label{tab:empbio-sch}
%\centering
%\small
%\footnotesize
%%\scriptsize
%\begin{subtable}[t]{\textwidth}
%\centering
%\caption{The relational table of \empbio\ for variants that only enables the feature \vThree\ out of
%the features \vOne--\vFive. The relation schema is captured by the name of the relation and its attributes.}
%\label{tab:empbio-v3}
%\begin{tabular} {c | l l l}
%%\hline
%\multirow{2}{*}{\empbio} & \empno & \sex & \birthdate\\
%\cline{2-4}
%%\hline
% &12001 & F& 1960-11-06\\
%  &12002 & M& 1961-04-15\\
%   &12003 & M& 1958-07-27\\
%   &\ldots & \ldots & \ldots \\
%\arrayrulecolor{white}\hline
%\end{tabular}
%\end{subtable}
%
%\medskip
%\medskip
%\medskip
%\begin{subtable}[t]{\textwidth}
%\centering
%\footnotesize
%\caption{The relational table of \empbio\ for variants that only enables the feature \vFour\ out of
%the features \vOne--\vFive.}
%%The relation schema is captured by the name of the relation and attributes.}
%\label{tab:empbio-v4}
%\begin{tabular} {c | l l l l}
%\multirow{2}{*}{\empbio}  & \empno & \sex & \birthdate & \name\\
%\cline{2-5}
% &80001 & M & 1956-09-30 & Nagui Merli \\
% & 80002 & M & 1963-04-25 & Mayuko Meszaros\\
% & 80003 & F & 1960-10-26 & Theirry Viele\\
% & \ldots & \ldots & \ldots & \ldots \\
%\arrayrulecolor{white}\hline
%\end{tabular}
%\end{subtable}
%
%\medskip
%\medskip
%\medskip
%\begin{subtable}[t]{\textwidth}
%\centering
%\caption{The relational table of \empbio\ for variants that only enables the feature \vFive\ out of
%the features \vOne--\vFive.}
%%The relation schema is captured by the name of the relation and attributes.}
%\label{tab:empbio-v5}
%\footnotesize
%\begin{tabular} {c | l l l l l}
%\multirow{2}{*}{\empbio}  & \empno & \sex & \birthdate & \fname & \lname\\
%\cline{2-6}
% & 200000 & M & 1960-01-11 & Selwyn & Koshiba \\
% & 200001 & M & 1957-09-10 & Bedrich & Markovitch\\
% & 200002 & F & 1961-02-07 & Pascal & Benzmuller \\
% & \ldots & \ldots & \ldots & \ldots & \ldots\\
%\arrayrulecolor{white}\hline
%\end{tabular}
%\end{subtable}
%
%\medskip
%\medskip
%\medskip
%\begin{subtable}[t]{\textwidth}
%\centering
%%\footnotesize
%\scriptsize
%%\tiny
%\caption{The variational relation of \empbio\ without accounting for variation at the content level.
%The relation schema is captured by the name of the relation and attributes in addition to their presence
%conditions which are colored blue. }
%%This table is present under a presence condition that applies
%%to the entire database $\dimMeta_{\mathit{mot}}$ which is given in \exref{vsch-mot}.}
%\label{tab:empbio-vsch}
%\begin{tabular} {c | l l l l l l l}
%%\hline
%%\hhline{-==}
%\textcolor{blue}{$\vThree \vee \vFour \vee \vFive$} & \textcolor{blue}{\texttt{true}} & \textcolor{blue}{\texttt{true}} & \textcolor{blue}{\texttt{true}} & \textcolor{blue}{$\vFour \wedge \neg \vThree \wedge \neg \vFive$} & \textcolor{blue}{$\vFive \wedge \neg \vThree \wedge \neg \vFour$} & \textcolor{blue}{$\vFive \wedge \neg \vThree \wedge \neg \vFour$}\\
%\arrayrulecolor{blue}\hdashline
%\multirow{2}{*}{\empbio}  & \empno & \sex & \birthdate & \name & \fname & \lname\\
%\arrayrulecolor{black}\cline{2-7}
% &12001 & F& 1960-11-06 & & & \\
%  &12002 & M& 1961-04-15 & & & \\
%   &12003 & M& 1958-07-27 & & & \\
% &80001 & M & 1956-09-30 & Nagui Merli & & \\
% & 80002 & M & 1963-04-25 & Mayuko Meszaros & & \\
% & 80003 & F & 1960-10-26 & Theirry Viele & & \\
% & 200001 & M & 1960-01-11 & & Selwyn & Koshiba \\
% & 200002 & M & 1957-09-10 & & Bedrich & Markovitch \\
% & 200003 & F & 1961-02-07 & & Pascal & Benzmuller  \\
% & \ldots & \ldots & \ldots & \ldots & \ldots & \ldots \\
%\arrayrulecolor{white}\hline
%%\job & \titleatt & \salary\\
%%\cline{2-3}
%%& Assistant Engineer & 61594\\
%%& Senior Engineer & 96646\\
%%& \ldots & \ldots \\
%%& Staff & 77935\\
%%& Technique Leader & 58345
%\end{tabular}
%\end{subtable}
%
%\end{table}

\section{Variational Table}
\label{sec:vtab}

\TODO{vtab}

\subsection{Variational Table Configuration}
\label{sec:vtabconf}

\TODO{vtab configuration}

%\section{Variational Database}
\label{sec:vdb}

\TODO{vdb}

\subsection{Variational Database Configuration}
\label{sec:vdbconf}

\TODO{vdb configuration}

\section{Properties of a Variational Database Framework}
\label{sec:vdbfprop}

\TODO{well-formed vdb properties.context-specific properties.}

\TODO{show that they hold for vdb.}


\NOTE{the following is taken from VaMoS}
% Since a single VDB can supply data for many different database variants at the
% same time, encoding variability explicitly in a database allows the developers
% to check for different properties over all database variants.

In this section, we describe a set of basic properties that a well-formed VDB
should satisfy.
%
These checks ensure that presence conditions are consistent and satisfiable,
which ensures that each element is present in at least one variant.
%
In the following, $\sat\dimMeta$ denotes a satisfiability check
that returns \t\ if the feature expression \dimMeta\ is satisfiable and \f\
otherwise.


A well-formed v-schema should have the following properties:
%
\begin{enumerate}
%
\item There is at least one valid configuration of the feature model $m$:
%
$\sat\fModel$
%
\item Every relation $r$ is present in at least one configuration of the
variational schema:
%
$\forall\vRel\in\vSch, \sat{\fModel\wedge\getPC\vRel}$
%
\item Every attribute $a$ in every relation $r$ is present in at least one
configuration of the variational schema:
%
$\forall\vAtt\in\vRel, \forall\vRel\in\vSch,
\sat{\fModel\wedge\getPC\vRel\wedge\getPC\vAtt}$
%
\item If $\vSch_\config$ denotes the expected plain relational schema for
configuration $c$ of the variational schema \vSch, then configuring the
variational schema with that configuration, written $\sem[\config]{\vSch}$,
actually yields that variant:
%
$\forall\config\in\confSet, \sem[\config]{\vSch} = \vSch_\config$
%
\end{enumerate}


\noindent
%
At the data level, a well-formed VDB should have these properties:
%
\begin{enumerate}
%
\item Every tuple $u$ in relation $r$ is present in at least one variant:
%
$\forall\vTuple\in\vRel, \forall\vRel\in\vSch,
\sat{\fModel\wedge\getPC\vRel\wedge\getPC\vTuple}$ 
%
\item For every tuple $u$ in relation $r$, if an attribute $a$ in $r$ is
not present in any variants of the tuple, then the value of that attribute in
the tuple, written $\mathit{value}_\vTuple(\vAtt)$, should be NULL:
$\forall\vTuple\in\vRel, \forall\vAtt\in\vRel, \forall\vRel\in\vSch,
\neg\sat{\fModel\wedge\getPC\vRel\wedge\getPC\vAtt\wedge\getPC\vTuple}
\Rightarrow \mathit{value}_\vTuple(\vAtt) = \text{NULL}$
%
\end{enumerate}


\noindent
%
We implemented these checks in our VDBMS tool and verified that both use cases
described in this paper satisfy all of them. 
%
Depending on the context of the VDB, more specialized properties can be checked
too. For example, if temporal variability in a database is accumulated over
variants (i.e.\ old data is included in more recent variants in addition to
newly added data), it is desirable to ensure that older variants are subsets of
newer variants.
%
This property should hold for our employee data set. To check this, 
assume that configurations \ensuremath{\config_1, \config_2, \cdots}
represent time-ordered configurations, then check
\ensuremath{
\forall \config_i, \config_j \in \confSet, i \le j, \sem[\config_i] {\vDB} \subseteq \sem[\config_j]{\vDB}
}, 
where \ensuremath{\sem[\config]{\vDB}} denotes configuring the VDB instance
\vDB\ for configuration \config. 
%\parisa{note to myself, impl todo: actually check this for employee db when you got the time!}

%v-table checks:
%- \ensuremath{\forall tuple \in relation \in schema : sat (fm \wedge pc_relation \wedge pc_tuple)}\\
%- \ensuremath{\forall attribute \in relation \in schema, \forall val : if unsat (fm \wedge pc_relation \wedge pc_attribute \wedge pc_tuple)} then value must be null\\



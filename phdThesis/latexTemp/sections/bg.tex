\chapter{Background}
\label{ch:bg}


The core of this thesis is injecting a new aspect to relational databases: \emph{variation}.
Thus, the goals of this chapter are twofold: 
%
first, to introduce how variation is encoded and represented in our variational database framework;
%
second, to provide the reader with the concepts and notations
used to build up the main contributions of this thesis, mainly relational databases and
relational algebra
%
and approaches used to add variation to them.
% of converting non-variational components into
%variational counterparts by using the introduced encoding of variation. 
%third, to familiarize the reader with
%main notations and design decision of the thesis. 

%
Throughout the thesis, we use types when defining concepts. 
A type is a set of possible values. For example, the type $\mathbf{Int}$
denotes all possible integers. In our formalization, we use the notation of $i \in \mathbf{Int}$ to
state that the variable $i$ is of type $\mathbf{Int}$. 
%
Types can be more general. Consider the type \settype \typevar\ that indicates the set all sets
 of values of type \typevar. Here, \typevar\ is type variable and stands for any possible type. 
Note that concrete types start with a capital letters but type variables do not.
For example, the type $\settype {\mathit{Int}}$ is the type of
sets of integers and it has values such as $\setDef {1,3,4}$ and $\setDef {\ }$.
We also use type synonyms for simplicity. For example, instead of referring to
$\settype {\mathit{Int}}$ we can give it a new name ($\mathit{Setint}= \settype {\mathit{Int}}$) 
and refer to it with the new name $\mathit{Setint}$.
\wrrite{define pair and function types}
%However, for brevity, we usually use 
%bold capital letters for types, for example, instead of $\mathit{Int}$ we write $\mathcal{I}$ to denote the type $\mathit{Int}$.
%
Sometimes we must extend a type with an additional ``bottom'' element $\bot$ and to account for this
extension at the type level we subscript the type with $\bot$. For example, $\maybe {\mathcal{I}}$
denotes the extension of the type $\mathcal{I}$ with $\bot$.
%

%
Throughout the thesis, we discuss relational concepts and their
variational counterparts. 
For clarity or when it is unclear from context, we use
an $\underline{underline}$ to distinguish a non-variational entity
from its variational counterpart, 
both at the value level and the type level,
%when we need to emphasize 
%an entity is not variational we underline it, 
e.g., \pElem\ is a 
non-variational entity while \elem\ is its variational counterpart,
if it exists.


%\secref{types} describes types. Types provide readers with a strong tool to understand 
%some formalizations easier and more intuitively.
%
\secref{rdb} describes the database model of relational databases and 
\eric{that is the schema}
the specification of
the structured used to store the data~\cite{AliceBook}. 
\secref{ra} describes the relational algebra, a query language used to query relational databases~\cite{AliceBook}.
%
\secref{encode-var} defines our encoding of the variation space used in 
variational database and how we describe parts of that space using propositional formulas of boolean variables~\cite{ATW18poly,ATW17dbpl}.
%
Finally, we introduce the main techniques used to incorporate variation into our variational 
database framework.
\secref{vset} introduces variation into sets which forms the basis of the variational database
framework~\cite{EWC13fosd,Walk14onward,ATW17dbpl} 
and \secref{fcc} describes the formula choice calculus used to incorporate 
variation into relational algebra~\cite{HW16fosd}.



%\section{Types}
\label{sec:types}

\TODO{types}


\section{Relational Databases}
\label{sec:rdb}

\TODO{relational database}


\section{The SPC Relational Algebra}
\label{sec:ra}

%\TODO{relational algebra}
%\TODO{add syntax definition}
\maybeAdd{add type system}
%\TODO{add examples with tables}
\maybeAdd{maybe add semantics later on}

Having introduced relational databases we now shift gears into querying
these databases, that is, extracting information from tables. Importantly
and almost across all relational query languages, the result of a query 
is also a table~\cite{AliceBook}. 
%
We base our variational query language on the SPC relational algebra.
Three primitive operators form the SPC algebra: \emph{selection}, \emph{projection},
and \emph{cross-product} (or Cartesian product)~\cite{AliceBook}.
We introduce these operators through \exref{ra} by stating an intent and then
building up a query to extract the information required by the intent. 

\begin{example}
Consider the database instance given in \tabref{rdb}. We want to get a list
of employees (by their names) whose salary is more than 65000 dollars. 
As the first step we use the selection operator to get the salaries that 
are more than 65000 dollars.\\
%
%\begin{equation*}
\centerline{
\ensuremath{
\underline{Q_1} = \sigma_{\salary \ge 65000} (\job)
%\end{equation*}
}}
%
\noindent
A sample of the results returned by the query $\underline{Q_1}$ is given in \tabref{ra1}.
Next a set of wide tuples is created by taking the cross-product of $\underline{Q_1}$
and \empacct.\\
%
%\begin{equation*}
\centerline{
\ensuremath{
\underline{Q_2} = \underline{Q_1} \times \empacct
%\end{equation*}
}}
%
\noindent
A sample of the results returned by the query $\underline{Q_2}$ is given in \tabref{ra2}.
However, looking closely at these results there is no connection between an employee
in the \empacct\ relation and their salary in the \job\ relation. Thus, we have to perform 
another selection to connect each employee with their title. \\
%
%\begin{equation*}
\centerline{
\ensuremath{
\underline{Q_3} = \sigma_{\empacct.\titleatt=\job.\titleatt} (\underline{Q_2})
%\end{equation*}
}}
%
\noindent
A sample of the results returned by the query $\underline{Q_3}$ is given in \tabref{ra3}.
At this point, we are only interested in two attributes, that is, \name\ and \salary.
Thus, we use projection to discard the unneeded columns.\\
%
%\begin{equation*}
\centerline{
\ensuremath{
\underline{Q_4} = \pi_{\name, \salary} (\underline{Q_3} )
%\end{equation*}
}}
%
A sample of the results returned by the query $\underline{Q_4}$ is given in \tabref{ra4}.
\end{example}

\begin{table}[!htbp]
\caption[Results of subqueries to build up the query in \exref{ra}]{Results of each step of building the final query in \exref{ra}.}
\label{tab:ra-ex}
\centering
\small
%\footnotesize
%\scriptsize
\begin{subtable}[t]{\textwidth}
\centering
\caption[shortcaption]{Result of the query \ensuremath{\sigma_{\salary \ge 65000} (\job)}.}
\label{tab:ra1}
\begin{tabular} {l l }
\titleatt & \salary \\
\hline
Senior Staff & 77935 \\
 Senior Engineer & 96646\\
 \ldots & \ldots \\
 Staff & 77935\\
Engineer & 96646
\end{tabular}
\end{subtable}

\medskip
\medskip
\medskip
\begin{subtable}[t]{\textwidth}
%\begin{center}
\centering
\tiny
\caption[shortcaption]{Result of the query \ensuremath{(\sigma_{\salary \ge 65000} (\job)) \times \empacct}.}
\label{tab:ra2}
\begin{tabular} {l l l l l l l}
%\hline
%\hhline{-==}
 \titleatt & \salary & \empno & \name & \hiredate & \titleatt & \deptname\\
\hline
Senior Engineer & 96646 & 13094 & Sanjay Servieres & 1986-01-01 & Engineer & Research \\
Staff & 77935 & 16099 & Mohan Ferretti & 1987-09-20 & Senior Staff & Human Resources\\
\ldots & \ldots & \ldots & \ldots & \ldots & \ldots & \ldots\\
Engineer & 80324 & 19162 & Chinho Fadgyas & 1986-05-19 & Technique Leader & Production \\
Senior Staff & 88070 & 22255 & Kristian Merel & 1986-09-12 & Senior Engineer & Development
\end{tabular}
%\end{center}
\end{subtable}

\medskip
\medskip
\medskip
\begin{subtable}[t]{\textwidth}
%\begin{center}
\centering
%\footnotesize
\tiny
\caption[shortcaption]{Result of the query \ensuremath{\sigma_{\empacct.\titleatt=\job.\titleatt}\left(\left(\sigma_{\salary \ge 65000} \left(\job\right)\right) \times \empacct\right)}.}
\label{tab:ra3}
\begin{tabular} {l l l l l l l}
%\hline
%\hhline{-==}
 \titleatt & \salary & \empno & \name & \hiredate & \titleatt & \deptname\\
\hline
Engineer & 96646 & 13094 & Sanjay Servieres & 1986-01-01 & Engineer & Research \\
Senior Staff & 77935 & 16099 & Mohan Ferretti & 1987-09-20 & Senior Staff & Human Resources\\
\ldots & \ldots & \ldots & \ldots & \ldots & \ldots & \ldots\\
Senior Engineer & 96646 & 22255 & Kristian Merel & 1986-09-12 & Senior Engineer & Development\\
Staff & 77935 & 43670 & JoAnna Randi & 1987-10-18 & Staff & Marketing
\end{tabular}
%\end{center}
\end{subtable}

\medskip
\medskip
\medskip
\begin{subtable}[t]{\textwidth}
\centering
\caption[shortcaption]{Result of the query \ensuremath{\pi_{\name, \salary} \left(\sigma_{\empacct.\titleatt=\job.\titleatt}\left(\left(\sigma_{\salary \ge 65000} \left(\job\right)\right) \times \empacct\right)\right)}.}
\label{tab:ra4}
\begin{tabular} {l l }
%\hline
%\hhline{-==}
\name & \salary\\
\hline
Sanjay Servieres & 96646\\
Mohan Ferretti & 77935\\
 \ldots & \ldots \\
 Kristian Mere & 96646\\
 JoAnna Randi & 77935
\end{tabular}
\end{subtable}

\end{table}



\TODO{then syntax and we add set op and empty}

\TODO{join syntactic sugar. natural join.}

\TODO{rename in imple. in examples used as arrow for understandablility.}


\fromppr{prelim}
\begin{figure}

\begin{syntax}

% feature expressions
%\synDef{\dimMeta}{\ffSet}
%  &\eqq& \multicolumn{2}{l}{%
%         \t \myOR \f \myOR \fName \myOR \neg\fName
%         \myOR \dimMeta\wedge\dimMeta \myOR \dimMeta\vee\dimMeta}
%\\[1.5ex]

% relation conditions
\synDef{\pCond}{\pCondSet}
  &\eqq& \multicolumn{2}{l}{%
         \t \myOR \f \myOR \att\bullet\cte \myOR \att\bullet\att
         \myOR \neg\pCond \myOR \pCond\vee\pCond} \\
%     &|& \multicolumn{2}{l}{\vCond\wedge\vCond \myOR \chc{\vCond,\vCond}}
\\[1.5ex]

% variational relational algebra
\synDef{\pQ}{\pQSet}
  &\eqq& \pRel                 & \textit{Relation reference} \\
     &|& \pRen[\pRel]{\pQ}     & \textit{Renaming} \\
     &|& \pPrj[\pAttList]{\pQ} & \textit{Projection} \\
     &|& \pSel\pQ              & \textit{Selection} \\
     &|& \pQ \times \pQ & \textit{Cartesian product}\\
     &|& \pQ \Join_{\pCond} \pQ  & \textit{Join} \\
     &|& \pQ \circ \pQ & \textit{Set operation}\\
%     &|& \chc{\vQ,\vQ}         & \textit{Choice} \\
%     &|& \empRel               & \textit{Empty relation} \\
%    &|& \vQ \times \vQ        & \textit{Cartesian Product} \\
%    &|& \vQ \circ \vQ         & \textit{Set operation} \\
\end{syntax}

\caption{Syntax of  relational algebra, where $\bullet$ ranges over
comparison operators ($<, \leq, =, \neq, >, \geq$), $\circ$ over 
set operations ($\cap, \cup$), \cte\ over constant values,
\att\ over attribute names, and \pAttList\ over lists of attributes.
The syntactic category
% \dimMeta\ represents feature expressions, 
 \pCond\
is relational conditions, and \pQ\ is  relational algebra terms.
}
%\vspace{-20pt}
\label{fig:rel-alg}
\end{figure}
%\vspace{-20pt}

\TODO{add bullet and conditions and attribute list to the definition.}

\figref{rel-alg} defines the syntax of 
relational algebra which allows users to query a relational database~\cite{AliceBook}.
%
The first five constructs are adapted from relational algebra:
%
A query may simply \emph{reference} a relation \pRel\ in the schema.
\emph{Renaming} allows giving a name to an intermediate query to be referenced
 later. Note that \pRel\ is an overloaded symbol that indicates both a relation
 and a relation name. 
%
A \emph{projection} enables selecting a subset of attributes from the results
of a subquery, for example, \vPrj[\pAtt_1]{\pRel} would return only attribute $\pAtt_1$
from $\pRel$.
%; we extend the standard project operator to work with annotated lists
%of attributes, for example, \vPrj[a_1,a_2^e]{r} would include $a_1$ for all
%configurations and also $a_2$ for configurations where $e$ is true.
%
A \emph{selection} enables filtering the tuples returned by a subquery based on a
given condition \pCond, for example, \vSel[\pAtt_1 > 3]{\pRel} would return all tuples
from $\pRel$ where the value for $\pAtt_1$ is greater than 3.
%; these conditions may be
%variational to enable returning different tuples for different configurations
%of the VDB.
%
A \emph{Cartesian products} simply cross products every tuple from its
left subquery with every tuple from its right subquery. 
%
The \emph{join} operation joins two subqueries based on a condition and
omitting its condition implies it is a natural join (i.e., join on the
shared attribute of the two subqueries).
For example, $\pRel_1 \bowtie_{\pAtt_1 = \pAtt_2} \pRel_2$ joins tuples from $\pRel_1$ 
and $\pRel_2$ where the attribute $\pAtt_1$ from relation $\pRel_1$ is equal to
attribute $\pAtt_2$ from relation $\pRel_2$. However, if we have $\pRel_1(\pAtt_1, \pAtt_3)$
and $\pRel_2 (\pAtt_1, \pAtt_2)$ then
$\pRel_1 \bowtie \pRel_2$ joins tuples from $\pRel_1$ and $\pRel_2$ where
attribute $\pAtt_1$ has the same value in $\pRel_1$ and $\pRel_2$. 
%
Also, note that join is simply a syntactic sugar for selection of cross product,
that is $\pQ_1 \bowtie_{\pCond} \pQ_2 = \vSel [\pCond] {(\pQ_1 \times \pQ_2)}$.
%
The set operations, union and intersection, require two subqueries to have the same set of attributes
and simply apply the operation, either union or intersection, to the tuples returned by
the subqueries.
For example, if we have $\pRel_1(\pAtt_1, \pAtt_2)$ with 
tuples $\{(1,2)(3,4)\}$
and $\pRel_2(\pAtt_1, \pAtt_2)$ with tuples $\{(1,2),(5,6) \}$
then $\pRel_1 \cup \pRel_2 $ returns the tuples $\{(1,2), (3,4), (5,6)\}$.

\fromppr{vldb}
We do not extend the notation of using underline for relational algebra
operations. Instead, relational algebra operations are overloaded and are used
as both plain relational and variational operations. It should be clear from
context when an operation is variational or not. 
%
We also extend relational algebra such that projection of an empty list of
attributes is a valid query that returns an empty set of tuples. We define the
\emph{empty} query \empPRel\ as shorthand for projecting an empty list of
attributes, that is, $\empPRel = \pi_{\{\}} \pQ$.

\section{Variation Space in a Variational Database Framework}
\label{sec:varspace}

\TODO{have to revise}
\TODO{define oplus in fig as syntactic sugar.}

%\point{using a feature set to represent variability within a context.}
We encode variability in terms of \emph{features}.
%
The \emph{feature space}, \fSet, of a variational database
is a closed set of boolean variables called features.
%
A feature \ensuremath{\fName \in \fSet} can be enabled (i.e., \fName = \t) or disabled (\fName = \f).
Features describe the variability in a given variational scenario.
%first organize the configuration space into
%a set of features \fSet.
%, denoted by \fSet.
%
%we require a \emph{set of features}, denoted by \fSet, 
%appropriate for the context that the database is used for.
%
For example, in the context of schema evolution, features can be generated from version 
numbers (e.g., features \vOne\ to \vFive\ and \tOne\ to \tFive\ in the 
motivating example, \tabref{mot}); for SPLs, 
the features can be adopted from the SPL feature set (e.g., the \edu\ feature in
our motivating example, \tabref{mot}); and 
for data integration, the features may represent different data sources.  
%For simplicity, the set of features is assumed to be closed and features are
%assumed to be boolean variables, however, it is easy to extend them
%to multi-valued variables that have finite set of values.
% and without loss of generality, 
%features are assumed to be boolean variables, although, it is easy to extend them
%to multi-valued variables. 
%A feature \ensuremath{\fName \in \fSet} can be enabled (i.e., \fName = \t) or disabled (\fName = \f).
%\point{configuration.}
%Assuming that all the features by default are set to \prog{false},
%enabling some of them specifies a variant. 

Features are used at variation points to indicate which variants a particular
element belongs to.
%, i.e,
%in \fSet\ 
%are used to 
%they indicate which parts of the database are present conditionally.
%of a variational entity within the database 
%are different 
%among different variants. 
Thus, enabling or disabling each of
the features in the feature set
%in \fSet\ 
produces a particular database \emph{variant} where
%of the entity in which 
all variation has been removed. 
%Enabling or disabling all features of \fSet\ specifies a non-variational \emph{variant}
%that can potentially be generated by \emph{configur}ing its variational counterpart 
%with the variant's \emph{configuration}.
%Hence, to specify a variant
%we define a function, called 
A \emph{configuration} is a \emph{total} function
that maps every feature in the feature set to a boolean value.
%By definition, a configuration is a \emph{total} function,
%i.e., it is defined for \emph{all} features in the feature set. 
%For brevity, 
We represent a configuration by the set of enabled features.
%which represents a variant. 
%\ensure{make sure referring to a variant can be done by the set of enabled features. HERE!}
For example, in our motivating scenario, the configuration \ensuremath{
\setDef {\vTwo,\tThree,\edu}
}
represents a database variant where only features \vTwo, \tThree, and \edu\ are
enabled (and the rest are disabled).
This database variant contains relation schemas in the yellow cells of \tabref{mot}.
%of the
%employee and education sub-schemas associated with \vTwo\ and
%\tThree\ in \tabref{mot}, respectively.
%For brevity, 
We refer to a variant by the configuration that produces it, e.g.,
%For example, 
variant \setDef {\vTwo,\tThree,\edu} refers to the variant produced by applying
that configuration.
%and education sub-schema associated with \tThree\ in \tabref{mot}.

%\point{represent variability in db by prop formula of features, called feature expression.}
%Having defined a set of features, we need to incorporate them into the database.
%To encode features in the database, we construct propositional formulas of features
%such that the formula evaluates to \t\ for a set of configurations. 
When describing variation points in the database, we need to 
refer to subsets of the configuration space. We do this with
propositional formulas of features.
Thus, 
such a propositional formula defines a condition that holds for 
a subset of configurations and their corresponding variants. 
%
%describing the condition where one or more variants are present,
%i.e., assigning features to their values defined in variant's configuration and 
%evaluating the propositional formula results in \prog{true}.
%
For example, the propositional formula $\neg \edu$ represents all variants of
our motivating example where the $\edu$ feature is disabled, i.e., variant
schemas of the  left schema column. 
%
We call a propositional formula of features a \emph{feature expression} and define
it formally in \figref{fexp-def}. 
%\figref{fexp-def} defines the syntax of feature expressions.
The evaluation function of feature expressions 
$\fSem \dimMeta : \ffSet \to \confSet \to \bSet$ simply substitutes each
feature \fName\ in the expression \dimMeta\ with the boolean value 
given by configuration \config\ and then
simplifies the propositional formula to a boolean value.
%  evaluates the 
%feature expression \dimMeta\ w.r.t. the configuration \config.
%and defined in
%our technical report~\cite{vldbArXiv}, 
%and defined in \appref{fexp},
%evaluates feature expression \dimMeta\ under configuration \config,
%also called \emph{configuration of feature expression \dimMeta\ under \config}. 
For example, 
$\fSem [\{\A\}] {\A \vee \B} = \t \vee \f = \t$, while
%which states that the feature expression
%$\A \vee \B$ evaluates to \t\ w.r.t. configuration that only enables $\A$, however,
$\fSem [\{\}] {\A \vee \B} = \f \vee \f = \f$.
%, where the empty set indicates neither \A\ nor \B\
%are enabled.
%which states that the same feature expression evaluates
%to \f\ when neither $\A$ nor $\B$ are enabled.
Additionally, in \figref{fexp-def}, we define a binary \emph{equivalence
relation} ($\equiv$) relation on feature expressions corresponding to logical
equivalence, and unary \emph{sat} and \emph{unsat} predicates that determine
whether a feature expression is satisfiable or unsatisfiable, respectively.
%
% some of the functions needed over feature
%expressions:
%1) \emph{equivalence} of two feature expressions, 
%\ensuremath{\dimMeta_1 \equiv \dimMeta_2}
%and
%2) \emph{satisfiability} of a feature expression, 
%\ensuremath{\sat \dimMeta}.
%However, we define the evaluation of feature expressions and functions over them
%in \appref{fexp}.
%We define two functions over feature expressions, as shown in \figref{fexp-def}:
%1) \emph{satisfiability}: feature expression \dimMeta\ is \emph{satisfiable} if there 
%exists configuration \config\ s.t. \fSem \dimMeta = \t\
%and 2) \emph{tautology}: feature expression \dimMeta\ is a \emph{tautology} if  
%for all valid configurations we have: \fSem \dimMeta = \t.

%\point{annotating elements of database with feature expressions.}
To incorporate feature expressions into the database,
we \emph{annotate} database elements (including attributes, relations, and tuples) 
with feature expressions. An \emph{annotated element} \elem\ with feature expression \dimMeta\
is denoted by \annot \elem. 
%The feature expression \dimMeta\ represents
%the set of configurations where their variants contain element \elem\ because
%
The feature expression attached to an element is called its \emph{presence
condition} since it determines the condition (set of configurations) under
which the element is present in the database. For example, assuming
$\ffSet=\set{\A,\B}$, the annotated number $\annot [\A \vee \B] 2$ is present
in variants \setDef{\A}, \setDef{\B}, and \setDef{\A,\B} but not in variant
\setDef{}.
%
The operation $\getPC{\annot{\elem}}=e$ returns the presence condition of an
annotated element.
% with a configuration
%that enables either $\A$ or $\B$ or both
%variants that disable both $\A$ and $\B$.
% Here, $\getPC {\annot [\A \vee \B] 2} = \A \vee \B$.

%\point{relationship between features is captured by a propositional formula, called feature model.} 
No matter the context, features often have a relationship with each other that
constrains the set of possible configurations. For example, only one of the
temporal features of \vOne--\vFive\ can be \t\ for a given variant. This
relationship is captured by a feature expression, called a \emph{feature model}
and denoted by \fModel, which restricts the set of \emph{valid configurations}.
That is, a configuration $c$ is only valid if $\fSem{\fModel}=\t$.
%
For example, the restriction that at a given time only one of temporal features
\vOne--\vFive\ 
can be enabled is represented by the feature model
$\vOne \oplus \vTwo \oplus \vThree \oplus \vFour \oplus \vFive$,
where $\A \oplus \B \oplus \ldots \oplus \fName_n$ is syntactic sugar for
$(\A \wedge \neg \B \wedge \ldots \wedge \neg \fName_n)
\vee (\neg \A \wedge \B \wedge \ldots \wedge \neg \fName_n)
\vee (\neg \A \wedge \neg \B \wedge \ldots \wedge \fName_n)$,
that is, the features are mutually exclusive.
%multiple features cannot be enabled simultaneously.
%$\left(\vOne \wedge \neg \left(\vTwo \vee \hdots \vee \vFive \right) \right)
%\vee \left(\vTwo \wedge \neg \left(\vOne \vee \vThree \vee \vFour \vee \vFive \right) \right) 
%\vee \hdots 
%\vee \left(\vFive \wedge \neg \left(\vOne \vee \hdots \vee \vFour \right) \right)$.
%Note that this is not the feature model for the entire motivating example.



\begin{figure}
%\textbf{Feature expression generic object:}
%\begin{syntax}
%\synDef \fName \fSet &\textit{Feature Name}
%%c \in \mathbf{C} &\textit{Configuration}
%\end{syntax}
%
%\medskip
\textbf{Feature expression syntax:}
\begin{syntax}
\synDef \fName \fSet &&&\textit{Feature Name}\\
\synDef \bTag \bSet &\eqq& \t \myOR \f & \textit{Boolean Value}\\
\synDef \dimMeta \ffSet &\eqq& \bTag \myOR \fName \myOR \neg \dimMeta \myOR \dimMeta \wedge \dimMeta \myOR \dimMeta \vee \dimMeta & \textit{Feature Expression}\\
\synDef \config \confSet &:& \fSet \to \bSet &\textit{Configuration}
\end{syntax}

\medskip
\textbf{Evaluation of feature expressions:}
\begin{alignat*}{1}
\fSem [] . &: \ffSet \to \confSet \to \bSet\\
\fSem \bTag &= \bTag\\
\fSem \fName &= \config \ \fName\\
\fSem {\neg \fName} &= \neg \fSem \fName\\
\fSem {\annd \dimMeta} &= \fSem {\dimMeta_1} \wedge \fSem {\dimMeta_2}\\
\fSem {\orr \dimMeta} &= \fSem {\dimMeta_1} \vee \fSem {\dimMeta_2}\\
\end{alignat*}

\medskip
\textbf{Relations over feature expressions:}
\begin{alignat*}{1}
\dimMeta_1 \equiv \dimMeta_2 &\textit{ iff \ } \forall \config \in \confSet. \fSem {\dimMeta_1} = \fSem {\dimMeta_2}\\
\sat {\dimMeta} &\textit{ iff \ } \exists \config \in \confSet. \fSem {\dimMeta} = \t\\
\unsat {\dimMeta} &\textit{ iff \ } \forall \config \in \confSet. \fSem {\dimMeta} = \f\\
\oneof {\A, \B, \ldots, \fName_n}
&= (\A \wedge \neg \B \wedge \ldots \wedge \neg \fName_n)
\vee (\neg \A \wedge \B \wedge \ldots \wedge \neg \fName_n)\\
&\qquad\vee (\neg \A \wedge \neg \B \wedge \ldots \wedge \fName_n)
%\dimMeta_1 \oplus \dimMeta_2 = (\dimMeta_1 \wedge \neg \dimMeta_2) \vee (\dimMeta_2 \wedge \neg \dimMeta_1)
\end{alignat*}

%\medskip
%\textbf{Syntactic sugar for mutually exclusive features:}
%\begin{alignat*}{1}
%\A \oplus \B \oplus \ldots \oplus \fName_n
%= (\A \wedge \neg \B \wedge \ldots \wedge \neg \fName_n)\\
%\vee (\neg \A \wedge \B \wedge \ldots \wedge \neg \fName_n)\\
%\vee (\neg \A \wedge \neg \B \wedge \ldots \wedge \fName_n)
%\end{alignat*}


\begin{comment}
\medskip
\textbf{Configuration constraint:}
\begin{equation*}
\forall \overrightarrow{f_i} \in c :
%( \neg o_1 \wedge o_2 \wedge o_3 \wedge \ldots \wedge o_{|f_i|})
\bigvee_{1\leq j \leq |f_i|}(\neg o_j \wedge \bigwedge_{\substack{k\not = j\\1 \leq k\leq |f_i|}} o_k)
= \prog{true}
\end{equation*}
\end{comment}

\caption{Feature expression syntax, relations, and evaluation.
% and functions over feature expressions. 
%We informally define \emph{exclusive or} \ensuremath{\oplus} of \ensuremath{n} features to be \t\ 
%only when one feature is \t.
}
\label{fig:fexp-def}
\end{figure}


\begin{figure}
\textbf{Evaluation of feature expressions:}
\begin{alignat*}{1}
\fSem [] . &: \ffSet \to \confSet \to \bSet\\
\fSem \bTag &= \bTag\\
\fSem \fName &= \config \ \fName\\
\fSem {\neg \fName} &= \neg \fSem \fName\\
\fSem {\annd \dimMeta} &= \fSem {\dimMeta_1} \wedge \fSem {\dimMeta_2}\\
\fSem {\orr \dimMeta} &= \fSem {\dimMeta_1} \vee \fSem {\dimMeta_2}\\
\end{alignat*}

\begin{comment}

\medskip
\textbf{Functions over feature expressions:}
\begin{alignat*}{1}
\mathit{sat}, \mathit{taut} &: \ffSet \to \bSet \\
\sat \dimMeta = \t &\textit{ iff \ } \exists \config \in \confSet: \fSem \dimMeta = \t\\
\taut \dimMeta = \t &\textit{ iff \ } \forall \config \in \confSet: \fSem \dimMeta = \t
\end{alignat*}
\end{comment}

\caption{Feature expression evaluation and functions.}
\label{fig:fexp-eval}
\end{figure}





\subsection{Variational Set}
\label{sec:vlist-vset}

%\point{vset.}
A \emph{variational set (v-set)} $\vset = \setDef {\annot [\dimMeta_1] {\elem_1},\ldots, \annot [\dimMeta_n] {\elem_n}}$ 
is a set of annotated elements~\cite{EWC13fosd,Walk14onward,ATW17dbpl}.
% where the presence condition of elements is satisfiable~\cite{EWC13fosd,Walk14onward,vdb17ATW}. 
%
Conceptually, a \emph{variational set} represents many different plain sets
that can be generated by enabling or disabling features
and including only the elements whose feature expressions evaluate to \t.
We typically omit the presence condition \prog{true} in a variational set,
e.g., the v-set 
$\setDef {\annot [\A] 2, \annot [\B] 3, 4}$
represents four plain sets under different configurations. These plain
sets can be generated by \emph{configuring} the variational set with a
given configuration: 
\setDef {2,3,4}, when $\A$ and $\B$
are enabled; \setDef {2,4}, when $\A$ is enabled but $\B$ is disabled;
\setDef {3,4}, when $\B$ is enabled but $\A$ is disabled;
and \setDef {4}, when both $\A$ and $\B$ are disabled.
%
%We indicate variational sets of elements $\elem \in \mathbf{\elemSet}$ with \elemSet.
%A variational set is conceptually a function from a configuration of its
%features to the corresponding plain set. 
%We typically omit the feature
%expression \prog{true} in a variational set, for example, in the
%variational set $\{5,6^{f_1}\}$, the feature expression for the value $5$ is
%implicitly \prog{true}, and so the element is included in both variants:
%$\{5,6\}$ when feature $f_1$ is enabled and $\{5\}$ when feature $f_1$ is
%disabled.
Note that elements with presence condition \prog{false} can be omitted
from the v-set, e.g., the v-set \ensuremath{\setDef {\annot [\f] {1}}} is 
equivalent to an empty v-set.
For simplicity and to comply with database notational conventions
we drop the brackets of a variational set when used in database
schema definitions and queries.
%for defining 
%variational relation schemas and the variational attribute set to be projected in a query.

%\point{annotated vset.}
A variational set itself can also be annotated with a feature expression.
%
%An \emph{annotated variational set} 
$\annot \vset = \setDef {\annot [\dimMeta_1] {\elem_1},\ldots,\annot [\dimMeta_n] {\elem_n}}^\dimMeta$ is an
\emph{annotated v-set}.
% that it is annotated itself by a \emph{feature expression} \dimMeta.
%We denote an annotated variational set of elements $\elem \in \mathbf{\elemSet}$ with
%\annot \elemSet.
Annotating a v-set with the feature expression \dimMeta\ 
restricts the condition under which its elements are present, i.e., it forces
elements' presence conditions to be more specific. This restriction 
can be applied to all elements of the set by \emph{pushing} in the
feature expression \dimMeta, done by the operation
%\NOTE{
\ensuremath{
\pushIn {\setDef {\annot [\dimMeta_1] {\elem_1},\ldots,\annot [\dimMeta_n] {\elem_n}}^\dimMeta}
= 
%\annot {\setDef{\annot [\dimMeta_i] \elem_i \myOR \sat {\dimMeta_i \wedge \dimMeta}, 1 \leq i \leq n}}}.}
\setDef {\annot [\dimMeta_1 \wedge \dimMeta] {\elem_1},\ldots, \annot [\dimMeta_n \wedge \dimMeta] {\elem_n}}
}.
%This restriction
%can be captured by the property:
%$\setDef {\annot [\dimMeta_1] {\elem_1} ,\ldots, \annot [\dimMeta_n] {\elem_n}}^\dimMeta
%\equiv 
%\setDef {\annot [\dimMeta_1 \wedge \dimMeta] {\elem_1},\ldots, \annot [\dimMeta_n \wedge \dimMeta] {\elem_n}}
%$.
%
For example, the annotated v-set
$\{\annot [\A] 2, \annot [\neg \B] 3, 4, \annot [\C] 5\}^{\A \wedge \B}$
indicates that all the elements of the set can only exist
when both $\A$ and $\B$ are enabled. Thus, pushing in the set's feature expression
results in
$\{\annot [\A \wedge \B] 2,\annot [\A \wedge \B] 4,\annot [\A \wedge \B \wedge \C] 5\}$. The element $3$ is dropped 
%from the set 
since 
\ensuremath{\neg \sat {\neg \B \wedge (\A \wedge \B)}},
where
\ensuremath{
\getPC 3 = \neg \B \wedge (\A \wedge \B)}.
%its presence condition is unsatisfiable, i.e., $\neg \sat {\neg \fName_2 \wedge (\fName_1 \wedge \fName_2)}$.
%%

We provide some operations over v-sets. Intuitively, these operations should 
behave such that configuring the result of applying a variational set operation
should be equivalent to applying the plain set operation on the configured 
input v-sets. 
 
%These operations are vastly used
%in \secref{type-sys}.

%
\begin{definition}[V-set union]
\label{def:vset-union}
The \emph {union} of two v-sets is the union of their elements with the disjunction of 
presence conditions if an element exists in both v-sets:
\ensuremath{
\vset_1 \cup \vset_2 = \setDef {\annot [\dimMeta_1] \elem \myOR \annot [\dimMeta_1] \elem \in \vset_1, \annot [\dimMeta_2] \elem \not \in \vset_2}
\cup \setDef {\annot [\dimMeta_2] \elem \myOR \annot [\dimMeta_2] \elem \in \vset_2, \annot [\dimMeta_1] \elem  \not \in \vset_1}
\cup \setDef {\annot [\dimMeta_1 \vee \dimMeta_2] \elem \myOR 
\annot [\dimMeta_1] \elem \in \vset_1, \annot [\dimMeta_2] \elem \in \vset_2}
}.
For example, \\
\ensuremath{
\setDef {2,\annot [\dimMeta_1] 3, \annot [\dimMeta_1] 4} \cup \setDef {\annot [\dimMeta_2] 3, \annot [\neg \dimMeta_1] 4} = \setDef {2, \annot [\dimMeta_1 \vee \dimMeta_2] 3, 4}
}.
\end{definition}

% 
% is needed for the implicitly-type lang:
\begin{definition}[V-set intersection]
\label{def:vset-intersect}
The \emph{intersection} of two v-sets is a v-set of their shared elements
annotated with the conjunction of their presence conditions, i.e., 
\ensuremath{
\vset_1 \cap \vset_2 = \setDef {
\annot [\dimMeta_1 \wedge \dimMeta_2 ]\elem \myOR
\annot [\dimMeta_1] \elem \in \vset_1, \annot [\dimMeta_2] \elem \in \vset_2,
\sat {\dimMeta_1 \wedge \dimMeta_2}
}
}.
For example, \ensuremath{
\setDef {2, \annot [\A] 3, \annot [\neg \B] 4} \cap
\pushIn {\annot [\B] {\setDef{2,3,4,5}}} =
\setDef{\annot [\B] 2, \annot [\A \wedge \B] 3}
}.
\end{definition}

\begin{definition} [V-set cross product]
\label{def:vset-cross}
The \emph{cross product} of two v-sets is a pair of every two elements of 
them annotated with the conjunction of their presence conditions.
\ensuremath{
\vset_1 \times \vset_2 = \setDef{
\annot [\dimMeta_1 \wedge \dimMeta_2] {(\elem_1, \elem_2)} \myOR
\annot [\dimMeta_1] \elem_1 \in \vset_1, \annot [\dimMeta_2] \elem_2 \in \vset_2
%\vset_1 \cap \vset_2 = \setDef \
}
}
%
\end{definition}

\begin{definition} [V-set equivalence]
\label{def:vset-eq}
Two v-sets are \emph{equivalent}, denoted by
\ensuremath{\vset_1 \equiv \vset_2}, iff
\ensuremath{
\forall \annot  \elem \in (\vset_1 \cup \vset_2). 
\annot [\dimMeta_1] \elem \in \vset_1, \annot [\dimMeta_2] \elem \in \vset_2, 
\dimMeta_1 \equiv \dimMeta_2},
i.e., they both cover the same set of elements and the presence conditions
of elements from the two v-sets are equivalent.
\end{definition}

%
\begin{definition} [V-set subsumption]
\label{def:vset-subsumption}
The v-set \ensuremath{\vset_1} \emph {subsumes} the v-set
\ensuremath{\vset_2}, $\subsume {\vset_2} {\vset_1}$, iff
\ensuremath{ \forall \annot [\dimMeta_2] \elem \in \vset_2.
\annot [\dimMeta_1] \elem \in \vset_1, 
%\neg \sat {\dimMeta_2 \wedge \neg \dimMeta_1}
\sat {\dimMeta_2 \wedge  \dimMeta_1}
},
i.e., all elements in $\vset_2$ also exist in $\vset_1$ 
s.t. the element is valid in a shared configuration between the v-sets.
For example, 
\ensuremath{
 \subsume {\pushIn {\annot [\A] {\setDef {2,3}}}} {\setDef {2, \annot [\A \vee \B] 3, 4}}},
however, 
\ensuremath{
 \nsubsume {\pushIn {\annot [\A] {\setDef {2,3}}}} {\setDef {2, \annot [\neg \A \wedge \B] 3}}}
and
\ensuremath{
\nsubsume {\setDef {\annot [\A] 2,\annot [\A] 3, 4}} {\setDef {2, \annot [\A \wedge \B] 3}}}.
\end{definition}

%\begin{definition} [V-set explicit subsumption]
%\dropit{drop this for vldb submission. remember you need it for popl.}
%\label{def:vset-strict-subsumption}
%The v-set \ensuremath{\vset_1} \emph {explicitly subsumes} the v-set
%\ensuremath{\vset_2}, $\subsumeExpl {\vset_2} {\vset_1}$, iff
%\ensuremath{ \forall \annot [\dimMeta_2] \elem \in \vset_2.
%\annot [\dimMeta_1] \elem \in \vset_1, 
%\neg \sat {\dimMeta_2 \wedge \neg \dimMeta_1}
%},
%i.e., all elements in $\vset_2$ also exist in $\vset_1$ 
%s.t. its presence condition in \ensuremath{\vset_2} is more specific than 
%its presence condition in \ensuremath{\vset_1}, captured by 
%\ensuremath{\neg \sat {\dimMeta_2 \wedge \neg \dimMeta_1}}
%which could also be defined as 
%\ensuremath{
%\nexists \config \in \confSet . \fSem {\dimMeta_1} = \t , \fSem {\dimMeta_2} = \f.
%%i.e. in set theory:
%% \dimMeta_2 \subset \dimMeta_1
%%\dimMeta_2 - \dimMeta_1 = \emptyset
%%i.e.
%%\dimMeta_2 \cap \bar{\dimMeta_1} = \emptyset 
%}
%\end{definition}





\section{The Formula Choice Calculus}
\label{sec:fcc}


%To account for variation, VRA combines relational algebra (RA) with 
%\emph{choices}~\cite{EW11tosem,HW16fosd,Walk13thesis}.
%%\point{choice.}
%A choice $\chc{\elem_1,\elem_2}$ consists of a feature expression \dimMeta, called
%the \emph{dimension} of the choice, and 
%two \emph{alternatives} $\elem_1$ and $\elem_2$. For a given configuration \config, 
%the choice $\chc{\elem_1, \elem_2}$ can be replaced by $\elem_1$ if \dimMeta\
%evaluates to \t\ under configuration \config, (i.e., \fSem{\dimMeta}),
%or $\elem_2$ otherwise. 

The second approach we use to incorporate variation into queries is
the formula choice calculus~\cite{HW16fosd} which is an extension of 
the choice calculus~\cite{Walk13thesis,EW11tosem}. 
%
The choice calculus is a metalanguage for
describing variation in programs and its elements such as data 
structures~\cite{Walk14onward,EWC13fosd}.
In the choice calculus, variation is represented in-place as
choices between alternative subexpressions. For example, 
the variational expression 
$\mathit{expr} = \chc [\A] {1,2} + \chc [\B] {3,4} + \chc [\A] {5,6}$
 contains three choices.
Each choice has an associated \emph{dimension}, which is a boolean
variable equivalent to a feature and is used to
synchronize the choice with other choices in different parts
of the expression. For example, expression $\mathit{expr}$ contains
two dimensions, $\A$ and $\B$, and the two choices in dimension
$\A$ are synchronized. Therefore, the variational expression
$\mathit{expr}$ represents four different plain expressions, depending
on whether the left or right alternatives are selected from each
dimension. Assuming that dimensions may be set to boolean values
where \t\ indicates the left alternative and \f\ indicates the
right alternative, we have: 
%(1) $1+3+5$, when $A$ and $B$ are \t,
%(2) $1+4+5$, when $A$ is \t\ and $B$ is \f,
%(3) $2+3+6$, when $A$ is \f\ and $B$ is \t,
%and (4) $2+4+6$, when $A$ and $B$ are \f.
\begin{alignat*}{1}
\chc [\A] {1,2} + \chc [\B] {3,4} + \chc [\A] {5,6} &=
\begin{cases}
  1+3+5,& \A =\t, \B = \t\\
  1+4+5,& \A =\t, \B = \f\\
  2+3+6,& \A =\f, \B = \t\\
  2+4+6,& \A =\f, \B = \f
\end{cases}
\end{alignat*}
%
\noindent
The formula
choice calculus extends the choice calculus 
by allowing dimensions to be propositional formulas~\cite{HW16fosd}. For example,
the variational expression $\VVal {\mathit{expr}} = \chc [\A \vee \B] {1,2} + \chc [\B] {3,4} + \chc [\A] {5,6}$ represents
four plain expressions: 
%(1) $1$, when $\A \vee \B$ evaluates to \t\
%and (2) $2$, when $\A \vee \B$ evaluates to \f. More explicitly, we have:
\begin{alignat*}{1}
\chc [\A \vee \B] {1,2} + \chc [\B] {3,4} + \chc [\A] {5,6}&=
\begin{cases}
  1+3+5,& \A =\t, \B = \t\\
  1+4+5,& \A =\t, \B = \f\\
  1+3+6,& \A =\f, \B = \t\\
  2+4+6,& \A =\f, \B = \f
\end{cases}
\end{alignat*}



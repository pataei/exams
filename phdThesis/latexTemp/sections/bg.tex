\chapter{Background}
\label{ch:bg}


The core of this thesis is injecting a new aspect to relational databases: \emph{variation}.
Thus, the goals of this chapter are twofold: 
%
first, to introduce how variation is encoded and represented in our variational database framework;
%
second, to provide the reader with the concepts and notations
used to build up the main contributions of this thesis, mainly relational databases and
relational algebra
%
and approaches used to add variation to them.
% of converting non-variational components into
%variational counterparts by using the introduced encoding of variation. 
%third, to familiarize the reader with
%main notations and design decision of the thesis. 

%
Throughout the thesis, we use types when defining concepts. 
A type is a set of possible values. For example, the type $\mathbf{Int}$
denotes all possible integers. In our formalization, we use the notation of $i \in \mathbf{Int}$ to
state that the variable $i$ is of type $\mathbf{Int}$. 
%
Types can be more general. Consider the type \settype \typevar\ that indicates the set all sets
 of values of type \typevar. Here, \typevar\ is type variable and stands for any possible type. 
Note that concrete types start with a capital letters but type variables do not.
For example, the type $\settype {\mathit{Int}}$ is the type of
sets of integers and it has values such as $\setDef {1,3,4}$ and $\setDef {\ }$.
We also use type synonyms for simplicity. For example, instead of referring to
$\settype {\mathit{Int}}$ we can give it a new name ($\mathit{Setint}= \settype {\mathit{Int}}$) 
and refer to it with the new name $\mathit{Setint}$.
\wrrite{define pair and function types}
%However, for brevity, we usually use 
%bold capital letters for types, for example, instead of $\mathit{Int}$ we write $\mathcal{I}$ to denote the type $\mathit{Int}$.
%
Sometimes we must extend a type with an additional ``bottom'' element $\bot$ and to account for this
extension at the type level we subscript the type with $\bot$. For example, $\maybe {\mathcal{I}}$
denotes the extension of the type $\mathcal{I}$ with $\bot$.
%

%
Throughout the thesis, we discuss relational concepts and their
variational counterparts. 
For clarity or when it is unclear from context, we use
an $\underline{underline}$ to distinguish a non-variational entity
from its variational counterpart, 
both at the value level and the type level,
%when we need to emphasize 
%an entity is not variational we underline it, 
e.g., \pElem\ is a 
non-variational entity while \elem\ is its variational counterpart,
if it exists.


%\secref{types} describes types. Types provide readers with a strong tool to understand 
%some formalizations easier and more intuitively.
%
\secref{rdb} describes the database model of relational databases and 
\eric{that is the schema}
the specification of
the structured used to store the data~\cite{AliceBook}. 
\secref{ra} describes the relational algebra, a query language used to query relational databases~\cite{AliceBook}.
%
\secref{encode-var} defines our encoding of the variation space used in 
variational database and how we describe parts of that space using propositional formulas of boolean variables~\cite{ATW18poly,ATW17dbpl}.
%
Finally, we introduce the main techniques used to incorporate variation into our variational 
database framework.
\secref{vset} introduces variation into sets which forms the basis of the variational database
framework~\cite{EWC13fosd,Walk14onward,ATW17dbpl} 
and \secref{fcc} describes the formula choice calculus used to incorporate 
variation into relational algebra~\cite{HW16fosd}.



%\section{Types}
\label{sec:types}

\TODO{types}


\begin{table}
\caption[shortcaption]{An example of a relational database corresponding to \vTwo\ of our motivating example
given in \tabref{mot-basic}.}
\label{tab:rdb}
\centering
\small
%\footnotesize
%\scriptsize
\begin{subtable}[t]{\textwidth}
\centering
\caption{The schema of a relational database.}
\label{tab:rdb-sch}
\begin{tabular} {| l | }
\hline
\empacct\ (\empno, \name, \hiredate, \titleatt, \deptname)\\
\job\ (\titleatt, \salary)\\
\hline
\end{tabular}
\end{subtable}

\medskip
\medskip
\medskip
\begin{subtable}[t]{\textwidth}
%\begin{center}
\centering
\caption{The \empacct\ table.}
\label{tab:rdb-empacct}
\begin{tabular} {c | l l l l l}
%\hline
%\hhline{-==}
\empacct & \empno & \name & \hiredate & \titleatt & \deptname\\
\cline{2-6}
& 10001 & Georgi Facello & 1986-06-26 & Senior Engineer & Development\\
& 10002 & Bezalel Simmel & 1985-11-21 & Staff & Sales\\
& \ldots & \ldots & \ldots & \ldots & \ldots \\
& 499998 & Patricia Breugel & 1993-10-13 & Senior Staff & Finance\\
& 499999 & Sachin Tsukuda & 1997-11-30 & Engineer & Production
\end{tabular}
%\end{center}
\end{subtable}

\medskip
\medskip
\medskip
\begin{subtable}[t]{\textwidth}
\centering
\caption{The \job\ table.}
\label{tab:rdb-job}
\begin{tabular} {c | l l }
%\hline
%\hhline{-==}
\job & \titleatt & \salary\\
\cline{2-3}
& Assistant Engineer & 96646\\
& Assistant Engineer & 61594\\
& \ldots & \ldots \\
& Technique Leader & 58345\\
& Technique Leader & 86641
\end{tabular}
\end{subtable}

\end{table}


\section{The SPC Relational Algebra}
\label{sec:ra}

%\maybeAdd{add type system}
%\maybeAdd{maybe add semantics later on}
Having introduced relational databases, we now shift gears into querying
these databases, that is, extracting information from tables.
For almost all relational query languages, the result of a query 
is a table called $\mathit{result}$. 
%
We base our variational query language on the SPC relational algebra.
Three primitive operators form the SPC algebra: \emph{selection}, \emph{projection},
and \emph{cross-product} (or Cartesian product)~\cite{AliceBook}.
We introduce these operators through \exref{ra} by stating an intent and then
building up a query to extract the information required by the intent. 

\begin{example}
\label{eg:ra}
Consider the database instance given in \tabref{rdb}. We want to get a list
of employees (by their names) whose salary is more than 65000 dollars. 
As the first step, we use the selection operator to get the tuples for all jobs with salaries that 
are more than 65000 dollars.\\
%
%\begin{equation*}
\centerline{
\ensuremath{
{\pQ_1} = \sigma_{\salary \ge 65000} (\job)
%\end{equation*}
}}
%
\noindent
A sample of the results returned by the query ${\pQ_1}$ is given in \tabref{ra1}.
Next a set of 
 tuples is created by taking the cross-product of ${\pQ_1}$
and \empacct.\\
%
%\begin{equation*}
\centerline{
\ensuremath{
{\pQ_2} = {\pQ_1} \times \empacct
%\end{equation*}
}}
%
\noindent
A sample of the results returned by the query ${\pQ_2}$ is given in \tabref{ra2}.
However, looking closely at these results, there is no connection between an employee
in the \empacct\ relation and their salary in the \job\ relation. Thus, we have to perform 
another selection to connect each employee with their title. \\
%
%\begin{equation*}
\centerline{
\ensuremath{
{\pQ_3} = \sigma_{\empacct.\titleatt=\job.\titleatt} ({\pQ_2})
%\end{equation*}
}}
%
\noindent
A sample of the results returned by the query ${\pQ_3}$ is given in \tabref{ra3}.
At this point, we are only interested in two attributes, that is, \name\ and \salary.
Thus, we use projection to discard the unneeded columns.\\
%
%\begin{equation*}
\centerline{
\ensuremath{
{\pQ_4} = \pi_{\name, \salary} ({\pQ_3} )
%\end{equation*}
}}
%
A sample of the results returned by the query ${\pQ_4}$ is given in \tabref{ra4}.
\end{example}

\begin{table}[!htbp]
\caption[Results of subqueries to build up the query in \exref{ra}]{Results of each step of building the final query in \exref{ra}.}
\label{tab:ra-ex}
\centering
\small
%\footnotesize
%\scriptsize
\begin{subtable}[t]{\textwidth}
\centering
\caption{Result of the query \ensuremath{\pQ_1 = \sigma_{\salary \ge 65000} (\job)}.}
\label{tab:ra1}
\begin{tabular} {c | l l }
\multirow{2}{*}{$\mathit{result}$} & \titleatt & \salary \\
\cline{2-3}
&Senior Staff & 77935 \\
& Senior Engineer & 96646\\
& \ldots & \ldots \\
& Staff & 77935\\
& Engineer & 96646
\end{tabular}
\end{subtable}

\medskip
\medskip
\medskip
\begin{subtable}[t]{\textwidth}
%\begin{center}
\centering
\tiny
\caption{Result of the query \ensuremath{\pQ_2 = (\sigma_{\salary \ge 65000} (\job)) \times \empacct}.}
\label{tab:ra2}
\begin{tabular} {c | l l l l l l l}
%\hline
%\hhline{-==}
\multirow{2}{*}{$\mathit{result}$}  & \titleatt & \salary & \empno & \name & \hiredate & \titleatt & \deptname\\
\cline{2-8}
&Senior Engineer & 96646 & 13094 & Sanjay Servieres & 1986-01-01 & Engineer & Research \\
&Staff & 77935 & 16099 & Mohan Ferretti & 1987-09-20 & Senior Staff & Human Resources\\
&\ldots & \ldots & \ldots & \ldots & \ldots & \ldots & \ldots\\
&Engineer & 80324 & 19162 & Chinho Fadgyas & 1986-05-19 & Technique Leader & Production \\
&Senior Staff & 88070 & 22255 & Kristian Merel & 1986-09-12 & Senior Engineer & Development
\end{tabular}
%\end{center}
\end{subtable}

\medskip
\medskip
\medskip
\begin{subtable}[t]{\textwidth}
%\begin{center}
\centering
%\footnotesize
\tiny
\caption{Result of the query \ensuremath{\pQ_3 = \sigma_{\empacct.\titleatt=\job.\titleatt}\left(\left(\sigma_{\salary \ge 65000} \left(\job\right)\right) \times \empacct\right)}.}
\label{tab:ra3}
\begin{tabular} {c | l l l l l l l}
%\hline
%\hhline{-==}
\multirow{2}{*}{$\mathit{result}$}  & \titleatt & \salary & \empno & \name & \hiredate & \titleatt & \deptname\\
\cline{2-8}
&Engineer & 96646 & 13094 & Sanjay Servieres & 1986-01-01 & Engineer & Research \\
&Senior Staff & 77935 & 16099 & Mohan Ferretti & 1987-09-20 & Senior Staff & Human Resources\\
&\ldots & \ldots & \ldots & \ldots & \ldots & \ldots & \ldots\\
&Senior Engineer & 96646 & 22255 & Kristian Merel & 1986-09-12 & Senior Engineer & Development\\
&Staff & 77935 & 43670 & JoAnna Randi & 1987-10-18 & Staff & Marketing
\end{tabular}
%\end{center}
\end{subtable}

\medskip
\medskip
\medskip
\begin{subtable}[t]{\textwidth}
\centering
\caption{Result of the query \ensuremath{\pQ_4 = \pi_{\name, \salary} \left(\sigma_{\empacct.\titleatt=\job.\titleatt}\left(\left(\sigma_{\salary \ge 65000} \left(\job\right)\right) \times \empacct\right)\right)}.}
\label{tab:ra4}
\begin{tabular} {c | l l }
%\hline
%\hhline{-==}
\multirow{2}{*}{$\mathit{result}$}  &\name & \salary\\
\cline{2-3}
& Sanjay Servieres & 96646\\
&Mohan Ferretti & 77935\\
&  \ldots & \ldots \\
& Kristian Mere & 96646\\
& JoAnna Randi & 77935
\end{tabular}
\end{subtable}

\end{table}



The relational algebra that we use also includes standard set operations, a join 
operation, and an empty relation. The syntax is defined in \figref{rel-alg}.
%
The set operations, union and intersection, require two subqueries to have the same relation schema
and simply applies the corresponding operation, either union or intersection, to the sets of tuples returned by
the subqueries.
%For example, if we have $\pRel_1(\pAtt_1, \pAtt_2)$ with 
%tuples $\{(1,2)(3,4)\}$
%and $\pRel_2(\pAtt_1, \pAtt_2)$ with tuples $\{(1,2),(5,6) \}$
%then $\pRel_1 \cup \pRel_2 $ returns the tuples $\{(1,2), (3,4), (5,6)\}$.
%
The \emph{join} operation is equivalent to selection applied to a cross-product, that is,
$\pQ_1 \bowtie_{\pCond} \pQ_2 = \vSel [\pCond] {(\pQ_1 \times \pQ_2)}$.
For example, ${\pQ_3}$ in \exref{ra} can be rewritten as\\
\centerline{
\ensuremath{
\VVal {{\pQ_3}} = \left(\sigma_{\salary \ge 65000} \left(\job\right)\right) \bowtie_{\empacct.\titleatt=\job.\titleatt} \empacct
}}.
%joins two subqueries based on a condition and
\noindent
Throughout our examples, omitting the condition of join  implies it is a \emph{natural join},
that is, join on the shared attribute of the two subqueries.
For example, $\VVal {{\pQ_3}}$ can be rewritten using the natural join\\
\centerline{
\ensuremath{
\VVVal {{\pQ_3}} =  \left(\sigma_{\salary \ge 65000} \left(\job\right)\right) \bowtie \empacct
}}.

\begin{figure}

\begin{syntax}

% feature expressions
%\synDef{\dimMeta}{\ffSet}
%  &\eqq& \multicolumn{2}{l}{%
%         \t \myOR \f \myOR \fName \myOR \neg\fName
%         \myOR \dimMeta\wedge\dimMeta \myOR \dimMeta\vee\dimMeta}
%\\[1.5ex]

% relation conditions
\synDef{\pCond}{\pCondSet}
  &\eqq& \multicolumn{2}{l}{%
         \t \myOR \f \myOR \att\bullet\cte \myOR \att\bullet\att
         \myOR \neg\pCond \myOR \pCond\vee\pCond} \\
%     &|& \multicolumn{2}{l}{\vCond\wedge\vCond \myOR \chc{\vCond,\vCond}}
\\[1.5ex]

% variational relational algebra
\synDef{\pQ}{\pQSet}
  &\eqq& \pRel                 & \textit{Relation reference} \\
     &|& \pRen[\pRel]{\pQ}     & \textit{Renaming} \\
     &|& \pPrj[\pAttList]{\pQ} & \textit{Projection} \\
     &|& \pSel\pQ              & \textit{Selection} \\
     &|& \pQ \Join_{\pCond} \pQ  & \textit{Join} \\
%     &|& \chc{\vQ,\vQ}         & \textit{Choice} \\
%     &|& \empRel               & \textit{Empty relation} \\
%    &|& \vQ \times \vQ        & \textit{Cartesian Product} \\
%    &|& \vQ \circ \vQ         & \textit{Set operation} \\
\end{syntax}

\caption{Syntax of  relational algebra, where $\bullet$ ranges over
comparison operators ($<, \leq, =, \neq, >, \geq$), \cte\ over constant values,
\att\ over attribute names, and \pAttList\ over lists of attributes.
The syntactic category
% \dimMeta\ represents feature expressions, 
 \pCond\
is relational conditions, and \pQ\ is  relational algebra terms.
}
%\vspace{-20pt}
\label{fig:v-alg-def}
\end{figure}
%\vspace{-20pt}


We also extend relational algebra such that projection of an empty set of
attributes is a valid query that returns an empty set of tuples. We define the
\emph{empty} query \empPRel\ as shorthand for projecting an empty set of
attributes, that is, $\empPRel = \pi_{\{\}} \pQ$.
%
Note that we do not extend the notation of using underline for relational algebra
operators. Instead, relational algebra operators are overloaded and are used
as both plain relational and variational operators. It should be clear from
context when an operation is variational or not. 


Although we do not consider renaming of queries in the formal definition of 
relational algebra, we do support this in our implementation. Furthermore, we use it
to rename subqueries of our examples to make them easier to understand. 
For example, query $\VVVal {{\pQ_3}}$ can be written as:
\begin{align*}
\VVVal {{\pQ_3}} &= \underline{temp} \bowtie \empacct\\
\underline{temp} &\leftarrow  \sigma_{\salary \ge 65000} \left(\job\right)
\end{align*}
\noindent
Making this renaming explicit is necessary to avoid names conflicting in some cases.





\section{Variation Space in a Variational Database Framework}
\label{sec:varspace}

\TODO{have to revise}
\TODO{define oplus in fig as syntactic sugar.}

%\point{using a feature set to represent variability within a context.}
To account for variability in a database we need to 
encode it.
%
%The first challenge of incorporating variability into a database
%is to represent variability. 
%
To encode variability we introduce a \emph{feature space} \fSet\ as 
a closed set of boolean variables. 
A feature \ensuremath{\fName \in \fSet} can be enabled (i.e., \fName = \t) or disabled (\fName = \f).
Features capture the variation in a given variational scenario.
%first organize the configuration space into
%a set of features \fSet.
%, denoted by \fSet.
%
%we require a \emph{set of features}, denoted by \fSet, 
%appropriate for the context that the database is used for.
%
For example, in the context of schema evolution, features can be generated from version 
numbers (e.g. features \vOne\ to \vFive\ and \tOne\ to \tFive\ in the 
motivating example, \tabref{mot}); for SPLs, 
the features can be adopted from the SPL feature set (e.g. the \edu\ feature in
our motivating example, \tabref{mot}); and 
for data integration, the features can be representatives of resources.  
%For simplicity, the set of features is assumed to be closed and features are
%assumed to be boolean variables, however, it is easy to extend them
%to multi-valued variables that have finite set of values.
% and without loss of generality, 
%features are assumed to be boolean variables, although, it is easy to extend them
%to multi-valued variables. 
%A feature \ensuremath{\fName \in \fSet} can be enabled (i.e., \fName = \t) or disabled (\fName = \f).
%\point{configuration.}
%Assuming that all the features by default are set to \prog{false},
%enabling some of them specifies a variant. 

Features describe local points of variation in a database.
%, i.e,
%in \fSet\ 
%are used to 
%they indicate which parts of the database are present conditionally.
%of a variational entity within the database 
%are different 
%among different variants. 
Thus, enabling or disabling all
the features 
%in \fSet\ 
produces a particular database \emph{variant} where
%of the entity in which 
all variation has been removed. 
%Enabling or disabling all features of \fSet\ specifies a non-variational \emph{variant}
%that can potentially be generated by \emph{configur}ing its variational counterpart 
%with the variant's \emph{configuration}.
%Hence, to specify a variant
%we define a function, called 
A \emph{configuration} is a \emph{total} function
that maps every feature in the feature set to a boolean value.
%By definition, a configuration is a \emph{total} function,
%i.e., it is defined for \emph{all} features in the feature set. 
%For brevity, 
We represent a configuration  by the set of enabled features.
%which represents a variant. 
%\ensure{make sure referring to a variant can be done by the set of enabled features. HERE!}
For example, in our motivating scenario, the configuration \ensuremath{
\setDef {\vTwo,\tThree,\edu}
}
represents a database variant where only features \vTwo, \tThree, and \edu\ are enabled.
This database variant contains relation schemas in the yellow cells of \tabref{mot}.
%of the
%employee and education sub-schemas associated with \vTwo\ and
%\tThree\ in \tabref{mot}, respectively.
%For brevity, 
We refer to a variant with configuration \config\ as variant \config, e.g.,
%For example, 
variant \setDef {\vTwo,\tThree,\edu} refers to the variant
with configuration \setDef {\vTwo,\tThree,\edu}.
%and education sub-schema associated with \tThree\ in \tabref{mot}.

%\point{represent variability in db by prop formula of features, called feature expression.}
%Having defined a set of features, we need to incorporate them into the database.
%To encode features in the database, we construct propositional formulas of features
%such that the formula evaluates to \t\ for a set of configurations. 
When describing variation points in the database, we need to 
refer to subsets of the configuration space. We achieve this by 
constructing propositional formulas of features.
Thus, 
such a propositional formula defines a condition that holds for 
a subset of configurations and their corresponding variants. 
%
%describing the condition where one or more variants are present,
%i.e., assigning features to their values defined in variant's configuration and 
%evaluating the propositional formula results in \prog{true}.
%
For example,
the propositional formula $\neg \edu$ represents all variants of our
motivating example that do not 
have the education part of the schema, i.e., variant schemas of the 
left schema column. 

We call a propositional formula of features a \emph{feature expression} and define
it formally in \figref{fexp-def}. 
%\figref{fexp-def} defines the syntax of feature expressions.
The evaluation function of feature expressions 
$\fSem \dimMeta : \ffSet \to \confSet \to \bSet$ simply substitutes each
feature \fName\ in the expression \dimMeta\ with its boolean value 
set in the given configuration \config\ and then
simplifies the propositional formula to a boolean value.
%  evaluates the 
%feature expression \dimMeta\ w.r.t. the configuration \config.
%and defined in
%our technical report~\cite{vldbArXiv}, 
%and defined in \appref{fexp},
%evaluates feature expression \dimMeta\ under configuration \config,
%also called \emph{configuration of feature expression \dimMeta\ under \config}. 
For example, assuming that 
\ensuremath{\fSet = \setDef{\A, \B}}
 $\fSem [\{\A\}] {\A \vee \B} = \t \vee \f = \t$, however,
%which states that the feature expression
%$\A \vee \B$ evaluates to \t\ w.r.t. configuration that only enables $\A$, however,
$\fSem [\{\}] {\A \vee \B} = \f \vee \f = \f$.
%, where the empty set indicates neither \A\ nor \B\
%are enabled.
%which states that the same feature expression evaluates
%to \f\ when neither $\A$ nor $\B$ are enabled.
Additionally, we define the binary \emph{equivalence ($\equiv$)} relation and
the unary \emph{satisfiable (sat)} and \emph{unsatisfiable (unsat)}
relations over feature expressions in \figref{fexp-def}.
%
% some of the functions needed over feature
%expressions:
%1) \emph{equivalence} of two feature expressions, 
%\ensuremath{\dimMeta_1 \equiv \dimMeta_2}
%and
%2) \emph{satisfiability} of a feature expression, 
%\ensuremath{\sat \dimMeta}.
%However, we define the evaluation of feature expressions and functions over them
%in \appref{fexp}.
%We define two functions over feature expressions, as shown in \figref{fexp-def}:
%1) \emph{satisfiability}: feature expression \dimMeta\ is \emph{satisfiable} if there 
%exists configuration \config\ s.t. \fSem \dimMeta = \t\
%and 2) \emph{tautology}: feature expression \dimMeta\ is a \emph{tautology} if  
%for all valid configurations we have: \fSem \dimMeta = \t.

%\point{annotating elements of database with feature expressions.}
To incorporate feature expressions into the database,
we \emph{annotate/tag} database elements (including attributes, relations, and tuples) 
with feature expressions. An \emph{annotated element} \elem\ with feature expression \dimMeta\
is denoted by \annot \elem. 
%The feature expression \dimMeta\ represents
%the set of configurations where their variants contain element \elem\ because
%
The feature expression attached to an element is called a \emph{presence condition}
since it determines the condition (set of configurations) under which the element is present.
\getPC {\annot \elem} returns the presence condition of the variational element
\annot \elem.
For example, the
annotated number $\annot [\A \vee \B] 2$ is present in variants
\setDef \A, \setDef \B, 
\ensuremath{\setDef {\A, \B}}
% with a configuration
%that enables either $\A$ or $\B$ or both
but it does not exist in variant \setDef {}.
%variants that disable both $\A$ and $\B$.
Here, $\getPC {\annot [\A \vee \B] 2} = \A \vee \B$.

%\point{relationship between features is captured by a propositional formula, called feature model.} 
No matter the context, features often have a relationship with each other that
constrains configurations. For example, only one of the temporal features of $\vOne - \vFive$
can be \t\ for a given variant.
This relationship is captured by a feature expression, called a \emph{feature model} and
denoted by \fModel,
which restricts the set of \emph{valid configurations}:
if configuration \config\ violates the relationship then 
%evaluating the feature model \fModel\
%under this configuration evaluates to \f: 
it follows from the \fSem . definition that \fSem \fModel = \f.
For example, the restriction that at a given time only one of temporal features $\vOne - \vFive$
can be enabled is represented by:
%the feature expression:
\ensuremath{
\vOne \oplus \vTwo \oplus \vThree \oplus \vFour \oplus \vFive
},
where $\A \oplus \B \oplus \ldots \oplus \fName_n$ is syntactic sugar for $(\A \wedge \neg \B \wedge \ldots \wedge \neg \fName_n) \vee (\neg \A \wedge \B \wedge \ldots \wedge \neg \fName_n) \vee (\neg \A \wedge \neg \B \wedge \ldots \wedge \fName_n)$
, i.e., features are mutually exclusive.
%multiple features cannot be enabled simultaneously.
%$\left(\vOne \wedge \neg \left(\vTwo \vee \hdots \vee \vFive \right) \right)
%\vee \left(\vTwo \wedge \neg \left(\vOne \vee \vThree \vee \vFour \vee \vFive \right) \right) 
%\vee \hdots 
%\vee \left(\vFive \wedge \neg \left(\vOne \vee \hdots \vee \vFour \right) \right)$.
%Note that this is not the feature model for the entire motivating example.



\begin{figure}
%\textbf{Feature expression generic object:}
%\begin{syntax}
%\synDef \fName \fSet &\textit{Feature Name}
%%c \in \mathbf{C} &\textit{Configuration}
%\end{syntax}
%
%\medskip
\textbf{Feature expression syntax:}
\begin{syntax}
\synDef \fName \fSet &&&\textit{Feature Name}\\
\synDef \bTag \bSet &\eqq& \t \myOR \f & \textit{Boolean Value}\\
\synDef \dimMeta \ffSet &\eqq& \bTag \myOR \fName \myOR \neg \dimMeta \myOR \dimMeta \wedge \dimMeta \myOR \dimMeta \vee \dimMeta & \textit{Feature Expression}\\
\synDef \config \confSet &:& \fSet \totype \bSet &\textit{Configuration}
\end{syntax}

\medskip
\textbf{Evaluation of feature expressions:}
\begin{alignat*}{1}
\fSem [] . &: \ffSet \totype \confSet \totype \bSet\\
\fSem \bTag &= \bTag\\
\fSem \fName &= \config \ \fName\\
\fSem {\neg \fName} &= \neg \fSem \fName\\
\fSem {\annd \dimMeta} &= \fSem {\dimMeta_1} \wedge \fSem {\dimMeta_2}\\
\fSem {\orr \dimMeta} &= \fSem {\dimMeta_1} \vee \fSem {\dimMeta_2}\\
\end{alignat*}

\medskip
\textbf{Relations over feature expressions:}
\begin{alignat*}{1}
\dimMeta_1 \equiv \dimMeta_2 &\textit{ iff \ } \forall \config \in \confSet. \fSem {\dimMeta_1} = \fSem {\dimMeta_2}\\
\sat {\dimMeta} &\textit{ iff \ } \exists \config \in \confSet. \fSem {\dimMeta} = \t\\
\unsat {\dimMeta} &\textit{ iff \ } \forall \config \in \confSet. \fSem {\dimMeta} = \f\\
\oneof {\A, \B, \ldots, \fName_n}
&= (\A \wedge \neg \B \wedge \ldots \wedge \neg \fName_n)
\vee (\neg \A \wedge \B \wedge \ldots \wedge \neg \fName_n)\\
&\qquad\vee (\neg \A \wedge \neg \B \wedge \ldots \wedge \fName_n)
%\dimMeta_1 \oplus \dimMeta_2 = (\dimMeta_1 \wedge \neg \dimMeta_2) \vee (\dimMeta_2 \wedge \neg \dimMeta_1)
\end{alignat*}

%\medskip
%\textbf{Syntactic sugar for mutually exclusive features:}
%\begin{alignat*}{1}
%\A \oplus \B \oplus \ldots \oplus \fName_n
%= (\A \wedge \neg \B \wedge \ldots \wedge \neg \fName_n)\\
%\vee (\neg \A \wedge \B \wedge \ldots \wedge \neg \fName_n)\\
%\vee (\neg \A \wedge \neg \B \wedge \ldots \wedge \fName_n)
%\end{alignat*}


\begin{comment}
\medskip
\textbf{Configuration constraint:}
\begin{equation*}
\forall \overrightarrow{f_i} \in c :
%( \neg o_1 \wedge o_2 \wedge o_3 \wedge \ldots \wedge o_{|f_i|})
\bigvee_{1\leq j \leq |f_i|}(\neg o_j \wedge \bigwedge_{\substack{k\not = j\\1 \leq k\leq |f_i|}} o_k)
= \prog{true}
\end{equation*}
\end{comment}

\caption[Feature expression syntax and evaluation]{Feature expression syntax, relations, and evaluation.
% and functions over feature expressions. 
%We informally define \emph{exclusive or} \ensuremath{\oplus} of \ensuremath{n} features to be \t\ 
%only when one feature is \t.
}
\label{fig:fexp-def}
\end{figure}


\begin{figure}
\textbf{Evaluation of feature expressions:}
\begin{alignat*}{1}
\fSem [] . &: \ffSet \to \confSet \to \bSet\\
\fSem \bTag &= \bTag\\
\fSem \fName &= \config \ \fName\\
\fSem {\neg \fName} &= \neg \fSem \fName\\
\fSem {\annd \dimMeta} &= \fSem {\dimMeta_1} \wedge \fSem {\dimMeta_2}\\
\fSem {\orr \dimMeta} &= \fSem {\dimMeta_1} \vee \fSem {\dimMeta_2}\\
\end{alignat*}

\begin{comment}

\medskip
\textbf{Functions over feature expressions:}
\begin{alignat*}{1}
\mathit{sat}, \mathit{taut} &: \ffSet \to \bSet \\
\sat \dimMeta = \t &\textit{ iff \ } \exists \config \in \confSet: \fSem \dimMeta = \t\\
\taut \dimMeta = \t &\textit{ iff \ } \forall \config \in \confSet: \fSem \dimMeta = \t
\end{alignat*}
\end{comment}

\caption{Feature expression evaluation and functions.}
\label{fig:fexp-eval}
\end{figure}





\section{Annotations and Variational Sets}
\label{sec:vset}


%\point{annotating elements of database with feature expressions.}
We now introduce the first approach used to incorporate variation into a database.
To incorporate feature expressions into the database,
we \emph{annotate} database elements (including attributes, relations, and tuples) 
with feature expressions. An \emph{annotated element} \elem\ with feature expression \dimMeta\
is denoted by \annot \elem, 
that is, if \elem\ has type \typevar\ (i.e., $\elem \in \typevar$)
then $\annot \elem$ has the corresponding variational type 
$\vartype \typevar$ (i.e., $\annot \elem \in \vartype \typevar$).
%The feature expression \dimMeta\ represents
%the set of configurations where their variants contain element \elem\ because
%
The feature expression attached to an element is called its \emph{presence
condition} since it determines the condition (set of configurations) under
which the element is present in the database. 
This is done by the \emph{configuration} function $\xeSem [] . : \elemSet \totype \confSet \totype \maybe {\pelemSet}$ defined in \figref{vset}.
For example, assuming
$\features=\set{\A,\B}$, the annotated number $\annot [\A \vee \B] 2$ is present
in variants \setDef{\A} (i.e., $\xeSem [\setDef{\A}] {\annot [\A \vee \B] 2}$ = 2), 
\setDef{\B} (i.e., $\xeSem [\setDef{\A}] {\annot [\A \vee \B] 2}$ = 2), 
and \setDef{\A,\B} (i.e.,$\xeSem [\setDef{\A, \B}] {\annot [\A \vee \B] 2} = 2$) 
but not in variant
\setDef{} (i.e., $\xeSem [\setDef { }] {\annot [\A \vee \B] 2} = \bot$). 
%
The operation $\getPC{\annot{\elem}}=e$ returns the presence condition of an
annotated element.
% with a configuration
%that enables either $\A$ or $\B$ or both
%variants that disable both $\A$ and $\B$.
% Here, $\getPC {\annot [\A \vee \B] 2} = \A \vee \B$.

\section{Annotations and Variational Sets}
\label{sec:vset}


%\point{annotating elements of database with feature expressions.}
We now introduce the first approach used to incorporate variation into a database.
To incorporate feature expressions into the database,
we \emph{annotate} database elements (including attributes, relations, and tuples) 
with feature expressions. An \emph{annotated element} \elem\ with feature expression \dimMeta\
is denoted by \annot \elem, 
that is, if \elem\ has type \typevar\ (i.e., $\elem \in \typevar$)
then $\annot \elem$ has the corresponding variational type 
$\vartype \typevar$ (i.e., $\annot \elem \in \vartype \typevar$).
%The feature expression \dimMeta\ represents
%the set of configurations where their variants contain element \elem\ because
%
The feature expression attached to an element is called its \emph{presence
condition} since it determines the condition (set of configurations) under
which the element is present in the database. 
This is done by the \emph{configuration} function $\xeSem [] . : \elemSet \totype \confSet \totype \maybe {\pelemSet}$ defined in \figref{vset}.
For example, assuming
$\features=\set{\A,\B}$, the annotated number $\annot [\A \vee \B] 2$ is present
in variants \setDef{\A} (i.e., $\xeSem [\setDef{\A}] {\annot [\A \vee \B] 2}$ = 2), 
\setDef{\B} (i.e., $\xeSem [\setDef{\A}] {\annot [\A \vee \B] 2}$ = 2), 
and \setDef{\A,\B} (i.e.,$\xeSem [\setDef{\A, \B}] {\annot [\A \vee \B] 2} = 2$) 
but not in variant
\setDef{} (i.e., $\xeSem [\setDef { }] {\annot [\A \vee \B] 2} = \bot$). 
%
The operation $\getPC{\annot{\elem}}=e$ returns the presence condition of an
annotated element.
% with a configuration
%that enables either $\A$ or $\B$ or both
%variants that disable both $\A$ and $\B$.
% Here, $\getPC {\annot [\A \vee \B] 2} = \A \vee \B$.

\section{Annotations and Variational Sets}
\label{sec:vset}


%\point{annotating elements of database with feature expressions.}
We now introduce the first approach used to incorporate variation into a database.
To incorporate feature expressions into the database,
we \emph{annotate} database elements (including attributes, relations, and tuples) 
with feature expressions. An \emph{annotated element} \elem\ with feature expression \dimMeta\
is denoted by \annot \elem, 
that is, if \elem\ has type \typevar\ (i.e., $\elem \in \typevar$)
then $\annot \elem$ has the corresponding variational type 
$\vartype \typevar$ (i.e., $\annot \elem \in \vartype \typevar$).
%The feature expression \dimMeta\ represents
%the set of configurations where their variants contain element \elem\ because
%
The feature expression attached to an element is called its \emph{presence
condition} since it determines the condition (set of configurations) under
which the element is present in the database. 
This is done by the \emph{configuration} function $\xeSem [] . : \elemSet \totype \confSet \totype \maybe {\pelemSet}$ defined in \figref{vset}.
For example, assuming
$\features=\set{\A,\B}$, the annotated number $\annot [\A \vee \B] 2$ is present
in variants \setDef{\A} (i.e., $\xeSem [\setDef{\A}] {\annot [\A \vee \B] 2}$ = 2), 
\setDef{\B} (i.e., $\xeSem [\setDef{\A}] {\annot [\A \vee \B] 2}$ = 2), 
and \setDef{\A,\B} (i.e.,$\xeSem [\setDef{\A, \B}] {\annot [\A \vee \B] 2} = 2$) 
but not in variant
\setDef{} (i.e., $\xeSem [\setDef { }] {\annot [\A \vee \B] 2} = \bot$). 
%
The operation $\getPC{\annot{\elem}}=e$ returns the presence condition of an
annotated element.
% with a configuration
%that enables either $\A$ or $\B$ or both
%variants that disable both $\A$ and $\B$.
% Here, $\getPC {\annot [\A \vee \B] 2} = \A \vee \B$.

\input{formulas/vset}

%\point{vset.}
A \emph{variational set} (\emph{v-set}) $\vset = \setDef {\annot [\dimMeta_1] {\elem_1},\ldots, \annot [\dimMeta_n] {\elem_n}}$ 
is a set of annotated elements, 
that is,
$\vset \in \vsetSet$~\cite{EWC13fosd,Walk14onward,ATW17dbpl}.
We typically omit the presence condition \t\ in a variational set,
e.g., $\annot [\t] 4 = 4$.
% where the presence condition of elements is satisfiable~\cite{EWC13fosd,Walk14onward,vdb17ATW}. 
%
Conceptually, a variational set represents many different plain sets simultaneously.
These plain sets can be generated by \emph{configuring} a variational set with a configuration.
This is done by the \emph{variational set configuration} function
\ensuremath{\osetSem \vset: \vsetSet \totype \confSet \totype \psetSet}, defined in \figref{vset}.
The configuration function evaluates the presence condition $\dimMeta_i$ of each 
element $\elem_i$ of the variational set with the configuration \config. 
If the evaluation results in \t\ it includes $\elem_i$ in the plain set and otherwise it
does not. \exref{vset-conf} illustrates the configuration of a variational set for all
possible configurations. 
\structure{it'd be nice to have the entire ex in the same page.}

\begin{example}
\label{eg:vset-conf}
Assume we have the feature space $\features = \setDef {\A, \B}$ 
and the variational set $\vset_1 = \setDef {\annot [\A] 2, \annot [\B] 3, 4}$.
$\vset_1$ represents four plain sets:
\begin{alignat*}{1}
\osetSem {\vset_1} &=
\begin{cases}
  \setDef{2,3,4}, & \config = \setDef{\A,\B}\\
  \setDef{2,4}, & \config = \setDef{\A}\\
  \setDef{3,4}, & \config = \setDef{\B}\\
  \setDef{4}, & \config = \setDef { }
\end{cases}
\end{alignat*}
This states that, for example, configuring $\vset_1$ for the variant that enables 
bot \A\ and \B\ (that is, \ensuremath{\A = \t, \B = \t}) results in the plain set
\ensuremath{ \osetSem [\setDef {\A, \B}] {\vset_1} = \setDef {2,3,4} }.
\end{example}

%
%\noindent
Following database notational conventions
we drop the brackets of a variational set when used in database
schema definitions and queries.

%\point{annotated vset.}
A variational set itself can also be annotated with a feature expression.
%
%An \emph{annotated variational set} 
$\annot \vset = \setDef {\annot [\dimMeta_1] {\elem_1},\ldots,\annot [\dimMeta_n] {\elem_n}}^\dimMeta$ is an
\emph{annotated variational set}, 
that is, $\annot \vset \in \annotvsetSet$.
% that it is annotated itself by a \emph{feature expression} \dimMeta.
%We denote an annotated variational set of elements $\elem \in \mathbf{\elemSet}$ with
%\annot \elemSet.
Annotating a variational set with the feature expression \dimMeta\ means that all
elements in the variational set are only present when \dimMeta\ evaluates to \t.
The \emph{normalization} operation $\pushIn {\annot \vset}$ applies this
restriction by pushing it into the presence conditions of the individual
elements:
\ensuremath{
\pushIn {\annot \vset}
= 
\setDef{\annot [\dimMeta_i \wedge \dimMeta] {\elem_i} \myOR 
\annot [\dimMeta_i] \elem_i \in \annot \vset, \sat {\dimMeta_i \wedge \dimMeta}
}}.
%\eric{added that both v-set and annot v-set are of the same type.}
%Thus, we consider both variational sets and annotated variational sets to 
%belong to the set of variational set \vsetSet, that is, we consider them to have the same type. 
Note that both the normalization operation and variational set configuration
are overloaded, that is, they are defined for both variational sets and 
annotated variational sets. 
Also, note that the \emph{normalization} operation also removes elements
with unsatisfiable presence conditions and may also be applied
to an unannotated variational set \vset\ since $\annot[\t]{\vset} = \vset$.
%\ensuremath{
%\vset = \setDef {\annot [\dimMeta_1] \elem_1, \ldots, \annot [\dimMeta_n] \elem_n}}, 
%which is equivalent to the annotated v-set \annot [\t] \vset. Thus,
%\ensuremath{
%\pushIn \vset = \setDef {
%\annot [\dimMeta_i] \elem_i \myOR \annot [\dimMeta_i] \elem_i, \sat {\dimMeta_i}
%}
%}.}
%This restriction
%can be captured by the property:
%$\setDef {\annot [\dimMeta_1] {\elem_1} ,\ldots, \annot [\dimMeta_n] {\elem_n}}^\dimMeta
%\equiv 
%\setDef {\annot [\dimMeta_1 \wedge \dimMeta] {\elem_1},\ldots, \annot [\dimMeta_n \wedge \dimMeta] {\elem_n}}
%$.
%
For example, the annotated variational set
$\vset_1 = \{\annot [\A] 2, \annot [\neg \B] 3, 4, \annot [\C] 5\}^{\A \wedge \B}$
indicates that all the elements of the set can only exist
when both $\A$ and $\B$ are enabled. Thus, normalizing the variational set $\vset_1$
%the set's feature expression
results in
$\{\annot [\A \wedge \B] 2,\annot [\A \wedge \B] 4,\annot [\A \wedge \B \wedge \C] 5\}$. The element $3$ is dropped 
%from the set 
since 
\ensuremath{\neg \sat {\getPCfrom 3 {\vset_1} }},
where
\ensuremath{
{\getPCfrom 3 {\vset_1} } = \neg \B \wedge (\A \wedge \B)}.
%its presence condition is unsatisfiable, i.e., $\neg \sat {\neg \fName_2 \wedge (\fName_1 \wedge \fName_2)}$.
%%
Note that we use the function \getPCfrom \elem {\annot \vset} to 
return the presence condition of a unique variational element within a bigger
variational structure. 
Note that,
without loss of generality, we assume that elements in a variational set
are unique since we can simply disjoin the presence conditions of a repeated 
element, that is, 
\ensuremath{\setDef {\annot [\dimMeta] \elem, \annot [\dimMeta] \elem, \annot [\dimMeta_1] \elem_1, \ldots, \annot [\dimMeta_n] \elem_n} = \setDef {\annot [\dimMeta \vee \VVal \dimMeta] \elem, \annot [\dimMeta_1] \elem_1, \ldots, \annot [\dimMeta_n] \elem_n}}.
% by just referring to the element itself without its
%annotation, i.e., \elem.

In \figref{vset}, we also define several operations, such as union and
intersection, over variational sets; these operations are used in \secref{type-sys}. The
semantics of a variational set operation is equivalent to applying the corresponding
plain set operation to every corresponding variant of the argument variational sets. For
example, the union of two variational sets $\vset_1\cup\vset_2$ should produce a new
variational set $\vset_3$ such that
%
$\forall c\in\confSet.\;
\osetSem{\vset_3} = \osetSem{\vset_1}\,\underline{\cup}\,\osetSem{\vset_2}$,
where $\underline{\cup}$ is the plain set union operation.
%
 This property must hold for all operations over variational sets, that is, for all possible operations, \vsetOp, defined on variational sets the property 
 \ensuremath{
 \Pone: 
 \forall \config \in \confSet. \osetSem {\pushIn {\vset_1} \vsetOp \pushIn {\vset_2}} 
 = \osetSem {\vset_1} \psetOp \osetSem {\vset_2}
 } must hold, where \psetOp\ is the counterpart operation on plain sets.%
\footnote{This property is proved for the operations we define over variational sets in Coq proof assistant~\cite{Khan21}.}



%\point{vset.}
A \emph{variational set} (\emph{v-set}) $\vset = \setDef {\annot [\dimMeta_1] {\elem_1},\ldots, \annot [\dimMeta_n] {\elem_n}}$ 
is a set of annotated elements, 
that is,
$\vset \in \vsetSet$~\cite{EWC13fosd,Walk14onward,ATW17dbpl}.
We typically omit the presence condition \t\ in a variational set,
e.g., $\annot [\t] 4 = 4$.
% where the presence condition of elements is satisfiable~\cite{EWC13fosd,Walk14onward,vdb17ATW}. 
%
Conceptually, a variational set represents many different plain sets simultaneously.
These plain sets can be generated by \emph{configuring} a variational set with a configuration.
This is done by the \emph{variational set configuration} function
\ensuremath{\osetSem \vset: \vsetSet \totype \confSet \totype \psetSet}, defined in \figref{vset}.
The configuration function evaluates the presence condition $\dimMeta_i$ of each 
element $\elem_i$ of the variational set with the configuration \config. 
If the evaluation results in \t\ it includes $\elem_i$ in the plain set and otherwise it
does not. \exref{vset-conf} illustrates the configuration of a variational set for all
possible configurations. 
\structure{it'd be nice to have the entire ex in the same page.}

\begin{example}
\label{eg:vset-conf}
Assume we have the feature space $\features = \setDef {\A, \B}$ 
and the variational set $\vset_1 = \setDef {\annot [\A] 2, \annot [\B] 3, 4}$.
$\vset_1$ represents four plain sets:
\begin{alignat*}{1}
\osetSem {\vset_1} &=
\begin{cases}
  \setDef{2,3,4}, & \config = \setDef{\A,\B}\\
  \setDef{2,4}, & \config = \setDef{\A}\\
  \setDef{3,4}, & \config = \setDef{\B}\\
  \setDef{4}, & \config = \setDef { }
\end{cases}
\end{alignat*}
This states that, for example, configuring $\vset_1$ for the variant that enables 
bot \A\ and \B\ (that is, \ensuremath{\A = \t, \B = \t}) results in the plain set
\ensuremath{ \osetSem [\setDef {\A, \B}] {\vset_1} = \setDef {2,3,4} }.
\end{example}

%
%\noindent
Following database notational conventions
we drop the brackets of a variational set when used in database
schema definitions and queries.

%\point{annotated vset.}
A variational set itself can also be annotated with a feature expression.
%
%An \emph{annotated variational set} 
$\annot \vset = \setDef {\annot [\dimMeta_1] {\elem_1},\ldots,\annot [\dimMeta_n] {\elem_n}}^\dimMeta$ is an
\emph{annotated variational set}, 
that is, $\annot \vset \in \annotvsetSet$.
% that it is annotated itself by a \emph{feature expression} \dimMeta.
%We denote an annotated variational set of elements $\elem \in \mathbf{\elemSet}$ with
%\annot \elemSet.
Annotating a variational set with the feature expression \dimMeta\ means that all
elements in the variational set are only present when \dimMeta\ evaluates to \t.
The \emph{normalization} operation $\pushIn {\annot \vset}$ applies this
restriction by pushing it into the presence conditions of the individual
elements:
\ensuremath{
\pushIn {\annot \vset}
= 
\setDef{\annot [\dimMeta_i \wedge \dimMeta] {\elem_i} \myOR 
\annot [\dimMeta_i] \elem_i \in \annot \vset, \sat {\dimMeta_i \wedge \dimMeta}
}}.
%\eric{added that both v-set and annot v-set are of the same type.}
%Thus, we consider both variational sets and annotated variational sets to 
%belong to the set of variational set \vsetSet, that is, we consider them to have the same type. 
Note that both the normalization operation and variational set configuration
are overloaded, that is, they are defined for both variational sets and 
annotated variational sets. 
Also, note that the \emph{normalization} operation also removes elements
with unsatisfiable presence conditions and may also be applied
to an unannotated variational set \vset\ since $\annot[\t]{\vset} = \vset$.
%\ensuremath{
%\vset = \setDef {\annot [\dimMeta_1] \elem_1, \ldots, \annot [\dimMeta_n] \elem_n}}, 
%which is equivalent to the annotated v-set \annot [\t] \vset. Thus,
%\ensuremath{
%\pushIn \vset = \setDef {
%\annot [\dimMeta_i] \elem_i \myOR \annot [\dimMeta_i] \elem_i, \sat {\dimMeta_i}
%}
%}.}
%This restriction
%can be captured by the property:
%$\setDef {\annot [\dimMeta_1] {\elem_1} ,\ldots, \annot [\dimMeta_n] {\elem_n}}^\dimMeta
%\equiv 
%\setDef {\annot [\dimMeta_1 \wedge \dimMeta] {\elem_1},\ldots, \annot [\dimMeta_n \wedge \dimMeta] {\elem_n}}
%$.
%
For example, the annotated variational set
$\vset_1 = \{\annot [\A] 2, \annot [\neg \B] 3, 4, \annot [\C] 5\}^{\A \wedge \B}$
indicates that all the elements of the set can only exist
when both $\A$ and $\B$ are enabled. Thus, normalizing the variational set $\vset_1$
%the set's feature expression
results in
$\{\annot [\A \wedge \B] 2,\annot [\A \wedge \B] 4,\annot [\A \wedge \B \wedge \C] 5\}$. The element $3$ is dropped 
%from the set 
since 
\ensuremath{\neg \sat {\getPCfrom 3 {\vset_1} }},
where
\ensuremath{
{\getPCfrom 3 {\vset_1} } = \neg \B \wedge (\A \wedge \B)}.
%its presence condition is unsatisfiable, i.e., $\neg \sat {\neg \fName_2 \wedge (\fName_1 \wedge \fName_2)}$.
%%
Note that we use the function \getPCfrom \elem {\annot \vset} to 
return the presence condition of a unique variational element within a bigger
variational structure. 
Note that,
without loss of generality, we assume that elements in a variational set
are unique since we can simply disjoin the presence conditions of a repeated 
element, that is, 
\ensuremath{\setDef {\annot [\dimMeta] \elem, \annot [\dimMeta] \elem, \annot [\dimMeta_1] \elem_1, \ldots, \annot [\dimMeta_n] \elem_n} = \setDef {\annot [\dimMeta \vee \VVal \dimMeta] \elem, \annot [\dimMeta_1] \elem_1, \ldots, \annot [\dimMeta_n] \elem_n}}.
% by just referring to the element itself without its
%annotation, i.e., \elem.

In \figref{vset}, we also define several operations, such as union and
intersection, over variational sets; these operations are used in \secref{type-sys}. The
semantics of a variational set operation is equivalent to applying the corresponding
plain set operation to every corresponding variant of the argument variational sets. For
example, the union of two variational sets $\vset_1\cup\vset_2$ should produce a new
variational set $\vset_3$ such that
%
$\forall c\in\confSet.\;
\osetSem{\vset_3} = \osetSem{\vset_1}\,\underline{\cup}\,\osetSem{\vset_2}$,
where $\underline{\cup}$ is the plain set union operation.
%
 This property must hold for all operations over variational sets, that is, for all possible operations, \vsetOp, defined on variational sets the property 
 \ensuremath{
 \Pone: 
 \forall \config \in \confSet. \osetSem {\pushIn {\vset_1} \vsetOp \pushIn {\vset_2}} 
 = \osetSem {\vset_1} \psetOp \osetSem {\vset_2}
 } must hold, where \psetOp\ is the counterpart operation on plain sets.%
\footnote{This property is proved for the operations we define over variational sets in Coq proof assistant~\cite{Khan21}.}



%\point{vset.}
A \emph{variational set} (\emph{v-set}) $\vset = \setDef {\annot [\dimMeta_1] {\elem_1},\ldots, \annot [\dimMeta_n] {\elem_n}}$ 
is a set of annotated elements, 
that is,
$\vset \in \vsetSet$~\cite{EWC13fosd,Walk14onward,ATW17dbpl}.
We typically omit the presence condition \t\ in a variational set,
e.g., $\annot [\t] 4 = 4$.
% where the presence condition of elements is satisfiable~\cite{EWC13fosd,Walk14onward,vdb17ATW}. 
%
Conceptually, a variational set represents many different plain sets simultaneously.
These plain sets can be generated by \emph{configuring} a variational set with a configuration.
This is done by the \emph{variational set configuration} function
\ensuremath{\osetSem \vset: \vsetSet \totype \confSet \totype \psetSet}, defined in \figref{vset}.
The configuration function evaluates the presence condition $\dimMeta_i$ of each 
element $\elem_i$ of the variational set with the configuration \config. 
If the evaluation results in \t\ it includes $\elem_i$ in the plain set and otherwise it
does not. \exref{vset-conf} illustrates the configuration of a variational set for all
possible configurations. 
\structure{it'd be nice to have the entire ex in the same page.}

\begin{example}
\label{eg:vset-conf}
Assume we have the feature space $\features = \setDef {\A, \B}$ 
and the variational set $\vset_1 = \setDef {\annot [\A] 2, \annot [\B] 3, 4}$.
$\vset_1$ represents four plain sets:
\begin{alignat*}{1}
\osetSem {\vset_1} &=
\begin{cases}
  \setDef{2,3,4}, & \config = \setDef{\A,\B}\\
  \setDef{2,4}, & \config = \setDef{\A}\\
  \setDef{3,4}, & \config = \setDef{\B}\\
  \setDef{4}, & \config = \setDef { }
\end{cases}
\end{alignat*}
This states that, for example, configuring $\vset_1$ for the variant that enables 
bot \A\ and \B\ (that is, \ensuremath{\A = \t, \B = \t}) results in the plain set
\ensuremath{ \osetSem [\setDef {\A, \B}] {\vset_1} = \setDef {2,3,4} }.
\end{example}

%
%\noindent
Following database notational conventions
we drop the brackets of a variational set when used in database
schema definitions and queries.

%\point{annotated vset.}
A variational set itself can also be annotated with a feature expression.
%
%An \emph{annotated variational set} 
$\annot \vset = \setDef {\annot [\dimMeta_1] {\elem_1},\ldots,\annot [\dimMeta_n] {\elem_n}}^\dimMeta$ is an
\emph{annotated variational set}, 
that is, $\annot \vset \in \annotvsetSet$.
% that it is annotated itself by a \emph{feature expression} \dimMeta.
%We denote an annotated variational set of elements $\elem \in \mathbf{\elemSet}$ with
%\annot \elemSet.
Annotating a variational set with the feature expression \dimMeta\ means that all
elements in the variational set are only present when \dimMeta\ evaluates to \t.
The \emph{normalization} operation $\pushIn {\annot \vset}$ applies this
restriction by pushing it into the presence conditions of the individual
elements:
\ensuremath{
\pushIn {\annot \vset}
= 
\setDef{\annot [\dimMeta_i \wedge \dimMeta] {\elem_i} \myOR 
\annot [\dimMeta_i] \elem_i \in \annot \vset, \sat {\dimMeta_i \wedge \dimMeta}
}}.
%\eric{added that both v-set and annot v-set are of the same type.}
%Thus, we consider both variational sets and annotated variational sets to 
%belong to the set of variational set \vsetSet, that is, we consider them to have the same type. 
Note that both the normalization operation and variational set configuration
are overloaded, that is, they are defined for both variational sets and 
annotated variational sets. 
Also, note that the \emph{normalization} operation also removes elements
with unsatisfiable presence conditions and may also be applied
to an unannotated variational set \vset\ since $\annot[\t]{\vset} = \vset$.
%\ensuremath{
%\vset = \setDef {\annot [\dimMeta_1] \elem_1, \ldots, \annot [\dimMeta_n] \elem_n}}, 
%which is equivalent to the annotated v-set \annot [\t] \vset. Thus,
%\ensuremath{
%\pushIn \vset = \setDef {
%\annot [\dimMeta_i] \elem_i \myOR \annot [\dimMeta_i] \elem_i, \sat {\dimMeta_i}
%}
%}.}
%This restriction
%can be captured by the property:
%$\setDef {\annot [\dimMeta_1] {\elem_1} ,\ldots, \annot [\dimMeta_n] {\elem_n}}^\dimMeta
%\equiv 
%\setDef {\annot [\dimMeta_1 \wedge \dimMeta] {\elem_1},\ldots, \annot [\dimMeta_n \wedge \dimMeta] {\elem_n}}
%$.
%
For example, the annotated variational set
$\vset_1 = \{\annot [\A] 2, \annot [\neg \B] 3, 4, \annot [\C] 5\}^{\A \wedge \B}$
indicates that all the elements of the set can only exist
when both $\A$ and $\B$ are enabled. Thus, normalizing the variational set $\vset_1$
%the set's feature expression
results in
$\{\annot [\A \wedge \B] 2,\annot [\A \wedge \B] 4,\annot [\A \wedge \B \wedge \C] 5\}$. The element $3$ is dropped 
%from the set 
since 
\ensuremath{\neg \sat {\getPCfrom 3 {\vset_1} }},
where
\ensuremath{
{\getPCfrom 3 {\vset_1} } = \neg \B \wedge (\A \wedge \B)}.
%its presence condition is unsatisfiable, i.e., $\neg \sat {\neg \fName_2 \wedge (\fName_1 \wedge \fName_2)}$.
%%
Note that we use the function \getPCfrom \elem {\annot \vset} to 
return the presence condition of a unique variational element within a bigger
variational structure. 
Note that,
without loss of generality, we assume that elements in a variational set
are unique since we can simply disjoin the presence conditions of a repeated 
element, that is, 
\ensuremath{\setDef {\annot [\dimMeta] \elem, \annot [\dimMeta] \elem, \annot [\dimMeta_1] \elem_1, \ldots, \annot [\dimMeta_n] \elem_n} = \setDef {\annot [\dimMeta \vee \VVal \dimMeta] \elem, \annot [\dimMeta_1] \elem_1, \ldots, \annot [\dimMeta_n] \elem_n}}.
% by just referring to the element itself without its
%annotation, i.e., \elem.

In \figref{vset}, we also define several operations, such as union and
intersection, over variational sets; these operations are used in \secref{type-sys}. The
semantics of a variational set operation is equivalent to applying the corresponding
plain set operation to every corresponding variant of the argument variational sets. For
example, the union of two variational sets $\vset_1\cup\vset_2$ should produce a new
variational set $\vset_3$ such that
%
$\forall c\in\confSet.\;
\osetSem{\vset_3} = \osetSem{\vset_1}\,\underline{\cup}\,\osetSem{\vset_2}$,
where $\underline{\cup}$ is the plain set union operation.
%
 This property must hold for all operations over variational sets, that is, for all possible operations, \vsetOp, defined on variational sets the property 
 \ensuremath{
 \Pone: 
 \forall \config \in \confSet. \osetSem {\pushIn {\vset_1} \vsetOp \pushIn {\vset_2}} 
 = \osetSem {\vset_1} \psetOp \osetSem {\vset_2}
 } must hold, where \psetOp\ is the counterpart operation on plain sets.%
\footnote{This property is proved for the operations we define over variational sets in Coq proof assistant~\cite{Khan21}.}


\section{The Formula Choice Calculus}
\label{sec:fcc}


%To account for variation, VRA combines relational algebra (RA) with 
%\emph{choices}~\cite{EW11tosem,HW16fosd,Walk13thesis}.
%%\point{choice.}
%A choice $\chc{\elem_1,\elem_2}$ consists of a feature expression \dimMeta, called
%the \emph{dimension} of the choice, and 
%two \emph{alternatives} $\elem_1$ and $\elem_2$. For a given configuration \config, 
%the choice $\chc{\elem_1, \elem_2}$ can be replaced by $\elem_1$ if \dimMeta\
%evaluates to \t\ under configuration \config, (i.e., \fSem{\dimMeta}),
%or $\elem_2$ otherwise. 

\eric{please read the entire section. thx!}
The second approach we use to incorporate variation into queries is
the formula choice calculus~\cite{HW16fosd} which is an extension of 
the choice calculus~\cite{Walk13thesis,EW11tosem}. 
%
The choice calculus~\cite{Walk13thesis,EW11tosem} is a metalanguage for
describing variation in programs and its elements such as data 
structures~\cite{Walk14onward,EWC13fosd}.
In the choice calculus, variation is represented in-place as
choices between alternative subexpressions. For example, 
the variational expression 
$\mathit{expr} = \chc [\A] {1,2} + \chc [\B] {3,4} + \chc [\A] {5,6}$
 contains three choices.
Each choice has an associated \emph{dimension}, which is used to
synchronize the choice with other choices in different parts
of the expression. For example, expression $\mathit{expr}$ contains
two dimensions, $\A$ and $\B$, and the two choices in dimension
$\A$ are synchronized. Therefore, the variational expression
$\mathit{expr}$ represents four different plain expressions, depending
on whether the left or right alternatives are selected from each
dimension. Assuming that dimensions may be set to boolean values
where \t\ indicates the left alternative and \f\ indicates the
right alternative, we have: 
%(1) $1+3+5$, when $A$ and $B$ are \t,
%(2) $1+4+5$, when $A$ is \t\ and $B$ is \f,
%(3) $2+3+6$, when $A$ is \f\ and $B$ is \t,
%and (4) $2+4+6$, when $A$ and $B$ are \f.
\begin{alignat*}{1}
\chc [\A] {1,2} + \chc [\B] {3,4} + \chc [\A] {5,6} &=
\begin{cases}
  1+3+5,& \A =\t, \B = \t\\
  1+4+5,& \A =\t, \B = \f\\
  2+3+6,& \A =\f, \B = \t\\
  2+4+6,& \A =\f, \B = \f
\end{cases}
\end{alignat*}
%
\noindent
The formula
choice calculus extends the choice calculus 
by allowing dimensions to be propositional formulas~\cite{HW16fosd}. For example,
the variational expression $\VVal {\mathit{expr}} = \chc [\A \vee \B] {1,2} + \chc [\B] {3,4} + \chc [\A] {5,6}$ represents
four plain expressions: 
%(1) $1$, when $\A \vee \B$ evaluates to \t\
%and (2) $2$, when $\A \vee \B$ evaluates to \f. More explicitly, we have:
\begin{alignat*}{1}
\chc [\A \vee \B] {1,2} + \chc [\B] {3,4} + \chc [\A] {5,6}&=
\begin{cases}
  1+3+5,& \A =\t, \B = \t\\
  1+4+5,& \A =\t, \B = \f\\
  1+3+6,& \A =\f, \B = \t\\
  2+4+6,& \A =\f, \B = \f
\end{cases}
\end{alignat*}



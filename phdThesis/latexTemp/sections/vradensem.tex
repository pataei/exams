\subsection{VRA Denotational Semantics }
\label{sec:vradensem}


Now that we have all required parts we define the denotational semantics of 
variational relational algebra using the denotational semantics of relational 
algebra. 
We assume the existence of the function
%The denotational semantics of relational queries 
$\mathit{rqSem} : \pQSet \totype \pInstSet \totype \pTabSet$, which given a plain query and
a plain database, returns a plain table named $\mathit{result}$
according to the standard semantics of plain relational algebra.
%We do not give a formal 
%definition of $\mathit{rqSem}$, however, examples of the semantics
%of a query are given in \tabref{vq-conf-res}.
%
We then define the VRA denotational semantics 
$\mathit{vqSem} : \qSet \totype \vdbInstSet \totype \tabletype$ as the 
accumulation of relational tables resulting from the semantics of its
configured queries over their corresponding configured databases for all 
valid configurations of a variational database. 
%
The $\mathit{mapRQSem}$ function takes a set of relational queries with their attached
configurations and a variational database instance and returns the set of query 
semantics over their configured database
with their attached configurations, that is, it maps $\mathit{rqSem}$ on the 
relational queries over their corresponding relational database.%
\footnote{In the implementation, the closed set of features and valid configurations
of a VDB are contained within, instead of extracting them from the database. However,
we keep the formalization simple and assume that they can also be retrieved from
the VDB.}
%
Finally, the $\mathit{qToConfRelQs}$ takes a well-typed, explicitly annotated 
variational query and the set of valid configurations and configures the variational
query for the given configurations and returns the set of configured queries paired 
with their corresponding configuration.


\begin{figure}

\textbf{VRA denotational semantics:}
\begin{alignat*}{1}
\mathit{vqSem} &: \qSet \totype \vdbInstSet \totype \tabletype\\
\mathit{vqSem} \  \vQ \ \vdbInst &= \mathit{accum} \ \mathit{fs} \ \mathit{tabs}\\
&\hspace{-30pt}\textit{where } \mathit{fs} =\mathit{featues} \ \vdbInst\\
&\hspace{2pt} \mathit{rqs} = \mathit{qToConfRelQs} \ \vQ \ (\mathit{validConfigs} \ \vdbInst)\\
&\hspace{2pt} \mathit{tabs} = \mathit{mapRQSem} \ \mathit{rqs} \ \vdbInst
\end{alignat*}


\medskip 
\textbf{Auxiliary functions for VRA denotational semantics:}
%\footnotesize
\begin{alignat*}{1}
\mathit{rqSem} &: \pQSet \totype \pInstSet \totype \pTabSet\\
\mathit{mapRQSem} &: \settype {\typepair \confSet \pQSet} \totype \vdbInstSet \totype \settype {\typepair \confSet \pTabSet}\\
\mathit{features} &: \vdbInstSet \totype \settype \fSet\\
\mathit{validConfigs} &: \vdbInstSet \totype \settype \confSet\\
\mathit{qToConfRelQs} &: \qSet \totype \settype \confSet \totype \settype {\typepair \confSet \pQSet}\\
\end{alignat*}


\caption[VRA denotational semantics]{Denotational semantics of variational relational algebra.
Note that the query \vQ\ is well-typed and explicitly annotated by the schema of the VDB instance \vdbInst.
}
\label{fig:densem}
\end{figure}



%\begin{example}
%\label{eg:sem}
%\wrrite{write the damn thing}
%\end{example}


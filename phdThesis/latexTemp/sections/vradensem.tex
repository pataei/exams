\subsection{VRA Denotational Semantics }
\label{sec:vradensem}

\eric{pls read this entire section. thx!}
Now that we have all required parts we define the denotational semantics of 
variational relational algebra using the denotational semantics of relational 
algebra. The denotational semantics of relational queries 
$\mathit{rqSem} : \pQSet \totype \pInstSet \totype \pTabSet$ takes a plain query and
a plain database and returns a table named $\mathit{result}$. We do not give a formal 
definition of $\mathit{rqSem}$, however, examples of the semantics
of a query are given in \tabref{vq-conf-res}.
%
We then define the VRA denotational semantics 
$\mathit{vqSem} : \qSet \totype \vdbInstSet \totype \tabletype$ as the 
accumulation of relational tables resulted from the semantics of its
configured queries over their corresponding configured databases for all 
valid configurations of a variational database. 
%
The $\mathit{mapRQSem}$ takes a set of relational queries with their attached
configurations and a variational database instance and returns the set of query 
semantics over their configured database
with their attached configurations, that is, it maps $\mathit{rqSem}$ on the 
relational queries over their corresponding relational database.%
\footnote{In implementation, the closed set of features and valid configurations
of a VDB are contained within, instead of extracting them from the database. However,
we keep the formalization simple and assume that they can also be retrieved from
the VDB.}

%\TODO{think if you want to have example or more explanation of helpers.}

\begin{figure}

\textbf{VRA denotational semantics:}
\begin{alignat*}{1}
\mathit{vqSem} &: \qSet \totype \vdbInstSet \totype \tabletype\\
\mathit{vqSem} \  \vQ \ \vdbInst &= \mathit{accum} \ \mathit{fs} \ \mathit{tabs}\\
&\hspace{-30pt}\textit{where } \mathit{fs} =\mathit{featues} \ \vdbInst\\
&\hspace{2pt} \mathit{rqs} = \mathit{qToConfRelQs} \ \vQ \ (\mathit{validConfigs} \ \vdbInst)\\
&\hspace{2pt} \mathit{tabs} = \mathit{mapRQSem} \ \mathit{rqs} \ \vdbInst
\end{alignat*}


\medskip 
\textbf{Auxiliary functions for VRA denotational semantics:}
%\footnotesize
\begin{alignat*}{1}
\mathit{rqSem} &: \pQSet \totype \pInstSet \totype \pTabSet\\
\mathit{mapRQSem} &: \settype {\typepair \confSet \pQSet} \totype \vdbInstSet \totype \settype {\typepair \confSet \pTabSet}\\
\mathit{features} &: \vdbInstSet \totype \settype \fSet\\
\mathit{validConfigs} &: \vdbInstSet \totype \settype \confSet\\
\mathit{qToConfRelQs} &: \qSet \totype \settype \confSet \totype \settype {\typepair \confSet \pQSet}\\
\end{alignat*}


\caption[VRA denotational semantics]{Denotational semantics of variational relational algebra.
Note that the query \vQ\ is well-typed and explicitly annotated by the schema of the VDB instance \vdbInst.
}
\label{fig:densem}
\end{figure}



%\begin{example}
%\label{eg:sem}
%\wrrite{write the damn thing}
%\end{example}


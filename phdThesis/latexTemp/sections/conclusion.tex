\chapter{Conclusion}
\label{ch:conclusion}

\TODO{conclusion}

\fromppr{from vamos}
We provide two use cases that illustrate how software variation leads to
corresponding variation in relational databases. These use cases demonstrate
the feasibility of VDBs and v-queries to capture the data needs of variational
software systems.
%
We argue that effectively managing such variation is an open problem, and we
believe that these use cases will form a useful basis for evaluating
research that addresses it, such as our own VDBMS framework.
%
% VDB
%and VRA.
%
%The case studies were developed by systematically combining existing data
%sources with software variation scenarios described in the literature. They
%each consist of a variational schema describing the structural variation of
%the database, the variational database itself containing the variational
%content, and a set of variational queries that satisfy realistic information
%needs over multiple variants of each database.


VDBs encode variation explicitly in the structure and content of databases.
%
This is a source of complexity that may impact understandability, as can be
observed in our use cases. However, it also has several advantages: it
is general in the sense that any set of variant databases and queries can be
encoded as a VDB and v-queries, and it enables directly associating variation
in databases to variation in software.
%
By applying variational typing to variational queries, this generality does not
come at the cost of safety. Future work can explore how tooling can mitigate
the usability concerns using techniques that have been developed in the SPL
community.

% We discussed that explicitly encoding variation in databases
% allows  tracing variation between the program and data. It also empowers
% developer to check properties over their database and queries to ensure that
% constraints over a database holds and queries are well-behaved.

%p: taken from rel work
% Although we have focused on variational databases to support SPL development,
% the broader motivation of \emph{effectively computing with variability} is at
% the heart of our work. This is why VDBs support not only structural variation
% but also content-level variation. Also, while variational queries can be
% statically configured in the same way that SPLs typically are, our prototype

%%VDBMS implementation also supports directly executing variational queries on
%%variational databases to yield variational results.


% These case studies can be used to 1) evaluate approaches and systems
% attempting to manage any kind of variation in database, 2) learn how a
% variational database can be generated from a scenario that describes such
% variation, and 3) design a system that automatically generates a variational
% database from non-variational databases and their corresponding variant. In
% particular, we use these use cases to evaluate our Variational Database
% Management System~\cite{vldbArXiv}. It would be interesting to investigate 1)
% how database systems that manage a specific kind of variation deal with
% variational databases and 2) how database systems that account for different
% kinds of variation can be improved to manage more specific kinds of
% variation.

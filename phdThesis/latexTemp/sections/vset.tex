\section{Variational Set}
\label{sec:vset}

\TODO{havve to revise this}

\NOTE{The sentence starting with ``Thus, \ldots'' is a bit confusing since
it's not really obvious what \psetSet\ and \vsetSet\ are. Also, it makes it
sound like any plain set is a variant of any variational set, which obviously
isn't true. I recommend just introducing configuration after the
``Conceptually, \ldots'' sentence. If we only need \psetSet\ and \vsetSet\ to
define the type of configuration, then we should not introduce them and just
not give this type (I think it's easy enough to explain and understand without
it).}

%\point{vset.}
A \emph{variational set} (\emph{v-set}) $\vset = \setDef {\annot [\dimMeta_1] {\elem_1},\ldots, \annot [\dimMeta_n] {\elem_n}}$ 
is a set of annotated elements~\cite{EWC13fosd,Walk14onward,ATW17dbpl}.
We typically omit the presence condition \t\ in a variational set,
e.g., $\annot [\t] 4 = 4$.
% where the presence condition of elements is satisfiable~\cite{EWC13fosd,Walk14onward,vdb17ATW}. 
%
Conceptually, a variational set represents many different plain sets simultaneously.
Thus, a plain set $\pset \in \psetSet$  is a \emph{variant} of the variational set $\vset \in \vsetSet$ 
and given variant $\pset$'s configuration $\config \in \confSet$
% which corresponds to \pset\ 
it can be generated by evaluating the presence condition of each element with \config\
and including the element if the said evaluation results in \t\ and excluding it otherwise.
%applying the corresponding configuration $\config \in \confSet$ to the 
%variational set, 
This is done by the \emph{v-set configuration} function
\ensuremath{\osetSem \vset : \vsetSet \to \confSet \to \psetSet}. 
%Thus, the v-set configuration function has the type
%\ensuremath{\osetSem . : \vsetSet \to \confSet \to \psetSet}.
%that can be generated by enabling or disabling features
%and including only the elements whose feature expressions evaluate to \t.
%We typically omit the presence condition \t\ in a variational set,
For example, 
assume we have the feature space $\fSet = \setDef {\A, \B}$ 
and the v-set $\vset_1 = \setDef {\annot [\A] 2, \annot [\B] 3, 4}$.
$\vset_1$ represents four plain sets:
% under different configurations. These plain
%sets 
%they can be generated by \emph{configuring} the variational set with a
%given configuration: 
%\revised{
\ensuremath{ \setDef {2,3,4} = \osetSem [\setDef {\A, \B}] {\vset_1}}, i.e., the variant 
 \setDef {2,3,4} is generated from the v-set $\vset_1$ under the configuration \ensuremath{\A = \t, \B = \t};
% $\A$ and $\B$
%are enabled; 
\ensuremath{ \setDef {2,4} = \osetSem [\setDef {\A}] {\vset_1}};
%\setDef {2,4}, when \ensuremath{\A = \t, \B = \f};
%$\A$ is enabled but $\B$ is disabled;
\ensuremath{ \setDef {3,4} = \osetSem [\setDef {\B}] {\vset_1}};
%\setDef {3,4}, when \ensuremath{\A = \f, \B = \t};
%$\B$ is enabled but $\A$ is disabled;
and \ensuremath{ \setDef {4} = \osetSem [\setDef {}] {\vset_1}}.
%and \setDef {4}, when \ensuremath{\A = \f, \B = \f}.
%both $\A$ and $\B$ are disabled.
%Configuring a v-set \ensuremath{\vset \in \vsetSet} for a given configuration
%\ensuremath{\config \in \confSet} is done by the \emph{configuration function}:
%\ensuremath{\osetSem . : \vsetSet \to \confSet \to \psetSet}.
%}
%
%We indicate variational sets of elements $\elem \in \mathbf{\elemSet}$ with \elemSet.
%A variational set is conceptually a function from a configuration of its
%features to the corresponding plain set. 
%We typically omit the feature
%expression \prog{true} in a variational set, for example, in the
%variational set $\{5,6^{f_1}\}$, the feature expression for the value $5$ is
%implicitly \prog{true}, and so the element is included in both variants:
%$\{5,6\}$ when feature $f_1$ is enabled and $\{5\}$ when feature $f_1$ is
%disabled.
%The \emph{normalization} operation \pushIn \vset drops elements with 
%unsatisfiable presence conditions from the v-set \vset. That is, 
%\ensuremath{
%\pushIn \vset = \setDef {\annot [\dimMeta_i] {\elem_i} \myOR  \annot [\dimMeta_i] {\elem_i} \in \vset, 
%\sat {\dimMeta_i}}
%}, e.g., 
%\ensuremath{
%\pushIn {\setDef {\annot [\f] 1, \annot [\A \wedge \neg \A] 2} }= \setDef {}}.
%Note that elements with unsatisfiable presence condition can be omitted
%from the v-set, e.g., the v-set \ensuremath{\setDef {\annot [\f] {1}}} is 
%equivalent to an empty v-set.
%\revised{When required, we use the \emph{filter} function,
%\filterSet \vset, to indicate that all elements with an unsatisfiable presence condition
%have been removed from the v-set \vset, e.g., \filterSet {\setDef {\annot [\f] 1}} = \{\}.}
%
%For simplicity and to comply with 
Following database notational conventions
we drop the brackets of a variational set when used in database
schema definitions and queries.
%for defining 
%variational relation schemas and the variational attribute set to be projected in a query.

%\point{annotated vset.}
A variational set itself can also be annotated with a feature expression.
%
%An \emph{annotated variational set} 
$\annot \vset = \setDef {\annot [\dimMeta_1] {\elem_1},\ldots,\annot [\dimMeta_n] {\elem_n}}^\dimMeta$ is an
\emph{annotated v-set}.
% that it is annotated itself by a \emph{feature expression} \dimMeta.
%We denote an annotated variational set of elements $\elem \in \mathbf{\elemSet}$ with
%\annot \elemSet.
Annotating a v-set with the feature expression \dimMeta\ means that all
elements in the v-set are only present when \dimMeta\ is true.
The \emph{normalization} operation $\pushIn {\annot \vset}$ applies this
restriction by pushing it into the presence conditions of the individual
elements:
% This restriction 
%can be applied to all elements of the set by \emph{normalizing}  the
%annotated v-set, done by the operation
%\NOTE{
\ensuremath{
\pushIn {\annot \vset}
% {\setDef {\annot [\dimMeta_1] {\elem_1},\ldots,\annot [\dimMeta_n] {\elem_n}}^\dimMeta}
= 
%\annot {\setDef{\annot [\dimMeta_i] \elem_i \myOR \sat {\dimMeta_i \wedge \dimMeta}, 1 \leq i \leq n}}}.}
%\setDef {\annot [\dimMeta_1 \wedge \dimMeta] {\elem_1},\ldots, \annot [\dimMeta_n \wedge \dimMeta] {\elem_n}}
\setDef{\annot [\dimMeta_i \wedge \dimMeta] {\elem_i} \myOR 
\annot [\dimMeta_i] \elem_i \in \annot \vset, \sat {\dimMeta_i \wedge \dimMeta}
}
}.
Note that the \emph{normalization} operation also removes elements
with unsatisfiable presence conditions and may also be applied
to an unannotated v-set \vset\ since $\annot[\t]{\vset} = \vset$.
%\ensuremath{
%\vset = \setDef {\annot [\dimMeta_1] \elem_1, \ldots, \annot [\dimMeta_n] \elem_n}}, 
%which is equivalent to the annotated v-set \annot [\t] \vset. Thus,
%\ensuremath{
%\pushIn \vset = \setDef {
%\annot [\dimMeta_i] \elem_i \myOR \annot [\dimMeta_i] \elem_i, \sat {\dimMeta_i}
%}
%}.}
%This restriction
%can be captured by the property:
%$\setDef {\annot [\dimMeta_1] {\elem_1} ,\ldots, \annot [\dimMeta_n] {\elem_n}}^\dimMeta
%\equiv 
%\setDef {\annot [\dimMeta_1 \wedge \dimMeta] {\elem_1},\ldots, \annot [\dimMeta_n \wedge \dimMeta] {\elem_n}}
%$.
%
For example, the annotated v-set
$\vset_1 = \{\annot [\A] 2, \annot [\neg \B] 3, 4, \annot [\C] 5\}^{\A \wedge \B}$
indicates that all the elements of the set can only exist
when both $\A$ and $\B$ are enabled. Thus, normalizing the v-set $\vset_1$
%the set's feature expression
results in
$\{\annot [\A \wedge \B] 2,\annot [\A \wedge \B] 4,\annot [\A \wedge \B \wedge \C] 5\}$. The element $3$ is dropped 
%from the set 
since 
\ensuremath{\neg \sat {\getPCfrom 3 {\vset_1} }},
where
\ensuremath{
{\getPCfrom 3 {\vset_1} } = \neg \B \wedge (\A \wedge \B)}.
%its presence condition is unsatisfiable, i.e., $\neg \sat {\neg \fName_2 \wedge (\fName_1 \wedge \fName_2)}$.
%%
Note that we use the function \getPCfrom \elem {\annot \vset} to 
return the presence condition of a unique variational element within a bigger
variational structure.
% by just referring to the element itself without its
%annotation, i.e., \elem.

In \appref{vset-ops} we define several operations, such as union and
intersection, over v-sets; these operations are used in \secref{type-sys}. The
semantics of a v-set operation is equivalent to applying the corresponding
plain set operation to every corresponding variant of the argument v-sets. For
example, the union of two v-sets $\vset_1\cup\vset_2$ should produce a new
v-set $\vset_3$ such that
%
$\forall c\in\confSet.\;
\osetSem{\vset_3} = \osetSem{\vset_1}\,\underline{\cup}\,\osetSem{\vset_2}$,
where $\underline{\cup}$ is the plain set union operation.
%
% Thus, for all possible operations, \vsetOp, defined on v-sets the property 
% \ensuremath{
% \Pone: \forall \config \in \confSet. \osetSem {\pushIn {\vset_1} \vsetOp \pushIn {\vset_2}} 
% = \osetSem {\vset_1} \psetOp \osetSem {\vset_2}
% } must hold, where \psetOp\ is the counterpart operation on plain sets.
%
%For example, we define the intersection of two v-sets as follow
%\ensuremath{
%\vset_1 \cap \vset_2 = \setDef {
%\annot [\dimMeta_1 \wedge \dimMeta_2 ]\elem \myOR
%\annot [\dimMeta_1] \elem \in \vset_1, \annot [\dimMeta_2] \elem \in \vset_2,
%\sat {\dimMeta_1 \wedge \dimMeta_2}
%}}
%.
%Note that the property \Pone\ holds and \A\ and \B\ are feature names and not 
%feature expressions. 

%\revised{Remember that \fName\ and \dimMeta\ are metavariables for feature names
%and feature expressions, respectively.}
 
%These operations are vastly used
%in \secref{type-sys}.

%
%\begin{definition}[V-set union]
%\label{def:vset-union}
%The \emph {union} of two v-sets is the union of their elements with the disjunction of 
%presence conditions if an element exists in both v-sets:
%\revised{
%\ensuremath{
%\vset_1 \cup \vset_2 = \setDef {\annot [\dimMeta_1] \elem \myOR \annot [\dimMeta_1] \elem \in \pushIn {\vset_1}, \exists \dimMeta_2. \annot [\dimMeta_2] \elem \not \in \pushIn{\vset_2}}
%\cup \setDef {\annot [\dimMeta_2] \elem \myOR \annot [\dimMeta_2] \elem \in \pushIn {\vset_2}, 
%\exists \dimMeta_1. \annot [\dimMeta_1] \elem  \not \in \pushIn {\vset_1}}
%\cup \setDef {\annot [\dimMeta_1 \vee \dimMeta_2] \elem \myOR 
%\annot [\dimMeta_1] \elem \in \pushIn {\vset_1}, \annot [\dimMeta_2] \elem \in \pushIn {\vset_2}}
%}.
%For example, the union of two normalized v-sets
%\ensuremath{\setDef {2,\annot [\dimMeta_1] 3, \annot [\dimMeta_1] 4}} and 
%\ensuremath{\setDef {\annot [\dimMeta_2] 3, \annot [\neg \dimMeta_1] 4}} is
%\ensuremath{
%%\setDef {2,\annot [\dimMeta_1] 3, \annot [\dimMeta_1] 4} \cup \setDef {\annot [\dimMeta_2] 3, \annot [\neg \dimMeta_1] 4} =
% \setDef {2, \annot [\dimMeta_1 \vee \dimMeta_2] 3, 4}
%}.}
%\end{definition}
%
%% 
%% is needed for the implicitly-type lang:
%\begin{definition}[V-set intersection]
%\label{def:vset-intersect}
%The \emph{intersection} of two v-sets is a v-set of their shared elements
%annotated with the conjunction of their presence conditions, i.e., 
%\ensuremath{
%\vset_1 \cap \vset_2 = \setDef {
%\annot [\dimMeta_1 \wedge \dimMeta_2 ]\elem \myOR
%\annot [\dimMeta_1] \elem \in \vset_1, \annot [\dimMeta_2] \elem \in \vset_2,
%\sat {\dimMeta_1 \wedge \dimMeta_2}
%}
%}.
%For example, \ensuremath{
%\setDef {2, \annot [\A] 3, \annot [\neg \B] 4} \cap
%\pushIn {\annot [\B] {\setDef{2,3,4,5}}} =
%\setDef{\annot [\B] 2, \annot [\A \wedge \B] 3}
%}.
%\end{definition}
%
%\begin{definition} [V-set cross product]
%\label{def:vset-cross}
%The \emph{cross product} of two v-sets is a pair of every two elements of 
%them annotated with the conjunction of their presence conditions.
%\ensuremath{
%\vset_1 \times \vset_2 = \setDef{
%\annot [\dimMeta_1 \wedge \dimMeta_2] {(\elem_1, \elem_2)} \myOR
%\annot [\dimMeta_1] \elem_1 \in \vset_1, \annot [\dimMeta_2] \elem_2 \in \vset_2
%%\vset_1 \cap \vset_2 = \setDef \
%}
%}
%%
%\end{definition}
%
%\begin{definition} [V-set equivalence]
%\label{def:vset-eq}
%%\revised{Two v-sets are \emph{equivalent}, denoted by
%%\ensuremath{\vset_1 \equiv \vset_2}, iff
%%\ensuremath{\forall \config \in \confSet.
%%%\forall \annot \elem \in  (\vset_1 \cup \vset_2).
%%\elemSem {\vset_1} \equiv \elemSem {\vset_2}},
%%where \ensuremath{\elemSem {.} : \vsetSet \to \confSet \to \psetSet}
%%is a configuration function for variational sets. 
%%Thus, two v-sets are equivalent iff all their configured sets are equivalent.
%%}
%%\TODO{Two v-sets are \emph{equivalent}, denoted by
%%\ensuremath{\vset_1 \equiv \vset_2}, iff
%%\ensuremath{
%%\forall \annot  \elem \in (\vset_1 \cup \vset_2),
%%\annot [\dimMeta_1] \elem \in \vset_1, \annot [\dimMeta_2] \elem \in \vset_2.
%%\dimMeta_1 \equiv \dimMeta_2, \dimMeta \equiv \dimMeta_1},
%%this fails b/c {1 fls == 2 fls}
%%i.e., they both cover the same set of elements and the presence conditions
%%of elements from the two v-sets are equivalent.}
%\revised{Two v-sets are \emph{equivalent}, denoted by
%\ensuremath{\vset_1 \equiv \vset_2}, iff
%\ensuremath{\forall \annot  \elem \in (\filterSet \vset_1 \cup \filterSet \vset_2),
%\annot [\dimMeta_1] \elem \in \filterSet \vset_1, \annot [\dimMeta_2] \elem \in \filterSet \vset_2.
%\dimMeta_1 \equiv \dimMeta_2, \dimMeta \equiv \dimMeta_1},
%i.e., they both cover the same set of elements and the presence conditions
%of elements from the two normalized v-sets are equivalent.}
%\end{definition}
%
%%
%\begin{definition} [V-set subsumption]
%\label{def:vset-subsumption}
%\revised{
%The v-set \ensuremath{\vset_1} \emph {subsumes} the v-set
%\ensuremath{\vset_2}, $\subsume {\vset_2} {\vset_1}$, iff
%\ensuremath{ \forall \annot [\dimMeta_2] \elem \in \pushIn {\vset_2}.
%\annot [\dimMeta_1] \elem \in \pushIn {\vset_1}, 
%%\neg \sat {\dimMeta_2 \wedge \neg \dimMeta_1}
%\sat {\dimMeta_2 \wedge  \dimMeta_1}
%},
%}
%i.e., all elements in $\vset_2$ also exist in $\vset_1$ 
%s.t. the element is valid in a shared configuration between the v-sets.
%For example, 
%\ensuremath{
% \subsume {\pushIn {\annot [\A] {\setDef {2,3}}}} {\setDef {2, \annot [\A \vee \B] 3, 4}}},
%however, 
%\ensuremath{
% \nsubsume {\pushIn {\annot [\A] {\setDef {2,3}}}} {\setDef {2, \annot [\neg \A \wedge \B] 3}}}
%and
%\ensuremath{
%\nsubsume {\setDef {\annot [\A] 2,\annot [\A] 3, 4}} {\setDef {2, \annot [\A \wedge \B] 3}}}.
%\end{definition}
%
%%\begin{definition} [V-set explicit subsumption]
%%\dropit{drop this for vldb submission. remember you need it for popl.}
%%\label{def:vset-strict-subsumption}
%%The v-set \ensuremath{\vset_1} \emph {explicitly subsumes} the v-set
%%\ensuremath{\vset_2}, $\subsumeExpl {\vset_2} {\vset_1}$, iff
%%\ensuremath{ \forall \annot [\dimMeta_2] \elem \in \vset_2.
%%\annot [\dimMeta_1] \elem \in \vset_1, 
%%\neg \sat {\dimMeta_2 \wedge \neg \dimMeta_1}
%%},
%%i.e., all elements in $\vset_2$ also exist in $\vset_1$ 
%%s.t. its presence condition in \ensuremath{\vset_2} is more specific than 
%%its presence condition in \ensuremath{\vset_1}, captured by 
%%\ensuremath{\neg \sat {\dimMeta_2 \wedge \neg \dimMeta_1}}
%%which could also be defined as 
%%\ensuremath{
%%\nexists \config \in \confSet . \fSem {\dimMeta_1} = \t , \fSem {\dimMeta_2} = \f.
%%%i.e. in set theory:
%%% \dimMeta_2 \subset \dimMeta_1
%%%\dimMeta_2 - \dimMeta_1 = \emptyset
%%%i.e.
%%%\dimMeta_2 \cap \bar{\dimMeta_1} = \emptyset 
%%}
%%\end{definition}

\begin{definition}[V-set union]
\label{def:vset-union}
The \emph {union} of two v-sets is the union of their elements with the disjunction of 
presence conditions if an element exists in both v-sets:

\ensuremath{
\vset_1 \cup \vset_2 = \setDef {\annot [\dimMeta_1] \elem \myOR \annot [\dimMeta_1] \elem \in \pushIn {\vset_1}, \exists \dimMeta_2. \annot [\dimMeta_2] \elem \not \in \pushIn{\vset_2}}
\cup \setDef {\annot [\dimMeta_2] \elem \myOR \annot [\dimMeta_2] \elem \in \pushIn {\vset_2}, 
\exists \dimMeta_1. \annot [\dimMeta_1] \elem  \not \in \pushIn {\vset_1}}
\cup \setDef {\annot [\dimMeta_1 \vee \dimMeta_2] \elem \myOR 
\annot [\dimMeta_1] \elem \in \pushIn {\vset_1}, \annot [\dimMeta_2] \elem \in \pushIn {\vset_2}}.
For example, the union of two normalized v-sets
\ensuremath{\setDef {2,\annot [\dimMeta_1] 3, \annot [\dimMeta_1] 4}} and 
\ensuremath{\setDef {\annot [\dimMeta_2] 3, \annot [\neg \dimMeta_1] 4}} is
\ensuremath{
%\setDef {2,\annot [\dimMeta_1] 3, \annot [\dimMeta_1] 4} \cup \setDef {\annot [\dimMeta_2] 3, \annot [\neg \dimMeta_1] 4} =
 \setDef {2, \annot [\dimMeta_1 \vee \dimMeta_2] 3, 4}
}.}
\end{definition}

% 
% is needed for the implicitly-type lang:
\begin{definition}[V-set intersection]
\label{def:vset-intersect}
The \emph{intersection} of two v-sets is a v-set of their shared elements
annotated with the conjunction of their presence conditions, i.e., 
\ensuremath{
\vset_1 \cap \vset_2 = \setDef {
\annot [\dimMeta_1 \wedge \dimMeta_2 ]\elem \myOR
\annot [\dimMeta_1] \elem \in \vset_1, \annot [\dimMeta_2] \elem \in \vset_2,
\sat {\dimMeta_1 \wedge \dimMeta_2}
}
}.
%For example, \ensuremath{
%\setDef {2, \annot [\A] 3, \annot [\neg \B] 4} \cap
%\pushIn {\annot [\B] {\setDef{2,3,4,5}}} =
%\setDef{\annot [\B] 2, \annot [\A \wedge \B] 3}
%}.
\end{definition}

\begin{definition} [V-set cross product]
\label{def:vset-cross}
The \emph{cross product} of two v-sets is a pair of every two elements of 
them annotated with the conjunction of their presence conditions.
\ensuremath{
\vset_1 \times \vset_2 = \setDef{
\annot [\dimMeta_1 \wedge \dimMeta_2] {(\elem_1, \elem_2)} \myOR
\annot [\dimMeta_1] \elem_1 \in \vset_1, \annot [\dimMeta_2] \elem_2 \in \vset_2
%\vset_1 \cap \vset_2 = \setDef \
}
}
%
\end{definition}

\begin{definition} [V-set equivalence]
\label{def:vset-eq}
%\revised{Two v-sets are \emph{equivalent}, denoted by
%\ensuremath{\vset_1 \equiv \vset_2}, iff
%\ensuremath{\forall \config \in \confSet.
%%\forall \annot \elem \in  (\vset_1 \cup \vset_2).
%\elemSem {\vset_1} \equiv \elemSem {\vset_2}},
%where \ensuremath{\elemSem {.} : \vsetSet \to \confSet \to \psetSet}
%is a configuration function for variational sets. 
%Thus, two v-sets are equivalent iff all their configured sets are equivalent.
%}
%\TODO{Two v-sets are \emph{equivalent}, denoted by
%\ensuremath{\vset_1 \equiv \vset_2}, iff
%\ensuremath{
%\forall \annot  \elem \in (\vset_1 \cup \vset_2),
%\annot [\dimMeta_1] \elem \in \vset_1, \annot [\dimMeta_2] \elem \in \vset_2.
%\dimMeta_1 \equiv \dimMeta_2, \dimMeta \equiv \dimMeta_1},
%this fails b/c {1 fls == 2 fls}
%i.e., they both cover the same set of elements and the presence conditions
%of elements from the two v-sets are equivalent.}
Two v-sets are \emph{equivalent}, denoted by
\ensuremath{\vset_1 \equiv \vset_2}, iff
\ensuremath{\forall \annot  \elem \in (\filterSet {\vset_1} \cup \filterSet {\vset_2}),
\annot [\dimMeta_1] \elem \in \filterSet {\vset_1}, \annot [\dimMeta_2] \elem \in \filterSet {\vset_2}.
\dimMeta_1 \equiv \dimMeta_2, \dimMeta \equiv \dimMeta_1},
i.e., they both cover the same set of elements and the presence conditions
of elements from the two normalized v-sets are equivalent.
\end{definition}

%
%\begin{definition} [V-set subsumption]
%\label{def:vset-subsumption}
%The v-set \ensuremath{\vset_1} \emph {subsumes} the v-set
%\ensuremath{\vset_2}, $\subsume {\vset_2} {\vset_1}$, iff
%\ensuremath{ \forall \annot [\dimMeta_2] \elem \in \pushIn {\vset_2}.
%\annot [\dimMeta_1] \elem \in \pushIn {\vset_1}, 
%%\neg \sat {\dimMeta_2 \wedge \neg \dimMeta_1}
%\sat {\dimMeta_2 \wedge  \dimMeta_1}
%},
%i.e., all elements in $\vset_2$ also exist in $\vset_1$ 
%s.t. the element is valid in a shared configuration between the v-sets.
%For example, 
%\ensuremath{
% \subsume {\pushIn {\annot [\A] {\setDef {2,3}}}} {\setDef {2, \annot [\A \vee \B] 3, 4}}},
%however, 
%\ensuremath{
% \nsubsume {\pushIn {\annot [\A] {\setDef {2,3}}}} {\setDef {2, \annot [\neg \A \wedge \B] 3}}}
%and
%\ensuremath{
%\nsubsume {\setDef {\annot [\A] 2,\annot [\A] 3, 4}} {\setDef {2, \annot [\A \wedge \B] 3}}}.
%\end{definition}

%\begin{definition} [V-set explicit subsumption]
%\dropit{drop this for vldb submission. remember you need it for popl.}
%\label{def:vset-strict-subsumption}
%The v-set \ensuremath{\vset_1} \emph {explicitly subsumes} the v-set
%\ensuremath{\vset_2}, $\subsumeExpl {\vset_2} {\vset_1}$, iff
%\ensuremath{ \forall \annot [\dimMeta_2] \elem \in \vset_2.
%\annot [\dimMeta_1] \elem \in \vset_1, 
%\neg \sat {\dimMeta_2 \wedge \neg \dimMeta_1}
%},
%i.e., all elements in $\vset_2$ also exist in $\vset_1$ 
%s.t. its presence condition in \ensuremath{\vset_2} is more specific than 
%its presence condition in \ensuremath{\vset_1}, captured by 
%\ensuremath{\neg \sat {\dimMeta_2 \wedge \neg \dimMeta_1}}
%which could also be defined as 
%\ensuremath{
%\nexists \config \in \confSet . \fSem {\dimMeta_1} = \t , \fSem {\dimMeta_2} = \f.
%%i.e. in set theory:
%% \dimMeta_2 \subset \dimMeta_1
%%\dimMeta_2 - \dimMeta_1 = \emptyset
%%i.e.
%%\dimMeta_2 \cap \bar{\dimMeta_1} = \emptyset 
%}
%\end{definition}






\subsection{Variational Set Configuration}
\label{sec:vsetconf}

\TODO{vset configuration.}




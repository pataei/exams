\section{Experiments and Discussion}
\label{sec:exp}

%what are the approaches.
%what machine we used and version of postgres. 
%future work compare different db engines. 

In this section, we compare the performance of VDBMS with regards
to the approaches used to generate SQL query(ies) introduced in \secref{apps}.
For our experiments, approaches introduced in \secref{apps} do not filter out 
tuples with unsatisfiable presence condition unless specifically mentioned otherwise.
Accounting for filtering such tuples explicitly is indicated by adding \emph{(f)} to
the approach name. For example, \nbff\ is the Naive Brute Force approach that 
filters out tuples with unsatisfiable presence conditions. 
%
The variational queries used for our experiments are available online
%
\footnote{All queries are available at: \url{https://zenodo.org/record/4321921}.}
% in \appref{queries}
and they are described in \chref{vdbusecase}.
%
The runtime of queries contains all elements of an approach including the query
passing the type system, being explicitly annotated by the variational schema, 
being optimized by the variation minimization rules, passed to the SQL generator, 
run on the VDB, and finally, the final variational table is generated. 
%
We run the experiments on a MacBook Pro with a 2.4 GHz Core i7 processor and 
8 GB 1600 MHz DDR3. All experiments are also run with the PostgreSQL 13.3 as 
the database engine. 

We conduct three main 
comparison of the approaches over our two use cases. The first one 
compares the five approaches with each other (\secref{exp-gen}). The second one
investigates the effect of number of (unique) variants on the approaches (\secref{exp-vars}). 
The last one investigates the effect filtering out
tuples with unsatisfiable presence conditions (\secref{exp-tuples}). 

\section{Analysis of Different SQL Generators}
\label{sec:exp-gen}


%\begin{wrapfigure}{r}{0.5\linewidth}
%\centering
%\rule{0.9\linewidth}{0.75\linewidth}
%\caption{Dummy figure.}
%\label{fig:myfig}
%\end{wrapfigure}
%\blindtext

\begin{figure}
\centering
\includegraphics[scale=0.12] {figs/plots/emp1-5.png}
\caption[Comparison of SQL generators \nbf, \nbfi, \uav, \ubf, and \ubfi\ over the employee VDB]{Comparison of SQL generators \nbf, \nbfi, \uav, \ubf, and \ubfi\ over the employee VDB}
\label{fig:emp1-5}
\end{figure}



\figref{emp1-5} and \figref{enron1-5} show the runtime for each query which is shown on top the plot
for the five approaches introduced in \secref{apps} over the employee and email VDBs, respectively.
%
\figref{emp1-5} implies that  \nbf\ mostly has a better
runtime than \nbfi\ and similarly \ubf\ mostly has a better runtime than \ubfi\ for this data set. However, 
\nbf, \uav, and \ubf\ have a close performance to each other but neither consistently performs 
better than the others. 
%
%Since the most of the variational queries of the employee VDB return a small table, showed in \figref{},
%this dataset provides us with little insight. Still, \figref{emp1-5} shows that \nbf\ mostly has a better
%runtime than \nbfi\ and similarly \ubf\ mostly has a better runtime than \ubfi. This insight is confirmed 
%by \figref{enron1-5} which shows the runtime of the email VDB's queries. 
%
\figref{enron1-5} also implies that \nbf\ consistently has a better
runtime than \nbfi\ and \ubf\ mostly has a better runtime than \ubfi\ for this data set. And while
\uav\ mostly performs better than \nbf\ it is mainly comparable to \ubf\ for this data set. Yet, \uav\
sometimes generates a non-runnable SQL query, showed by the striped bars in \figref{enron1-5}.
\exref{uav-fail} explains this in detail. 


\begin{figure}[!t]
\centering
\includegraphics[width = \linewidth] {figs/plots/enron1-5.png}
\caption[Comparison of SQL generators \nbf, \nbfi, \uav, \ubf, and \ubfi\ over the email VDB]{Comparison of SQL generators \nbf, \nbfi, \uav, \ubf, and \ubfi\ over the email VDB}
\label{fig:enron1-5}
\end{figure}


%
Based on our experiments, the query construction (from type system to generating SQL queries) takes
similar time between the approaches. Their main difference comes down to the gross runtime
of queries on the VDB and building the returning variational table. 
%
\uav\ does not take any time to build a variational table since the result already have the
desired schema and presence conditions, however, it spends more time on running the 
SQL query since queries generated by \uav\ are usually more complicated. 
%
On the other hand, although \nbf\ and \ubf\ run multiple SQL queries per variational query
their generated SQL queries are simpler than the ones generated by \uav. However,
as opposed to \uav\ they have to adjust the returned table for each SQL query and apply
the correct presence condition to the tuples. 
%
Finally, the main difference between the performance of  \nbf\ and \nbfi\ (and similarly, \ubf\ and \ubfi)
is where they apply the correct presence condition to the tuples. While \nbfi\ and \ubfi\ pass this
task to the underlying database engine the \nbf\ and \ubf\ approaches load this to the Haskell 
backend. Note that all four of these approaches still have to fix the schema of the returned
tables to the variational table schema of the variational query. 

\begin{example}
\label{eg:uav-fail}
In this example we explain the reason why the SQL query generated by the \uav\ approach
sometimes cannot be run. PostgreSQL forces the type of an attribute \vAtt\ that is projected as \nul\
(that is, \texttt{NULL as \vAtt}) to be a string. Thus, using the union operation between 
subqueries when \vAtt\ has a different type causes an error. Assume the following query is 
generated by the \uav\ approach:
%
\begin{lstlisting}[basicstyle=\footnotesize\ttfamily,columns=flexible,lineskip=0.5\baselineskip]
(SELECT body,
         NULL AS is_encrypted
 FROM messages)
UNION ALL
(SELECT body,
         is_encrypted
 FROM messages)
\end{lstlisting}
%
PostgreSQL forces \isencrypted\ to have the type string while in the second subquery it assumes the boolean 
type for \isencrypted\ since that is its defined type in the  database. This causes a conflict in the assumption
that subqueries of a query that uses the union operation must have the same schema and their 
attributes must have the same type.
\end{example}

\secref{exp-vars} sheds light on the impact of number of (unique) variants on each approach and
\secref{exp-tuples} explores the effect of the number of returned tuples on our approaches. 


\section{The Effect of Number of Variants on SQL Generators}
\label{sec:exp-vars}


%\begin{wrapfigure}{r}{0.5\linewidth}
%\centering
%\rule{0.9\linewidth}{0.75\linewidth}
%\caption{Dummy figure.}
%\label{fig:myfig}
%\end{wrapfigure}
%\blindtext


\begin{figure}
\centering
\includegraphics[scale = 0.1] {figs/plots/emp-comp-var.png}
\caption[blah]{blah}
\label{fig:-}
\end{figure}

\begin{figure}
\centering
\includegraphics[scale=0.1] {figs/plots/enron-comp-var.png}
\caption[blah]{blah}
\label{fig:-}
\end{figure}


\begin{figure}
\begin{subfigure}{.5\linewidth}
\centering
\includegraphics[width=.7\textwidth]{figs/plots/emp-nbf-ubf.png}
\caption{}
\label{fig:sub1}
\end{subfigure}%
\begin{subfigure}{.5\linewidth}
\centering
\includegraphics[width=.7\textwidth]{figs/plots/emp-nbfi-ubfi.png}
\caption{}
\label{fig:sub2}
\end{subfigure}\\[1ex]
\begin{subfigure}{0.5\linewidth}
\centering
\includegraphics[width=.7\textwidth]{figs/plots/emp-nbff-ubff.png}
\caption{}
\label{fig:sub3}
\end{subfigure}
\begin{subfigure}{0.5\linewidth}
\centering
\includegraphics[width=.7\textwidth]{figs/plots/emp-nbfif-ubfif.png}
\caption{}
\label{fig:sub4}
\end{subfigure}
\caption{Three subfigures}
\label{fig:test}
\end{figure}

\begin{figure*}[t!]
    \centering
    \begin{subfigure}[t]{0.5\textwidth}
        \centering
        \includegraphics[scale=0.09]{figs/plots/enron-nbf-ubf.png}
        \caption{Lorem ipsum}
    \end{subfigure}%
    ~ 
    \begin{subfigure}[t]{0.5\textwidth}
        \centering
        \includegraphics[scale=0.09]{figs/plots/enron-nbfi-ubfi.png}
        \caption{Lorem ipsum, lorem ipsum,Lorem ipsum, lorem ipsum,Lorem ipsum}
    \end{subfigure}
    \caption{Caption place holder}
\end{figure*}


\subsection{The Effect of Filtering Invalid Tuples}
\label{sec:exp-tuples}

In this section, we explore the effect of filtering out invalid tuples, that is, tuples with
an unsatisfiable presence condition. \figref{enron-nbfs-filter} and \figref{emp-nbfs-filter}
illustrate that filtering out invalid tuples increases the runtime of queries significantly. 
%
This increase is very significant for the \ubf, \ubfi, and \uav\ approaches compared to
\nbf\ and \nbfi\footnote{In our experiments these approaches took longer than 30 minutes for 
all queries from both data sets except for the ones showed in \figref{emp-ubff-ubff}.
%Thus, we did not include
}
since the former approaches check the satisfiability of tuples' 
presence conditions while the latter applies the configuration to a tuple's 
presence condition and if it returns false the tuple is dropped. Thus,
the calls to the SAT solver are more expensive than applying the configuration.
%
Furthermore, \figref{enron-nbfi-tuple}, \figref{enron-nbf-tuple}, 
\figref{emp-nbfi-tuple}, and \figref{emp-nbf-tuple} illustrate that increasing the number of 
returned tuples reduces the performance of the \nbf\ and \nbfi\ approaches. 
%
A possible solution would be to either use an incremental SAT solver since most of the
SAT problems have lots of common parts or to cluster the tuples based on their presence
conditions and send unique presence conditions to the SAT solver. 

Finally, we explore the performance of approaches that filter out invalid tuples. 
As shown in \figref{filter-comp}, there is not a clear ranking of
the performance of the approaches that filter out invalid tuples and there is also
not a clear connection as the number of tuples increases. This is due to the fact that 
the complexity of presence conditions also plays a role in the runtime of these approaches. 

\begin{figure*}[t!]
    \centering
    \begin{subfigure}[t]{0.5\textwidth}
        \centering
        \includegraphics[width=\textwidth]{figs/plots/enron-nbfi-comp-f.png}
        \caption[The \nbfi\ approach versus \nbfif]{The \nbfi\ approach versus \nbfif.}
    \end{subfigure}%
    ~ 
    \begin{subfigure}[t]{0.5\textwidth}
        \centering
        \includegraphics[scale=0.09]{figs/plots/enron-nbfi-f-comp-scatter.png}
        \caption[The \nbfi\ approach versus \nbfif\ as the number of returned tuples increases]{The \nbfi\ approach versus \nbfif\ as the number of returned tuples increases.}
        \label{fig:enron-nbfi-tuple}
    \end{subfigure}\\[1ex]
     \begin{subfigure}[t]{0.5\textwidth}
        \centering
        \includegraphics[width=\textwidth]{figs/plots/enron-nbf-comp-f.png}
        \caption[The \nbf\ approach versus \nbff]{The \nbf\ approach versus \nbff.}
    \end{subfigure}%
    ~ 
    \begin{subfigure}[t]{0.5\textwidth}
        \centering
        \includegraphics[scale=0.09]{figs/plots/enron-nbf-f-comp-scatter.png}
        \caption[The \nbf\ approach versus \nbff\ as the number of returned tuples increases]{The \nbf\ approach versus \nbff\ as the number of returned tuples increases.}
        \label{fig:enron-nbf-tuple}
    \end{subfigure}
    \caption[The effect of filtering out tuples with unsatisfiable presence conditions on SQL generator 
    approaches over the email VDB]{The effect of filtering out tuples with unsatisfiable presence conditions on SQL generator 
    approaches over the email VDB.}
    \label{fig:enron-nbfs-filter}
\end{figure*}

%\begin{figure*}[t!]
%    \centering
%    \begin{subfigure}[t]{0.5\textwidth}
%        \centering
%        \includegraphics[width=\textwidth]{figs/plots/enron-nbf-f.png}
%        \caption{Lorem ipsum}
%    \end{subfigure}%
%    ~ 
%    \begin{subfigure}[t]{0.5\textwidth}
%        \centering
%        \includegraphics[scale=0.09]{figs/plots/enron-nbf-f-scatter.png}
%        \caption{Lorem ipsum, lorem ipsum,Lorem ipsum, lorem ipsum,Lorem ipsum}
%    \end{subfigure}
%    \caption{Caption place holder}
%\end{figure*}

%\begin{figure*}[t!]
%    \centering
%    \begin{subfigure}[t]{0.5\textwidth}
%        \centering
%        \includegraphics[width=\textwidth]{figs/plots/enron-nbf-comp-f.png}
%        \caption{Lorem ipsum}
%    \end{subfigure}%
%    ~ 
%    \begin{subfigure}[t]{0.5\textwidth}
%        \centering
%        \includegraphics[scale=0.09]{figs/plots/enron-nbf-f-comp-scatter.png}
%        \caption{Lorem ipsum, lorem ipsum,Lorem ipsum, lorem ipsum,Lorem ipsum}
%    \end{subfigure}
%    \caption{Caption place holder}
%\end{figure*}


\begin{figure*}[t!]
    \centering
    \begin{subfigure}[t]{0.5\textwidth}
        \centering
        \includegraphics[scale=0.09]{figs/plots/emp-nbfi-comp-f.png}
        \caption[The \nbfi\ approach versus \nbfif]{The \nbfi\ approach versus \nbfif.}
    \end{subfigure}%
    ~ 
    \begin{subfigure}[t]{0.5\textwidth}
        \centering
        \includegraphics[scale=0.09]{figs/plots/emp-nbfi-f-comp-scatter.png}
        \caption[The \nbfi\ approach versus \nbfif\ as the number of returned tuples increases]{The \nbfi\ approach versus \nbfif\ as the number of returned tuples increases.}
        \label{fig:emp-nbfi-tuple}
    \end{subfigure}\\[1ex]
    \begin{subfigure}[t]{0.5\textwidth}
        \centering
        \includegraphics[scale=0.09]{figs/plots/emp-nbf-comp-f.png}
        \caption[The \nbf\ approach versus \nbff]{The \nbf\ approach versus \nbff.}
    \end{subfigure}%
    ~ 
    \begin{subfigure}[t]{0.5\textwidth}
        \centering
        \includegraphics[scale=0.09]{figs/plots/emp-nbf-f-comp-scatter.png}
        \caption[The \nbf\ approach versus \nbff\ as the number of returned tuples increases]{The \nbf\ approach versus \nbff\ as the number of returned tuples increases.}
        \label{fig:emp-nbf-tuple}
    \end{subfigure}    
    \caption[The effect of filtering out tuples with unsatisfiable presence conditions on SQL generator 
    approaches over the employee VDB]{The effect of filtering out tuples with unsatisfiable presence conditions on SQL generator 
    approaches over the employee VDB.}
    \label{fig:emp-nbfs-filter}
\end{figure*}

%\begin{figure*}[t!]
%    \centering
%    \begin{subfigure}[t]{0.5\textwidth}
%        \centering
%        \includegraphics[scale=0.09]{figs/plots/emp-nbf-f.png}
%        \caption{Lorem ipsum}
%    \end{subfigure}%
%    ~ 
%    \begin{subfigure}[t]{0.5\textwidth}
%        \centering
%        \includegraphics[scale=0.09]{figs/plots/emp-nbf-f-scatter.png}
%        \caption{Lorem ipsum, lorem ipsum,Lorem ipsum, lorem ipsum,Lorem ipsum}
%    \end{subfigure}
%    \caption{Caption place holder}
%\end{figure*}

%\begin{figure*}[t!]
%    \centering
%    \begin{subfigure}[t]{0.5\textwidth}
%        \centering
%        \includegraphics[scale=0.09]{figs/plots/emp-nbf-comp-f.png}
%        \caption{Lorem ipsum}
%    \end{subfigure}%
%    ~ 
%    \begin{subfigure}[t]{0.5\textwidth}
%        \centering
%        \includegraphics[scale=0.09]{figs/plots/emp-nbf-f-comp-scatter.png}
%        \caption{Lorem ipsum, lorem ipsum,Lorem ipsum, lorem ipsum,Lorem ipsum}
%    \end{subfigure}
%    \caption{Caption place holder}
%\end{figure*}
\begin{figure*}[t!]
    \centering
    \begin{subfigure}[t]{0.5\textwidth}
        \centering
        \includegraphics[scale=0.06]{figs/plots/emp-nbf-f.png}
        \caption[Comparison of the \nbff\ and \nbfif\ approaches on the employee VDB]{Comparison of the \nbff\ and \nbfif\ approaches on the employee VDB.}
    \end{subfigure}%
    ~ 
    \begin{subfigure}[t]{0.5\textwidth}
        \centering
        \includegraphics[scale=0.06]{figs/plots/emp-nbf-f-scatter.png}
        \caption[Comparison of the \nbff\ and \nbfif\ approaches on the employee VDB]{Comparison of the \nbff\ and \nbfif\ approaches on the employee VDB.}
    \end{subfigure}\\[1 ex]
    \begin{subfigure}[t]{0.5\textwidth}
        \centering
        \includegraphics[scale=0.06]{figs/plots/emp-ubf-f.png}
        \caption[Comparison of the \ubff\ and \ubfif\ approaches on the employee VDB]{Comparison of the \ubff\ and \ubfif\ approaches on the employee VDB}
    \end{subfigure}%
    ~ 
    \begin{subfigure}[t]{0.5\textwidth}
        \centering
        \includegraphics[scale=0.06]{figs/plots/emp-ubf-f-scatter.png}
        \caption[Comparison of the \ubff\ and \ubfif\ approaches on the employee VDB]{Comparison of the \ubff\ and \ubfif\ approaches on the employee VDB}
    \end{subfigure}\\[1 ex]
    \begin{subfigure}[t]{0.5\textwidth}
        \centering
        \includegraphics[width=\textwidth]{figs/plots/enron-nbf-f.png}
        \caption[Comparison of the \nbff\ and \nbfif\ approaches on the email VDB]{Comparison of the \nbff\ and \nbfif\ approaches on the email VDB}
    \end{subfigure}%
    ~ 
    \begin{subfigure}[t]{0.5\textwidth}
        \centering
        \includegraphics[scale=0.07]{figs/plots/enron-nbf-f-scatter.png}
        \caption[Comparison of the \nbff\ and \nbfif\ approaches on the email VDB]{Comparison of the \nbff\ and \nbfif\ approaches on the email VDB}
    \end{subfigure}
    \caption[The effect of filtering out tuples with unsatisfiable presence conditions when presence conditions are injected into a query in SQL generator approaches]{The effect of filtering out tuples with unsatisfiable presence conditions when presence conditions are injected into a query in  on SQL generator approaches.}
    \label{fig:filter-comp}
\end{figure*}
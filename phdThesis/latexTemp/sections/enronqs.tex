%\vspace{-7.5pt}
\subsection{Email Query Set}
\label{sec:enron-qs}

%\point{We write queries based on SPL developers needs when different SPL features
%interact with each other~\cite{blah}.}\\
%- give an example.

To produce a set of queries for the email SPL use case, we collected all of
the information needs that we could identify in the description of the email
SPL by \citet{Hall05}.. In order to make the information needs more concrete, we
viewed the requirements of the email SPL mostly through the lens of
constructing an email header.
%
An email header includes all of the relevant information needed to send an
email and is used by email systems and clients to ensure that an email is sent to
the right place and interpreted correctly. 
More specifically, the email header
includes the sender and receiver of the email, whether an email is signed and
the location of a signature verification key, whether an email is encrypted and
the location of the corresponding public key, the subject and body of the
email, the mail host it belongs to, whether the email should be filtered,
% the verification status of the email and who has verified the email, 
and so on.
%
Although there is obviously other infrastructure involved, the fundamental
information needs of an email system can be understood by considering how to
construct email headers
 that ensures the email would get where it needs to go
 and be interpreted correctly on the other end.


Hall's decomposition focuses on enumerating the features of the email SPL and
enumerating the potential interactions of those features.
%
We deduce the information need for each feature by asking: ``what information
is needed to modify the email header in a way that incorporates the new
functionality?''. We deduce the information need for each interaction by
asking: ``what information is needed to modify the email header in a way that
avoids the undesirable feature interaction?''.
%
We can then translate these information needs into queries on the underlying
variational database.


In total, we provide $27$ queries for the email SPL.
%, encoded in both VRA
%format usable by our VDBMS tool and as \cpp{ifdef} annotated SQL, as described
%above.%
%
%\footnote{\href{https://github.com/lambda-land/VDBMS/tree/master/usecases/space-emailSPL/queries}{usecases/space-emailSPL/queries}}
%
This consists of $1$ query for constructing the basic email header, $8$ queries
for realizing the information needs corresponding to each feature, and $18$
queries for realizing the information needs to correctly handle the feature
interactions described by Hall.


We start by presenting the query to assemble the basic email header, \Qbasic.
This corresponds to the information need of a system with no features enabled.
We use $X$ to stand for the specific message ID (\midatt) of the email whose
header we want to construct.
%
%\vspace{-2pt}
\begin{align*}
\Qbasic &= 
%\pi_{\sender,\rvalue,\subject,\body} 
%((\selectRel\midCond\messages)\\
%&\hspace{20pt} \bowtie_{\messages.\midatt = \recipientinfo.\midatt} \recipientinfo)
\projectRel{\sender,\rvalue,\subject,\body} \mrtable \\
\mrtable &\leftarrow (\selectRel\midCond\messages) \bowtie \recipientinfo
%\mrtable &\leftarrow (\selectRel\midCond\messages) \\
%&\qquad \bowtie_{\messages.\midatt = \recipientinfo.\midatt} \recipientinfo
\end{align*}
%\vspace{-13pt}
%
This query extracts the sender, recipient, subject, and body of the email to
populate the header. The projection is applied to an intermediate result
\mrtable\ constructed by joining the \messages\ table with the \recipientinfo\
table on recipient IDs; we reuse this intermediate result also in subsequent
queries.


Taking \Qbasic\ as our starting point, we next construct our set of $8$
\emph{single-feature queries} that capture the information needs specific to
each feature.
%
When a feature is enabled in the SPL, more information is needed to construct
the header of email $X$. For example, if the feature \filtermessages\ is
enabled, then the query \Qfilter\ extends \Qbasic\ with the \suffix\ attribute
used in filtering. This additional information allows the system to filter a
message if its address contains any of the suffixes set by the receiver.
%
%% Since this queries involve features that are optional, these
%% queries are variational and correspond to multiple different plain queries.
%% However, as described in \secref{background}...
%
%\vspace{-2pt}
\begin{align*}
\Qfilter &= 
%\pi_{\sender,\rvalue,\suffix,\subject,\body}
%((\employees \\
%&\bowtie_{\rvalue=\emailid}
% \mrtable) \bowtie\filtermsg )
%%_{\employees.\eid = \filtermsg.\eid}\filtermsg )\\
%% \bowtie_{\rvalue=\emailid}{\employees}\\
\projectRel{\sender,\rvalue,\suffix,\subject,\body} \temp \\
\temp & \leftarrow \mretable
 \bowtie \filtermsg \\
%&\qquad \bowtie_{\employees.\eid = \filtermsg.\eid}\filtermsg \\
\mretable &\leftarrow \joinRel{\mrtable}{\employees}{\rvalue=\emailid}
\end{align*}
%\vspace{-1pt}
%
%The intermediate result \mretable\ formed by joining \mrtable\ with the
%\employees\ relation will be reused in later queries.
%
We can construct a query that retrieves the required header information whether
\filtermessages\ is enabled or not by combining \Qbasic\ and \Qfilter\ in a
choice, as $\Qbf=\chc[\filtermessages]{\Qfilter,\Qbasic}$. 
%
Although we do not show the process in this paper, we can use equivalence laws
from the choice calculus~\cite{EW11tosem,HW16fosd} to factor commonalities out
of choices and reduce redundancy in queries like \Qbf.
% For clarity, we distribute the un-factored form presented here in our case
% studies.
The other single-feature queries are written similarly.
% to the one shown here.


% \centerline{$\choiceS \forwardmessages {\Qforward} {\Qbasic}$}

As another example of a single-feature query, \Qforward\ captures the
information needs for implementing the \forwardmessages\ feature. It is similar
to the previous queries except that it extracts the \forwardaddr\ from the
\automsg\ table, which is needed to construct the message header for the new
email to be forwarded when email $X$ is received by a user with a \forwardaddr\
set.
%
\begin{align*}
\Qforward &=
\projectRel{\rvalue,\forwardaddr,\subject,\body}{\temp} \\
\temp &\leftarrow \mretable \\
&\qquad \bowtie_{\employees.\eid = \forwardmsg.\eid} \automsg
\end{align*}

The other single-feature queries are similar to those shown here.


Besides single-feature queries, we also provide queries that gather information
needed to identify and address the undesirable feature interactions described
by \citet{Hall05}. Out of Hall's 27 feature interactions, we determined 16 of
them to have corresponding information needs related to the database; 2 of the
interactions require 2 separate queries to resolve. Therefore, we define and
provide 18 queries addressing all 16 of the relevant feature interactions.
%
As before, we deduced the information needs through the lens of constructing an
email header; in these cases, the header would correspond to an email produced
after successfully resolving the interaction.
%
However, some interactions can only be detected but not automatically resolved.
In these cases, we constructed a query that would retrieve the relevant
information to detect and report the issue.


One undesirable feature interaction occurs between the \signature\ and
\forwardmessages\ features: if Philippe signs a message and sends it to Sarah,
and Sarah forwards the message to an alternate address Sarah-2, then signature
verification may incorrectly interpret Sarah as the sender rather than Philippe
and fail to verify the message (Hall's interaction \#4).
%
A solution to this interaction is to embed the original sender's verification
information into the email header of the forwarded message so that it can be
used to verify the message, rather than relying solely on the message's
``from'' field.


Below, we show a variational query \Qsf\ that includes four variants
corresponding to whether \signature\ and \forwardmessages\ are enabled or not
independently. The information need for resolving the interaction is satisfied
by the first alternative of the outermost choice with condition
$\signature\wedge\forwardmessages$. The alternatives of the choices nested to
the right satisfy the information needs for when only \signature\ is enabled,
only \forwardmessages\ is enabled, or neither is enabled (\Qbasic). We don't
show the single-feature \Qsig\ query, but it is similar to other
single-feature queries shown above.
%
\begin{align*}
\Qsf &=  \signature\wedge\forwardmessages \\
&\quad\langle
  \vPrj[\rvalue,\forwardaddr,\mathit{emp1}.\issigned,\mathit{emp1}.\verificationkey]{\temp}, \\
&\quad\signature\langle\Qsig, 
\chc[\forwardmessages]{\Qforward,\Qbasic} \rangle\rangle \\
\temp &\leftarrow
  ((((\sigma_{\midCond} \messages) 
 \bowtie
%_{\messages.\midatt = \recipientinfo.\midatt}
 \recipientinfo) \\
& \bowtie_{\sender = \mathit{emp1}.\emailid}
    (\vRen[\mathit{emp1}]{\employees})) \\
& \bowtie_{\rvalue = \mathit{emp2}.\emailid}
    (\vRen[\mathit{emp2}]{\employees})) \bowtie \forwardmsg
%\temp &\leftarrow
%  ((((\sigma_{\midCond} \messages) \\
%& \bowtie_{\messages.\midatt = \recipientinfo.\midatt} \recipientinfo) \\
%& \bowtie_{\messages.\sender = \mathit{emp1}.\emailid}
%    (\vRen[\mathit{emp1}]{\employees})) \\
%& \bowtie_{\recipientinfo.\rvalue = \mathit{emp2}.\emailid}
%    (\vRen[\mathit{emp2}]{\employees}))\\
%& \bowtie_{\mathit{emp2}.\eid = \forwardmsg.\eid} \forwardmsg
\end{align*}
%
 The query $\Qsf$ also resolves another consequence of the interaction between
 these two features. This time Sam successfully verifies message $X$ and
 forwards it to Sam2 which changes the header in the system such that it states
 message $X$ has been successfully verified, thus, the message could be
 altered by hackers while it is being forwarded (Hall's interaction \#27). The
 system can use $\Qsf$ to generate the correct header in this scenario again.

%\begin{comment}
\noindent
%
Some feature interactions require more than one query to satisfy their
information need.
% due to VRA's limitation that values cannot be variational.
For example, assume both \encryption\ and \forwardmessages\ are enabled.
Philippe sends an encrypted email $X$ to Sarah; upon receiving it the message
is decrypted and forwarded it to Sarah-2 (Hall's interaction \#9). This
violates the intention of encrypting the message and the system should warn the
user.
%
Queries \Qef\ and $\Qef'$ satisfy the information need for this interaction
when a message is encrypted or unencrypted, respectively.
%
\begin{align*}
\small
\Qef &= \encryption\wedge\forwardmessages \\
  &\langle
    \projectRel{\rvalue}{(\selectRel{\midCond\wedge\isencrypted}{\messages})},
    \encryption\langle\Qencrypt, \\
  &\qquad \chc[\forwardmessages]{\Qforward,\Qbasic} \rangle \\
\OB{\Qef'} &= \encryption\wedge\forwardmessages
  \langle \temp, \encryption \langle \Qencrypt, \\
  &\qquad \chc[\forwardmessages]{\Qforward,\Qbasic} \rangle \rangle \\
% \end{align*}
% %
% \begin{align*}
\temp &\leftarrow
  \pi_{\rvalue,\forwardaddr,\subject,\body} (\sigma_{\midCond \wedge \neg\isencrypted} \\
  & (\mretable \bowtie_{\employees.\eid = \forwardmsg.\eid} \forwardmsg))
\end{align*}

\noindent
%
However, managing feature interactions is not necessarily complicated. Some
interactions simply require projecting more attributes from the corresponding
single-feature queries. For example, assume both \filtermessages\ and
\mailhost\ features are enabled. Philippe sends a message to a non-existant
user in a mailhost that he has filtered. The mailhost generates a non-delivery
notification and sends it to Philippe, but he never receives it since it is filtered out
(Hall's interaction \#26). The system can check the \issystemnotification\
attribute for the $\Qfilter$ query and decide whether to filter a message or
not. Therefore, we can resolve this interaction by extending the single-feature
query for \filtermessages\ to $\Qfilter'$.
%
\begin{align*}
\small
\OB{\Qfilter'} &= 
\projectRel {\sender, \rvalue, \suffix, \issystemnotification, \subject, \body} \temp\\
\temp &\leftarrow 
{\joinRel \mretable
\filtermsg
{\employees . \eid = \filtermsg . \eid }
}
\end{align*}
%\end{comment}
\noindent
%
Overall, for the 18 interaction queries we provide, 12 have 4 variants, 3 have
3 variants, 2 have 2 variants, and 1 has 1 variant.

%%\input{formulas/enronQs}

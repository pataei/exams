\section{VRA Type System}
\label{sec:type-sys}

%\begin{figure}
%\begin{minipage}[t]{0.5\textwidth}
\textbf{V-queries typing rules:}

  \begin{mathpar}
  \small
  
  \inferrule[\empRelE]
  {}
  {\env {\empRel} {\annot [\f] {\setDef \ }}}
%  \inferrule[\judge]
 % 	{\env{\vQ}{\vType}}
 %   {}
%
% explicitly-typed vra:
%    \inferrule[\relationE]
%  	{\vRel (\vType)^{\VVal \dimMeta} \in \vSch \\
%	\neg \sat{\vctx \wedge \neg \VVal \dimMeta} }
%     {\envWithSchema{\envInContext [\vctx ] {\vType}}}

%implicitly-typed lang:
    \inferrule[\relationE]
  	{ \vRel (\vType)^{\VVal \dimMeta} \in \vSch \\
	\sat {\vctx \wedge \getPCfrom \vRelSch \vSch}}
%	\sat{\vctx \wedge \VVal \dimMeta} }
     {\envWithSchema{\envInContext [\vctx \wedge \VVal \dimMeta] {\vType}}}

% explicitly-typed vra:  
%  \inferrule[\prjE]
%  	{\envPrime \\
%    	\subsume {\annot \vType}  {\annot [\VVal \vctx] {\VVal \vType}}}
%    {\env{\vPrj[\vType] \vQ} {\envInContext [\vctx] \vType}}

%implicitly-typed lang:
  \inferrule[\prjE]
  	{\envPrime \\
	|\pushIn {\annot \vType}| = | \vType | \\
%	\pushIn {\annot \vType} \neq \setDef \ \\
%	\pushIn {\annot [\VVal \dimMeta] {\VVal \vType}} \neq \setDef \ \\
    	\subsume { \vType}  {\pushIn {\annot [\VVal \vctx] {\VVal \vType}}}}
    {\env{\vPrj[\vType] \vQ} {\envInContext [\VVal \vctx] {\left(\vType \cap {\VVal \vType} \right)}}}
%    the older version 8/8/20:
%    	\subsume {\pushIn {\annot \vType}}  {\pushIn {\annot [\VVal \vctx] {\VVal \vType}}}}
%    {\env{\vPrj[\vType] \vQ} {\envInContext [\VVal \vctx] {\left(\pushIn{\annot {\vType}} \cap {\VVal \vType} \right)}}}


  \inferrule[\selE]
  	{\env \vQ {\envInContext [\VVal \vctx] \vType} \\
    	\envCondAnnot \vCond}
    {\env{\vSel \vQ}{\envInContext [\VVal \vctx] \vType}}
    
  \inferrule[\choiceE]
  	{\envOne[\vctx \wedge \VVal \dimMeta] \\
    	\envTwo[\vctx \wedge \neg \VVal \dimMeta]}
    {\env{\chc[\VVal \dimMeta]{\vQ_1, \vQ_2}}{
     \envInContext [(\vctx_1 \wedge \VVal \dimMeta) \vee (\vctx_2 \wedge \neg \VVal \dimMeta)] 
     {\left({\pushIn {\envInContext [\vctx_1] \vType_1}} \cup
    							{\pushIn {\envInContext [\vctx_2] \vType_2}}\right)}}}
% older version 8/8/20:
%    {\env{\chc[\VVal \dimMeta]{\vQ_1, \vQ_2}}{
%     \envInContext [(\vctx_1 \wedge \VVal \dimMeta) \vee (\vctx_2 \wedge \neg \VVal \dimMeta)] 
%     {\left({\pushIn {\envInContext [\vctx_1 \wedge \VVal \dimMeta] \vType_1}} \cup
%    							{\pushIn {\envInContext [\vctx_2 \wedge \neg \VVal \dimMeta] \vType_2}}\right)}}}
    
  \inferrule[\productE]
  	{\envOne \\
    	\envTwo\\
	\pushIn {\annot [\vctx_1] \vType_1} \cap \pushIn {\annot [\vctx_2] \vType_2} = \{\}}
    {\env{\vQ_1 \times \vQ_2}{\envInContext [\vctx_1 \wedge \vctx_2] 
      {\left(\pushIn {\annot [\vctx_1] \vType_1} \cup \pushIn {\annot [\vctx_2] \vType_2} \right)}}}


  \inferrule[\setopE]
  	{\envOne \\
    	\envTwo \\
	\envEval {\pushIn {\annot [\vctx_1] \vType_1}} {\pushIn {\annot [\vctx_2] \vType_2}}}
%        \envEval{\envInContext{\vType_1}} \vType \\
%        \envEval{\envInContext{\vType_2}} \vType}
    {\env{\vQ_1 \circ \vQ_2} {\envInContext [\vctx_1] \vType_1} }

%  \inferrule[\diffE]
%  	{\envOne \\
%    	\envTwo \\
%        \envEval{\envInContext{\vType_1}} \vType \\
%        \envEval{\envInContext{\vType_2}} \vType}
%    {\env{\vQ_1 \setminus \vQ_2} \vType}
  \end{mathpar}
  
\medskip
\textbf{V-condition typing rules:}
% (b: boolean tag, \pAtt: plain attribute, k: constant value):}
%(b: boolean tag, A: plain attribute, k: constant value)}
  \begin{mathpar}
  \small    

  \inferrule[\conjC]
  	{\envCond \vCond_1\\
    	\envCond \vCond_2}
    {\envCond{\vCond_1 \wedge \vCond_2}}
    
  \inferrule[\disjC]
  	{\envCond \vCond_1\\
    	\envCond \vCond_2}
    {\envCond{\vCond_1 \vee \vCond_2}}
    


  \inferrule[\choiceC]
%  	{\defType{\relInContext{\vContext''}}\in \vSch \\
    	{\envCond[\vctx \wedge \VVal \dimMeta, \vType]{\vCond_1} \\
        \envCond[\vctx \wedge \neg \VVal \dimMeta, \vType]{\vCond_2}}
    {\envCond{\chc[\VVal \dimMeta]{\vCond_1, \vCond_2}}}
    

  \inferrule[\notC]
  	{\envCond \vCond}
    {\envCond \neg \vCond}
        
%  \inferrule[]
%  	{\envCond[\vContext \wedge \dimMeta]{\vCond_1} \\
%    	\envCond[\vContext \wedge \neg\dimMeta]{\vCond_2}}
%    {\envCond{\chc{\vCond_1, \vCond_2}}}
    

    
  \inferrule[\attValC]
  	{
	%\defType{\relInContext{\vContext'}}\in \vSch \\
    	\optAtt [\VVal \dimMeta] \in \vType \\
%	\taut{{\VVal \dimMeta} \imply \vctx} \\
        \sat {\VVal \dimMeta \wedge \vctx}}
%        \\
%        \cte \in \dom \vAtt}
    {\envCond{\op \pAtt \cte}}
    
  \inferrule[\boolC]
  	{}
    {\envCond \bTag}
    

    
  \inferrule[\attAttC]
  	{
	%\defType{\relInContext{\vContext'}}\in \vSch \\
    	\optAtt [\dimMeta_1] [\vAtt_1]\in \vType \\
         {\optAtt [\dimMeta_2] [\vAtt_2]} \in \vType \\
%         \taut{\dimMeta_1 \imply \vctx} \\
%         \taut{\dimMeta_2 \imply \vctx} \\
        \sat { \dimMeta_1 \wedge \dimMeta_2 \wedge \vctx}}
%        \\
%        \type[\vAtt_1] = \type[ \vAtt_2]}
    {\envCond{\op{\pAtt_1}{\pAtt_2}}}
    
  \end{mathpar}

%\caption{V-condition typing relation. A v-condition \vCond\ is well-typed if 
%it is valid in the variational context \vctx\ and type environment \vType, i.e., 
%\envCond \vCond. Note that the type rules for v-conditions return a boolean, if
%the v-condition is type-correct the rules return \t, otherwise they return \f.}
\caption[\TODO{shortcaption}]{VRA and v-condition typing relation. 
The rules assume that the underlying VDB is well-formed. 
Remember that our theory assumes all attributes have the same type
and all constants belong to attributes' domain. 
%The typing rule of a join query is the combination
%of rules \selE\ and \productE.
}
\label{fig:vq-stat-sem}
%\end{minipage}
\end{figure}


%\input{formulas/vRelAlgTypingRulesHorz}
%\input{formulas/vRelAlgTypingRulesExplicitAnnotation}

%\input{formulas/vRelAlgTypingRulesExplicitAnnotation}

\rewrite{adjust after Eric reviews it in VLDB}
%\point{aspects we need to type check v-queries.}
In this section, we introduce a static type system for VRA. The type system
ensures that queries are consistent with the underlying v-schema. That is, that
all referenced relations and attributes are present in the variation contexts
in which they are used.
%
For example, consider the VDB from \exref{conf-vq} that contains only the
relation $\vRel(\optAtt[\fOne][\vAtt_1],\vAtt_2,\vAtt_3)^{\fOne\vee\fTwo}$. The
query $\vPrj[\vAtt_4]{\vRel}$ is ill-typed since $\vAtt_4$ is not present in
$\vRel$. Similarly, the queries $\vPrj[{\optAtt[\neg\fOne][\vAtt_1]}]{\vRel}$
and $\chc[\fOne]{\vPrj[\vAtt_2]{\vRel},\vPrj[\vAtt_1]{\vRel}}$ are both
ill-typed since $\vAtt_1$ is not present in \vRel\ when \fOne\ is disabled.

%is ill-typed because the variation encoded in the query is violating the 
%corresponding variation in v-schema, i.e., the feature expression
%\ensuremath{\neg \fOne \wedge \getPC{\vAtt_1}} is not satisfiable.
%For example, while projecting an annotated attribute \optAtt\ from a 
%v-relation \vRel\ not only the attribute must belong to the v-relation, i.e., 
%$\vRel \annot [\dimMeta_\vRel] {\paran {\optAtt [\dimMeta_1], \vAttList}}$, but 
%the feature expression $\dimMeta \wedge \dimMeta_\vRel \wedge \dimMeta_1$ 
%must also be satisfiable, i.e., 
%%a database variant in the intersection of variability
%%encoded in the v-schema and the query exists s.t. attribute \vAtt\ is present in 
%%relation \vRel.
%the attribute \vAtt\ must be present in the relation \vRel\
%under the condition imposed by the query and the v-schema. 
%
%
%A v-query that is not well-typed is \emph{ill-typed}.

The type of a VRA query is a v-relation schema
$\mathit{result}\annot{(\vAttList)}$. However, since the
relation name is the same for all queries, we shorten this to
$\annot{\vAttList}$, that is, an annotated v-set of attributes.
%
The annotation $e$ corresponds to the presence condition of the 
returned table.
%
The presence conditions of attributes within $A$ may differ from the
corresponding presence conditions in the original v-schema due to variation
constraints imposed by the query.
%
For example, continuing with relation
$\vRelSch=\vRel(\optAtt[\fOne][\vAtt_1],\vAtt_2,\vAtt_3)^{\fOne\vee\fTwo}$, the
query $\vPrj[{\optAtt[\fOne][\vAtt_2]}]{\vRel}$ has type
$\{\annot[\fOne]{\vAtt_2}\}^{\fOne\vee\fTwo}$. 
%
In the original schema, $\vAtt_2$ is present when $\A\vee\B$, while in the
query it is present only when $\A$ is enabled.
%
% while 
%according to \vRel's schema 
%\ensuremath{\getPC {\vAtt_2} = \fOne \vee \fTwo}, 
% i.e., the presence
% condition of attribute \ensuremath{{\vAtt_2}} changes through the query.
%
% The presence condition of the entire set determines the condition under
% which the entire table (i.e., attributes and tuples) are present. 
%
%Note that it is essential to consider the type of a query an \emph{annotated}
%v-set to account for the presence condition of the entire table including the tuples. 
%similar to how we encode v-relation schemas. 
%If we
%consider the type of a query a variational attribute set we lose information (i.e.,
%the condition under which tuples are valid).
%The final variation context, after running a v-query, is the
%presence condition of the returned v-table. That is why we 
%consider the type environment as a variational set of attributes instead of 
%a relation schema. 

%\begin{example}
%\label{eg:vq-affect-vctx}
%Assume we have the VDB defined in \exref{vsch}. 
%Consider the query $\pi_{\name} \empbio$. The returned 
%table has the type: $(\optAtt [\vFour] [\name])^\fModel$. 
%However, if we change the query to: 
%$\pi_{\optAtt [\edu] [\name]} \empbio$, the returned table has the
%type: $(\optAtt [\vFour \wedge \edu] [\name])^\fModel$. 
%\end{example}

%
%\eric{Eric, feel free to summarize rule explanations. I basically
%explained them how I'd read them.}

\begin{figure}
%\begin{minipage}[t]{0.5\textwidth}
\textbf{V-queries typing rules:}

  \begin{mathpar}
  \small
  
  \inferrule[\empRelE]
  {}
  {\env {\empRel} {\annot [\f] {\setDef \ }}}
%  \inferrule[\judge]
 % 	{\env{\vQ}{\vType}}
 %   {}
%
% explicitly-typed vra:
%    \inferrule[\relationE]
%  	{\vRel (\vType)^{\VVal \dimMeta} \in \vSch \\
%	\neg \sat{\vctx \wedge \neg \VVal \dimMeta} }
%     {\envWithSchema{\envInContext [\vctx ] {\vType}}}

%implicitly-typed lang:
    \inferrule[\relationE]
  	{ \vRel (\vType)^{\VVal \dimMeta} \in \vSch \\
	\sat {\vctx \wedge \getPCfrom \vRelSch \vSch}}
%	\sat{\vctx \wedge \VVal \dimMeta} }
     {\envWithSchema{\envInContext [\vctx \wedge \VVal \dimMeta] {\vType}}}

% explicitly-typed vra:  
%  \inferrule[\prjE]
%  	{\envPrime \\
%    	\subsume {\annot \vType}  {\annot [\VVal \vctx] {\VVal \vType}}}
%    {\env{\vPrj[\vType] \vQ} {\envInContext [\vctx] \vType}}

%implicitly-typed lang:
  \inferrule[\prjE]
  	{\envPrime \\
	|\pushIn {\annot \vType}| = | \vType | \\
%	\pushIn {\annot \vType} \neq \setDef \ \\
%	\pushIn {\annot [\VVal \dimMeta] {\VVal \vType}} \neq \setDef \ \\
    	\subsume { \vType}  {\pushIn {\annot [\VVal \vctx] {\VVal \vType}}}}
    {\env{\vPrj[\vType] \vQ} {\envInContext [\VVal \vctx] {\left(\vType \cap {\VVal \vType} \right)}}}
%    the older version 8/8/20:
%    	\subsume {\pushIn {\annot \vType}}  {\pushIn {\annot [\VVal \vctx] {\VVal \vType}}}}
%    {\env{\vPrj[\vType] \vQ} {\envInContext [\VVal \vctx] {\left(\pushIn{\annot {\vType}} \cap {\VVal \vType} \right)}}}


  \inferrule[\selE]
  	{\env \vQ {\envInContext [\VVal \vctx] \vType} \\
    	\envCondAnnot \vCond}
    {\env{\vSel \vQ}{\envInContext [\VVal \vctx] \vType}}
    
  \inferrule[\choiceE]
  	{\envOne[\vctx \wedge \VVal \dimMeta] \\
    	\envTwo[\vctx \wedge \neg \VVal \dimMeta]}
    {\env{\chc[\VVal \dimMeta]{\vQ_1, \vQ_2}}{
     \envInContext [(\vctx_1 \wedge \VVal \dimMeta) \vee (\vctx_2 \wedge \neg \VVal \dimMeta)] 
     {\left({\pushIn {\envInContext [\vctx_1] \vType_1}} \cup
    							{\pushIn {\envInContext [\vctx_2] \vType_2}}\right)}}}
% older version 8/8/20:
%    {\env{\chc[\VVal \dimMeta]{\vQ_1, \vQ_2}}{
%     \envInContext [(\vctx_1 \wedge \VVal \dimMeta) \vee (\vctx_2 \wedge \neg \VVal \dimMeta)] 
%     {\left({\pushIn {\envInContext [\vctx_1 \wedge \VVal \dimMeta] \vType_1}} \cup
%    							{\pushIn {\envInContext [\vctx_2 \wedge \neg \VVal \dimMeta] \vType_2}}\right)}}}
    
  \inferrule[\productE]
  	{\envOne \\
    	\envTwo\\
	\pushIn {\annot [\vctx_1] \vType_1} \cap \pushIn {\annot [\vctx_2] \vType_2} = \{\}}
    {\env{\vQ_1 \times \vQ_2}{\envInContext [\vctx_1 \wedge \vctx_2] 
      {\left(\pushIn {\annot [\vctx_1] \vType_1} \cup \pushIn {\annot [\vctx_2] \vType_2} \right)}}}


  \inferrule[\setopE]
  	{\envOne \\
    	\envTwo \\
	\envEval {\pushIn {\annot [\vctx_1] \vType_1}} {\pushIn {\annot [\vctx_2] \vType_2}}}
%        \envEval{\envInContext{\vType_1}} \vType \\
%        \envEval{\envInContext{\vType_2}} \vType}
    {\env{\vQ_1 \circ \vQ_2} {\envInContext [\vctx_1] \vType_1} }

%  \inferrule[\diffE]
%  	{\envOne \\
%    	\envTwo \\
%        \envEval{\envInContext{\vType_1}} \vType \\
%        \envEval{\envInContext{\vType_2}} \vType}
%    {\env{\vQ_1 \setminus \vQ_2} \vType}
  \end{mathpar}
  
\medskip
\textbf{V-condition typing rules:}
% (b: boolean tag, \pAtt: plain attribute, k: constant value):}
%(b: boolean tag, A: plain attribute, k: constant value)}
  \begin{mathpar}
  \small    

  \inferrule[\conjC]
  	{\envCond \vCond_1\\
    	\envCond \vCond_2}
    {\envCond{\vCond_1 \wedge \vCond_2}}
    
  \inferrule[\disjC]
  	{\envCond \vCond_1\\
    	\envCond \vCond_2}
    {\envCond{\vCond_1 \vee \vCond_2}}
    


  \inferrule[\choiceC]
%  	{\defType{\relInContext{\vContext''}}\in \vSch \\
    	{\envCond[\vctx \wedge \VVal \dimMeta, \vType]{\vCond_1} \\
        \envCond[\vctx \wedge \neg \VVal \dimMeta, \vType]{\vCond_2}}
    {\envCond{\chc[\VVal \dimMeta]{\vCond_1, \vCond_2}}}
    

  \inferrule[\notC]
  	{\envCond \vCond}
    {\envCond \neg \vCond}
        
%  \inferrule[]
%  	{\envCond[\vContext \wedge \dimMeta]{\vCond_1} \\
%    	\envCond[\vContext \wedge \neg\dimMeta]{\vCond_2}}
%    {\envCond{\chc{\vCond_1, \vCond_2}}}
    

    
  \inferrule[\attValC]
  	{
	%\defType{\relInContext{\vContext'}}\in \vSch \\
    	\optAtt [\VVal \dimMeta] \in \vType \\
%	\taut{{\VVal \dimMeta} \imply \vctx} \\
        \sat {\VVal \dimMeta \wedge \vctx}}
%        \\
%        \cte \in \dom \vAtt}
    {\envCond{\op \pAtt \cte}}
    
  \inferrule[\boolC]
  	{}
    {\envCond \bTag}
    

    
  \inferrule[\attAttC]
  	{
	%\defType{\relInContext{\vContext'}}\in \vSch \\
    	\optAtt [\dimMeta_1] [\vAtt_1]\in \vType \\
         {\optAtt [\dimMeta_2] [\vAtt_2]} \in \vType \\
%         \taut{\dimMeta_1 \imply \vctx} \\
%         \taut{\dimMeta_2 \imply \vctx} \\
        \sat { \dimMeta_1 \wedge \dimMeta_2 \wedge \vctx}}
%        \\
%        \type[\vAtt_1] = \type[ \vAtt_2]}
    {\envCond{\op{\pAtt_1}{\pAtt_2}}}
    
  \end{mathpar}

%\caption{V-condition typing relation. A v-condition \vCond\ is well-typed if 
%it is valid in the variational context \vctx\ and type environment \vType, i.e., 
%\envCond \vCond. Note that the type rules for v-conditions return a boolean, if
%the v-condition is type-correct the rules return \t, otherwise they return \f.}
\caption[\TODO{shortcaption}]{VRA and v-condition typing relation. 
The rules assume that the underlying VDB is well-formed. 
Remember that our theory assumes all attributes have the same type
and all constants belong to attributes' domain. 
%The typing rule of a join query is the combination
%of rules \selE\ and \productE.
}
\label{fig:vq-stat-sem}
%\end{minipage}
\end{figure}



\figref{vq-stat-sem} defines a typing relation that relates VRA queries to
their types.
%
The judgement form \env{\vQ}{\envInContext[\VVal{\vctx}]{\vType}} states that
in variation context \vctx\ within v-schema \vSch, v-query \vQ\ has type
\envInContext[\VVal{\vctx}]{\vType}. 
%
If a query does not have a type, it is \emph{ill-typed}.
%
A \emph{variation context} is a feature expression that tracks which variants
the current subquery is present in. The variation context is initiated to be
the schema's feature model and extended when entering the alternative of a choice.
%
We sometimes use the judgment form \envWithoutVctx{\vQ}{\envInContext[\VVal{\vctx}]{\vType}}
when the variation context is the unextended feature model, i.e., 
\env[\getPC{\vSch}]{\vQ}{\envInContext[\VVal{\vctx}]{\vType}}.
%
We assume that the v-set of attributes $A$ is normalized to remove elements
with unsatisfiable presence conditions, but this normalization is only shown
explicitly in the rules where strictly necessary.

%Note that attributes with an
%unsatisfiable presence condition are not present in any 
%database variant, i.e., they are not present for any configuration.
%Thus, the existence of such attribute in a type does not change
%the type semantically, based on the defined equivalence rule for 
%v-sets, given in \defref{vset-eq}. Hence, we do not filter out such attributes
%explicitly in \figref{vq-stat-sem}, however, for simplicity, 
%the implemented type system
%drops the attributes with an unsatisfiable presence condition.
 

The rule \relationE\ looks up the relation \vRel\ in the 
v-schema \vSch\ and returns its v-set of attributes \vAttList\ annotated
with the variation context and its presence condition $\vctx \wedge \VVal \dimMeta \wedge \getPC \vSch$. 
%
That is, it states that, in variation context \vctx\ with
underlying variational schema \vSch, assuming that
1) \vSch\ contains
the relation \vRel\ with presence condition $\VVal \dimMeta$
and v-set of attributes \vType\ 
and
2) there exists a valid variant in the intersection of variation context \vctx\
and \vRel's presence condition \ensuremath{\dimMeta \wedge \VVal \dimMeta \wedge \getPC \vSch},
then query \vRel\ has type \ensuremath {\annot [\vctx \wedge \VVal \dimMeta \wedge \getPC \vSch] \vType}.
% \ensuremath{\annot [\vctx \wedge \getPCfrom \vRelSch \vSch] \vType}.

% 
The rule \prjE\ 
checks that first, the subquery is well-typed (i.e., $\envPrime$);
second, all projected attributes are present under the variation context 
(i.e., $|\pushIn {\annot \vType}| = | \vType |$);
third, projected attributes are subsumed by the type of the subquery
(i.e., $\subsume { \vType}  {\pushIn {\annot [\VVal \vctx] {\VVal \vType}}}$).
If all the premises hold, it returns the intersection of the projected attributes
and the subquery attributes annotated with the presence condition of the
subquery type (i.e., ${\envInContext [\VVal \vctx] {\left(\vType \cap {\VVal \vType} \right)}}$).
%
That is, it states that, in variation context \vctx\ within v-schema \vSch, assuming 
that the subquery \vQ\ has type $\envInContext [\VVal \vctx] {\VVal \vType}$,
v-query $\pi_\vType \vQ$
has type \ensuremath {\envInContext [\VVal \vctx] {\left( \vType \cap \VVal \vType\right)}}, 
if all attributes in \vType\ are present in \vctx\
and
 \ensuremath {\pushIn {\envInContext [\VVal \vctx] {\VVal \vType}}} subsumes \vType.
%
Note that the variation context \dimMeta\ is applied in the presence condition of
the subquery (\VVal \dimMeta), thus, there is no need to repeat it in the returned 
type of the query $\pi_\vAttList \vQ$.
%
Assuming that $\VVVal \vType = \annot [\VVal \dimMeta] {\VVal \vType}$,
the subsumption of $\subsume \vType { \VVVal \vType}$ is defined as
\ensuremath{ \forall \annot [\dimMeta_1] \vAtt \in \pushIn {\vType}.
\exists \dimMeta_2. \annot [\dimMeta_2] \vAtt \in \pushIn {\VVVal \vType}, 
\sat {\dimMeta_2 \wedge  \dimMeta_1}} and it 
ensures that the subquery \vQ\ does not have an empty type
and it includes all attributes in 
the projected attribute set and attributes' presence conditions do not 
contradict each other. 
Returning the intersection of types, defined in 
\defref{vset-intersect}, filters both 
attributes and their presence conditions.
\exref{type} illustrates how the type system generates a type for a query
and how this type helps enforcing the variation encoded in the query to its result.


\begin{example}
\label{eg:type}
Consider the query \ensuremath{\vQ_1} given in \exref{vq-specific}.
Through this example, we simplify feature expressions when possible.
The \prjE\ rule is applied under
the variation context initiated to 
\ensuremath{\fModel_2 = \vThree \oplus \vFour \oplus \vFive}
and schema \ensuremath{\vSch_2}.
Now the
\relationE\ rule applies to the subquery \ensuremath {(\empbio)} 
under the same variation context and schema,
resulting in the type
\ensuremath{
\vAttList_\empbio =  \{\empno, \sex, \birthdate,}
\ensuremath{ 
\optAtt [\vFour] [\name], \optAtt [\vFive] [\fname], \optAtt [\vFive] [\lname]\}^{\fModel_2}}.
Now that the type system has the type of the subquery \empbio\ 
it verifies that the projected attribute v-set
\ensuremath{
\vAttList_{\mathit{prj}} =
 \{\optAtt [\vFour \vee \vFive] [\empno],
\name,}
\ensuremath{ \fname, \lname\}^{\fModel_2}},
is subsumed by \ensuremath{\vAttList_\empbio}. 
Thus, it generates the type of \ensuremath{\vQ_1} by
intersecting \ensuremath{\vAttList_{\mathit{prj}}} and \ensuremath{\vAttList_\empbio}
annotated with \ensuremath{\vAttList_\empbio}'s presence condition, resulting in the type
\ensuremath{
\vAttList_{\vQ_1} = 
\{\optAtt [\vFour \vee \vFive] [\empno],
\optAtt [\vFour] [\name], }
\ensuremath{
\optAtt [\vFive] [\fname], \optAtt [\vFive] [\lname]\}^{\fModel_2}}.
%
This type proves that $\vQ_1$ is well-typed and dictates the structure of its result.
\ensure{don't forget to inlcude this}
\figref{vq1-type} illustrates the tree derivation of $\vQ_1$'s type.
%
Now consider \ensuremath{\vQ_2} introduced in \exref{vq-specific}.
The \choiceE\ rule is applied
under the variation 
context initiated to \ensuremath{\fModel_2} and schema \ensuremath{\vSch_2}.
It then applies the \prjE\ and \empRelE\ rules to the left and right
alternatives of the choice, respectively, which generates the types
\ensuremath{
\vType_\mathit{left} = \annot [\fModel_2 \wedge (\vFour \vee \vFive)] {(\empno, \annot [\vFour] \name,
 \annot [\vFive] \fname,\annot [\vFive] \lname)}}
and \ensuremath{\vType_\mathit{right} = \annot [\f] {\setDef \ }}, respectively.
Finally, it generates the type of \ensuremath{\vQ_2} by 
annotating the union of \ensuremath{\vType_\mathit{left}} and \ensuremath{\vType_\mathit{right}}
with \ensuremath{\fModel_2 \wedge (\vFour \vee \vFive)}, resulting in the 
final type of \\
\ensuremath{\vType_{\vQ_2} = 
\annot [\fModel_2 \wedge (\vFour \vee \vFive)] {(\empno, \annot [\vFour] \name,
 \annot [\vFive] \fname,\annot [\vFive] \lname)}}.
 Note that \ensuremath{\vType_{\vQ_2}}'s presence condition 
 explicitly accounts for only two variants
 while \ensuremath{\vType_1} does not do so even though \ensuremath{\vQ_1}
 does not return any tuple that belongs to variant \ensuremath{\setDef \vThree} because
 of its attributes presence condition.
 \ensure{don't forget to inlcude this}
 \figref{vq2-type} illustrates the tree derivation of $\vQ_2$'s type.
\end{example}

%\begin{landscape}
\begin{figure}
%\begin{minipage}[t]{0.5\textwidth}
\caption[Example of derivation tree to determine the type of a query]{Derivation tree for \ensuremath{\vQ_1} in \exref{type}.}
%\textbf{Derivation tree for \ensuremath{\vQ_1} in \exref{type}:}

%\footnotesize
\scriptsize

\textbf{Assumption 1:}
\begin{prooftree*}
\hypo {\empbio \annot [\fModel_2] {(\empno, \sex, \birthdate, \annot [\vFour] \name, \annot [\vFive] \fname, \annot [\vFive] \lname) }\in \vSch_2}
\hypo {\sat {\fModel_2 \wedge \fModel_2}}
\Infer2 [\textsc{Relation-E}] {\env [\fModel_2] [\vSch_2] {\empbio} {\annot [\fModel_2] {(\empno, \sex, \birthdate, \annot [\vFour] \name, \annot [\vFive] \fname, \annot [\vFive] \lname)}}}
\end{prooftree*}

\medskip
%\begin{prooftree*}
\textbf{Assumption 2:}
\begin{alignat*}{1}
%\hypo {
&  { \{\annot [\vFour \vee \vFive] \empno, \name, \fname, \lname  \} }\\ 
&\qquad \prec
{\{\annot [\fModel_2] \empno, \annot [\fModel_2] \sex, \annot [\fModel_2] \birthdate, \annot [\vFour] \name, \annot [\vFive] \fname, \annot [\vFive] \lname\} } 
%}
%\end{prooftree*}
\end{alignat*}

\medskip
\textbf{Final derivation tree:}
\begin{prooftree*}
\hypo {\textit{\bf Assumption 1}}
%\hypo {\empbio \annot [\fModel_2] {(\empno, \sex, \birthdate, \annot [\vFour] \name, \annot [\vFive] \fname, \annot [\vFive] \lname) }\in \vSch_2}
%\hypo {\sat {\fModel_2 \wedge \fModel_2}}
%\Infer2 [\relationE] {\env [\fModel_2] [\vSch_2] {\empbio} {\annot [\fModel_2] {(\empno, \sex, \birthdate, \annot [\vFour] \name, \annot [\vFive] \fname, \annot [\vFive] \lname)}}}
\hypo {\textit{\bf Assumption 2}}
%\hypo {\subsume { \{\annot [\vFour \vee \vFive] \empno, \name, \fname, \lname  \} } 
%{\{\annot [\fModel_2] \empno, \annot [\fModel_2] \sex, \annot [\fModel_2] \birthdate, \annot [\vFour] \name, \annot [\vFive] \fname, \annot [\vFive] \lname\} } }
\Infer2 [\textsc{Project-E}] {\env [\fModel_2] [\vSch_2] {\vQ_1} {\annot [\fModel_2] {( \annot [(\vFour \vee \vFive) \wedge \fModel_2] \empno, \annot [\vFour] \name, \annot [\vFive] \fname, \annot [\vFive] \lname)} }}
\end{prooftree*}
\end{figure}

\begin{figure}
\caption[Example of derivation tree to determine the type of a query]{Derivation tree for \ensuremath{\vQ_2} in \exref{type}.}
%\medskip
%\textbf{Derivation tree for \ensuremath{\vQ_2} in \exref{type}:}

\scriptsize
\textbf{Assumption 1:}
\begin{prooftree*}
\hypo {\empbio \annot [\fModel_2] {(\empno, \sex, \birthdate, \annot [\vFour] \name, \annot [\vFive] \fname, \annot [\vFive] \lname) }\in \vSch_2}
\end{prooftree*}

\medskip
\textbf{Assumption 2:}
\begin{prooftree*}
\hypo {\textit{\bf Assumption 1}}
\hypo {\sat {(\fModel_2 \wedge (\neg \vThree)) \wedge \fModel_2}}
\Infer2 [\textsc{Relation-E}] {\env [\fModel_2 \wedge (\neg \vThree)] [\vSch_2] {\empbio} {\annot [\fModel_2 \wedge (\vFour \vee \vFive)] {(\empno, \sex, \birthdate, \annot [\vFour] \name, \annot [\vFive] \fname, \annot [\vFive] \lname)}}}
\end{prooftree*}

%\begin{prooftree*}
\medskip
\textbf{Assumption 3:}
\begin{alignat*}{1}
%\hypo {
%\subsume
& { \{ \empno, \name, \fname, \lname  \} } \\
&\qquad \prec {\{\annot [\fModel_2 \wedge (\neg \vThree)] \empno, \annot [\fModel_2 \wedge (\neg \vThree)] \sex, \annot [\fModel_2 \wedge (\neg \vThree)] \birthdate, \annot [\vFour] \name, \annot [\vFive] \fname, \annot [\vFive] \lname\} } 
%}
%\end{prooftree*}
\end{alignat*}

\medskip
\textbf{Assumption 4 (derivation tree for $\mathit{left} = {\pi_{\empno, \name, \fname, \lname} (\empbio)}$):}
\begin{prooftree*}
\hypo {\textit{\bf Assumption 2}}
\hypo {\textit{\bf Assumption 3}}
\Infer2 [\textsc{Project-E}] {\env [\fModel_2 \wedge (\neg \vThree)] [\vSch_2] {\mathit{left}} {\annot [\fModel_2 \wedge (\neg \vThree)] {(\annot [\fModel_2 \wedge (\neg \vThree)] \empno, \annot [\vFour] \name, \annot [\vFive] \fname, \annot [\vFour] \lname)}}}
\end{prooftree*}

\medskip
\textbf{Final derivation tree for $\vQ_2$:}
\begin{prooftree*}
%\hypo {Assumption 2}
\hypo {\textit{\bf Assumption 4}}
\Infer0 [\textsc{EmptyRelation-E}] {\env [\fModel_2 \wedge \neg (\neg \vThree)] [\vSch_2] {\empRel} {\annot [\texttt{false}] {\setDef \ }}}
\Infer2 [\textsc{Choice-E}] {\env [\fModel_2] [\vSch_2] {\vQ_2} {\annot [(\fModel_2 \wedge (\neg \vThree)) \vee (\texttt{false} \wedge \neg (\neg \vThree))] {(\annot [(\neg \vThree) \wedge \fModel_2] \empno, \annot [\vFour] \name, \annot [\vFive] \fname, \annot [\vFive] \lname)}}}
\end{prooftree*}


%\begin{prooftree*}
%\hypo {\empbio \annot [\fModel_2] {(\empno, \sex, \birthdate, \annot [\vFour] \name, \annot [\vFive] \fname, \annot [\vFive] \lname) }\in \vSch_2}
%\hypo {\sat {(\fModel_2 \wedge (\vFour \vee \vFive)) \wedge \fModel_2}}
%\Infer2 [\relationE] {\env [\fModel_2 \wedge (\vFour \vee \vFive)] [\vSch_2] {\empbio} {\annot [\fModel_2 \wedge (\vFour \vee \vFive)] {(\empno, \sex, \birthdate, \annot [\vFour] \name, \annot [\vFive] \fname, \annot [\vFive] \lname)}}}
%\hypo {\subsume { \{ \empno, \name, \fname, \lname  \} } {\{\annot [\fModel_2 \wedge (\vFour \vee \vFive)] \empno, \annot [\fModel_2 \wedge (\vFour \vee \vFive)] \sex, \annot [\fModel_2 \wedge (\vFour \vee \vFive)] \birthdate, \annot [\vFour] \name, \annot [\vFive] \fname, \annot [\vFive] \lname\} } }
%\Infer2 [\prjE] {\env [\fModel_2 \wedge (\vFour \vee \vFive)] [\vSch_2] {\prj {\empno, \name, \fname, \lname} \empbio} {\annot [\fModel_2 \wedge (\vFour \vee \vFive)] {(\annot [\fModel_2 \wedge (\vFour \vee \vFive)] \empno, \annot [\vFour] \name, \annot [\vFive] \fname, \annot [\vFour] \lname)}}}
%\Infer0 [\empRelE] {\env [\fModel_2 \wedge \neg (\vFour \vee \vFive)] [\vSch_2] {\empRel} {\annot [\f] {\setDef \ }}}
%\Infer2 [\choiceE] {\env [\fModel_2] [\vSch_2] {\vQ_2} {\annot [(\fModel_2 \wedge (\vFour \vee \vFive)) \vee (\f \wedge \neg (\vFour \vee \vFive))] {(\annot [(\vFour \vee \vFive) \wedge \fModel_2] \empno, \annot [\vFour] \name, \annot [\vFive] \fname, \annot [\vFive] \lname)}}}
%\end{prooftree*}

%  \begin{mathpar}
%  \small
%  
%  \inferrule[\prjE]
%  {\inferrule [\relationE] 
%  {blah}
%  {\env [\fModel_2] [\vSch_2] {\empbio} {\annot [\fModel_2] {(\empno, \sex, \birthdate, \annot [\vFour] \name, \annot [\vFive] \fname, \annot [\vFive] \lname)}}}\\
%\subsume { \{\annot [\vFour \vee \vFive] \empno, \name, \fname, \lname  \} } 
%{\{\annot [\fModel_2] \empno, \annot [\fModel_2] \sex, \annot [\fModel_2] \birthdate, \annot [\vFour] \name, \annot [\vFive] \fname, \annot [\vFive] \lname\}}}
%  %
%  {\env [\fModel_2] [\vSch_2] {\vQ_1} {\annot [\fModel_2] {\setDef {\annot [(\vFour \vee \vFive) \wedge \fModel_2] {(\empno, \annot [\vFour] \name, \annot [\vFive] \fname, \annot [\vFive] \lname)} }}}}
%
%%implicitly-typed lang:
%    \inferrule[\relationE]
%  	{\vRel (\vType)^{\VVal \dimMeta} \in \vSch \\
%	\sat{\vctx \wedge \VVal \dimMeta} }
%     {\envWithSchema{\envInContext [\vctx \wedge \VVal \dimMeta] {\vType}}}
%
%% explicitly-typed vra:  
%%  \inferrule[\prjE]
%%  	{\envPrime \\
%%    	\subsume {\annot \vType}  {\annot [\VVal \vctx] {\VVal \vType}}}
%%    {\env{\vPrj[\vType] \vQ} {\envInContext [\vctx] \vType}}
%
%%implicitly-typed lang:
%  \inferrule[\prjE]
%  	{\envPrime \\
%    	\subsume {\pushIn {\annot \vType}}  {\pushIn {\annot [\VVal \vctx] {\VVal \vType}}}}
%    {\env{\vPrj[\vType] \vQ} {\envInContext [\VVal \vctx] {\left(\pushIn{\annot {\vType}} \cap {\VVal \vType} \right)}}}
%
%
%  \inferrule[\selE]
%  	{\env \vQ {\envInContext [\VVal \vctx] \vType} \\
%    	\envCondAnnot \vCond}
%    {\env{\vSel \vQ}{\envInContext [\VVal \vctx] \vType}}
%    
%  \inferrule[\choiceE]
%  	{\envOne[\vctx \wedge \VVal \dimMeta] \\
%    	\envTwo[\vctx \wedge \neg \VVal \dimMeta]}
%    {\env{\chc[\VVal \dimMeta]{\vQ_1, \vQ_2}}{
%     \envInContext [(\vctx_1 \wedge \VVal \dimMeta) \vee (\vctx_2 \wedge \neg \VVal \dimMeta)] 
%     {\left({\pushIn {\envInContext [\vctx_1 \wedge \VVal \dimMeta] \vType_1}} \cup
%    							{\pushIn {\envInContext [\vctx_2 \wedge \neg \VVal \dimMeta] \vType_2}}\right)}}}
%    
%  \inferrule[\productE]
%  	{\envOne \\
%    	\envTwo\\
%	\pushIn {\annot [\vctx_1] \vType_1} \cap \pushIn {\annot [\vctx_2] \vType_2} = \{\}}
%    {\env{\vQ_1 \times \vQ_2}{\envInContext [\vctx_1 \wedge \vctx_2] 
%      {\left(\pushIn {\annot [\vctx_1] \vType_1} \cup \pushIn {\annot [\vctx_2] \vType_2} \right)}}}
%
%
%  \inferrule[\setopE]
%  	{\envOne \\
%    	\envTwo \\
%	\envEval {\pushIn {\annot [\vctx_1] \vType_1}} {\pushIn {\annot [\vctx_2] \vType_2}}}
%%        \envEval{\envInContext{\vType_1}} \vType \\
%%        \envEval{\envInContext{\vType_2}} \vType}
%    {\env{\vQ_1 \circ \vQ_2} {\envInContext [\vctx_1] \vType_1} }
%
%%  \inferrule[\diffE]
%%  	{\envOne \\
%%    	\envTwo \\
%%        \envEval{\envInContext{\vType_1}} \vType \\
%%        \envEval{\envInContext{\vType_2}} \vType}
%%    {\env{\vQ_1 \setminus \vQ_2} \vType}
%  \end{mathpar}
%  
%\medskip
%\textbf{V-condition typing rules (b: boolean tag, \pAtt: plain attribute, k: constant value):}
%%(b: boolean tag, A: plain attribute, k: constant value)}
%  \begin{mathpar}
%  \small    
%
%  \inferrule[\conjC]
%  	{\envCond \vCond_1\\
%    	\envCond \vCond_2}
%    {\envCond{\vCond_1 \wedge \vCond_2}}
%    
%  \inferrule[\disjC]
%  	{\envCond \vCond_1\\
%    	\envCond \vCond_2}
%    {\envCond{\vCond_1 \vee \vCond_2}}
%    
%
%
%  \inferrule[\choiceC]
%%  	{\defType{\relInContext{\vContext''}}\in \vSch \\
%    	{\envCond[\vctx \wedge \VVal \dimMeta, \vType]{\vCond_1} \\
%        \envCond[\vctx \wedge \neg \VVal \dimMeta, \vType]{\vCond_2}}
%    {\envCond{\chc[\VVal \dimMeta]{\vCond_1, \vCond_2}}}
%    
%
%  \inferrule[\notC]
%  	{\envCond \vCond}
%    {\envCond \neg \vCond}
%        
%%  \inferrule[]
%%  	{\envCond[\vContext \wedge \dimMeta]{\vCond_1} \\
%%    	\envCond[\vContext \wedge \neg\dimMeta]{\vCond_2}}
%%    {\envCond{\chc{\vCond_1, \vCond_2}}}
%    
%
%    
%  \inferrule[\attValC]
%  	{
%	%\defType{\relInContext{\vContext'}}\in \vSch \\
%    	\optAtt [\VVal \dimMeta] \in \vType \\
%%	\taut{{\VVal \dimMeta} \imply \vctx} \\
%        \sat {\VVal \dimMeta \wedge \vctx}\\
%        \cte \in \dom \vAtt}
%    {\envCond{\op \pAtt \cte}}
%    
%  \inferrule[\boolC]
%  	{}
%    {\envCond \bTag}
%    
%
%    
%  \inferrule[\attAttC]
%  	{
%	%\defType{\relInContext{\vContext'}}\in \vSch \\
%    	\optAtt [\dimMeta_1] [\vAtt_1]\in \vType \\
%         {\optAtt [\dimMeta_2] [\vAtt_2]} \in \vType \\
%%         \taut{\dimMeta_1 \imply \vctx} \\
%%         \taut{\dimMeta_2 \imply \vctx} \\
%        \sat { \dimMeta_1 \wedge \dimMeta_2 \wedge \vctx}\\
%        \type[\vAtt_1] = \type[ \vAtt_2]}
%    {\envCond{\op{\pAtt_1}{\pAtt_2}}}
%    
%  \end{mathpar}
%
%%\caption{V-condition typing relation. A v-condition \vCond\ is well-typed if 
%%it is valid in the variational context \vctx\ and type environment \vType, i.e., 
%%\envCond \vCond. Note that the type rules for v-conditions return a boolean, if
%%the v-condition is type-correct the rules return \t, otherwise they return \f.}
%\caption{VRA and v-condition typing relation. The typing rule of a join query is the combination
%of rules \selE\ and \productE.}
%\label{fig:vq-stat-sem}
%\end{minipage}
\end{figure}
%\end{landscape}


%
The rule \selE\ checks if its subquery and v-condition are well-typed
and if so it returns the subquery's type. 
%
That is, it states that, in variation context \vctx\ within v-schema \vSch, assuming 
that the subquery \vQ\ has type {\envInContext [\VVal \vctx] \vType}, 
the v-query $\sigma_{\vCond} \vQ$
has type {\envInContext [\VVal \vctx] \vType},
if the variational condition \vCond\ is \emph{well-formed} w.r.t.
 variation context \vctx\ and  normalized type {\pushIn {\envInContext [\VVal \vctx] \vType}}, 
denoted by v-condition's typing relation 
\envCondAnnot \vCond.
%
The v-condition typing relation, as defined in \figref{vq-stat-sem}, has the 
judgement form \envCond \vCond, which states that the v-condition \vCond\
is well-formed in variation context \vctx\ within attribute v-set \vType.
Remember that  {\pushIn {\envInContext [\VVal \vctx] \vType}} normalizes 
{\envInContext [\VVal \vctx] \vType} to be a v-set.
The v-condition typing rules state that attributes used in a
variational condition must be present in \vType\ and 
attribute's presence condition \ensuremath {\VVal \dimMeta} 
in type \vType\ must exists within variation context \vctx,
denoted by \ensuremath{\sat {\VVal \dimMeta \wedge \vctx}}.
%be more specific than the variation context \vctx,
%denoted by \ensuremath{\taut {\VVal \dimMeta \to \vctx}}, 
%since \vType\ is the exact type and specification of the subquery within
%a selection query which is at least as specific as the variation context under which
%the selection query is written. 
%the \fexpTxt\ attached to an attribute in a 
%variational condition must be more specific than its presence condition in type \vType. 
%They also
%check the constraints of traditional relational databases, such as the type of two 
%compared attributes must be the same.

%
The rule \choiceE\ adjusts the variation contexts of its alternative 
subqueries, checks if they are both well-typed, and if so it returns
the union of their types annotated with the disjunction of their presence conditions. 
%
That is, it states that, in variation context \vctx\ within v-schema \vSch, the type of 
a choice of two subqueries is the \emph{union of types}, defined in 
\defref{vset-union}, of its subqueries annotated with the disjunction of their presence
conditions conjuncted with the corresponding condition of the choice's dimension.
%A choice query is well-typed iff both of 
%its subqueries $\vQ_1$ and $\vQ_2$ are well-typed.
%
Note that \choiceE\ is the only rule that refines the variation context and that 
mainly uses the \empRelE\ rule which states that an empty relation has the 
type of an empty set annotated with \f.
%
The rest of the rules are congruence rules for product and set operations over queries. 

% 
%The rule \productE\ states that the type of a product query in variation context
%\vctx\ is the union of the type of its subqueries annotated with the 
%disjunction of their presence conditions, assuming that 
%they are disjoint. 
%%Note that 
%%\ensuremath{
%%\annot [\vctx_1 \wedge \vctx_2] {\left(\envInContext [\vctx_1] \vType_1 \cup \envInContext [\vctx_2] \vType_2\right)}
%%\equiv 
%%\envInContext [\vctx_1 \wedge \vctx_2] \vType_1 \cup \envInContext [\vctx_1 \wedge \vctx_2] \vType_2
%%\equiv
%%\annot [\vctx_1 \wedge \vctx_2] {\left(\vType_1 \cup \vType_2\right)}
%%}.
%
%% 
%The rule \setopE\ denotes the typing rule for set operation queries such as 
%union and difference. It states that, if the subqueries $\vQ_1$ and $\vQ_2$
%have \emph{equivalent} types $\envInContext [\vctx_1] \vType_1$ and 
%$\envInContext [\vctx_2] \vType_2$
%%, respectively, 
%in variation context \vctx,
%then the v-query of their set operation has type $\envInContext [\vctx_1] \vType_1$.
%%, iff 
%%$\pushIn {\envInContext [\vctx_1] \vType_1}$ and $\pushIn {\envInContext [\vctx_2] \vType_2}$ are \emph{equivalent}.
%The \emph{type equivalence} is v-set equivalence, defined in \defref{vset-eq}, for normalized v-set of attributes.
%%,
%%for v-sets of attributes.


%9-19-18 notes:
%* type soundness theorem: 
%    * ideally we want to prove this, but at least we should formulate it
%    * the way you do this usually, is that you have a semantics and you want to show 
%        * progress: if your program is well-typed your either done evaluating it or you can keep evaluate it -> in our case is kind of given, because we don't have a turing complete language!!
%        * preservation: when we evaluate sth it doesn't change its type! -> what we want to show is the relation that semantics gives us back fits the schema that the type system gives us. so there is consistency between type system and semantics. 
%    * so in the context of variational queries what that would mean is that when we evaluate a query, the relation that we get back actually has the type that we said it has via the type system
%    * PROBLEM: our semantics is via SQL so proving this will be kind of hairy but we should at least write this down and try to convince ourselves that it's correct, and we'll figure out what we can say in the context of the paper.
%    * [TODO] writing test cases that it?s correct!! in terms of mini database at two levels, haskell and database. the test cases will be queries against this mini database with lots of variability in it. you can write a quick check property. so you basically are testing the commuting diagram, so you either:
%        1. configure the query first and then give its type
%        2. give its variational type first and then configure it
%        3. and you should get the same thing from both
 


%The purpose of establishing \emph{type safety} is to ensure that the
%static and dynamic semantics are consistent with each other. 
%@Eric do we need to say anything about why we don't take the standard 
%approach?
%\NEED{Since 
%we define \vqTxt\ dynamic semantics in terms of relational algebra,
%i.e. we translate a \vqTxt\ into a set of relational algebra expression 
%and then combine the result of them into a \vrelTxt, we do not take 
%the standard approach of defining and proving \emph{progress} and 
%\emph{preservation} properties.
%}

%We 
%follow the approach developed by \TODO{fill in later!!}, which distinguishes
%two type safety properties, preservation and progress. The preservation
%theorem establishes that \vqsTxt\ preserve type assignments, i.e. that the
%type of a \vqTxt\ accurately predicates the type of the result of evaluating 
%that \vqTxt. \TODO{state the context and type you start out with}


%pierce def:
%progress: a well-typed term is not stuck (either it is a value or it can take a step 
%according to the evaluation rules)
%\begin{theorem}
%\label{thm:progress}
%Suppose \vQ\ is a closed, well-typed \vqTxt\ (that is, \env {\vQ} {\vType} for some 
%\vType). Then either \vQ\ is a value (a \vrelTxt ) or else there is some 
%\TODO{in order to define this we need to define:\\
%small step of relational algebra\\
%translation rules to rel alg\\
%combining the set of queries resulted from translation to output a var table\\
%canonical forms\\
%
%\end{theorem}


%pierce def:
%preservation: if a well-typed term takes a step of evaluation, then the resulting 
%term is also well typed
%\begin{theorem}
%\label{thm:type-pres}
%
%\end{theorem}

%\TODO{double check the following ph and write the properties based on it:}
%Conceptually, a variational query describes a query that can
%be executed over any database instance consistent with the 
%variational schema. The property that must hold between a 
%variational query \vQ\ and a variational schema \vSch
%is that for every plain query $\pQ_\config$ obtained from \vQ 
%by configuring with a function $\config: \fSet \to \bSet$, $\pQ_\config$ 
%is consistent with the corresponding plain schema $\pSch_\config$ 
%obtained by $\osSem  \vSch$ with the same function \config. That 
%is, every variant query matches the corresponding variant schema.
%%\REMEMBER{In \secref{prop-q-lang}, we prove that our query language,
%%variational relational algebra, can encode this idea and also
%%recover any of the conceptually potential results for any instance
%%of the variational database.}

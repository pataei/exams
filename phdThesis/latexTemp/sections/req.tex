\subsection{Requirements of a Variation-Aware Database Framework}
\label{sec:req}

%In this section, we define the requirements that make a database framework
%variational. 
For a variation-aware framework  to be expressive enough to encode
any kind of variation in databases, it must satisfy some requirements.
Thus, we define the requirements that make a database framework
variational through studying different kinds of variation in databases.  
%These requirements distinguish a database framework that 
%simulates the effect of variation compared to one that is variational. 
A variational artifact, including databases, encompasses multiple 
\emph{variants} of the artifact and provides a way to distinguishes 
between different variants that are all encoded in one place, the 
variational artifact. It also provides a way to get the variants from
the variational artifact, we call this \emph{configuration function}.
%
These requirements that help distinguish a database framework that 
simulates the effect of variation compared to one that is variational are
listed below:

%Our example demonstrates how instances of variation
%% in time and space
%arise and interact with each other, an unavoidable consequence of modern software
%development.
%We categorize the needs of a variational database framework at different stages
%such as development, test, and deployment and
%use our motivating example to illustrate them. Throughout the paper,
%we show how our framework achieves these needs via examples, proofs, and tests.
%%To be concrete throughout the paper we itemize
%%the needs of users working with a database with variation
%%through the needs of SPL developers and \revised{analysts and} DBAs:
%% for our motivating example.
%%These needs cover all general needs when variation appears in databases
%%because our motivating example considers variation in both possible 
%%dimensions; time and space:
%
\begin{enumerate}
%[wide, labelwidth=!, labelindent=0pt, topsep=1pt]
%[leftmargin=*]
%\itemsep0em
\item [\textbf{(\nZero)}]
\emph{All database variants must be accessible at a given time}.
For example, in our motivating example, a company that started with 
\vOne\ of the \basic\ schema evolves over time but its different 
branches adopt the new schema at different paces, thus, it 
requires access to all variants of the \basic\ schema. 
%For example,
%SPL developers need to access all database variants while
%writing code to be able to 
% extract information for 
% all software variants
%they are developing. 
%
%Hence,
%\emph{users need to have access to all database variants at a given time}.
%
%\secref{vdb} explains how a VDB achieves this.
%%\tabref{mot-vsch} shows how v-schema achieves
%\exref{vsch} shows how v-schema achieves
%this at schema level for a part of our motivating example.
%%make sure the following is the case.
%%\TODO{and \secref{exp-disc} discusses two databases
%%that achieve this at both content and schema level.}
%
\item [\textbf{(\nOne)}]
\emph{The query language must allow for querying
%One must be able to query 
multiple database variants simultaneously.
Additionally, it must allow for filtering tuples to specific variants.}
That is, the framework must provide a query language that allows users to query multiple
database variants at the same time in addition to giving them the freedom to choose
the variants that they want to query. 
For example, an SPL tester that is testing a piece of code for the not highlighted 
variants of the software in \tabref{mot} needs to write queries that exclude the 
 variants associated with yellow cells of \tabref{mot}. 
%For example,
%depending on what component they are working on,
%\revised{SPL testers} need to be able to query all or some of the variants.
%
%Hence, 
%\emph{users need to query multiple database variants simultaneously and selectively}.
%
%\secref{vq} shows how our query language achieves this.
%% by introducing variation into queries, \secref{vq}.
%\exref{vq-specific} and \exref{vq-same-intent-mult-vars} illustrate this 
%for our motivating example.
%
\item [\textbf{(\nTwo)}]
\emph{Every piece of data must clearly state the variant it belongs to and 
this information must be kept throughout the entire framework.}
Continuing the example of the SPL tester, they need to know the variant that
some results belong to in order to be able to debug the software correctly and
accordingly. 
%The framework needs to 
%keep track of which variants a piece of data belongs to and ensuring that 
%it is maintained throughout a query}.
%For example,
%given that one can access all database variants and that they
%can query all variants simultaneously, for test purposes, 
%analysts need to know which variant a tuple belongs to.
%
%Hence,
%\emph{the framework needs to 
%keep track of which variants a piece of data belongs to and ensuring that 
%it is maintained throughout a query}.
%
%\revised{\secref{vdb} and \secref{type-sys} illustrate how our framework achieves this.}
%%We call this \emph{variation-preserving} property and }
%%Storing variants that a tuple belongs to in a VDB achieves the first part, \secref{vdb}, and
%%VRA's type system ensures the second part, \secref{type-sys}.
%\exref{var-pres} illustrates this for a given query.
%\TODO{we should show this for content level too. but for this submission we can say that 
%all approaches account for this and don't lose that info. discuss with Eric. say that it has been tested for all queries but not proved.}
%
\item [\textbf{(\nThree)}]
\emph{The variational database must provide a way for generating database
and query variants.}
For example, the SPL developers need to deploy the management software for each client,
thus, they need to configure the database schema and its queries in the code 
for each software variant.  
%Users need to deploy one variant of the database and its associated queries}.
%For example,
%SPL developers need to deploy the database and its queries
%to generate a specific software product for a client based on their
%requested features. 
%
%Hence,
%\emph{users need to deploy one variant of the database and its associated queries}.
%
%We define a \emph{configure} function for a VDB and its elements, \figref{vdb-conf}, 
%in addition to queries, \figref{v-alg-conf-sem}, that achieves this. 
%\exref{conf-vq} illustrate deploying an example query.
%
\end{enumerate}

Throughout the thesis,
we show how the proposed variational database framework satisfies these requirements 
via examples, proofs, and tests.


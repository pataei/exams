\begin{table}[!htbp]
\caption[Example of step three of table accumulation]{Step three of table accumulation adds the 
presence condition values to relation contents. The table illustrates a set of relation contents that 
are separated by the red bold line between them. The tuples follow the order of attributes in the
relation schema.}
\label{tab:pcadded}
\centering
\small
%\footnotesize
%\scriptsize
\arrayrulecolor{blue}
\begin{tabular} {c !{\color{black}\vrule} l l l l : l }
%\multirow{2}{*}{$\mathit{result}$}  & \empno & \name & \fname & \lname & \pcatt \\
\multirow{2}{*}{\textcolor{white}{result}} & & & & &  \\
\arrayrulecolor{black}\cline{2-6}
& & & & &  \textcolor{blue}{$\vThree \wedge \neg \vFour \wedge \neg \vFive$}\\
\arrayrulecolor{red}\specialrule{.2em}{.1em}{.1em}
 &80001 & Nagui Merli & & & \textcolor{blue}{$\neg \vThree \wedge \vFour \wedge \neg \vFive$}\\
 & 80002 & Mayuko Meszaros & & & \textcolor{blue}{$\neg \vThree \wedge \vFour \wedge \neg \vFive$}\\
 & 80003 & Theirry Viele & & & \textcolor{blue}{$\neg \vThree \wedge \vFour \wedge \neg \vFive$}\\
&\ldots & \ldots  & \ldots & \ldots & \textcolor{blue}{\ldots}\\
\arrayrulecolor{red}\specialrule{.2em}{.1em}{.1em}
 & 200001 & & Selwyn & Koshiba & \textcolor{blue}{$\neg \vThree \wedge \neg \vFour \wedge \vFive$}\\
 & 200002 & & Bedrich & Markovitch &\textcolor{blue}{$\neg \vThree \wedge \neg \vFour \wedge \vFive$} \\
 & 200003 & & Pascal & Benzmuller &\textcolor{blue}{$\neg \vThree \wedge \neg \vFour \wedge \vFive$} \\
 & \ldots & \ldots & \ldots & \ldots & \textcolor{blue}{\ldots}\\
\arrayrulecolor{white}\hline
\end{tabular}

\end{table}


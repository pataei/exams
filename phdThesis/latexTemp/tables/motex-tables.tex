\begin{table}
\caption{Schema variants of the employee database developed for multiple software variants by an SPL. Note that an \educational\ database variant must contain a \basic\ database variant too. 
%Employee schema evolution of a database for a SPL.
%%(evolution features \vOne -- \vFive\ for employee part of the schema (the left schema column), evolution features \tOne -- \tFive\ for education part
%%of the schema (the right schema column), and \edu\ feature representing the education feature of SPL). 
%A feature (a boolean variable) represents 
%inclusion/exclusion of tables/attributes.  
%The dash-underlined attributes in the basic schema 
%have \emph{variational} present in their relation, i.e.,
%they only exist in their relation when \edu\ = \t.
%%in left schema column only exist when the \edu\ feature of SPL is enabled. 
%%This table represents 1056 possible schema variants:
%%\ensuremath{2^5} schema variants when \edu\ is disabled 
%%(by enabling any combination of \vOne--\vFive) 
%%in addition to \ensuremath{2^5 \times 2^5} when \edu\ is enabled.
}
\label{tab:mot}
\centering
\small
%\footnotesize
%\scriptsize
\begin{subtable}[t]{\textwidth}
\centering
\caption{Database schema variants for \basic\ software variants.}
\label{tab:mot-basic}
\begin{tabular} {| c | l |}
\hline
\textbf{Temporal Features} & \textbf{\basic\ Database Schema Variants}\\
\hline
\multirow{3}{*}{\vOne} &  \engemp\ (\empno, \name, \hiredate, \titleatt, \deptname) \\
& \othemp\ (\empno, \name, \hiredate, \titleatt, \deptname) \\
& \job\ (\titleatt, \salary) \\
\hline
\multirow{2}{*}{\vTwo} & \cellcolor{yellow}{\empacct\ (\empno, \name, \hiredate, \titleatt, \deptname)}\\
& \cellcolor{yellow}{\job\ (\titleatt, \salary)} \\
\hline
\multirow{4}{*}{\vThree} & \empacct\ (\empno, \name, \hiredate, \titleatt, \deptno)\\
& \job\ (\titleatt, \salary) \\
& \dept\ (\deptname, \deptno, \managerno) \\
& \empbio\ (\empno, \sex, \birthdate)\\
\hline
\multirow{4}{*}{\vFour} & \empacct\ (\empno, \hiredate, \titleatt, \deptno, \dashuline{\isstudent}, \dashuline{\isteacher}) \\
& \job\ (\titleatt, \salary) \\
& \dept\ (\deptname, \deptno, \managerno) \\
& \empbio\ (\empno, \sex, \birthdate, \name) \\
\hline
\multirow{4}{*}{\vFive} & \empacct\ (\empno, \hiredate, \titleatt, \deptno,  \dashuline{\isstudent}, \dashuline{\isteacher}, \salary)\\
& \dept\ (\deptname, \deptno, \managerno,  \dashuline{\studentnum}, \dashuline{\teachernum}) \\
& \empbio\ (\empno, \sex, \birthdate, \fname, \lname) \\
\hline
\end{tabular}
\end{subtable}

\medskip
\medskip
\medskip
\begin{subtable}[t]{\textwidth}
\centering
\caption{Database schema variants for \educational\ software variants.}
\label{tab:mot-edu}
\begin{tabular} {| c | l |}
\hline
\textbf{Temporal Feature} & \textbf{\educational\ Database Schema Variants }\\
%\cline{2-3}
%\textbf{\tiny Features} & \multicolumn{1}{ c ||} {\basic} & \multicolumn{1}{ c |} {\educational} & \textbf{\tiny Features}\\
\hline
%\cline{2-3}
%\cline{2-3}
%\hhline{-==}
\multirow{2}{*}{\tOne} & \course\ (\cname, \tno)\\
& \student\ (\sno, \cname) \\
\hline
\multirow{2}{*}{\tTwo} & \course\ (\cno, \cname, \tno)\\
%\cdashline{2-3}
& \student\ (\sno, \cno)\\
\hline
\multirow{3}{*}{\tThree} & \cellcolor{yellow}{\course\ (\cno, \cname)} \\
& \cellcolor{yellow}{\teach\ (\tno, \cno)} \\
& \cellcolor{yellow}{\student\ (\sno, \cno, \grade)} \\
\hline
\multirow{4}{*}{\tFour} & \ecourse\ (\cno, \cname) \\
& \course\ (\cno, \cname, \timeatt, \class) \\
& \teach\ (\tno, \cno) \\
& \student\ (\sno, \cno, \grade) \\
\hline
\multirow{4}{*}{\tFive}  & \ecourse\ (\cno, \cname, \deptno) \\
& \course\ (\cno, \cname, \timeatt, \class, \deptno) \\
& \teach\ (\tno, \cno)\\
& \take\ (\sno, \cno, \grade)\\
\hline
\end{tabular}
\end{subtable}
\end{table}


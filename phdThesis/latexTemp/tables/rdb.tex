\begin{table}
\caption[Example of a relational database]{An example of a relational database corresponding to \vTwo\ of our motivating example
given in \tabref{mot-basic}.}
\label{tab:rdb}
\centering
\small
%\footnotesize
%\scriptsize
\begin{subtable}[t]{\textwidth}
\centering
\caption{The schema of a relational database.}
\label{tab:rdb-sch}
\begin{tabular} {| l | }
\hline
\empacct\ (\empno, \name, \hiredate, \titleatt, \deptname)\\
\job\ (\titleatt, \salary)\\
\hline
\end{tabular}
\end{subtable}

\medskip
\medskip
\medskip
\begin{subtable}[t]{\textwidth}
%\begin{center}
\centering
\caption{The \empacct\ table.}
\label{tab:rdb-empacct}
\begin{tabular} {c | l l l l l}
%\hline
%\hhline{-==}
\multirow{2}{*}{\empacct} & \empno & \name & \hiredate & \titleatt & \deptname\\
\cline{2-6}
& 10001 & Georgi Facello & 1986-06-26 & Senior Engineer & Development\\
& 10002 & Bezalel Simmel & 1985-11-21 & Staff & Sales\\
& \ldots & \ldots & \ldots & \ldots & \ldots \\
& 499998 & Patricia Breugel & 1993-10-13 & Senior Staff & Finance\\
& 499999 & Sachin Tsukuda & 1997-11-30 & Engineer & Production
\end{tabular}
%\end{center}
\end{subtable}

\medskip
\medskip
\medskip
\begin{subtable}[t]{\textwidth}
\centering
\caption{The \job\ table.}
\label{tab:rdb-job}
\begin{tabular} {c | l l }
%\hline
%\hhline{-==}
\multirow{2}{*}{\job} & \titleatt & \salary\\
\cline{2-3}
& Assistant Engineer & 61594\\
& Senior Engineer & 96646\\
& \ldots & \ldots \\
& Staff & 77935\\
& Technique Leader & 58345
\end{tabular}
\end{subtable}

\end{table}


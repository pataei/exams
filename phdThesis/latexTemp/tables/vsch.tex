\begin{table}
\caption[shortcaption]{The relation schema of \empbio\ for variants that enable one of the features \vThree, \vFour, or \vFive and the variational relation schema of \empbio\ encompassing 
the three variants of the plain relation \empbio.}
\label{tab:empbio}
\centering
\small
%\footnotesize
%\scriptsize
\begin{subtable}[t]{\textwidth}
\centering
\caption{The relation schema of \empbio\ for variants that enable the feature \vThree.}
\label{tab:empbio-v3}
\begin{tabular} {c | l l l}
\empbio & \empno & \sex & \birthdate\\
\end{tabular}
\end{subtable}

\medskip
\medskip
\medskip
\begin{subtable}[t]{\textwidth}
\centering
\caption{The relation schema of \empbio\ for variants that enable the feature \vFour.}
\label{tab:empbio-v4}
\begin{tabular} {c | l l l l}
\empbio & \empno & \sex & \birthdate & \name\\
\end{tabular}
\end{subtable}

\medskip
\medskip
\medskip
\begin{subtable}[t]{\textwidth}
\centering
\caption{The relation schema of \empbio\ for variants that enable the feature \vFive.}
\label{tab:empbio-v5}
\begin{tabular} {c | l l l l l}
\empbio & \empno & \sex & \birthdate & \fname & \lname\\
\end{tabular}
\end{subtable}


\begin{subtable}[t]{\textwidth}
\centering
\caption{The variational relation schema of \empbio.}
\label{tab:empbio-vsch}
\begin{tabular} {c | l l l l l l}
%\hline
%\hhline{-==}
\empbio & \empno & \sex & \birthdate & \fname & \lname\\
%\job & \titleatt & \salary\\
%\cline{2-3}
%& Assistant Engineer & 61594\\
%& Senior Engineer & 96646\\
%& \ldots & \ldots \\
%& Staff & 77935\\
%& Technique Leader & 58345
\end{tabular}
\end{subtable}

\end{table}

\begin{table*}
\caption[Variational schema of the motivating example]{Variational schema $\vSch_\mot$ 
%of the motivating example given in \secref{mot} 
with feature model $\dimMeta_\mot$.
This variational schema encompasses 30 relational schemas: five schemas when \edu\ = \f\ and 25 schemas otherwise. 
%A v-schema encoding the variation of the employee schema introduced in \tabref{mot}. The feature model \fModel\ only allows one temporal feature to be true from a set of temporal features at the time: 
%\ensuremath{
%\fModel = \edu \vee 
%\left( 
%\vOne \oplus \vTwo \oplus \vThree \oplus \vFour \oplus \vFive
%\right) 
%\vee 
%\left( 
%\tOne \oplus \tTwo \oplus \tThree \oplus \tFour \oplus \tFive
%\right)
%}. 
%%where \ensuremath{\fName_1 \oplus \fName_2 = (\fName_1 \wedge \neg \fName_2) \vee (\fName_2 \wedge \neg \fName_1)}.
%However, the feature model can be encoded differently to allow more than one temporal feature to be true at the time. Hence, 
%this is not the only v-schema capturing variation of the employee schema. 
%Additionally, the encoding can change by formulating presence conditions differently while representing the same v-schema, 
%e.g., the presence condition of the \job\ relation can be changed to \ensuremath{\neg \vFive}. 
%This v-schema encompasses 30 relational schemas: five schemas when \edu\ = \f\ and 25 schemas otherwise. 
%%Note that the feature model restricts the schema s.t. only one variant of a sub-schema can exists in the schema, e.g., both \vOne\ and \vTwo cannot be enabled at the same time.
}
\label{tab:mot-vsch}
\arrayrulecolor{black}
\begin{center}
\small
\begin{tabular} {| l |}
\hline
\ensuremath{
\engemp (\empno, \name, \hiredate,\titleatt,\deptname )^{\textcolor{blue}{\vOne}}
}\\
\ensuremath{
\othemp (\empno, \name, \hiredate,\titleatt,\deptname )^{\textcolor{blue}{\vOne}}
}\\
\ensuremath{
\empacct (\empno, \optAtt [{\textcolor{blue}{\vTwo \vee \vThree}}] [\name], \hiredate, \titleatt, \optAtt [{\textcolor{blue}{\vTwo}}] [\deptname], \optAtt [{\textcolor{blue}{\vThree \vee \vFour \vee \vFive}}] [\deptno], \optAtt [{\textcolor{blue}{\vFive}}] [\salary],} \\
\hspace{40pt} \ensuremath{\optAtt [{\textcolor{blue}{\edu \wedge (\vFour \vee \vFive)}}] [\isstudent], \optAtt [{\textcolor{blue}{\edu \wedge (\vFour \vee \vFive)}}] [\isteacher] )^{\textcolor{blue}{\vTwo \vee \vThree \vee \vFour \vee \vFive}}
}\\
\ensuremath{
\job \left(\titleatt, \salary  \right)^{\textcolor{blue}{\vOne \vee \vTwo \vee \vThree \vee \vFour}}
}\\
\ensuremath{
\dept \left(\deptname, \deptno, \managerno, \optAtt [{\textcolor{blue}{\edu \wedge \vFive}}] [\studentnum], \optAtt [{\textcolor{blue}{\edu \wedge \vFive}}] [\teachernum] \right)^{\textcolor{blue}{\vThree \vee \vFour \vee \vFive}}
}\\
\ensuremath{
\empbio \left(\empno, \sex, \birthdate, \optAtt [{\textcolor{blue}{\vFour}}] [\name], \optAtt [{\textcolor{blue}{\vFive}}] [\fname], \optAtt [{\textcolor{blue}{\vFive}}] [\lname] \right)^{\textcolor{blue}{\vThree \vee \vFour \vee \vFive}}
}\\
%\hdashline
\ensuremath{
\course \left(\optAtt [{\textcolor{blue}{\neg \tOne}}] [\cno], \cname, \optAtt [{\textcolor{blue}{\tOne \vee \tTwo}}] [\tno], \optAtt [{\textcolor{blue}{\tFour \vee \tFive}}] [\timeatt], \optAtt [{\textcolor{blue}{\tFour \vee \tFive}}] [\class], \optAtt [{\textcolor{blue}{\tFive}}] [\deptno] \right)^{\textcolor{blue}{\edu}}
}\\
\ensuremath{
\student \left(\sno, \optAtt [{\textcolor{blue}{\tOne}}] [\cname], \optAtt [{\textcolor{blue}{\neg \tOne}}] [\cno], \optAtt [{\textcolor{blue}{\tThree \vee \tFour}}] [\grade] \right)^{\textcolor{blue}{\edu \wedge \neg \tFive}}
}\\
\ensuremath{
\teach \left(\tno, \cno \right)^{\textcolor{blue}{\edu \wedge \left(\tThree \vee \tFour \vee \tFive\right)}}
}\\
\ensuremath{
\ecourse \left(\cno, \cname, \optAtt [{\textcolor{blue}{\tFive}}] [\deptno] \right)^{\textcolor{blue}{\edu \wedge \left(\tFour \vee \tFive\right)}}
}\\
\ensuremath{
\take \left(\sno, \cno, \grade \right)^{\textcolor{blue}{\edu \wedge \tFive}}
}\\
%\multirow{3}{*}{\vOne} &  \engemp\ (\empno, \name, \hiredate,\titleatt,\deptname) & 
%\course\ (\cname, \tno) & \multirow{3}{*}{\tOne}\\
%& \othemp\ (\empno, \name, \hiredate, \title, \deptname)  & \student\ (\sno, \cname) &\\
%& \job\ (\titleatt, \salary) &  &\\
%\hline
%\multirow{2}{*}{\vTwo} & \empacct\ (\empno, \name, \hiredate, \titleatt, \deptname) & \course\ (\cno, \cname, \tno) & \multirow{2}{*}{\tTwo}\\
%%\cdashline{2-3}
%& \job\ (\titleatt, \salary) & \student\ (\sno, \cno) & \\
%\hline
%\multirow{4}{*}{\vThree} & \empacct\ (\empno, \name, \hiredate, \titleatt, \deptno) & \course\ (\cno, \cname) & \multirow{4}{*}{\tThree}\\
%& \job\ (\titleatt, \salary) & \teach\ (\tno, \cno) &\\
%& \dept\ (\deptname, \deptno, \managerno) & \student\ (\sno, \cno, \grade) &\\
%& \empbio\ (\empno, \sex, \birthdate) & &\\
%\hline
%\multirow{4}{*}{\vFour} & \empacct\ (\empno, \hiredate, \titleatt, \deptno, \dashuline{\isstudent}, \dashuline{\isteacher}) & \ecourse\ (\cno, \cname) & \multirow{4}{*}{\tFour}\\
%& \job\ (\titleatt, \salary) & \course\ (\cno, \cname, \timeatt, \class) & \\
%& \dept\ (\deptname, \deptno, \managerno) & \teach\ (\tno, \cno) & \\
%& \empbio\ (\empno, \sex, \birthdate, \name) & \student\ (\sno, \cno, \grade) & \\
%\hline
%\multirow{4}{*}{\vFive} & \empacct\ (\empno, \hiredate, \titleatt, \deptno,  \dashuline{\isstudent}, \dashuline{\isteacher}, \salary) & \ecourse\ (\cno, \cname, \deptno) & \multirow{4}{*}{\tFive}\\
%& \dept\ (\deptname, \deptno, \managerno,  \dashuline{\studentnum}, \dashuline{\teachernum}) & \course\ (\cno, \cname, \timeatt, \class, \deptno) & \\
%& \empbio\ (\empno, \sex, \birthdate, \fname, \lname) & \teach\ (\tno, \cno) & \\
%&& \take\ (\sno, \cno, \grade) & \\
\hline
\end{tabular}
\end{center}
\end{table*}


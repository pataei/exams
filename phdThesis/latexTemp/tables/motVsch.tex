\tabref{mot-vsch} provides the v-schema of our motivating example, given in \tabref{mot}.
The feature model \fModel\ only allows one temporal feature to be true from a set of temporal features at the time: 
\ensuremath{
\fModel = 
\left( 
\neg \edu \wedge \left(
\vOne \oplus \vTwo \oplus \vThree \oplus \vFour \oplus \vFive
\right) \right)
\vee 
\left( \edu \wedge \left(
\tOne \oplus \tTwo \oplus \tThree \oplus \tFour \oplus \tFive
\right) \wedge
 \left(
\vOne \oplus \vTwo \oplus \vThree \oplus \vFour \oplus \vFive
\right) \right)
}. 
%where \ensuremath{\fName_1 \oplus \fName_2 = (\fName_1 \wedge \neg \fName_2) \vee (\fName_2 \wedge \neg \fName_1)}.
However, the feature model can be encoded differently. For example, we can restrict it s.t. it only allows 
the two sets of temporal features to change together, i.e.,
\ensuremath{
\VVal \fModel = 
\left( \vOne \vee \left(\vOne \wedge \edu \wedge \tOne \right) \right)
\oplus
\left( \vTwo \vee \left(\vTwo \wedge \edu \wedge \tTwo \right) \right)
\oplus
\left( \vThree \vee \left(\vThree \wedge \edu \wedge \tThree \right) \right)
\oplus
\left( \vFour \vee \left(\vFour \wedge \edu \wedge \tFour \right) \right)
\oplus
\left( \vFive \vee \left(\vFive \wedge \edu \wedge \tFive \right) \right)
}. 
%  to allow more than one temporal feature to be true at the time. 
Hence, 
\fModel\ is not the only v-schema capturing variation of the employee schema. 
Additionally, the encoding can change by formulating presence conditions differently while representing the same v-schema, 
e.g., the presence condition of the \job\ relation can be changed to \ensuremath{\neg \vFive}. 

\begin{table*}
\caption{V-schema of the employee motivating example given in \tabref{mot} with feature model \fModel.
This v-schema encompasses 30 relational schemas: five schemas when \edu\ = \f\ and 25 schemas otherwise. 
%A v-schema encoding the variation of the employee schema introduced in \tabref{mot}. The feature model \fModel\ only allows one temporal feature to be true from a set of temporal features at the time: 
%\ensuremath{
%\fModel = \edu \vee 
%\left( 
%\vOne \oplus \vTwo \oplus \vThree \oplus \vFour \oplus \vFive
%\right) 
%\vee 
%\left( 
%\tOne \oplus \tTwo \oplus \tThree \oplus \tFour \oplus \tFive
%\right)
%}. 
%%where \ensuremath{\fName_1 \oplus \fName_2 = (\fName_1 \wedge \neg \fName_2) \vee (\fName_2 \wedge \neg \fName_1)}.
%However, the feature model can be encoded differently to allow more than one temporal feature to be true at the time. Hence, 
%this is not the only v-schema capturing variation of the employee schema. 
%Additionally, the encoding can change by formulating presence conditions differently while representing the same v-schema, 
%e.g., the presence condition of the \job\ relation can be changed to \ensuremath{\neg \vFive}. 
%This v-schema encompasses 30 relational schemas: five schemas when \edu\ = \f\ and 25 schemas otherwise. 
%%Note that the feature model restricts the schema s.t. only one variant of a sub-schema can exists in the schema, e.g., both \vOne\ and \vTwo cannot be enabled at the same time.
}
\label{tab:mot-vsch}
\begin{center}
\small
\begin{tabular} {| l |}
\hline
\ensuremath{
\engemp \left(\empno, \name, \hiredate,\titleatt,\deptname \right)^{\vOne}
}\\
\ensuremath{
\othemp \left(\empno, \name, \hiredate,\titleatt,\deptname \right)^{\vOne}
}\\
\ensuremath{
\empacct \left(\empno, \optAtt [\vTwo \vee \vThree] [\name], \hiredate, \titleatt, \optAtt [\vTwo] [\deptname], \optAtt [\vThree \vee \vFour \vee \vFive] [\deptno], \optAtt [\vFive] [\salary], \optAtt [\vFour \vee \vFive] [\isstudent], \optAtt [\vFour \vee \vFive] [\isteacher] \right)^{\vTwo \vee \vThree \vee \vFour \vFive}
}\\
\ensuremath{
\job \left(\titleatt, \salary  \right)^{\vOne \vee \vTwo \vee \vThree \vee \vFour}
}\\
\ensuremath{
\dept \left(\deptname, \deptno, \managerno, \optAtt [\vFive] [\studentnum], \optAtt [\vFive] [\teachernum] \right)^{\vThree \vee \vFour \vee \vFive}
}\\
\ensuremath{
\empbio \left(\empno, \sex, \birthdate, \optAtt [\vFour] [\name], \optAtt [\vFive] [\fname], \optAtt [\vFive] [\lname] \right)^{\vFour \vee \vFive}
}\\
\hdashline
\ensuremath{
\course \left(\optAtt [\neg \tOne] [\cno], \cname, \optAtt [\tOne \vee \tTwo] [\tno], \optAtt [\tFour \vee \tFive] [\timeatt], \optAtt [\tFour \vee \tFive] [\class], \optAtt [\tFive] [\deptno] \right)^\edu
}\\
\ensuremath{
\student \left(\sno, \optAtt [\tOne] [\cname], \optAtt [\neg \tOne] [\cno], \optAtt [\tThree \vee \tFour] [\grade] \right)^{\edu \wedge \neg \tFive}
}\\
\ensuremath{
\teach \left(\tno, \cno \right)^{\edu \wedge \left(\tThree \vee \tFour \vee \tFive\right)}
}\\
\ensuremath{
\ecourse \left(\cno, \cname, \optAtt [\tFive] [\deptno] \right)^{\edu \wedge \left(\tFour \vee \tFive\right)}
}\\
\ensuremath{
\take \left(\sno, \cno, \grade \right)^{\edu \wedge \tFive}
}\\
%\multirow{3}{*}{\vOne} &  \engemp\ (\empno, \name, \hiredate,\titleatt,\deptname) & 
%\course\ (\cname, \tno) & \multirow{3}{*}{\tOne}\\
%& \othemp\ (\empno, \name, \hiredate, \title, \deptname)  & \student\ (\sno, \cname) &\\
%& \job\ (\titleatt, \salary) &  &\\
%\hline
%\multirow{2}{*}{\vTwo} & \empacct\ (\empno, \name, \hiredate, \titleatt, \deptname) & \course\ (\cno, \cname, \tno) & \multirow{2}{*}{\tTwo}\\
%%\cdashline{2-3}
%& \job\ (\titleatt, \salary) & \student\ (\sno, \cno) & \\
%\hline
%\multirow{4}{*}{\vThree} & \empacct\ (\empno, \name, \hiredate, \titleatt, \deptno) & \course\ (\cno, \cname) & \multirow{4}{*}{\tThree}\\
%& \job\ (\titleatt, \salary) & \teach\ (\tno, \cno) &\\
%& \dept\ (\deptname, \deptno, \managerno) & \student\ (\sno, \cno, \grade) &\\
%& \empbio\ (\empno, \sex, \birthdate) & &\\
%\hline
%\multirow{4}{*}{\vFour} & \empacct\ (\empno, \hiredate, \titleatt, \deptno, \dashuline{\isstudent}, \dashuline{\isteacher}) & \ecourse\ (\cno, \cname) & \multirow{4}{*}{\tFour}\\
%& \job\ (\titleatt, \salary) & \course\ (\cno, \cname, \timeatt, \class) & \\
%& \dept\ (\deptname, \deptno, \managerno) & \teach\ (\tno, \cno) & \\
%& \empbio\ (\empno, \sex, \birthdate, \name) & \student\ (\sno, \cno, \grade) & \\
%\hline
%\multirow{4}{*}{\vFive} & \empacct\ (\empno, \hiredate, \titleatt, \deptno,  \dashuline{\isstudent}, \dashuline{\isteacher}, \salary) & \ecourse\ (\cno, \cname, \deptno) & \multirow{4}{*}{\tFive}\\
%& \dept\ (\deptname, \deptno, \managerno,  \dashuline{\studentnum}, \dashuline{\teachernum}) & \course\ (\cno, \cname, \timeatt, \class, \deptno) & \\
%& \empbio\ (\empno, \sex, \birthdate, \fname, \lname) & \teach\ (\tno, \cno) & \\
%&& \take\ (\sno, \cno, \grade) & \\
\hline
\end{tabular}
\end{center}
\end{table*}

